\documentclass[a4paper,12pt,oneside,openany]{memoir}
\usepackage{sty/preamble}

\newlength{\alphabet}
\settowidth{\alphabet}{\normalfont abcdefghijklmnopqrstuvwxyz}

%% Page setup
\usepackage{geometry}
\geometry{
    % margin=2cm,
    vmargin=2cm,
    hmargin=3cm,
    % hmargin=1in,
    % textwidth=2.3\alphabet,
    includehead,
    % includefoot,
    heightrounded,
    % showframe,
    % pass,
}
\pagestyle{fancy}
\fancyfoot{}
\fancyfoot[R]{\thepage}

\setcounter{tocdepth}{2}
\setcounter{secnumdepth}{1}

\setlength{\parindent}{2pc}

%% Frame around the text area
% \usepackage{showframe}
% \renewcommand\ShowFrameLinethickness{0.1pt}
% \renewcommand*\ShowFrameColor{\color{lightgray}}

% \usepackage{indentfirst}
% \usepackage{tocbibind}
\usepackage{verse}
\usepackage{paracol}
\usepackage{xlop}

\let\providelength\undefined
\let\providecounter\undefined
\usepackage{moredefs}


\title{ГЁДЕЛЬ, ЭШЕР, БАХ: \\ эта бесконечная гирлянда}
\author{Дуглас Ричард Хофштадтер}


\begin{document}

\frontmatter

\maketitle

\mainmatter

% Праздничное предисловие автора к русскому изданию книги «Гёдель, Эшер Бах»
\subfile{parts/preface}

\begingroup
\hypersetup{hidelinks}
\clearpage
% \begin{KeepFromToc}
\tableofcontents
% \end{KeepFromToc}
\clearpage
\endgroup

% Обзор
\subfile{parts/overview}
% Список иллюстраций
\subfile{parts/illustrations}
% Благодарность
\subfile{parts/acknowledgment}

\part{ЧАСТЬ~I}

% TODO: illustration 1
% \emph{Рис. I. Иоганн Себастиан Бах в 1748. С портрета кисти Элиаса Готтлиба Хауссманна.}

% Интродукция: музыко-логическое приношение
\subfile{parts/introduction}
% Трехголосная инвенция
\subfile{parts/dial01}
% Глава 1. Головоломка MU
\subfile{parts/ch01}
% Двухголосная инвенция
\subfile{parts/dial02}
% Глава 2. Содержание и форма в математике
\subfile{parts/ch02}
% Соната для Ахилла соло
\subfile{parts/dial03}
% Глава 3. Рисунок и фон
\subfile{parts/ch03}
% Акростиконтрапунктус
\subfile{parts/dial04}
% Глава 4. Непротиворечивость, полнота и геометрия
\subfile{parts/ch04}
% Маленький гармонический лабиринт
\subfile{parts/dial05}
% Глава 5. Рекурсивные структуры и процессы
\subfile{parts/ch05}
% Канон с интервальным увеличением
\subfile{parts/dial06}
% Глава 6. Местонахождение значения
\subfile{parts/ch06}
% Хроматическая фантазия и фига
\subfile{parts/dial07}
% Глава 7. Исчисление Высказываний
\subfile{parts/ch07}
% Крабий канон
\subfile{parts/dial08}
% Глава 8. Типографская теория чисел
\subfile{parts/ch08}
% Приношение «МУ»
\subfile{parts/dial09}
% Глава 9. Мумон и Гёдель
% \subfile{parts/ch09}
% %
% \subfile{parts/dial10}
% %
% \subfile{parts/dial11}
% %
% \subfile{parts/dial12}
% %
% \subfile{parts/dial13}
% %
% \subfile{parts/dial14}

% \subsubsection{Интродукция: музыко-логическо приношение}

\emph{Автор:}

КОРОЛЬ ПРУССИИ Фридрих Великий пришел к власти в 1740 году. Исторические трактаты упоминают о нем в основном как о проницательном и умелом полководце - однако, кроме военной деятельности, Фридрих Великий в немалой степени посвящал себя жизни умственной и духовной. Его двор в Потсдаме был центром интеллектуальной деятельности Европы восемнадцатого столетия. Прославленный математик Леонард Эйлер провел там двадцать пять лет. Многие математики, ученые и философы посетили в то время Потсдам; Вольтер и Ламеттри написали там некоторые из своих важнейших сочинений.

Но настоящей любовью короля была музыка. Сам он был страстным флейтистом и композитором; некоторые его сочинения исполняются иногда по сей день. Фридрих Великий был одним из первых покровителей искусств, признавших замечательные качества только что изобретенного фортепиано («тихогрома», как когда-то пытались окрестить этот инструмент в России). Фортепиано было изобретено в первой половине восемнадцатого века; оно представляло из себя не что иное, как модификацию клавесина. Дело в том, что на клавесине невозможно было варьировать громкость; все звуки получались одинаковыми. Тихогром, как показывает само название, был выходом из положения.

Зародившись в Италии, где Бартоломео Кристофори изготовил первое фортепиано, идея тихогрома распространилась широко. Готтфрид Зильберман, лучший мастер того времени по изготовлению органов, получил заказ на изготовление «совершенного» фортепиано. Фридрих Великий, без сомнения, явился самым большим энтузиастом этого начинания; говорят, что он приобрел целых пятнадцать инструментов, сделанных Зильберманом!

Бах

Король был горячим поклонником не только фортепиано; его вниманием пользовался также органист и композитор по имени И. С. Бах. Баховские композиции были довольно интересны; некоторые считали их напыщенными и запутанными, в то время как другие ценители восхищались ими как несравненными шедеврами. Однако никто не оспаривал способности Баха исполнять импровизации на органе. В то время умение импровизировать, наравне с исполнительским мастерством, считалось необходимым качеством органиста, а Бах имел славу превосходного импровизатора. (Прелестные рассказы о Баховских импровизациях читатель может найти в книге Дэвида и Менделя «\emph{Баховская хрестоматия»} (David~\& Mendel, «The Bach Reader».))

В 1747 году слава 62-летнего Баха докатилась до Потсдама. Там же очутился и один из его сыновей, Карл Филипп Эмануэль Бах, ставший капельмейстером при дворе короля Фридриха. В течение нескольких лет король деликатно намекал Филиппу Эмануэлю, насколько приятен был бы Его Величеству визит в Потсдам Баха-старшего. В особенности Фридриху хотелось, чтобы Бах опробовал его новые рояли Зильбермана, которые, как он правильно предвидел, были началом больших перемен в музыке. Это королевское желание, однако, долго не исполнялось.

При дворе Фридриха Великого были обычаем вечерние концерты камерной музыки. В концертах для флейты часто солировал сам монарх. Я привожу здесь репродукцию картины немецкого художника Адольфа фон Менцеля, кто в 1800-х годах написал серию произведений из жизни Фридриха Великого. На клавесине играет К. Ф. Э. Бах; крайний справа - Иоахим Кванц, учивший короля игре на флейте и единственный, кому было даровано право исправлять ошибки в игре Его Величества. Однажды майским вечером 1747 года на королевский концерт явился неожиданный гость. Иоганн Николаус Форкель, один из первых биографов Баха, рассказывает эту историю так.

Однажды вечером, когда король уже достал свою флейту и все музыканты были готовы, вошел слуга со списком новоприбывших гостей. Не выпуская флейты из рук, король стал проглядывать список; вдруг он быстро повернулся к собравшимся музыкантам и взволнованно воскликнул: «Господа, приехал старый Бах!» Флейта была отложена, и Баха, остановившегося у сына, тут же пригласили во дворец. Вильгельм Фридеман Бах, сопровождавший своего отца, передал мне эту историю, и, должен признаться, я до сих пор вспоминаю его рассказ с удовольствием. В то время в моде были многословные и цветистые любезности. Первое появление Баха, даже не успевшего сменить дорожное платье, перед Его Величеством, разумеется, сопровождалось пышными и изысканными извинениями. Не буду останавливаться на них подробно; замечу лишь, что в устах Вильгельма Фридемана они представляли из себя настоящий формальный диалог между Королем и Приносящим Извинения.

Самым главным, однако, было то, что король отложил свой вечерний концерт и пригласил Баха, уже тогда известного как «старый Бах», опробовать Зильбермановские фортепиано, стоявшие в нескольких залах дворца. (Здесь Форкель делает сноску: «Фортепиано, изготовленные Зильберманом из Фрейбурга, так понравились королю, что он решил скупить их все. Его коллекция насчитывала пятнадцать инструментов. Говорят, что все они, ныне непригодные, еще хранятся по углам королевского дворца.»)

Бах был приглашен играть свои импровизации; музыканты сопровождали его из залы в залу. Спустя некоторое время он предложил королю предоставить ему тему для фуги, чтобы обработать ее тут же, без подготовки. Результат привел короля в восторг. Возможно, чтобы узнать, каковы пределы импровизаторского мастерства Баха, Фридрих Великий выразил желание услышать фугу с шестью облигатными голосами. Так как не всякая тема подходит к такой полной гармонии, Бах выбрал тему сам~и тут же сыграл на нее фугу так же блистательно и легко, как и на королевскую тему, чем поразил всех присутствующих.

Его Величество захотел затем услышать игру Баха на органе; на следующий день Баху пришлось совершить турне по всем органам Потсдама, так же как накануне - по всем Зильбермановским фортепиано.

После своего возвращения в Лейпциг Бах обработал тему, данную ему королем, создав трехголосную и шестиголосную композиции. К ним он добавил несколько искусных проведений темы в форме строгого канона, назвал свое произведение \emph{«Музыкальным приношением»} и посвятил его автору темы.\footnote{H.~D. David and A Mendel «The Bach Reader» стр. 305 6}

\emph{Рис. 2. Адольф фон Мензель. «Концерт флейтистов в Сансуси».}

\emph{Рис. 3. Королевская Тема.}

Посылая королю «Музыкальное приношение», Бах приложил к нему письмо-посвящение, интересное уже самим своим стилем, смиренным и льстивым. С нынешней точки зрения это кажется смешным. Письмо это также дает некоторое представление о стиле Баховских извинений перед королем за свой «непрезентабельный» вид во время их первой встречи.\footnote{Там же стр. 179}

ВСЕМИЛОСТИВЕЙШИЙ ГОСУДАРЬ,

В глубочайшем смирении я осмеливаюсь посвятить Вашему Величеству музыкальное приношение, наилучшая часть коего создана Августейшей рукой Вашего Величества. С благоговейным и счастливым трепетом я вспоминаю особую королевскую милость, когда, во время моего визита в Потсдам, Ваше Величество собственной персоной снизошли до того, чтобы сыграть на клавире тему для фуги, и тогда же всемилостивейше поручили мне развить эту тему в присутствии Вашего Августейшего Величества. Со смирением повиновался я тогда высочайшему повелению. Однако очень скоро я заметил, что за недостатком специальной подготовки я был не в состоянии выполнить это задание так, как того требовала сия превосходная тема Засим я решился и с готовностию посвятил себя работе над более полным развитием прекрасной Королевской темы с тем, чтобы сделать ее известной всему миру По мере своих сил я исполнил это решение, движимый желанием прославить, хотя бы в ничтожной степени, Монарха, чье величие и могущество, как в науках военных и мирных, так и в музыке, достойно восхищения и преклонения каждого. Осмелюсь смиренно просить Ваше Величество снизойти до принятия моего скромного труда и продолжить дарить Августейшую милость

Его покорнейшему и смиреннейшему слуге

АВТОРУ.

Лейпциг, 7 июля 1747

Спустя двадцать четыре года после смерти Баха (он умер в 1750 году) барон по имени Готфрид ван Свитен, кому, кстати, Форкель посвятил свою биографию Баха, а Бетховен~--- свою Первую симфонию, имел беседу с королем Фридрихом. Барон вспоминает об этом так:

Он (Фридрих) говорил со мной, среди прочего, о музыке и о великом органисте по имени Бах, проведшем некоторое время в Берлине. Речь шла о Вильгельме Фридемане Бахе Я сказал, что этот музыкант наделен талантом, по глубине понимания гармонии и по исполнительской мощи превосходящим все, о чем я слышал и что я могу себе вообразить; те же, кто знавал его отца, утверждают, что тот был еще более велик. Король согласился с этим мнением и в подтверждение спел мне хроматическую тему для фуги, которую он когда-то дал старому Баху; по его словам, Бах тогда же, не сходя с места, превратил эту тему в фугу, сначала для четырех, потом для пяти и, наконец, для восьми голосов.\footnote{Там же стр. 260}

Сейчас уже невозможно сказать, кто украсил случившееся фантастическими подробностями~--- Фридрих Великий или барон Ван Свитен. Однако этот случай показывает, что уже в то время Бах стал легендарной личностью. Представление о том, насколько удивительна шестиголосная фуга, дает тот факт, что среди 48 прелюдий и фуг \emph{«Хорошо темперированного клавира»} встречаются только две пятиголосные фуги. Шестиголосных фуг там нет. Импровизацию такой фуги можно, пожалуй, сравнить с сеансом одновременной игрой в шахматы вслепую на шестидесяти досках, где мастер побеждает во всех партиях! Импровизация же восьмиголосной фуги находится за пределами человеческих возможностей.

В рукописи, которую Бах послал Фридриху Великому, на странице, предшествующей нотам, была следующая надпись:

\emph{Рис. 4. Акростих Баха «РИЧЕРКАР».}

(«По повелению Короля мелодия и дополнение разрешены каноническим искусством».) Здесь Бах играет со словом «канонический», обозначающим не только «при помощи канонов», но также «наилучшим образом». Начальные буквы этой надписи составляют итальянское слово

RICERCAR

(РИЧЕРКАР), означающее «искать», «исследовать». Действительно, \emph{«Музыкальное приношение»} представляет собой достойный объект для исследования! Оно состоит из трехголосной и шестиголосной фуг, десяти канонов и триосонаты. Музыковеды считают, что трехголосная фуга, скорее всего, та самая, которую Бах симпровизировал для короля. Шестиголосная фуга~--- одна из самых сложных Баховских композиций; она основана, конечно же, на Королевской теме. Читатель найдет эту знаменитую тему на рис. 3. Она очень сложна, ритмически причудлива и полна хроматизмов (то есть звуков в другой тональности). Для среднего музыканта было бы нелегко написать даже приличную двухголосную фугу, основанную на такой теме.

Обе фуги носят у Баха название «ричеркар»~--- это слово было также старинным названием музыкальной формы, известной сейчас как фуга. Во времена Баха название \emph{«фуга»} стало стандартным; термин же «ричеркар» приобрел новое значение. Теперь он обозначал изощренную, сложную фугу, возможно, слишком холодную и интеллектуальную для среднего слушателя. Подобное значение сохранилось и в других языках; французское (употребляющееся так­же и в английском) «recherche» означает что-то необычное и имеет смысловой оттенок эзотеричности и утонченной интеллектуальности.

Трио-соната~--- приятный отдых от холодной строгости фуг и канонов; она мелодична и радостна и местами звучит как танцевальная музыка. Однако и эта соната основана все на той же Королевской Теме! То, что Бах сумел~использовать эту строгую по форме тему для такой приятной интерлюдии, похоже на чудо.

Десять канонов \emph{«Музыкального приношения»} находятся в числе самых сложных канонов, написанных когда-либо Бахом. Любопытно, однако, что они не закончены. Это было сделано умышленно; каноны были своего рода головоломками, которые Бах задал королю. В те дни была популярна следующая музыкальная игра; давалась тема и вместе с ней~--- несколько «подсказок», в свою очередь довольно непростых. Играющие должны были «найти» канон, основанный на этой теме. Чтобы понять, как это возможно, читатель должен знать кое-что о канонах.

Каноны и фуги

Идея канона заключается в том, что одна и та же тема играется на фоне самой себя: «копии» темы повторяются в нескольких голосах. Существуют разные способы построения канонов; самые простые каноны~--- круговые, такие как «Дядя Ваня». Тема здесь начинается в первом голосе~--- спустя определенное время вступает второй голос, исполняя «копию» темы. Через то же время вступает третий голос, в свою очередь имитируя тему, и так далее. При этом все голоса исполняют тему в одной и той же тональности. Большинство мелодий не будут гармонировать сами с собой таким образом; для того, чтобы тема могла служить основой канона, каждая ее нота должна быть способной исполнять как минимум две роли: во-первых, быть частью мелодии и, во-вторых, быть частью гармонизации этой же мелодии. В трехголосном каноне, например, каждая нота темы должна к тому же участвовать в двух различных гармонизациях. Таким образом, каждая нота канона имеет несколько музыкальных значений; ухо и мозг слушателя автоматически выбирают нужное значение, исходя из контекста.

Разумеется, существуют и более сложные типы канонов. На следующей ступеньке находятся такие каноны, в которых копии темы отстоят друг от друга не только по \emph{времени,} но и по \emph{тональности} скажем, первый голос начинает с ноты до, а второй голос, накладываясь на первый, вступает на четыре ступени выше, с соль. Третий голос вступает опять на кварту выше, с ре, в свою очередь накладываясь на первый и второй голоса\ldots{} Следующая ступень сложности~--- каноны, в которых голоса исполняют мелодию в разном темпе, второй голос, например, вдвое быстрее или вдвое медленнее первого. Этот прием называется, соответственно, \emph{уменьшением} или \emph{увеличением,} и дает эффект сокращения или растягивания мелодии.

Это еще не все! Еще более сложные каноны используют \emph{обращенную} тему, «копия» мелодии \emph{обращает} все \emph{восходящие} ходы в \emph{нисходящие,} сохраняя в них те же интервалы. Это довольно странное музыкальное преобразование; однако, привыкнув к звучанию обращенных тем, слушатель находит их вполне естественными. Бах особенно любил обращения и часто использовал их в своих композициях~--- \emph{«Музыкальное приношение»} в этом смысле не составляет исключения. (Примером обращенного канона является «Good King Wenceslas» Скотта Кима, приведенный на рис. 4а.)

\emph{Рис. 4а. Канон «Добрый король Венсеслас».}

Пожалуй, самая причудливая из всех «копий»~--- «пятящаяся», в которой тема играется «задом наперед», с конца к началу. Канон, использующий этот прием, известен во многих языках под ласковым прозвищем «ракоход» или «крабий канон», поскольку он запечатлевает в музыке особенности походки этих милых созданий. Нет нужды говорить, что в Баховском \emph{«Музыкальном приношении»} есть ракоход (крабий канон). Обратите внимание на то, что каждый тип «копии» полностью сохраняет информацию, заложенную в первоначальной теме; это значит, что эта тема может быть легко восстановлена по любой своей копии. Такая сохраняющая информацию трансформация часто называется \emph{изоморфизмом;} в этой книге мы еще не раз обратимся к изоморфизмам разного рода.

Иногда бывает желательно ослабить строгость канонической формы. Немного отступив от точного копирования темы, можно достигнуть более полной гармонии. Некоторые каноны имеют к тому же «свободные» голоса, не повторяющие тему, а просто состоящие в приятном согласовании с «каноническими» голосами.

Каждый канон \emph{«Музыкального приношения»} построен на вариации Королевской темы; при этом Бах выжимает все возможное из замысловатых приемов, описанных выше. Иногда композитор даже комбинирует несколько из них в развитии одной темы. Например, в одном из трехголосных канонов «Приношения» под названием «Canon per Augmentationem, contario Motu» средний голос является свободным и исполняет Королевскую тему, в то время как два других голоса канонически «танцуют» выше и ниже Королевской темы, используя приемы увеличения и обращения. Другой канон носит загадочное название «Quaerendo invenietis» («Ищущий обрящет»). Все канонические головоломки Баха были решены; ответы на них нашел один из его учеников. Иоганн Филипп Кирнбергер. Однако читатель может попытаться найти и другие решения; очень вероятно, что возможности загадочных канонов Баха еще не исчерпаны до конца!

Теперь я должен вкратце объяснить, что такое фуга. Фуга похожа на канон тем, что основная мелодия и ее имитации исполняются несколькими голосами в различных тональностях, а также иногда в разном темпе, снизу вверх или от конца к началу. Однако фуга гораздо менее строга по форме, чем канон, что придает ей больший артистизм и эмоциональность. Безошибочной определяющей приметой фуги является её начало: один голос исполняет тему до конца. Затем вступает второй голос, четырьмя тонами выше или тремя тонами ниже. Первый голос в это время ведет дополнительную тему, подобранную так, чтобы дать ритмический, гармонический и мелодический контраст к основной теме. Последующие голоса вступают по очереди, исполняя основную тему, часто являющуюся аккомпанементом дополнительной темы; остальные голоса в это время занимаются тем, что, следуя прихотливой фантазии композитора, «украшают» фугу различными мелодиями. Когда все голоса «прибывают» к концу темы, правил больше не существует. Существуют, конечно, некоторые типичные приемы; но они не настолько стандартны, чтобы по ним, как по формулам, можно было бы строить фуги. Две фуги из \emph{«Музыкального приношения»} --- яркий пример композиций, которые никогда не могли бы быть «сочинены по формулам». В них есть нечто гораздо более глубокое, чем простая «фугообразность».

В целом, \emph{«Музыкальное приношение»} - одно из высших достижений Баха в области контрапункта. Оно само по себе является одной большой интеллектуальной фугой, где переплетаются множество идей и форм и на каждом шагу встречаются шутливые иносказания и тонкие намеки. Это прекрасное создание человеческого ума, которым мы не устанем восхищаться. (Все произведение замечательно описано в книге X. Т. Дэвида \emph{«Музыкальное приношение» И. С. Баха} (Н.T.David, «J.S.Bach's Musical Offering»).

Естественно растущий канон

Один из канонов \emph{«Музыкального приношения»} особенно необычен. Это трехголосный канон под названием «Canon per tonos» («Тональный канон»). Самый высокий голос исполняет Королевскую тему; два других голоса дают каноническую гармонизацию, основанную на второй теме, причем нижний голос ведет свою мелодию в до миноре (общая тональность всей фуги), а верхний - ту же мелодию, но на пять ступеней выше. Отличительным свойством этого канона является то, что в конце~--- или, вернее, когда нам \emph{кажется,} что канон заканчивается~--- его тональность меняется с до минора на ре минор. Бах каким-то образом ухитряется \emph{смодулировать} (поменять тональность) прямо под носом слушателей! Канон сконструирован таким образом, что его кажущийся финал неожиданно плавно переходит в начало; этот процесс можно повторить, придя на этот раз к тональности ми минор, которая в свою очередь оказывается началом! Эти последовательные модуляции уводят слушателя во все более далекие тональные «провинции», так что после нескольких из них он чувствует себя уже безнадежно далеко от начальной тональности. Однако, чудесным образом, после шести модуляций мы возвращаемся все к тому же до минору. Все голоса теперь звучат ровно на октаву выше, чем в начале - пьеса может быть естественным образом прервана на этом месте. Вы можете подумать, что Бах именно это и намеревался сделать~--- однако Бах, несомненно, упивался возможностью продолжать этот процесс бесконечно. Может быть, поэтому он и написал на полях «Пусть Королевская слава возрастает подобно этой модуляции». Чтобы подчеркнуть заложенную в описанном каноне возможность естественного бесконечного движения, я буду называть его «Естественно Растущий Канон».

В этом каноне Баха мы впервые сталкиваемся с примером \emph{«Странных Петель».} «Странная Петля» получается каждый раз, когда, двигаясь вверх или вниз по уровням иерархической системы, мы неожиданно оказываемся в исходном пункте. (В нашем примере это система музыкальных тональностей.) Иногда, описывая систему со Странной Петлей, я использую термин \emph{Запутанная Иерархия} . В дальнейшем тема Странных Петель прозвучит еще не раз. Иногда она будет спрятана, а иногда будет лежать на поверхности; иногда она будет проводиться слева направо, иногда~--- вверх ногами, а иногда~--- ракоходом. Мой совет читателю~--- «Quaerendo invenietis».

Эшер

Как мне кажется, самые яркие и впечатляющие зрительные реализации идеи Странных Петель представлены в работах голландского графика М. К. Эшера, жившего с 1898 по 1971 год Эшер был создателем одних из самых интеллектуально стимулирующих рисунков всех времен Многие из них берут свое начало в парадоксе, иллюзии или двояком значении. Среди первых поклонников графики Эшера оказались математики, это неудивительно, так как его рисунки часто основаны на математических принципах симметрии или структуры. Однако типичный рисунок Эшера представляет из себя нечто гораздо большее, чем только лишь симметрию или определенную структуру часто в его основе лежит некая идея, представленная в художественной форме В частности, Странная Петля - одна из наиболее часто повторяющихся в работах Эшера тем. Взгляните, например, на литографию \emph{«Водопад»} (рис. 5) и сравните ее бесконечно спускающуюся шестиступенчатую Петлю с бесконечно поднимающейся шестиступенчатой Петлей «Тонального канона». Сходство поистине удивительное! Бах и Эшер проводят одну и ту же тему в двух различных «ключах»: музыка и изобразительное искусство.

\emph{Рис. 5. М. К. Эшер. «Водопад».}

В работах Эшера встречаются различные типы Странных Петель: они могут быть расположены по порядку в зависимости от того, как туго они «затянуты». Литография \emph{«Подъем и спуск»} (рис. 6), на которой монахи плетутся по лестнице, навсегда уловленные Петлей, является самой свободной версией, так как Петля здесь содержит множество ступеней.

\emph{Рис. 6. М. К. Эшер. «Подъем и спуск».}

Более «тугая» Петля представлена в \emph{«Водопаде»,} который, как мы уже видели, содержит всего шесть ступеней. Читатель может возразить, что понятие «ступени» весьма неопределенно: к примеру, можно считать, что \emph{«Подъем и спуск»} имеет не сорок восемь (ступеньки), а всего четыре (лестничные клетки) уровня.

\emph{Рис. 7. М. К. Эшер. «Рука с зеркальным шаром».}

\emph{Рис. 8. М. К. Эшер. «Метаморфоза II».}

Действительно, подсчету ступеней-уровней всегда свойственна некоторая неопределенность; это верно не только для картин Эшера, но и для любых многоступенчатых иерархических систем. Позже мы постараемся глубже понять причину этой неопределенности. Однако не будем отвлекаться! Если затянуть Петлю еще туже, мы получим замечательную картину \emph{«Рисующие руки»} (рис. 135), на которой каждая из рук рисует другую~--- двуступенчатая Странная Петля. Наконец, самая тугая Петля представлена в \emph{«Картинной галерее»} (рис. 142): это картина картины, содержащей саму себя. Или это картина галереи, содержащей саму себя? Или города, содержащего самого себя? Или молодого человека, содержащего самого себя? (Между прочим, иллюзия, лежащая в основе \emph{«Подъема и спуска»} и \emph{«Водопада»} была изобретена не Эшером, а английским математиком Роджером Пенроузом в 1958 году. Однако тема Странных Петель появилась в работах Эшера уже в 1948 году, когда он создал свои \emph{«Рисующие руки» «Картинная галерея»} датируется 1956 годом.)

В концепции Странных Петель скрыта идея бесконечности, ибо что такое Петля, как не способ представить бесконечный процесс в конечной форме? Бесконечность играет важную роль во многих картинах Эшера. Копии какой-либо «темы» часто «вставлены» друг в друга, создавая зрительные аналогии с канонами Баха. Несколько таких структур можно увидеть на знаменитой Эшеровской гравюре \emph{«Метаморфоза»} (рис. 8). Она немного напоминает «Естественно Растущий Канон»: уходя все дальше и дальше от начального пункта, мы внезапно возвращаемся обратно к началу. В черепичных плоскостях \emph{«Метаморфозы»} уже есть намек на бесконечность; однако другие картины Эшера являют еще более смелые образы бесконечного, На некоторых его рисунках одна и та же тема «звучит» на нескольких уровнях реальности. Скажем, один из планов легко узнается как фантастический, в то время как другой представляет реальность. Сама картина, возможно, содержит только эти два плана; однако само наличие подобной «двусмысленности» приглашает зрителя увидеть самого себя как часть еще одного плана. Сделав этот шаг, он уже околдован предложенной Эшером возможностью бесконечной последовательности планов, где для каждого данного уровня существует высший, более «реальный», и низший, более «фантастичный» уровни. Такая ситуация сама по себе является достаточно удивительной и пугающей. Однако что произойдет, если цепь уровней к тому же будет не линейная, а замкнутая саму на себя, образуя Петлю? Что тогда будет реальностью, а что фантазией? Гений Эшера заключается в том, что он не только придумал, но и сумел изобразить десятки полуреальных, полумифических миров, миров, полных Странных Петель, куда он приглашает войти Зрителя.

Гёдель

\emph{Рис. 9. Курт Гёдель}

Во всех примерах Странных Петель, которые мы видели у Баха и Эшера, присутствует конфликт между конечным и бесконечным, конфликт, рождающий ощущение парадокса. Интуиция подсказывает, что здесь замешано нечто, связанное с математикой. В самом деле, не так давно~--- в нашем веке~--- было найдено математическое соответствие этого явления. Это открытие оказало огромное влияние на развитие логической мысли. Подобно Петлям Баха и Эшера, основанным на простых и привычных образах (музыкальная гамма, лестница), открытие Странных Петель в математических системах, принадлежащее К. Гёделю, берет свое начало в простых и интуитивных идеях. В самой упрощенной форме открытие Гёделя сводится к переводу на язык математики одного из старинных философских парадоксов, так называемого \emph{парадокса Эпименида} (или \emph{парадокса лжеца).} Критский философ Эпименид был автором бессмертного суждения: «Все критяне~--- лжецы». В более прямой форме парадокс звучит так: «Я лгу» или «Это высказывание~--- ложь». В дальнейшем, говоря о парадоксе Эпименида, я буду иметь в виду последний вариант. Это суждение грубо нарушает обычное представление о том, что все суждения делятся на истинные и ложные, так как если мы на минуту представим, что оно истинно, то тут же увидим, что мы ошиблись, и на самом деле суждение ложно. Точно так же, из предпосылки ложности этого суждения вытекает, что оно должно быть истинным, Попробуйте сами!

Парадокс Эпименида является Странной Петлей «в одну ступеньку», так же, как \emph{«Картинная галерея»} Эшера. Но какое отношение имеет он к математике? В этом как раз и заключается открытие, сделанное Гёделем. Он попытался использовать математические рассуждения для анализа самих же математических рассуждений. Идея заставить математику заняться «самоанализом» оказалась необычайно продуктивной; \emph{теорема Гёделя о неполноте,} пожалуй, самое важное её следствие. То, что эта теорема утверждает, и то, как это утверждение в ней доказывается, это разные вещи, которые мы подробно рассмотрим в дальнейшем. Саму теорему можно сравнить с жемчужиной, а метод доказательства~--- с устрицей, её скрывающей. Мы восхищаемся сияющей простотой жемчужины; устрица же является сложным живым организмом, в чьем нутре зарождается эта таинственно простая драгоценность.

Теорема Гёделя впервые увидела свет как «теорема VI» в его статье 1931 года «О формально неразрешимых суждениях в \emph{„Principia Mathematica``} и родственных системах, I». Теорема утверждает следующее:

Каждому \&\#969;-непротиворечивому рекурсивному классу \emph{формул k} соответствует рекурсивный \emph{символ классов~r} такой, что ни \emph{v} Gen r ни Neg (\emph{v} Gen r\emph{)} не принадлежат к Flg \emph{(к),} где \emph{v - свободная переменная r.}

В оригинале это было написано по-немецки; читатель, возможно, думает, что с тем же успехом можно было бы это на немецком и оставить. Постараемся привести перевод на более понятный язык.

\emph{Все непротиворечивые аксиоматические формулировки теории чисел содержат неразрешимые суждения.}

Это наша жемчужина.

В ней трудно увидеть Странную Петлю, потому что эта Петля спрятана в «устрице»~--- в доказательстве. Доказательство теоремы Гёделя о неполноте вращается вокруг автореферентного (описывающего самого себя) математического суждения, так же как парадокс Эпименида~--- вокруг такого суждения в языке. Говорить о языке, используя для этого сам язык, несложно; гораздо труднее вообразить, как может говорить само о себе математическое суждение о числах. На самом деле, уже для того, чтобы связать идею автореферентного суждения с теорией чисел, понадобился гениальный ум. Интуитивно придя к мысли о возможности такого суждения, Гёдель преодолел одну из основных трудностей. Само же создание автореферентного суждения было делом техники, раздуванием костра из блистательной искры мгновенного прозрения.

Мы остановимся на теореме Гёделя в последующих главах; но чтобы покуда не оставить читателя в полной тьме, я несколькими штрихами обрисую суть идеи в надежде на то, что это заставит вас задуматься. Для начала уясним, в чем здесь основная трудность. Математические суждения описывают свойства целых чисел (мы будем говорить здесь о суждениях теории чисел). Ни целые числа, ни их свойства не являются сами по себе суждениями. Суждения теории чисел не говорят ничего про суждения теории чисел; они не более как суждения теории чисел. В этом и заключается проблема; однако Гёдель сумел увидеть глубже того, что лежит на поверхности.

Гёдель предположил, что суждение теории чисел могло бы быть о суждении теории чисел (возможно даже о себе самом), если бы сами числа могли обозначать суждения. Иными словами, в центре его построения находится идея кода. В этом коде, обычно именуемом «Гёделевой нумерацией», символы и последовательности символов обозначаются числами. Таким образом, любое суждение теории чисел, будучи последовательностью специальных символов, получает Гёделев номер, что-то вроде телефонного номера или номерного знака машины. В дальнейшем, для ссылки на данное суждение используется соответствующий Гёделев номер. С помощью этого кодирующего трюка суждения теории чисел приобретают двоякое значение: они могут быть поняты как суждения теории чисел, а так же как \emph{суждения о суждениях} теории чисел.

После того, как Гёдель изобрел эту кодирующую схему, ему пришлось разработать в деталях способ перевода парадокса Эпименида на формальный язык теории чисел. Конечный результат «пересадки» Эпименида на формальную почву звучит так: «Это суждение теории чисел не имеет доказательства» (вместо «Это суждение теории чисел ложно»). Эта формулировка может создать немалую путаницу. так как «доказательство» для многих является весьма приблизительным понятием. В действительности, труды Геделя были лишь частью долгих поисков, предпринятых математиками в надежде выяснить, что же такое доказательства. Необходимо помнить тот факт, что доказательства являются таковыми только \emph{внутри жестких систем} теорем. В Гёделевской работе такой жесткой системой, к которой относится слово «доказательство», является огромный труд Бертрана Рассела и Альфреда Норта Уайтхеда \emph{«Principia Mathematical» («Основания математики»)} , опубликованный между 1910 и 1913 годами. Следовательно, Гёделево высказывание Г должно бы звучать более правильно как:

\emph{Это суждение теории чисел не имеет доказательств в системе «Оснований математики».}

Заметим, между прочим, что Гёделево высказывание Г само по себе не является теоремой Гёделя, так же как высказывание Эпименида не является замечанием «Высказывание Эпименида~--- парадокс». Теперь мы можем установить, какой эффект произвело открытие Г. В то время как высказывание Эпименида создает парадокс, потому что оно не является ни истинным, ни ложным, Гёделево высказывание Г~--- истинно, хотя и не доказуемо в системе \emph{«Оснований математики».} Из этого следует замечательный вывод: система \emph{«Оснований математики»} неполна, так как существуют истинные суждения теории чисел, не доказуемые методами самой теории (эти методы доказательства оказываются слишком «слабыми».)

\emph{«Основания математики»} явились первой, но далеко не последней жертвой удара. Выражение «и родственные системы» в заглавии Гёделевой статьи говорит о многом. Если бы результат, полученный Гёделем, указывал бы только на дефект в работе Рассела и Уайтхеда, другие математики могли бы попытаться исправить ошибки в «Основаниях математики» и «перехитрить» теорему Гёделя. Однако это оказалось невозможным: теорема Гёделя была приложима ко \emph{всем} аксиоматическим системам, ставившим своей целью то же, что и система Рассела и Уайтхеда. Для различных систем подходил один и тот же основной трюк. Короче, Гёдель показал, что понятие «доказуемости» уже, слабее понятия истинности вне зависимости от того, какую аксиоматическую систему мы выбираем.

Таким образом, теорема Гёделя произвела электризующий эффект на логиков, математиков и философов, заинтересованных в основах математики, поскольку она показала, что ни одна установленная система, какой бы сложной она не была, не может отразить всей сложности целых чисел: 0,1, 2, 3\ldots{} Современный читатель, возможно, не окажется от этого в таком замешательстве, как читатели 1931 года, так как за прошедшее время наша культура впитала теорему Гёделя вместе с революционными идеями теории относительности и квантовой механики, и широкая публика получила доступ к этим концепциям, поражающим и дезориентирующим мышление даже в смягченном прослойкой переводов (а зачастую и затемненном этими переводами) виде. Сейчас идея «ограничивающих» результатов витает в воздухе; тогда, в 1931 году, она была как гром с ясного неба.

Математическая логика: краткий обзор

Чтобы полностью оценить теорему Гёделя, необходим определенный контекст. Я попытаюсь здесь дать обзор истории математической логики до 1931 года на нескольких страницах~--- невозможная задача! (Хорошее изложение истории этого предмета читатель может найти у Делонга, Нибоуна, или Нагеля и Ньюмена). Все началось с попытки механизировать мыслительный процесс логических рассуждений. Обратите внимание, что умение мыслить всегда рассматривалось как отличительная черта человека; на первый взгляд, желание механизировать самую человеческую черту кажется парадоксальным. Тем не менее, уже древние греки знали, что логическое мышление - структурный процесс, до некоторой степени управляемый определенными законами. Эти законы можно описать. Аристотель систематизировал силлогизмы, а Эвклид~--- геометрию; однако с тех пор прошло много веков до того, как в изучении логического мышления снова наступила эра прогресса.

Одним из важнейших открытий геометров девятнадцатого столетия были различные геометрии, равно имеющие право на существование. Под \emph{геометрией} здесь понимается теория, описывающая свойства абстрактных точек и линий. До этого считалось, что геометрия~--- это система, кодифицированная Эвклидом; она могла иметь незначительные недостатки, которые могли быть со временем исправлены. Таким образом, любой прогресс в этой области означал исправление и дополнение Эвклида. Это убеждение было разбито вдребезги, когда несколько математиков почти одновременно открыли неэвклидову геометрию~--- открытие, потрясшее математический мир, поскольку оно сильно поколебало бытовавшее мнение, что математика изучает реальную действительность. Каким образом в одной и той же реальности могли существовать различные типы точек и линий? Сегодня решение этой дилеммы может быть очевидно даже для некоторых далеких от математики людей, но в то время она посеяла панику в математических кругах.

Позже в девятнадцатом веке английские логики Джордж Буль и Август де Морган пошли значительно дальше Аристотеля в кодификации строго дедуктивных~рассуждений. Буль даже назвал свою книгу \emph{«Законы мысли»,} что, безусловно, было некоторым преувеличением; однако его попытки внесли серьезный вклад в общие усилия. Льюис Кэрролл был очарован механическими методами рассуждений и изобрел множество головоломок, решавшихся с помощью этих методов. Готтлоб Фреге в Йене и Джузеппе Пеано в Турине работали над соединением формальных рассуждений с изучением чисел и множеств. Дэвид Гильберт в Геттингене трудился над более строгой, чем у Эвклида, формализацией геометрии. Все эти усилия были направлены на выяснение вопроса о том, что же такое «доказательство».

Между тем, в классической математике тоже происходили интересные события. В 1880-х годах Георг Кантор развил теорию о различных типах бесконечности, известную под именем \emph{теории множеств.} Теория Кантора была глубока и красива, но шла вразрез с интуицией; вскоре на свет появилось целое семейство парадоксов, основанных на теории множеств. Ситуация была не из приятных. Только математики начали оправляться от удара, нанесенного по математическому анализу парадоксами, связанными с теорией пределов, как попали из огня в полымя из-за нового, еще худшего набора парадоксов!

Самый известный из них~--- парадокс Рассела. По всей видимости, большинство множеств не являются элементами самих себя: скажем, множество моржей~--- это не морж; множество, содержащее только одного члена, Жанну д'Арк, само не является Жанной (множество не человек!), и так далее. В этом смысле, большинство множеств совершенно заурядны. Однако существуют такие «самозаглатывающие» множества, которые содержат самих себя, как, например, множество всех множеств, или множество всех вещей за исключением Жанны Д'Арк, и тому подобные. Ясно, что множества могут быть только одного из этих двух типов~--- либо заурядные, либо самозаглатывающие~--- и ни одно множество не может входить сразу в два класса. Однако ничто не мешает нам изобрести множество \textbf{R} всех заурядных множеств. На первый взгляд, \textbf{R} кажется довольно заурядным изобретением, но вам придется пересмотреть свое мнение, если вы спросите себя, является ли множество \textbf{R} самозаглатывающим или заурядным. Вы придете к следующему ответу: \textbf{R} не является ни тем, ни другим, так как любой из этих двух ответов приводит к парадоксу. Попробуйте и убедитесь сами!

Но если \textbf{R} не заурядное и не самозаглатывающее, тогда что же оно такое? По меньшей мере, ненормальное. Однако такой уклончивый ответ никого не удовлетворял. Тогда люди стали пытаться докопаться до основ теории множеств; при этом они задавали себе следующие вопросы: «В чем заключается ошибка нашего интуитивного понимания понятия „множество``? Можно ли создать строгую теорию множеств, которая бы не противоречила нашей интуиции и в то же время исключала бы парадоксы?» Здесь, так же как и в теории чисел и в геометрии, проблема заключалась в том, чтобы примирить интуицию с формальными, аксиоматическими системами логических рассуждений.

Удивительный вариант парадокса Рассела, называющийся парадоксом Греллинга, получается, если вместо множеств использовать прилагательные. Разделите все прилагательные русского языка на две категории: те, которые описывают самих себя, \emph{«самоописывающие»,} («пятисложное», «шелестящий,» «пренеестественнейший» и т. п.), и те, которые таким свойством не обладают («съедобный», «двусложный», «кратчайший»). Рассмотрим теперь прилагательное «несамоописывающий». К какому классу оно относится? Попробуйте ответить!

У всех этих парадоксов есть общий виновник: автореферентность, или «страннопетельность». Таким образом, если наша цель~--- избавиться от всех парадоксов, то почему бы нам не попытаться избавиться от автореферентности и тех условий, которые ее порождают? Это не так легко, как кажется, так как иногда бывает трудно найти, где именно происходит автореференция. Иногда она бывает распределена по Странной Петле в несколько ступеней, как в следующей расширенной версии парадокса Эпименида, напоминающей Эшеровские \emph{«Рисующие руки» ---}

\emph{Следующее высказывание ложно} .

\emph{Предыдущее высказывание истинно} .

Вместе эти высказывания производят такой же эффект, как первоначальный парадокс Эпименида; однако взятые по отдельности они безобидны и даже полезны Ни одно из них не может нести ответственности за Странную Петлю; виновато их объединение, то, как они указывают друг на друга. Точно так же каждый взятый по отдельности кусок \emph{«Подъема и спуска»} совершенно правилен; невозможно лишь подобное соединение этих кусков в одно целое Видимо, существуют прямой и косвенный типы автореферентности; если мы считаем, что в автореферентности~--- корень зла, то мы должны найти способ избавиться сразу от обоих типов.

Изгнание Странных Петель

Рассел и Уайтхед считали именно таких труд «Основания математики» («ОМ») был титаническим усилием, направленным на изгнание Странных Петель из логики, теории множеств и теории чисел. В основе их системы лежала следующая идея. Множество «низшего» типа могло иметь своими элементами лишь «предметы», а не множества. На следующей ступени стояли множества, которые могли содержать предметы или множества первого типа. Вообще, любое данное множество могло содержать лишь множества низшего типа или предметы. Каждое множество принадлежало к определенному типу. Ясно, что никакое множество не могло содержать самого себя, так как оно оказалось бы тогда принадлежащим к более высокому типу, чем его собственный. В такой системе существуют лишь обыкновенные множества; более того, наш старый знакомец, множество \textbf{R} , теперь вообще не считается множеством, так как оно не принадлежит ни к одному конечному типу! По всей видимости, эта \emph{теория типов,} которую мы также могли бы именовать «теорией уничтожения Странных Петель», преуспела в избавлении теории множеств от парадоксов~--- но только ценой введения искусственной иерархии и запрета на определенный тип множеств, такой, например, как множество всех «заурядных» множеств. Интуитивно это идет вразрез с нашим представлением о множествах.

Теория типов справилась с парадоксом Рассела, но ничего не предприняла в отношении парадоксов Эпименида или Греллинга. Для тех, чей интерес не шел дальше теории множеств, этого было достаточно; однако людям, заинтересованным в уничтожении парадоксов вообще, казалось необходимым создание подобной иерархии в языке, чтобы изгнать оттуда Странные Петли. На первой ступеньке такой иерархии стоял бы \emph{предметный язык,} на котором возможно говорить лишь об определенной сфере предметов, но нельзя говорить о самом предметном языке, обсуждать его грамматику или какие-либо высказывания, для этого понадобился бы \emph{метаязык.} (Опыт двух различных лингвистических уровней знаком любому, кто изучал иностранные языки.) В свою очередь, что­бы говорить о метаязыке, потребовался бы метаметаязык, и так далее. Каждое высказывание должно было принадлежать к определенному уровню иерархии. Таким образом, если бы мы не могли найти для данного высказывания места в иерархической структуре, мы должны были бы считать такое высказывание бессмысленным и как можно скорее выбросить его из головы.

Можно попытаться проанализировать таким образом двуступенчатую петлю Эпименида, приведенную выше. Первое высказывание, поскольку оно говорит о втором, должно быть уровнем выше последнего; однако точно такое же рассуждение применимо и ко второму высказыванию. Поскольку это невозможно, оба высказывания «бессмысленны». Точнее, они вообще не могут существовать в системе, основанной на строгой иерархии языков. Это предупреждает возникновение любых версий парадокса Эпименида или Греллинга (К какому уровню принадлежит «самоописывающий»?)

В теории множеств, имеющей дело с абстракциями, далекими от повседневной жизни стратификация теории типов еще приемлема, хотя и выглядит натянутой. Когда же дело доходит до языка, важнейшей и ежедневно употребляемой части нашей жизни, такая стратификация кажется абсурдом. Трудно поверить что, разговаривая, мы скачем вверх и вниз по иерархии языков. Довольно обычное высказывание, такое как, например, «В этой книге я критикую теорию типов», было бы дважды запрещено в подобной системе. Во-первых, оно упоминает «эту книгу», которая должна бы упоминаться только в «метакниге», и во-вторых, оно упоминает обо мне~--- существе, о котором я не должен бы говорить вообще. Этот пример показывает, насколько нелепо выглядит теория типов в повседневном контексте. В данном случае, лекарство хуже самой болезни метод, используемый этой теорией, чтобы избавиться от парадоксов, заодно объявляет бессмыслицей множество вполне правильных конструкций. Эпитет «бессмысленный» кстати, был бы приложим к любому обсуждению теории лингвистических типов (и в частности, к данному параграфу), так как ясно, что никакое из них не может принадлежать ни к одному из уровней~--- ни к предметному ни к метаязыку, ни к метаметаязыку, и т. д. Таким образом, сам акт обсуждения теории оказывался бы ее грубейшим нарушением.

Конечно, мы могли бы попытаться защитить подобные теории, обговорив, что они имеют дело только с формальными языками, а не с повседневным, обыкновенным языком. Может, оно и так, но тогда такие теории оказываются чисто академическими и имеют дело с парадоксами только тогда, когда те возникают в специальных сделанных по заказу системах. К тому же, стремление уничтожить парадоксы любой ценой, особенно ценой создания чрезвычайно~искусственных формализмов, придает слишком много значения плоской последовательности и логичности, и слишком мало~--- тому причудливому и замысловатому, что придает вкус жизни и математике. Вне сомнения, стараться быть последовательным важно, но когда это старание приводит к созданию удивительно неуклюжих и уродливых теорий, становится ясно, что здесь что-то не в порядке.

В начале двадцатого века, проблемы подобного типа в основах математики вызвали живой интерес к кодификации методов логического мышления. Математики и философы начали сомневаться в том, что даже самые конкретные теории, такие, как теория чисел, построены на прочном фундаменте. Если парадоксы могли возникнуть в теории множеств, основанной на простых интуитивных понятиях, то почему бы им не проникнуть и в другие области математики? А что, если логические парадоксы, такие как парадокс Эпименида, свойственны математике в целом, и, таким образом, ставят всю ее под сомнение? Подобные проблемы тревожили в первую очередь тех~--- а их было немало~--- кто твердо верил в то, что математика~--- лишь один из разделов логики (или, наоборот, что логика~--- лишь один из разделов математики). Уже сам этот вопрос, «являются ли математика и логика отдельными и непохожими дисциплинами?», вызывал горячие споры.

Изучение самой математики получило название \emph{метаматематики} или, иногда, \emph{металогики,} поскольку математика и логика тесно переплетены. Важнейшей задачей метаматематиков было определение природы математических рассуждений. Что является законным методом рассуждений и что~--- незаконным? Поскольку рассуждения велись на каком-либо «естественном языке», скажем, французском или латинском, всегда были возможны двусмысленные и неясные толкования. Одно и то же слово может иметь разные значения для разных людей, вызывать различные образы, и так далее. Хорошей и важной идеей казалось установление единой нотации, с помощью которой велись бы все математические рассуждения, так чтобы два математика всегда могли договориться о том, верно ли предложенное доказательство. Эта задача потребовала бы кодификации всех общепринятых методов человеческих рассуждений, по крайней мере постольку, поскольку они~приложимы к математике.

Последовательность, полнота, и программа Гильберта

Такая кодификация являлась основной идеей системы «\emph{Оснований математики»} («ОМ»), авторы которой задались целью вывести всю математику из логики, причем без малейших противоречий! Многие восхищались их грандиозным трудом, но никто не был уверен в том, что 1) методы Рассела и Уайтхеда действительно описывают всю математику и 2) эти методы достаточно последовательны и корректны. Действительно ли при следовании этим методам \emph{никогда и не при каких условиях} не могло возникнуть парадоксов?

Этот вопрос особенно тревожил знаменитого немецкого математика (и метаматематика) Дэвида Гильберта, кто поставил перед математиками (и метаматематиками) всего мира следующую задачу: со всей строгостью доказать, возможно, при помощи самих методов Рассела и Уайтхеда, что эти методы, во-первых, непротиворечивы и во-вторых, полны (иными словами, что в системе «ОМ» может быть выведено любое истинное высказывание). Эта задача весьма непростая, и ее можно критиковать за некоторую «порочную кругообразность», как можно пытаться доказать какие-либо методы рассуждения, пользуясь этими же методами? Это все равно, что пытаться поднять самого себя на воздух за шнурки от собственных ботинок. (Кажется, нам-таки никуда не деться от этих Странных Петель)

Гильберт, разумеется, полностью отдавал себе отчет в этой дилемме; однако он надеялся, что доказательство полноты и непротиворечивости удастся найти с помощью только небольшой группы так называемых «финитных» методов рассуждения, признаваемых большинством математиков. В этом смысле Гильберт надеялся, что математикам все же удастся «поднять самих себя на воздух за шнурки ботинок», доказав правильность \emph{всех} математических методов путем использования лишь нескольких из них. Эта цель может показаться слишком эзотерической, однако именно она занимала умы многих великих математиков в первые тридцать лет двадцатого столетия.

Однако в тридцать первом году Гёдель опубликовал работу, подорвавшую основы Гильбертовой программы. Эта работа показала не только наличие незаполнимых «дыр» в аксиоматической системе, предложенной Расселом и Уайтхедом, но и то, что ни одна аксиоматическая система не может породить все истинные высказывания теории чисел, если она не является противоречивой! Наконец, Гёдель показал, насколько тщетна надежда доказать непротиворечивость системы «ОМ» если бы такое доказательство было найдено только при помощи методов, используемых в «ОМ»~--- и это одно из самых удивительных следствий Гёделевской работы~--- сами «ОМ» оказались бы противоречивы!

Последний иронический штрих для доказательства теоремы Гёделя о неполноте потребовалось внедрить парадокс Эпименида прямо в сердце «Оснований математики»~--- бастиона, считавшегося недоступным для Странных Петель. Хотя Гёделева Странная Петля и не разрушила «Оснований математики», она сделала их гораздо менее интересными для математиков, доказав иллюзорность цели, первоначально поставленной Расселом и Уайтхедом.

Баббидж, компьютеры, искусственный разум...

Как раз когда работа Гёделя вышла в свет, мир был накануне создания электронных цифровых компьютеров. Идея механических счетных машин носилась в воздухе уже давно В семнадцатом веке Паскаль и Лейбниц разработали машины для выполнения установленных операций сложения и умножения. К сожалению, эти машины не имели памяти и не были, в современном понимании этого слова, программируемыми

Первым человеком, понявшим, какой огромный счетный потенциал заключают в себе машины, был лондонец Чарльз Баббадж (Charles Babbage, 1792- 1871), фигура, словно сошедшая со страниц «Пиквикского клуба». При жизни он был известен более всего тем, что вел энергичные кампании по очистке Лондона от «нарушителей спокойствия», в первую очередь, шарманщиков.

Эти паразиты любили подразнить Баббаджа и исполняли для него «серенады» в любой час дня и ночи, а он, в ярости, гнал их вдоль по улице. Сегодня мы признаем, что Баббадж был человеком, обогнавшим свое время лет на сто он не только изобрел основные принципы современных компьютеров, но и был первым борцом за охрану окружающей среды от шума.

Его первое изобретение, «разностная машина», могла вычислять математические таблицы многих типов по «методу разностей». Однако, прежде чем была создана первая модель «РМ», Баббаджем завладела идея гораздо более революционная его «аналитическая машина». Довольно нескромно, Баббадж писал: «Я пришел к этой мысли таким сложным и запутанным путем, какой, возможно, впервые прошел человеческий ум».\footnote{Charles Babbage «Passages from the Life of a Philosopher» стр 145 6} В отличие от созданных ранее машин, «AM» должна была иметь «склад» (память) и «фабрику» (считающее и принимающее решения устройство). Оба устройства должны были быть построены из тысяч цилиндров, сцепленных самым сложным и причудливым образом. Баббадж представлял себе числа, влетающие и вылетающие из «фабрики» под контролем некоторой \emph{программы,} содержащейся в перфорированных картах~--- на эту идею его натолкнул ткацкий станок Жаккара, изготовлявший при помощи подобных карт удивительно сложные узоры. Подруга Баббаджа графиня Ада Лавлейс, дочь Байрона, женщина незаурядного таланта и горькой судьбы, поэтично про­комментировала: «Аналитическая машина ткет алгебраические узоры, наподобие того, как станок Жаккара ткет узоры из цветов и листьев». К сожалению, использование графиней настоящего времени вводит читателя в заблуждение: «AM» так никогда и не была построена, и Баббадж умер горько разочаровавшимся человеком.

Леди Лавлейс не менее, чем Баббадж, отдавала себе отчет в том, что, пытаясь создать аналитические машины, человечество флиртовало с искусственным разумом~--- в особенности, если эти машины способны «укусить себя за хвост» (так Баббадж описывал Странную Петлю, получавшуюся, когда его машина «залезала внутрь себя» и меняла заложенную в нее программу). В 1842 году она написала в своих мемуарах,\footnote{Lady A. A. Lovelace «Notes upon the Memoir „Sketch of the Analytical Engine Invented by Charles Babbage``» записано L. F. Menabrea (Женева 1842) и воспроизведено в книге~E. Morrison «Charles Babbage and His Calculating Engines» стр. 248 9 284} что аналитическая машина «может воздействовать не только на цифры, но и на другие вещи». В то время, как Баббадж мечтал о создании шахматного или «крестико-ноликового» автомата, леди Лавлейс предположила, что если записать на цилиндры машины тона и гармонии, то она могла бы «создавать искусно сделанные научные музыкальные композиции любой сложности и длины». Впрочем, там же она объясняет: «Аналитическая машина не претендует на создание чего-то нового, она может делать только то, что мы умеем ей приказать». Верно поняв, какая мощь заложена в механических вычислениях, она, тем не менее, оставалась скептически настроенной по отношению к механическому разуму. Однако могла ли она, со всей своей проницательностью, предположить, какие возможности откроются, когда человечество подчинит себе электричество?

В нашем веке пришло время для компьютеров, превзошедших самые смелые мечты Паскаля, Лейбница, Баббаджа или леди Лавлейс. В 1930-х и 1940-х годах были разработаны и построены первые «блестящие электронные головы». Это послужило катализатором к соединению трех ранее совершенно различных областей науки, теории аксиоматических рассуждений, изучения механических вычислений и исследований по психологии человеческого разума. В те же годы гигантскими скачками двигалась вперед теория компьютеров. Эта теория была тесно связана с математикой. Фактически, теорема Гёделя имеет параллель в теории вычислений: Алан Тюринг открыл существование неизбежных «дыр» в возможностях даже самого могучего компьютера. Словно в насмешку, как раз когда делались эти довольно мрачные прогнозы, строились новые компьютеры, чьи возможности росли на глазах, далеко превосходя самые смелые предсказания их создателей. Баббадж, сказавший однажды, что он с радостью отдал бы остаток жизни за возможность вернуться на три дня лет через пятьсот, чтобы получить возможность ознакомиться с наукой будущего, возможно, потерял бы дар речи от удивления уже через сто лет после своей смерти, пораженный как новыми машинами, так и их неожиданными ограничениями.

В начале 1950-х годов казалось, что до механического разума~--- рукой подать: однако за каждой преодоленной вершиной вставала новая, препятствуя созданию по-настоящему думающей машины. Возможно ли, что это упорное отдаление цели имело глубинные причины?

Никто не знает, где пролегает граница между разумным и не-разумным поведением; в самом деле, возможно, что само предположение о существовании четкой границы звучит глупо. Однако мы с уверенностью можем перечислить основные критерии разума:

гибко реагировать на различные ситуации;

извлекать преимущество из благоприятного стечения обстоятельств;

толковать двусмысленные или противоречивые сообщения;

оценивать различные элементы данной ситуации по степени их важности;

находить сходство между ситуациями, несмотря на возможные различия;

находить разницу между ситуациями, несмотря на возможное сходство;

создавать новые понятия, по-новому соединяя старые;

выдвигать новые идеи.

Здесь мы сталкиваемся с кажущимся парадоксом. Компьютеры, по определению, являются самыми негибкими, безвольными и послушными приказам существами. Несмотря на свою быстроту, они, тем не менее, сама бессознательность. Как же, в таком случае, можно запрограммировать разумное поведение? Не является ли уже само это предположение кричащим противоречием? Одна из основных идей этой книги~--- показать, что это вовсе не противоречие. Одна из основных целей этой книги~--- побудить каждого читателя встретиться с кажущимся парадоксом во всеоружии, попробовать его на вкус, вывернуть его наизнанку, разобрать его на части, пошлепать в нем, как ребенок в луже, чтобы в результате читатель смог взглянуть по-новому на кажущуюся неприступной пропасть между формальным и неформальным, одушевленным и неодушевленным, гибким и негибким

Это и составляет предмет исследований науки об искусственном интеллекте (ИИ). Работа специалистов по ИИ кажется странной и удивительной именно потому, что они разрабатывают строго формальные правила, говорящие негибким машинам, как стать гибкими

Что же это за правила такие, могущие описать всю сложность поведения разумных существ? Безусловно, это должны быть правила самых разных уровней: «простые» правила, «метаправила» для модификации «простых», «метаметаправила» для модификации метаправил, и так далее. Гибкость нашего разума зависит именно от огромного количества правил и сложности их иерархии. Некоторые ситуации вызывают стереотипные реакции, для которых годятся «простые» правила. Другие ситуации представляют собой комбинации из стереотипных ситуаций; тут нужны правила, говорящие, какие из «простых» правил приложимы к данной ситуации. Некоторые ситуации вообще не поддаются классификации~--- следовательно, требуются правила для изобретения новых правил\ldots{} ит. д., и т. п. Без сомнения, Странные Петли, правила, изменяющие сами себя, находятся в самом сердце разума. Иногда сложность нашего разума кажется нам настолько поразительной, что у нас опускаются руки перед задачей понять и описать его; тогда нам кажется, что никакие, даже самые замысловатые иерархические правила не способны управлять поведением разумных существ.

...и Бах

В 1754 году, четыре года спустя после смерти И. С. Баха, лейпцигский теолог Иоганн Микаэль Шмидт написал в своем труде о музыке и о душе следующие достойные внимания строки:

Не так давно из Франции сообщили, что там сделана была статуя, способная исполнить несколько пьес на Fleuttraversiere; статуя эта подносит флейту к губам и затем ее опускает, двигает глазами и т. д. Однако никто еще не изобрел образа, который бы думал, желал, сочинял или делал бы что-либо отдаленно подобное. Пусть любой, кто желает в этом убедиться, обратится к последним фугам Баха, которому мы уже воздали почести ранее. (Эти фуги были выгравированы на меди, но не были закончены, так как этому помешала слепота композитора.) Пусть увидит наблюдатель, какое искусство содержится в этой музыке~--- еще более он будет поражен чудесным Хоралом, который был записан под диктовку слепого Баха. «Wenn wir in h\&\#337;chen Nothen seyn». Я уверен, что наблюдателю вскоре понадобится душа, ежели он желает прочувствовать всю содержащуюся в этой музыке красоту или, более того, исполнить эту музыку и составить суждение об авторе. Все аргументы чемпионов Материализма должны рассыпаться в прах лишь от одного этого примера.\footnote{David and Mendel стр. 255 6}

Скорее всего, под главным «чемпионом Материализма» здесь имеется в виду не кто иной как Жюльен Оффрой де Ламеттри, придворный философ Фридриха Великого, автор книги «Человек как машина» и материалист до мозга костей. С тех пор прошло более двухсот лет, но битва между сторонниками Иоганна Микаэля Шмидта и Жюльена Оффроя де Ламеттри все еще в полном разгаре. В этой книге я надеюсь дать читателю некоторую перспективу этой битвы.

«Гёдель, Эшер, Бах»

Эта книга построена необычно~--- как контрапункт между Диалогами и Главами. С помощью такой структуры я смог вводить новые понятия дважды: каждое из них сначала представлено в метафорической форме в диалоге, что дает читателю конкретные зрительные образы; эти образы затем служат интуитивным фоном для более серьезного, абстрактного обсуждения того же понятия. Многие Диалоги создают поверхностное впечатление, что я говорю о какой-то определенной идее, когда на самом деле я имею в виду совсем иную идею, тщательно замаскированную.

Сначала единственными действующими лицами моих Диалогов были Ахилл и Черепаха, пришедшие ко мне от Зенона из Элей через посредство Льюиса Кэрролла. Зенон, изобретатель парадоксов, жил в 5 веке до н.э. Один из его парадоксов был аллегорией, в которой действовали Ахилл и Черепаха. История изобретения Зеноном этой счастливой парочки рассказана в первом Диалоге, «Трехголосная инвенция». В 1895 году Льюис Кэрролл воссоздал Ахилла и Черепаху для иллюстрации своего собственного нового парадокса о бесконечности. Парадокс Кэрролла, заслуживающий гораздо большей популярности, играет значительную роль в этой книге. В оригинале он называется «Что Черепаха сказала Ахиллу»~--- здесь он приведен как «Двухголосная инвенция».

Вскоре после того, как я начал писать Диалоги, каким-то образом они связались в моим воображении с музыкальными формами. Не помню того момента, когда это произошло, помню лишь, как однажды я в задумчивости написал «фуга» над текстом одного из ранних Диалогов. Идея привилась, и с тех пор я стал писать Диалоги, формально составленные по образцу различных композиций Баха. Это оказалось неплохой мыслью. Сам Бах часто напоминал своим ученикам, что различные части их композиций должны вести себя как «люди, беседующие друг с другом в избранном обществе». Возможно, что я вложил в этот совет более буквальный смысл, чем Бах, надеюсь все же, что результат оказался верен также и его духу. Особенно меня вдохновили некоторые поразительные аспекты Баховских композиций, которые так прекрасно описаны Менделем и Давидом в их книге «Баховская хрестоматия» (Mendel \& David, «The Bach Reader»):

Форма у Баха в основном опиралась на соотношения между отдельными частями от полного сходства с одной стороны до повторения какого-либо одного композиционного принципа или просто мелодической переклички с другой стороны. Получившиеся композиции часто бывали симметричными но это никоим образом не являлось необходимым следствием. Иногда соотношения между частями создают запутанный клубок, который можно распутать только путем детального анализа. Обычно впрочем, несколько доминирующих черт позволяют сориентироваться с первого взгляда или прослушивания, хотя при дальнейшем изучении мы можем открыть для себя множество тонкостей нас никогда не покидает чувство единства, связывающего каждое произведение Баха в одно гармоничное целое.\footnote{Там же стр. 40}

Я решил попытаться сплести Бесконечную Гирлянду из этих трех прядей Гедель, Эшер, Бах. По началу я планировал написать эссе, центральной темой которого была бы теорема Геделя о неполноте. Я думал, что у меня получится тоненькая брошюрка, однако мой проект стал расти, как снежный ком, и вскоре затронул Баха и Эшера. Некоторое время я не знал, выразить ли эту связь открыто или же оставить ее при себе как источник собственного вдохновения. В конце концов я понял, что Гедель, Эшер и Бах для меня~--- только тени, отбрасываемые в разные стороны некой единой центральной сущностью. Я попробовал реконструировать этот центральный объект, результатом моей попытки явилась эта книга.


% \subsubsection{Трехголосная инвенция}
% \subsubsection{Трехголосная инвенция}

\emph{Ахилл (греческий воин, самый быстроногий из смертных) и Черепаха стоят рядом на пыльной беговой дорожке; жара, палит солнце. Далеко в конце дорожки на высоком флагштоке висит большой прямоугольный ярко-красный флаг. В центре флага вырезана дыра в форме кольца, сквозь которую видно небо.}

\emph{Ахилл} : Что~это за странный флаг там, на другом конце дорожки? Он чем-то напоминает мне гравюру моего любимого художника, Эшера.

\emph{Черепаха} : Это флаг Зенона.

\emph{Ахилл} : Не кажется ли вам, что дыра в нем похожа на отверстия в листе Мёбиуса на одной из картин Эшера? Могу поспорить, что с этим флагом что-то не в порядке.

\emph{Черепаха} : В нем вырезано кольцо в форме нуля --- любимого числа Зенона.

\emph{Ахилл} : Но ведь в то время нуль еще не был изобретен! Он будет придуман неким индусским математиком только несколько тысяч лет спустя. Это доказывает, дорогая г-жа Ч, что подобный флаг невозможен.

\emph{Черепаха} : Ваши доводы убедительны, Ахилл, и я должна согласиться, что такой флаг, действительно, не может существовать. Но все равно он замечательно красив, не правда ли?

\emph{Ахилл} : В этом я не сомневаюсь.

\emph{Черепаха} : Интересно, не связана ли его красота с его невозможностью? Не знаю, не знаю.. У меня никогда не доходили лапы до анализа Красоты. Это Сущность с Большой Буквы, а у меня никогда не хватало времени на Сущности с Большой Буквы.

\emph{Ахилл} : Кстати, о Сущностях с Большой Буквы --- вы никогда не задавались вопросом о Смысле Жизни?

\emph{Черепаха} : Бог мой, конечно же, нет!

\emph{Ахилл} : Не спрашивали ли вы себя, зачем мы здесь и кто нас изобрел?

\emph{Черепаха} : Ну, это совершенно другое дело. Нас изобрел Зенон (в чем вы сами скоро убедитесь); мы находимся здесь, чтобы бежать наперегонки.

\emph{Ахилл} : Мы --- наперегонки?. Это возмутительно! Я, самый быстроногий из смертных --- и вы медлительная, как\ldots{} как\ldots{} как Черепаха!

\emph{Черепаха} : Вы могли бы дать мне фору.

\emph{Ахилл} : Это была бы огромная фора.

\emph{Черепаха} : Ну что же, я не возражаю.

\emph{Ахилл} : Все равно я вас нагоню, раньше или позже --- скорее всего, раньше.

\emph{Черепаха} : А вот и нет, если верить парадоксу Зенона. Зенон надеялся с помощью нашего маленького соревнования доказать, что движение невозможно. По Зенону, движение происходит только в нашем воображении. Это значит, что Мир Изменяется Исключительно Иллюзорно. Он доказывает этот постулат весьма элегантно.

\emph{Ахилл} : Ах, да, теперь я припоминаю~ знаменитый коан мастера дзен-буддизма Дзенона\ldots{} тьфу!. Зенона, я имею в виду. Действительно, очень просто.

\emph{Черепаха} : Дзен коан? Дзен мастер? О чем вы говорите?

\emph{Ахилл} : Вот, послушайте\ldots{} Два монаха спорили о флаге Один сказал; «Этот флаг движется». Другой возразил: «Нет, это ветер движется». В это время мимо проходил шестой патриарх, Зенон, который сказал монахам: «Не флаг и не ветер --- движется ваша мысль!»

\emph{Рис. 10. М.К. Эшер. «Лист Мёбиуса I» (гравюра на дереве, отпечатанная с четырех блоков, 1961).}

\emph{Черепаха} : Что-то вы все путаете, Ахилл. Зенон вовсе не мастер дзен-буддизма. На самом деле, он греческий философ из города Элей, лежащего на полпути между точками А и Б. Спустя столетия, его все еще будут славить как автора парадоксов движения. В центре одного из них --- наше соревнование по бегу.

\emph{Ахилл} : Вы меня совсем сбили с толку. Я отчетливо помню, как много раз повторял наизусть имена шести патриархов дзена: «Шестой патриарх --- Зенон, шестой патриарх --- Зенон...» (Внезапно поднимается теплый ветер.) Взгляните, госпожа Черепаха, как развевается флаг! Как приятно смотреть на волны, бегущие по его мягкой ткани. И кольцо, вырезанное в нем, развевается вместе с флагом!

\emph{Черепаха} : Не смешите меня. Этот флаг в принципе невозможен, следовательно, он не может развеваться. Это движется ветер.

\emph{(В этот момент мимо идет Зенон.)}

\emph{Зенон} : День добрый! Приветствую вас! Что слышно?

\emph{Ахилл} : Флаг движется!

\emph{Черепаха} : Ветер движется!

\emph{Зенон} : О мои дражайшие друзья! Прекратите ваши словопрения! Оставьте ваши разногласия! Поберегите ваше красноречие! Я разрешу ваш спор, не сходя с места. Эгей, и в такой чудный денек!

\emph{Ахилл} : Этот тип явно дурака валяет.

\emph{Черепаха} : Нет, подождите, Ахилл, давайте-ка его послушаем. О неизвестный господин, будьте так любезны поделиться с нами вашими соображениями по этому поводу.

\emph{Зенон} : С превеликим удовольствием. Не ветер и не флаг --- на самом деле, вообще ничто не движется, что следует из моей великой Теоремы. Она гласит: «Мир Изменяется Исключительно Иллюзорно». А из этой Теоремы вытекает еще более великая Теорема, Теорема Зенона: «Мир Ультранеподвижен».

\emph{Ахилл} : Теорема Зенона? Вы, случаем, уж не Зенон ли из Элей будете?

\emph{Зенон} : Он самый, Ахилл.

\emph{Ахилл (чешет голову в замешательстве)} : Откуда он знает, как меня зовут?

\emph{Зенон} : Возможно ли убедить вас выслушать меня, чтобы вы поняли, почему это так? Я прошел сегодня от точки А до самой Элей, только затем, чтобы найти кого-нибудь, кто согласился бы послушать мои тщательно отточенные доводы. Но все встречные сразу разбегались. Им, видите ли, было некогда. Вы не представляете себе, как это разочаровывает, когда встречаешь отказ за отказом\ldots{} Однако простите меня --- я совсем замучил вас пересказом моих неприятностей. Я прошу вас только об одном: не согласитесь ли вы ублажить старика-философа и уделить несколько минут --- обещаю вам, всего лишь несколько минут --- его экстравагантным теориям?

\emph{Ахилл} : О, без сомнения! Сделайте милость, просветите нас! Я знаю, что говорю за обоих, так как моя приятельница, госпожа Черепаха, только что отзывалась о вас весьма уважительно и упоминала как раз о ваших парадоксах.

\emph{Зенон} : Благодарю вас. Видите ли, мой Мастер, пятый патриарх, учил меня, что реальность всегда одна и та же, единая и неизменная. Все разнообразие, изменение и движение --- не более, чем иллюзии наших органов чувств. Некоторые смеялись над его взглядами, но я могу доказать всю абсурдность их насмешек. Мои доводы весьма просты. Я покажу их на примере двух персонажей моего собственного изобретения: Ахилл (греческий воин, самый быстроногий из смертных) и Черепаха. В моем рассказе, прохожий убеждает их бежать наперегонки к флагу, развевающемуся на ветру в конце беговой дорожки. Предположим, что Черепаха, как гораздо более медленный бегун, получит фору, скажем, в пятьдесят локтей. Соревнование начинается. В несколько прыжков Ахилл добегает до того места, откуда стартовала Черепаха.

\emph{Ахилл} : Ха!

\emph{Зенон} : Теперь Черепаха впереди него лишь на пять метров. Ахилл вмиг достигает того места.

\emph{Ахилл} : Хо-хо!

\emph{Зенон} : Все же за этот миг Черепаха успела немного продвинуться вперед. В мгновение ока Ахилл покрывает и эту дистанцию.

\emph{Ахилл} : Хи-хи-хи!

\emph{Зенон} : Но и в это кратчайшее мгновение Черепаха чуточку продвинулась, и опять Ахилл оказался позади. Теперь вы видите, что если Ахилл хочет нагнать Черепаху, ему придется играть в эти «догонялки» БЕСКОНЕЧНО --- а следовательно, он НИКОГДА ее не догонит!

\emph{Черепаха} : Хе-хе-хе-хе!

\emph{Ахилл} : Хм\ldots{} хм\ldots{} хм\ldots{} хм\ldots{} хм\ldots{} Этот довод кажется мне неверным. Однако я никак не могу понять, в чем здесь ошибка.

\emph{Зенон} : Хороша головоломочка? Это мой любимый парадокс.

\emph{Черепаха} : Прошу прощения, Зенон, но мне кажется, что вы рассказали нам что-то не то. Через несколько веков этот ваш рассказ будет известен как парадокс Зенона «Ахилл и Черепаха»; он показывает --- гм! --- что Ахилл никогда не догонит Черепаху. Доказательство же того, что Мир Изменяется Исключительно Иллюзорно (а следовательно, Мир Ультранеподвижен) содержится в вашем «Дихотомическом Парадоксе», не так ли?

\emph{Зенон} : Ах, какой стыд. Конечно же, вы правы. Это тот парадокс, где объясняется, что идя от А до Б, надо сначала пройти половину пути --- но от этой половины также придется сначала пройти половину\ldots{} и так далее. Оба эти парадокса очень похожи; честно говоря, я просто обыгрывал мою Великую Идею с разных сторон.

\emph{Ахилл} : Могу поклясться, что эти аргументы содержат ошибку. Хотя я не вижу, где в них ошибка, зато прекрасно понимаю, что они не могут быть верными.

\emph{Зенон} : Так вы сомневаетесь в правильности моих парадоксов? Отчего же вам самим не попробовать? Видите тот красный флаг в конце дорожки?

\emph{Ахилл} : Невозможный, сделанный по гравюре Эшера?

\emph{Зенон} : Тот самый. Как насчет того, чтобы вам с Черепахой пробежаться к флагу наперегонки? Конечно, ей надо будет дать приличную фору, скажем\ldots{}

\emph{Черепаха} : Как насчет пятидесяти локтей?

\emph{Зенон} : Отлично --- пусть будут пятьдесят локтей.

\emph{Ахилл} : Я-то всегда готов.

\emph{Зенон} : Вот и чудесно. Все это захватывающе интересно! Сейчас мы проверим мою строго доказанную Теорему на опыте! Госпожа Черепаха, будьте так добры, займите позицию на пятьдесят локтей впереди Ахилла.

\emph{(Черепаха продвигается на пятьдесят локтей ближе к флагу.)}

Ну как, вы оба готовы?

\emph{Черепаха и Ахилл} : Готовы!

\emph{Зенон} : На старт\ldots{} Внимание\ldots{} Марш!


% % \subsubsection{ГЛАВА I: Головоломк MU}
% \subsubsection{ГЛАВА I: Головоломк MU}

Формальные системы

ОДНИМ ИЗ центральных понятий этой книги является понятие \emph{формальной системы.} Формальные системы того типа, который я использую, были изобретены американским логиком Эмилем Постом в 1920-х годах; их часто называют \emph{системами продукции} или \emph{системами Поста.} Эта глава познакомит вас с одной из таких формальных систем. Надеюсь, что вам захочется хотя бы немного ее исследовать --- чтобы вас заинтересовать, я придумал небольшую головоломку.

Головоломка формулируется просто: «Можете ли вы получить \textbf{MU} ?» Для начала вам будет дана некая строчка (последовательность букв).\protect\hyperlink{c_1}{\textsuperscript{\uline{\{1\}}}} Чтобы не мучить вас неизвестностью, сообщу эту строчку сразу --- это будет MI. Кроме этого, вам будут даны правила, с помощью которых вы сможете превращать одну строчку в другую. Вы можете использовать любое правило, применимое в данный момент; при этом, если таких правил несколько, у вас имеется свободный выбор. Именно в этот момент игра с формальной системой ближе всего подходит к искусству. Само собой, главное требование игры --- следование правилам. Это ограничение может быть названо «требованием формальности». Возможно, что в данной главе нам не придется подробно на нем останавливаться. Однако, как бы удивительно это вам не казалось, работая с формальными системами последующих глав, вы увидите, что вам частенько захочется нарушать требование формальности, если у вас раньше не было навыка работы с подобными системами.

Наша формальная система --- назовем ее \emph{системой \textbf{MIU}} --- использует лишь три буквы: \textbf{М} , \textbf{U} , \textbf{I} . Это означает, что единственными строчками системы \textbf{MIU} будут те, которые используют только эти буквы. Ниже приводятся некоторые строчки системы \textbf{MIU} :

\textbf{MU}

\textbf{UIM}

\textbf{MUUMUU}

\textbf{UIIUMIUUIMUIIUMIUUIMUIIU}

Однако, хотя все эти строчки и правильны, вы еще не можете ими распоряжаться. Пока у вас имеется единственная строчка --- \textbf{MI} . Вы можете расширить вашу «коллекцию» путем применения правил. Первое правило нашей системы:

ПРАВИЛО I: Если у вас есть строчка, кончающаяся на \textbf{I} , вы можете прибавить \textbf{U} в конце.

Кстати, надо отметить, если вы уже сами об этом не догадались, что в понятии «строчка» важен определенный порядок букв. Например, \textbf{MI} и \textbf{IM} --- две разные строчки. Строчка символов совсем не то же самое, что «мешок» с символами, где порядок символов не играет никакой роли.

Второе правило нашей системы:

ПРАВИЛО II: Если у вас имеется \textbf{М} \emph{x} , вы можете прибавить к вашей коллекции \textbf{М} \emph{xx.}

Поясним это правило на нескольких примерах.

Из \textbf{MIU} вы можете получить \textbf{MIUIU} .

Из \textbf{MUM} вы можете получить \textbf{MUMUM} .

Из \textbf{MU} вы можете получить \textbf{MUU} .

Таким образом, буква~\emph{x} означает здесь любую строчку; однако, после того, как вы выбрали определенную строчку, вам придется держаться вашего выбора до тех пор, пока вы не используете снова то же правило --- тогда вы можете сделать новый выбор. Обратите внимание на третий пример. Он показывает, каким образом вы можете получить новую строчку из \textbf{MU} --- но сначала вам необходимо иметь в вашей коллекции \textbf{MU} ! Хочу добавить еще одно, последнее замечание, касающееся буквы «\emph{x} » она не является частью формальной системы в том смысле, как буквы «\textbf{М} », «\textbf{I} » и «\textbf{U} ». Тем не менее, нам нужен способ говорить о строчках системы вообще --- и в этом нам помогает «\emph{x} »\emph{,} символизирующий любую произвольную строчку. Если в вашей коллекции оказывается строчка, содержащая «\emph{x} »\emph{,} это значит, что вы где-то ошиблись, так как в строчках системы \textbf{MIU} эта буква не встречается.

Третье правило нашей системы:

ПРАВИЛО III: Если в какой-либо строчке встречается \textbf{III} , вы можете получить новую строчку, где вместо \textbf{III} будет \textbf{U} .

Примеры.

Из \textbf{UMIIIMU} вы можете получить \textbf{UMUMU} .

Из \textbf{MIIII} вы можете получить \textbf{MIU} (а также \textbf{MUI} ).

Из \textbf{IIMII} вы не можете, применяя правило III, получить ничего нового. (Все три \textbf{I} должны стоять подряд.)

Ни в коем случае нельзя думать, что это правило можно применять в обратном порядке, как в следующем примере:

Из \textbf{MU} можно получить \textbf{MIII} . \textless= Это неверно.

Все правила читаются только в одном направлении, слева направо.

Последнее правило нашей системы:

ПРАВИЛО IV: Если в какой-либо строчке встречается последовательность \textbf{UU} , вы можете ее опустить.

Из \textbf{UUU} можно получить \textbf{U} . Из \textbf{MUUUIII} можно получить \textbf{MUIII} .

Теперь у вас есть все, что нужно, чтобы попытаться вывести \textbf{MU} . Не волнуйтесь, если у вас не будет получаться; просто попробуйте поиграть с~системой и постарайтесь схватить суть головоломки \textbf{MU} . Надеюсь, что вы получите удовольствие!

Теоремы, аксиомы и правила

Ответ на головоломку \textbf{MU} вы найдете дальше в тексте. Сейчас для нас важен сам процесс поиска решения. Возможно, что вы уже попытались это сделать; если так, то теперь у вас оказалась целая коллекция строчек. Подобные строчки, выведенные путем применения правил, называются \emph{теоремами.} Термин «теорема», разумеется, широко используется в математике и имеет там совсем другое значение: какое-либо утверждение на естественном языке, доказанное с помощью строгих рассуждений (например, Теорема Зенона о «невозможности» движения или Теорема Эвклида о бесконечном количестве простых чисел). Однако в формальных системах теоремы --- не утверждения, а лишь строчки символов. Такие теоремы не \emph{доказываются,} а просто \emph{производятся} автоматически при помощи неких типографских правил. Чтобы подчеркнуть это важное отличие, в дальнейшем, говоря о «теоремах» в обыденном значении, я буду писать это слово с заглавной буквы: Теорема --- это утверждение на каком-либо естественном языке, которое было доказано с помощью логических рассуждений. Слово «теорема», написанное с маленькой буквы, будет употребляться в техническом значении: теорема --- это строчка, выводимая в какой-либо формальной системе. В этих терминах головоломка \textbf{MU} состоит в том, чтобы выяснить, является ли \textbf{MU} теоремой системы \textbf{MIU} .

В начале этой главы я «подарил» вам теорему \textbf{MI} . Такая «дареная» теорема называется \emph{аксиомой.} Также и в этом случае, техническое значение этого слова отличается от повседневного. Формальная система может иметь ноль, одну, несколько и даже бесконечное множество аксиом. Далее в книге приводятся примеры формальных систем всех трех видов.

Каждая формальная система обладает набором правил обращения с символами, таких, как четыре правила системы \textbf{MIU} . Подобные правила называются \emph{порождающими правилами} или \emph{правилами вывода} ; в дальнейшем я буду пользоваться обоими терминами.

И, наконец, последний термин --- \emph{вывод.} Ниже приводится вывод теоремы \textbf{MUIIU} :

\textbf{(1)~~ MI} ~ аксиома

\textbf{(2)~~ MII} ~~из (1) по правилу II

\textbf{(3)~~ MIIII} ~из (2) по правилу II

\textbf{(4)~~ MIIIIU} ~из (3) по правилу I

\textbf{(5)~~ MUIU} ~из (4) по правилу III

\textbf{(6)~~ MUIUUIU} ~из (5) по правилу II

\textbf{(7)~~ MUIIU} ~из (6) по правилу IV

Выводом теоремы называется последовательное, шаг за шагом, объяснение того, как можно получить данную теорему согласно правилам формальной системы. Понятие вывода основывается на понятии доказательства, являясь, однако, лишь его дальним родственником. Было бы странным утверждать, что мы \emph{доказали} строчку \textbf{MUIIU} ; скорее, мы ее \emph{вывели.}

Внутри и снаружи системы

Большинство читателей, пытаясь решить головоломку \textbf{MU} , начинает выводить теоремы наобум и смотрят, что при этом получается. Вскоре, однако, они замечают, что полученные теоремы обладают некими свойствами; в этот момент в работу включается разум. Возможно, что пока вы не вывели несколько теорем, для вас не было очевидным, что все они будут начинаться с \textbf{M} . В какой-то момент вы заметили некую закономерность и смогли ее объяснить, исходя из правил они таковы; что каждая новая теорема наследует первую букву предыдущей. В результате первые буквы всех теорем восходят к первой букве нашей единственной аксиомы \textbf{MI} --- и это доказательство того, что все теоремы системы \textbf{MIU} должны начинаться с \textbf{M} .

То, что произошло, очень важно. Это указывает на одно из различий между человеком и машиной. Было бы возможно --- и даже весьма нетрудно --- запрограммировать компьютер на вывод теорем системы \textbf{MIU} ; мы можем включить в программу команду, велящую машине не останавливаться, пока она не выведет \textbf{U} . Читатель уже знает, что компьютер, запрограммированный таким образом, не остановится никогда.

В этом нет ничего удивительного. Но что, если бы вы попросили вывести \textbf{U} одного из ваших приятелей? Вы не удивились бы, если бы он через некоторое время подошел к вам, жалуясь, что он никак не может избавиться от M, и что эти поиски --- сумасбродная затея.

Даже не очень сообразительный человек не может не заметить закономерности в том, что он делает; эти наблюдения помогают ему лучше понять поставленную перед ним задачу. Компьютерная программа, которую мы только что упомянули, этого сделать не может.

Когда я сказал, что этот факт показывает различие между человеком и машиной, я имел в виду следующее: компьютер \emph{возможно} запрограммировать таким образом, что тот никогда не заметит даже самых очевидных закономерностей в том, что он делает; человеку, однако, свойственно подмечать определенные закономерности в его занятиях. Все это читатель, конечно, знал и раньше. Если вы возьмете калькулятор, нажмете на 1, прибавите 1, снова прибавите 1, и будете делать то же самое еще много раз подряд, калькулятор никогда не научится делать этого сам; однако любой человек очень быстро заметил бы схему в ваших действиях Еще один простой пример: автомобиль, как бы долго и хорошо его не водили, никогда не научится избегать аварий и никогда не выучит даже самые частые маршруты своего хозяина.

Таким образом, различие в том, что машина \emph{может} не делать наблюдений, в то время как для человека это невозможно. Заметьте, что я не говорю, что \emph{вообще никакие} машины не способны делать сложных наблюдений; я имею в виду лишь некоторые из них. Я также не хочу сказать, что все люди способны делать сложные наблюдения; на самом деле, многие из них весьма ненаблюдательны. Но машины, в отличие от людей, могут быть сделаны \emph{совершенно} ненаблюдательными. На самом деле, большинство машин, созданных до сих пор, весьма близки к полной ненаблюдательности; именно поэтому, многие считают, что отсутствие наблюдательности --- одна из основных характеристик машин. Например, говоря о «механической» работе, мы не имеем в виду, что люди не могут с ней справиться; мы хотим сказать, что только машина способна безропотно проделывать такую работу снова и снова.

Прыжки за пределы системы

Человеческому интеллекту свойственно умение, выпрыгивая за пределы системы, смотреть на то, что он делает, со стороны; при этом он ищет --- и часто находит --- какую-либо схему, закономерность. В то же время, сказав, что разум способен взглянуть на свою работу со стороны, я не говорю, что он делает это всегда. Зачастую, однако, для этого бывает достаточно лишь небольшого толчка. Например, человеку, читающему книгу, может захотеться спать. Вместо того, чтобы дочитать книгу до конца, он, скорее всего, отложит ее в сторону и потушит свет. При этом он «выходит из системы»; нам это кажется вполне естественным. Другой пример: человек А смотрит телевизор. В комнату входит человек Б и показывает явное неудовольствие ситуацией. Человек А может решить, что он понимает, в чем дело, и попытаться исправить положение, выходя из данной системы (той программы телевизора, которую он смотрел) и переключая телевизор на другой канал в поисках лучшей передачи. Б, однако, может иметь в виду более радикальный «выход из системы» --- а именно, вообще выключить телевизор! В некоторых случаях только редкие личности могут заметить систему, управляющую жизнью многих людей --- систему, никогда раньше таковой не считавшуюся. Подобные личности зачастую посвящают жизнь тому, чтобы убедить остальных, что система действительно существует, и что из нее необходимо выйти!

Насколько хорошо можно научить компьютер выскакивать за пределы системы? Я приведу пример, в свое время удививший многих наблюдателей. Не так давно на шахматном чемпионате среди компьютеров у одной из программ (самой слабой) оказалась необычайная особенность --- сдаваться задолго до конца партии. Она не была хорошим игроком, зато умела увидеть, когда позиция становилась безнадежной, и сдаться в этот момент, вместо того, чтобы ждать, пока другая программа пройдет через скучную процедуру матования. Хотя та программа проиграла все свои партии, она сделала это с шиком, удивив многих местных знатоков шахмат. Таким образом, если мы определим здесь «систему» как «делать ходы шахматной партии», ясно, что та программа имела сложную, заранее запрограммированную способность выходить из системы. С другой стороны, если вы считаете, что «системой» в данном случае является «все то, что компьютер запрограммирован делать», несомненно, что та программа вовсе не умела выходить из системы.

Изучая формальные системы, очень важно отличать работу \emph{внутри} системы от наших наблюдений \emph{над} системой. Наверное, подобно большинству читателей, вы начали работу над головоломкой \textbf{MU} внутри системы; однако в какой-то момент ваше терпение истощилось и вы вышли из системы, пытаясь проанализировать результаты вашей работы и понять, почему вам до сих пор не удалось получить \textbf{MU} . Возможно, вы смогли ответить на этот вопрос; это --- пример размышления о системе. Вероятно, в какой-то момент вы вывели \textbf{MIU} ; это --- пример работы внутри системы. Я не хочу сказать, что эти два метода совершенно несовместимы; напротив, я уверен, что любой человек до определенной степени способен одновременно работать внутри системы и размышлять над тем, что он делает. Более того, в человеческих делах часто почти невозможно точно отделить работу внутри системы от ее анализа; жизнь состоит из такого количества сложных, переплетенных между собой систем, что подобное деление вообще кажется слишком большим упрощением. Однако сейчас для нас важно четко сформулировать простые идеи, чтобы в дальнейшем мы могли опираться на них при анализе более сложных систем. Именно поэтому я рассказываю вам о формальных системах; кстати, нам пора вернуться к обсуждению системы \textbf{MIU} .

Режим М, Режим I, Режим U.

Головоломка \textbf{MU} была сформулирована таким образом, чтобы читатель некоторое время работал внутри системы, выводя теоремы. В то же время, ее формулировка не обещала, что, оставаясь внутри системы, он сможет добиться результата. Таким образом, система \textbf{MIU} предполагает некоторое колебание между двумя режимами работы. Эти режимы можно разделить, используя два листа бумаги: на одном из них вы работаете «в качестве машины», заполняя лист теоремами; на другом вы работаете «в качестве мыслящего существа» и можете делать все, что вам подскажет смекалка: использовать русский язык, записывать идеи, работать в обратном порядке, использовать иксы, сжимать несколько шагов в один, менять правила системы, чтобы посмотреть, что из этого выйдет --- короче, все, что придет вам в голову. Вы можете заметить, что числа 3 и 2 играют важную роль в системе, так как \textbf{I} сокращаются группами по 3, a \textbf{U} --- группами по 2; кроме того, правило II позволяет удвоение букв (кроме \textbf{M} ). На втором листе бумаги у вас могут содержаться какие-то размышления по этому поводу. Позже мы еще вернемся к этим двум способам работы с формальными системами; мы будем называть их механический режим (способ \textbf{M} ) и интеллектуальный режим (способ \textbf{I} ). Каждой букве системы \textbf{MIU} соответствует один из режимов. В дальнейшем я опишу последний режим --- ультра-режим (режим \textbf{U} ), свойственный дзен-буддистскому подходу к вещам. Подробнее об этом через несколько глав.

Алгоритм разрешения

Работая над этой головоломкой, вы, вероятно, заметили, что она включает правила двух противоположных типов \emph{удлиняющие} и \emph{укорачивающие.} Два правила (I и II) позволяют нам удлинять строчки (естественно, лишь строго определенным образом), два других правила позволяют укорачивать строчки (опять же, следуя строгому закону). Кажется, что порядок применения этих правил можно бесконечно варьировать; таким образом, возникает надежда, что рано или поздно мы придем к искомой строчке \textbf{MU} . Возможно, нам придется создать для этого гигантскую строчку и затем сокращать ее, пока не останутся только два символа; или, того хуже, нам придется попеременно удлинять и сокращать, удлинять и сокращать, и так далее. При этом успех не гарантирован. На самом деле, мы уже заметили, что получить \textbf{U} вообще невозможно, даже если бы мы удлиняли и сокращали строчки до второго пришествия.

Тем не менее, кажется, что с \textbf{MU} ситуация иная, чем с \textbf{U} . Наше заключение о том, что \textbf{U} вывести невозможно, основывалось на очевидном свойстве этой строчки она не начинается с \textbf{M} , как все остальные теоремы. Иметь такой простой способ отличать не-теоремы весьма удобно. Однако кто может поручиться, что подобный способ укажет нам \emph{все} не-теоремы? Вполне возможно, что существует множество начинающихся с~\textbf{M} строчек, которые, тем не менее, невыводимы. Это означало бы, что проверка «по первой букве» указывает нам только на ограниченное количество не-теорем, оставляя «за бортом» все остальные. Однако существует возможность найти некий более сложный метод проверки, точно говорящий нам, какие строчки могут быть выведены с помощью данных правил, а какие --- нет. Тут перед нами возникает вопрос: что мы подразумеваем под словом «проверка»? Читателю может быть не совсем понятно, какой смысл задаваться этим вопросом и почему он столь важен в данном контексте. Приведу пример такой «проверки», которая, как кажется, идет вразрез с самим смыслом этого слова.

Представьте себе джинна, в распоряжении которого имеется все время на свете. Джинн тратит это время на вывод теорем системы \textbf{MIU} . Делает он это весьма методично, скажем, следующим образом:

Шаг 1: Приложить все подходящие правила к аксиоме \textbf{MI} . Это дает две новые теоремы: \textbf{MIU} , \textbf{MII} .

Шаг 2: Приложить все подходящие правила к теоремам, полученным в шаге 1. Это дает три новые теоремы: \textbf{MIIU} , \textbf{MIUIU} , \textbf{MIIII} .

Шаг 3: Приложить все подходящие правила к теоремам, полученным в шаге 2. Это дает пять новых теорем: \textbf{MUIIIIU} , \textbf{MIIUIIU} , \textbf{MIUIUIUIU} , \textbf{МIIIIIIII} , \textbf{MUI} .

.

.

.

Следуя этому методу, рано или поздно мы выведем каждую теорему системы, так как правила применяются во всех мыслимых комбинациях. (См. рис. 11) Все удлиняющие и укорачивающие трансформации, упомянутые выше, со временем будут осуществлены.

\emph{Рис. 11. Систематически построенное «дерево» всех теорем системы MIU. N-ный уровень внизу содержит теоремы, для вывода которых понадобилось ровно N шагов. Номера в кружках говорят нам, с помощью какого правила была получена данная теорема. Растет ли на этом дереве MU?}

Неясно, однако, как долго нам придется ждать появления той или иной строчки, поскольку теоремы расположены согласно длине их вывода. Это не очень-то полезное расположение, в особенности, если вы заинтересованы в какой-то определенной строчке (например, \textbf{MU} ) и при этом не знаете не только того, какой длины ее вывод, но даже того, существует ли этот вывод вообще. Теперь давайте взглянем на обещанную «проверку теоремности»:

Ждите, пока данная строчка будет выведена; когда это случится, вы будете знать, что это --- теорема. Если же этого не случится никогда, вы можете быть уверены, что данная строчка --- не теорема.

Это звучит нелепо, так как здесь имеется в виду, что мы согласны ждать ответа до скончания веков. Таким образом, мы опять подошли к вопросу о том, что может считаться «проверкой». Прежде всего, нам необходима гарантия, что мы получим ответ за ограниченный промежуток времени. Такая проверка теоремности, которая завершается в конечный отрезок времени, называется \emph{алгоритмом разрешения} для данной формальной системы.

Когда у вас имеется алгоритм разрешения, все теоремы системы приобретают конкретную характеристику. С первого взгляда может показаться, что правила и аксиомы формальной системы сами по себе характеризуют ее теоремы не менее полно, чем алгоритм разрешения; однако проблема здесь заключается в слове «характеризуют». Безусловно, как правила вывода, так и аксиомы системы \textbf{MIU} косвенно характеризуют строчки, являющиеся теоремами; еще более косвенно они характеризуют строчки, теоремами \emph{не} являющиеся. Однако косвенная характеристика часто недостаточна. Если кто-нибудь утверждает, что он имеет в своем распоряжении характеристику всех теорем, но при этом тратит бесконечное время, чтобы установить, что данная строчка не является теоремой, вы, скорее всего, подумаете, что в его характеристике чего-то не хватает --- она недостаточно конкретна. Именно поэтому так важно установить, есть ли в данной системе алгоритм разрешения. Положительный ответ будет означать, что вы всегда можете проверить, является ли данная строчка теоремой; при этом, какой бы длинной проверка ни была, она \emph{непременно придет к концу.} В принципе, проверка --- такой же простой, механический, конечный и верный процесс, как установление того, что первая буква строчки --- \textbf{M} . Алгоритм разрешения --- это «лакмусовая бумажка» для установления теоремности!

Кстати, одним из требований формальной системы является наличие алгоритма разрешения для аксиом: каждая формальная система должна иметь свою Лакмусовую бумажку для определения аксиомности. Таким образом, у нас не будет проблем по крайней мере в начале работы. Разница между множеством аксиом и множеством теорем в том, что первые всегда имеют алгоритм разрешения, в то время как последние могут его и не иметь.

Уверен, что вы согласитесь, что, когда вы начали работать с системой \textbf{MIU} , вам пришлось столкнуться именно с этой проблемой. Вам была известна единственная аксиома системы и простые правила вывода, косвенно характеризующие теоремы --- и все же было неясно, каковы последствия этой характеристики. В частности, было совершенно непонятно, является ли MU теоремой.

\emph{Рис. 12. М. К. Эшер. «Воздушный замок» (гравюра на дереве), 1928.}


% % \subsubsection{\texorpdfstring{\emph{Двухголосная инвенция} }{Двухголосная инвенци }}
% \subsubsection{\texorpdfstring{\emph{Двухголосная инвенция} }{Двухголосная инвенци }}

\emph{или Что Черепаха сказала Ахиллу (записано Льюисом Кэрроллом)} \footnote{Lewis Carroll «What the Tortoise Said to Achilles» в журнале «Mind» n s 4 1895) стр. 255 6}

\emph{Ахилл перегнал Черепаху и с комфортом уселся отдыхать на ее широкой спине.}

«Так вам все же удалось добежать до финиша?» --- сказала Черепаха. «Несмотря на то, что дистанция состояла из бесконечного ряда отрезков? Я-то думала, какой-то умник доказал, что это невозможно сделать?»

«Это ВОЗМОЖНО сделать», --- сказал Ахилл: «И я это СДЕЛАЛ! \emph{Solvitur ambulando.} Видите ли, дистанции постоянно УМЕНЬШАЛИСЬ\ldots»

«А если бы они постоянно УВЕЛИЧИВАЛИСЬ?» --- перебила Черепаха, --- «Что тогда?»

«Тогда бы меня здесь еще не было,» --- скромно ответил Ахилл, --- «А Вы уже успели бы обежать несколько раз вокруг света.»

«Вы весьма великодушны, Ахилл. Вы меня просто подавили\ldots{} я хочу сказать, придавили, поскольку вы нешуточный тяжеловес. А теперь, не угодно ли вам послушать про такую беговую дорожку, о которой большинство людей воображают, что могут преодолеть ее в два-три шага, когда на самом деле она состоит из бесконечного числа расстояний, где каждое последующее больше предыдущего?»

«С превеликим удовольствием,» --- ответствовал греческий воин, доставая из шлема (в те дни мало кто из греческих воинов мог похвастаться карманами) огромный блокнот с карандашом. «Приступайте к своему рассказу, да говорите, пожалуйста, помедленнее --- ведь стенография еще не изобретена!»

«Этот прекрасный Первый Постулат Эвклида\ldots» --- пробормотала мечтательно Черепаха, --- «вы восхищаетесь Эвклидом?»

«Страстно! Постольку, конечно, поскольку можно восхищаться трудом, который будет опубликован лишь через несколько столетий\ldots»

«Давайте, в таком случае, рассмотрим первые два пункта его доводов, и выводы, которые из них следуют. Будьте так любезны, запишите их к себе в блокнот --- для удобства обозначим их А, В и Z:

(A) Вещи, равные одному и тому же, равны между собой.

(B) Две стороны этого треугольника суть вещи, равные одному и тому же.

(Z) Две стороны этого треугольника равны между собой.

Читатели Эвклида согласятся, я думаю, что Z логически следует из А и В, так что тот, кто согласен с истинностью А и В, ДОЛЖЕН считать истинным и Z?»

«Несомненно! Уж с ЭТИМ-то легко согласится любой старшеклассник --- как только старшие классы будут изобретены, каких-нибудь пару тысяч лет спустя.»

«И если какой-нибудь читатель не принимает А и В за истинные, он, тем не менее, должен согласиться с тем, что ВЗЯТАЯ ЦЕЛИКОМ, эта последовательность имеет смысл?»

«Без сомнения, такого читателя можно вообразить. Он мог бы сказать: „Я принимаю за истинное Гипотетическое Утверждение, что ЕСЛИ А и В истинны, то Z должно быть тоже истинно.`` Такой читатель поступил бы мудро, если бы он оставил Эвклида и занялся футболом».

«А что, если какой-нибудь другой читатель сказал бы: „Я принимаю за истинные А и В, но НЕ Гипотетическое Утверждение``?»

«Наверное, и такой читатель мог бы существовать. Ему, впрочем, тоже было бы лучше заняться футболом.»

«И никакой из этих читателей ПОКА не обязан соглашаться с тем, что логически Z должно быть истинно?»

«Совершенно верно,» --- кивнул Ахилл.

«Теперь представьте на минуту, что я --- тот второй читатель, и попробуйте логически заставить меня признать, что Z истинно.»

«Черепаха, играющая в футбол, была бы\ldots» --- начал Ахилл.

«\ldots{} совершеннейшей аномалией, конечно,» --- торопливо перебила Черепаха. «Не будем отвлекаться; сначала давайте разберемся с Z, а потом уж поговорим о футболе!»

«Я должен заставить вас принять Z, не так ли?» --- задумчиво пробормотал Ахилл. «И вы утверждаете, что принимаете А и В, но тем не менее не принимаете Гипотетическое Утверждение\ldots»

«Назовем его С», --- вставила Черепаха.

«Но вы не принимаете

(С) Если А и В истинны, следовательно Z должно быть истинно.»

«Именно это я и утверждаю,» --- сказала Черепаха.

«В таком случае я должен попросить вас согласиться с С.»

«Я, пожалуй, уважу вашу просьбу, как только вы занесете ее в свой блокнот. Кстати, что у вас там еще записано?»

«Только несколько заметок на память,» --- сказал Ахилл, нервно шурша страницами: «несколько заметок о\ldots{} о сражениях в которых я отличился!»

«Здесь полно чистых страниц, как я погляжу!» --- радостно заметила Черепаха. «Нам понадобятся ВСЕ они, до последней странички!» (Ахилл содрогнулся.) «Теперь пишите за мной:

(A) Вещи, равные одному и тому же, равны между собой.

(B) Две стороны этого треугольника суть вещи, равные одному и тому же.

(C) Если А и В истинны, следовательно Z должно быть истинно.

(Z) Две стороны этого треугольника равны между собой.»

«Вы должны бы называть последнее утверждение D, а не Z, поскольку оно прямо следует за первыми тремя. Если вы принимаете А, В, и С, вам ПРИДЕТСЯ принять Z.»

«Почему это мне „придется``?»

«Потому что Z ЛОГИЧЕСКИ следует из них. Если А, и В, и С истинны, Z ДОЛЖНО быть истинно. С этим-то вы, надеюсь, не станете спорить?»

«Если А, и В, и С истинны, Z ДОЛЖНО быть истинно,» --- в раздумьи повторила Черепаха. «Это еще одно Гипотетическое Утверждение, не правда ли? И если я его не приму, я все еще могу считать истинными А, В и С, но не принимать Z, не так ли, мой друг?»

«Пожалуй, что и так,» --- согласился простодушный герой, --- «хотя такое упрямство было бы просто феноменально. Все же, это событие ВОЗМОЖНО. А раз так, я должен попросить вас принять еще одно Гипотетическое Утверждение.»

«Прекрасно! Я согласен принять и это Утверждение, как только вы его запишете. Мы назовем его D.

(D) Если А, и В, и С истинны, Z ДОЛЖНО быть истинно.

Уже записали?»

«Записал, записал!» --- радостно воскликнул Ахилл, вкладывая карандаш в футляр. «Наконец-то мы пришли к концу нашей воображаемой беговой дорожки! Теперь, когда вы принимаете А, и В, и С, и D, вы, КОНЕЧНО, принимаете и Z.»

«Неужели?» --- спросила Черепаха с невинным видом. «Давайте-ка это выясним. Я принимаю А, и В, и С, и D. Что, если я ВСЕ ЕЩЕ отказываюсь принять Z?»

«Тогда госпожа Логика возьмет вас за горло и ЗАСТАВИТ!» --- торжествующе ответил Ахилл. «Логика скажет вам: „У вас нет выхода. Теперь, когда вы согласились с А, и В, и С и D, вы ОБЯЗАНЫ согласиться с Z!{}`` Так что у вас нет выбора, как видите.»

«То, что произносит госпожа Логика, уж конечно стоит того, чтобы быть ЗАПИСАНО,» --- сказала Черепаха. «Так что, пожалуйста занесите и это в ваш блокнот. Мы назовем это

(E) Если А и В и С и D истинны, Z должно быть истинным.»

«До тех пор, пока я не согласилась с ЭТИМ утверждением, я не обязана принимать Z за истинное. Теперь вы видите, что это совершенно НЕОБХОДИМЫЙ шаг?»

«Вижу, вижу\ldots» --- сказал Ахилл, и в его голосе явственно послышались грустные нотки.

В этот момент рассказчику пришлось покинуть счастливую парочку, так как ему срочно нужно было в банк. Он снова попал в те места только через несколько месяцев. Доблестный герой Ахилл все еще восседал на спине долготерпеливой Черепахи и писал в своем блокноте, который уже почти заполнился, а Черепаха говорила: «Записали последний шаг? Если я не сбилась со счета, у нас набралось уже 1001. Осталось всего каких-нибудь несколько миллионов\ldots{} Зато подумайте только, какую ОГРОМНУЮ пользу наша беседа принесет Логикам Девятнадцатого Века!»

«Не думаю, что кто-нибудь из них сможет разобраться во всей этой чепухе», --- отвечал усталый воин, в отчаянии пряча лицо в ладонях. «Сделайте милость, разрешите мне позаимствовать каламбур, который в девятнадцатом столетии придумает знакомая Алисы, ваша кузина Черепаха Квази, и переименовать вас в г-жу Чепупаху.»

«Ахиллес, бедняга, вы видно совсем устали, такую вы несете ахиллею\ldots{} по этому поводу, я, пожалуй, позволю себе каламбур, до которого моя кузина Черепаха Квази не додумается, и переименую вас в Ахинесса.»


% % \subsubsection{ГЛАВА II: Содержание и форма  математике}
% \subsubsection{ГЛАВА II: Содержание и форма  математике}

ЭТА ДВУХГОЛОСНАЯ ИНВЕНЦИЯ оказалась для моих героев~вдохновляющей идеей. Так же, как Льюис Кэрролл позволил себе вольное обращение с Ахиллом и Черепахой Зенона, я позволил себе некоторые вольности с Ахиллом и Черепахой Льюиса Кэрролла. У Кэрролла одни и те же события повторяются снова и снова, каждый раз на более высоком уровне; это замечательная~аналогия Баховского Естественно Растущего Канона. Если лишить диалог Кэрролла его блестящего остроумия, в нем останется глубокая философская проблема: \emph{подчиняются ли слова и мысли каким-либо формальным правилам?} Это и есть основной вопрос, на который пытается ответить моя книга.

В этой и следующей главах мы рассмотрим несколько новых формальных систем; это поможет нам лучше понять саму идею \emph{формальной системы} .~Когда вы дочитаете эти две главы до конца, у вас должно сложиться неплохое представление о мощности формальных систем и о том, почему они~представляют интерес для математиков и логиков.

Система «pr»

В этой главе мы будем рассматривать \emph{систему} \textbf{pr} . Ни математики, ни физики ею не заинтересуются; признаться, она --- всего лишь мое собственное изобретение. Система~pr интересна лишь постольку, поскольку она хорошо иллюстрирует многие идеи, играющие в этой книге важную роль. В этой~системе три символа:

\textbf{p r -} --- буквы \textbf{p} и~\textbf{r} и~тире.

Система~\textbf{pr} имеет бесконечное множество аксиом. Поскольку мы не~можем записать их все, мы должны придумать какой-нибудь метод их описания. На самом деле, нам нужно не просто описание этих аксиом; нам нужен способ, позволяющий узнать, является ли данная последовательность символов~аксиомой. Простое описание аксиом охарактеризовало бы их полностью, но~недостаточно сильно; именно в этом была проблема с описанием теорем системы MIU.

Мы не собираемся возиться в течении неопределенного --- возможно,~бесконечного --- времени, чтобы определить, является ли некая строчка символов~аксиомой. Нам необходимо такое определение аксиом, которое предоставит в наше распоряжение надежный алгоритм разрешения, устанавливающий аксиоматичность любой строчки, состоящей из символов \textbf{p} ,~\textbf{r} и тире.

ОПРЕДЕЛЕНИЕ: \emph{x} \textbf{p-r} \emph{x} \textbf{-} является аксиомой, когда~\emph{x} состоит только из тире.

~Обратите внимание, что каждый из этих двух \emph{x} -ов замещает одинаковое число тире. Например,~\textbf{-\/-p-r-\/-\/-} ~является аксиомой. Само выражение \emph{x} \textbf{p-r} \emph{x} \textbf{-} , разумеется, не аксиома, так как~\emph{x} не принадлежит системе \textbf{pr} ; оно, скорее, походит на форму, в которой отливаются все аксиомы данной системы. Такая «форма» называется схемой аксиом.

Система~\textbf{pr} имеет только одно правило вывода:

ПРАВИЛО: Пусть \emph{x} , \emph{у} и \emph{z} --- строчки, состоящие только из тире. Пусть~\emph{x} \textbf{p} \emph{y} \textbf{r} \emph{z} является теоремой. Тогда \emph{x} \textbf{p} \emph{y} \textbf{-r} \emph{z} \textbf{-} также будет теоремой.

Пусть, например,~\emph{x} будет «\textbf{-\/-»} , \emph{у} --- «\textbf{-\/-\/-»} и \emph{z} ~--- «\textbf{-»} . Правило говорит нам:

Если \textbf{-\/-p-\/-\/-r-} является теоремой, то~\textbf{-\/-p-\/-\/-\/-r-\/-} также будет теоремой.

Это утверждение типично для правил вывода: оно устанавливает связь между двумя строчками, не сообщая нам ничего о том, является ли каждая из них по отдельности теоремой.

Очень полезное упражнение --- попытаться найти разрешающий алгоритм для теорем системы \textbf{pr} . Это нетрудно --- после нескольких попыток вы, скорее всего, найдете решение. Попробуйте!

Разрешающий алгоритм

Надеюсь, что вы уже попытались найти решение. Во-первых, хотя это и кажется очевидным, я хотел бы заметить, что каждая теорема системы \textbf{pr} имеет три отдельных группы тире, и что разделяющими элементами являются \textbf{p} и \textbf{r} , именно в таком порядке. (Это можно доказать, основываясь на аргументах «наследственности», так же, как мы смогли доказать, что теоремы системы \textbf{MIU} всегда должны начинаться с \textbf{М} .) Это означает, что уже сама форма такой строчки как \textbf{-\/-p-\/-p-\/-p-\/-r-\/-\/-\/-\/-\/-\/-\/-~} исключает ее из числа теорем.

Читатель может подумать, что, подчеркивая фразу «уже сама форма», автор поступает довольно глупо: что еще может быть в такой строчке, кроме формы? Что, кроме ее формы, может играть какую-либо роль в определении особенностей данной строчки? Совершенно ясно, что ничего больше! Однако имейте в виду, читатель, что по мере того, как мы будем углубляться в обсуждение формальных систем, понятие «формы» будет становиться все сложнее и абстрактнее и нам придется все чаще задумываться о значении самого этого слова. Во всяком случае, мы будем называть \emph{«правильно составленной строчкой»} любую строчку следующей структуры: группа тире, одно \textbf{p,} вторая группа тире, одно \textbf{r} , завершающая группа тире.

Вернемся к алгоритму разрешения. Для того, чтобы данная строчка считалась теоремой, первые две группы тире в сумме должны давать третью группу тире. Так, например, \textbf{-\/-p-\/-r-\/-\/-\/-} ~является теоремой, так как 2 плюс 2 равняется 4, в то время как \textbf{-\/-p-\/-r-} теоремой не является, так как 2 плюс 2 не равняется 1. Чтобы понять, почему этот критерий верен, взгляните сначала на схему аксиом. Очевидно, она производит только такие аксиомы, которые удовлетворяют критерию сложения. Теперь обратитесь к правилу вывода. Если первая строчка удовлетворяет критерию сложения, то же условие необходимо будет выполняться и во второй строчке. И, наоборот, если первая строчка не удовлетворяет критерию сложения, не будет удовлетворять ему и вторая строчка. Это правило превращает критерий сложения в наследственное качество теорем; каждая теорема передает его своим «отпрыскам». Это показывает, почему критерий сложения верен.

Кстати, в системе \textbf{pr} есть некое свойство, позволяющее нам с уверенностью сказать, что данная система имеет разрешающий алгоритм, еще до того, как мы нашли критерий сложения. Это свойство заключается в том, что система\textbf{~pr} не усложнена встречными \emph{укорачивающими} и \emph{удлиняющими} правилами; в ней имеются лишь удлиняющие правила. Любая формальная система, которая говорит нам, как получать более длинные теоремы из более коротких, но никогда не говорит нам обратного, должна иметь алгоритм разрешения для своих теорем. Представьте себе, что вам дана какая-либо строчка. Прежде всего, проверьте, является ли эта строчка аксиомой (я предполагаю, что у нас имеется алгоритм разрешения для аксиом, иначе наше предприятие было бы безнадежным). Если это аксиома, то, следовательно, по определению она является теоремой, и проверка на этом заканчивается. Предположим теперь, что наша строчка --- не аксиома. В таком случае, чтобы быть теоремой, она должна была быть получена из более короткой строчки путем применения одного из правил. Перебирая правила одно за другим, вы всегда можете установить, какие из них были использованы для получения данной строчки, а также какие более короткие строчки предшествуют ей на «генеалогическом древе». Таким образом, проблема сводится к определению того, какие из новых, более коротких строчек являются теоремами. Каждая из них, в свою очередь, может быть подвергнута такой же проверке. В худшем случае, нам придется проверить огромное количество все более коротких строчек. Продолжая продвигаться таким образом назад, вы медленно, но верно приближаетесь к источнику всех теорем: схеме аксиом. Строчки не могут укорачиваться бесконечно; в один прекрасный момент вы либо установите, что одна из новых коротеньких строчек является аксиомой, либо застрянете на строчках, которые, не являясь аксиомами, тем не менее, не поддаются дальнейшему сокращению. Таким образом, системы, имеющие лишь удлиняющие правила, не особенно интересны; по-настоящему любопытны лишь системы, где взаимодействуют удлиняющие и укорачивающие правила.

Снизу вверх \emph{vs.} сверху вниз

Метод, описанный выше, можно назвать нисходящим алгоритмом разрешения; сравним его с восходящим алгоритмом, описание которого я сейчас приведу. Он весьма напоминает метод, используемый джинном для производства теорем в системе \textbf{MIU} ; однако он несколько осложнен присутствием схемы аксиом. Мы возьмем что-то вроде корзины, куда будем бросать теоремы по мере их рождения.

(1а) Бросьте в корзину самую простую (\textbf{-p-r-\/-} ) из возможных теорем.

(1б) Приложите правило вывода к предмету в корзине и положите в корзину результат.

(2а) Положите в корзину следующую по простоте аксиому.

(2б) Приложите правило в каждому имеющемуся в корзине предмету и бросьте в корзину результаты.

(За) Положите третью по простоте аксиому в корзину.

(3б) Приложите правило к каждому имеющемуся в корзине предмету и бросьте в корзину результаты.

И т. д. и т. п.

Ясно, что, действуя таким образом, вы не можете пропустить не одной теоремы системы \textbf{pr.} С течением времени корзина будет наполняться все более длинными теоремами; это --- следствие отсутствия сокращающих правил Таким образом, если вы желаете проверить, является ли данная строчка (например, \textbf{-\/-p-\/-\/-r-\/-\/-\/-\/-} ) теоремой, вам придется следуя шаг за шагом, бросать в корзину все новые теоремы и сравнивать их с данной строчкой. Если таковая обнаружится, значит, это --- теорема. Если же в какой-то момент вы заметите, что все, что попадает в корзину, длиннее искомой строчки, можете прекратить поиски --- это не теорема. Такой разрешающий алгоритм называется восходящим, так как он исходит из основы, фундамента системы --- аксиом. Предыдущий алгоритм разрешения, наоборот, спускался сверху, приближаясь к фундаменту системы.

Изоморфизмы порождают смысл

Теперь мы подошли к центральному вопросу данной главы --- и книги в целом. Возможно, у вас уже мелькнула мысль, что теоремы~\textbf{pr} напоминают сложение. Строчка~\textbf{-\/-p-\/-r-\/-\/-\/-} ~является теоремой, потому что 2 плюс 3 равняется 5. Может быть, вы даже подумали, что теорема~\textbf{-\/-p-\/-\/-r-\/-\/-\/-\/-} ~не что иное как записанное необычной нотацией утверждение, означающее, что 2 плюс 3~равняется 5. На самом деле я нарочно выбрал буквы \textbf{p} и \textbf{r} , чтобы напомнить вам о словах «плюс» и «равняется». Так что же, строчка~\textbf{-\/-p-\/-\/-r-\/-\/-\/-\/-} ~на самом деле означает 2 плюс 3 равняется 5?

Что заставляет нас думать подобным образом? Мне кажется, что в этом виноват замеченный нами изоморфизм между системой~\textbf{pr} и сложением. Во введении термин «изоморфизм» был определен как трансформация,~сохраняющая информацию Теперь мы можем далее углубиться в это понятие и~рассмотреть его в иной перспективе. Слово «изоморфизм» приложимо к тем случаям, когда две сложные структуры могут быть отображены одна в другой таким образом, что каждой части одной структуры соответствует какая-то часть~другой структуры («соответствие» здесь означает, что эти части выполняют в своих структурах сходные функции). Такое использование слова «изоморфизм»~восходит к более точному математическому понятию.

Обнаружить изоморфизм между двумя известными ему структурами --- большая радость для математика. Часто это открытие изумительно и~неожиданно, как гром с ясного неба. Осознание изоморфизма между двумя хорошо известными структурами --- большой шаг вперед по дороге познания, и я считаю, что именно это порождает значения в человеческом мозгу. Для~полноты картины заметим, что поскольку изоморфизмы бывают самых различных типов, иногда не совсем ясно, когда же мы в действительности имеем дело с изоморфизмом. Таким образом, слову «изоморфизм», как и вообще всем~словам, присуща некая расплывчатость, что является одновременно и~достоинством, и недостатком.

В данном случае, у нас имеется великолепный прототип для понятия~«изоморфизм». Во первых, у нас есть «низший уровень» нашего изоморфизма --- соответствие между частями двух структур:

\textbf{p} ~\textless==\textgreater{} плюс

\textbf{r} ~\textless==\textgreater{} равняется

\textbf{-} ~\textless==\textgreater{} один

\textbf{-\/-} ~\textless==\textgreater{} два

\textbf{-\/-\/-} ~\textless==\textgreater{} три

и т. д.

Подобное соответствие между словами и символами называется~интерпретацией.

Во-вторых, на более высоком уровне, у нас имеется соответствие между~истинными утверждениями и теоремами. Заметим, однако, что это соответствие высшего уровня не может быть осознано, пока мы не выберем интерпретации для символов. Исходя из этого, правильнее будет говорить о соответствии между истинными~суждениями и интерпретированными теоремами. В любом случае, мы установили~соответствие между двумя порядками --- нечто типичное для изоморфизма. Когда вы имеете дело с формальной системой, о которой ничего не знаете и в которой желаете найти скрытое значение, ваша задача --- интерпретировать символы таким образом, чтобы установить соответствие между истинными~высказываниями и теоремами. Возможно, что сначала вам придется искать наугад, прежде чем вы найдете набор слов, подходящий для ассоциации с символами системы. Эта процедура весьма напоминает попытки расшифровать секретный код или прочитать надпись на незнакомом языке, как, например, критский~линейный В: единственный возможный путь состоит в использовании метода проб и ошибок, а также «разумных» догадок. Когда вы найдете правильный, «значащий» вариант, внезапно все приобретает смысл, и работа начинает идти во много раз быстрее. Очень скоро все встает на свои места. Счастливое волнение, испытываемое при этом исследователем, хорошо описано Джоном Чадвиком в его книге «Расшифровка линейного языка В» (John Chadwick, The Decipherment of Linear B).

Однако мало кому приходится расшифровывать формальные системы,~найденные в раскопках древних цивилизаций. Больше всего с формальными~системами имеют дело математики (а в последнее время также лингвисты,~философы и некоторые другие ученые); они придерживаются определенной~интерпретации в формальных системах, которые они изучают и используют. Эти~специалисты пытаются создать такую формальную систему, теоремы которой~изоморфно отражали бы какие-либо фрагменты действительности. В этом случае~выбор символов так же важен, как выбор типографских правил вывода. Задумав систему \textbf{pr} , я очутился как раз в таком положении. Читателю, вероятно, уже понятно, почему я выбрал именно такие символы. Теоремы системы~\textbf{pr} не~случайно изоморфны сложению; это получилось потому, что я искал способ~представить сложение типографским путем.

Интерпретации значащие и незначащие

Вы можете выбрать интерпретации, отличные от моей. При этом не~обязательно, чтобы каждая теорема оказывалась истинной. Однако какой смысл в такой интерпретации, при которой, скажем, все теоремы оказывались бы ложными? Еще более бессмысленной выглядит интерпретация, при которой теоремы~вообще никоим образом не соотносятся с критериями истинности или ложности. Нам придется поэтому различать два типа интерпретации формальных систем. Во-первых, мы можем говорить о \emph{незначащей} интерпретации, которая не~устанавливает никакой изоморфной связи между теоремами системы и~реальностью Подобных интерпретаций сколько угодно, годится любой случайный~выбор. Возьмем, например, такую интерпретацию

\textbf{p} \textless==\textgreater{} лошадь

\textbf{r} \textless==\textgreater{} счастливая

\textbf{-} ~\textless==\textgreater{} яблоко

Теперь строчка \textbf{-p-r-\/-} ~приобретает новую интерпретацию «Яблоко лошадь яблоко счастливая яблоко яблоко» Это поэтическое выражение, пожалуй,~может понравиться лошадям и даже показаться им наилучшей интерпретацией строчек данной системы. Однако в такой интерпретации весьма мало~«осмысленности», теоремы системы звучат ничуть не истинней и не лучше, чем не-теоремы. Утверждение «счастливая счастливая счастливая яблоко лошадь» (соответственно, \textbf{rrr-p} ) доставит нашей лошадке точно такое же удовольствие, как и любая интерпретированная теорема.

Другой тип интерпретации может быть назван \emph{значащим.} В такой~интерпретации, теоремы и истины совпадают --- то есть, между теоремами и~фрагментами реального мира существует изоморфизм. По этой причине мы будем различать \emph{интерпретацию} и \emph{значение.} Интерпретацией~\textbf{p} могло бы быть~любое слово, но «плюс» кажется мне единственным \emph{значащим} вариантом. Короче, наиболее вероятно что значение~«\textbf{p} » --- «плюс», хотя этот символ может иметь миллион различных интерпретаций.

Активные и пассивные значения

Возможно, что прочитавшие внимательно эту главу найдут самым важным в ней следующий факт: система \textbf{pr} , по всей видимости, заставляет нас признать, что \emph{поначалу абстрактные символы неизбежно приобретают некое~значение, по крайней мере, если мы находим какой-либо изоморфизм.} Однако между значением в формальных системах и значением в языке есть важное различие. Различие это заключается в том, что, выучив значение какого-либо слова, мы составляем затем новые предложения, основанные на этом значении. В определенном смысле значение становится \emph{активным} , так как оно~порождает новые правила создания предложений. Это означает, что наше владение языком не является законченным продуктом, правил производства~предложений становится все больше по мере того, как мы выучиваем новые значения. С другой стороны, в формальных системах теоремы предопределены правилами вывода. Мы можем выбирать «значения», основанные на изоморфизме (если таковой удается найти) между теоремами и истинными утверждениями. Однако это еще не разрешает нам по своему усмотрению прибавлять новые теоремы к уже имеющимся в системе. Именно об этом предупреждало нас в первой главе правило формальности.

В системе~\textbf{MIU} , разумеется, у нас не возникает искушения выйти за пределы четырех правил, так как мы не собираемся искать в ней никаких~интерпретаций. Однако здесь, в нашей новой системе, мы можем соблазниться новоприобретенным «значением» каждого символа и решить, что строчка

\textbf{-\/-p-\/-p-\/-p-\/-r-\/-\/-\/-\/-\/-\/-\/-}

является теоремой. По крайней мере, у нас может появиться такое \emph{желание} ; однако это не меняет того факта, что эта строчка --- не теорема. Было бы грубой ошибкой думать, что она «должна» быть теоремой, только лишь потому, что 2 плюс 2 плюс 2 плюс 2 равняется 8. Более того, было бы неверно~приписывать этой строчке вообще какое бы то ни было значение, поскольку она не является правильно построенной, в то время как наша интерпретация~полностью выводится из наблюдения над правильно построенными строчками.

В формальной системе значение должно оставаться \emph{пассивным} ; мы~можем прочитывать каждую строчку в зависимости от значения символов, ее составляющих, но нам не позволено создавать новые теоремы,основываясь назначениях, которые мы придаем этим символам. Интерпретированные~формальные системы находятся на границе между системами без значения и системами со значением. Мы можем считать, что их строчки что-то выражают, но это является не более как следствием формальных особенностей данной системы.

Double-entendre!

А теперь я хочу рассеять ваши иллюзии по поводу того, что мы нашли~единственно правильное значение для символов системы \textbf{pr} . Рассмотрим~следующее соотношение:

\textbf{p} \textless==\textgreater~равняется

\textbf{r} \textless==\textgreater~отнятое от

\textbf{-} ~\textless==\textgreater{} один

\textbf{-\/-} ~\textless==\textgreater{} два

и т. д.

Теперь~\textbf{-\/-p-\/-\/-r-\/-\/-\/-\/-~} приобретает новое значение: «2 равняется 3 отнятым от 5». Разумеется, это истинное утверждение; более того, в новой интерпретации все теоремы системы будут истинны. Новая интерпретация ровно настолько же~осмыслена, насколько и прежняя. Ясно, что глупо спрашивать, какое из двух~значений является истинным \emph{на самом деле} . Любая интерпретация истинна, если только она аккуратно отражает определенный изоморфизм с действительностью. Когда какие-либо аспекты действительности (в данном случае, сложение и~вычитание) изоморфны между собой, одна и та же система может быть изоморфна обоим этим аспектам и в результате иметь два пассивных значения. Тот факт, что одни и те же символы могут иметь различные значения, чрезвычайно важен. В нашем примере это могло показаться вам тривиальным, или любопытным, или вообще неинтересным; однако когда мы вернемся к этой теме в более сложном контексте, читатель увидит, какое богатство идей она заключает.

Подведем итоги тому, что мы сказали о системе \textbf{pr} . В каждой из двух значащих интерпретаций, любая правильно построенная строчка соответствует какому-либо грамматическому высказыванию. Некоторые из этих высказываний окажутся истинными, некоторые --- ложными. В любой формальной системе \emph{правильно построенными строчками} являются те, которые, будучи проинтерпретированы символ за символом, порождают грамматические высказывания. (Безусловно, это зависит от самой интерпретации, но обычно мы уже имеем в виду какую-то одну из них.) Среди правильно построенных строчек некоторые являются теоремами. Теоремы определяются схемой аксиом и~правилом вывода. Моей целью, когда я придумывал систему \textbf{pr} , являлась имитация сложения: каждая теорема, интерпретированная определенным образом,~выражает истинный пример сложения; наоборот, каждое уравнение сложения двух целых положительных чисел может быть записано в форме строчки,~оказывающейся теоремой. Эта цель была достигнута. Таким образом, заметьте, что все ошибочные примеры сложения, такие, как, например, 2 плюс 3 равняется 6, соответствуют правильно построенным строчкам, которые, однако, не являются теоремами.

Формальные системы и действительность

Это был наш первый пример того, как формальная система может быть~основана на фрагменте действительности и точно отображать его в том смысле, что теоремы этой системы изоморфны истинным утверждениям данной части~действительности. Однако надо иметь в виду, что действительность и формальные системы не зависят друг от друга. Никто не обязан знать об изоморфизме~между ними. Каждая из этих систем существует сама по себе: 1 плюс 1 равняется 2, независимо от того, знаем ли мы, что~\textbf{-p-r-\/-} ~является теоремой; с другой~стороны,~\textbf{-p-r-\/-} ~является теоремой, независимо от того, соотносим ли мы ее с~примером сложения.

Читатель может спросить, помогает ли создание этой (или любой другой) формальной системы узнать что-либо новое об области ее интерпретации.~Выучили ли мы какие-нибудь новые примеры сложения путем производства \textbf{pr} -теорем? Разумеется, нет; однако мы узнали что-то новое о самом процессе~сложения, а именно, что оно легко может быть имитировано с помощью~типографского правила, управляющего абстрактными символами. Это пока не~удивительно, так как сложение --- весьма простое понятие. Всем известно, что суть сложения может быть «уловлена» скажем, при наблюдении за вращающимися шестеренками кассового аппарата.

Ясно, что мы затронули лишь самые начатки формальных систем;~естественно, возникает вопрос, какие именно фрагменты действительности могут быть отражены при помощи набора бессмысленных символов, управляемых~формальными законами? Может ли вся реальность быть превращена в формальную~систему? В очень широком смысле кажется, что на этот вопрос можно ответить положительно. Мы можем предположить, например, что вся действительность --- это не более чем весьма сложная формальная система. Ее символы находятся не на бумаге, а в трехмерном вакууме (пространстве); это элементарные частицы, из которых устроена вселенная. (Мы предполагаем здесь, что материя не делится до бесконечности, и что, таким образом, выражение «элементарные частицы» имеет смысл.) «Типографские правила» такой формальной системы --- законы физики, которые, учитывая положение и скорость всех частиц в данный момент, говорят нам, какие изменения произойдут, и каковы будут новая скорость и положение частиц в «следующий» момент. Таким образом, теоремами этой огромной формальной системы являются все возможные конфигурации частиц во все времена истории вселенной. Единственной аксиомой здесь является (или являлось)~первоначальное положение всех частиц в «начале времен». Однако это концепция столь грандиозна, что представляет лишь сугубо теоретический интерес; к тому же, достижения квантовой механики (и других областей физики) вносят некие сомнения даже и в чисто теоретическую ценность этой идеи. Проблема сводится к вопросу, функционирует ли вселенная по законам детерминизма; этот вопрос пока остается открытым.

Математика и манипуляция символами

Вместо того, чтобы иметь дело с такой огромной картиной, возьмем в качестве нашей «действительности» математику. Тут мы сталкиваемся с серьезным~вопросом: можем ли мы быть уверены в точности нашей формальной системы, моделирующей какую-либо область математики, в особенности, если мы еще не изучили данную часть математики вдоль и поперек? Предположим, что цель формальных систем --- дать нам новые знания по данной дисциплине. Каким образом мы узнаем, что интерпретация каждой теоремы истинна? Для этого пришлось бы доказать, что между формальной системой и данной частью~математики существует полный изоморфизм. С другой стороны, подобное~доказательство возможно только в том случае, если нам с самого начала уже~известны все истинные утверждения данной дисциплины!

Представьте себе, что в каких-то раскопках мы обнаружили некую~таинственную формальную систему. Вероятно, мы опробовали бы несколько~интерпретаций, пока не наткнулись бы на такую, в которой каждая теорема была бы истинной и каждая не-теорема --- ложной. Однако мы можем проверить это лишь на ограниченном количестве случаев, в то время как теорем, скорее всего, бесконечное множество. Можно ли утверждать, что все теоремы выражают истину в данной интерпретации, если нам еще не известно все и о формальной системе, и об области ее интерпретации?

В таком же положении мы оказываемся, когда пытаемся при помощи~типографских символов формальной системы описать фрагмент~действительности, представленный натуральными числами (то-есть, неотрицательными~целыми числами: 0, 1, 2,\ldots), . Попробуем понять отношение между тем, что мы называем «истиной» в теории чисел, и тем, к чему мы можем придти путем манипуляции символами.

Для начала посмотрим, какие основания у нас существуют для того, чтобы называть одни утверждения теории чисел истинными, а другие --- ложными? Сколько будет 12 умножить на 12? Любой знает, что 144. Однако многие ли из тех, кто уверенно дает этот ответ, когда-либо рисовали прямоугольник~размером 12~x 12 и подсчитывали составляющие его квадратики? Большинство людей считают, что эта процедура совсем не нужна. Вместо нее в доказательство своей правоты они предлагают несколько значков на бумаге, вроде тех, что показаны ниже:

Это и будет «доказательством». Почти все верят, что если посчитать квадратики, получится 144; мало кто когда-либо усомнился в этом результате. Конфликт между двумя точками зрения становится еще заметнее, когда мы рассматриваем такую проблему, как нахождение произведения 987654321~\&\#215; 123456789. Прежде всего, практически невозможно построить прямоугольник нужного размера; но хуже всего то, что, даже если бы нам и удалось таковой построить и армии людей потратили бы столетия на подсчет квадратиков, все равно конечному результату поверил бы разве что особенно доверчивый человек. Слишком велика вероятность того, что кто-нибудь обязательно что-то напутал. Возможно ли, в таком случае, узнать ответ? Да, если вы доверяете символическому процессу манипуляции числами при помощи некоторых простых законов. Этот процесс объясняют детям как способ нахождения верного ответа; при этом мало кто из них видит, какой смысл скрывается за этим арифметическим трюком. Правила, маневрирующие цифрами при умножении, основаны на нескольких основных свойствах сложения и умножения, которые считаются верными для всех чисел.

Основные законы арифметики

Свойства, которые я имею в виду, можно пояснить на следующем примере. Представьте, что вы выкладываете несколько палочек:

/ // // // / /

и начинаете их считать. В то же время кто-то подсчитывает эти же палочки, начиная с другого конца. Читателю, вероятно, понятно, что результат получится одинаковый. Результат подсчета не зависит от того, как этот подсчет делается. Было бы бессмысленно пытаться доказать это предположение о свойствах сложения, настолько оно первично: либо вы его понимаете, либо нет --- но в последнем случае вам не поможет никакое доказательство. Из этого предположения вытекают свойства коммутативности и ассоциативности сложения (первое заключается в том, что \emph{b + с = с + b} во всех случаях; второе --- в том, что \emph{b + (с + d) = (b + с) + d} во всех случаях). То же предположение приводит нас к свойствам коммутативности и ассоциативности в умножении; достаточно представить множество кубиков, собранных вместе таким образом, что они составляют большое прямоугольное твердое тело. Коммутативность и ассоциативность умножения означают, что как бы вы ни поворачивали это тело, количество кубиков в нем не изменится. Эти предположения невозможно проверить во всех случаях, так как количество комбинаций бесконечно. Мы принимаем их как данное и верим в них (если мы вообще когда-нибудь о них задумываемся) так глубоко, как только можно во что-либо верить. Количество монет у нас в кармане не меняется оттого, что при ходьбе они перемещаются и бренчат; количество наших книг не изменится, если мы упакуем их в коробку, бросим коробку в багажник машины, отъедем на 100 километров, распакуем коробку и поставим книги на новую полку. Все это --- часть того, что мы понимаем под словом \emph{число} .

Встречаются люди, которые, столкнувшись с формулировкой какого-либо очевидного факта, находят удовольствие в том, что тут же пытаются доказать обратное. Я сам такой Фома Неверующий: записав свои примеры с палочками, деньгами и книгами, я сразу выдумал ситуации, в которых эти примеры перестают быть правильными. Вы, возможно, сделали то же самое. Все это я говорю к тому, чтобы показать, что числа как математическая абстракция весьма отличны от чисел, которые мы употребляем в повседневной жизни.

Все мы любим изобретать поговорки, которые, нарушая основные законы арифметики, иллюстрируют некие более глубокие «истины»:~«1 да 1 равно 1» (любовники) или~«1 плюс 1 плюс 1 равно 1» (святая Троица). Можно легко найти изъяны в подобных «формулах» --- скажем, показав, что употребление знака «плюс» в них неверно. Так или иначе, подобных высказываний множество. По забрызганному дождем оконному стеклу сползают две капли; у самой рамы они сливаются в одну. Значит ли это, что 1 + 1 = 1? Из одного облака рождаются два; не доказательство ли это той же идеи? Отличить случаи, в которых мы можем говорить о сложении, от тех, где нам нужно какое-то другое понятие, не так-то просто. Размышляя об этом, мы, возможно, додумаемся до таких критериев, как разделение объектов в пространстве и их четкое отличие друг от друга. Но как подсчитать идеи? Или количество газов в атмосфере? Во многих источниках можно встретить высказывания типа: «В Индии 17 языков и 462 диалекта». В точных утверждениях такого рода есть нечто странное, так как сами понятия «язык» и «диалект» довольно расплывчаты.

Идеальные числа

В повседневном мире числа часто ведут себя плохо. Однако у людей имеется врожденное, пришедшее из древности чувство, что этого быть не должно. В абстрактном понятии числа, взятого вне связи с подсчетом бусинок, диалектов или облаков, есть нечто чистое и точное; должен существовать способ говорить о числах, не примешивая к ним глупую повседневность. Твердые правила, управляющие идеальными числами, являются основой арифметики, в то время как их следствия лежат в основе теории чисел. При переходе от чисел как объектов повседневной жизни к числам как объектам формальной системы возникает следующий важный вопрос: возможно ли заключить всю теорию чисел в рамки одной формальной системы? Действительно ли числа так чисты, ясны и регулярны, что их природа может быть полностью описана правилами какой-либо формальной системы? Картина «Освобождение», одно из самых прекрасных произведений Эшера, иллюстрирует этот удивительный контраст между формальным и неформальным и поразительную зону перехода между ними. Действительно ли числа свободны, как птицы? Страдают ли они, уловленные в тесную клетку формальной системы? Существует ли магическая зона перехода между числами, используемыми в повседневной жизни, и числами, написанными на бумаге?

Говоря о свойствах натуральных чисел, я имею в виду не только такие свойства, как, скажем, сумма определенной пары чисел. Ее легко можно подсчитать; никто из нас, выросших в двадцатом веке, не сомневается в возможности механизации таких процессов, как подсчет, сложение, умножение, и т. д. Я имею в виду такие свойства чисел, исследованием которых занимаются математики и для познания которых не достаточно, даже теоретически, никакого подсчета. Рассмотрим классический пример: утверждение «существует бесконечно много простых чисел». Прежде всего, не существует такого метода подсчета, который мог бы доказать или опровергнуть это утверждение. Лучшее, что мы~можем сделать, --- это затратить некоторое время на подсчет простых чисел и заключить, что их действительно имеется «целая куча». Однако никакой подсчет не скажет нам того, конечно или бесконечно количество простых чисел; любой подсчет всегда останется неполным. Это утверждение, называющееся «Теорема Эвклида» (обратите внимание на заглавную «Т»), совсем не очевидно. Однако со времен Эвклида все математики считают его истинным. В чем же дело?

\emph{Рис. 13. М. К. Эшер «Освобождение» (литография, 1955)}

Доказательство Эвклида

Дело в том, что этот факт следует из неких рассуждений. Давайте проследим за этими \emph{рассуждениями} . Рассмотрим вариант доказательства Эвклида, показывающий, что какое бы число мы ни взяли, всегда найдется большее простое число. Возьмем число N. Перемножим все положительные целые числа, начиная с 1 и кончая N; иными словами, найдем факториал N (он пишется «N!») Полученный результат делится на все числа, меньшие чем N. Если прибавить 1 к N!, то результат

не будет делиться на 2 (так как при делении на 2 получится 1 в остатке);

не будет делиться на 3 (так как при делении на 3 получится 1 в остатке);

не будет делиться на 4 (так как при делении на 4 получится 1 в остатке);

.

.

.

не будет делиться на N (так как при делении на N получится 1 в остатке);

Другими словами, если N!+1 и делимо на какое-то число, кроме самого себя и единицы, оно делимо только на числа, большие, чем N. Следовательно, либо N!+1 само простое число, либо его простые делители больше N. В любом случае, мы показали, что должно существовать простое число, большее N, и что, следовательно, количество простых чисел бесконечно.

Кстати, этот последний шаг называется \emph{обобщением} ; мы еще встретимся с этим понятием в более сложном контексте. Оно заключается в том, что, начав наши рассуждения с какого-либо числа N, мы указываем, что N может быть любым числом --- следовательно, наше доказательство носит общий характер.

Эвклидово доказательство типично для так называемой «реальной математики». Оно просто, точно и изящно и иллюстрирует тот факт, что несколько коротких шагов могут увести нас весьма далеко от начального пункта. В нашем случае, таким начальным пунктом являлись основные идеи о свойствах умножения, деления, и так далее. Короткие шаги --- это этапы рассуждения. Хотя каждый отдельный шаг кажется очевидным, конечный результат таковым не является. Нам никогда не удастся проверить, верно ли это утверждение Эвклида; однако мы верим в его истинность, поскольку мы верим в логические рассуждения. Если вы принимаете эти рассуждения, вам не остается выхода; раз вы согласились выслушать Эвклида, вам придется согласиться с его выводом. Этот отрадный факт означает, что математики всегда могут придти к согласию по поводу того, какие утверждения считать «истинными», а какие --- «ложными».

Это доказательство --- пример упорядоченного процесса мысли. Каждое утверждение соотносится с предыдущим неоспоримым образом; именно поэтому мы говорим скорее о «доказательстве», чем об «очевидном свидетельстве». Целью математики всегда являлось нахождение строгого доказательства какого-либо неочевидного утверждения. Сам факт строгого соотношения шагов доказательства указывает на то, что должна существовать определенная схема, связывающая эти утверждения в одно логическое целое. Об этой схеме лучше всего рассуждать при помощи специального нового лексикона, состоящего из символов, годных только для описания утверждений о числах. Таким образом, мы сможем рассмотреть версию доказательства в «переводе». Это будет набор утверждений, строго соотносящихся между собой; причем эти отношения всегда можно описать. Утверждения, поскольку они записаны компактными, стилизованными символами, выглядят как определенные \emph{структуры} . Другими словами, при прочтении вслух мы видим, что эти утверждения говорят о числах и их свойствах; записанные же на бумаге, они выглядят как абстрактные структуры. Таким образом, последовательно, строка за строкой прочитанная схема доказательства начинает казаться постепенной трансформацией структур по определенным типографским правилам.

Минуя бесконечность

Хотя Эвклид доказывает, что \emph{каждое} число обладает определенным свойством, он, тем не менее, не рассматривает в отдельности каждый из бесконечно многих случаев. Для этого он использует выражения типа «каким бы числом N ни было», или «неважно, какое N мы возьмем». Мы могли бы перефразировать доказательство, используя фразу «все N». Умело обращаясь с подобными выражениями, мы всегда можем избежать возни с бесконечным количеством утверждений. Вместо этого мы будем иметь дело лишь с двумя-тремя понятиями, например, такими, как слово «все». Сами по себе конечные, они воплощают в себе бесконечность и поэтому позволяют нам обойти такое препятствие, как необходимость доказывать бесконечное количество фактов.

Мы используем слово «все» по-разному, что определено нашим мыслительным процессом: существуют правила, которым подчиняется наш выбор. Возможно, что мы не сознаем этого и утверждаем, что руководствуемся \emph{значением} слова; однако это лишь иносказание, выражающее все ту же идею; наше мышление подчиняется определенным негласным законам. Всю жизнь мы используем слова как часть определенных структур; но, вместо того, чтобы называть эти структуры «правилами», мы приписываем их возникновение и развитие «значениям» слов. Это открытие было решающим шагом на пути формализации теории чисел.

Рассмотрев доказательство Эвклида более внимательно, мы увидели бы, что оно складывается из многих крохотных, почти бесконечно малых шагов. Если бы мы записали их одно за другим, доказательство показалось бы невероятно сложным. Оно кажется нам легче, когда несколько шагов складываются на манер телескопа и составляют одно-единственное предложение. Если бы мы рассмотрели это доказательство, как в замедленной съемке, перед нами предстали бы отдельные «секции». Другими словами, деление может идти лишь до определенного предела, за которым мы сталкиваемся с «атомной» природой мыслительных процессов. Доказательство может быть разбито на серию крохотных, но отдельных этапов; рассмотренные «издалека», они сливаются в непрерывный поток. В главе VIII я приведу пример такой «атомизации» доказательства, и вы увидите, какое множество шагов в нем участвует. Возможно, что это вас не удивит. В мозгу у Эвклида, когда он изобретал свое доказательство, работали миллионы нейронов, многие из которых давали сотни импульсов в секунду. Чтобы произнести одно-единственное предложение, в мозгу задействованы сотни тысяч нейронов. Если мысли Эвклида были настолько сложны, логично ожидать, что его доказательство также состоит из огромного количества шагов! (Хотя, скорее всего, прямой связи между нейронной активностью мозга и доказательством в нашей формальной системе не существует, они, тем не менее, сравнимы по своей сложности --- словно природа желает сохранить сложность доказательства бесконечного множества простых чисел, несмотря на то, что это доказательство представлено в таких различных системах.)

В последующих главах мы разработаем такую формальную систему, которая (1) включает стилизованный лексикон, способный выразить все высказывания о натуральных числах и (2) имеет правила, соответствующие всем необходимым типам рассуждений. При этом возникает вопрос, сравнима ли мощность подобных формальных правил (по крайней мере, в сфере теории чисел) с мощностью тех правил, которыми мы регулярно пользуемся в наших мыслительных процессах. Иными словами, существует ли теоретическая возможность, используя формальную систему, достичь уровня наших мыслительных способностей?


% % \subsubsection{Соната для Ахилла соло}
% \subsubsection{Соната для Ахилла соло}

\emph{Звонит телефон~--- Ахилл берет трубку.}

\emph{Ахилл} : Алло, Ахилл слушает.

\emph{Ахилл} : А, здравствуйте, г-жа Черепаха. Как дела?

\emph{Ахилл} : Кривошея и чихиллит? Что такое чихи\ldots~--- а, теперь понимаю. Будьте здоровы!\ldots{} Что и говорить, неприятная комбинация. Как это вы ухитрились такое подцепить?

\emph{Ахилл} : И долго вы ее так продержали?

\emph{Ахилл} : Еще на самом сквозняке~--- не удивительно, что вам в шею надуло!

\emph{Ахилл} : Что же вас заставило так долго там проторчать?

\emph{Ахилл} : Многие из них удивительные? Какие, например?

\emph{Ахилл:} Фантасмагорические чудища? Что вы имеете в виду?

\emph{Ахилл} : И вам не страшно было в такой компании?

\emph{Ахилл} : Гитара? Вот странно~--- откуда взялась гитара среди этих диковинных созданий. Кстати, вы играете на гитаре?

\emph{Ахилл} : Ах, для меня это одно и то же.

\emph{Ахилл} : Вы правы удивительно, как это я сам до сих пор не заметил, в чем разница между гитарой и скрипкой. Кстати о скрипках: не хотите ли вы~~заглянуть ко мне и послушать сонату для скрипки соло вашего любимого композитора, И. С. Баха? Я только что купил отличную запись.~Поразительно, как это Баху удалось, используя одну-единственную скрипку,~создать такую интересную вещь.

\emph{Ахилл} : Головная боль тоже? Бедняжка\ldots{} Пожалуй, вам лучше лечь в постель и постараться заснуть.

\emph{Ахилл} : Понятно. Овец считать уже пробовали? Где-то у меня была целая~картотека подобных трюков~--- говорят, они здорово помогают от бессоницы.

\emph{Ахилл} : Ах, да. Я отлично понимаю, что вы имеете в виду~--- я это тоже пробовал. Может быть, если уж эта задачка так застряла у вас в голове, вы~поделитесь ею со мной, чтоб и я мог попробовать свои силы?

\emph{Ахилл} : Слово, внутри которого встречаются подряд буквы «Р», «Т», «О», «Т», «Е»\ldots{} Г-м-м\ldots{} Как насчет «ретотра»?

\emph{Ахилл} : Ах, какой стыд\ldots{} Конечно вы правы~--- я опять все перепутал. К тому же в слове «реторта» эти буквы все равно идут задом наперед.

\emph{Ахилл} : Уже несколько часов? Хорошенькую вы мне задали задачку\ldots{} Где вы откопали такую дьявольскую головоломку?

\emph{Ахилл} : Вы имеете в виду, что он только делал вид, что размышляет над~эзотерическими буддистскими проблемами, когда на самом деле он пытался придумать сложные словесные головоломки?

\emph{Ахилл} : Ага! Улитка знала, чем он занимается. Как же вам удалось с ней~переговорить?

\emph{Ахилл} : Вы знаете, я как-то слышал похожую головоломку. Хотите, я вам ее задам? Или это еще хуже вас отвлечет?

\emph{Ахилл} : Согласен~--- хуже уже вряд ли будет. Так вот: какое слово начинается с «КА» и кончается на «КА»?

\emph{Ахилл} : Очень остроумно~--- но это нечестно. Я совершенно не это имел в виду!

\emph{Ахилл} : Согласен, это слово выполняет условие; но все равно это какое-то~дегенеративное решение.

\emph{Ахилл} : Абсолютно верно! Как вам удалось так быстро найти ответ?

\emph{Ахилл} : Это~--- еще один пример того, какой полезной может оказаться картотека трюков от бессоницы. Прекрасно! Но я все еще блуждаю в потемках с вашей задачкой о «PTOTE».

\emph{Ахилл} : Поздравляю~--- теперь вам, может быть, удастся заснуть. Скажите же мне решение!

\emph{Ахилл} : Вообще-то я не люблю подсказок, но на этот раз ладно, валяйте.

\emph{Ахилл} : Не понимаю. Что вы имеете в виду под «рисунком» и «фоном»?

\emph{Ахилл} : Разумеется, я знаком с «Мозаикой II». Я знаю ВСЕ работы Эшера. В конце концов, это мой любимый художник! Кстати, репродукция «Мозаики II» висит прямо у меня перед носом.

\emph{Ахилл} : Всех черных зверей? Конечно, вижу!

\emph{Ахилл} : Верно: их «негативное пространство»~--- то, что остается свободным~--- определяет белых зверей.

\emph{Ахилл} : А, так вот что вы называете «рисунком» и «фоном»! Но какое отношение это имеет к головоломке о «Р-Т-О-Т-Е»?

\emph{Ахилл} : Это для меня слишком сложно\ldots{} Теперь и у меня начинает болеть голова; пойду, пожалуй, поищу мою спасительную картотеку, может быть она мне поможет забыться сном.

\emph{Ахилл} : Вы хотите зайти сейчас? Но я думал\ldots{}

\emph{Ахилл} : Ну что ж, хорошо. А я пока постараюсь решить эту задачку с помощью вашей подсказки о рисунке и фоне и моей головоломки.

\emph{Ахилл} : С удовольствием сыграю их для вас.

\emph{Ахилл} : Вы изобрели о них теорию?

\emph{Ахилл:} В сопровождении какого инструмента?

\emph{Ахилл} : В таком случае, как странно, что он не записал также и партию~клавесина, и не опубликовал их в таком виде.

\emph{Ахилл} : А, понимаю~--- нам предоставляется выбор: слушать ее с~аккомпанементом или без оного. Но откуда мы знаем, как он должен звучать?

\emph{Ахилл} : Да, вы правы~--- наверное, лучше всего оставить эту работу воображению слушателя. Согласен~--- может быть, у Баха в мыслях вообще не было никакого аккомпанемента. Действительно, эти сонаты и так звучат~замечательно.

\emph{Ахилл} : Точно. Ну, до скорого.

\emph{Ахилл:} Пока, г-жа Ч.

\emph{Рис. 14. М К. Эшер. «Мозаика II» (литография, 1957).}


% % \subsubsection{ГЛАВА III: Рисунок и фон}
% \subsubsection{ГЛАВА III: Рисунок и фон}

Простые и составные числа

ТО, ЧТО некоторые понятия можно выразить при помощи простых~манипуляций типографскими символами, кажется довольно странным. До сих пор мы передали таким образом лишь понятие сложения, и это, возможно, не~показалось нам удивительным. Предположим, однако, что мы захотим создать~формальную систему с теоремами вида \textbf{P} \emph{x} , где~\emph{x} было бы строчкой, состоящей из тире. Количество этих тире должно было бы выражаться простым числом. Так,~\textbf{P-\/-~} ---~было бы теоремой, в то время как~\textbf{P-\/-\/-} теоремой бы не являлось. Как это может быть выражено с помощью типографских операций? Сначала~необходимо точно определить, что мы имеем в виду под «типографскими операциями». Полное описание было дано в системах~\textbf{MIU} и \textbf{pr} , так что сейчас мы~ограничимся только списком наших возможностей:

(1) читать и узнавать любое из конечных множеств символов;

(2) записывать любой из символов, принадлежащий такому множеству,

(3) повторять любой из этих символов в другом месте;

(4) стирать любой из этих символов;

(5) проверять, одинаковы ли два символа;

(6) сохранять и использовать список ранее выведенных теорем.

Список получился немного повторяющимся, но это не столь важно.~Главное то, что он позволяет только самые тривиальные операции, намного проще, чем операция отличения простого числа от не простого. Как же, в таком случае, мы сможем совместить несколько операций и создать такую формальную~систему, в которой простые числа отличались бы от составных?

Система~ur

Первым шагом может стать решение более простой, но сходной задачи. Мы можем попытаться придумать систему, похожую на систему \textbf{pr} , но которая вместо сложения представляла бы умножение. Назовем ее системой~\textbf{ur} (u = «умноженное на»). Предположим, что X, Y, Z , соответственно, --- это~количество тире в строчках x, y, z. (Обратите внимание, что я специально делаю упор на различии между строчкой, и количеством тире, которое эта строчка~содержит.) Мы хотим, чтобы строчка~x\textbf{u} y\textbf{rz} была теоремой только в том случае, когда X, умноженное на Y, равняется Z. Например,~\textbf{-\/-u-\/-\/-r-\/-\/-\/-\/-\/-} ~должно быть теоремой, так как 2, умноженное на 3, равняется 6, в то время как~\textbf{-\/-u-\/-r-\/-\/-~} теоремой быть не должно. Систему~\textbf{ur} так же просто описать, как и систему \textbf{pr} . Для этого нужны всего лишь одна аксиома и одно правило вывода

СХЕМА АКСИОМ:~\emph{x} \textbf{u-r} \emph{x} является аксиомой, когда~\emph{x} ~--- строчка, состоящая из тире.

ПРАВИЛО ВЫВОДА: Предположим, что \emph{x} , \emph{у} , и \emph{z} ~--- строчки тире, и что~\emph{x} \textbf{u} \emph{y} \textbf{r} z --- старая теорема. Тогда~\emph{x} \textbf{u} \emph{y\textbf{-}} \textbf{r} \emph{zx} будет новой теоремой.

Ниже приводится вывод теоремы~\textbf{-\/-u-\/-\/-r-\/-\/-\/-\/-\/-}

(1)~\textbf{-\/-u-r-\/-~} (аксиома)

(2)~\textbf{-\/-u-\/-r-\/-\/-\/-} ~(по правилу вывода, используя (1) в качестве старой теоремы)

(3)~\textbf{-\/-u-\/-\/-r-\/-\/-\/-\/-\/-} ~(по правилу вывода, используя (2) в качестве старой теоремы)

Обратите внимание, что количество тире в средней строке возрастает на~единицу каждый раз, когда мы применяем правило вывода, таким образом, мы можем предсказать, что если мы хотим получить теорему с десятью тире в середине, нам придется применить правило вывода девять раз подряд.

Уловление Составности

Умножение (немного более сложное понятие, чем сложение) теперь уловлено нами в сети типографских правил, подобно птицам в Эшеровском «Освобождении». А как же насчет простых чисел? Следующий план кажется неплохим: используя систему \textbf{ur} , определить новое множество теорем вида S\emph{x} , которые характеризуют \emph{составные числа}

ПРАВИЛО: Предположим, что \emph{x} , \emph{у} , \emph{z} --- строчки тире. Если~\emph{x\textbf{-}} \textbf{u} \emph{y\textbf{-}} \textbf{r} \emph{z} является теоремой, то~S\emph{z} также будет теоремой.

Это означает, что Z (число тире в z) является составным, если оно ---~произведение двух чисел, больших единицы (а именно, X+1 --- число тире в \emph{x\textbf{-}} и Y+1 --- число тире в \emph{y\textbf{-}} ). Я объясняю вам это новое правило в «интеллектуальном режиме», поскольку вы, как существо мыслящее, желаете знать, почему такое правило существует. Если бы вы работали исключительно в «механическом режиме», вам бы не понадобились никакие объяснения, так как работающие в режиме~\textbf{M} следуют правилам чисто механически, никогда не задавая вопросов, и при этом совершенно счастливы!

Поскольку вы работаете в режиме \textbf{I} , вы будете склонны забывать о~различии между строчками и их интерпретацией. Ситуация может стать довольно запутанной, как только вы обнаружите смысл в символах, которыми вы~манипулируете. Вам придется бороться с собой, чтобы не решить, что строчка «\textbf{-\/-\/-»} ~--- то же самое, что число 3. Требование формальности, казавшееся совершенно очевидным в главе I, здесь становится весьма каверзным и приобретает~первостепенную важность Именно оно не дает вам спутать режим \textbf{I} с режимом \textbf{M} , иными словами, оно не позволяет вам смешивать арифметические факты с типографскими теоремами.

«Нелегальная» характеристика простых чисел

Весьма соблазнительно от теорем типа S сразу перескочить к теоремам типа P, путем введения следующего правила

ПРЕДЛОЖЕННОЕ ПРАВИЛО: Предположим, что~\emph{x} --- строчка тире. Если S\emph{x} не является теоремой, то~P\emph{x} является теоремой.

Роковая ошибка здесь заключается в том, что проверка «нетеоремности» S\emph{x} --- не типографская операция. Чтобы узнать наверняка, что \textbf{MU} --- не теорема \textbf{MIU} , нам пришлось бы \emph{выйти из системы} ; в такую же ситуацию мы попадаем и с Предложенным Правилом. Оно подрывает сами основы формальных систем тем, что предлагает вам действовать неформально, вне системы. Типографская операция (6) позволяет вам рассматривать предварительно выведенные~теоремы; однако Предложенное Правило отсылает вас к гипотетической «таблице не-теорем». Чтобы получить подобную таблицу, вам придется работать \emph{вне~системы} , показывая, почему некоторые строчки не могут быть получены в~данной системе. Конечно, может оказаться, что существует другая формальная система, в которой «таблица не-теорем» может быть получена чисто~типографскими способами. На самом деле, наша цель --- найти именно такую систему.Однако Предложенное Правило --- не типографское, а посему нам придется от него отказаться.

Это настолько важный момент, что мы остановимся на нем поподробнее. В нашей \emph{системе S} (включающей систему~\textbf{ur} и правила, определяющие теоремы типа S) у нас есть теоремы вида S\emph{x} , где \emph{x} , как обычно, обозначает строчку тире. В ней имеются также не-теоремы вида \emph{S} x. Говоря о не-теоремах, я имею в виду именно эту разновидность, хотя, конечно, существует множество не-теорем в виде неправильно сформированных строчек:~\textbf{u u-S r r} ~и пр. Между теоремами и не-теоремами есть следующая разница: количество тире в первых ---~составное число, во вторых --- простое. К тому же, все теоремы похожи по форме, так как все они выведены при помощи одного и того же набора типографских правил. Можем ли мы сказать, что в этом смысле все не-теоремы также имеют что-то общее в форме? Ниже приводится список теорем типа S, без их вывода. Число в скобках указывает на количество тире в соответствующей теореме.

S\textbf{-\/-\/-\/-} ~(4)

S\textbf{-\/-\/-\/-\/-\/-} ~(6)

S\textbf{-\/-\/-\/-\/-\/-\/-\/-} ~(8)

S\textbf{-\/-\/-\/-\/-\/-\/-\/-\/-} ~(9)

S\textbf{-\/-\/-\/-\/-\/-\/-\/-\/-\/-} ~(10)

S\textbf{-\/-\/-\/-\/-\/-\/-\/-\/-\/-\/-\/-} (12)

S\textbf{-\/-\/-\/-\/-\/-\/-\/-\/-\/-\/-\/-\/-\/-} (14)

S\textbf{-\/-\/-\/-\/-\/-\/-\/-\/-\/-\/-\/-\/-\/-\/-} ~(15)

S\textbf{-\/-\/-\/-\/-\/-\/-\/-\/-\/-\/-\/-\/-\/-\/-\/-} ~(16)

S\textbf{-\/-\/-\/-\/-\/-\/-\/-\/-\/-\/-\/-\/-\/-\/-\/-\/-\/-} ~(18)

.

.

.

«Дырки» в этом списке как раз и являются не-теоремами. Есть ли у них какая-то общая «форма»? Можно ли предположить, что лишь постольку,~поскольку они являются пробелами в неком упорядоченном списке, они обладают какими-то общими чертами? И да, и нет. Нельзя отрицать, что у них есть общие типографские черты; вопрос в том, правомочно ли называть эти черты «формой». Дело в том, что дырки определены только негативно: они представляют из себя то, что осталось от позитивно определенного списка.

~Рисунок и фон

Это напоминает известное разграничение между рисунком и фоном в~живописи. Когда предмет или «положительное пространство» (например, человеческая фигура, буква или натюрморт) рисуется внутри рамки, неизбежным следствием этого является появление на картине дополняющей формы, также~называющейся «фоном», или «негативным пространством». В большинстве картин~отношение между фоном и рисунком почти не играет роли; как правило, художник в основном занят рисунком. Однако иногда его внимание привлекает также и фон.

Существуют замечательные шрифты, обыгрывающие это различие между рисунком и фоном. Послание, написанное таким шрифтом, приводится ниже. На первый взгляд это просто несколько клякс; но если вы посмотрите на них издали, попристальнее, то увидите семь букв на этом РИСУНКЕ (специальным шрифтом, так, что черный фон, создающий белые буквы, похож на кляксы.)

\emph{Рис. 15. Рисунок}

Такой же эффект производит мой рисунок «Знак из дыма» (рис. 139).~Продолжая в том же ключе, попробуйте решить следующую задачку: возможно ли нарисовать такую картину, чтобы слова были как на рисунке, так и в фоне?

Давайте условимся различать между двумя типами рисунков: \emph{курсивно рисуемыми} и \emph{рекурсивными} (эти термины не являются~общеупотребительными~--- их придумал я сам). В \emph{курсивно рисуемом} рисунке фон является лишь побочным продуктом. В \emph{рекурсивном} рисунке, наоборот, фон может~рассматриваться как отдельный самостоятельный рисунок. Обычно художник делает это вполне сознательно. Приставка «ре» здесь выражает тот факт, что как~рисунок, гак и фон могут быть нарисованы курсивно, то есть, такая картина «дву-курсивна». Любой контур на рекурсивном рисунке --- это обоюдоострый меч. М. К. Эшер был мастером подобных картин; взгляните, например, на его~великолепную рекурсивную гравюру «Птицы» (рис. 16).

\emph{Рис. 16. M. K. Эшер. «Деление пространства при помощи птиц» (из блокнота 1942 года).}

Различие здесь не такое строгое, как в математике; кто может с~уверенностью утверждать, что некий фон не является в то же время и рисунком? При достаточно внимательном рассмотрении, любой фон не лишен собственного интереса. В этом смысле любой рисунок можно назвать рекурсивным. Однако, вводя эти термины, я имел в виду нечто другое. Существует естественное,~интуитивное понятие узнаваемых форм. Являются ли и рисунок и фон узнаваемыми формами? Если да, то такой рисунок рекурсивен. Посмотрев на фон~большинства контурных рисунков, вы обнаружите, что в нем трудно признать какую-либо форму. Это доказывает, что:

Существуют узнаваемые формы, чье негативное пространство не является никакой узнаваемой формой. Или, выражаясь более технично:

Существуют курсивно рисуемые рисунки, которые не рекурсивны.

\emph{Рис. 17. Скотт Е. Ким Рисунок «РИСУНОК-РИСУНОК».}

На рис. 17 показано решение предложенной выше головоломки,~принадлежащее Скотту Киму; я называю это решение «рисунок РИСУНОК --- РИСУНОК». На какую бы часть --- белую или черную --- вы не посмотрели, вы увидите только «ФИГУРЕ» (= английское «РИСУНОК»), и никакого «ФОНА». Великолепный образчик рекурсивного рисунка! Черные области этого~хитроумного рисунка можно охарактеризовать двумя способами:

(1) как \emph{негативное пространство} белых областей;

(2) как \emph{видоизмененные копии} белых областей (полученные путем их окраски и сдвига каждой белой области).

(В данном случае обе характеристики эквивалентны; для большинства~черно-белых рисунков это не так.) В главе VIII, создавая Типографскую Теорию Чисел (ТТЧ), мы будем надеяться, что нам удастся охарактеризовать~множество всех ложных утверждений аналогичными способами:

(1) как \emph{негативное пространство} множества всех теорем ТТЧ;

(2) как \emph{модифицированные копии} множества всех теорем ТТЧ (полученные путем отрицания каждой теоремы ТТЧ).

Однако этой надежда окажется напрасной, так как:

(1) среди множества всех не-теорем существуют некоторые истинные~утверждения;

(2) вне множества всех отрицаний теорем, существуют некоторые ложные утверждения.

Отчего так получается, вы увидите в главе XIV; а пока можете поразмыслить над графическим изображением данной ситуации (Рис. 18).

\emph{Рис. 18. Эта диаграмма отношений между различными классами строчек ТТЧ весьма богата зрительным символизмом. Самый большой прямоугольник --- множество всех строчек ТТЧ. Следующий прямоугольник --- все правильно построенные строчки ТТЧ. Внутри него находится множество всех предложений ТТЧ. Именно на этом уровне начинают происходить интересные вещи. Множество теорем изображено в виде~дерева, чей ствол --- множество аксиом. Символ дерева был выбран из-за того, что оно растет «рекурсивно» новые ветви (теоремы) вырастают из старых. Пальцеобразные ветви проникают во все уголки области представляющей множество истинных~высказываний, однако они не могут занять эту область целиком. Граница между областями истинных и ложных высказываний представляет собой изломанную «береговую линию», которая, как бы близко вы ее не рассматривали, всегда имеет еще более мелкие уровни структуры и таким образом, не поддается описанию каким либо конечным методом (См. книгу Мандельбродта «Фракталы» (В. Mandelbrodt Fractals)). Отраженное дерево справа представляет отрицания теорем все они ложны, но вкупе они не в состоянии заполнить всю область ложных высказываний (Рисунок автора)}

Рисунок и фон в музыке

Аналогию с понятием рисунка и фона можно также найти и в музыке. Примером может служить различие между мелодией и аккомпанементом: мелодия всегда на первом плане, тогда как аккомпанемент в каком-то смысле второстепенен. Поэтому нам кажется удивительным, когда мы узнаем мелодии на «низшем» уровне музыкального произведения. Для пост-барочной музыки это редкое~явление --- обычно гармонии там не выходят на первый план. Но в барочной музыке --- и прежде всего, у Баха --- все уровни «работают» в качестве «рисунка». В этом смысле баховские композиции могут быть названы рекурсивными.

В музыке есть еще одно различие между рисунком и фоном --- ударные и безударные такты. Если вы начнете отмечать ритм счетом «раз-и, два-и, три-и\ldots», большинство нот мелодии придутся на числа, а не на «и». Иногда, однако, мелодия бывает нарочно смещена на «и», чем достигается интересный эффект. Это происходит, например, в нескольких фортепианных этюдах Шопена. Тот же прием можно найти у Баха, в особенности, в сонатах и партитурах для скрипки соло и в сюитах для виолончели соло. В этих композициях Баху удается~поместить несколько мелодий одновременно на разных уровнях. Иногда он~достигает этого эффекта, заставляя солирующий инструмент играть дублировки --- две ноты сразу. В других случаях, однако, он помещает один голос на ударные такты, а другой --- на безударные, так что слух различает две разные мелодии, вплетающиеся одну в другую и гармонически сочетающиеся. Нет нужды~говорить, что Бах не останавливается на этом уровне сложности\ldots{}

Рекурсивно счетные и рекурсивные множества

Перенесем понятие рисунка и фона обратно в область формальных систем. В нашем примере роль позитивного пространства играют теоремы типа S, а роль негативного пространства --- строчки, в которых количество тире выражается простым числом. Пока что единственный способ, который нам удалось найти~~~для выражения простых чисел типографским путем, это негативное пространство. Существует ли какой нибудь способ выразить простые числа в виде позитивного пространства, то есть в виде множества теорем некой системы?

Интуиция подсказывает разным людям разные ответы. Я отчетливо помню, как был озадачен и заинтригован, заметив разницу между негативной и позитивной характеристиками. Я был совершенно уверен в том, что не только простые числа, но и вообще любое негативно определяемое множество чисел может быть определено позитивно. Интуитивное обоснование моей уверенности заключалось в следующем вопросе: «\emph{Как это возможно, чтобы рисунок и фон не содержали совершенно одинаковой информации} ?» Мне казалось, что они представляют собой одну и ту же информацию, закодированную двумя разными способами. А что думаете по этому поводу вы, читатель?

Выяснилось, что я был прав насчет простых чисел, но ошибался в остальном. Тогда это меня поразило и продолжает поражать и по сей день. Оказывается, что:

\emph{существуют такие формальные системы, чье негативное пространство (множество не-теорем) не является позитивным пространством никакой другой формальной системы} .

Как выяснилось, этот результат сравним по глубине с Теоремой Гёделя --- так что неудивительно, что моя интуиция не могла принять его сразу. Подобно математикам начала двадцатого века, я считал мир формальных систем и натуральных чисел более предсказуемым, чем он оказался в действительности. Выраженное более техническим языком, это утверждение звучит так:

\emph{Существуют рекурсивно счетные множества, не являющиеся рекурсивными} .

Выражение «рекурсивно счетные» (часто сокращаемое как р.с.) --- математическое соответствие нашему художественному понятию «курсивно рисуемые», а \emph{рекурсивный} --- соответствие «рекурсивным». Множество строчек является р. с., когда все они могут быть выведены путем применения типографских правил --- например, множество теорем типа S или множество теорем системы \textbf{MIU} ; на самом деле, это определение приложимо ко множеству теорем любой формальной системы. Оно сравнимо с понятием о «рисунке» как о «множестве линий, которые могут быть произведены в соответствии с художественными правилами» (что бы это последнее не означало!). А «рекурсивное множество» подобно рисунку, чей фон, в свою очередь, также является рисунком --- в таком случае не только рисунок, но и его дополнение будут р. с. Из этого вытекает следующий результат:

\emph{Существуют такие формальные системы, у которых нет типографского алгоритма разрешения} .

Из чего это следует? Очень просто. Типографский алгоритм разрешения --- это метод, отличающий теоремы от не-теорем. Он позволяет нам выводить не-теоремы систематически, идя по списку \emph{всех} строчек и отбрасывая те, что не являются теоремами. Эту процедуру можно назвать типографским методом вывода множества не-теорем. Однако из предыдущего утверждения (которое мы пока принимаем на веру) следует, что для \emph{некоторых} формальных систем это невозможно.

Предположим, что мы нашли множество R («R» --- рисунок) натуральных чисел, которое мы можем вывести каким-либо формальным путем --- вроде множества составных чисел. Предположим, что его дополнением является множество F («F» --- фон) --- простые числа. Вместе взятые, R и F дают все натуральные числа. Мы знаем правило, позволяющее вывести все числа множества R, для чисел множества F такого правила не существует. Важно, что если числа R выводятся исключительно \emph{в возрастающем порядке} , то мы всегда можем охарактеризовать F. Трудность заключается в том, что многие р. с. множества производятся при помощи таких методов, которые выводят элементы в произвольном порядке, так что не известно, появится ли какое-либо число, до сих пор пропускаемое, если подождать еще чуть-чуть.

На вопрос «Все ли рисунки рекурсивны?» мы ответили отрицательно. Теперь мы видим что придется ответить отрицательно и на аналогичный вопрос математиков «Все ли множества рекурсивны?» Имея это в виду, давайте вернемся к этому расплывчатому понятию «формы». Обратимся снова к нашим множествам R --- рисунки и F --- фон. Легко согласиться с тем, что все числа во множестве R имеют какую-то общую «форму» --- но можно ли сказать то же самое о числах множества F? Странный вопрос. С самого начала имея дело с бесконечным множеством всех натуральных чисел, весьма сложно прямо и четко определить «дырки», остающиеся в списке после изъятия оттуда неких чисел. Таким образом, возможно что на самом деле у этих дырок нет никаких общих характеристик «формы». Неясно, стоит ли вообще использовать здесь такое соблазнительное словечко как «форма». Может быть лучше не определять этого понятия оставив ему некую интуитивную гибкость.

Вот вам еще одна головоломка, над которой вы можете поразмыслить в связи с изложенным выше Можете ли вы охарактеризовать следующее множество чисел (или его негативное пространство)?

1 3 7 12 18 26 35 45 56 69

Чем данная последовательность напоминает рисунок РИСУНОК-РИСУНОК?

Простые числа в качестве рисунка, а не фона

Как же насчет формальной системы для вывода простых чисел? Как это~сделать? Способ состоит в том чтобы, не останавливаясь на умножении,~обратиться прямо к неделимости, представив ее позитивно. Ниже дана схема аксиом и правило вывода теорем, представляющих понятие числа, не являющегося~делителем других чисел (\textbf{ND} = не делитель).

СХЕМА АКСИОМ:~~\emph{xy} \textbf{ND} \emph{x} , где~\emph{x} и \emph{у} --- строчки тире

Например,~\textbf{-\/-\/-\/-\/-ND-\/-} , где~\emph{x} заменен на «\textbf{-\/-»} и \emph{y} --- на~«\textbf{-\/-\/-»}

ПРАВИЛО: Если~\emph{x} \textbf{ND} \emph{y} является теорема, то~\emph{x} \textbf{ND} \emph{xу} также будет теоремой

Приложив это правило дважды, вы можете вывести теорему

\textbf{-\/-\/-\/-\/-ND-\/-\/-\/-\/-\/-\/-\/-\/-\/-\/-\/-}

которая интерпретируется как~«5 не делитель 12». Однако~\textbf{-\/-\/-ND-\/-\/-\/-\/-\/-} ~не является теоремой. В чем будет ошибка, если вы попытаетесь вывести эту строчку?

Чтобы определить, что данное число простое, у нас должны быть какие-то сведения о его свойствах неделимости. В частности, мы хотим знать, что это число не делится на 2, 3, 4, и т. д., до числа, меньшего его на единицу. Однако в формальных системах мы не можем позволить себе таких расплывчатых~формулировок как «и так далее». Здесь нужна исчерпывающая точность. Нам бы хотелось иметь возможность сказать на языке системы: «число Z \emph{свободно от делителей} до X» (\textbf{SOD} ~= свободно от делителей), имея в виду, что не одно число между 2 и X не является делителем Z. Это можно сделать, но здесь есть небольшой трюк. Если хотите, можете попытаться найти его.

Решение заключается в следующем:

ПРАВИЛО: Если~\textbf{-\/-ND} \emph{z} является теоремой, то \emph{z} \textbf{SOD-\/-} ~также будет~теоремой.

ПРАВИЛО: Если \emph{z} \textbf{SOD} \emph{x} и \emph{x\textbf{-}} \textbf{ND} \emph{z} являются теоремами, то \emph{z} \textbf{SOD} \emph{x} также будет теоремой.

Эти два правила, в совокупности, характеризуют понятие свободы от~делителей. Все что нам нужно, это указать, что простые числа --- это числа, \emph{свободные от делителей} , включая число на единицу меньшее их самих:

ПРАВИЛО: Если \emph{z\textbf{-}} \textbf{SOD} \emph{z} является теоремой, то \textbf{P} \emph{z\textbf{-}} также будет теоремой.

Не будем забывать, что 2 --- тоже простое число!

АКСИОМА: \textbf{P-\/-}

Вот и все, что нам необходимо. Принцип формального выражения «просто-численности» заключается в том, что существует метод проверки, не~требующий никакого отступления назад. Вы всегда двигаетесь вперед, проверяя~данное число на делимость --- сначала на 2, потом на 3, и так далее. Именно эта «монотонность» или однонаправленность --- отсутствие игры между~укорачивающими и удлиняющими правилами --- позволила нам уловить суть простых чисел. И именно этой потенциальной сложностью формальных систем,~могущих вместить сколько угодно взаимодействий между движением вперед и~назад, объясняются такие ограничивающие результаты как Теорема Гёделя и~Проблема Остановки Тюринга, как и тот факт, что не все рекурсивно счетные множества рекурсивны.


% % \subsubsection{Акростиконтрапунктус}
% \subsubsection{Акростиконтрапунктус}

\emph{Ахилл} : Хорошая у вас коллекция бумерангов, я такой нигде не видал!

\emph{Черепаха} : Обыкновенная, не преувеличивайте, пожалуйста. У любой Черепахи можно увидеть коллекцию ничуть не хуже.

\emph{Ахилл} : Феноменально! Вы, Черепахи, никогда не перестанете удивлять меня своей любовью к собиранию бумерангов.

\emph{Черепаха} : Шутить изволите? Да страсть к коллекционированию этого оружия у нас в крови. А сейчас, не угодно ли пройти в гостиную?

\emph{Ахилл} : Только после Вас, как обычно, госпожа Черепаха. (Следуя за Черепахой, Ахилл входит в гостиную и направляется в угол комнаты.) Я вижу, что у вас также неплохое собрание пластинок. Какую музыку вы предпочитаете?

\emph{Черепаха} : Актуальный вопрос. Видите ли, хотя я всегда была и остаюсь поклонницей Баха, должна признаться, что сейчас я увлекаюсь довольно необычной музыкой.

\emph{Ахилл} : Да? Что же это за музыка?

\emph{Черепаха} : Такая, о которой вы, скорее всего, никогда не слыхали. Я называю ее «разбивальная музыка».

\emph{Ахилл} : Едва ли не самая поразительная вещь, которую я слыхал от вас за последнее время. Что значит это необычное название?

\emph{Черепаха} : Рада удовлетворить ваше любопытство. Эта музыка --- для разбивания патефонов.

\emph{Ахилл} : О ужас!

\emph{Черепаха} : Вы полагаете?

\emph{Ахилл} : С ума сойти! Воображаю, как вы, пританцовывая с кувалдой в руке, сокрушаете один патефон за другим под звуки «Битвы при Виттории» Бетховена.

\emph{Черепаха} : Какое у вас образное мышление! Должна вас разочаровать, эта музыка не совсем то, что вы предполагаете. Однако ее истинная природа тоже любопытна. Могу дать вам кое-какие разъяснения\ldots{}

\emph{Ахилл} : Интересно\ldots{} Я весь внимание!

\emph{Черепаха} : Йоркширский мой приятель, старый Краб (вы с ним, часом, не знакомы?) пришел ко мне однажды с визитом\ldots{}

\emph{Ахилл} : Архибольшая умница, этот Краб. Я много о нем наслышан, но сам с ним никогда не встречался. Уверен, что знакомство со стариком принесло бы мне немалое удовольствие.

\emph{Черепаха} : Конечно, он личность незаурядная. Надо бы мне устроить вашу встречу; может быть, мы все как-нибудь увидимся в парке на прогулке. Думаю, что вы понравитесь друг другу!

\emph{Ахилл} : Расчудесная идея! Буду ждать этого с нетерпением\ldots{} Однако мы отклонились от темы вы, кажется, хотели объяснить мне, что такое разбивальная музыка?

\emph{Черепаха} : Ох, да, чуть не забыла. Так вот, пришел, значит, Краб ко мне в гости. Вы, наверное, слыхали, что у него всегда была страсть ко всяческим машинкам и приспособлениям; в то время он прямо-таки сходил с ума по патефонам. Он тогда только что приобрел свой первый патефон и, будучи наивным и доверчивым покупателем, поверил во всю ту белиберду, что нам обычно говорят усердные клерки в надежде сбыть свой товар. На этот раз клерк объявил, что понравившийся Крабу патефон может верно воспроизвести любой звук. Короче говоря, Краб уверился в том, что он купил Идеальный Патефон.

\emph{Ахилл} : Само собой разумеется, вы с этим не согласились.

\emph{Черепаха} : Точно, но он заупрямился и твердил, что его проигрыватель может воспроизвести какие угодно мелодии. Спорить не было толку, и каждый остался при своем мнении. Вскоре, однако, я опять пришла к Крабу в гости, на этот раз не с пустыми руками: я принесла с собой запись песни моего собственного сочинения. Песня называется «Меня нельзя воспроизвести на Патефоне №1.»

\emph{Ахилл} : Идея неординарная, ничего не скажешь! Это вы ему в подарок принесли?

\emph{Черепаха} : Конечно. Я предложила ему прослушать мое сочинение на его новом патефоне, и он с радостью согласился. Он поставил пластинку и включил патефон; но после первых же тактов бедный аппарат завибрировал, затрясся и вдруг --- БА-БАХ! --- разбился на мельчайшие кусочки, разлетевшиеся по всей комнате. Натурально, пластинка тоже разбилась вдребезги\ldots{}

\emph{Ахилл} : О, Боже!\ldots{} Какой удар для бедняги. Что-то было не в порядке с патефоном?

\emph{Черепаха} : Ничего. Абсолютно ничего. Просто он не мог воспроизвести мелодию моей песни --- эти звуки вызвали в нем такую сильную вибрацию, что он разбился.

\emph{Ахилл} : Так значит, это все же был не Идеальный Патефон. А ведь клерк ему такого наговорил\ldots{}

\emph{Черепаха} : Разве вы, Ахилл, верите всему тому, что говорят продавцы? Неужели вы так же наивны, как старый Краб?

\emph{Ахилл} : Абсолютно нет! Краб гораздо наивнее. Я-то знаю, что все торговцы --- известные пройдохи и надувалы. Поверьте, я не вчера родился!

\emph{Черепаха} : Представьте себе тогда, что тот клерк мог несколько преувеличить выдающиеся качества нового приобретения Краба. Скорее всего, его патефон вовсе не идеальный, а значит, не может воспроизвести любые звуки.

\emph{Ахилл} : Увы, кажется, так оно и есть\ldots{} Но как вы объясняете тот удивительный факт, что именно ваша запись оказалась той самой «невоспроизводимой» мелодией?

\emph{Черепаха} : Ничего удивительного; я сделала это специально. Перед тем, как снова отправиться к Крабу, я пошла в магазин, продавший ему патефон и спросила, где эта модель была сделана. Узнав адрес, я послала на фабрику запрос и со следующей почтой получила полное описание патефона Краба. Я работала, не покладая лап, проанализировала всю конструкцию, и мне удалось найти именно ту мелодию, которая, если ее сыграть вблизи от этого патефона, разобьет его вдребезги!

\emph{Ахилл} : Какое коварство! Зачем вы мне все это выложили\ldots{} Значит, вы сами записали эту музыку, да еще и принесли эту подлую штуку ему в подарок!

\emph{Черепаха} : Точно, вы угадали, мой проницательный друг! Однако это еще не конец. Краб не поверил, что его патефон оказался не Идеальным\ldots{}

\emph{Ахилл} : Упрямец!

\emph{Черепаха} : Совершенно верно! Он отправился в магазин, где приобрел себе еще один патефон, значительно дороже. На этот раз клерк пообещал вернуть ему деньги в удвоенном размере, если тот найдет хотя бы один звук, который новый патефон не сможет воспроизвести.

\emph{Ахилл} : Блестящая идея! Выходит, Краб ничем не рисковал\ldots{}

\emph{Черепаха} : Ловкий трюк, это верно. Так вот, Краб тут же похвастался мне своим приобретением; старик был вне себя от радости, и я пообещала ему придти в гости и посмотреть его очередное любимое детище.

\emph{Ахилл} : Естественно, перед тем как выполнить обещание, вы снова написали на фабрику и с учетом конструкции нового патефона Краба скомпоновали еще одну вредительскую мелодию, на этот раз под названием «Меня нельзя воспроизвести на патефоне №2»?

\emph{Черепаха} : Совершенно верно! Вижу, что вы вполне прониклись моей идеей\ldots{}

\emph{Ахилл} : Так что же случилось на этот раз?

\emph{Черепаха} : Я поставила мою запись и, как вы сами можете догадаться, история повторилась и патефон, и пластинка разлетелись вдребезги.

\emph{Ахилл} : Щелчок по Крабьему самолюбию изрядный! Тут уж ему, конечно, пришлось признать, что Идеальных Патефонов в природе не существует?

\emph{Черепаха} : Если бы. На самом деле он решил, что следующий патефон наверняка окажется «выигрышным билетом», а поскольку у него теперь была куча денег, он.

\emph{Ахилл (перебивает)} : Еще раз пошел в магазин\ldots{} Постойте-ка: ведь он бы мог вас запросто перехитрить, купив посредственный патефон, не воспроизводящий с достаточной точностью никакую, в том числе и разбивальную, музыку. Тогда вам пришлось бы спасовать\ldots{}

\emph{Черепаха} : Соблазнительная мысль. Однако она противоречит первоначальной идее иметь патефон, на котором можно воспроизвести даже его собственную разбивальную мелодию (что, естественно, невозможно).

\emph{Ахилл} : Конечно Теперь я понимаю, в чем здесь загвоздка. Любой достаточно качественный патефон (назовем его X), который сможет воспроизвести разбивальную музыку, от нее же и погибнет! Значит, патефон X не совершенный. Избежать подобной участи может только какой-нибудь плохонький патефон, который, однако, уже по определению не будет Идеальным! Любой патефон будет непременно «увечен» в том или ином смысле, а значит, все они дефектны!

\emph{Черепаха} : Разумеется, они не идеальны, но почему вы называете их «дефектными»? Никакой патефон не способен сделать все то, чего бы нам от него хотелось. Уж если говорить о дефектах, то изъян не в самих патефонах, а в наших представлениях о том, на что они способны. Краб, к примеру, был полон самых фантастических надежд

\emph{Ахилл} : Искать Идеальный патефон --- неблагодарное занятие. Купит ли Краб высококачественный или посредственный аппарат, он все равно проигрывает. Бедняга, мне его искренно жаль?\ldots{}

\emph{Черепаха} : В таком духе наш «поединок» с Крабом продолжался еще несколько раундов, пока Краб не раскусил принципа моих композиций. Тогда старик попытался меня перехитрить. Он послал фабрикантам описание патефона своего изобретения, который они и изготовили по его чертежам. Краб назвал свое детище «Патефон Омега» --- этот аппарат был намного сложнее чем все предыдущие.

\emph{Ахилл} : А, понимаю: у него вообще не было движущихся частей\ldots{} Может быть, он был сделан из ваты? Или\ldots{}

\emph{Черепаха} : Если вы будете пытаться угадать, то мы просидим здесь до завтра. Позвольте вам помочь: «Омега» имела встроенную телекамеру, сканирующую любую пластинку, перед тем как поставить ее на патефон. Эта камера была подключена к компьютеру, который, в свою очередь, устанавливал по форме дорожек, что за музыка записана на данной пластинке.

\emph{Ахилл} : Тривиальной эту конструкцию не назовешь, но пока мне все понятно. Однако как же Омега использовала полученную информацию?

\emph{Черепаха} : Интереснейшим образом: компьютер при помощи сложных вычислений устанавливал, какой эффект данная мелодия произведет на патефон. Если музыка оказывалась «опасной», Омега делала что-то поистине удивительное: она меняла структуру частей патефона, перестраиваясь на ходу! Только сделавшись неуязвимой для данной разбивальной мелодии, Омега включала свой патефон и проигрывала пластинку.

\emph{Ахилл} : Могу себе представить, как вы разочаровались: ведь это означало, что вашим проделкам пришел конец!

\emph{Черепаха} : Я удивлена, Ахилл, что вы так считаете. Видимо, вы не слишком хорошо знакомы с теоремой Гёделя о неполноте.

\emph{Ахилл} :~\ldots{} гммм\ldots{} Чьей теоремой?

\emph{Черепаха} : Имя ее создателя --- Гёдель. Суть теоремы заключается в том, что\ldots{}

\emph{Ахилл (перебивает)} : Гёдель? Не слыхал\ldots{} Послушайте, я уверен, что все это захватывающе интересно, но я, право, предпочел бы услыхать продолжение истории о разбивальной музыке. Мне думается, что я могу сам угадать ее конец\ldots{}

\emph{Черепаха} : Рада вашей проницательности. Вы, вероятно, думаете, что Краб победил?

\emph{Ахилл} : А как же! Признайтесь, что вам пришлось трусливо капитулировать. Не так ли?

\emph{Черепаха} : Ей-Богу, Ахилл, ну и засиделись мы с вами! Уж полночь близится\ldots{} Я с удовольствием пообщалась бы с вами еще, но у меня уже глаза слипаются.

\emph{Ахилл} : То-то я чувствую, что и меня в сон клонит\ldots{} Пойду я, пожалуй. (Направляется к двери, но внезапно поворачивает обратно.) Однако какой я забывчивый! Принес вам маленький презент и чуть не унес его обратно домой. (Протягивает Черепахе небольшой аккуратный сверток.)

\emph{Черепаха} : Стоило ли беспокоиться\ldots{} Благодарю! (Нетерпеливо распаковывает пакет.)

\emph{Ахилл} : Безделушка, право слово\ldots{}

\emph{Черепаха} : Ах\ldots{} (Срывает последнюю обертку и на свет появляется изящный стеклянный бокал.) Какая прелесть! Как вы узнали, что я прямо-таки с ума схожу по стеклянным бокалам?

\emph{Ахилл} : Разве? Не имел ни малейшего понятия, но я рад, что вам понравилось.

\emph{Черепаха} : Обворожительно! Послушайте, если вы умеете хранить секреты, я вам кое-что расскажу. Я пытаюсь найти Идеальный Бокал, так сказать, Генерал-бокал, Гроссмейстер-бокал, чья форма не имела бы ни малейшего изъяна. Представляете, если бы ваш подарок, назовем его Бокал Г, оказался бы искомым сокровищем! Сделайте милость, поделитесь: где вы отыскали это чудо?

\emph{Ахилл} : Частная коллекция, друг мой, частная коллекция --- а больше того, не обессудьте, я вам открыть не могу: секрет! Могу, ежели желаете, сообщить, кому принадлежал сей бокальчик.

\emph{Черепаха} : Не томите душу, говорите!

\emph{Ахилл} : Имейте терпенье, друг мой. Слыхали ли вы когда-нибудь о знаменитом коллекционере бокалов по имени И.С. Бах?

\emph{Черепаха} : Мало кто не слышал хотя бы однажды этого блестящего имени; но позвольте, я впервые слышу, что И. С. Бах занимался коллекционированием!

\emph{Ахилл} : Артистичные натуры часто бывают весьма разносторонни. Конечно, Бах был в первую очередь известен не как коллекционер, однако это занятие было его излюбленным хобби, хотя почти ни одна душа об этом не знает. Этот бокальчик --- его последнее приобретение.

\emph{Черепаха} : Клянусь небом, это удивительно! Последнее приобретение? Если это так, то ему цены нет! Но почему вы так уверены, что бокал Г действительно принадлежал Баху?

\emph{Ахилл} : Рассмотрите-ка его на свет: видите, внутри выгравирована надпись В-А-С-H?

\emph{Черепаха} : О, вижу, вижу. Убедительно, ничего не скажешь. Поразительная вещь\ldots{} (аккуратно ставит Бокал Г на полку). Кстати, знаете ли вы, что каждая буква в имени BACH --- это также название музыкальной ноты?

\emph{Ахилл} : Странно, как же это возможно? Я знаю, что во многих языках ноты обозначаются буквами, но там используются буквы только от А до G.

\emph{Черепаха} : Точно, в большинстве стран так оно и есть. Однако на родине Баха, в Германии, система немного другая. Например, нота «си» будет по-немецки «H», а «си бемоль» --- «В». Так, си-бемоль минорная месса Баха по-немецки называется «H-moll Mess». Понимаете?

\emph{Ахилл} : Изрядная путаница\ldots{} Подождите-ка. Нота «си» по-немецки «H», а «си бемоль» --- «В»\ldots{} Значит, само имя Баха --- мелодия?

\emph{Черепаха} : Хотя это и странно, но так оно и есть! На самом деле, Бах незаметно включил эту мелодию в одну из сложнейших композиций «Искусства фуги», финальный «Контрапункт». Это была последняя фуга, написанная Бахом.

\emph{Ахилл} : О, какое совпадение! Последний бокал, последняя фуга\ldots{} Продолжайте, друг мой, прошу вас\ldots{}

\emph{Черепаха} : Милейший Ахилл, наберитесь терпения, берите пример с нас, Черепах\ldots{} На чем, бишь, я остановилась? Когда я слушала «Контрапункт» впервые, я понятия не имела, какой будет финал. Внезапно, без малейшего предупреждения, музыка оборвалась. Затем --- мертвая тишина\ldots{} Я тут же поняла, что как раз в тот момент композитор умер. Этот миг, неописуемо печальный, так на меня подействовал, что я почувствовала себя совершенно разбитой. Так или иначе, В-А-С-H --- последняя тема этой фуги и она спрятана внутри произведения. Бах никому не сказал об этом, но, зная эту мелодию, ее можно найти без труда. Ах, Ахилл, сколько существует ловких способов спрятать тайные послания в музыке\ldots{}

\emph{Ахилл} : Хитроумные уловки для этого есть и в поэзии. Поэты часто прибегали к похожим трюкам; теперь, к сожалению, это вышло из моды. Скажем, Льюис Кэрролл частенько прятал слова и имена в первых буквах строк своих стихов. Поэма, скрывающая таким образом какое-нибудь послание, называется «акростих».

\emph{Черепаха} : Осведомлены ли вы, Ахилл, о том, что Бах тоже иногда писал акростихи?

\emph{Ахилл} : Фантастическая разносторонность!

\emph{Черепаха} : Широкие интересы у него были, ничего не скажешь! Но странного тут ничего нет: ведь контрапункт и акростих, с их скрытым смыслом, имеют очень много общего. Большинство акростихов прячут только одно послание, однако может существовать и акростих, так сказать, «с двойным дном», где первое послание, в свою очередь, является акростихом для второго. Можно представить себе и «контракростих», где секретное послание надо читать справа налево. Бог мой, да эта форма представляет почти неограниченные возможности! Более того, кто сказал, что акростихи --- область исключительно поэтов? Их может сочинять кто угодно, даже диалогики.

\emph{Ахилл} : Так, так\ldots{} Дело логики? Значит, мне это будет трудновато. Я с Госпожой Логикой не в ладах.

\emph{Черепаха} : Ахилл, вы опять все перепутали. Я сказала не «дело логики», а «диалогики», то есть сочинители диалогов. Гммм\ldots{} (Чешет лапой за ухом с задумчивым видом.)

\emph{Ахилл} : Друг мой, я по глазам вижу, что вы еще что-то замышляете\ldots{}

\emph{Черепаха} : Так, пустяки\ldots{} Я подумала: а что если какой-нибудь диалогик задумает написать один из своих диалогов в форме акростического контрапункта, в честь И. С. Баха? Маловероятно, конечно, чтобы такая странная идея пришла кому-нибудь в голову\ldots{} Все же, в таком случае, какое имя будет правильнее зашифровать: его собственное или Баховское? Впрочем, зачем нам волноваться о таких пустячных материях, пусть этот вопрос решает тот, кто задумает написать подобный диалог!\ldots{} Вернемся лучше к нашему «музыкальному» имени: знаете ли вы, что мелодия В-А-С-H, если ее сыграть снизу вверх и задом наперед, звучит точно также, как оригинал?

\emph{Ахилл} : Если ее сыграть снизу вверх? Не понимаю. Задом наперед, это ясно: H-С-А-В- но снизу вверх? Вы, вероятно, меня разыгрываете?

\emph{Черепаха} : Разрешите вам продемонстрировать; сейчас, только принесу скрипку\ldots{} (Идет в соседнюю комнату и возвращается со старинным инструментом.) Сейчас я вам, скептику, сыграю эту мелодию задом наперед, вверх тормашками, шиворот навыворот и в любом виде, в каком вашей душеньке будет угодно\ldots{} Ну что ж, начнем\ldots{} (Кладет на пюпитр ноты «Искусства фуги» и открывает их на последней странице.) Вот он, последний «Контрапунктус» и вот она, последняя тема.

(Черепаха начинает играть: В-А-С-~\ldots{} но когда она пытается взять финальное «H», внезапно, без малейшего предупреждения, резкий звук бьющегося стекла грубо прерывает ее игру.Черепаха и Ахилл оборачиваются как раз вовремя, чтобы успеть увидеть, как крохотные блестящие осколки осыпаются дождем с полки, где только что стоял Бокал Г. Затем --- мертвая тишина\ldots)

\emph{Рис. 19. Последняя страница «Искусства фуги» Баха. На подлиннике рукой сына композитора, Карла Филиппа Эммануэля, написано: «NB: Во время~исполнения этой фуги, в тот момент когда прозвучала мелодия В-А-С-H,~композитор скончался.» (На рисунке мелодия В-А-С-H взята в рамку) Пусть последняя страница Баховского «Контрапункта» послужит здесь как~эпитафия. (Ноты отпечатаны при помощи компьютерной программы СМУТ,~разработанной Дональдом Бирдом в Индианском университете США.)}

% % \subsubsection{ГЛАВА IV: Непротиворечивость, полнота  геометрия}
% \subsubsection{ГЛАВА IV: Непротиворечивость, полнота  геометрия}

Смысл явный и неявный

В главе II мы видели пример того, как смысл --- по крайней мере, в относительно простом контексте формальных систем --- рождается из изоморфизма между управляемыми правилами символами и вещами реального мира. В большинстве случаев, чем сложнее изоморфизм, тем больше «техники» --- как аппаратуры, так и программного обеспечения --- бывает необходимо, чтобы извлечь смысл из символов. Если изоморфизм очень прост (или хорошо нам знаком), то есть соблазн считать, что смысл, который мы замечаем, выражен явно. Мы видим смысл, не замечая изоморфизма. Один из самых ярких тому примеров --- человеческий язык. Люди часто приписывают значения самим словам, абсолютно не осознавая существования сложного «изоморфизма», эти значения порождающего. Эту ошибку совершить нетрудно; она состоит в том, что значение приписывается скорее \emph{объекту} (слову), чем \emph{связи} между данным объектом и реальностью. Вы можете сравнить это с наивным представлением о том, что шум является необходимым побочным эффектом столкновения двух предметов. Это, разумеется, неверно если два предмета столкнутся в вакууме, столкновение будет совершенно бесшумным. Здесь ошибка также заключается в том, что шум приписывается исключительно столкновению, и при этом игнорируется роль среды, переносящей звук от столкнувшихся предметов к уху.

Выше я использовал слово «изоморфизм» в кавычках, чтобы показать, что его здесь надо понимать с долей скептицизма. Символические процессы, лежащие в основе человеческого языка, настолько неизмеримо сложнее символических процессов в формальных системах, что, если мы хотим по-прежнему считать, что значение --- порождение изоморфизмов, то нам придется принять более гибкое определение изоморфизма, чем то, каким мы пользовались до сих пор. Мне кажется, что именно понимание природы изоморфизма, стоящего за значением, --- ключ к загадке человеческого сознания.

Явный смысл «Акростиконтрапунктуса»

Все это было подготовкой к обсуждению «Акростиконтрапунктуса» --- исследованию уровней его значения. В Диалоге есть как явный, так и неявный смысл. Самое явное значение --- та история, которая в нем рассказана. Это «явное» значение, строго говоря, крайне \emph{неявно} --- ведь мозгу приходится проделать невероятно сложную работу, чтобы, основываясь на черных значках на бумаге, понять происходящие в этой истории события. Несмотря на это, мы будем считать эти события явным значением Диалога, предполагая, что любой русскоязычный читатель, извлекая смысл из значков на бумаге, использует более или менее одинаковый «изоморфизм».

И все же, я хотел бы сделать явное значение истории еще более явным. Сначала немного о пластинках и патефонах. Обратите внимание на то, что дорожки на пластинке имеют два уровня значения. Первый уровень --- музыка. Но что же такое музыка --- последовательность колебаний в воздухе или последовательность эмоциональных реакций в человеческом мозгу? И то и другое, скажете вы. Но для того, чтобы эти эмоции возникли, сначала необходимы колебания. Колебания «извлекаются» из звуковых дорожек при помощи патефона --- относительно несложного устройства. На самом деле, вы можете сделать то же самое, ведя по дорожкам булавкой. После этого ухо превращает колебания в реакции слуховых нейронов мозга, которые, в свою очередь, трансформируют линейную последовательность вибраций в схему взаимодействующих эмоциональных откликов. Схема эта настолько сложна, что мне придется, вопреки желанию, воздержаться от ее обсуждения здесь. Так что давайте пока считать, что звуки в воздухе --- это «Первый Уровень» значения звуковых дорожек. Что же является «Вторым Уровнем» их значения? Это та вибрация, которая возникает в патефоне. Поэтому Второй Уровень значения зависит от цепи \emph{двух} изоморфизмов:

(1) изоморфизм между произвольным узором звуковых дорожек и колебаниями воздуха;

(2) изоморфизм между произвольными колебаниями воздуха и вибрацией патефона.

Эта цепь двух изоморфизмов изображена на рис. 20. Обратите внимание, что изоморфизм 1 порождает Первый Уровень значения. Второй Уровень значения --- менее явный, чем Первый, поскольку он порожден двумя изоморфизмами. Именно Второй Уровень значения является виновником того, что патефон разбивается. Интересно то, что рождение Первого Уровня значения немедленно влечет за собой рождение Второго Уровня значения --- один уровень невозможен без другого. Таким образом, именно неявное значение пластинки «атаковало» и разрушило патефон. Те же комментарии приложимы и к бокалу. Разница лишь в том, что здесь имеется еще один уровень изоморфизма --- соответствие между музыкальными нотами и буквами алфавита --- который мы будем называть «транскрипцией». За ней следует «перевод»: превращение музыкальных нот в звуки, после чего вибрация действует на бокал точно так же, как она действовала на серию все усложняющихся патефонов.

\emph{Рис. 20. Наглядное объяснение принципа, лежащего в основе Теоремы Геделя: два тесно связанных изоморфизма, дающие неожиданный эффект бумеранга. Первый --- от звуковых дорожек к звуку, получаемый при помощи патефона. Другой --- знакомый всем нам, но обычно оставляемый без внимания --- от звука к вибрации патефона. Обратите внимание на то, что второй изоморфизм существует независимо от первого: не только музыка, играемая на патефоне, но и вообще все звуки вблизи от него вызывают в нем вибрацию. Перефразировка Теоремы Геделя звучит так: для любого патефона существуют такие пластинки, которые нельзя на нем проигрывать, так как это косвенно способствует разрушению патефона. (Рисунок автора.)}

Неявные значения «Акростиконтрапунктуса»

Что же можно сказать о неявных значениях Диалога? (Множественное число не случайно --- в Диалоге их несколько.) О самом простом из них мы уже упомянули выше --- события в двух частях Диалога приблизительно изоморфны: патефон становится скрипкой, Черепаха --- Ахиллом, Краб --- Черепахой, звуковые дорожки --- выгравированным автографом, и т. д. После того, как вы заметили этот простой изоморфизм, вы можете продвинуться дальше. Обратите внимание, что в первой половине истории Черепаха --- виновник всех проказ, в то время как во второй половине она --- жертва. Ее же собственный метод обратился против нее! Не напоминает ли это вам об «атаке» на патефон пластинок, которые на нем проигрывают, или о надписи на бокале, явившейся «виновницей» его гибели, или о Черепахиной коллекции бумерангов? Безусловно. Это --- история о «плевках против ветра» на двух уровнях:

Первый уровень: «самоатакующие» бокалы и пластинки;

Второй уровень: «самоатакующий» дьявольский метод Черепахи,~использующий для «самоатаки» неявные значения.

Таким образом, мы можем установить изоморфизм между двумя уровнями истории, сравнив то, как пластинки и бокалы, подобно бумерангам, «замыкаются» сами на себя и в результате гибнут, с тем, как предательский метод Черепахи оборачивается против нее самой. Рассматриваемая таким образом, сама история --- пример «самоатак», которые в ней обсуждаются. Поэтому мы можем считать, что «Акростиконтрапунктус» косвенно говорит о себе самом, в том смысле, что его структура изоморфна событиям, которые в нем происходят. (Совершенно так же, как пластинки и бокал косвенно «говорят» о себе самих путем соседствующих изоморфизмов между игрой и вызыванием вибрации.) Конечно, можно прочитать Диалог, не замечая этого изоморфизма; тем не менее, он там присутствует.

Соответствие между «Акростиконтрапунктусом» и Теоремой Гёделя

Читатель, возможно, уже чувствует некоторое головокружение --- однако это еще только цветочки, а ягодки впереди. (На самом деле, некоторые уровни неявного значения даже не будут здесь затронуты --- если пожелаете, можете попробовать докопаться до них сами.) Я написал этот Диалог в основном для того, чтобы проиллюстрировать Теорему Гёделя, которая, как я уже говорил во введении, зависит от двух различных уровней значения высказываний теории чисел. Каждая из двух половин Диалога --- «изоморфная копия» Теоремы Гёделя. Поскольку это сложное соответствие --- центральная идея диалога, я попытался представить его на следующей диаграмме.

патефон \textless==\textgreater{} система аксиом теории чисел

патефон низкого качества~\textless==\textgreater{} «слабая» система аксиом

качественный патефон \textless==\textgreater{} «сильная» система аксиом

«совершенный» патефон~\textless==\textgreater{} полная система для теории чисел

«схема устройства» патефона~\textless==\textgreater{} аксиомы и правила формальной системы

пластинка~\textless==\textgreater{} строчка формальной системы

«проигрываемая» пластинка \textless==\textgreater{} теоремы формальной системы

«непроигрываемая» пластинка~\textless==\textgreater{} не-теоремы формальной системы

звук~\textless==\textgreater{} истинное высказывание теории чисел

воспроизводимый звук~\textless==\textgreater{} интерпретированная теорема системы

невоспроизводимый звук~\textless==\textgreater{} истинное высказывание, не являющееся теоремой

название песий «Меня нельзя воспроизвести на патефоне X»~\textless==\textgreater{} неявное значение строчки Геделя «Меня нельзя вывести в формальной системе X»

На этой диаграмме приводится основа изоморфизма между Теоремой Гёделя и «Акростиконтрапунктусом». Не волнуйтесь, если вы пока не вполне понимаете суть Теоремы Гёделя --- мы дойдем до нее только через несколько глав! Однако, прочитав этот Диалог, вы уже до некоторой степени прониклись духом этой Теоремы, даже если это и произошло незаметно для вас самих. Теперь я оставляю вас, читатель, с тем, чтобы вы попытались найти другие типы неявных значений в «Акростиконтрапунктусе». «Quaerendo invenietis»

«Искусство фуги»

Несколько слов об «Искусстве фуги»\ldots{} Написанное в последний год жизни Баха, оно состоит из восемнадцати фуг, основанных на одной и той же теме. По-видимому, создание «Музыкального приношения» вдохновило Баха еще на один цикл фуг, на этот раз с менее сложной исходной темой, где он решил показать все возможности этой формы. Простую тему «Искусства фуги» Бах обыгрывает множеством разных способов. Большинство фуг четырехголосные; их сложность и глубина выражения постепенно возрастают. Ближе к концу фуги достигают такой степени сложности, что кажется невероятным, что композитору удается поддерживать этот уровень. Однако это ему удается\ldots{} до последнего «Контрапункта».

«Искусство фуги» (а также жизнь композитора) были прерваны следующими обстоятельствами: Бах, у которого в течение многих лет были проблемы со зрением, наконец решился на операцию. Операция прошла неудачно, и Бах ослеп. Однако это не остановило его от работы над монументальным проектом, целью которого было описание всех возможностей искусства полифонической композиции; одной из важных черт проекта было использование многих тем. В композицию, которая была задумана как предпоследняя, Бах включил собственное имя, закодированное в третьей теме. Однако сразу после этого его здоровье так ухудшилось, что работу над любимым проектом пришлось прекратить. Несмотря на болезнь, Баху удалось продиктовать своему зятю финальную хоральную прелюдию, о которой Форкель, биограф композитора, написал следующее: «Когда я исполняю эту прелюдию, я всегда бываю глубоко тронут духом набожного смирения и веры; не могу сказать, чего мне не хватало бы больше: этого Хорала, или окончания последней фуги.»

Незадолго до смерти к Баху неожиданно вернулось зрение. Через несколько часов после этого с ним случился удар, и десять дней спустя он скончался, оставив загадку неполноты своего «Искусства фуги». Не связано ли это с тем, что Бах использовал там автореференцию?

Проблемы, связанные с Гёделевским результатом

Черепаха утверждает, что никакой достаточно мощный патефон не может быть совершенен --- то есть способен воспроизвести любые звуки, записанные на пластинке. Гёдель утверждает, что никакая достаточно мощная формальная система не может быть совершенна --- то есть способна представить любое истинное высказывание в виде теоремы. Так же, как и в случае с патефонами, это кажется дефектом только тогда, когда мы предъявляем слишком высокие требования к возможностям формальных систем. Однако для математиков начала столетия подобные завышенные требования были обычным делом; в то время во всемогуществе логических рассуждений никто не сомневался. Доказательство обратного было найдено в 1931 году. Тот факт, что в любой достаточно сложной формальной системе истинных утверждений больше, чем теорем, называется «неполнотой» этой системы. Удивительно то, что методы рассуждения, используемые Гёделем в его доказательстве, по-видимому, невозможно заключить в рамки формальных систем. С первого взгляда кажется, что Гёделю впервые удалось выразить необычайно глубокую и важную разницу между человеческой логикой и логикой машины. Это загадочное несоответствие между мощью живых и неживых систем отражено в несоответствии между понятием «истинности» и понятием «теоремности»; таков возможный романтический взгляд на эту ситуацию.

Модифицированная система~pr и противоречивость

Чтобы взглянуть на ситуацию более реалистично, нам необходимо глубже понять, почему и каким образом смысл выражается в формальных системах при помощи изоморфизма. (Мне кажется, что на самом деле это приводит к еще более романтическому взгляду на вещи.) Итак, сейчас мы приступаем к изучению некоторых новых для нас аспектов отношения между значением и формой. Первым делом, давайте создадим новую формальную систему, чуть-чуть изменив нашу старую знакомую, систему пр. Добавим к ней еще одну схему аксиом, сохранив при этом как старую схему, так и единственное правило вывода.

СХЕМА АКСИОМ II: Если~\emph{x} является строчкой тире, то~\emph{x} \textbf{p-r} \emph{x} будет аксиомой.

Ясно, что как~\textbf{-\/-p-r-\/-} , так и~\textbf{-\/-p-r-\/-\/-} ~будут теоремами новой системы. Однако они интерпретируются, соответственно, как~«2 плюс 1 равняется 2» и~«2 плюс 2 равняется 3». Легко увидеть, что такая система будет содержать массу ложных высказываний (если считать строчку высказыванием). Таким образом, наша новая система \emph{противоречива по отношению к окружающему миру} .

Как говорится, беда не приходит одна, в новой системе есть также и \emph{внутренние} проблемы. Она содержит высказывания, противоречащие друг другу, такие как~\textbf{-p-r-\/-} ~(старая аксиома) и~\textbf{-p-r-} (новая аксиома). Это означает, что наша система противоречива также и в другом смысле --- внутренне.

Так что же, лучше совсем отказаться от новой системы?

Ни в коем случае! Я нарочно описал эти «противоречия» в «лапшевешательном» стиле, изложив довольно туманные аргументы с уверенностью, призванной запутать читателя. Вполне возможно, что вы уже заметили ошибки в моих рассуждениях. Основная ошибка состоит в том, что я безоговорочно принял для новой системы ту же интерпретацию, что была верна для прежней системы. Вспомните, что мы тогда остановились на словах «плюс» и «равняется» только потому, что в такой интерпретации символы действовали изоморфно понятиям, с которыми мы их сравнивали. Когда мы изменяем правила системы, этот изоморфизм неизбежно страдает. С этим ничего не поделаешь. Таким образом, проблемы, на которые я жаловался в предыдущих абзацах, могут рассеяться как дым, \emph{как только мы найдем подходящую интерпретацию для некоторых символов новой системы} . Обратите внимание, что я сказал «некоторых» ---~совсем не обязательно в каждом случае менять интерпретацию всех символов. Некоторые из них могут сохранить прежнее значение, в то время как другие изменятся.

Снова непротиворечивость

Предположим, например, что мы интерпретируем по-новому лишь символ \textbf{r} , оставляя все остальные символы без изменения; в частности, символ~\textbf{r} будет означать «больше или равно». Теперь наши «противоречивые» теоремы~\textbf{-p-r-} и~\textbf{-p-r-\/-} ~звучат совершенно безобидно:~«1 плюс 1 больше или равно 1» и~«1 плюс 1 больше или равно 2». Мы одновременно избавились от противоречий (1) с окружающим миром и (2) внутри системы. К тому же, наша новая интерпретация \emph{значима} , в то время как прежняя не имела смысла. Я имею в виду, что она не имела смысла в новой системе --- в нашей первоначальной системе~\textbf{pr} она работала превосходно. Пытаться же использовать ее в новой системе так же глупо, как использовать интерпретацию «лошадь-яблоко-счастливая» в старой системе \textbf{pr} .

История эвклидовой геометрии

Несмотря на мои попытки застать вас врасплох и сбить с толку, этот урок по интерпретации символов при помощи слов, возможно, не показался вам слишком трудным, как только вы поняли, в чем тут дело. Действительно, это несложно. Однако это было одним из глубочайших прозрений математики девятнадцатого века! Все началось с Эвклида, который около 300 года до нашей эры собрал и систематизировал все, что было известно о геометрии в то время. Получившийся труд оказался таким солидным, что в течение более чем двух тысячелетий он практически считался библией геометрии --- одна из наиболее «долголетних» работ! Почему так получилось?

Основная причина в том, что Эвклид был основоположником строгости в математических рассуждениях. Его «Элементы» начинаются с простых понятий, определений и так далее; при этом постепенно накапливается множество результатов, организованных таким образом, что каждый данный результат строго основан на предыдущих. В результате, работа имела определенный план, архитектуру, делавшую ее мощной и прочной.

Однако эта архитектура весьма отличалась от, скажем, архитектуры небоскреба. (См. рис. 21.) В последнем случае, сам факт того, что небоскреб стоит и не падает, доказывает, что его структура «правильна». С другой стороны, в книге по геометрии, где предполагается, что каждое утверждение логически следует из предыдущих, одно ошибочное доказательство не вызовет видимого краха всей структуры. Перекладины и подпорки здесь не физические, а абстрактные. На самом деле, в Эвклидовых «Элементах» доказательства были построены из весьма капризного материала, полного скрытых ловушек. Этим материалом был человеческий язык. Как же в таком случае быть с архитектурной мощью «Элементов»? Верно ли, что они основаны на прочной структуре, или же в ней есть некие изъяны?

\emph{Рис. 21.~М. К. Эшер «Вавилонская башня» (гравюра на дереве, 1928)}

Каждое слово, которое мы произносим, имеет определенный смысл, диктующий нам, как это слово использовать. Чем обычнее слово, тем больше ассоциаций связано с ним и тем глубже укоренилось в нас его значение. Таким образом, когда кто-то пытается дать определение какому-либо употребительному слову, в надежде на то, что все мы с этим определением согласимся, обычно происходит следующее: вместо того, чтобы принять данное нам определение, мы, по большей части бессознательно, предпочитаем руководствоваться ассоциациями, хранящимися на «складе» нашего мозга. Я упоминаю об этом потому, что именно с такой проблемой столкнулся Эвклид, пытаясь дать определения таких обыденных слов как «точка», «прямая линия», «круг» и так далее. Как можно определить нечто, о чем у каждого уже есть вполне сформировавшаяся идея? Единственный способ заключается в том, чтобы указать, что ваше слово --- технический термин, который не должно путать с обычным, повседневным словом. Необходимо подчеркнуть, что связь с обычным значением слова здесь лишь кажущаяся. Эвклид этого не сделал, так как он был убежден в том, что точки и прямые в его «Элементах» были, на самом деле, точками и прямыми реального мира. Эвклид не предостерег читателей от ложных ассоциаций, тем самым пригласив их к свободной игре воображения\ldots{}

Это звучит почти анархично и, пожалуй, немного несправедливо по отношению к Эвклиду --- ведь он установил аксиомы или постулаты, которые должны были использоваться при доказательстве утверждений. На самом деле, он считал, что доказательства должны были быть основаны \emph{исключительно} на этих аксиомах и постулатах. К несчастью, именно здесь и случилась осечка! Неизбежным следствием использования ординарных слов явилось то, что некоторые вызванные этими словами ассоциации проникли и в Эвклидовы доказательства. Однако не думайте, что, читая «Элементы», вы найдете там зияющие «провалы» в рассуждениях. Напротив, ошибки там почти незаметны, поскольку Эвклид был слишком глубоким и проницательным мыслителем, чтобы допускать элементарные промахи. Тем не менее, в его рассуждениях все-таки есть «прорехи» --- небольшие дефекты в классическом труде. Однако вместо того, чтобы жаловаться, мы можем выучить кое-что новое о разнице между абсолютной и относительной строгостью математических рассуждений. На самом деле, именно отсутствие абсолютной строгости в работе Эвклида явилось причиной многих плодотворных открытий в математике более чем через две тысячи лет после того, как он написал свой труд.

Эвклид привел пять постулатов, легших в фундамент бесконечного небоскреба геометрии (Эвклидовы «Элементы» составили лишь первые несколько сотен этажей этого небоскреба). Четыре первые постулата кратки и элегантны:

(1) Любые две точки могут быть соединены отрезком прямой;

(2) Любой отрезок прямой может быть продолжен бесконечно и превращен в прямую линию;

(3) На основе любого отрезка прямой можно нарисовать круг, принимая этот отрезок за радиус и один из его концов --- за центр круга;

(4) Все прямые углы конгруэнтны.

Пятый постулат далеко не так грациозен:

(5) Если две прямые пересекают третью так, что сумма внутренних углов с одной стороны меньше двух прямых углов, то это прямые рано или неизбежно пересекутся на этой стороне.

Хотя Эвклид нигде не сказал об этом прямо, он считал свой пятый постулат в каком-то смысле хуже других, поскольку он нигде не использовал его в доказательстве первых двадцати восьми утверждений. Таким образом, мы можем сказать, что эти утверждения составляют так называемую «геометрию четырех постулатов» --- ту часть геометрии, которая может быть выведена на основе первых четырех постулатов «Элементов», без помощи пятого. (Ее также часто называют \emph{абсолютной геометрией} .) Безусловно, Эвклид предпочел бы найти доказательство этого «гадкого утенка», но за неимением такового, утенка пришлось принять на веру\ldots{}

Ученики Эвклида также были не в восторге от пятого постулата. В течение многих лет несказанное количество математиков посвящало несказанное число лет своей жизни попыткам доказать, что сам пятый постулат --- всего лишь часть геометрии четырех постулатов. К 1763 году были опубликованы по крайней мере двадцать восемь доказательств --- и все ошибочные! (Они были раскритикованы в диссертации некоего Г. С. Клюгеля.) Во всех этих ошибочных доказательствах присутствовала путаница между повседневной интуицией и строго формальными свойствами. Пожалуй, можно сказать, что на сегодняшний день эти «доказательства» не представляют интереса ни для математиков, ни для историков; однако имеются и некоторые исключения.

Многоликий Неэвклид

Во времена Баха жил некий Джироламо Саккери 1667-1733), питавший надежду освободить труд Эвклида от всех его недостатков. Основываясь на своих работах в области логики, он решил подойти к доказательству пятого постулата по-новому: предположим, что мы принимаем за истинное утверждение, \emph{обратное данному постулату} . Теперь попробуем работать \emph{с этим утверждением} в качестве пятого постулата. Через некоторое время мы наверняка придем к противоречию. Поскольку никакая математическая система не может содержать противоречия, тем самым мы докажем несостоятельность нашего пятого постулата --- а следовательно, состоятельность пятого постулата Эвклида. Необязательно вдаваться в подробности истории; достаточно сказать, что Саккери с большой изобретательностью начал работать над «Саккерианской геометрией», выводя одно утверждение за другим, пока ему не надоело. В один прекрасный день он решил, что очередное выведенное им утверждение «противно самому понятию прямой линии». Это, как ему показалось, было именно тем, чего он так долго искал --- желанным противоречием! Сразу после этого, незадолго до смерти, Саккери опубликовал свой труд под названием «Эвклид, освобожденный от недостатков».

Этим он лишил себя большей доли посмертной славы, так как не подозревал, что открыл то, что стало позже известно под именем «гиперболической геометрии». Через пятьдесят лет после Саккери, Ж. Г. Ламберт повторил ту же попытку, на этот раз подойдя еще ближе к цели. Наконец, через сорок лет после Ламберта и через пятьдесят лет после Саккери, \emph{неэвклидова геометрия} была признана как новая, полноправная область геометрии. На доселе прямой дороге математики появилась развилка. В 1928 году неэвклидова геометрия одновременно, по одному из необъяснимых совпадений, была открыта венгерским математиком Яношем (Иоганном) Больяйем, которому тогда был двадцать один год, и тридцатилетним русским, Николаем Лобачевским. По иронии судьбы, в том же году великий французский математик Адриен-Мари Лежандр решил, что он нашел доказательство пятого постулата Эвклида. Его рассуждения весьма напоминали рассуждения Саккери.

Кстати, отец Яноша, Фаркаш (или Волфганг) Больяй, близкий друг великого Гаусса, также вложил много сил в попытку доказать Пятый постулат. В письме к своему сыну он пытался отговорить того от подобных занятий:

Не пытайся пробовать этот подход к параллельным линиям. Я прошел этот путь до самого конца. Я пережил эту бездонную ночь, погасившую всякий свет и радость в моей жизни. Молю тебя, оставь науку о параллельных прямых в покое. Я думал, что жертвовал собой во имя истины. Я был готов стать мучеником, который освободил бы геометрию от ее недостатков и, очищенную, возвратил бы ее человечеству. Я предпринял огромный, чудовищный труд; мои создания --- неизмеримо лучше, чем у моих предшественников. И все же я не смог добиться полного удовлетворения. Поистине,~si paullum a summo discessit, vergit ad imum~. Убедившись, что ни один смертный не может достичь дна этой темной бездны, я повернул обратно, безутешный, жалея себя и все человечество\ldots{} Я проплыл мимо всех рифов этого дьявольского мертвого моря, всегда возвращаясь со сломанной мачтой и разодранными в клочья парусами. Именно в это время у меня испортился характер и в жизни моей началась осень. Я легкомысленно поставил на карту мое счастье и саму мою жизнь --- aut Caesar aut nihil.\footnote{Herbert Meschkowski «Non Euclidean Geometry» стр. 31 2}

Однако позже, убежденный, что его сын действительно чего-то достиг, Фаркаш настоятельно советовал ему опубликовать свои результаты, правильно предвидя такую частую в науке проблему одновременности:

Когда для определенных вещей пришло время, они появляются в разных местах, подобно тому, как фиалки появляются на свет ранней весной.\footnote{Там же стр. 33}

Насколько верным это оказалось в случае с неэвклидовой геометрией! В Германии сам Гаусс и еще несколько человек одновременно набрели на неэвклидовы идеи. Среди них были адвокат Ф. К. Швайкарт, который в 1818 году послал Гауссу письмо с описанием новой «астральной» геометрии, племянник Швайкарта Ф. А. Тауринус, который занимался неэвклидовой тригонометрией и Ф. Л. Вахтер, студент Гаусса, который умер в 1817 году в возрасте двадцати пяти лет, успев получить несколько глубоких результатов в неэвклидовой геометрии.

Ключом к неэвклидовой геометрии являлось «принятие всерьез» постулатов, на которых основаны такие геометрии как геометрия Саккери или Ламберта. Постулаты Саккери кажутся «отвратительными самой природе понятия прямой линии» только в том случае, если вы не можете освободиться от предвзятого мнения о том, что называть «прямой линией». Однако если вы можете отказаться от подобных идей и считать, что «простая линия» --- это то, что удовлетворяет новым постулатам, то ваша точка зрения радикально изменится.

Неопределяемые понятия

Эти рассуждения, вероятно, уже начинают звучать знакомо. В частности, они возвращают нас к теме системы~\textbf{pr} и ее варианта, где символы приобретали пассивное значение, зависящее от их роли в теоремах. Особенно интересен был символ \textbf{r} , поскольку его «значение» изменилось, когда мы прибавили новую схему аксиом. Совершенно так же з\emph{начения понятий «точка», «линия» и т. д. могут определяться множеством теорем (или постулатов), в которых они встречаются} . Очень важно, что открыватели неэвклидовой геометрии это осознали. Они нашли различные неэвклидовы геометрии, по-разному отрицая пятый постулат Эвклида и изучая последствия этого. Строго говоря, они (как и Саккери) не отрицали пятого постулата прямо; вместо этого они отрицали эквивалентный, так называемый параллельный постулат:

Через точку, лежащую вне прямой, можно провести одну и только одну прямую, не пересекающуюся с первой прямой, сколько бы мы их не продолжали.

В таком случае мы говорим, что вторая линия параллельна первой. Предполагая, что таких линий вообще не существует, вы входите в область \emph{эллиптической} геометрии; утверждая же, что таких прямых существует по крайней мере две, вы оказываетесь в гиперболической геометрии. Говоря об этих вариантах, мы все еще используем термин «геометрия» поскольку в них присутствует основной элемент --- абсолютная геометрия или геометрия четырех постулатов. Именно это «ядро» позволяет нам считать, что эти варианты --- описания свойств некого геометрического пространства, хотя это пространство не так легко интуитивно представить, как обычное.

На самом деле, эллиптическую геометрию нетрудно представить зрительно. Все «точки», «линии» и т. д. должны быть частью поверхности обыкновенной сферы. Давайте условимся писать «ТОЧКА» когда имеется в виду технический термин, и «точка» --- когда речь идет о повседневном значении. Мы можем сказать что ТОЧКА состоит из пары диаметрально противоположных точек на поверхности сферы. ЛИНИЯ --- это большой круг на сфере (круг, центр которого, как и центр экватора, совпадает с центром самой сферы). В этой интерпретации утверждения эллиптической геометрии, хотя и содержат такие слова как «ТОЧКА» и «ЛИНИЯ», описывают происходящее на сфере, а не на плоскости. Обратите внимание, что две ЛИНИИ всегда пересекаются в диаметрально противоположных точках --- а значит, в одной ТОЧКЕ! И, точно так же как две ЛИНИИ определяют ТОЧКУ, две ТОЧКИ определяют ЛИНИЮ.

Считая, что значения таких слов как «ТОЧКА» и «ЛИНИЯ» полностью зависят от утверждений, в которых эти слова встречаются, мы делаем шаг к полной формализации геометрии. Эта полуформальная версия еще употребляет множество слов русского языка в их обыденном значении (\emph{«и», «если», «имеет», «соединяет»} и т. п.), однако такие слова как «ТОЧКА» и «ЛИНИЯ» своего обыденного значения здесь лишены --- поэтому мы называем их \emph{неопределяемые понятия} . Неопределяемые понятия, такие как~\textbf{p~} и \textbf{r~} ~системы \textbf{pr} , в каком-то смысле определены \emph{косвенно} , --- совокупностью всех утверждений, в которых они встречаются, --- скорее чем прямо, в некоем определении.

Можно было бы утверждать, что полное определение неопределяемых понятий находится только в постулатах, так как там уже содержатся все вытекающие из них утверждения. Подобная точка зрения означала бы, что постулаты являются косвенными определениями всех неопределяемых понятий, поскольку те получают определение через какие-либо другие понятия.

Возможность множественных интерпретаций

Полная формализация геометрии означала бы, что каждый термин превратился бы в неопределяемое понятие, то есть стал бы «бессмысленным» символом какой-либо формальной системы. Я заключил слово «бессмысленный» в кавычки, поскольку, как вы знаете, символы автоматически приобретают различные пассивные значения, зависящие от теорем, в которых эти символы встречаются. Однако обнаружат ли люди эти значения --- это уже другой вопрос, так как для этого необходимо найти такое множество понятий, которое может быть связано изоморфизмом с символами данной формальной системы. По идее, желая формализовать геометрию, мы обычно уже имеем в виду определенную интерпретацию для каждого символа, так что пассивные значения оказываются уже встроенными в систему. Именно это я и сделал с символами~\textbf{p} и \textbf{r} , когда придумывал систему \textbf{pr} .

Но ведь могут существовать и другие пассивные значения, которые, в принципе, возможно подметить --- только до сих пор еще никто этого не сделал! Например, первоначальная система \textbf{pr} допускала довольно неожиданную интерпретацию~\textbf{r} как «равняется» и~\textbf{p} как «отнятое от». Хотя это довольно тривиальный пример, он неплохо передает суть идеи о том, что символы могут иметь множество значимых интерпретаций; искать их --- дело наблюдателя!\ldots{}

Все, что мы до сих сказали, может быть сведено к понятию «непротиворечивости». Мы начали с введения формальной системы, которая, на первый взгляд, не только находилась в противоречии с внешним миром, но и имела внутренние противоречия. Однако через несколько минут нам пришлось взять эти «обвинения» обратно и признать свою ошибку; оказывается, дело было в том, что мы выбрали неудачную интерпретацию для символов системы. Изменив интерпретацию, мы вернули системе ее непротиворечивость! Становится ясно, что непротиворечивость --- не свойство формальных систем как таковых, но зависит от интерпретации, предложенной для данной системы. Совершенно так же не является свойством формальных систем как таковых и противоречивость.

Разные виды непротиворечивости

Все это время мы говорили о «непротиворечивости» и «противоречивости», не давая определений этим понятиям. При этом мы опирались на старый добрый здравый смысл. Давайте теперь точно определим, что имеется в виду под \emph{непротиворечивостью} формальной системы (вместе с ее интерпретацией). Это означает, что каждая теорема, будучи интерпретирована, становится истинным утверждением. С другой стороны, если среди интерпретированных теорем найдется хоть одно ложное утверждение, мы говорим о \emph{противоречивости} данной системы.

Это определение говорит нам о противоречивости по отношению к внешнему миру --- а как насчет \emph{внутренних} противоречий? По идее, система была бы внутренне противоречива, если бы она содержала по крайней мере две теоремы, чьи интерпретации были бы несовместимы друг с другом, и непротиворечива, если бы все теоремы были совместимы между собой. Рассмотрим, например, формальную систему, имеющую только следующие три теоремы: ЧвЗ, ЗвЭ и ЭвЧ. Если Ч интерпретируется как «Черепаха»,~З --- как «Зенон», Э --- как «Эгберт» и~\emph{x} \textbf{в} \emph{y} --- как~\emph{x} всегда выигрывает в шахматы~у «\emph{y»} , то мы имеем следующие интерпретированные теоремы:

Черепаха всегда выигрывает в шахматы у Зенона.

Зенон всегда выигрывает в шахматы у Эгберта.

Эгберт всегда выигрывает в шахматы у Черепахи.

Эти утверждения нельзя назвать несовместимыми, хотя они и описывают довольно странную компанию шахматистов. Таким образом, в этой интерпретации формальная система, в которой эти три строчки являются теоремами, внутренне непротиворечива, хота на самом деле ни одна из ее теорем не является истинной! Внутренняя непротиворечивость требует от теорем не того, чтобы все они были истинными, а лишь того, чтобы все они были \emph{совместимы} друг с другом.

А теперь давайте предположим, что \emph{x} \textbf{в} \emph{y} интерпретируется как «\emph{x} был изобретен \emph{y} ». Тогда у нас было бы:

Черепаха была изобретена Зеноном

Зенон был изобретен Эгбертом

Эгберт был изобретен Черепахой

В этом случае неважно, какие из отдельных высказываний истинны --- а может быть, вообще нельзя установить, какие из них истинны и какие ложны. Однако мы можем с уверенностью сказать, что \emph{все три высказывания} не могут быть истинными \emph{одновременно.} Таким образом, данная интерпретация делает систему внутренне противоречивой. Противоречие здесь зависит не от интерпретации заглавных букв, а от интерпретации \textbf{в} и от того, как заглавные буквы передвигаются по кругу вокруг \textbf{в.} Следовательно, можно говорить о внутренней противоречивости, не интерпретируя \emph{всех} символов системы. (В данном случае хватило интерпретации одного-единственного символа) Возможно, что интерпретировав достаточное количество символов, мы уже ясно увидим, что никакая дальнейшая интерпретация не сделает все теоремы истинными. Дело здесь, однако, не только в истине, а в возможности. Все три теоремы оказались бы ложными, если бы мы интерпретировали заглавные буквы как имена реальных персонажей, однако мы называем систему внутренне противоречивой по другой причине. Мы основываем наше суждение на интерпретации буквы \textbf{в} в сочетании с кругообразностью (Еще кое-что об этом «авторском треугольнике» вы найдете в главе XX).

Гипотетические миры и непротиворечивость

Мы привели два взгляда на непротиворечивость: первый утверждает, что система вместе с ее интерпретацией \emph{непротиворечива по отношению к внешнему миру} , когда любая из ее интерпретированных теорем оказывается истинной. Другой говорит нам, что система вместе с ее интерпретацией \emph{внутренне непротиворечива} , когда все ее теоремы, будучи интерпретированы, \emph{совместимы друг с другом} . Эти два типа непротиворечивости тесно связаны. Чтобы определить совместимы ли друг с другом несколько высказываний, мы пытаемся представить себе такой мир, в котором все они могут быть истинными одновременно. Таким образом, внутренняя непротиворечивость зависит от непротиворечивости с внешним миром --- только теперь «внешний мир» может быть \emph{любым воображаемым миром} , вместо того, в котором мы живем. Однако это весьма неопределенное и неудовлетворительное заключение. Что составляет такой «воображаемый мир»? В конце концов, возможно вообразить и такой мир, в котором три героя изобретают друг друга по очереди. Или нет? Возможно ли вообразить мир, в котором есть квадратные круги? Или мир, в котором действительны законы Ньютона, а не законы относительности? Возможно ли вообразить такой мир, в котором что-то было бы одновременно зеленым и не зеленым? Или мир, в котором животные не сделаны из клеток? Мир, в котором Бах сымпровизировал восьмиголосную фугу на тему короля Фридриха Великого? В котором комары умнее людей? В котором Черепахи умеют играть в футбол и говорить? Разумеется, Черепаха, говорящая о футболе, была бы аномалией.

Некоторые из этих миров кажется легче вообразить, чем другие, так как некоторые из них включают \emph{логические} противоречия --- например, зеленый и не зеленый --- в то время как другие кажутся, за неимением лучшего слова, возможными, сюда например, относятся Бах, импровизирующий восьмиголосную фугу, или животные, состоящие не из клеток. Или даже такие миры, в которых законы физики отличаются от наших\ldots{} Пожалуй, можно сказать, что имеются разные типы непротиворечивости. Например, самым широким был бы «логически непротиворечивый» класс, так как для вхождения в него не существует никаких ограничений, кроме логических. Система является \emph{логически непротиворечивой} , когда никакие из ее двух теорем, будучи интерпретированы как суждения, прямо не противоречат одна другой; \emph{математически непротиворечивой} , когда интерпретированные теоремы не нарушают законов математики и \emph{физически непротиворечивой} , когда интерпретированные теоремы совместимы с законами физики. За этим следует биологическая непротиворечивость и так далее. В биологически непротиворечивой системе может существовать теорема, интерпретация которой --- суждение «Шекспир написал оперу», но не теорема, интерпретируемая как «Существуют неклеточные животные». Подобные причудливые типы противоречивости никто не изучает, так как их весьма сложно различить. Какая именно противоречивость заключена в задаче о трех героях, изобретающих друг друга по кругу? Логическая? Физическая? Биологическая? Литературная?

Обычно граница между интересным и неинтересным проводится между физической и математической непротиворечивостью. (Разумеется, эту линию проводят сами математики и физики --- компания, которую вряд ли можно назвать беспристрастной!\ldots) Это значит, что при рассмотрении формальных систем «учитываются» два типа противоречивости --- математическая и логическая. В соответствии с этим критерием мы еще не нашли такой интерпретации, в которой тройка теорем ЧвЗ, ЗвЭ, ЭвЧ была бы противоречива. Для этого мы могли бы интерпретировать \textbf{в} как «больше чем». Как насчет Ч, 3, и Э? Они могут быть интерпретированы, например, как 0, 2 и 11, соответственно. Обратите внимание, что таким образом две теоремы оказываются истинными, и одна --- ложной. Если бы мы интерпретировали~З как 3, у нас получилось бы две ложных и одна истинная теорема. Однако в обоих случаях система была бы противоречива. На самом деле неважно, какое значение мы придаем Ч,~З и Э, если мы не выходим за пределы натуральных чисел. Здесь мы опять сталкиваемся со случаем, когда, для того чтобы обнаружить внутреннюю противоречивость, необходима лишь \emph{частичная} интерпретация символов системы.

Включение одной формальной системы в другую

Предыдущий пример, в котором были интерпретированы только некоторые из символов, чем-то напоминает занятия геометрией на натуральном языке, когда мы используем некоторые слова как неопределяемые понятия. В таком случае слова делятся на два класса: те, чье значение неизменно и четко определено, и те, чье значение меняется до тех пор, пока система не станет непротиворечивой.(Последние и являются неопределяемыми понятиями). Такой подход к геометрии требует, чтобы слова первого класса уже имели определения, приобретенные вне геометрии. Эти слова формируют скелет системы, ее глубинную структуру, которая может быть затем наполнена различным материалом (эвклидова или неэвклидова геометрия).

Формальные системы часто строятся именно по такому последовательному или иерархическому типу. Например, можно придумать Формальную Систему I, с правилами и аксиомами, дающими некие пассивные значения ее символам. Эта Формальная Система I включается в более широкую систему с большим количеством символов --- Формальную систему II. Поскольку правила и аксиомы Формальной Системы I являются частью Формальной Системы II, пассивные значения символов Формальной Системы I остаются в силе и формируют жесткий скелет, играющий важную роль в определении пассивных значений новых символов Формальной Системы II. Вторая система может, в свою очередь, являться скелетом для третьей системы, и так далее. Может также существовать система (например, абсолютная геометрия) которая \emph{частично} дает пассивные значения своих неопределяемых понятий и которая может быть дополнена правилами и аксиомами, \emph{далее} ограничивающими эти значения. Именно это и происходит в случае эвклидовой геометрии в сравнении с неэвклидовой.

Уровни стабильности в зрительном восприятии

Подобным иерархическим образом мы приобретаем новые знания, расширяем наш словарный запас или воспринимаем незнакомые предметы. Это особенно интересно, когда мы пытаемся понять картины Эшера, скажем, такие, как «Относительность» (рис. 22), где часто встречаются совершенно невозможные образы. Можно предположить, что в таком случае мы должны пытаться интерпретировать картину снова и снова, пока не найдем непротиворечивой интерпретации --- однако мы поступаем совершенно иначе. Мы сидим перед картиной, заинтригованные лестницами, ведущими во всех воображаемых направлениях, и людьми, идущими по одной и той же лестнице в противоречащих друг другу направлениях. Лестницы являются тем «островком уверенности», на котором мы основываем нашу интерпретацию всей картины. Увидев в них знакомый предмет, мы пытаемся затем установить, как они связаны друг с другом. На этом этапе мы сталкиваемся с проблемой. Однако если бы мы попытались отказаться от своих взглядов и поставить под сомнение сами «островки уверенности», то столкнулись бы с трудностями иного рода. Мы никак не можем «перерешить» то, что лестницы --- это лестницы. Не рыбы, кнуты или руки, а именно лестницы. (На самом деле, выход у нас все-таки есть: можно оставить все линии картины вообще без интерпретации, как «бессмысленные символы» формальной системы. Этот путь --- пример «способа U», или отношения дзен-буддизма к символизму.)

\emph{Рис. 22. М. К. Эшер. «Относительность» (литография, 1953).}

Таким образом, иерархическая природа нашего восприятия заставляет нас видеть либо сумасшедший мир, либо кучу бессмысленных линий. Так же можно проанализировать и многие другие картины Эшера, опирающиеся на какие-либо стандартные формы, соединенные нестандартным образом. Когда зритель видит парадокс на высшем уровне, уже поздно возвращаться и пытаться поменять исходные интерпретации объектов нижнего уровня. Разница между рисунками Эшера и неэвклидовой геометрией заключается в том, что в последней возможно найти значимые интерпретации для неопределяемых понятий таким образом, что система становится понятной, в то время как в первой конечный результат несовместим с нашей концепцией мира, как бы долго мы не рассматривали картину. Конечно, можно придумать такие гипотетические миры, в которых Эшеровские события могут произойти\ldots{} но эти миры подчинялись бы законам биологии, физики, математики и даже логики на одном уровне, одновременно нарушая их на другом уровне. Что за странные миры! (Примером этого может служить «Водопад» (рис. 5), где вода подчиняется нормальным законам гравитации, в то время как природа пространства идет вразрез с законами физики.)

Одинакова ли математика во всех возможных мирах?

До сих пор мы подчеркивали тот факт, что \emph{внутренняя} непротиворечивость формальной системы (взятой вместе с ее интерпретацией) требует наличия некоего \emph{возможного} мира, в котором все интерпретированные теоремы были бы истинны; единственным ограничением этого мира было бы то, что математика и логика работали бы в нем так же, как и в нашем мире. С другой стороны, \emph{внешняя} непротиворечивость --- непротиворечивость с внешним миром --- требует того, чтобы все теоремы были бы истинны в \emph{реальном} мире. В том случае, когда мы хотим создать непротиворечивую формальную систему, теоремы которой интерпретировались бы как математические суждения, разница между этими двумя типами непротиворечивости, по всей видимости, должна исчезнуть, поскольку мы только что сказали, что \emph{математика во всех воображаемых мирах такая же, как и в нашем мире.} Таким образом, во всех возможных мирах 1+1 должно равняться 2, в любом мире должно быть бесконечно много простых чисел, прямые углы должны быть конгруэнтны и, разумеется, через точку, лежащую вне прямой должна проходить только одна прямая, параллельная данной.

Минуточку! Последнее --- постулат параллельности, и утверждать, что он универсален, было бы ошибкой, как мы только что показали. Если бы постулат параллельности был верен во всех воображаемых мирах, то неэвклидова геометрия была бы невозможна! Это отбрасывает нас назад, в ту же ситуацию, в которой находились Саккери и Ламберт --- безусловно, не лучший выход! \emph{Что же, если не математика, является общим для всех воображаемых миров} ? Может быть, логика? Или и она тоже находится «под подозрением»? Могут ли существовать миры, в которых противоречия --- нормальное и обыденное явление, миры, где противоречия не являются противоречиями?

В каком-то смысле уже лишь потому, что мы изобрели подобное понятие, такие миры, действительно возможны --- однако в более глубоком смысле они также весьма невероятны (Что уже само по себе маленькое противоречие). Говоря серьезно, если мы хотим хоть как-то общаться, то, по видимости, нам придется установить некую общую базу, включающую логику. (Существуют системы верований, отрицающие подобную точку зрения за то, что она слишком логична. В частности, дзен-буддизм с одинаковой готовностью принимает как противоречия, так и непротиворечия Это может показаться непоследовательным, но непоследовательность --- органическая часть дзен-буддизма, ну что тут можно сказать?)

Является ли теория чисел одинаковой во всех возможных мирах?

Если мы допустим, что именно логика --- одна и та же во всех возможных мирах (заметьте, что мы еще не определили, что такое логика --- определение будет дано в последующих главах), будет ли этого достаточно? Возможно ли, что в каких-то мирах количество простых чисел не бесконечно? Не должны ли числа подчиняться одним и тем же законам во всех возможных мирах? Или же лучше вообще считать число неопределяемым понятием, как «ТОЧКА» или «ЛИНИЯ»? В этом случае, теория чисел раздвоилась бы, подобно геометрии, на стандартную и нестандартную. Тогда должно было бы существовать соответствие абсолютной геометрии, некая центральная теория, общая для всех теорий чисел, отличающая их, скажем, от теорий какао, бананов или резины. Большинство современных математиков считают, что такая центральная теория чисел существует ---~вкупе с логикой она является необходимой частью всех возможных миров. Эта сердцевина теории чисел, соответствующая абсолютной геометрии, называется \emph{арифметика Пеано} , ее определение будет дано в главе VIII. Также уже точно установлено, что теория чисел действительно разветвляется на стандартную и нестандартные версии. (Это прямое следствие Теоремы Гёделя.) В отличие от ситуации с геометрией, однако, количество «сортов» теории чисел бесконечно, что делает положение с ней значительно более сложным.

Для \emph{практических} целей все теории чисел одинаковы. Иными словами, если бы конструкция мостов зависела бы от теории чисел (и в каком-то смысле так оно и есть), было бы совершенно неважно, что существует множество ее вариантов --- в аспектах, касающихся реального мира, все теории чисел совпадают. Этого нельзя сказать о различных геометриях; например, сумма углов в треугольнике равняется 180 градусам только в эвклидовой геометрии, она больше в эллиптической геометрии и меньше --- в гиперболической. Говорят, что однажды Гаусс попытался измерить сумму углов в огромном треугольнике, образованном вершинами трех гор, чтобы раз и навсегда определить, какой именно тип геометрии управляет нашей вселенной. Через сто лет Эйнштейн открыл теорию (общую теорию относительности), утверждающую, что геометрия вселенной определяется количеством материи, в ней содержащейся --- таким образом, никакой тип геометрии не присущ пространству как таковому. Это значит, что на вопрос «какой тип геометрии является истинным?» природа дает двусмысленный ответ не только в математике, но и в физике. А как же насчет соответственного вопроса «какой тип теории чисел истинен?»? Мы вернемся к нему после детального разбора Теоремы Гёделя.

Полнота

Если непротиворечивость --- это минимальное условие, при котором символы приобретают пассивные значения, то ее дополнение, \emph{полнота} --- максимальное признание этих пассивных значений. Непротиворечивость означает, что «все, что производит система, истинно»; полнота же, наоборот, утверждает, что «все истинные утверждения производятся данной системой». Точнее, мы не имеем в виду все истинные утверждения в мире, а только находящиеся в области, которую мы пытаемся воспроизвести в данной системе. Таким образом, более точное определение полноты следующее: «Каждое истинное утверждение, которое может быть выражено в нотации данной системы, является теоремой.»

\emph{Непротиворечивость} : когда каждая теорема, будучи интерпретирована, оказывается истинной (в каком-либо из возможных миров).

\emph{Полнота} : когда все утверждения, которые истинны (в каком-либо из возможных миров) и выразимы в виде правильно сформированных строчек системы, являются теоремами.

Пример формальной системы, полной на своем скромном уровне --- наша система~\textbf{pr} в ее первоначальной интерпретации. Все правильные суммы двух положительных целых чисел представлены теоремами данной системы. Можно сказать то же самое по-другому: «Все правильные суммы двух положительных целых чисел \emph{доказуемы} в данной системе.» (Внимание: используя термин «доказуемые утверждения» вместо термина «теоремы», мы начинаем стирать границу между формальными системами и их интерпретациями. Это не страшно, если мы четко осознаем этот факт, а также то, что некоторые системы допускают множественные интерпретации.) Система~\textbf{pr} в первоначальной интерпретации \emph{полна} ; она также \emph{непротиворечива} , поскольку не содержит таких ложных утверждений, которые были бы --- используем наш новый термин --- доказуемы внутри системы.

Некоторые читатели могут возразить, что система вовсе не полна, так как она не включает сложения \emph{трех} положительных целых чисел (например, 2+3+4=9), хотя оно и может быть записано в нотации системы (\textbf{-\/-p-\/-\/-p-\/-\/-\/-r-\/-\/-\/-\/-\/-\/-\/-\/-} ). Однако эта строчка не является хорошо сформированной и поэтому должна считаться такой же бессмысленной как и~\textbf{prp-\/-\/-rpr} . Тройное сложение \textbf{просто не может быть выражено} в данной системе, поэтому полнота системы сохраняется.

Несмотря на полноту системы~\textbf{pr} в данной интерпретации, эта система, безусловна, далека от того, чтобы полностью выразить понятие истины в теории чисел. Она, например, не может сказать нам, сколько всего простых чисел. Теорема Гёделя о неполноте говорит, что любая «достаточно мощная» система уже в силу своей мощности является неполной, в том смысле, что имеются хорошо сформированные строчки, которые выражают истинные утверждения теории чисел, не являясь при этом теоремами. (Иными словами, в теории чисел имеются истинные утверждения, не доказуемые внутри самой системы.) Системы типа \textbf{pr} , полные но не очень мощные, напоминают патефоны низкого качества --- мы сразу видим, что они настолько несовершенны, что никак не могут сделать то, чего бы нам от них хотелось --- а именно, сказать нам все о теории чисел.

Как интерпретация может создать или разрушить полноту

Что означает выражение, употребленное мною выше, что «полнота --- это максимальное подтверждение пассивных значений»? Оно означает, что если система непротиворечива, но не полна, то существует несоответствие между символами системы и их интерпретациями. Система недостаточно мощна, чтобы оправдать данную интерпретацию. Иногда, если интерпретации немного «подправить», система может стать полной. Для иллюстрации этой идеи давайте взглянем на модифицированную систему~\textbf{pr} (включая схему аксиом II) и на выбранную нами интерпретацию.

Изменив систему \textbf{pr} , мы изменили также и интерпретацию символа~\textbf{r} с «больше» на «больше или равняется». Мы нашли, что измененная система~\textbf{pr} в такой интерпретации непротиворечива; однако в новой интерпретации есть что-то сомнительное. Проблема весьма проста: теперь имеется множество истинных утверждений, не являющихся теоремами. Например,~«2 + 3 больше или равняется 1» выражено не-теоремой~\textbf{-\/-p-\/-\/-r-} . Просто эта интерпретация слишком небрежна! Она не отражает того, что делают теоремы системы. В такой неряшливой интерпретации система~\textbf{pr} неполна. Мы могли бы поправить дело одним из двух способов: (1) \emph{прибавив к системе новые правила} и, таким образом, сделав ее более мощной и (2) \emph{заменив интерпретацию на более аккуратную} . В данном случае, заменить интерпретацию кажется более разумной альтернативой. Вместо того, чтобы интерпретировать~\textbf{r} как «больше или равняется», мы должны сказать «равняется или больше на 1». После такой модификации система \textbf{pr} становится как непротиворечивой, так и полной. И эта полнота подтверждает правильность нашей интерпретации.

Неполнота формализованной теории чисел

В теории чисел мы снова встретимся с неполнотой; но в этом случае, чтобы исправить ситуацию, нам придется пойти в другом направлении --- сделать систему более мощной путем прибавления новых правил. Ирония здесь заключается в том, что каждый раз, когда мы прибавляем новое правило, мы думаем, что уж \emph{теперь-то} система станет полной! Эта дилемма может быть проиллюстрирована с помощью следующей аллегории.

Представьте себе, что у вас есть патефон и пластинка, которой мы пока дадим пробное название «Канон на тему В-А-С-H». Однако, когда мы проигрываем запись на нашем патефоне, вибрации, производимые записью, создают сильные помехи, мешающие нам узнать мелодию. Следовательно, заключаем мы, \emph{что-то} должно быть не в порядке --- или пластинка, или наш патефон. Чтобы проверить качество \emph{пластинки} , мы должны прослушать ее на патефоне товарища; а чтобы проверить качество \emph{патефона} , нам придется проигрывать на нем пластинки товарища, и смотреть, соответствует ли музыка этикеткам на них. Если наш патефон выдержит экзамен, тогда мы заключим, что дефект был в пластинке; с другой стороны, если пластинка пройдет свое испытание, то мы решим, что дефект --- в нашем патефоне. Однако каково будет наше заключение, если тест выдержат \emph{оба} ? Вспомните цепь двух изоморфизмов (рис. 20) и подумайте над ответом!


% % \subsubsection{Маленький гармонический лабиринт}
% \subsubsection{Маленький гармонический лабиринт}

\emph{Черепаха и Ахилл проводят день в Кони Айленде, огромном парке аттракционов. Купив себе по палочке «сахарной ваты», они решают прокатиться на колесе обозрения.}

Черепаха: Это мой любимый аттракцион. Кажется, что едешь так далеко --- а на самом деле никуда не попадаешь!

Ахилл: Понятно, почему это вам так нравится. Вы уже пристегнулись?

Черепаха: Да, все ремни на месте. Поехали! Ур-ра!

Ахилл: Я вижу, вы сегодня предовольны.

Черепаха: И не без основания: моя тетушка-гадалка предсказала мне на сегодня необыкновенную удачу. Так что я вся трепещу в предвкушении.

Ахилл: Неужели вы верите в предсказания судьбы?

Черепаха: Вообще-то нет\ldots{} но говорят, что они действуют, даже когда в них не веришь.

Ахилл: Ну, в таком случае, вам действительно повезло.

Черепаха: Ах, какой вид! Пляж, толпа, океан, город\ldots{}

Ахилл: И правда, великолепно. Взгляните-ка на вертолет --- вон там. Кажется, он летит в нашем направлении. На самом деле, он уже почти над нами.

Черепаха: Странно, оттуда свисает какая-то веревка\ldots{} и она совсем близко к нам --- можно ухватиться\ldots{}

Ахилл: Смотрите-ка: на конце веревки огромный крюк и на нем --- записка.

\emph{(Он протягивает руку и срывает записку. Колесо начинает опускаться.)}

Черепаха: Ну как, что там написано? Можете разобрать?

Ахилл: Да\ldots{} Здесь написано: «Приветик, друзья. Будете снова наверху --- хватайтесь за крюк, и получите Сюрприз!»

Черепаха: Записка грубовата\ldots{} но кто знает, к чему это может привести. Может, это начинается обещанное везенье. Давайте попробуем!

Ахилл: Давайте!

\emph{(Когда колесо снова начинает подниматься, они расстегивают свои ремни и на самой высокой точке хватаются за гигантский крюк. Внезапно веревка взлетает вверх, унося их к зависшему над их головами вертолету. Большая сильная рука втаскивает их внутрь.)}

Голос: Добро пожаловать на борт, лопухи!

Ахилл: К-кто\ldots{} кто вы такой?

Голос: Позвольте представиться: Гексахлорофен Ж. Удача, Знаменитый Похититель Детишек и Пожиратель Черепах --- к вашим услугам!

Черепаха: Ой!

Ахилл (шепотом Черепахе): Вот так «Удача»! Не совсем то, на что мы надеялись\ldots{} (Удаче): Гм-м-м\ldots{} если я могу позволить себе смелость спросить куда вы нас везете?

Удача: Хо-хо! На мою небесную кухню с полным электрическим оборудованием, где я собираюсь приготовить вот этот лакомый кусочек (бросая плотоядный взгляд на Черепаху) --- райский супчик получится, пальчики оближешь! И не сомневайтесь --- я проделываю все это исключительно в усладу моему чревоугодию! Хо-хо-хо!

Ахилл: На это я могу сказать лишь то, что смех у вас довольно злодейский.

Удача (злодейски смеясь): Хо-хо-хо! За эти слова, мои дорогой друг, ты мне дорого заплатишь! Хо-хо!

Ахилл: Ах, господи! Интересно, что он имеет в виду?

Удача: Все очень просто: у меня для вас обоих уготовлена Ужасная Судьба! Погодите --- вы у меня попляшете! Хо-хо- хо! Хо-хо-хо!

Ахилл: Ой, мамочка!\ldots{}

Удача: Вот мы и приехали. Высаживайтесь, друзья, прямо в мою электрическую небесную кухню. (Они заходят внутрь.) Располагайтесь и чувствуйте себя как дома, пока я буду решать вашу судьбу. Вот моя спальня. Вот мой кабинет. Присаживайтесь и подождите меня --- я ненадолго, только ножи наточу. Можете пока попробовать мои вина. Мое последнее приобретение --- «Витаскин»; там что-то еще на этикетке понаписано, да только я языка не понимаю, так что я называю эту штуку просто: «Вытаскин». Вон та бутылочка, на лосьон смахивает\ldots{} Я его еще сам не пробовал. Ну, я пошел. Хо-хо-хо! Черепаший супчик! Черепаший супчик! Мое любимое блюдо! (Уходит.)

Ахилл: Вытаскин! Давайте напьемся с горя!

Черепаха: Ахилл! Вы же уже выпили две кружки пива в парке! Да и как вы можете думать об этом в такой момент, именно когда нам необходима ясная голова?

Ахилл: А мне до лампочки\ldots{} (Поет.) Шуме-ел камы-ы-ыш\ldots{} о, миль пардон, я не должен петь подобных песен в присутствии дамы, да еще в такую ужасную минуту.

Черепаха: Боюсь, что наша песенка так и так спета.

Ахилл: Это еще бабушка надвое сказала. Давайте пока от нечего делать посмотрим, что за книги у нашего хозяина на полках. Ну и коллекция, только для посвященных: «Садовые головы, с которыми я был знаком», «Шахматы и верчение зонтиков --- без труда», «Концерт для чечеточника и оркестра»\ldots{} Гм-м-м.

Черепаха: Что это за открытая книжица лежит там на столе, рядом с додекаэдром и альбомом для рисования?

Ахилл: Эта? Она называется: «Занимательные приключения Ахилла и Черепахи или Вокруг света от кочки до кочки.»

Черепаха: Довольно занимательное название.

Ахилл: Действительно --- и приключение, на котором книга открыта, выглядит занимательно. Оно называется «Джинн и Настойка».

Черепаха: Гм-м-м\ldots{} Интересно, почему. Может, попробуем почитать? Я буду читать за Черепаху, а вы --- за Ахилла.

Ахилл: Согласен. Терять нам все равно нечего\ldots{}

\emph{(Они начинают читать «Джинна и настойку».)}

\emph{(Ахилл пригласил Черепаху в гости, посмотреть коллекцию гравюр его любимого художника, Эшера.)}

\emph{Черепаха} : Чудесные гравюры, Ахилл.

\emph{Ахилл} : Я так и знал, что вам понравится. Какая ваша любимая гравюра?

\emph{Черепаха} : Одна из моих любимых --- «Выпуклое и вогнутое», где совмещаются два внутренне непротиворечивых мира. В результате получается составной, абсолютно невозможный мир. Противоречивые миры всегда забавно посетить, но жить там мне бы не хотелось.

\emph{Ахилл} : «Забавно посетить?» Что вы имеете в виду? Как можно посетить противоречивые миры, если их вообще НЕ СУЩЕСТВУЕТ?

\emph{Черепаха} : Прошу прощения --- но разве мы только что не согласились, что на этой картине Эшера изображен противоречивый мир?

\emph{Ахилл} : Да, но это же двухмерный мир, фикция, картинка. Этот мир посетить не удастся.

\emph{Черепаха} : У меня есть свои способы\ldots{}

\emph{Ахилл} : Как же вам удается затолкать себя в плоский мир картины?

\emph{Черепаха} : Для этого надо выпить стаканчик ПРОТАЛКИВАЮЩЕГО ЗЕЛЬЯ.

\emph{Ахилл} : Что это за штука такая --- проталкивающее зелье?

\emph{Черепаха} : Это жидкость, обычно содержащаяся в маленьких керамических пузырьках; когда вы, глядя на картину, выпиваете немного, жидкость эта проталкивает вас прямо в мир картины. Люди, которые ничего не знают о свойствах проталкивающего зелья, часто бывают поражены тем, в какие ситуации они попадают.

\emph{Ахилл} : А как насчет противоядия? Когда человек таким образом оказывается протолкнутым в картину, он что, так и остается там на всю жизнь?

\emph{Черепаха} : Иногда это не такое уж большое несчастье\ldots{} Но, разумеется, имеется другое зелье --- на самом деле, это скорее что-то вроде бальзама\ldots{} или эликсира\ldots{}

Черепаха: Она, кажется, имеет в виду «настойку».

\emph{Ахилл} : Настойка?

\emph{Черепаха} : Точно, именно это я и имела в виду! ВЫТАЛКИВАЮЩАЯ НАСТОЙКА, так она и называется. Если вы держите ее в правой руке, когда глотаете проталкивающее зелье, то она тоже оказывается протолкнутой в картину вместе с вами. Как только вы возжаждете быть вытолкнутым обратно в реальный мир, отхлебните немного выталкивающей настойки и --- але-оп! --- вы в реальном мире, точно на том же месте, где вы были, когда отведали проталкивающего зелья.

\emph{Ахилл} : Все это звучит захватывающе интересно. А что получится, если принять выталкивающую настойку, не протолкнувшись предварительно в картину?

\emph{Черепаха} : Я точно не знаю, Ахилл, но я бы не стала играть с этими странными жидкостями. Когда-то у меня был друг Фома, который мне не поверил и решил сделать именно это --- и с тех пор никто о нем ничего не слыхал.

\emph{Ахилл} : Жаль. А можно ли взять с собой бутылочку проталкивающего зелья?

\emph{Черепаха} : О, конечно. Надо зажать ее в левой руке и она тоже оказывается протолкнутой в картину вместе с вами.

Ахилл: А если внутри этой картины окажется еще одна, и вы снова примете глоточек проталкивающего зелья?

Черепаха: Случится именно то, чего вы ожидаете: вы очутитесь внутри картины-в-картине.

Ахилл: И, наверное, тогда придется выталкиваться дважды, чтобы вытащить себя из вписанных друг в друга картин и вновь вернуться в реальную жизнь.

Черепаха: Совершенно верно. На каждое проталкивание приходится одно выталкивание, так как первое вводит вас в картину, а второе это действие отменяет.

Ахилл: Знаете, все это звучит подозрительно. Вы уверены, что вы говорите это не только с целью испытать пределы моей доверчивости?

Черепаха: Клянусь! Поглядите: вот тут, в кармане, у меня два пузырька. (Засовывает руку в жилетный карман и вытаскивает два довольно больших пузырька без этикетки; слышно, как в них булькает жидкость, в одном красная, в другом --- голубая.) Ежели желаете, можем попробовать!

Ахилл: Э-э-э\ldots{} ну ладно\ldots{} может быть\ldots{}

Черепаха: Ну и славно! Я так и думала, что вам захочется попробовать. Хотите протолкнуться в мир Эшеровского «Выпуклого и вогнутого?»

Ахилл: Ну, как вам сказать\ldots{}

Черепаха: Значит, решено. Не забыть захватить с собой бутылочку настойки, чтобы мы смогли вытолкнуться обратно. Возьмете на себя эту ответственность, Ахилл?

Ахилл: Знаете, я немного нервничаю, и, если вы не возражаете, я предпочел бы, чтобы вы, с вашим опытом, управляли бы этой операцией.

Черепаха: Отлично. Итак\ldots{}

\emph{(С этими словами Черепаха наливает две маленькие порции проталкивающего зелья, протягивает Ахиллу его стакан и зажимает в правой лапе пузырек с настойкой Оба подносят стаканы к губам.)}

Черепаха: Пей до дна!

\emph{(Они делают по глотку.)}

\emph{Ахилл} : Что за странный привкус!

\emph{Черепаха} : К нему постепенно привыкаешь.

\emph{Ахилл} : А у настойки такой же странный вкус?

\emph{Черепаха} : Что вы, никакого сравнения! После первого же глотка вы чувствуйте такое удовлетворение, будто вы всю жизнь только о ней и мечтали.

\emph{Ахилл} : Прямо не терпится попробовать!

\emph{Черепаха} : Ну, Ахилл, где мы находимся?

\emph{Ахилл (оглядываясь)} : Мы в маленькой гондоле, скользим вниз по каналу! Я хочу сойти на берег. Синьор гондольер, остановите здесь, пожалуйста!

\emph{Рис. 23. М. К. Эшер «Выпуклое и вогнутое» (литография, 1955)}

~\emph{(Гондольер не обращает на эту просьбу ни малейшего внимания)}

~\emph{\textbf{Черепаха}} : Он не понимает по-русски. Придется нам выпрыгивать на берег, пока гондола не вошла в этот ужасный «Туннель любви», прямо перед нами.

~\emph{(Ахилл, слегка побледнев, выпрыгивает из гондолы с быстротой молнии и вытаскивает свою более медлительную спутницу.)}

\emph{\textbf{Ахилл}} : Что-то мне в этом названии определенно не по вкусу. Я очень рад что нам удалось вовремя вылезти. Послушайте, а откуда вы так хорошо знаете эти места? Вы здесь уже бывали раньше?

\emph{\textbf{Черепаха}} : Много раз, но я всегда попадала сюда из других картин Эшера. Знаете ли, позади рам они все соединены. Войдя в одну из картин, можно оттуда попасть в любую другую.

~\emph{\textbf{Ахилл}} : Удивительно! Если бы я не видел всего этого своими глазами, я бы ни за что в это не поверил. (Они выходят наружу сквозь небольшую арку.) Ой, что это там за смешная парочка ящериц?

\emph{\textbf{Черепаха}} : Смешные? Никакие они не смешные --- я вся дрожу при одной мысли о них! Это же злобные стражи волшебной медной лампы. Вон она, висит на потолке. Одно прикосновение языка, и любой смертный превращается в огурчик для закуски!

\emph{\textbf{Ахилл}} : Соленый или маринованный?

\emph{\textbf{Черепаха}} : Маринованный.

\emph{\textbf{Ахилл}} : Какая горькая судьба! Все-таки, если лампа действительно волшебная, я, пожалуй, рискну\ldots{}

\emph{\textbf{Черепаха}} : Это чистое безумие, мой друг. Я бы на вашем месте не стала этого делать.

\emph{\textbf{Ахилл}} : Всего один разочек\ldots{}

\emph{(Крадется к лампе, стараясь не разбудить спящую поблизости ящерицу. Внезапно нога его попадает в странную выемку в форме ракушки --- Ахилл скользит и взлетает в воздух. Судорожно пытаясь за что-то уцепиться, он нащупывает лампу и хватается за нее одной рукой. Лампа раскачивается. Ахилл беспомощно болтается в воздухе, а взбешенные ящерицы шипят и высовывают языки, пытаясь до него достать.)}

\emph{\textbf{Ахилл}} : На по-о-о-мощь!

\emph{(Его крик привлекает внимание стоящей поблизости женщины --- та сбегает с лестницы и будит спящего внизу мальчишку. Оценив ситуацию, он ободряюще улыбается Ахиллу и жестами показывает ему, что все будет в порядке. На странном гортанном наречии мальчишка кричит что-то двум трубачам, глядящим из окон. Они тут же начинают играть. Чудные мелодии сплетаются друг с другом, в необычном ритмическом узоре. Сонный паренек кивает в сторону ящериц, и Ахилл видит, что музыка действует усыпляюще и на них. Вскоре они вновь замирают. Тогда услужливый мальчишка зовет двух товарищей, взбирающихся по лестницам. Они составляют из лестниц что-то вроде моста. Повинуясь их настойчивым приглашающим жестам, Ахилл хватается за перекладины --- но прежде он осторожно разгибает верхнее звено цепи, на которой висит лампа, и снимает ее. Потом он взбирается на лестничный мост и мальчики вытаскивают его на безопасное место. Благодарный воин поочередно обнимает каждого из них.)}

\emph{\textbf{Ахилл}} : Г-жа Черепаха, как мне их отблагодарить?

\emph{\textbf{Черепаха}} : Я слыхала, что эти смельчаки неравнодушны к кофе --- а там внизу, в городе, есть местечко, где подают несравненный кофе-экспресс. Пригласите-ка их на чашечку!

\emph{\textbf{Ахилл}} : Это то что надо!

\emph{(С помощью комической серии жестов, улыбок и слов, Ахиллу удается растолоковать паренькам, что он их приглашает. Компания спускается по крутой лестнице в город. Они подходят к небольшому уютному кафе, усаживаются за один из столиков на улице, и заказывают пять чашечек экспресса. Пока друзья попивают кофе, Ахилл внезапно вспоминает про свою волшебную лампу.)}

\emph{\textbf{Ахилл}} : Чуть не забыл, г-жа Черепаха, лампа-то здесь! А что же в ней такого магического?

\emph{\textbf{Черепаха}} : Да как обычно --- джинн.

\emph{\textbf{Ахилл}} : Что? Вы имеете в виду, что стоит ее потереть, появится джинн и исполнит все ваши желания?

\emph{\textbf{Черепаха}} : Именно. А вы чего ожидали? Манны небесной?

\emph{\textbf{Ахилл}} : Да это же просто фантастика! Любое желание, а? Я всегда мечтал о чем-нибудь подобном\ldots{}

\emph{(Ахилл начинает тихонько тереть большую букву Л, выгравированную на медном боку лампы. Внезапно из лампы вырывается клуб дыма, в котором пятеро друзей различают очертания огромной призрачной фигуры, похожей на башню.)}

\emph{\textbf{Ахилл}} : Джинн!

\emph{\textbf{Черепаха}} : Дух!

\emph{\textbf{Фигура}} : Можно звать просто Гением\ldots{} Приветствую вас, о высокочтимые друзья, и благодарю за спасение моей Лампы от злобной Ящеричной Парочки. (С этими словами Гений подбирает Лампу и сует ее в карман, спрятанный в складках его длинного призрачного одеяния, струящегося из Лампы.) В благодарность за ваш героический поступок, я хотел бы предложить вам, от лица моей Лампы, осуществить три ваших желания.

\emph{\textbf{Ахилл}} : Потрясающе! Как вы думаете, г-жа Ч.?

\emph{\textbf{Черепаха}} : Безусловно. Что ж, друг мой, говорите ваше первое желание.

\emph{\textbf{Ахилл}} : Ух ты!.. Чего же мне пожелать? А, знаю: это пришло мне в голову еще когда я в первый раз читал «Тысячу и одну ночь» --- эти немудреные сказочки, вставлены одна в другую наподобие матрешки. Я хочу иметь не три, а СТО желаний? Здорово, правда, г-жа Ч.? Никогда не~~понимал, почему эти балбесы в сказках не догадываются попросить то же самое?

\emph{\textbf{Черепаха}} : Может быть, сейчас вы поймете.

\emph{\textbf{Гений}} : Мне очень жаль, Ахилл, но я не исполняю мета-желаний.

\emph{\textbf{Ахилл}} : Мне бы хотелось знать, что такое мета-желание\ldots{}

\emph{\textbf{Гений}} : Но это уже мета-мета-желание, Ахилл, а их я тоже не могу исполнить.

\emph{\textbf{Ахилл}} : Что-о? Ничего не понимаю\ldots{}

\emph{\textbf{Черепаха}} : Почему бы вам не выразить вашу просьбу как-нибудь по-другому?

\emph{\textbf{Ахилл}} : Что вы имеете в виду? Почему по-другому?

\emph{\textbf{Черепаха}} : Дело в том, что вы начинаете со слов «Мне бы хотелось\ldots» Но, поскольку вы хотите получить информацию, почему бы вам просто не задать вопрос?

\emph{\textbf{Ахилл}} : Ну хорошо, хотя я не совсем понимаю\ldots{} Скажите, пожалуйста, мистер Гений, что такое мета-желание?

\emph{\textbf{Гений}} : Это всего-навсего желание о желаниях. У меня нет права исполнять мета-желания. В моей власти только самые обыкновенные желания; ящик пива, скатерть-самобранка, готовая на все красотка, миллион долларов\ldots{} Понимаете, что-нибудь простенькое. Но мета-желание --- не могу. БОГ не велит.

\emph{\textbf{Ахилл}} : БОГ? Кто такой БОГ? И почему он не велит вам исполнять мета-желания? Это кажется совсем легко по сравнению с желаниями, о которых вы только что упомянули.

\emph{\textbf{Гений}} : Как вам сказать\ldots{} На самом деле, это довольно сложно. Почему бы вам просто не загадать три желания? Или, для начала, хотя бы одно? Я, знаете ли, не могу сидеть тут у вас до скончания веков.

\emph{\textbf{Ахилл}} : Ах, какое разочарование\ldots{} А я-то так надеялся получить мои сто желаний.

\emph{\textbf{Гений}} : Боже мой, как неприятно разочаровывать людей. К тому же, мета-желания --- мой любимый вид желаний. Пожалуй, я могу постараться вам помочь. Это отнимет только одну минуточку\ldots{}

\emph{(Гений вынимает из легких складок своей, одежды почти такую же Лампу, какую он недавно положил в карман . На этот раз она не медная, а серебряная. На месте буквы «Л» на ней, помельче, выгравировано «МЛ.»)}

\emph{\textbf{Ахилл}} : А это что такое?

\emph{\textbf{Гений}} : Это моя Мета-Лампа.

\emph{(Он начинает тереть Мета-Лампу, из которой вырывается огромный клуб дыма. В дымных водоворотах вырисовывается гигантская призрачная~~фигура, нависшая над ними подобно башне. На этот раз джинн оказывается женщиной.)}

\textbf{Мета-Гений} : Я --- Мета-Гений. Вы звали меня, о, высокочтимый Гений? Каково ваше желание?

\emph{\textbf{Гений}} : Я хочу попросить вас, о Гений, и также БОГа, даровать мне исполнение специального желания: отмены ограничений на типы желаний на время одного Нетипового Желания. Можете ли вы это сделать?

\textbf{Мета-Гений} : Придется, разумеется, направить вашу просьбу по соответствующим каналам\ldots{} Это отнимет только полминутки.

\emph{(Вдвое быстрее чем Гений, она вынимает из легких складок своего платья почти такую же Лампу, какую тот недавно положил в карман. На этот раз она не серебряная, а золотая На месте букв «МЛ» на ней, помельче, выгравировано «ММЛ.»)}

\textbf{Ахилл (его голос теперь звучит на октаву выше)} : Что это такое?

\textbf{Мета-Гений} : Это моя Мета-Мета-Лампа\ldots{}

\emph{(Она начинает тереть Мета-Мета-Лампу и из нее вырывается огромный клуб дыма, в котором они различают смутные очертания фигуры, нависшей над ними, подобно башне.)}

\textbf{Мета-Мета-Гений} : Я Мета-Мета-Гений. Вы звали меня, о Мета-Гений? Чего вы желаете?

\textbf{Мета-Гений} : Я хочу попросить вас, о Гений, и также БОГа, даровать мне исполнение специального желания, отмены ограничений на типы желаний, на время одного Нетипового Желания. Можете ли вы это сделать?

\textbf{Мета-Мета-Гений} : Придется, разумеется, направить вашу просьбу по соответствующим каналам\ldots{} Это отнимет только четверть минутки.

\emph{(И, вдвое быстрее чем Мета-Гений, он достает из складок своего одеяния предмет, напоминающий золотую Мета-Мета-Лампу, с той разницей, что он сделан из\ldots{}}

~\emph{(\ldots втягивается обратно в Мета-Мета-Мета-Лампу, которую Мета-Мета-Гений прячет обратно в складки своего одеяния, вдвое медленнее, чем это делал Мета-Мета-Мета-Гений.)}

Ваше желание исполнено, о Мета-Гений.

\textbf{Мета-Гений} : Благодарю вас, о Гений, и БОГ. (И Мета-Мета-Гений, подобно всем высшим Гениям, исчезает в Мета-Мета-Лампе, которую Мета-Гений затем прячет в складках своего платья, вдвое медленнее, чем Мета-Мета-Гений.) Ваше желание исполнено, Гений.

\emph{\textbf{Гений}} : Благодарю вас, о Гений и БОГ. (И Мета-Гений, подобно всем высшим Гениям, исчезает в Мета-Лампе, которую Гений затем прячет в складках его одеяния, вдвое медленнее, чем Мета-Гений.)Ваше желание исполнено, Ахилл.

\emph{(Ровно минута прошла с тех пор, как он сказал: «Это отнимет только одну минуту».)}

\emph{\textbf{Ахилл}} : Благодарю вас, О Гений и БОГ.

\emph{\textbf{Гений}} : Рад вам сказать, Ахилл, что вам даровано право ровно на одно Нетиповое Желание. Это может быть просто желание, или мета-желание, или мета-мета-желание --- столько «мета», сколько вашей душеньке угодно --- даже бесконечно много, ежели желаете.

\emph{\textbf{Ахилл}} : Я вам бесконечно благодарен, Гений. Но вы задели мое любопытство. Прежде чем я скажу свое желание, не могли бы вы мне ответить, кто такой --- или что такое --- БОГ?

\emph{\textbf{Гений}} : Нет ничего проще. «БОГ» --- это сокращение. Оно расшифровывается так: «БОГ, Одолевающий Гения.»~~Слово «Гений» обозначает Гениев, Мета-Гениев, Мета-Мета-Гениев и т. д. Это Нетиповое слово.

\emph{\textbf{Ахилл}} : Но\ldots{} Но как БОГ может быть словом в своем собственном сокращении? Это совершенная бессмыслица!

\emph{\textbf{Гений}} : Разве вы ничего не слыхали о рекурсивных сокращениях? Я думал, это общеизвестно. Видите ли, БОГ означает «БОГ, Одолевающий Гения», что, в свою очередь, может быть расширено «БОГ, Одолевающий Гения, Одолевающий Гения», что также может быть расширено до «БОГ, Одолевающий Гения, Одолевающий Гения, Одолевающий Гения», что, в свою очередь, может быть расширено\ldots{} и так расширять его можно сколько угодно.

\emph{\textbf{Ахилл}} : Но я так никогда не кончу!

\emph{\textbf{Гений}} : Разумеется, нет. БОГа невозможно познать до конца.

\emph{\textbf{Ахилл}} : Гм-м-м\ldots{} Изрядная путаница. Что вы имели в виду, когда попросили Мета-Гения, а также БОГа, даровать исполнение специального желания?

\emph{\textbf{Гений}} : Я обращался не только к Мета-Гению, но и ко всем Гениям выше нее. С помощью рекурсивного сокращения это делается просто. Услыхав мою просьбу, Мета-Гений передала ее своему БОГу. Так просьба достигла Мета-Мета-Гения, который, в свою очередь, направил ее Мета-Мета-Мета-Гению\ldots{} Поднимаясь таким образом по инстанциям, просьба в конце концов достигает БОГа.

\emph{\textbf{Ахилл}} : Понятно. Значит, БОГ сидит наверху лестницы Гениев?

\emph{\textbf{Гений}} : Да нет же! Наверху ничего нет, так как никакого «верха» не существует Именно поэтому БОГ --- рекурсивное сокращение. БОГ --- не какой-то последний Супер-Гений; это «башня» всех Гениев, находящихся над данным Гением.

\emph{\textbf{Черепаха}} : Мне кажется, что в таком случае каждый Гений имеет свое представление о том, что такое БОГ, так как для каждого Гения БОГ --- это множество высших Гениев, и нет двух таких Гениев, у которых это множество было бы одинаковым.

\emph{\textbf{Гений}} : Вы совершенно правы --- и поскольку я самый «низкий» Гений из всех, мое представление о БОГе самое возвышенное. Бедные высшие Гении --- они воображают, что находятся ближе к БОГу. Какое кощунство!

\emph{\textbf{Ахилл}} : Ух ты! Слишком все это сложно. Поистине, чтобы изобрести БОГа, нужны Гении\ldots{}

\emph{\textbf{Черепаха}} : Вы действительно верите всем этом сказкам о БОГе, Ахилл?

\emph{\textbf{Ахилл}} : Ну конечно, верю. А вы что же, атеистка г-жа Черепаха? Или агностик?

\emph{\textbf{Черепаха}} : Не думаю. Может быть, я --- мета-агностик.

\emph{\textbf{Ахилл}} : Что-о-о? Ничего не понимаю.

\emph{\textbf{Черепаха}} : Понимаете, если бы я была мета-агностиком, я~бы сомневалась в том, агностик ли я --- но я не уверена, что я в этом сомневаюсь. Значит, я, наверное, мета-мета-агностик\ldots{} Ну, ладно. Скажите мне, Гений, а случается ли какому-нибудь Гению ошибиться и перепутать путешествующее вверх или вниз по цепи послание?

\emph{\textbf{Гений}} : Такое иногда случается; это самая распространенная причина того, что Нетиповые Желания не разрешаются. Видите ли, вероятность того, что путаница произойдет на каком-то ОПРЕДЕЛЕННОМ этапе, ничтожно мала --- но когда у вас имеется цепь из бесконечного числа этапов, становится практически неизбежным, что ГДЕ-НИБУДЬ выйдет ошибка. На самом деле, как это ни странно, ошибок бывает бесконечное множество, хотя они и встречаются весьма редко.

\emph{\textbf{Ахилл}} : Тогда это просто чудо, когда какое-нибудь Нетиповое Желание вообще бывает даровано.

\emph{\textbf{Гений}} : Не совсем так. Большинство ошибок остается без последствий, а некоторые ошибки взаимоуничтожаются. Но иногда, хотя и довольно редко, причиной неисполнения Нетипового Желания может быть ошибка какого-то одного несчастного Гения. Когда такое происходит, виновник прогоняется сквозь бесконечный строй, и БОГ наказывает его шлепками. Это большое развлечение для шлепающих и к тому же совсем не больно для виновника. Вас бы позабавило это зрелище.

\emph{\textbf{Ахилл}} : Было бы интересно посмотреть! Но это бывает только в том случае, когда не исполняется Нетиповое Желание?

\emph{\textbf{Гений}} : Верно.

\emph{\textbf{Ахилл}} : Гм-м-м\ldots{} Кажется, я знаю, чего мне пожелать.

\emph{\textbf{Черепаха}} : Да? Чего же?

\emph{\textbf{Ахилл}} : Я бы хотел, чтобы мое желание не исполнилось!

\emph{(В этот момент происходит такое странное событие --- да можно ли это вообще назвать «событием»? --- что его невозможно описать; а значит, мы и пытаться не будем.)}

Ахилл: Интересно, что означает этот загадочный комментарий?

Черепаха: Он относится к Нетиповому Желанию, исполнения которого попросил Ахилл.

Ахилл: Но он еще ничего не пожелал!

Черепаха: Напротив; он сказал: «Я хотел бы, чтобы мое желание не исполнилось,» и Гений принял ЭТИ СЛОВА за желание.

\emph{(В этот момент в коридоре раздаются шаги; они медленно приближаются.)}

Ахилл: Ой! Какой кошмар!

\emph{(Шаги останавливаются и затем начинают удаляться.)}

Черепаха: Уф-ф!\ldots{}

Ахилл: История продолжается, или это уже конец? Переверните-ка страницу и давайте проверим.

\emph{(Черепаха переворачивает страницу «Джинна и настойки», и они обнаруживают, что история продолжается.)}

\emph{\textbf{Ахилл}} : Эй! Что стряслось? Где мой Гений? Моя лампа? Моя чашка кофе-экспресса? Что случилось с нашими юными друзьями из Выпуклого и Вогнутого Миров? И что здесь делают все эти ящерицы?

\emph{\textbf{Черепаха}} : Боюсь, что наш контекст был восстановлен неправильно.

\emph{\textbf{Ахилл}} : Интересно, что означает этот загадочный комментарий?

\emph{\textbf{Черепаха}} : Я имею в виду Нетиповое Желание, исполнения которого вы попросили.

\emph{\textbf{Ахилл}} : Но я еще ничего не пожелал!

\emph{\textbf{Черепаха}} : Напротив --- вы сказали: «Я хотел бы, чтобы мое желание не исполнилось», и Гений принял ЭТИ СЛОВА за желание.

\emph{\textbf{Ахилл}} : Ой! Какой кошмар!

\emph{\textbf{Черепаха}} : Это называется ПАРАДОКС. Чтобы исполнить это Нетиповое Желание, надо отказать в его исполнении. В то же время отказать в его исполнении значило бы исполнить его!

\emph{\textbf{Ахилл}} : Так что же произошло? Земля остановилась? Пространство закуклилось?

\emph{\textbf{Черепаха}} : Нет --- просто система отказала.

\emph{\textbf{Ахилл}} : Что это значит?

\emph{\textbf{Черепаха}} : Это значит, что мы оба мгновенно очутились в Лимбедламии.

\textbf{\emph{Ахилл}} : Где?

\emph{\textbf{Черепаха}} : Лимбедламия --- страна прошедшей икоты и перегоревших лампочек. Это что-то вроде зала ожидания, где дремлют программы в ожидании компьютеров. Нельзя сказать, как долго мы пробыли в Лимбедламии --- может быть, несколько минут, часов или дней, а может быть, и несколько лет.

\emph{\textbf{Ахилл}} : Я не знаю, при чем здесь программы или компьютеры. Я знаю только то, что не успел загадать желания! Верните моего Гения обратно!

\emph{\textbf{Черепаха}} : Мне очень жаль, Ахилл, но вы упустили свой шанс. Из-за вас отказала Система. Благодарите Бога, что мы вообще куда-то попали. Все могло быть гораздо хуже. Не имею ни малейшего понятия, где мы очутились\ldots{}

\emph{\textbf{Ахилл}} : Я знаю, это другая картина Эшера. Она называется «Рептилии».

\emph{\textbf{Черепаха}} : Ага! Система попыталась запомнить как можно больше нашего контекста перед тем, как отказать; ей~~удалось сохранить в памяти то, что мы находились в картине Эшера с ящерицами. Весьма похвально!

\emph{Рис. 24. М. К. Эшер. «Рептилии» (литография, 1943).}

\emph{\textbf{Ахилл}} : И взгляните не наш ли это флакончик с Выталкивающей настойкой там на столе, рядом с ящеричным хороводом?

\emph{\textbf{Черепаха}} : Безусловно, это он, Ахилл. Должна сказать, что нам действительно везет. Система обошлась с нами по-божески, вернув нам эту драгоценную жидкость!

\emph{\textbf{Ахилл}} : Это верно. Теперь мы можем вытолкнуться из эшеровского мира и вернуться ко мне домой.

\emph{\textbf{Черепаха}} : Интересно, что это за книги там, рядом с настойкой? (Она берет книгу поменьше, открытую в середине.) Эта книжица выглядит довольно занимательно.

\emph{\textbf{Ахилл}} : Правда? Как она называется?

\emph{\textbf{Черепаха}} : «Занимательные приключения Черепахи и Ахилла или Вокруг света от точки до точки.» Интересно было было бы почитать немного.

\emph{\textbf{Ахилл}} : Вы можете читать, если хотите, а я не собираюсь рисковать, какая-нибудь ящерица может запросто толкнуть флакон и разлить настойку. Я выпью свою порцию немедленно! (Он бросается к столу и протягивает руку к пузырьку с настойкой; при этом он случайно толкает его. Пузырек падает со стола и катится.) Ой! Г-жа Ч, смотрите! Я нечаянно столкнул настойку на пол и она покатилась\ldots~ к лестнице! Быстрее, а то свалится вниз!

\emph{(Но Черепаха погружена в свою книгу.)}

\emph{\textbf{Черепаха (бормочет)}} : А? Эта история выглядит захватывающе.

\emph{\textbf{Ахилл}} : Г-жа Ч, скорей, на помощь! Помогите поймать пузырек!

\emph{\textbf{Черепаха}} : Что за шум?

\emph{\textbf{Ахилл}} : Пузырек с настойкой, я столкнул его со стола, и сейчас он катится, и\ldots{} (В этот момент пузырек достигает первой ступеньки и падает вниз ) Ох! Что теперь делать? Г-жа Черепаха, вас это не волнует? Мы теряем настойку! Она только что свалилась с лестницы. Единственная наша надежда --- перейти на другой этаж!

\emph{\textbf{Черепаха}} : Перейти на другой рассказ? С превеликим удовольствием! Желаете ко мне присоединиться?

\emph{(Она начинает читать вслух, Ахилл застывает в нерешительности, не зная, что предпринять. Наконец он решает остаться и начинает читать за Черепаху.)}

\textbf{Ахилл} : Как здесь темно, г-жа Ч Я ничего не вижу. Ой! Я натолкнулся на стену. Осторожнее!

\textbf{Черепаха} : У меня есть пара тросточек Вот, держи~те одну. Вы можете прощупывать дорогу, чтобы ни с чем не сталкиваться.

\textbf{Ахилл} : Отличная идея. (Он берет трость.) Вам не кажется, что дорога слегка изгибается влево?

\textbf{Черепаха} : Да, пожалуй.

\textbf{Ахилл} : Интересно, где мы находимся. И увидим ли мы когда-нибудь дневной свет опять. Как жаль, что я вас послушался и проглотил эту штуковину «Выпей меня».

\textbf{Черепаха} : Уверяю вас, она совершенно безвредна. Я делала это много раз и никогда еще об этом не пожалела. Лучше расслабьтесь и постарайтесь получить удовольствие от того, что вы так чудесно уменьшились.

\textbf{Ахилл} : Уменьшился? Что вы со мной сделали, г-жа Черепаха?

\textbf{Черепаха} : Пожалуйста, не обвиняйте меня. Вы проделали все по вашему собственному желанию.

\textbf{Ахилл} : Так вы меня уменьшили? А вдруг лабиринт, в котором мы находимся, такой крохотный, что кто-нибудь может на него наступить?

\textbf{Черепаха} : Лабиринт? Лабиринт? Может ли это быть? Неужели мы попали в знаменитый лабиринт ужасного Мажотавра?

\textbf{Ахилл} : Ой, мамочка! Что это такое?

\textbf{Черепаха} : Говорят --- хотя я лично в это никогда не верила --- что злобный Мажотавр создал миниатюрный лабиринт и сидит в углублении в центре, поджидая невинных жертв, затерявшихся в чудовищно запутанных переходах. Когда они, окончательно заблудившись, забредают в центр, он начинает над ними смеяться, да так громко, что засмеивает их до смерти!

\textbf{Ахилл} : О боже, не может быть!

\textbf{Черепаха} : Это только миф. Смелее, Ахилл!

\emph{(И храбрая парочка осторожно двигается вперед.)}

\textbf{Ахилл} : Потрогайте эти стены. Они напоминают сморщенные жестяные листы --- только все морщины разного размера.

\emph{(Чтобы подчеркнуть свои слова, он прикладывает конец трости к стене и идет вперед. Трость подпрыгивает на неровностях стены --- длинный изогнутый коридор, в котором они находятся, наполняется странными звуками.)}

\textbf{Черепаха (встревоженно)} : Что это такое?

\textbf{Ахилл} : Это я веду тросточкой по стене.

\textbf{Черепаха} : Ох --- я было подумала, что это рев кровожадного Мажотавра.

\textbf{Ахилл} : Я думал, вы сказали, что это все выдумки.

\textbf{Черепаха} : Конечно. Бояться совершенно нечего.

\emph{(Ахилл снова прикладывает трость к стене и идет вперед. При этом слышна музыка; звуки исходят из того места, где трость прикасается к стене.)}

\textbf{Черепаха} : Ох, Ахилл, у меня дурное предчувствие --- мне кажется, что этот Лабиринт не такой уж и миф.

\textbf{Ахилл} : Погодите-ка, что это заставило вас так внезапно передумать?

\textbf{Черепаха} : Слышите эту музыку? (Чтобы лучше слышать, Ахилл опускает трость, и мелодия прекращается.) Эй! Поставьте трость обратно! Я хочу послушать конец этой пьесы!

\emph{Рис. 25. Критский лабиринт (Итальянская гравюра; школа Финигерры) Из книги У. Г. Маттьюса «Лабиринты: их история и развитие» (W.H. Mattews, Mazes and Labyrinths. Their History and Development.)}

(Ахилл, сбитый с толку, повинуется и музыка возобновляется.) Благодарю. Теперь я догадалась, где мы находимся.

\textbf{Ахилл} : Правда? Где же?

\textbf{Черепаха} : Мы идем по звуковой дорожке пластинки, лежащей в конверте. Ваша трость, скребущая по морщинам на стене, действует как иголка, бегущая по звуковой дорожке, позволяя нам слушать музыку.

\textbf{Ахилл} : Ох, нет, нет\ldots{}

\textbf{Черепаха} : Что такое? Разве вы не радуетесь? Когда еще вы находились в таком интимном контакте с музыкой?

\textbf{Ахилл} : Как же я смогу выигрывать соревнования по бегу против людей в натуральную величину, если я теперь меньше блохи, г-жа Черепаха?

\textbf{Черепаха} : Ах, так вот что вас волнует? Право, Ахилл, стоит ли из-за этого беспокоиться\ldots{}

\textbf{Ахилл} : Вы говорите так, что у меня создается впечатление, что вы вообще никогда не волнуетесь.

\textbf{Черепаха} : Не знаю, не знаю\ldots{} Я уверена только в~~одном: о чем я не жалею, так это о том, что я уменьшилась. В особенности тогда, когда нам грозит страшная опасность от чудовищного Мажотавра.

\textbf{Ахилл} : О ужас!.. Вы хотите сказать, что\ldots{}

\textbf{Черепаха} : Боюсь, что да, Ахилл. Музыка выдала его с головой.

\textbf{Ахилл} : Каким же это образом?

\textbf{Черепаха} : Очень просто. Когда я услышала мелодию В-А-С-H в верхнем голосе, меня осенило: на звуковых дорожках, по которым мы идем, записано не что иное, как «Маленький гармонический лабиринт», одна из наименее известных органных пьес Баха. Она названа так из-за модуляций, таких частых, что от них начинает кружиться голова.

\textbf{Ахилл} : Ч-что --- что это такое и с чем это едят?

\textbf{Черепаха} : Как вы знаете, большинство музыкальных произведений написано в какой-нибудь тональности --- например, «до мажор», как эта пьеса.

\textbf{Ахилл} : Я уже слышал это название раньше. Не правда ли, это значит, что «до» --- та нота, на которой произведение должно заканчиваться?

\textbf{Черепаха} : Да, «до» --- это что-то вроде ключа от дома, куда вы хотите попасть. Ключ бывает и в музыке.

\textbf{Ахилл} : Значит, сначала мы удаляемся от этого «дома», чтобы потом туда возвратиться?

\textbf{Черепаха} : Правильно. В музыкальных произведениях часто используются мелодии, уводящие в сторону от ключевой тональности. Мало-помалу нарастает напряжение, и слушатель начинает все сильнее скучать по «дому» ---~ему хочется вновь услышать ключевую тональность.

\textbf{Ахилл} : Таким образом, в конце пьесы я всегда буду чувствовать такое удовлетворение, как будто я всю жизнь желал услышать именно эти звуки?

\textbf{Черепаха} : Точно. Композитор использует свои знания о гармонической прогрессии, чтобы таким образом управлять нашими чувствами и пробудить в нас желание услышать ключевую тональность\ldots{}

\textbf{Ахилл} : Понятно, но, кажется, вы собирались рассказать мне о модуляциях\ldots{}

\textbf{Черепаха} : Ах, да. Один из важных приемов, которые композитор может использовать где-то в середине пьесы, называется модуляцией; это означает, что он устанавливает временную~~«цель», отличную от конечного разрешения в ключевую тональность.

\textbf{Ахилл} : А-а-а\ldots{} кажется, я понимаю. Вы имеете в виду, что определенная серия аккордов изменяет гармоническое напряжение таким образом, что я начинаю желать разрешения в новой тональности?

\textbf{Черепаха} : Именно так. Это усложняет ситуацию, поскольку, наряду с этим новым желанием, подсознательно вы все время ощущаете, что ваша конечная цель --- ключевая тональность, в данном случае, «до мажор». И когда временная цель бывает достигнута, то\ldots{}

\textbf{Ахилл (внезапно начиная возбужденно жестикулировать)} : О, послушайте только: какие восхитительные поднимающиеся вверх аккорды! Какой прекрасный конец у «Маленького гармонического лабиринта»!

\textbf{Черепаха} : Нет, Ахилл, это не конец, это просто ---

\textbf{Ахилл} : Да нет, разумеется, это конец! Вот это да! Какой могучий финал! Какое облегчение! Вот разрешение так разрешение! Гениально! (Поет): Ля-ля-ля\ldots{} (И точно, в этот момент музыка прекращается; стен больше нет, и Черепаха с Ахиллом оказываются в открытом пространстве.) Вот видите, музыка действительно кончилась. Ну, что я вам говорил?

\textbf{Черепаха} : Что-то здесь не так. Эта запись позорит музыкальный мир.

\textbf{Ахилл} : Почему это?

\textbf{Черепаха} : Я только что вам объяснил: Бах промодулировал здесь от «до» в «ля», так что временной целью было услышать мелодию в ключе «ля». Это значит, что вы чувствуете сразу два желания: с одной стороны, вы ожидаете разрешения в «ля», а с другой стороны, вы все время помните, что конечная цель --- триумфальное возвращение в «до мажор».

\textbf{Ахилл} : Почему надо все время о чем-то помнить, когда слушаешь музыку? Разве музыка --- только упражнение для ума?

\emph{Черепаха} : Нет, конечно. Некоторые произведения весьма интеллектуальны, но большинство довольно просты. Обычно наше ухо или мозг делают все «расчеты» за нас, в то время как чувства решают, что именно нам хочется услышать. Нам не приходится думать об этом. Но в этой пьесе Бах проделывает разные трюки, в надежде сбить слушателя с толку --- и надо сказать, что в вашем случае, Ахилл, он вполне преуспел!

\textbf{Ахилл} : Вы хотите сказать, что я среагировал на разрешение во «второстепенной» тональности?

\textbf{Черепаха} : Правильно.

\textbf{Ахилл} : Все же я и сейчас уверен, что это был конец!

\textbf{Черепаха} : Именно этого эффекта Бах и добивался. Вы угодили прямиком в его ловушку. Это место написано так, что оно звучит как финал; но если вы внимательно следите за развитием гармонической прогрессии, вы увидите, что оно не в том ключе. Видимо, не только вы, но и та несчастная студия звукозаписи решила, что это конец, и записала только часть пьесы!

\textbf{Ахилл} : Какую недостойную шутку сыграл со мной старик Бах!

\textbf{Черепаха} : Как раз этого он и хотел --- заставить вас заблудиться в его «Лабиринте». Видите ли, злодей Мажотавр --- сообщник Баха. Если вы не остережетесь, он засмеет вас до смерти --- а может быть, и меня вместе с вами!

\textbf{Ахилл} : Надо срочно уносить ноги отсюда! Скорее! Если мы побежим обратно по звуковым дорожкам, то выберемся из пластинки прежде, чем страшный Мажотавр нас обнаружит!

\textbf{Черепаха} : Ну нет, мое ухо слишком чувствительно, чтобы вынести странные аккорды, получающиеся, когда время обращается вспять!

\textbf{Ахилл} : Ах, г-жа Ч, как же мы выберемся отсюда, если мы не можем вернуться по нашим следам?

\textbf{Черепаха} : Хороший вопрос\ldots{} (Почти отчаявшись, Ахилл начинает бегать взад-вперед в темноте. Внезапно раздается сдавленный крик и затем --- БА-БАХ! --- глухой звук падения.) Ахилл? С вами все в порядке?

\textbf{Ахилл} : Ничего особенного, только маленькая встряска: я свалился в какую-то ямину.

\textbf{Черепаха} : Вы угодили прямиком в логово Страшного Мажотавра! Постараюсь вас вытащить --- нам надо удирать побыстрее!

\textbf{Ахилл} : Осторожнее, г-жа Ч --- я совсем не хочу, чтобы и Вы тоже попали в западню\ldots{}

\textbf{Черепаха} : Да не суетитесь вы, Ахилл. Все будет в порядке\ldots{} (Внезапно раздается сдавленный крик и затем --- БА-БАХ! --- глухой звук~~падения.)

\textbf{Ахилл} : Г-жа Ч, вы тоже упали? Не ушиблись?

\textbf{Черепаха} : Кроме моей гордости, ничего не пострадало.

\textbf{Ахилл} : Вот теперь мы действительно попали в переплет!

\emph{(Внезапно, в опасной близости от них, друзья слышат оглушительный хохот.)}

\textbf{Черепаха} : Осторожно, Ахилл --- тут дело нешуточное!

\textbf{Мажотавр} : Ха-ха-ха! Хи-хи-хи! Хо-хо-хо!

\textbf{Ахилл} : Я слабею на глазах, г-жа Ч\ldots{}

\textbf{Черепаха} : Старайтесь не обращать внимания на его смех --- это ваша единственная надежда.

\textbf{Ахилл} : Я сделаю все, что в моих силах --- ах, если бы сейчас пропустить для храбрости рюмочку-другую\ldots{}

\textbf{Черепаха} : Мне кажется, я чувствую знакомый запах\ldots{} Не вытаскин ли это?

\textbf{Ахилл} : И правда\ldots{} откуда этот запах?

\textbf{Черепаха} : По-моему, это здесь\ldots{} О! Я нашла целую бутыль! Это он и есть!

\textbf{Ахилл} : Вытаскин! Давайте напьемся с горя!

Черепаха: Надеюсь, что это не протолкин --- они до того похожи, что их трудно различить.

\textbf{Ахилл} : Что вы сказали про Толкиена?

\textbf{Черепаха} : Я ничего подобного не говорила. У вас уже галлюцинации начинаются\ldots{}

\textbf{Ахилл} : Б-батюшки мои! Надеюсь, что нет\ldots{} Ну что же, поехали!

\emph{(И друзья начинают отхлебывать вытаскин (или протолкин?) --- и вдруг --- ХЛОП! Кажется, это-таки оказался вытаскин\ldots)}

\emph{\textbf{Черепаха}} : Забавная история, ничего не скажешь. Вам понравилось?

\emph{\textbf{Ахилл}} : Так, ничего себе\ldots{} Интересно, выбрались ли они в конце концов из ямы страшного Мажотавра? Бедняга Ахилл, он так хотел опять стать большим.

\emph{\textbf{Черепаха}} : Не беспокойтесь --- они выбрались, и Ахилл снова вырос до своих обычных размеров. Вытаскин оказался весьма кстати\ldots{}

\emph{\textbf{Ахилл}} : Не знаю, не знаю\ldots{} Единственное, в чем я сейчас АБСОЛЮТНО уверен, это в том, что нам не мешало бы найти нашу бутылочку с настойкой --- у меня уже давно горло пересохло. И ничто так не утоляет жажду, как выталкивающая настойка

\emph{\textbf{Черепаха}} : Она к тому же известна своим тонизирующим~~действием. Известны случаи, когда народ просто с ума по ней сходил. Например, когда в начале века продуктовая фабрика Шёнберга перестала производить джин с тоником и начала производство какао, вы не представляете себе, какой из-этого поднялся шум --- настоящая какаофония!

\emph{\textbf{Ахилл}} : Воображаю\ldots{} Но давайте же искать настойку! Погодите --- взгляните-ка на этих ящериц на столе! Не кажется ли вам, что в них есть что-то необычное?

\emph{\textbf{Черепаха}} : Не вижу ничего особенного. А что такое?

\emph{\textbf{Ахилл}} : Посмотрите: они вылезают из плоскости картины без помощи выталкивающей настойки! Как они это делают?

\emph{\textbf{Черепаха}} : Разве я вам не говорила? Вы можете вылезти из картины, двигаясь перпендикулярно ее плоскости. Ящерки научились лезть НАВЕРХ, когда они хотят выбраться из двухмерного мира альбома.

\emph{\textbf{Ахилл}} : Может быть, мы можем так же выбраться из этой картины Эшера наружу?

\emph{\textbf{Черепаха}} : Разумеется --- нужно только подняться уровнем выше. Хотите попытаться?

\emph{\textbf{Ахилл}} : Все что угодно, только бы попасть домой! Я уже сыт по горло этими занимательными приключениями.

\emph{\textbf{Черепаха}} : В таком случае, следуйте за мной наверх.

\emph{(И они поднимаются на один уровень.)}

\emph{Ахилл} : Хорошо быть снова у себя дома\ldots{} Но постойте, здесь что-то не то! Это вовсе не мой дом --- это ВАШ дом, г-жа Черепаха!

\emph{Черепаха} : Вы правы --- и я предовольна, так как перспектива тащиться от вас к себе домой мне совершенно не улыбалась. Я прямо-таки валюсь с лап от усталости.

\emph{Ахилл} : Что ж, я как раз не возражаю против небольшой прогулки; так что, мне кажется, все сложилось довольно удачно.

\emph{Черепаха} : Я думаю! Вот это удача так удача!


% % \subsubsection{ГЛАВА V: Рекурсивные структуры  процессы}
% \subsubsection{ГЛАВА V: Рекурсивные структуры  процессы}

Что такое рекурсия?

ЧТО ТАКОЕ РЕКУРСИЯ? То, что было проиллюстрировано в диалоге «Маленький гармонический лабиринт»: вложенность схем одна в другую и варианты этой вложенности. Рекурсия --- весьма общее понятие. (Рассказы внутри рассказов, фильмы внутри фильмов, картины внутри картин, матрешечки внутри матрешек (даже скобки внутри скобок!) --- вот лишь несколько симпатичных примеров.) Однако читатель должен иметь в виду, что в этой главе термин «рекурсия» употребляется в ином значении, чем в главе III, и эти два значения связаны только косвенно. Эта связь должна проясниться к концу главы.

Иногда рекурсия приближается к парадоксу. Например, существуют \emph{рекурсивные} определения. С первого взгляда может показаться, что в этом случае нечто определяется \emph{через себя самоё} . Из этого получился бы если не парадокс, то порочный круг и бесконечное возвращение к началу. На самом деле, правильно сформулированное рекурсивное определение никогда не приводит ни к тому, ни к другому. Дело в том, что рекурсивные определения никогда не определяют предметы или идеи через них самих --- вместо этого они используют \emph{более простые версии} определяемого понятия. Чтобы вам стало понятнее, что я имею в виду, приведу несколько примеров рекурсивных определений.

Один из часто встречающихся типов рекурсии в повседневной жизни --- это прекращение какого-либо дела на время, с тем, чтобы сделать более простое дело, зачастую того же типа, что и первое. Вот хороший пример. У директора фирмы на столе стоит сложный телефон, по которому ему могут звонить несколько человек одновременно. Директор разговаривает с А; в этот момент звонит Б. Директор спрашивает А, может ли тот подождать минутку. На самом деле, ему совершенно все равно, может ли А подождать, --- он просто нажимает кнопку и переключается на разговор с Б. Тут звонит В. Теперь уже и Б приходится подождать. Так может продолжаться до бесконечности --- однако не будем увлекаться. Предположим, разговор с В закончился; наш директор «выталкивается» обратно и продолжает беседу с Б. Между тем, на другом конце провода А в раздражении барабанит пальцами по столу и слушает сладенькие мелодии, передающиеся по телефону чтобы скрасить его ожидание\ldots{} Самый простой случай был бы, если бы звонок Б закончился и директор наконец вернулся бы к А. Но может случиться, что когда он разговаривает с Б, звонит Д. Б снова оказывается «протолкнутым» в стек ждущих своей очереди. По окончанию разговора с Д директор вернется к Б, а затем к А. Разумеется, здесь он действует совершенно механически --- я пытаюсь показать рекурсию в самой чистой форме.

Проталкивание, выталкивание и стек

В предыдущем примере я ввел основные термины, касающиеся рекурсии, по крайней мере так, как их понимают специалисты по компьютерам: проталкивание, выталкивание и стек. Все эти термины связаны между собой. Они вошли в обиход в конце 1950-х годов в составе ИПЛ, одного из первых языков для искусственного разума. Вы уже встречались с «проталкиванием» и «выталкиванием» в Диалоге; однако я объясню здесь эти термины еще раз. \emph{Протолкнуть} означает прервать работу над очередным делом, при этом не забывая, на чем вы остановились, и начать работать над следующим заданием. Обычно говорят, что новое дело находится на «низшем уровне» по сравнению с предыдущим занятием. \emph{Вытолкнуть} означает обратное: прекратить работу на одном уровне и вновь приняться за работу на высшем уровне, начав с того, на чем вы остановились.

Как же нам удается точно помнить, где мы были на каждом уровне? Для этого мы сохраняем нужную информацию в \emph{стеке} . Таким образом, стек --- это просто табличка, сообщающая нам 1) на чем было прервано каждое незаконченное занятие (на компьютерном жаргоне это называется «обратный адрес») и 2) какие факты нам надо знать о моменте, когда задание было прервано («переменная связка»). Когда вы выталкиваетесь наверх, чтобы возобновить работу над чем-либо, именно стек восстанавливает ваш контекст, чтобы вы не потерялись. В примере с телефонными звонками стек сообщает вам, \emph{кто ждет} вас на каждой линии и \emph{в каком месте} беседа была прервана.

К слову сказать, происхождение терминов «проталкивать», «выталкивать» и «стек» восходит к образу сложенных один на другой подносов в кафетерии (\emph{stack} по-английски --- куча, стеллаж). Обычно внизу такой стопки помещается нечто вроде пружины, поддерживающей верхний поднос приблизительно на одном и том же уровне --- так что каждый новый поднос «проталкивает» всю стопку вниз, в то время как при снятии одного подноса все стопка «выталкивается» наверх.

Еще один пример из повседневной жизни. Когда вы слушаете новости по радио, часто случается, что слово предоставляется иностранному корреспонденту. «Говорит Адам Зайчиков из Минска, Беларусь.» Адам, в свою очередь, включает запись местного репортера, берущего у кого-то интервью: «С вами Иван Петровский; я нахожусь недалеко от того места, где совершилось ограбление банка. Предоставляю слово главе оперативной группы\ldots» Теперь вы уже тремя уровнями ниже. Может случиться, что и тот, у кого берут интервью, тоже включит какую-то запись. Спускаться таким образом по уровням, слушая новости --- дело весьма обычное; мы даже не всегда отдаем себе отчет в том, что сообщение на одном уровне прерывается. Наше подсознание следит за этим автоматически. Может быть, это так легко для нас потому, что уровни здесь сильно отличаются друг от друга. Если бы они были схожими, мы потеряли бы ориентацию в мгновение ока.

Пример более сложной рекурсии --- наш Диалог. Ахилл и Черепаха присутствовали там на каждом из нескольких различных уровней. Иногда они читали историю, в которой сами были действующими лицами. Тут было легко запутаться, и приходилось напрягать все внимание, чтобы не потерять нить. «Так, посмотрим\ldots{} \emph{настоящие} Ахилл и Черепаха все еще наверху, в вертолете господина Удачи --- \emph{вторичные} сейчас находятся в картине Эшера --- а теперь они нашли ту книгу и начали читать; значит, Ахилл и Черепаха, блуждающие по звуковым дорожкам „Маленького гармонического лабиринта``, --- третичны. Стоп --- я, кажется, пропустил один уровень\ldots» Чтобы уследить за рекурсией в Диалоге, нам необходим сознательный мысленный стек, подобный такому, какой изображен ниже.

\emph{Рис. 26. Диаграмма структуры Диалога «Маленький гармонический лабиринт». Вертикальные спуски --- проталкивание, подъемы --- выталкивание. Обратите внимание, что диаграмма напоминает структуру абзацев в Диалоге. Из нее ясно следует, что угроза Удачи так никогда и не была выполнена --- Ахилл и Черепаха остались висеть между небом и землей. Некоторые читатели, возможно, придут в отчаяние от этого недовытолкутого проталкивания, в то время как другие даже глазом не моргнут. В рассказе Баховский музыкальный лабиринт тоже был оборван слишком, скоро --- но Ахилл не заметил в этом ничего особенного. Нарастающее напряжение почувствовала только Черепаха.}

Стеки в музыке

Говоря о «Маленьком гармоническом лабиринте», мы должны обсудить следующую идею, которая косвенно упоминалась в диалоге: мы слушаем музыку рекурсивно --- в частности, мы создаем мысленный стек ключей, и каждая новая модуляция проталкивает туда новый ключ. Если развить эту идею дальше, получится, что мы хотим услышать последовательность ключей в обратном порядке --- выталкивая из стека ключи один за другим, пока не дойдем до основной тональности. Это, разумеется, преувеличение, но в нем есть доля правды. Слушая музыку, любой сколько-нибудь музыкальный человек автоматически создает минимальный стек с двумя ключами. В этом «коротком стеке» содержатся основная тональность, а также ближайший «псевдоключ», тональность, в которой композитор «находится» в данный момент. (Иными словами, самый общий и самый «местный» ключи. Таким образом слушатель знает, когда достигается тоника, и испытывает от этого сильное чувство «удовлетворения». В отличие от Ахилла, он также чувствует разницу между \emph{местным} разрешением напряжения --- например, разрешением в псевдотонику --- и \emph{глобальным} разрешением. Псевдоразрешение нагнетает напряжение, вместо того, чтобы его ослабить. Оно подобно иронической шутке --- совсем как спасение Ахилла от ящериц, в то время как мы знаем, что на самом деле и он, и Черепаха все еще ожидают погибели от ножа месье Удачи.

Поскольку напряжение и разрешение --- душа и сердце музыки, существует множество примеров на эту тему. Давайте взглянем на пару примеров из Баховской музыки. Бах написал много композиций в форме «ААББ»: обе части пьесы повторяются дважды. Возьмем джигу из «Французской сюиты \#5», типичную для данной формы. Ее энергично введенная танцевальной мелодией тоника --- «соль». Вскоре, однако, модуляция в части А вводит тесно связанную с первоначальной тональность «ре» (доминанта). Когда часть А кончается, мы находимся в тональности «ре». Может даже показаться, что пьеса заканчивается в ключе «ре»! (По крайней мере, так может подумать Ахилл.) Но тут случается странная вещь --- мы внезапно прыгаем обратно к началу, снова в тональность «соль», и снова слышим тот же переход в «ре». Но тут случается странная вещь --- мы внезапно прыгаем обратно к началу, снова в тональность «соль», и снова слышим тот же переход в «ре».

Затем следует часть Б. В результате тематического сдвига, мелодия здесь начинается с «ре», словно «ре» являлось тоникой с самого начала --- но в конце концов, мелодия модулирует обратно в «соль»; это означает, что мы выталкиваемся обратно в тонику, и что часть Б оканчивается именно так, как надо. Тут случается это забавное повторение, отбрасывая нас, безо всякого предупреждения, назад к «ре», и затем возвращаясь к «соль» еще раз. Тут случается это забавное повторение, отбрасывая нас, безо всякого предупреждения, назад к «ре», и затем возвращаясь к «соль» еще раз.

Психологический эффект, достигаемый этими переходами, то внезапными, то плавными, трудно описать. Магия музыки отчасти и заключается в том, что мы способны автоматически уследить за этими переходами. А может быть, это магия Баха, сумевшего внести такую грацию в эту сложную структуру, что мы даже не замечаем, что именно там происходит.

Баховский «Маленький гармонический лабиринт» --- это пьеса, в которой композитор пытается запутать слушателя быстрой сменой ключей. Вскоре вы настолько сбиты с толку, что совершенно теряете ориентацию. Вы не знаете, где настоящая тоника, если только у вас нет абсолютного слуха или вы, подобно Тезею, не прибегаете к помощи друга, который, словно Ариадна, дал бы вам нить, ведущую к началу. В данном случае, такой нитью являлись бы ноты. Эта пьеса, наряду с Естественно Растущим Каноном, показывает, что у нас, как у слушателей музыки, отсутствуют надежные глубокие стеки.

Рекурсия в языке

Наш интеллектуальный стек, пожалуй, более надежен для работы с языком. Грамматическая структура всех языков включает весьма сложные схемы для проталкивания в стек; трудность фразы, разумеется, возрастает с количеством проталкиваний. Знаменитое немецкое явление «глагола-в-конце», о котором забавные истории о рассеянных профессорах, начинающих фразу, продолжающуюся все лекцию, и под завязку выдающих цепочку глаголов, в которой аудитория, давно потерявшая нить в этом стеке, не видит никакого смысла, часто рассказываются, представляет из себя прекрасный пример лингвистического проталкивания и выталкивания. Замешательство в аудитории, которое неправильное выталкивание из стека, куда были сложены глаголы профессора, забавно вообразить, может произвести. Однако в повседневном немецком такие глубокие стеки почти никогда не встречаются; на самом деле, немцы частенько невольно нарушают правила, проталкивающие глагол в конец, с тем, чтобы избежать усилий, связанных с напряжением внимания в течение всей фразы. В любом языке имеются конструкции, где задействованы стеки, хотя обычно не такие впечатляющие, как в немецком. При этом всегда имеется возможность перефразировать предложение таким образом, чтобы уменьшить глубину стека.

Схемы рекурсивных переходов

Синтаксическая структура предложений хорошо подходит для метода описания рекурсивных схем и процессов --- этот метод называется \emph{Схемой Рекурсивных Переходов} (СРП). СРП представляет из себя диаграмму, показывающую различные пути для выполнения данного задания. Каждый такой путь состоит из нескольких \emph{узлов} --- маленьких квадратов, в которых что-то написано. Узлы соединены \emph{ребрами} , или стрелками. Общее название данной СРП написано отдельно, слева от диаграммы, и в первом и последнем узле написано, соответственно, \emph{начало} и \emph{конец} . Остальные узлы содержат либо краткие инструкции, либо названия других СРП. Попав в определенный узел, вы должны либо выполнить указания, в нем написанные, либо перейти в указанную в нем СРП и работать уже там.

Возьмем простую СРП, под названием УКРАШЕННОЕ СУЩЕСТВИТЕЛЬНОЕ, которая говорит нам, как создать определенную русскую фразу (см. рис. 27а) Двигаясь по схеме горизонтально, мы попадаем в \emph{начало} , затем создаем прилагательное, затем --- существительное, и затем приходим к \emph{концу} . Например, «глупое мыло», или «неблагодарная закуска». Но ребра позволяют и другие возможности, например, повторить или совсем опустить прилагательное. Так мы можем сконструировать «молоко» или «огромная красная голубая зеленая зевота» и так далее.

Находясь в узле имя существительное, вы просите некий черный ящик под названием имя существительное выдать вам любое существительное с его склада. В компьютерной терминологии это называется \emph{процедурой вызова} . Это означает, что вы временно передаете контроль некой \emph{процедуре} (здесь, СУЩЕСТВИТЕЛЬНОМУ), которая 1) выполняет свою инструкцию (производит существительное) и 2) передает контроль вам обратно. В нашей СРП есть вызовы для двух таких процедур имя существительное и ИМЯ ПРИЛАГАТЕЛЬНОЕ. Обратите внимание, что СРП УКРАШЕННОЕ СУЩЕСТВИТЕЛЬНОЕ может, в свою очередь, быть вызвана из какой-либо другой СРП --- например, ПРЕДЛОЖЕНИЕ. В этом случае, схема УКРАШЕННОЕ СУЩЕСТВИТЕЛЬНОЕ произвела бы «глупое мыло» и вернулась бы на свое место в предложении, откуда она была вызвана. Эта ситуация напоминает примеры со вложенными один в другой телефонными звонками или фрагментами новостей, где вы возвращаетесь к прерванному занятию.

Однако, хотя мы и назвали это «схемой рекурсивных переходов», мы еще не привели примера настоящей рекурсии.

\emph{Ри. 27. Схема рекурсивных переходов для УКРАШЕННОГО СУЩЕСТВИТЕЛЬНОГО~~И СВЕРХУКРАШЕННОГО СУЩЕСТВИТЕЛЬНОГО.}

Рекурсия --- и, по видимости, кругообразность --- появляется тогда, когда мы переходим к такой СРП как СВЕРХУКРАШЕННОЕ СУЩЕСТВИТЕЛЬНОЕ (Рис 27б). Как вы заметили, любая дорожка к СВЕРХУКРАШЕННОМУ СУЩЕСТВИТЕЛЬНОМУ проходит через узел УКРАШЕННОЕ СУЩЕСТВИТЕЛЬНОЕ --- таким образом, у нас обязательно появится какое-либо существительное. Мы можем на этом закончить и прийти к ФИНИШУ с «молоком» или «огромной красной голубой зеленой зевотой». Но остальные три пути к финишу сами включают \emph{рекурсивный} вызов СВЕРХУКРАШЕННОГО СУЩЕСТВИТЕЛЬНОГО. Это выглядит как порочный круг --- определение чего-либо в терминах его самого. Действительно ли это происходит? На этот вопрос мы ответим так: «Да, но это не страшно.» Представьте, что в процедуре ПРЕДЛОЖЕНИЕ есть узел, вызывающий СВЕРХУКРАШЕННОЕ СУЩЕСТВИТЕЛЬНОЕ, и мы попадаем именно в этот узел. Это означает, что мы прежде всего запоминаем (проталкиваем в стек) место этого узла внутри ПРЕДЛОЖЕНИЯ, чтобы знать, куда нам вернуться; после этого, мы переходим к самой процедуре СВЕРХУКРАШЕННОЕ СУЩЕСТВИТЕЛЬНОЕ --- мы должны найти способ его сконструировать. Предположим, что мы выбираем нижнюю из двух верхних дорожек:

УКРАШЕННОЕ СУЩЕСТВИТЕЛЬНОЕ, ОТНОСИТЕЛЬНОЕ МЕСТОИМЕНИЕ, СВЕРХУКРАШЕННОЕ СУЩЕСТВИТЕЛЬНОЕ, ГЛАГОЛ.

Итак, за дело: сначала мы выдаем «на-гора» УКРАШЕННОЕ СУЩЕСТВИТЕЛЬНОЕ: «странные бублики»; затем, относительное местоимение: «которые»\ldots{} теперь мы должны воспроизвести СВЕРХУКРАШЕННОЕ СУЩЕСТВИТЕЛЬНОЕ --- но ведь мы как раз и находимся в процессе создания СВЕРХУКРАШЕННОГО СУЩЕСТВИТЕЛЬНОГО! Это верно, но вспомните наш пример с директором, которому позвонили в середине другого телефонного разговора. Он «отложил» первый разговор в стек и начал новую беседу так, словно ничего необычного не случилось. Давайте и мы сделаем так же.

Прежде всего запасемся обратным адресом: запишем в стек, в каком узле мы находились во время второго вызова СВЕРХУКРАШЕННОГО СУЩЕСТВИТЕЛЬНОГО. Затем снова перейдем в начало схемы, словно ничего необычного не случилось. Теперь мы должны снова выбрать путь. Давайте, для разнообразия, попробуем пройти по нижней дорожке: УКРАШЕННОЕ СУЩЕСТВИТЕЛЬНОЕ, ПРЕДЛОГ, СВЕРХУКРАШЕННОЕ СУЩЕСТВИТЕЛЬНОЕ. Это значит, что сначала мы производим УКРАШЕННОЕ СУЩЕСТВИТЕЛЬНОЕ (например, «пурпурная корова»), затем ПРЕДЛОГ (например, «без»)\ldots{} и опять упираемся в рекурсию. Придется нам снова спуститься уровнем ниже --- смотрите не споткнитесь! Чтобы избежать осложнений, давайте на этот раз выберем прямую дорогу. УКРАШЕННОЕ СУЩЕСТВИТЕЛЬНОЕ (например, «рога»). Этот вызов тут же попадает в узел КОНЕЦ, что позволяет нам вытолкнуться на предыдущий уровень. Мы обращаемся к стеку за обратным адресом, который отсылает нас к фразе «пурпурная корова без». Закончив дела на этом уровне и попав в узел КОНЕЦ, мы выталкиваемся еще раз. Теперь нам необходим ГЛАГОЛ (например, «слопала»). На этом вызов СВЕРХУКРАШЕННОГО СУЩЕСТВИТЕЛЬНОГО на высшем уровне заканчивается. У нас получилась фраза:

\emph{«странные бублики, которые пурпурная корова без рогов слопала»} .

Когда мы вытолкнемся в последний раз, эта фраза будет передана наверх, к терпеливо ожидающей схеме ПРЕДЛОЖЕНИЕ.

Как видите, бесконечной регрессии не произошло, так как по крайней мере на одной из дорожек внутри СРП СВЕРХУКРАШЕННОЕ СУЩЕСТВИТЕЛЬНОЕ мы не встретились с вызовом самого СВЕРХУКРАШЕННОГО СУЩЕСТВИТЕЛЬНОГО. Конечно, мы могли бы упорствовать в выборе нижней дорожки внутри СВЕРХУКРАШЕННОГО СУЩЕСТВИТЕЛЬНОГО --- тогда бы нам никогда не удалось закончить работу, подобно тому, как нам не удалось полностью раскрыть сокращение БОГ. Однако если мы выбираем дорожки наугад, подобной бесконечной регрессии не случается.

«Спуск на дно» и гетерархии

Мы только что описали основные различия между круговыми и рекурсивными определениями --- в последних всегда есть определенная часть без автореферентности. Таким образом, рано или поздно мы коснемся дна: наша цель --- построение объекта, отвечающего определению --- будет достигнута. Существуют и другие, менее прямые, чем самовызовы, пути для получения рекурсивности в СРП. Примером может служить картина Эшера «Рисующие руки» (рис. 135), где каждая процедура вызывает не саму себя, а другую. Например, можно представить СРП под названием ПРИДАТОЧНОЕ ПРЕДЛОЖЕНИЕ, вызывающую СВЕРХУКРАШЕННОЕ СУЩЕСТВИТЕЛЬНОЕ, когда ей понадобится дополнение для переходного глагола --- с другой стороны, высшая дорожка СВЕРХУКРАШЕННОГО СУЩЕСТВИТЕЛЬНОГО может вызывать ОТНОСИТЕЛЬНОЕ МЕСТОИМЕНИЕ и затем ПРЕДЛОЖЕНИЕ каждый раз, когда нам потребуется придаточное предложение. Это пример \emph{косвенной} рекурсии; он напоминает двухступенчатую версию парадокса Эпименида.

Нет нужды говорить, что может существовать также \emph{трио} процедур, вызывающих одну другую по кругу --- и так далее. Может существовать даже целая семья СРП, спутанных между собой и что есть силы вызывающих друг друга и самих себя. Программа со структурой, в которой нет «высшего уровня» или «монитора», называется \emph{гетерархией} (в отличие от \emph{иерархии} ). Этот термин изобретен Уорреном Мак Каллохом, одним из первых кибернетиков, посвятивших себя изучению мозга и интеллекта.

Расширение узлов

Есть также и другая возможность представить СРП графически. Каждый раз, когда, двигаясь по одной из дорожек, вы попадаете в узел, вызывающий другую СРП, вы «расширяете» этот узел, заменяя его на уменьшенную копию требуемой СРП (см. рис. 28). После этого вы приступаете к исполнению этой уменьшенной СРП.

\emph{Рис. 28. СРП СВЕРХУКРАШЕННОЕ СУЩЕСТВИТЕЛЬНОЕ с одним рекурсивно расширенным узлом.}

Выталкиваясь из расширенного узла, вы автоматически оказываетесь в нужном месте большой схемы. С другой стороны, находясь в маленькой схеме, вы можете конструировать внутри нее еще более миниатюрные СРП. Расширяя узлы по мере того, как вы в них попадаете, вы избегаете построения бесконечной схемы даже в том случае, когда СРП вызывает саму себя. Расширение узлов немного напоминает замену буквы в аббревиатуре на то слово, которое она представляет. Сокращение БОГ рекурсивно, но его дефект --- или преимущество --- заключается в том, что мы должны все время расширять букву «Б» и, таким образом, она никогда не достигнет «дна». Однако когда СРП является частью настоящей компьютерной программы, в ней всегда есть по крайней мере одна дорожка, избегающая как прямой, так и косвенной рекурсивности. Поэтому бесконечного регресса там не бывает. Даже самая гетерархическая программа рано или поздно заканчивается --- иначе она вообще не работала бы! Она продолжала бы расширять узлы один за другим до скончания веков.

Диаграмма G и рекурсивные ряды

Бесконечные геометрические структуры могут быть определены именно так-как расширение узлов один за другим. Давайте попробуем определить бесконечную диаграмму --- назовем ее «диаграммой G». Воспользуемся следующим условным обозначением, в двух узлах напишем просто букву «G», которая, однако, будет представлять всю диаграмму G. На рис. 28 показана диаграмма G, использующая такую условную нотацию. Если мы захотим представить эту диаграмму более явно, мы должны расширить каждый узел, обозначенный буквой G, то есть заменить его на уменьшенную копию той же диаграммы G (см. рис. 29 б). Эта версия диаграммы G «второго порядка» дает нам некоторое представление о том, как бы выглядела конечная, невыполнимая диаграмма G. На рис. 30 показана большая часть диаграммы G; все узлы пронумерованы снизу вверх и слева направо. Внизу добавлены два дополнительных узла под номерами 1 и 2. У этого бесконечного «дерева» есть некоторые весьма интересные математические свойства. Двигаясь по нему справа налево, мы получаем знаменитый ряд чисел Фибоначчи:

1, 1, 2,~3, 5, 8, 13, 21, 34, 55, 89, 144, 233\ldots{}

Этот рад был открыт в 1202 году Леонардом из Пизы, сыном Боначчи --- отсюда Филиус Боначчи или, сокращенно, Фибоначчи.

\emph{Рис. 29. а) Диаграмма G, нерасширенная; б) Диаграмма G,~расширенная один раз;~в) Диаграмма H, нерасширенная; г) Диаграмма H, расширенная один раз один раз}

\emph{Рис. 30. Диаграмма G, расширенная далее. Узлы пронумерованы.}

Это числа описываются рекурсивно при помощи следующей пары формул:

FIBO~(n) =~FIBO (n --- 1)~+~FIBO (n --- 2) for~n \textgreater{} 2

FIBO (n)~= FIBO (2) = 1

\emph{Рис. 31. СРП для чисел Фибоначчи}

Таким образом, вы можете вычислить ФИБО(15) с помощью ряда рекурсивных вызовов описанной в этой схеме процедуры. Это рекурсивное определение касается дна, когда вы доходите до явно выраженных ФИБО(1) и ФИБО(2). Для этого надо пройти по схеме назад, к меньшим и меньшим значениям \emph{n} . Пятиться раком довольно неудобно, вместо этого можно начать с ФИБО(1) и ФИБО(2) и идти вперед, складывая два предыдущих числа, пока вы не получите ФИБО(15). Так вам не придется следить за стеком.

Но это еще не самое интересное свойство диаграммы G! Ее структура может быть целиком закодирована в следующем рекурсивном определении.

G(n)~= n-G(G(n-1)) для n\textgreater0

G(0) = 0

Каким образом эта формула G(n) отражает структуру дерева? Очень просто: если вы начнете строить дерево, помещая G(n) под~\emph{n} для всех значений \emph{n} , у вас получится диаграмма G. На самом деле, именно так я и открыл эту диаграмму. Я занимался исследованием \emph{функции} G; однажды, пытаясь ускорить вычисления, я решил представить уже имеющиеся у меня значения в форме дерева. К моему удивлению оказалось, что это дерево обладает очень аккуратной геометрической рекурсивностью.

Еще более занимательным получается аналогичное дерево для функции H(n), имеющей на одно рекурсивное вложение больше, чем G:

H(n) = n - H(H(H(n-1))) для n\textgreater0

H(0) = 0

Таким образом, соответствующая диаграмма~H косвенно определяется так, как показано на рис. 29 в). Правая ветвь отличается от G только тем, что в ней на один узел больше. И так далее, для любого количества вложений. Рекурсивные геометрические структуры проявляют замечательную регулярность, в точности соответствующую рекурсивным алгебраическим определениям.

Вопрос для любознательных читателей: представьте себе, что вы перевернули диаграмму G так, что у вас получилось ее зеркальное отображение. Номера узлов нового дерева возрастают теперь слева направо. Можете ли вы найти рекурсивное \emph{алгебраическое} определение для такого «дерева-перевертыша»? Как насчет определения для перевертыша дерева H? И так далее?

Другая забавная задача включает пару рекурсивно сплетенных функций F(n) и M(n) --- так сказать, супружеская парочка функций --- определенных следующим образом:

F(n) = n-M(F(n-1))

для n\textgreater0

M(n) = n-F(M(n-1))

F(0) = 1, M(0) = 0

СРП для этих двух функций вызывают как друг друга, так и самих себя. Задача состоит в том, чтобы найти рекурсивные структуры диаграмм~M и F. Они весьма просты и элегантны.

Хаотическая последовательность

Последний пример рекурсии в теории чисел приводит к небольшой загадке. Рассмотрим следующее рекурсивное определение функции.

Q(n) = Q(n-Q(n-1)) + Q(n-Q(n-2) для n\textgreater2

Q(1) = Q(2) = 1

Это напоминает определение Фибоначчи тем, что каждое новое значение является суммой двух предыдущих значений --- но не \emph{ближайших} ! Вместо этого, два ближайших предыдущих значения указывают нам, \emph{насколько далеко мы должны отступить} , чтобы найти числа, которые надо сложить для получения нового значения. Вот первые семнадцать чисел Q.

Чтобы получить следующее число, надо продвинуться налево (считая от многоточия), соответственно, на 9 и 10 шагов; вы получите 5 и 6 (отмеченные стрелками). Их сумма --- 11 --- и дает новое значение: Q(18). Странный процесс: список уже известных чисел Q используется для расширения самого ряда. Получающаяся последовательность, мягко выражаясь, беспорядочна, и чем дальше мы продвигаемся, тем бессмысленнее она кажется. Это один из тех странных случаев, когда естественное определение приводит к весьма странному результату --- хаос, полученный упорядоченным способом. При этом возникает вопрос: нет ли в кажущемся хаосе какого-то скрытого порядка? Разумеется, из определения следует, что некий порядок существует. Но интересно, есть ли иной способ определить данный ряд --- если повезет, нерекурсивно?

Два удивительных рекурсивных графика

Чудес рекурсии в математике множество, и я не собираюсь здесь говорить о них подробно. Я остановлюсь лишь на двух особо интересных случаях с которыми мне пришлось столкнуться. Речь пойдет о двух графиках. Один из них --- часть моих исследований по теории чисел. Другой возник в процессе моей работы над докторской диссертацией по физике твердых тел. Особенно поразительно то, что эти графики находятся в родстве между собой.

Первый (рис. 32) --- график функции, которую я называю INT (\emph{x} ). Здесь она дана для~\emph{x} между 0 и 1. Чтобы найти~\emph{x} между любой другой парой чисел \emph{n} и \emph{n} +1, вы должны вычислить INT (\emph{x} -\emph{n} ) и затем снова прибавить \emph{n} . Как видите, структура этого графика прерывиста. Она состоит из бесконечного числа изогнутых кусочков, уменьшающихся ближе к краям. Если вы посмотрите на любой такой кусочек попристальнее, вы увидите, что перед вами --- копия целого графика, только слегка изогнутая! Последствия этого удивительны; одним из них является то, что график INT состоит исключительно из копий себя самого, вложенных одна в другую до бесконечности. Если вы возьмете любую, сколь угодно малую часть графика, у вас окажется полная копия всего графика --- на самом деле, бесконечное количество таких копий!

\emph{Рис. 32. График функции INT(x). В точках рациональных значений~x функция прерывается.}

Вы можете подумать, что INT слишком эфемерна, чтобы существовать в действительности, поскольку она состоит лишь из копий самой себя. Ее определение выглядит слишком круговым.

Как начинается эта функция? Где ее «исток»? Это очень интересный вопрос. Важно отметить, что, описывая INT человеку, никогда не видевшему графика этой функции, недостаточно просто сказать, что она состоит из копий себя самой. Вторая, нерекурсивная часть описания должна содержать сведения о том, \emph{где} эти копии лежат внутри графика~и \emph{каким образом} они деформированы по отношению к нему. Только взятые вместе, эти два аспекта INT определяют ее структуру. Точно так же, чтобы определить числа Фибоначчи, нам понадобились две строчки --- одна, определяющая \emph{рекурсию} , и другая, определяющая \emph{дно} --- первоначальные значения функции. Приведу конкретный пример: если вы замените одно из двух первоначальных значений на 3 вместо 1, то получите совершенно иную последовательность, известную под названием \emph{ряда Лукаса} :

В определении INT «дну» соответствует рисунок (рис. 33а), состоящий из множества квадратов, указывающих, \emph{где} находятся копии и \emph{каким образом} они деформированы. Я называю это «скелетом» INT. Чтобы построить INT на основе скелета, вы должны действовать следующим образом. Сначала для каждого квадрата надо проделать две операции: (1) вложите туда уменьшенную и изогнутую копию скелета, следуя направлению изогнутой линии внутри; (2) сотрите квадрат-рамку и линию внутри него. Закончив этот процесс для каждого квадрата первоначального скелета, вы получите вместо одного большого скелета множество скелетов-«деток». Теперь тот же процесс повторяется уровнем ниже, для каждого скелета-детки. Затем то же самое повторяется еще раз, и еще, и еще\ldots{} В пределе вы приближаетесь к точному графику INT, хотя никогда его не достигаете. Снова и снова вкладывая скелет графика внутрь себя самого, вы постепенно строите график «из ничего». Но, по сути, «ничто» не было таковым --- оно было рисунком.

\emph{Рис. 33 а. Скелет, на базе которого путем рекурсивной замены строится INT.}

\emph{Рис. 33 б. Скелет, на базе которого путем рекурсивной замены строится график G.}

Поясним сказанное на еще более впечатляющем примере: вообразите, что вы оставляете рекурсивную часть определения INT, но заменяете начальный рисунок, скелет. Вариант скелета показан на рис. 33б); также и здесь квадраты уменьшаются ближе к углам. Если вы начнете вкладывать этот скелет в себя самого снова и снова, вы получите основной график моей докторской диссертации, который я назвал Графиком G (рис. 34). (На самом деле, там также потребовались определенные сложные деформации, но основной идеей остается «самовложение».) Таким образом, График G --- член семьи INT. Это дальний родственник, так как его скелет намного сложнее скелета INT; однако рекурсивные части их определений идентичны, и именно в этом заключается их родство.

Я не буду слишком долго держать вас в неведении относительно происхождения этих замечательных графиков. INT (сокращенное interchange --- обмен) связан с проблемой непрерывных дробей, а еще точнее --- «последовательностей ETA». В основе INT лежит идея о том, что знаки плюс и минус взаимозаменяемы для определенного вида непрерывных дробей. Отсюда следует то, что INT(INT(\emph{x} ))=\emph{x} . Когда~\emph{x} рационально, ITN(\emph{x} ) также рациональна; квадратичные значения~\emph{x} дают квадратичные значения INT(\emph{x} ). He знаю, верна ли эта тенденция для высших алгебраических степеней. Другим любопытным свойством INT является то, что в точках рациональных значений~\emph{x} функция разрывается скачками, в то время как в точках иррациональных значений~\emph{x} она непрерывна.

\emph{Рис. 34. График G: рекурсивный график, показывающий энергетические полосы~для электронов в идеализированном кристалле, помещенном в магнитное поле. a, представляющая силу магнитного поля, изменяется вертикально от 0 до 1.Энергия показана на горизонтальной оси. Сегменты горизонтальных линий --- разрешенные энергии электронов.}

График G представляет собой сильно упрощенный ответ на вопрос «Какую энергию может иметь электрон в кристалле, помещенном в магнитное поле?» Это очень интересная проблема, так как она совмещает две фундаментальные физические ситуации: электрон в совершенном кристалле и электрон в однородном магнитном поле. Решения этих простых проблем хорошо известны и кажутся почти несовместимыми; тем интереснее выяснить, как природе удается их совместить. Оказывается, что ситуации «электрон в кристалле без магнитного поля» и «электрон в магнитном поле без кристалла» все-таки имеют одну общую черту: в обоих случаях электрон ведет себя периодично во времени. Когда две ситуации совмещаются, отношение их периодов является ключевым параметром, так как оно выражает возможные уровни энергии электронов. Однако свой секрет это отношение выдает только тогда, когда оно записано в форме непрерывной дроби.

График G показывает это распределение. Горизонтальные оси представляют энергию, вертикальные --- упомянутое выше отношение временных периодов, которое мы называем «\emph{а} ». Внизу а равняется нулю, наверху --- единице. Когда \emph{а} равняется нулю, магнитное поле отсутствует. Каждый из составляющих график G сегментов --- энергетическая полоса, представляющая возможные уровни энергии. Каждая из разномасштабных пустых полос, пересекающих график G, представляет районы запрещенных энергий. Одним из самых удивительных свойств графика G является то, что когда \emph{а} рациональна (иными словами, может быть представлена в форме p/q), то существует ровно \emph{q} таких пустых полос (хотя, когда \emph{q} четно, две из них «целуются» в центре).

Когда \emph{а} иррационально, полосы сжимаются до точек, бесконечное число которых разбросано по так называемому «множеству Кантора» --- еще один рекурсивно определяемый объект, берущий начало в топологии.

У читателя может возникнуть вопрос, можно ли получить такую сложную структуру экспериментальным путем. Честно говоря, я бы сам удивился больше всех, если бы в результате какого-нибудь эксперимента получился График G. График G «физичен» в том смысле, что он указывает, как можно математически подходить к менее идеальным физическим проблемам. Другими словами, График G принадлежит к области теоретической физики, а не указывает физикам-практикам на то, что они могут получить в результате экспериментов. Как-то раз один из моих друзей-агностиков, пораженный бесконечным количеством бесконечностей Графика G, именовал этот график «портретом Бога» --- и это совсем не показалось мне богохульством.

Рекурсия на низшем уровне материи

Мы уже встретились с рекурсией в грамматике языков, видели рекурсивные геометрические деревья, тянущие свои ветви в бесконечность, и привели пример рекурсии в физике твердых тел. Теперь давайте взглянем еще на один способ рекурсивного устройства мира. Я имею в виду элементарные частицы: электроны, протоны, нейтроны и крохотные кванты электромагнитного излучения, называемые «фотонами». Мы увидим, что эти частицы в некотором роде «вставлены» друг в друга (это определено со всей строгостью только в релятивистской квантовой механике), и что это положение можно описать рекурсивно --- может быть, даже с помощью какой-либо «грамматики».

Начнем с того, что если бы элементарные частицы не взаимодействовали друг с другом, мир был бы невероятно прост. В таком мире физики были бы наверху блаженства, так как там они могли бы с легкостью вычислить поведение всех частиц! (Конечно, при условии, что в таком мире существовали бы сами физики --- что кажется весьма сомнительным.) Невзаимодействующие частицы называются \emph{голыми} , и являются чисто гипотетическими --- в реальном мире их не существует.

Теперь представьте себе, что мы «включаем» взаимодействия --- частицы связываются между собой так же, как связаны между собой функции~M и F или женатые пары. Эти реальные частицы называются \emph{ренормализованными} --- неуклюжий, но интересный термин. Теперь каждая частица определяется через совокупность всех других частиц, которые, в свою очередь, определяются через первую частицу, и так далее. Получается движение кругом и кругом, по бесконечной петле.

Давайте теперь перейдем на более конкретные темы и ограничимся двумя частицами --- \emph{электронами} и \emph{фотонами} . Нам также придется включить сюда и античастицу электрона --- позитрон. (Фотон является античастицей себя самого.) Вообразите себе скучный мир, в которой голый электрон желает добраться от точки А до точки В, как Зенон в моей «Трехголосной инвенции».

Физик нарисовал бы такую картину:

Существует весьма простое математическое выражение, соответствующее этому отрезку и его конечным точкам. С его помощью, физик может понять поведение голого электрона на этой траектории.

Теперь давайте «включим» электромагнитное взаимодействие, так что электроны и фотоны начнут взаимодействовать. Хотя в этой сцене фотоны не участвуют, наше допущение будет иметь серьезные последствия даже для этой простой траектории. В частности, электроны теперь способны испускать и снова поглощать \emph{виртуальные фотоны} --- фотоны, рождающиеся и умирающие прежде, чем их заметят. Этот процесс выглядит так:

По мере того, как электрон распространяется, он может испускать и снова поглощать один фотон за другим, иногда вкладывая один фотон в другой, как показано на рисунке ниже:

Математические выражения, соответствующие этим диаграммам --- так называемым «диаграммам Файнмана» --- легко записать, но труднее вычислить, чем соответствующие выражения для голых электронов. Самое сложное то, что фотон --- реальный или виртуальный --- может на мгновение превратиться в пару электрон-позитрон. Между ними происходит аннигиляция, и, как по волшебству, первоначальный фотон появляется снова! Этот процесс показан на рисунке ниже:

Стрелка, указывающая направо, --- электрон, налево --- позитрон. Как вы, наверно, догадались, эти виртуальные процессы могут вставляться один в другой до любой глубины. В результате может получиться довольно сложная диаграмма, такая, как показана на рис. 35. На данной диаграмме Файнмана один электрон входит слева в точке А, и после серии удивительных акробатических трюков выходит справа в точке В. Отсюда видно, что линии как электрона, так и фотона могут быть сколько угодно «украшены». Такую диаграмму чрезвычайно трудно вычислить.

\emph{Рис. 35. Диаграмма Файнмана. показывающая распространение ренормализованного электрона от А до В. Время возрастает слева направо, это значит, что в тех местах, где стрелка указывает справа налево, электрон движется «обратно во времени». Или, говоря более интуитивно, антиэлектрон(позитрон) движется вперед во времени. Фотоны --- свои собственные античастицы, и поэтому их линии не нуждаются в стрелках}

У этих диаграмм своя «грамматика», позволяющая воплотиться в жизнь только определенным картинкам. Например, ситуация, изображенная ниже, невозможна:

Вы можете возразить, что это не является «правильно-сформированной» диаграммой Файнмана. Грамматика, о которой мы говорим, берет начало в основных законах физики, таких, как сохранение энергии, сохранение заряда, и т. д. Подобно грамматикам человеческих языков, эта грамматика рекурсивна --- в ней возможны структуры, вставленные одну в другую. Можно было бы нарисовать серию схем рекурсивных переходов, определяющих «грамматику» электромагнитных взаимодействий.

Когда голые электроны и голые фотоны вступают в подобные сложные, запутанные взаимодействия, результатом являются \emph{ренормализованные} электроны и фотоны. Таким образом, чтобы понять, каким образом реальный, физический электрон распространяется от А до В, физик должен найти что-то вроде среднего арифметического для бесконечного множества всех возможных графиков, включающих виртуальные частицы. Что это, если не дзен-буддизм, да еще в превосходной степени?\ldots{}

Таким образом, физическая --- ренормализованная --- частица включает (1) голую частицу и (2) путаницу виртуальных частиц, сложнейшим рекурсивным образом связанных между собой. Значит, существование каждой реальной частицы включает существование бесконечного множества других частиц, содержащихся в виртуальном «облаке», окружающем эту частицу, когда она движется. И, разумеется, каждая из виртуальных частиц в облаке несет с собой свое собственное виртуальное облако --- и так далее, до бесконечности.

Физики, изучающие элементарные частицы, не в состоянии справиться с подобной сложностью; чтобы понять поведение электронов и фотонов, они используют приближения, принимающие во внимание только самые простые диаграммы Файнмана. К счастью, чем сложнее диаграмма, тем меньше ее значимость. Никто не знает, каким образом можно учесть все бесконечное множество возможных диаграмм, чтобы описать поведение полностью ренормализованного физического электрона. Однако, рассматривая сотни простейших диаграмм некоторых процессов, физики смогли правильно предсказать одну из величин (так называемый g-фактор муона) с точностью до девяти знаков!

Ренормализация происходит не только среди электронов и фотонов. Физики используют идею ренормализации каждый раз, когда они пытаются понять поведение любых взаимодействующих частиц. Так что протоны и нейтроны, нейтрино, пи-мезоны, кварки --- все звери этого субатомного зверинца --- все имеют голые и ренормализованные версии в физических теориях. И из миллиардов этих пузырей в пузырях состоят все штуковины и чепуховины мира.

Копии и схожесть

Давайте теперь снова взглянем на График G. Возможно, читатель помнит, что во введении мы говорили о разных формах канонов. Каждый тип канона использовал основную тему и копировал ее с помощью изоморфизма, или сохраняющей информацию трансформации. Иногда копии получались вверх ногами, иногда задом наперед, а иногда растянутые или сокращенные\ldots{} В Графике G есть все эти типы трансформации, и даже больше. Отображение всего графика в его частях требует изменения размеров, искажения, отражения и так далее. И все же мы можем говорить о некоей основной тождественности, которую при определенном усилии можно заметить --- особенно, если ваш глаз уже натренирован на INT.

Эшер использовал идею о частях объекта, являющихся копией самого этого объекта, в своей гравюре на дереве «Рыбы и чешуйки» (Рис. 36). Конечно, эти рыбы и чешуйки схожи только в том случае, если мы рассматриваем картину на достаточно абстрактном уровне. Каждый знает, что рыбьи чешуйки --- вовсе не уменьшенные копии самой рыбы, так же как и клетки рыбы не являются ее крохотными копиями. Однако ДНК, содержащаяся в каждой из рыбьих клеток, и есть, в действительности, сильно уменьшенная «копия» самой рыбы --- таким образом, гравюра Эшера правдивее, чем кажется.

\emph{Рис. 36. М. К. Эшер. Рыбы и чешуйки. (Гравюра на дереве, 1959).}

Что именно делает всех бабочек «похожими» друг на друга? Отображение одной бабочки на другую не совпадает по клеткам; скорее, оно совпадает по функциональным органам, отчасти на макроскопическом и отчасти на микроскопическом масштабе. Вместо точных пропорций сохраняются функциональные отношения частей. Именно этот тип изоморфизма связывает между собой бабочек на гравюре Эшера «Бабочки» (рис. 37). То же верно и для более абстрактных бабочек Графика G, связанных между собой математическими отображениями одной функциональной части в другую. При этом полностью игнорируются пропорции линий, углы, и тому подобное.

\emph{Рис. 37. М. К. Эшер. «Бабочки» (гравюра на дереве, 1950).}

Перенося это исследование схожести на еще более высокий уровень абстракции, мы можем спросить: «Что же делает схожими все картины Эшера?» Было бы смешно пытаться отобразить их, часть за частью, одну на другую. Удивительно то, что ответ содержится даже в самом крохотном фрагменте картины Эшера или композиции Баха. Подобно тому, как ДНК рыбы содержится внутри самого малюсенького кусочка этой рыбы, авторская «подпись» содержится в самом маленьком кусочке его произведения. Для этого явления у нас нет другого обозначения, кроме расплывчатого и ускользающего понятия «стиля». Мы снова и снова сталкиваемся со «схожестью-внутри-различия» и с вопросом:

Когда два предмета схожи между собой?

В этой книге мы вернемся к нему еще не раз и, рассмотрев его под всевозможными углами, увидим, насколько такой простой вопрос связан с природой разума. То, что этот вопрос возник в главе, посвященной рекурсии, не случайно, рекурсия --- это область, в которой схожесть-внутри-различия играет центральную роль. Рекурсия основана на «одном и том же» событии, происходящем одновременно на нескольких различных уровнях. При этом события на разных уровнях \emph{не одинаковы} --- скорее мы находим в них какую-либо общую черту, как бы они не различались. Например, в «Маленьком гармоническом лабиринте» истории на разных уровнях весьма отличны друг от друга, и их схожесть заключается лишь в двух фактах: (1) все они --- истории и (2) во всех историях действуют Ахилл и Черепаха. В остальном, эти истории радикально отличаются одна от другой.

Программирование и рекурсия: модульность, петли, процедуры

Одна из основных способностей, необходимых в компьютерном программировании, --- это умение заметить, когда два явления схожи в широком смысле, поскольку это ведет к \emph{модуляризации} --- разбиванию задачи на несколько естественных подзадач. Представьте, например, что вам надо исполнить одну за другой серию схожих операций. Вместо того, чтобы записывать каждую из них, мы можем написать \emph{петлю} (или \emph{цикл} ), которая говорит компьютеру, что, выполнив некое множество операций, он должен вернуться к началу и повторять тот же процесс снова и снова, пока не будет выполнено некое условие. Тело петли --- определенные повторяющиеся команды --- не должно быть жестко установленным; в нем допустимы предсказуемые вариации. Примером может служить несложная проверка того, является ли число N простым. Вначале вы делите число N на 2, потом на 3, 4, 5, и так далее, до N-1. Если N не делится ни на одно из этих чисел, то N --- простое число.

Обратите внимание на то, что каждый шаг цикла здесь похож на другие, но не тождественен им. Заметьте также, что количество шагов варьируется в зависимости от N, поскольку петля постоянной длины не могла бы служить общей проверкой для простых чисел. Существуют два критерия для «прерывания» петли: (1) если N делится без остатка на какое-либо число, то петля прерывается и ответом будет «НЕТ»; (2) если мы достигли N-1 и N «выжило», не разделившись, то петля прерывается и ответом будет «ДА».

Основная идея петель такова: повторять серию родственных шагов до тех пор, пока не выполняется определенное условие. Иногда максимальное количество шагов в петле заранее известно, а иногда мы начинаем и ждем, пока петля прервется. Второй тип петель, который я называю \emph{свободными} , опасен, поскольку условия для прерывания петли могут никогда не выполниться, в результате чего компьютер застрянет на так называемом «бесконечном цикле». Разница между \emph{свободными} и \emph{ограниченными петлями} , или \emph{циклами} , является одним из важнейших понятий в теории вычислительной техники; этой теме будет посвящена глава «БлууП и ФлууП и ГлууП».

Петли могут быть также вложены одна в другую. Предположим, например, что мы хотим найти все простые числа от 1 до 5000. Для этого можно написать вторую петлю, повторяющую описанную проверку снова и снова, начиная с N=1 и кончая N=5000. Таким образом, у нашей программы будет структура «петли-в-петле». Хорошие программисты обычно составляют программы именно в этом «стиле». Подобные вложенные петли встречаются в инструкциях для сборки простых предметов, а также в таких видах деятельности, как вязание и вышивание, где маленькие петли повторяются несколько раз внутри больших петель, которые, в свою очередь, тоже повторяются несколько раз\ldots{} Результатом петли на нижнем уровне может быть всего пара стежков, в то время как петля на высшем уровне производит большую часть изделия.

В музыке также часто встречаются вложенные одна в другую петли --- например, когда гамма (маленькая петля) проигрывается несколько раз, возможно, сдвинутая при этом выше или ниже. Последние части Пятого концерта Прокофьева и Второй симфонии Рахманинова содержат длинные пассажи, в которых разные инструменты одновременно проигрывают гаммы-петли в быстром, среднем и медленном темпе --- эффект получается потрясающий. Гаммы Прокофьева идут вверх, гаммы Рахманинова --- вниз. Выбор за вами!

Более широким, чем понятие петли, является понятие \emph{подпрограммы} или процедуры, которое мы уже затронули. Группа операций при этом рассматривается как одно целое, носящее определенное название --- например, процедура УКРАШЕННОЕ СУЩЕСТВИТЕЛЬНОЕ. Как мы видели в СРП, процедуры могут вызывать одна другую по имени --- таким образом кратко описывается последовательность необходимых операций. Это --- основа модульности в программировании. Разумеется, модульность существует также в качественных системах звуковоспроизведения, в мебели, в живых клетках и в человеческом обществе --- везде, где есть иерархическая структура.

Чаще всего, нам нужна процедура, которая может варьироваться в зависимости от контекста. Такая процедура может согласовывать выбор действий с информацией, хранящейся в памяти, или же действовать согласно данному списку \emph{параметров} . Иногда используются оба эти метода. В терминах СРП выбор последовательности действий называется \emph{выбором пути} . СРП, улучшенная добавлением параметров и условий, контролирующих выбор путей внутри нее, называется Увеличенная Схема Переходов (УСП). Скорее всего, вы предпочтете УСП вместо СРП, если вам надо получить осмысленные русские предложения на основе набора слов; при этом базой служит грамматика, выраженная в УСП. Параметры и условия позволят вам ввести определенные семантические ограничения, запрещающие случайные соединения типа «неблагодарная закуска». Однако мы еще вернемся к этой теме в главе XVIII.

Рекурсия в шахматных программах

Классическим примером рекурсивной процедуры с параметрами может служить программа для выбора лучших ходов в шахматной партии. Лучшим ходом можно, по-видимому, считать тот, что оставляет противника в наихудшей ситуации. Таким образом, проверка лучшего хода весьма проста: представьте себе, что вы сделали ход\ldots{} а теперь мысленно переверните доску и оцените позицию с точки зрения вашего противника. Но каким образом оценивает позицию ваш противник? Он ищет \emph{свой} лучший ход. Это значит, что он мысленно перебирает все возможные варианты и оценивает их, как ему кажется, с \emph{вашей} точки зрения, надеясь, что вы найдете их опасными для себя. Обратите внимание, что мы определили «лучший ход» рекурсивно: то, что лучше для одного противника, хуже для другого. Рекурсивная процедура, занятая поисками лучшего хода, пробует один ход за другим и каждый раз \emph{вызывает саму себя в качестве противника} ! В этой роли она пробует следующий ход, и вызывает себя в качестве противника противника --- то есть, снова себя самой.

Эта рекурсия может спуститься на несколько уровней --- но рано или поздно она должна достичь дна! Как можно оценить позицию на доске, \emph{не заглядывая вперед} ? Для этого существуют несколько полезных критериев, таких как, например, количество фигур с обеих сторон, количество и тип фигур, находящихся под атакой, контроль над центром, и так далее. Оценивая позицию таким образом в начале, «на дне», рекурсивный генератор ходов может вернуться наверх и оценить позицию с точки зрения каждого отдельного хода. Таким образом, один из параметров в этом самовызове должен определить, на сколько ходов вперед просчитывать. Самый внешний вызов процедуры будет использовать некое установленное извне значение для этого параметра. После этого, каждый раз, когда процедура будет вызывать саму себя, параметр, указывающий на какое количество ходов вперед надо просчитывать каждый вариант, будет сокращаться на единицу. Таким образом, когда параметр достигнет нуля, процедура последует по другому пути и обратится к не-рекурсивной оценке.

В программах подобного «игрового» типа, каждый анализируемый ход порождает «дерево анализа вариантов», где сам ход является стволом, возможные ответы --- основными ветвями, контр-ответы --- ветвями потоньше, и так далее. На рис. 38 я показал простое дерево анализа, иллюстрирующее начало игры в крестики-нолики. Существуют способы, позволяющие избежать анализа каждой ветви до конца. В искусстве выращивания шахматных деревьев лидируют люди, а не компьютеры. Известно, что лучшие игроки просчитывают варианты на относительно небольшое число ходов, в сравнении с компьютером --- и играют при этом намного лучше! В начале развития компьютерных шахмат считалось, что не пройдет и десяти лет, как компьютер (или программа) станет чемпионом мира. Однако, эта цель не достигнута и по сей день\ldots{} Это может служить еще одним подтверждением очень рекурсивного

\emph{Закона Хофштадтера:}

\emph{На любое дело требуется больше времени, чем казалось в начале, даже если вы учитывали при этом закон Хофштадтера.}

\emph{Рис. 38.~Разветвляющееся дерево ходов и контрходов в начале игры в крестики и нолики.}

Рекурсия и непредсказуемость

В чем связь между рекурсивными множествами предыдущей главы и рекурсивными процедурами этой главы? Ответ на этот вопрос затрагивает понятие \emph{рекурсивно перечислимых множеств.} Чтобы множество было р.п., оно должно быть получено на основе начальных точек (аксиом) при помощи повторного применения правил вывода. Таким образом, множество растет и растет, и каждый новый элемент так или иначе составлен из предыдущих --- что-то вроде «математического снежного кома». Но ведь это и есть основа рекурсии: вместо явного определения нечто определяется через более простые версии себя самого. Числа Фибоначчи и Лукаса --- превосходные примеры р.п. множеств, вырастающих из двух данных элементов до бесконечности путем применения рекурсивного правила. По соглашению, множество, чье дополнение также р.п., называется «рекурсивным».

Рекурсивное перечисление --- это процесс, в котором новые элементы вырастают из старых при помощи определенных правил. В подобных процессах немало сюрпризов --- например, непредсказуемость ряда Q. Может показаться, что подобные рекурсивно определенные ряды обладают некой врожденной возрастающей сложностью поведения --- чем дальше вы идете, тем менее предсказуемы они становятся. Развивая эту идею, мы приходим к мысли, что достаточно сложная рекурсивная система может быть настолько мощной, что она в конце концов вырвется за пределы любой установленной заранее схемы. Но не это ли одно из основных свойств разума? Вместо того, чтобы рассматривать программы, просто \emph{вызывающие} самих себя, нельзя ли попытаться создать изменяющиеся программы --- программы, действующие на другие программы, улучшая, расширяя, обобщая и налаживая их? В самом сердце разума, возможно, лежит именно такой тип «переплетающейся рекурсивности».


% % \subsubsection{Канон с интервальны увеличением}
% \subsubsection{Канон с интервальны увеличением}

\emph{Ахилл и Черепаха только что доели превосходный ужин на двоих в лучшем китайском ресторане города.}

\emph{Ахилл} : Здорово вы управляетесь с палочками, г-жа Ч.

\emph{Черепаха} : Приходится --- я с детства питаю слабость к восточной кухню. Как насчет вас, Ахилл --- вам понравилось?

\emph{Ахилл} : Еще как! Я никогда раньше не пробовал китайской еды, и сегодняшний ужин был приятным знакомством с ней. А сейчас, если вы не торопитесь мы можем еще немного посидеть и поболтать.

\emph{Черепаха} : Что ж, с удовольствием побеседую с вами, пока мы пьем чай. Официант! (Подходит официант.) Пожалуйста, принесите наш счет. И еще немного чая! (Официант торопливо уходит.)

\emph{Ахилл} : Вы можете понимать больше меня в китайской кухне, г- жа Ч, но могу поспорить, что о японской поэзии я знаю побольше вас. Читали ли вы когда-нибудь хайку?

\emph{Черепаха} : Боюсь, что нет. Что это такое?

\emph{Ахилл} : Хайку --- это японская поэма, в которой семнадцать слогов. Правильнее сказать, что это мини-поэма, наводящая на размышление так,же, как благоуханный розовый лепесток или покрытые росой кувшинки в пруду. Обычно хайку состоит из группы пяти слогов, затем --- семи, и затем --- снова пяти.

\emph{Черепаха} : Такая краткость --- всего семнадцать слогов --- но где же здесь смысл?

\emph{Ахилл} : Смысл живет также в голове читателя --- не только в хайку.

\emph{Черепаха} : Гм-м-м\ldots{} Это утверждение наводит на размышления.

\emph{(Подходит официант со счетом, чайничком, полным чая, и парой печений «с сюрпризом» --- бумажкой, на которой написана судьба едока.)}

Премного благодарна. Еще чайку не желаете, Ахилл?

\emph{Ахилл} : Пожалуй. Эти печеньица выглядят весьма аппетитно. (Берет печенье, откусывает кусочек и начинает жевать.) Эй --- что эта за штуковина тут внутри? Клочок бумаги?

\emph{Черепаха} : Это ваша судьба, Ахилл. Во многих китайских ресторанах вместе со счетом подают печенья с судьбой-сюрпризом, чтобы смягчить удар. Завсегдатаи китайских ресторанов обычно считают их не за печенья, а за посланцев судьбы. К несчастью, вы, кажется, проглотили кусочек своей судьбы. Что там написано, на оставшемся клочке?

~\emph{Ахилл} : Странно --- все буквы сгрудились в кучу, нет никакого деления на слова. Может быть, это надо расшифровать? О, я понял если расставить промежутки там, где надо, получится: «НИС КЛАДУН ИЛ АДУ». Поистине, адская бессмыслица! Может быть, это что-то вроде хайку, от которого я отъел большинство слогов.

\emph{Черепаха} : В таком случае, ваша судьба теперь всего лишь 6/17 хайку. Веселенькие ассоциации все это вызывает. Колдуны, болота, черти, клады\ldots{} Что и говорить, картинка унилая\ldots{} унылая, я имею в виду. Это звучит как комментарий к новой форме искусства --- 6/17 хайку. Можно мне взглянуть?

\emph{Ахилл (протягивая Черепахе узкий клочок бумаги)} : Конечно.

\emph{Черепаха} : Но, Ахилл, в моей «расшифровке» получается нечто совершенно другое! Это вовсе не 6/17 хайку, а шестисложное послание --- и вот что в нем написано «НИ СКЛАДУ НИ ЛАДУ». Поистине, глубокий комментарий к этой новой форме искусства --- 6/17 хайку!

\emph{Ахилл} : Вы правы. Удивительно, что это послание содержит комментарий о самом себе!

\emph{Черепаха} : Я только передвинула рамку чтения на единицу --- сдвинула все промежутки между словами на один интервал.

\emph{Ахилл} : Посмотрим, какая судьба выпала сегодня вам.

\emph{Черепаха (ловко разламывая печенье, читает)} : «Судьбу едока не печенье содержит, а его рука».

\emph{Ахилл} : Ваша «судьба» тоже хайку, г-жа Черепаха --- по крайней мере, в ней семнадцать слогов. 5-7-5.

\emph{Черепаха} : Потрясающе! Я бы сама этого ни за что не заметила, Ахилл --- такие вещи только вы подмечаете. То, что меня больше всего удивило, это сам текст послания; разумеется, его можно интерпретировать по-разному.

\emph{Ахилл} : Наверное, мы все интерпретируем послания по-своему, когда с ними сталкиваемся\ldots{} (Лениво рассматривает чаинки на дне чашки.)

\emph{Черепаха} : Подлить вам чаю?

\emph{Ахилл} : Да, спасибо. Кстати, как поживает ваш товарищ, старый Краб? Я частенько о нем вспоминаю, с тех пор, как вы рассказали мне о его диковиной патефонной войне.

\emph{Черепаха} : Я ему о вас кое-что рассказала, и ему тоже не терпится с вами встретиться. У него все в порядке;на днях он приобрел новую штуковину из серии проигрывателей,~какой-то странный проигрыватель-автомат.

\emph{Ахилл} : Расскажите-ка мне об этом поподробнее. Обожаю эти автоматы --- кругом разноцветные огоньки, и когда опустишь монетку, машина играет глупые песни, которые так и окунают тебя в старое доброе прошлое\ldots{}

\emph{Черепаха} : Этот проигрыватель слишком велик, чтобы держать его дома, и Краб построил для него во дворе специальный навес.

\emph{Ахилл} : Не представляю себе, почему он такой большой? Может, в нем огромная коллекция пластинок?

\emph{Черепаха} : На самом деле, в нем всего одна запись.

\emph{Ахилл} : Что? Проигрыватель-автомат с одной пластинкой? Это уже само по себе противоречие! Почему же он так велик? Может, его единственная пластинка --- гигант двадцати футов в диаметре?

\emph{Черепаха} : Да нет, пластинка самая обыкновенная.

\emph{Ахилл} : Ах, г-жа Черепаха, не иначе как вы надо мной смеетесь. Ну скажите на милость, что это за автомат с единственной песней?

\emph{Черепаха} : Кто сказал хотя бы слово о единственной песне?

\emph{Ахилл} : Любой проигрыватель-автомат, с которым я когда-либо сталкивался, подчинялся фундаментальной аксиоме этих аппаратов: «одна пластинка, одна песня.»

\emph{Черепаха} : Этот автомат не таков, Ахилл. Единственная пластинка в нем расположена вертикально, и за ней находится небольшая, но сложная система рельсов, на которых подвешены проигрыватели. Когда вы нажимаете на пару кнопок, скажем, В-1, вы выбираете один из проигрывателей. Это пускает в действие механизм, и проигрыватель со скрипом отправляется по ржавым рельсам. Вскоре он прибывает к краю пластинки, и --- щелк! --- устанавливается в нужную позицию.

\emph{Ахилл} : И тогда пластинка начинает вращаться, и раздается музыка, правда?

\emph{Черепаха} : Не совсем. Пластинка остается неподвижной --- вращается сам проигрыватель.

\emph{Ахилл} : Я мог бы догадаться. Но каким же образом, если у вас только одна пластинка, вы можете выудить из этой сумасшедшей конструкции больше одной песни?

\emph{Черепаха} : Я и сама спрашивала Краба об этом. Он посоветовал мне попробовать самой. Я нашла в кармане монетку (ее хватало на три песни), засунула ее в щель и нажала наугад: В-1, С-3, и V-10.

\emph{Ахилл} : Значит, патефон В-1 поехал по рельсам, подкатился к вертикальной пластинке и стал вращаться?

\emph{Черепаха} : Точно. Получилась довольно приятная музыка, основанная на знаменитой старой мелодии В-А-С-H, которую, я полагаю, вы еще помните\ldots{}

\emph{Ахилл} : Могу ли я ее забыть?

\emph{Черепаха} : Это был патефон В-1. Когда мелодия закончилась, он отъехал назад, чтобы дать место патефону С-3.

\emph{Ахилл} : Неужели С-3 заиграл другую мелодию?

\emph{Черепаха} : Именно так.

\emph{Ахилл} : А, понимаю. Он проиграл другую сторону пластинки, или, может быть, другую полосу на этой стороне.

\emph{Черепаха} : Нет, на этой пластинке дорожки только с одной стороны и на ней только одна полоса.

\emph{Ахилл} : Ничего не понимаю. Получить разные песни из одной записи НЕВОЗМОЖНО!

\emph{Черепаха} : Я тоже так думала, пока не увидела проигрыватель м-ра Краба.

\emph{Ахилл} : Как звучала эта вторая песня?

\emph{Черепаха} : Это-то как раз интересно: она была основана на мелодии C-A-G-E.

\emph{Ахилл} : Но это совершенно иная мелодия!

\emph{Черепаха} : Верно.

\emph{Ахилл} : Кажется, Джон Кэйдж --- это композитор, создатель авангардистской музыки? Мне кажется, я читал о нем в одной из моих книг хайку.

\emph{Черепаха} : Точно. Многие его творения довольно известны, например, 4'33'' --- трехчастная пьеса, состоящая из безмолвий разной длины. Она необыкновенно выразительна --- если у вас есть вкус к подобным вещам.

\emph{Ахилл} : Что ж, если бы я находился в шумном ресторане, я с удовольствием поставил бы 4'33" Кэйджа на музыкальном автомате. Это могло бы быть некоторым облегчением!

\emph{Черепаха} : Правильно --- кому хочется слушать звон тарелок и стук ножей? Эта пьеса пришлась бы весьма кстати еще в одном месте, в Павильоне~Гигантских Кошек, во время кормления.

\emph{Ахилл} : Вы намекаете на то, что Кэйджу место в зверинце? Что ж, если учесть, что его фамилия в переводе с английского значит «клетка»\ldots{} Но вернемся к крабьему музыкальному автомату --- я ничего не понимаю. Как могут на одной и той же записи быть сразу В-А-С-H и C-A-G-E?

\emph{Черепаха} : Если вы посмотрите повнимательней, Ахилл, вы можете подметить, что между ними есть некоторая связь. Вот, взгляните: что у вас получится, если вы последовательно запишете интервалы мелодии В-А-С-H?

\emph{Ахилл} : Ну-ка, посмотрим\ldots{} Сначала она понижается на полтона, от В до А (я имею в виду немецкое В); затем поднимается на три полутона до С, и, наконец, опускается на полутон, до H. Получается следующая схема:

-1, +3, -1

\emph{Черепаха} : Совершенно верно. А как насчет C-A-G-E?

\emph{Ахилл} : Здесь мелодия сначала идет на три полутона вниз, потом поднимается на десять полутонов, и снова опускается на три полутона. Получается:

-3, +10, -3

Очень похоже на первую мелодию, правда?

\emph{Черепаха} : Действительно, похоже. В некотором смысле, у этих двух мелодий совершенно одинаковый «скелет». Вы можете получить C-A-G-E из~В-А-С-H, умножив все интервалы на 3,5 и беря ближайшее целое число.

\emph{Ахилл} : Вот это да! Это значит, что на звуковых дорожках записан только некий основной код, который разные проигрыватели интерпретируют по-разному?

\emph{Черепаха} : Я не уверена --- этот уклончивый Краб не посвятил меня во все детали. Но мне удалось услышать третью песню, произведенную на проигрывателе В-10.

\emph{Ахилл} : И как она звучала?

\emph{Черепаха} : Ее мелодия состояла из огромных интервалов: В-С-А-H.

Схема в полутонах была такая:

-10, +33, -10

Эта мелодия получается из C-A-G-E, если снова умножить интервалы на 3,3 и округлить результаты до ближайшего целого числа.

\emph{Ахилл} : Есть ли какое-то название у такого умножения интервалов?

\emph{Черепаха} : Его можно назвать «интервальным увеличением». Оно похоже на прием ритмического увеличения темы канона. При этом длительность всех нот мелодии умножается на какое-либо постоянное число. В результате мелодия замедляется. Здесь же интересным образом расширяется диапазон мелодии.

\emph{Ахилл} : Удивительно. Так что все три мелодии, что вы услышали, были интервальными увеличениями одной и той же схемы звуковых дорожек?

\emph{Черепаха} : Таково мое заключение.

\emph{Ахилл} : Забавно, когда мы увеличиваем В-А-С-H, у нас получается C-A-G-E, a когда мы опять увеличиваем C-A-G-E, то снова получаем В-А-С-H, только теперь он весь перевернут, словно В-А-С-H разнервничался, проходя через промежуточный этап C-A-G-E.

\emph{Черепаха} : Поистине, глубокий комментарий к этой новой форме искусства --- музыке Кэйджа.


% % \subsubsection{ГЛАВА VI: Местонахождени значения}
% \subsubsection{ГЛАВА VI: Местонахождени значения}

Когда одна и та же вещь не не похожа сама на себя?

В ПОСЛЕДНЕЙ ГЛАВЕ, мы сформулировали вопрос. «Когда две вещи похожи друг на друга?» В этой главе мы рассмотрим оборотную сторону этого вопроса. «Когда одна и та же вещь не похожа сама на себя?» Мы попытаемся выяснить, присуще ли значение самому сообщению или же оно всегда порождается взаимодействием разума (или механизма) с этим сообщением --- как в предыдущем Диалоге. В последнем случае нельзя было бы сказать ни что значение находится в каком-то одном месте, ни что сообщение имеет некое универсальное или объективное значение --- поскольку каждый наблюдатель привносил бы в каждое сообщение свое собственное значение. Но в первом случае значение имело бы постоянное место и было бы универсально. В этой главе я постараюсь показать универсальность по крайней мере некоторых сообщений, не утверждая этого для всех сообщений вообще. Как мы увидим, идея «объективности значения» некоего сообщения интересным образом соотносится с тем, насколько легко может быть описан разум.

Носители информации и обнаружители информации

Начну с моего любимого примера отношения между пластинками, музыкой и проигрывателями. Мы привыкли к мысли о том, что пластинка содержит ту же информацию, что и музыкальное произведение, так как существуют проигрыватели, которые способны «читать» записи и превращать структуру звуковых дорожек в звуки. Иными словами, между звуковыми дорожками и звуками существует изоморфизм, проигрыватель --- механизм, осуществляющий этот изоморфизм физически. Таким образом, естественно думать о пластинках как о \emph{носителях информации} и о проигрывателях, как об \emph{обнаружителях информации} . Другой пример этих понятий --- система \textbf{pr} . Там «носителями информации» являлись теоремы, а ее «обнаружителем» была интерпретация, такая прозрачная, что для извлечения информации из теорем нам не понадобились никакие электронные машины.

Эти два примера наводят на мысль, что изоморфизмы и декодирующие механизмы (то есть, обнаружители информации) всего лишь «проявляют» информацию, уже имеющуюся в структуре сообщения и только ждущую своего часа, чтобы быть извлеченной. Отсюда следует, что в любой структуре есть некая информация, которую \emph{возможно} извлечь, так же как и информация, которую извлечь \emph{нельзя} . Но что именно означает фраза «извлечь информацию»? С какой силой нам позволено ее «вытягивать»? В некоторых случаях, приложив достаточно усилий, удается извлечь очень глубоко запрятанную информацию. На самом деле, извлечение информации может потребовать настолько сложных операций, что вам может показаться, что вы вкладываете больше информации, чем извлекаете.

Генотип и фенотип

Рассмотрим пример генетической информации, содержащейся в двойной спирали дезоксирибонуклеиновой кислоты (ДНК). Молекула ДНК --- генотип --- превращается в физический организм --- фенотип --- путем весьма сложного процесса, включающего выработку белков, воспроизведение ДНК, воспроизведение клеток, постепенное различение типов клеток, и т. д. Процесс превращения генотипа в фенотип --- эпигенез --- представляет собой пример наиболее запутанной из запутанных рекурсий; мы уделим ему все внимание в главе XVI. Эпигенез зависит от множества сложнейших химических реакций и петель обратной связи. К тому времени, когда создание организма закончено, его физические характеристики не имеют ни малейшего сходства с его генотипом.

Тем не менее, считается, что физическая структура организма восходит к его ДНК --- и только к ней. Впервые это подтвердили эксперименты Освальда Авери, проведенные в 1944 году; с тех пор собрано много убедительных данных в пользу этой идеи. Эксперименты Авери показали, что из множества молекул только ДНК обладает свойством передавать наследственные качества. Можно изменить другие молекулы в организме, например, белки, но эти изменения не будут переданы последующим поколениям. Однако когда меняется ДНК, изменения наследуются всеми последующими поколениями. Эти эксперименты доказали, что единственный способ изменить инструкции по построению нового организма заключается в изменении его ДНК; из этого, в свою очередь, следует, что эти инструкции должны быть закодированы где-то в структуре ДНК.

Изоморфизмы экзотические и прозаические

По-видимому, приходится заключить, что, структура ДНК содержит информацию о структуре фенотипа --- иными словами, эти структуры \emph{изоморфны} . Это пример \emph{экзотического} изоморфизма; я имею в виду, что разделить фенотип и генотип на «части», которые могут быть отображены друг в друге --- весьма нетривиальная задача. Напротив, \emph{прозаическим} изоморфизмом являлся бы такой, в котором части двух структур отображались бы друг в друге без труда. Пример тому --- изоморфизм между пластинкой и музыкальным произведением ---~мы знаем, что для каждого звука в произведении существует его точное «изображение» в структуре звуковых дорожек и что, если потребуется, его можно всегда аккуратно указать. Другой пример прозаического изоморфизма --- изоморфизм между Графиком G и любой из составляющих его бабочек.

Изоморфизм между структурой ДНК и структурой фенотипа никак нельзя назвать прозаическим --- физически осуществляющий его механизм необыкновенно сложен. Например, было бы весьма трудно найти ту часть ДНК, которая в ответе за форму вашего носа или кончиков пальцев. Это немного похоже на попытку найти ту \emph{единственную} ноту, которая создает эмоциональный настрой музыкального произведения в целом. Конечно, такой ноты не существует, поскольку эмоциональное значение создается на гораздо высшем уровне --- не единственной нотой, а большими «кусками» произведения. Кстати, эти «куски»~~не обязательно состоят из нот, идущих подряд --- могут существовать также отдельные фрагменты, которые, взятые вместе, создают определенный эмоциональный настрой.

Подобно этому, «генетическое значение» --- то есть, информация о структуре фенотипа --- рассеяно по нескольким крохотным частям молекулы ДНК. Пока никто еще не понимает этого «языка» (Внимание понять этот «язык» --- вовсе не то же самое, что разгадать Генетический Код, последнее произошло в начале шестидесятых годов. Генетический Код объясняет, как «перевести» небольшие порции ДНК в различные аминокислоты. Таким образом, разгадка Генетического Кода сравнима с нахождением фонетических значений букв иностранного алфавита --- при этом мы еще не знаем ни грамматики данного языка, ни значений его слов. Разгадка Генетического Кода явилась важнейшим шагом на пути к извлечению значения из ДНК, но это всего лишь первый шаг по длинной дороге, лежащей перед нами )

Проигрыватели-автоматы и пусковые механизмы

Генетическая информация, содержащаяся в ДНК, --- это один из лучших примеров неявного значения Чтобы превратить генотип в фенотип, требуются механизмы гораздо более сложные, чем сам генотип. Некоторые части генотипа служат пусковыми механизмами для этих процессов. Проигрыватель-автомат --- обыкновенный, не крабий --- хорошо поясняет эту идею: пара кнопок определяет серию действий, которые предстоит выполнить механизму. В этом смысле можно сказать, что кнопки «пустили в ход» песню, играемую на проигрывателе. В процессе, превращающем генотип в фенотип, клеточные «проигрыватели-автоматы» приводятся в действие с помощью «кнопок», каковыми являются короткие отрезки спирали ДНК и полученные таким образом «песни» часто служат кирпичиками для построения дальнейших «проигрывателей». Это можно сравнить с настоящими проигрывателями-автоматами, которые вместо лирических песенок проигрывали бы песни, объясняющие, как построить более сложные проигрыватели. Части ДНК запускают создание белков, эти белки пускают в ход сотни новых реакций, которые, в свою очередь, запускают операцию воспроизводства, которая в несколько этапов повторяет структуру ДНК --- и так далее, и тому подобное. Это дает понятие о том, насколько рекурсивен этот процесс. Конечным результатом работы этого много раз запущенного пускового механизма является фенотип --- индивид. Мы говорим, что фенотип --- это раскрытие информации, содержавшейся в ДНК в скрытом состоянии (Термин «раскрытие» в этом контексте принадлежит Жаку Моноду, одному из лучших и оригинальнейших специалистов двадцатого века по молекулярной биологии). Никто не сказал бы, что песня, выходящая из динамиков музыкального автомата --- это раскрытие информации, содержавшейся в паре нажатых нами кнопок, они послужили всего лишь \emph{триггером} для пуска в действие содержащих информацию механизмов самого автомата. С другой стороны, естественно говорить об извлечении музыки из звукозаписи как о «раскрытии» содержащейся в данной записи информации по нескольким причинам:

(1) музыка не запрятана в механизмах самого проигрывателя;

(2) возможно сопоставить части ввода (запись) с частями вывода (музыка) с любой степенью аккуратности;

(3) можно проигрывать на одном и том же проигрывателе разные записи и получать различные мелодии;

(4) запись и проигрыватель легко отделить друг от друга.

Совершенно другим вопросом является тот, присуще ли значение частям \emph{разбитой} пластинки. Края разбитой пластинки можно сложить вместе и таким образом восстановить значение --- но вопрос здесь гораздо сложнее. Есть ли собственное значение у неразборчивого телефонного разговора?\ldots{} Спектр степеней собственных значений весьма широк. Интересно попытаться найти в этом спектре место для эпигенеза. Когда организм развивается, можем ли мы сказать, что информация извлекается из ДНК? Там ли находится вся информация о структуре организма?

ДНК и необходимость химического контекста

Благодаря экспериментам, подобным экспериментам Авери, в определенном смысле кажется, что ответ на этот вопрос положителен. Но в другом смысле кажется, что ответом будет «нет», поскольку процесс извлечения информации здесь в большой степени зависит от сложнейших клеточных химических процессов, которые не закодированы в самой ДНК. ДНК «надеется» на то, что они произойдут, но, по всей видимости, не содержит никакого кода, который вызывал бы эти процессы. Таким образом, у нас имеются два противоречивых взгляда на природу информации в генотипе. Один из них утверждает, что, поскольку такое большое количество информации содержится вне ДНК, мы должны рассматривать ДНК не более как очень сложный набор пусковых механизмов, что-то вроде кнопок на музыкальном автомате; другой взгляд --- что вся \emph{информация} содержится в ДНК, только в очень неявной форме.

Можно подумать, что эти две точки зрения --- лишь разные формы выражения одной и той же идеи; однако это вовсе не обязательно верно. Одна точка зрения утверждает, что ДНК почти бесполезна вне контекста; другая --- что даже вне контекста структура молекулы ДНК живого существа имеет настолько \emph{убедительную внутреннюю логику} , что извлечь из нее информацию возможно в любом случае. Выражая ту же мысль короче, первый взгляд утверждает, что для выяснения значения ДНК необходим \emph{химический} контекст; другая точка зрения утверждает, что для раскрытия присущего ДНК значения необходим только разум.

Фантастический НЛО

Чтобы взглянуть на этот спорный вопрос в перспективе, вообразим себе странное гипотетическое событие. Запись фа-минорной сонаты Баха для скрипки и клавира в исполнении Давида Ойстраха и Льва Оборина отправлена в пространство в спутнике. Затем запись выброшена из спутника и направлена за пределы солнечной системы, а, может быть, и всей галактики --- просто пластмассовый диск с дыркой в середине, крутящийся в межгалактическом пространстве. Безусловно, запись потеряла свой контекст. Каково теперь ее значение?

Если бы иная цивилизация нашла эту пластинку, она была бы удивлена ее формой и весьма заинтересована ее назначением. Форма, действуя как пусковой механизм, сообщила бы им что, возможно, речь идет об искусственно сделанном предмете, и что этот предмет, может быть, несет определенную информацию. Эта мысль, сообщенная или «пущенная в действие» самой пластинкой, \emph{создает теперь новый контекст} , в котором пластинка будет рассматриваться в дальнейшем. Сама расшифровка может отнять гораздо больше времени --- но нам об этом трудно судить. Можно представить себе, что если бы подобная запись попала на землю во времена Баха, никто не знал бы, что с ней делать, и, скорее всего, она так и осталась бы нерасшифрованной. Однако это не уменьшает нашей уверенности в том, что информация \emph{была там} изначально; просто мы знаем, что в то время человеческие знания о хранении, трансформации и извлечении информации были недостаточны.

Уровни понимания сообщения

В наши дни идея расшифровки распространена весьма широко; дешифровка составляет значительную часть работы астрономов, лингвистов, археологов, военных специалистов, и так далее. Существует предположение, что мы плаваем в море радиопосланий из других цивилизаций --- посланий, которые мы пока еще не умеем расшифровывать. Технике расшифровки подобных посланий было посвящено немало серьезных исследований. Одной из главных проблем --- может быть, даже самой трудной --- является следующая: «Как распознать шифрованное сообщение и поместить его в определенный контекст?» Посылка пластинки кажется простым решением; ее физическая структура сразу привлекает внимание, и у нас есть разумная надежда на то, что достаточно развитый интеллект попытается найти спрятанную в ней информацию. Однако по технологическим причинам пока не представляется возможным посылать твердые объекты в другие солнечные системы. Это, разумеется, не мешает нам размышлять на эту тему.

Теперь представьте себе, что наша гипотетическая цивилизация догадалась, что для расшифровки записи нужен механизм, превращающий структуру звуковых дорожек в звуки. Это все еще весьма далеко от настоящей расшифровки. Что же было бы удачной расшифровкой записи? Ясно, что для этого цивилизация должна найти смысл в звуках. Простое производство звуков было бы бесполезным, если бы оно не вызывало соответствующей реакции в мозгах (если можно так выразиться) у инопланетян. А что мы имеем в виду под «соответствующей реакцией»? Пуск в действие механизмов, вызывающих в их мозгах такой же эмоциональный настрой, какой возникает при прослушивании этой пьесы у нас. На самом деле, весь звуковоспроизводящий процесс можно было бы опустить, если бы инопланетянам удалось использовать пластинку как-то иначе, тем не менее получив при этом нужный эмоциональный эффект. (Если бы мы, земляне, умели бы последовательно активировать нужные механизмы в нашем мозгу так, как это делает музыка, возможно, что мы предпочли бы обходиться без звуков. Однако кажется маловероятным, что это может быть достигнуто без помощи слуха. Глухие композиторы --- Бетховен, Дворжак, Форе --- или музыканты, способные «слышать» музыку, глядя на ноты, не являются опровержением, так как их умение основано на долгом предварительном опыте прямого слушания музыки.)

Здесь все становится весьма расплывчато и неясно. Испытывают ли вообще инопланетяне какие-либо эмоции? Могут ли их эмоции --- предполагая, что они у них есть --- быть сравнимы с нашими? Если их эмоции схожи с нашими, группируются ли они, подобно нашим? Поймут ли они такие комбинации, как трагическая красота или мужественное страдание? Если окажется, что существа других миров разделяют с нами познавательные структуры до такой степени, что даже их эмоции совпадают с нашими, то, в некотором смысле, запись никогда не может оказаться полностью вне контекста --- контекст оказывается частью схемы самой природы. Если дело действительно обстоит таким образом, то вполне возможно, что наша бродяга-пластинка, если не сломается по дороге, попадет в конце концов к какому-нибудь существу или группе существ и будет удачно расшифрована.

Воображаемый космопейзаж

Рассуждая о значении молекулы ДНК, я употребил выражение «убедительная внутренняя логика»; это кажется мне ключевым понятием. В качестве иллюстрации возьмем нашу гипотетическую посылку пластинки в пространство, на этот раз заменив Баха «Воображаемым пейзажем \#4» Джона Кейджа. Эта пьеса --- классический пример «случайной» музыки, в которой вместо того, чтобы пытаться сообщить определенные эмоции, звукосочетания выбираются путем различных случайных процессов. В этом случае, двадцать четыре исполнителя держатся за двадцать четыре ручки двенадцати радио. Во время пьесы они крутят эти ручки кто во что горазд, так что настройка и громкость каждого радио все время меняются. Совокупность всех этих звуков и есть пьеса Кейджа. Композитор выразил свое намерение лаконично: «Позволим звукам быть самими собой, вместо того, чтобы заставлять их выражать придуманные человеком теории о его чувствах.»

Вообразите теперь, что это и есть пьеса, посланная в пространство на пластинке. Инопланетянам было бы весьма нелегко, если не невозможно, разгадать значение такого объекта. Скорее всего, они были бы удивлены противоречием между «рамкой» послания, говорящей: «Я --- сообщение; расшифруйте меня», и хаосом его внутренней структуры. В этой пьесе Кейджа есть несколько кусочков, за которые можно ухватиться при расшифровке. С другой стороны, в пьесе Баха есть множество структур, структур структур и так далее. Мы не можем знать, являются ли эти структуры универсально привлекательными. Мы не знаем достаточно о природе разуме, эмоций или музыки, чтобы судить, настолько ли привлекательна внутренняя логика пьес Баха, что их значение способно пересечь галактики.

Однако вопрос здесь не в том, достаточно ли внутренней логики в пьесах Баха; вопрос в том, достаточно ли в любом отдельно взятом сообщении внутренней логики для того, чтобы его контекст был восстановлен автоматически при контакте с любой достаточно развитой цивилизацией. Если бы какое-либо сообщение обладало такой внутренней логикой, то разумно было бы сказать, что значение такого сообщения является его внутренним свойством.

Героические расшифровыватели

Еще один блестящий пример подобных идей --- расшифровка старинных текстов, написанных на неизвестных языках и алфавитах. Интуиция говорит нам, что в подобных сообщениях есть смысл, независимо от того, удается ли нам этот смысл извлечь. Это чувство так же сильно, как и наша вера в то, что в газете, написанной по-китайски, есть внутренний смысл, даже если мы и не понимаем по-китайски ни слова. После того, как письменность или язык текста оказываются расшифрованными, никто не сомневается, что значение лежит в самом тексте, а не в методах расшифровки --- так же как музыка «живет» в \emph{записи} , а не в проигрывателе. Именно так мы и определяем декодирующие механизмы, они не \emph{добавляют} никакого значения к знакам или предметам, которые служат им вводом, они лишь \emph{выявляют} значение, присущее этим знакам или предметам. Музыкальный автомат не является декодирующим механизмом, поскольку он не выявляет никакого значения вводных символов, напротив, он привносит значение, лежащее внутри него самого

Расшифровка старинного текста может потребовать многолетней работы коллективов ученых, пользующихся материалами множества библиотек всего мира\ldots{} Не добавляет ли и этот процесс определенную информацию? Насколько внутренним является значение самого текста, если его расшифровка требует таких гигантских усилий? Вкладывается ли при этом значение в текст, или оно уже в нем находилось? Моя интуиция говорит, что значение там уже было и что весь грандиозный труд по расшифровке не привнес в текст ничего нового. Это чувство основано на факте, что расшифровка была неизбежна, если не этой группой ученых, то другой, и если не теперь, так позже --- и что результат был бы одним и тем же.

Значение содержится в самом тексте именно потому, что его воздействие на разум предсказуемо. В итоге мы можем утверждать, что значение является частью самого предмета постольку, поскольку этот предмет воздействует на разум определенным предсказуемым способом.

На рис. 39 показан камень Розетты, одно из важнейших исторических открытий. Он явился ключом к расшифровке египетских иероглифов, поскольку он содержит параллельный текст, написанный тремя древними письменностями: иероглифической, демотической и греческой. Надпись на базальтовой пластине была впервые расшифрована Жаном Франсуа Шамполионом, «отцом египтологии»; это декрет Мемфисского собрания священников в поддержку Птолемея V Эпифания.

\emph{Рис. 39. Камень Розетты (С разрешения Британского музея )}

Три уровня любого сообщения

В этих примерах расшифровки помещенных вне контекста сообщений можно ясно различить три уровня информации: (1) \emph{сообщение-рамка} ; (2) \emph{внешнее сообщение} ; (3) \emph{внутреннее сообщение} . Мы лучше всего знакомы с (3) --- внутренним сообщением. Оно передается явно, как эмоциональные ощущения в музыке, фенотип в генетике, описание династий и ритуалов древних цивилизаций в старинных надписях, и так далее.

\emph{Понять внутреннее сообщение означает извлечь значение, вложенное в сообщение его отправителем} .

Сообщение-рамка гласит: «Я --- сообщение; расшифруйте меня, если сможете!». Эта информация содержится в структурном аспекте предмета --- носителя сообщения.

\emph{Понять сообщение-рамку означает признать необходимость декодирующего механизма} .

Если мы видим сообщение-рамку, то наше внимание направляется на уровень (2) --- внешнее сообщение. Это информация, явно переданная с помощью схем символов и общей структуры сообщения; она сообщает, как расшифровать внутреннее сообщение.

\emph{Понять внешнее сообщение означает построить --- или знать, как построить --- правильный декодирующий механизм для внутреннего сообщения} .

Сообщение внешнего уровня всегда неявно, поскольку отправитель послания не может гарантировать, что оно будет понято. Пытаться послать инструкции по расшифровке внешнего послания было бы напрасным усилием, так как они являлись бы частью внутреннего сообщения --- а его можно понять только после того, как найден декодирующий механизм. Поэтому внешнее сообщение всегда \emph{представляет собой скорее набор триггеров} , чем какое-либо послание, поддающееся расшифровке.

Выделение этих трех «уровней» --- только самое начало анализа того, как значение содержится в сообщениях. Сообщения могут иметь не один, а множество внешних и внутренних уровней. Взгляните, например, на то, насколько сложны и связаны между собой внутренний и внешний уровни сообщения на камне Розетты. Чтобы полностью расшифровать это послание и понять отправителя в самом глубоком смысле, нам пришлось бы восстановить всю семантическую структуру, лежащую в основе его создания. После этого мы могли бы вообще выбросить внутреннее сообщение, так как полное понимание всех тонкостей внешнего сообщения позволило бы нам это внутреннее сообщение восстановить.

Подробное обсуждение отношения между внутренним и внешним сообщениями имеется в книге Джорджа Стайнера «После Вавилона» (George Steiner, «After Babel»), хотя автор не использует этой терминологии. Тон этой книги хорошо передает следующая цитата:

Обычно мы используем сокращенную запись, за которой просвечивает богатство подсознательных ассоциаций, иногда нарочно затемненных, а иногда явных --- ассоциаций, таких глубоких и сложных, что, взятые в сумме, они, возможно, передают все своеобразие нашего статуса как индивидуума. \footnote{George Sterner «After Babel» стр. 172 3}

Подобные мысли можно также найти в книге Леонарда Б. Мейера «Музыка, искусство, идеи» (Leonard В. Mayer, «Music, Art, Ideas»):

Манера, в которой мы слушаем композиции Элиотта Картера, весьма отличается от манеры, в которой мы слушаем работы Джона Кейджа. Таким же образом, роман Беккета должен читаться по-иному, чем роман Беллоу Картина, написанная Виллемом де Кунингом нуждается в другом восприятии, чем картина, написанная Энди Вархолем. \footnote{Leonard В. Meyer «Music The Arts and Ideas» стр. 87 8}

Может быть, произведения искусства пытаются, прежде всего, передать некий стиль. В таком случае, если бы мы могли полностью понять и прочувствовать, что именно представляет собой тот или иной \emph{стиль} , мы могли бы обойтись без произведений, написанных в данном стиле. «Стиль», «внешнее сообщение», «декодирующий механизм» --- все это только разные способы выражения одной и той же идеи.

Апериодические кристаллы Шредингера

Что заставляет нас замечать сообщение-рамку в некоторых предметах и не видеть ее в других? Почему инопланетянин, поймавший заблудшую пластинку, должен решить, что в ней спрятано какое-то послание? Чем отличается пластинка от метеорита? Ясно, что ее геометрическая форма является первым ключом к тому, что здесь «что-то не то». Следующий ключ --- то, что на микроскопическом уровне она состоит из очень длинной последовательности апериодических структур, расположенных по спирали. Если расправить эту спираль, то мы получили бы гигантский (около 600 метров) ряд, состоящий из миниатюрных символов. Это не так уж отличается от молекулы ДНК, символы которой, записанные алфавитом из четырех различных оснований, расположены в одномерной последовательности, которая затем скручена в спираль. Еще до того, как Авери установил связь между генами и ДНК, физик Эрвин Шредингер в своем труде «Что такое жизнь?» (Ervin Schroedinger, «What is life?») предсказал, основываясь на чисто теоретических соображениях, что генетическая информация должна содержаться в «апериодических кристаллах». В действительности, сами книги представляют собой апериодические кристаллы, содержащиеся внутри аккуратных геометрических форм. Эти примеры наводят на мысль, что апериодические кристаллы, «упакованные» внутри регулярной геометрической структуры, могут скрывать внутреннее сообщение. (Я не хочу сказать, что это является исчерпывающей характеристикой сообщения-рамки; однако многие типичные сообщения имеют именно такие рамки. На рис. 40 приведены хорошие примеры этого.)

\emph{Рис. 40. Коллаж из различных письменностей. В верхнем левом углу --- надпись на еще нерасшифрованной бустрофедонской системе с острова Пасхи, в которой каждая вторая строчка перевернута. Знаки вырезаны на деревянной табличке размером 9x89 см. Двигаясь по часовой стрелке, мы находим вертикально записанный монгольский; над ним --- современный монгольский, а под ним --- документ, датирующийся 1314 годом. В правом нижнем углу мы находим поэму Рабиндраната Тагора, написанную по-бенгальски. Рядом с ней --- газетный заголовок на майаламе (язык западной Кералы, провинции в южной Индии), над которым --- элегантно изогнутая письменность тамильского (восточная Керала). Самый маленький фрагмент --- отрывок сказания на бугинезском, языке островов Селибеса в Индонезии. В центре --- абзац на тайском языке; над ним --- манускрипт, написанный руническим письмом (четырнадцатый век), содержащий пример законов провинции Скании (южная Швеция). Наконец, налево вклинен фрагмент законов Хаммураби, написанный ассирийской клинописью. Как сторонний наблюдатель, я чувствую очарование тайны, думая о том, как передается значение в странных изгибах и углах этих прекрасных апериодических кристаллов. В самой форме здесь присутствует содержание. (Из книги Ханса Йенсена «Знак, символ и письменность» (Нью-Йорк, 1969), стр. 89 (клинопись), 356 (остров Пасхи), 386, 417 (монгольский), 552 (руническое письмо); из книги Кеннета Катцнера «Языки мира» (Нью-Йорк, 1975), стр. 190 (бенгальский), 237 (бугинезский); из книги И. А. Ричардса и Кристины Гибсон «Английский в картинках» (Нью-Йорк, 1960), стр. 73 (тамильский), 82 (тайский).)}

Языки для трех уровней

Идею трех уровней сообщения хорошо поясняет пример бутылки, выброшенной на берег прибоем. С первым уровнем, рамкой, мы сталкиваемся, когда видим, что бутылка запечатана и внутри нее --- сухой листок бумаги. Даже не видя, написано ли там что-нибудь, мы знаем, что этот предмет --- носитель информации. Чтобы отбросить бутылку, не попытавшись ее открыть, понадобилось бы потрясающее --- почти нечеловеческое --- отсутствие любопытства. Итак, мы открываем бутылку и исследуем значки на бумаге. Может быть, они написаны по-японски; это можно установить, узнав символы, но при этом не поняв ничего из внутреннего сообщения. Внешнее сообщение может быть передано русской фразой «Я --- сообщение, написанное по-японски». Как только этот факт установлен, мы можем обратиться к внутреннему сообщению, которое может оказаться чем угодно: призывом к помощи, стихотворением хайку, жалобой влюбленного\ldots{}

Было бы бесполезно включать в перевод внутреннего сообщения фразу «Это сообщение написано по-японски», поскольку человек, это читающий, должен был бы знать японский. До того, как прочесть внутреннее сообщение, он знал бы, что, поскольку оно написано по-японски, он сможет его прочесть. Можно было бы вывернуться, предложив перевод фразы «Это сообщение написано по-японски» на несколько различных языков. Практически это помогло бы; но теоретически остается та же трудность. Человек, говорящий по-русски, должен сначала узнать «русскость» сообщения --- иначе толку все равно мало. Следовательно, мы не можем избежать проблемы расшифровки внутреннего сообщения \emph{снаружи} ; само внутреннее сообщение может дать нам подсказки и подтверждения, но они не более, чем пусковые механизмы, действующие на человека, нашедшего бутылку (или на его помощников).

С подобными проблемами встречается слушатель коротковолнового радио. Прежде всего, он должен решить, являются ли звуки, которые он слышит, сообщением или просто шумом. Звуки сами по себе не дают ответа на этот вопрос, даже в том маловероятном случае, когда внутреннее сообщение оказывается на языке слушателя и состоит из фразы «Эти звуки --- не шум, а сообщение!» Если слушатель узнает в звуках сообщение-рамку, он пытается установить, на каком языке идет передача --- и ясно, что он находится все еще извне; он принимает \emph{пусковые механизмы} , исходящие из радио, но они не могут дать ему явного ответа.

В самой природе внешних сообщений заложено то, что они не могут быть выражены на явном языке. Найти такой явный язык, на котором можно было бы передать внешнее сообщение, не было бы шагом вперед --- это было бы противоречием в терминах! Понять внешнее сообщение всегда остается заботой слушателя. Если ему это удается, он проникает внутрь, в каковом случае отношение пусковых механизмов к явным значениям сдвигается в пользу последних. По сравнению с предыдущими этапами, понимание внутреннего сообщения весьма нетрудно; оно словно бы входит в нас само собой.

Теория значения «музыкальный автомат»

Эти примеры могут показаться подтверждением идеи, что у сообщений нет присущего им значения --- ведь для того, чтобы понять сколь угодно простое внутреннее сообщение, необходимо сначала понять его рамку и его внешнее сообщение, представляющие из себя пусковые механизмы (такие, как японский алфавит или звуковые дорожки на пластинке). Начинает казаться, что от теории «музыкального автомата» нам никуда не деться. Эта теория гласит, что \emph{никакое сообщение не имеет присущего ему значения} , поскольку, чтобы понять какое-либо сообщение, его надо сначала ввести в «музыкальный автомат»; это значит, что информация, содержащаяся в этом автомате должна быть добавлена к сообщению --- только тогда у него появится значение.

Этот довод весьма похож на ловушку, в которую Черепаха поймала Ахилла в Диалоге Льюиса Кэрролла. Там идея состояла в том, что, прежде чем использовать какое-то правило, необходимо иметь правило, говорящее нам, как использовать первое правило; иными словами, что существует бесконечная иерархия уровней правил, которая не позволяет исполниться ни одному из них. Здесь идея в том, что, прежде чем понять любое сообщение, нам необходимо сообщение, говорящее нам, как понять это сообщение; иными словами, что существует бесконечная иерархия уровней сообщений, которая не позволяет понять ни одного из них. Однако все мы знаем, что эти парадоксы недействительны, поскольку правила все-таки используются и сообщения понимаются. Как же это происходит?

Против теории «музыкального автомата»

Это происходит потому, что наш разум не бестелесен; он расположен в физических объектах --- в наших мозгах. Их структура сформировалась в процессе долгой эволюции, и их действие подчиняются законам физики. Поскольку они являются физическими телами, \emph{наши мозги действуют, не нуждаясь в инструкциях к действию} . Именно на том уровне, где, повинуясь физическим законам, рождаются мысли, парадокс Кэрролла перестает действовать. Точно так же на том уровне, где мозг интерпретирует входящую информацию как сообщение, перестает действовать «парадокс сообщения». По-видимому, в нашем мозгу уже есть встроенная «аппаратура», позволяющая нам распознавать сообщения в некоторых объектах --- и затем эти сообщения декодировать. Эта минимальная врожденная способность извлекать внутренние сообщения делает возможным в высшей степени рекурсивный, подобный снежному кому, процесс усвоения языков Эта врожденная аппаратура --- что-то вроде музыкального автомата она дает недостающую информацию, превращающую простые пусковые механизмы в целые сообщения.

Значение врожденно, если разум естественен

Если бы «музыкальные автоматы» разных людей содержали бы разные «песни» и по-разному отвечали бы на одни и те же пусковые механизмы, нам не пришло бы в голову говорить о том, что этим механизмам присуще определенное значение. Однако человеческие мозги устроены так, что при равенстве остальных условий, один мозг отвечает на данный пусковой механизм почти так же, как и другой. Именно поэтому ребенок может выучить любой язык: все дети одинаково реагируют на «пусковой механизм» разных языков. Это единообразие «человеческого музыкального автомата» устанавливает общий «язык», на котором передаются рамки и внешние сообщения. Более того, если считать, что человеческий разум является лишь одним из примеров общего явления природы --- появления разумных существ в самых разных ситуациях --- то можно предположить, что «язык» на котором передаются рамки и внешние сообщения среди людей, является «диалектом» \emph{универсального} языка, на котором могут договориться между собой любые разумные существа. В таком случае, некоторые пусковые механизмы обладали бы \emph{универсальной пусковой мощью} в том смысле, что любое разумное существо отвечало бы на них примерно так же, как и мы.

Сказанное позволяет нам изменить наше описание того, где находится значение. Мы можем приписать все значения (рамку, внешнее и внутреннее) самому сообщению, поскольку сами декодирующие механизмы универсальны --- иными словами, они представляют собой универсальные формы природы, возникающие в различных контекстах. Приведу конкретный пример: предположим, что кнопки «А-5» запустили одну и ту же песню на всех автоматах --- и представьте также, что автоматы эти сделаны не человеком, а встречаются в природе повсеместно, как галактики или атомы углерода. В этой ситуации, пожалуй, было бы уместно назвать универсальную пусковую мощь кнопок «А-5» «присущим им значением»; кроме того, «А-5» заслуживали бы называться «сообщением» вместо «пускового механизма», и песня была бы «выявлением» внутреннего --- хотя и неявного --- значения этих кнопок.

Земной шовинизм

Таким образом, значение приписывается сообщению в том случае, когда это сообщение понимается одинаково представителями любой, в том числе инопланетной, цивилизации. В этом смысле оно напоминает массу, приписываемую предметам. В древности вес должен был казаться свойством, присущим самим предметам. Но, по мере того, как были лучше поняты законы тяготения, стало ясно, что вес предметов меняется в зависимости от различных гравитационных полей, действующих на данный предмет. Однако существует родственное свойство --- масса; оно не варьируется в зависимости от гравитационного поля. Из этой неизменности вытекает заключение, что масса является свойством, присущим самим предметам. Если окажется, что масса тоже зависит от контекста, то нам придется пересмотреть нашу уверенность в том, что масса --- свойство самих предметов. Таким же образом допустимо, что могут существовать другие типы «музыкальных автоматов» --- разумных существ --- которые общаются между собой при помощи сообщений, которые мы никогда бы не распознали как таковые; с другой стороны, эти существа также не могли бы распознать природу \emph{наших сообщений} . В таком случае, нам пришлось бы пересмотреть наше заключение о том, что наборам символов присущее определенное значение. С другой стороны, как бы мы вообще узнали о существовании подобных созданий?

Интересно сравнить эти рассуждения о неотъемлемости значения с аналогичными рассуждениями о неотъемлемости веса. Предположим, что мы определяем вес тела как «сила, с которой тело давит вниз, находясь на планете Земля». Согласно этому определению, для силы, с которой тело давит вниз, находясь на планете Марс, мы должны использовать иной термин. Это определение делает вес неотъемлемым свойством предметов, но происходит это за счет геоцентризма --- «земного шовинизма». Это что-то вроде «гринвичского шовинизма» --- отказа признавать местное время на всем земном шаре, за исключением гринвичского меридиана.

Возможно, что мы, сами того не сознавая, отягощены подобным шовинизмом в отношении разума, а следовательно и в отношении значения. Будучи такими шовинистами, мы назвали бы «разумными» существа, чей мозг достаточно похож на наш собственный, и отказались бы признавать разум за иными типами объектов. Вот немного преувеличенный пример: представьте себе метеорит, который, вместо того, чтобы пытаться расшифровать Баховскую запись, с абсолютным безразличием протыкает ее и весело устремляется дальше по своей орбите. В нашем понимании, его контакт с пластинкой не затронул ее значения. Поэтому нам может захотеться обозвать метеорит «тупицей». Но что если мы ошибаемся, и метеорит обладает неким «высшим разумом», который мы в своем земном шовинизме не в состоянии обнаружить? В таком случае его взаимодействие с пластинкой могло бы быть проявлением этого высшего разума. Возможно, что пластинка обладает неким «высшим значением», совершенно отличным от того, который приписываем ей мы; может быть, ее значение зависит от типа разума, ее интерпретирующего. Может быть\ldots{}

Было бы прекрасно, если бы могли определить разум как-нибудь иначе, чем «то, что интерпретирует символы таким же образом, как и мы». Ведь если это --- единственное определение, которое мы можем дать разуму, то наше доказательство неотъемлемости значения было бы круговым, а следовательно, свободным от содержания. Мы должны попытаться определить множество характеристик, заслуживающих имя «разума», независимым способом. Эти характеристики представляли бы собой эссенцию разума, которую мы, люди, разделяем с другими разумными существами. На сегодня у нас еще нет полного списка подобных характеристик. Однако весьма вероятно, что в ближайшие десятилетия в попытках определения человеческого разума будет сделан большой прогресс. В частности, не исключено, что специалисты по психологии познания, искусственному разуму и неврологии сумеют совместить их результаты и объяснить, что такое разум. Это определение может все равно оставаться человеко-шовинистическим --- с этим ничего не поделаешь. Но чтобы это уравновесить, может существовать некий элегантный и красивый --- и, возможно, даже простой --- способ дать абстрактную характеристику того, что лежит в сердце разума. Это может уменьшить нашу неловкость от того, что мы сформулировали антропоцентрическое понятие. И, разумеется, если бы мы вступили в контакт с представителями цивилизации из другой звездной системы, мы уверились бы в том, что наш разум --- не счастливая случайность, а пример естественного явления, которое возникает в природе в различных контекстах, так же как звезды и урановые ядра. В свою очередь, это подтвердило бы идею о неотъемлемости значения.

В заключение рассмотрим некоторые новые и старые примеры и обсудим степень неотъемлемости значения в каждом из них, представив на минуту, что мы находимся в положении инопланетянина, нашедшего странный объект\ldots{}

Две пластинки в пространстве

Представьте себе прямоугольную пластинку, сделанную из неразрушимого металлического сплава, на которой выгравированы две точки, одна над другой: такую же картинку представляет только что напечатанное двоеточие. Несмотря на то, что форма этого объекта наводит на мысль, что он искусственный и может содержать некую информацию, двух точек недостаточно, чтобы что-либо сообщить. (Можете ли вы, прежде чем читать далее, поразмышлять над тем, что они могут значить?) Представьте теперь, что мы изготовили вторую пластинку с большим количеством точек, а именно:

~~~~~~~~~~~~~~~~~~~ .

~~~~~~~~~~~~~~~~~~~ .

~~~~~~~~~~~~~~~~~~ ..

~~~~~~~~~~~~~~~~~ ...

~~~~~~~~~~~~~~~~ .....

~~~~~~~~~~~~~~ ........

~~~~~~~~~~~ .............

~~~~~~ .....................

..................................

Теперь естественнее всего --- по крайней мере, для земного разума --- было бы посчитать точки в каждом из рядов и записать получившуюся последовательность:

1, 1, 2, 3, 5, 8, 13, 21, 34.

Очевидно, что существует правило, управляющее количеством точек при переходе с одной линии на следующую. На самом деле, из этого списка мы можем с некоторой степенью уверенностью вывести рекурсивную часть определения чисел Фибоначчи. Предположим, что мы принимаем начальную пару значений (1, 1) за «генотип», из которого при помощи рекурсивного правила производим «фенотип» --- весь ряд чисел Фибоначчи. Посылая лишь один генотип --- первую версию пластинки --- мы опускаем информацию, позволяющую реконструировать фенотип. Таким образом, генотип не содержит полного определения фенотипа. С другой стороны, если мы примем за генотип вторую версию пластинки, у нас будет гораздо больше шансов на то, что фенотип будет восстановлен. Эта новая версия генотипа --- «длинный генотип» --- содержит столько информации, что \emph{механизм, производящий фенотип из генотипа может быть выведен разумными существами из самого генотипа.}

Как только этот механизм для производства фенотипа из генотипа твердо установлен, мы можем вернуться к использованию «краткого генотипа» --- первой версии пластинки. Например, краткий генотип (1, 3) произвел бы фенотип

1, 3, 4, 7, 11, 18, 29, 47,~\ldots{}

--- последовательность Лукаса. Для любого набора двух начальных значений --- то есть, для любого краткого генотипа --- существует соответствующий фенотип. Однако краткие генотипы, в отличие от длинных, действуют только как пусковые механизмы --- кнопки на музыкальном автомате, в который встроено рекурсивное правило. Длинные генотипы содержат достаточное количество информации, чтобы разумное существо могло бы определить, какой именно «музыкальный автомат» надо сконструировать. В этом смысле, длинные генотипы содержат информацию о фенотипе, в то время как краткие --- нет. Иными словами, длинные генотипы передают не только внутреннее сообщение, но и то внешнее сообщение, которое позволяет нам это внутреннее сообщение понять. Кажется, что ясность внешнего сообщения здесь зависит лишь от его длины. Это вовсе не является неожиданностью: то же самое верно и в случае дешифровки старинных текстов. Очевидно, что возможность успеха находится в прямой зависимости от количества имеющегося текста.

Снова Бах против Кейджа

Однако одного длинного текста может оказаться недостаточно. Давайте снова обратимся к разнице между посылкой в космос пластинки с музыкой Баха и пластинки с музыкой Кэйджа. Посмотрим, какое значение имеет для нас музыка Кэйджа. Его произведения должны рассматриваться в широком культурном контексте --- как протест против определенных традиций. Таким образом, если мы хотим передать это значение, мы должны посылать не только ноты данной пьесы, но и всю историю западной культуры. Справедливо будет заключить, что, взятая сама по себе, музыка Кэйджа \emph{не имеет внутреннего значения} . Для слушателя, который достаточно искушен в западной и восточной культурах и, в особенности, в тенденциях западной музыки за последние десятилетия, она \emph{имеет} смысл --- но такой слушатель будет подобен музыкальному автомату, а пьеса Кэйджа --- паре кнопок на нем. Смысл прежде всего находится в голове у слушателя, и музыка служит лишь пусковым механизмом. И этот «музыкальный автомат», в отличие от чистого разума, вовсе не универсален; он связан с земной культурой и зависит от серии событий, происходивших на земном шаре в течение долгого времени. Надеяться на то, что музыка Кэйджа была бы понята инопланетянами, все равно что ожидать, что любимый вами мотивчик зазвучал бы из лунного музыкального автомата при нажатии тех же кнопок, что и на музыкальном автомате в кафе вашего родного городка.

С другой стороны, понимание музыки Баха нуждается в гораздо меньшем знании земной культуры. Это может звучать парадоксально, поскольку Бах сложен и организован, в то время как Кэйдж полностью лишен интеллектуальности. Дело в том, что разум любит организованность и избегает случайности. Для большинства слушателей случайная музыка Кэйджа требует подробных объяснений, даже после которых им все еще может казаться, что они ее не понимают. С другой стороны, большинство Баховских композиций не нуждаются в словах. В этом смысле в музыке Баха больше значения, чем в музыке Кэйджа. И все же мы не можем в точности сказать, в какой степени в Бахе отражена человеческая культура.

Например, в музыке есть три основных структуры (мелодия, гармония и ритм), каждая из которых может быть в свою очередь подразделена на основной, промежуточный и мелкомасштабный аспекты. В каждом из этих измерений есть определенный уровень сложности, который наш мозг способен усвоить, прежде чем начать путаться; очевидно, что композитор, создавая свои произведения, принимает это в расчет --- скорее всего, бессознательно. Эти уровни «терпимой сложности» в различных измерениях, возможно, зависят от специфических условий эволюции человеческого рода; другие разумные существа могли развить музыкальную культуру с совершенно иными уровнями терпимой сложности. Таким образом, вполне возможно, что пьеса Баха должна была бы сопровождаться значительным количеством информации о человеческом роде, которая не может быть выведена лишь из самой музыкальной структуры. Если сравнить музыку Баха с генотипом, а производимые ею эмоции --- с фенотипом, то вопрос заключается в том, содержит ли генотип всю информацию, необходимую для восстановления фенотипа.

Насколько универсально сообщение, содержащееся в ДНК?

Основная проблема, с которой мы сталкиваемся, и которая весьма напоминает проблему двух пластинок, формулируется следующим образом: «Какое количество контекста, необходимого для понимания данного сообщения, может быть восстановлено на основе этого сообщения?» Теперь мы можем вернуться к первоначальному, биологическому значению терминов «генотип» и «фенотип» --- ДНК и живой организм --- и задать те же вопросы. Является ли ДНК универсальным пусковым механизмом? Или ему необходим «био-музыкальный автомат», чтобы раскрыть свое значение? Может ли ДНК вызвать фенотип, не используя соответствующего химического контекста? Ответ на этот вопрос --- нет; но это «нет» --- относительное. Разумеется, молекула ДНК в вакууме не создаст ничего. Однако если бы молекула ДНК была послана «искать счастья» в космос, как пластинки Баха и Кэйджа в нашем воображаемом примере, ее могли бы найти разумная цивилизация. Прежде всего, они могли бы узнать ее сообщение-рамку. После этого, они могли бы попытаться заключить, основываясь на химической структуре ДНК, какой тип химической среды является для нее подходящим, и обеспечить именно этот тип. Постепенно усложняющиеся попытки такого рода могли бы в конце концов привести к полному восстановлению химического контекста, необходимого для выявления фенотипного значения ДНК. Это звучит довольно неправдоподобно, но если дать на эксперименты много миллионов лет, то возможно, что значение ДНК в конце концов было бы восстановлено.

С другой стороны, если бы последовательность основ, составляющих цепь ДНК, была бы послана в космос в виде абстрактных символов (как на рис. 41) вместо длинной спиральной молекулы, шансов на то, что такое внешнее послание пустило бы в действие механизм декодирования, способный восстановить фенотип из генотипа, почти не было бы. Это пример того, как внутреннее послание может быть «завернуто» в настолько абстрактное внешнее послание, что возможности последнего к восстановлению контекста теряются. Практически этот набор символов здесь не имеет собственного смысла. Если вы считаете, что все это звучит безнадежно абстрактно и заумно, имейте в виду, что точный момент, когда фенотип может быть получен из генотипа, является сегодня предметом ожесточенных споров во многих странах, это вопрос о допустимости аборта.

\emph{Рис. 41. Этот громадный апериодический кристалл --- последовательность оснований хромосомы бактериофага фX174. Это первый геном живого организма, который удалось полностью отобразить. Чтобы показать основную последовательность лишь одной клетки кишечной бактерии, понадобилось бы около 2000 таких бустрофедонических страниц; для описания же человеческой клетки потребовалось бы около миллиона страниц. Книга, которую вы держите в руках, содержит приблизительно такое же количество информации, как и молекулярный отпечаток одной-единственной клетки кишечной бактерии.}


% % \subsubsection{Хроматическая фантазия и фига}
% \subsubsection{Хроматическая фантазия и фига}

\emph{Вдоволь наплававшись в пруду, Черепаха вылезает и отряхивается; тут мимо идет Ахилл.}

\emph{Черепаха} : День добрый, Ахилл. Я о вас только что вспоминала, пока купалась.

\emph{Ахилл} : Ну не забавно ли? И вы у меня из головы не выходили, пока я бродил по лугам. Смотрите, я нашел для вас фигу. Правда, она еще зеленая\ldots{}

\emph{Черепаха} : Вы полагаете? Это напоминает мне об одной идейке\ldots{} Хотите послушать?

\emph{Ахилл} : С превеликим удовольствием. Только, пожалуйста, без этих злодейских логических ловушек, г-жа Ч.

\emph{Черепаха} : Злодейских ловушек? Хорошо же вы обо мне думаете! Какая же я злодейка? Я мирная душа, никому не мешаю, живу спокойной травоядной жизнью. Мои мысли текут себе среди странностей и завихрений мироздания (так как я его вижу). Я, скромная наблюдательница явлений, бреду себе потихоньку и бросаю на ветер всякие глупости, которые, боюсь, никого не впечатляют. Но не волнуйтесь, Ахилл, сегодня я собиралась поговорить всего-навсего о своем панцире --- он-то уж не имеет к логике ни малейшего отношения.

\emph{Ахилл} : Вы меня НА САМОМ ДЕЛЕ успокоили, г-жа Ч. И, честно говоря, мое любопытство задето. Охотно вас послушаю, даже если это и не очень впечатляюще.

\emph{Черепаха} : Ну что ж\ldots{} с чего мне начать? Гмм\ldots{} Присмотритесь-ка к моему панцирю --- вас ничего не удивляет?

\emph{Ахилл} : Как будто почище стал?

\emph{Черепаха} : Премного благодарна. Я только что оставила в пруду несколько слоев грязи, накопившихся на мне за последнее столетие. Теперь вы можете увидеть, какой у меня зеленый панцирь!

\emph{Ахилл} : Такой крепкий, зеленый панцирь --- и как ярко он блестит на солнце!

\emph{Черепаха} : Зеленый? Он вовсе не зеленый.

\emph{Ахилл} : Вы же сами только что сказали, что ваш панцирь зеленый!

\emph{Черепаха} : Я так и сказала.

\emph{Ахилл} : В таком случае, мы согласны: он зеленый.

\emph{Черепаха} : Нет, он не зеленый.

\emph{Ахилл} : О, я понимаю: вы намекаете на то, что то, что вы говорите, не обязательно истинно, что Черепахи играют с языком, что ваши утверждения не всегда совпадают с действительностью, что\ldots{}

\emph{Черепаха} : Ничего подобного у меня и в мыслях не было! Слово для Черепах --- святыня; Черепахи преклоняются перед точностью.

\emph{Ахилл} : Хорошо, тогда почему же вы говорите, что ваш панцирь зеленый, и что он не зеленый?

\emph{Черепаха} : Никогда я ничего такого не говорила --- а жаль!

\emph{Ахилл} : Вы хотели бы это сказать?

\emph{Черепаха} : Нисколько. Я сожалею о том, что я это сказала, и совершенно с этим не согласна.

\emph{Ахилл} : Но это противоречит тому, что вы только что сказали!

\emph{Черепаха} : Противоречит? Противоречит? Я никогда себе не противоречу. Это не в черепашьем характере.

\emph{Ахилл} : Ну, на этот раз я вас поймал, хитрюга этакая! Это же самое настоящее противоречие!

\emph{Черепаха} : Вероятно, вы правы.

\emph{Ахилл} : Опять! Теперь вы противоречите себе еще больше! Вы настолько запутались в противоречиях, что с вами невозможно спорить!

\emph{Черепаха} : Вовсе нет. Я спорю сама с собой постоянно, и у меня это прекрасно получается. Может быть, дело в вас самих. Позволю себе предположить, что противоречивы именно вы --- но, поскольку вы сами себя совершенно запутали, вы не в состоянии заметить собственной непоследовательности.

\emph{Ахилл} : Какое оскорбительное предположение! Я вам покажу, что противоречите себе именно вы, и что об этом не может быть двух мнений.

\emph{Черепаха} : Что ж, если это так, Ахилл, то это дело должно быть вам по плечу. Нет ничего легче, чем указать на противоречие. Валяйте, доказывайте, Ахилл!

\emph{Ахилл} : Гмм\ldots{} Даже не знаю, с чего начать\ldots{} А! Теперь вижу. Вы сказали сначала, что (1) ваш панцирь зеленый и тут же, что (2) ваш панцирь не зеленый. Что тут добавишь?

\emph{Черепаха} : Осталось только указать на противоречие. Будьте любезны, перестаньте, наконец, ходить вокруг да около.

\emph{Ахилл} : Но\ldots{} но\ldots{} но\ldots{} О, теперь я понимаю. (Видите ли, иногда я такой тугодум!) Наверное, мы с вами по-разному понимаем противоречие. В этом-то вся загвоздка. Позвольте мне объясниться: противоречие возникает, когда кто-то утверждает одну вещь и одновременно ее отрицает.

\emph{Черепаха} : Вот ловкий трюк! Хотела бы я увидеть, как подобное возможно. Наверное, лучше всего противоречия получались бы у чревовещателей, которые могут говорить одновременно двумя сторонами рта. Но я-то не чревовещатель\ldots{}

\emph{Ахилл} : На самом деле, я имел в виду только то, что кто-то утверждает одну вещь и ее же отрицает в одном и том же предложении. Это не должно быть буквально в один и тот же момент.

\emph{Черепаха} : Однако в вашем примере не ОДНО предложение, а два!

\emph{Ахилл} : Да --- два предложения, противоречащих друг другу.

\emph{Черепаха} : Ну и путаница у вас в голове, бедняга! Сначала вы мне говорите, что противоречие --- это что-то, что должно быть в одном и том же предложении. Тут же вы утверждаете, что вы нашли противоречие в паре моих предложений. Так и есть --- ваш мыслительный процесс настолько запутан, что вы сами не видите, как вы непоследовательны. Со стороны, однако, это ясно как день.

\emph{Ахилл} : Вы меня совсем сбили с толку вашими отвлекающими маневрами. Я уже перестал понимать, идет ли речь о каких-то чепуховых мелочах или же о чем-то важном и глубоком.

\emph{Черепаха} : Уверяю вас, Черепахи не занимаются мелочами. Следовательно, верно второе.

\emph{Ахилл} : Вы меня успокоили, благодарю вас. Теперь, поразмыслив, я вижу логический шаг, необходимый, чтобы уверить вас в том, что вы противоречили себе.

\emph{Черепаха} : Отлично! Надеюсь, что этот шаг столь же легок, сколь бесспорен.

\emph{Ахилл} : Так и есть --- даже вы с ним согласитесь. Идея в том, что если вы считаете истинным предложение 1 («Мой панцирь зеленый») и предложение 2 («Мой панцирь не зеленый»), то вы должны считать истинной комбинацию этих двух предложений. Не так ли?

\emph{Черепаха} : Безусловно. Это только естественно\ldots{} если, конечно, все согласны с тем, КАК эти предложения комбинировать.

\emph{Ахилл} : Разумеется --- и тут-то я вас поймаю! Я предлагаю такую комбинацию ---

\emph{Черепаха} : С комбинированием предложений надо быть осторожнее. Разрешите мне это продемонстрировать. Наверняка, Ахилл, вы согласитесь со следующим предложением, описывающим ваш странный род:

\emph{У людей пять пальцев.}

К тому же, истинность его весьма нетрудно проверить, не так ли?

\emph{Ахилл (неуверенно)} : Соглашусь ли я? То есть, э-э, гмм\ldots{} как это я могу не согласиться с таким скучным и плоским утверждением? Минуточку\ldots{} (Смотрит себе на пальцы и бормочет.) Раз, два, три, четыре\ldots{} (Вслух, Черепахе) Г-жа Черепаха, а мизинец тоже считается за палец?

\emph{Черепаха} : Я думаю, да.

\emph{Ахилл (снова бормочет)} : Ага! Получается пять. Кажется, правильно. Я проверил все необходимые и достаточные условия истинности, так что\ldots{} (Вслух, на этот раз гораздо более уверенно): ЛЮБОЙ знает, что тривиальное суждение «у людей пять пальцев» --- истинно! Что может быть более очевидно?

\emph{Черепаха} : Разумеется. А теперь потрудитесь проверить почти такое же очевидное утверждение, а именно:

\emph{В этом предложении пять слов.}

\emph{Ахилл (бормочет себе под нос)} : Гмм\ldots{} раз\ldots{} два\ldots{} три\ldots{} четыре\ldots{} пять! Да, действительно, я должен согласиться с истинностью и этого утверждения. В ЭТОМ предложении я не вижу никаких проблем.

\emph{Черепаха} : Превосходно! Теперь, когда мои теоретические предположения получили экспериментальное подтверждение в ваших строгих исследованиях, я чувствую себя значительно лучше. Сейчас же, поскольку мы согласны по всем статьям, нам остается только соединить эти два невинных предложения в одно подлиннее, с помощью вашего безопасного слова «и».

\emph{Ахилл} : Именно «безопасного», г-жа Ч. Вам не удастся обвести меня вокруг пальца! Что ж, начнем, пожалуй\ldots{}

\emph{Черепаха} : Прекрасно. Посмотрим\ldots{} у меня получается следующее предложение, которое, безусловно, должно оказаться истинным:

\emph{У людей пять пальцев и в этом предложении пять слов.}

\emph{Ахилл} : Постойте, г-жа Ч! Что-то здесь не то!

\emph{Черепаха (всем своим видом выражая невинное удивление)} : Что? Что вы имеете в виду?

\emph{Ахилл} : Вы соединяете эти предложения неправильно!

\emph{Черепаха} : Я только последовала вашему совету и использовала ваше любимое «и».

\emph{Ахилл} : Не знаю, не знаю\ldots{} То, что у вас получилось, НЕЛОГИЧНО! Где-то здесь должна быть ошибка\ldots{}

\emph{Черепаха} : Ну вот, вы снова заговорили о г-же Логике и ее великих принципах\ldots{} Будьте так любезны, увольте --- хотя бы на сегодня.

\emph{Ахилл} : Г-жа Черепаха, у меня уже черепушка трещит от всего этого. Признайтесь, что вы немного сжульничали\ldots{}

\emph{Черепаха} : Пожалуйста, не обвиняйте меня в собственных грехах, кто из нас хотел соединить два высказывания с помощью «и».Мне кажется, я только следовала вашим пожеланиям --- и какова же ваша благодарность? Ну и молодежь нынче пошла\ldots{}

\emph{Ахилл} : Ну вот, я же и виноват. Ведь я хотел, как лучше\ldots{}

\emph{Черепаха} : Добрыми намерениями, мой юный друг, вымощена дорога в преисподнюю\ldots{}

\emph{Ахилл} : Я чувствую себя ужасно\ldots{}

\emph{Черепаха} : Я отлично понимаю, куда вы клонили: вы хотели заставить меня принять за истинную фразу «Мой панцирь зеленый и мой панцирь не~зеленый». О, создатель!\ldots{} Какая страшная ложь, и как она противна черепашьему духу!

\emph{Ахилл} : Умоляю вас простить меня, дурака\ldots{} Честное слово, у меня и в мыслях~не было вас обидеть.

\emph{Черепаха} : Ничего, мой друг --- мы, черепахи, привыкли к людской бестактности. Я ценю вашу компанию, Ахилл, пусть ваши мысли и не так кристально ясны, как у созданий нашей хладнокровной породы.

\emph{Ахилл (вздыхая)} : Надеюсь, что для меня еще не все потеряно --- хотя я, наверняка, сделаю еще немало ложных шагов на пути к истине\ldots{}

\emph{Черепаха} : Мужайтесь, Ахилл. Может быть, наша сегодняшняя беседа вам поможет\ldots{} Кстати, не забудьте отдать мне ту фигу, что вы мне принесли. Хоть она и зеленая, все равно пригодится!

\emph{Ахилл} : Вот, возьмите.

\emph{Черепаха} : Что ж, до скорого, мой друг.

\emph{Ахилл} : До скорого.


% % \subsubsection{ГЛАВА VII: Исчислени Высказываний}
% \subsubsection{ГЛАВА VII: Исчислени Высказываний}

Слова и символы

ПРЕДЫДУЩИЙ ДИАЛОГ напоминает «Двухголосную инвенцию» Льюиса Кэрролла. В обоих диалогах Черепаха отказывается использовать обычные повседневные слова в их обычном повседневном значении --- по крайней мере, когда ей это невыгодно. В предыдущей главе был предложен один из возможных взглядов на парадокс Кэрролла. В этой главе мы проделаем при помощи символов то, что Ахиллу не удалось проделать словами. Иными словами, мы построим такую формальную систему, один из символов которой будет действовать так, как Ахилл хотел заставить действовать Черепахино «и»; другой символ будет вести себя так, как должны были вести себя слова «если\ldots{} то\ldots». Кроме того, мы будем иметь дело со словами «или» и «не». Рассуждения, зависящие исключительно от правильного употребления этих четырех слов, называются \emph{пропозициональными рассуждениями} .

Алфавит и первое правило исчисления высказываний

Я буду представлять эту новую формальную систему, называемую исчислением высказываний, в форме загадки, объясняя сначала лишь часть и предоставляя читателю догадываться о некоторых вещах самому. Начнем со списка символов:

\textbf{\textless{} \textgreater{}}

\textbf{P Q R '}

\textbf{\&\#923; V э \textasciitilde{}}

\textbf{{[} {]}}

(прим. символ «э»~заменяет символ импликации «superset of»)

~Первое правило нашей системы таково:

ПРАВИЛО ОБЪЕДИНЕНИЯ: Если~\emph{x} и \emph{у} --- теоремы системы, то строчка~\textless{}\emph{x} \&\#923; \emph{y} \textgreater{} --- тоже теорема.

Это правило соединяет две теоремы в одну. Оно должно напомнить вам о предыдущем Диалоге.

Правильно сформированные строчки

У нас будет еще несколько правил вывода; вскоре я их объясню. Однако сначала необходимо определить некое подмножество всех строчек, а именно --- \emph{правильно сформированные строчки} . Они будут определены рекурсивным путем, начиная с атомов. АТОМЫ: \textbf{P} , \textbf{Q} , и \textbf{R} называются \emph{атомами} . Новые атомы получаются путем добавления штрихов справа от старых атомов, таким образом, получаются \textbf{R'} , \textbf{Q''} , \textbf{R'''} и т. д. Это дает нам бесконечные ресурсы атомов. Все атомы правильно сформированы.

Далее, у нас имеются четыре рекурсивных правила.

ПРАВИЛА ОБРАЗОВАНИЯ: Если~\emph{x} и \emph{у} правильно сформированы, то следующие четыре строчки также правильно сформированы.

(1) \textasciitilde{}\emph{x}

(2) \textless{}\emph{x} \&\#923; \emph{y} \textgreater{}

(3) \textless{}\emph{x} V \emph{y} \textgreater{}

(4) \textless{}\emph{x} э \emph{y} \textgreater{}

Например, все следующие строчки правильны:

\textbf{P} ~атом

\textbf{\textasciitilde P} по правилу (1)

\textbf{\textasciitilde\textasciitilde P} по правилу (1)

\textbf{Q'} атом

\textbf{\textasciitilde Q'} по правилу (1)

\textless{}\textbf{P} \&\#923; \textbf{\textasciitilde Q'} \textgreater{} по правилу (2)

\textbf{\textasciitilde{}} \textless{}\textbf{P} \&\#923; \textbf{\textasciitilde Q'} \textgreater{} по правилу (1)

\textless{}\textbf{\textasciitilde\textasciitilde P} э \textbf{Q'} \textgreater~по правилу (4)

\textless{}\textbf{\textasciitilde{}} \textless{}\textbf{P} \&\#923; \textbf{\textasciitilde Q'} \textgreater V\textless{}\textbf{\textasciitilde\textasciitilde P} э \textbf{Q'} \textgreater\textgreater{} по правилу (3)

Последняя строчка может показаться весьма сложной, но на самом деле, она построена всего лишь из двух компонентов --- двух предыдущих строчек. Каждая из них, в свою очередь, построена из предыдущих строчек\ldots{} и так далее. Происхождение любой правильно сформированной строчки может быть прослежено до ее элементарных составляющих --- атомов. Для этого вы просто применяете правила в обратном порядке до тех пор, пока это возможно. Этот процесс рано или поздно должен кончиться, поскольку каждое правило вывода --- удлиняющее правило; идя в обратном порядке, мы непременно дойдем до атомов.

Таким образом, метод разложения строчек служит проверкой их правильности. Это --- нисходящая процедура разрешения для \emph{правильно-сформированности} . Можете проверить, как вы поняли эту процедуру, найдя, какие из ниже приведенных строчек правильно сформированы:

(1) \textless{}\textbf{P} \textgreater{}

(2) \textless{}\textbf{\textasciitilde P} \textgreater{}

(3) \textless{}\textbf{P} \&\#923; \textbf{Q} \&\#923; \textbf{R} \textgreater{}

(4) \textless{}\textbf{P} \&\#923; \textbf{Q} \textgreater{}

(5) \textless\textless{}\textbf{P} \&\#923; \textbf{Q} \textgreater\&\#923;\textless{}\textbf{Q\textasciitilde{}} \&\#923;~\textbf{P} \textgreater\textgreater{}

(6) \textless{}\textbf{P} \&\#923; \textbf{\textasciitilde P} \textgreater{}

(7) \textless\textless{}\textbf{P} V\textless{}\textbf{Q} э \textbf{R} \textgreater\textgreater\&\#923;\textless{}\textbf{\textasciitilde P} V \textbf{\textasciitilde R'} \textgreater\textgreater{}

(8) \textless{}\textbf{P} \&\#923; \textbf{Q} \textgreater\&\#923;\textless{}\textbf{Q} \&\#923; \textbf{P} \textgreater{}

(Ответ. Те строчки, номера которых являются числами Фибоначчи, сформированы неправильно; остальные --- правильно.)

Еще правила вывода

Сейчас мы познакомимся с остальными правилами вывода, при помощи которых строятся теоремы системы. Во всех этих правилах символы «\emph{x} » и «\emph{y»} всегда относятся к правильно сформированным строчкам.

ПРАВИЛО РАЗДЕЛЕНИЯ: Если \textless{}\emph{x} \&\#923; \emph{y} \textgreater~--- теорема, то и~\emph{x} и \emph{у} --- также теоремы.

Вероятно, вы уже догадались, что значит символ «\&\#923;». (Подсказка: это то самое слово, что причинило столько проблем в Диалоге.) Из следующего правила вы сможете вывести значение тильды («\textbf{\textasciitilde{}} »):

ПРАВИЛО ДВОЙНОЙ ТИЛЬДЫ: Строчка «\textbf{\textasciitilde\textasciitilde{}} » может быть выброшена из любой теоремы. Она также может быть вставлена в любую теорему, если при этом получается правильно сформированная строчка.

Правило фантазии

Эта система отличается тем, что в ней \emph{нет аксиом} --- одни лишь правила. Вспомнив наши предыдущие формальные системы, вы можете спросить: как же здесь могут вообще существовать теоремы? Откуда они появляются? Ответом является правило, фабрикующее теоремы «из воздуха» --- оно не требует ввода «старых теорем». (Остальные правила, наоборот, нуждаются во вводных данных.) Это правило называется «\emph{правилом фантазии} .» Почему я его так окрестил? Ответ прост.

Чтобы использовать это правило, вы должны записать любую приглянувшуюся вам правильно сформированную строчку \emph{x} , и затем спросить себя: что бы произошло, если строчка~\emph{x действительно} оказалась бы аксиомой или теоремой? После чего вы предлагаете системе ответить на этот вопрос; это значит, что вы начинаете вывод, используя~\emph{x} как первую строчку. Пусть \emph{у} будет последней строчкой. От~\emph{x} до \emph{у} включительно все является \emph{фантазией} ;~\emph{x} --- \emph{посылка} фантазии, а \emph{у} --- ее результат. Следующий шаг --- выход из области фантазии; мы узнали, что

Если бы~\emph{x} являлось теоремой, то у также являлось бы теоремой.

Вы можете спросить: «Где же здесь \emph{настоящая} теорема?» Это строчка:

\textless{}\emph{x} э \emph{y} \textgreater{}

Обратите внимание на то, как эта строчка напоминает предложение, напечатанное выше.

Чтобы отметить вход и выход в область фантазии, мы будем использовать квадратные скобки «{[}» и «{]}», соответственно. Таким образом, увидев левую квадратную скобку, вы будете знать, что вы «проталкиваетесь» в область фантазии, и \emph{следующая} строчка будет \emph{посылкой} . Увидев правую квадратную скобку, вы будете знать, что вы «выталкиваетесь» обратно из воображаемого мира, и что \emph{предыдущая} строчка была \emph{результатом} . Удобно (хотя и не необходимо) начинать те строчки вывода, что относятся к области фантазии, с нового абзаца.

Ниже приводится иллюстрация правила фантазии в действии. Строчка~P служит посылкой. (На самом деле,~P не \emph{является} теоремой, но для нас это не важно --- мы просто задаем вопрос «а что, если бы она была теоремой?») Мы воображаем следующее:

\textbf{{[}} ~ проталкивание в область фантазии

\textbf{~ P} ~ посылка

\textbf{~~\textasciitilde\textasciitilde P} ~ результат (по правилу двойной тильды)

\textbf{{]}} ~ выталкивание из области фантазии

Наша фантазия показывает, что:

если бы~\textbf{P} было теоремой,~\textbf{\textasciitilde\textasciitilde P} также было бы теоремой.

Теперь мы постараемся «затолкать» это высказывание русского языка (метаязык) в рамки формальной нотации (предметный язык): \textless{}\textbf{P} э \textbf{\textasciitilde\textasciitilde P} \textgreater. Таким образом, наша первая теорема исчисления высказываний должна подсказать вам интерпретацию символа «э».

Вот еще один пример вывода с помощью правила фантазии:

\textbf{{[}} проталкивание в область фантазии

~ \textless{}\textbf{P} \&\#923; \textbf{Q} \textgreater{} посылка

\textbf{~ P} отделение

\textbf{~ Q} отделение

~ \textless{}\textbf{Q} \&\#923; \textbf{P} \textgreater{} соединение

\textbf{{]}} выталкивание из области фантазии

\textless\textless{}\textbf{P} \&\#923; \textbf{Q} \textgreater э\textless{}\textbf{Q} \&\#923; \textbf{P} \textgreater\textgreater{} правило фантазии

Необходимо помнить, что только последняя строчка здесь является настоящей теоремой; все остальное --- чистая фантазия.

Рекурсия и правило фантазии

Как вы могли догадаться из рекурсивной терминологии («проталкивание» и «выталкивание»), правило фантазии может быть использовано рекурсивно --- так что могут существовать фантазии внутри фантазий, фантазии, вложенные друг в друга три раза, и так далее. Это означает, что для этого правила существуют различные уровни реальности, так же как и во вставленных друг в друга рассказах или фильмах. Когда вы выталкиваетесь из фильма, вставленного внутрь другого фильма, на мгновение вам кажется, что вы достигли реального мира, хотя вас все еще отделяет от него один уровень. Точно так же, когда вы выталкиваетесь из фантазии внутри фантазии, вы находитесь в «более реальном», чем предыдущий, мире, хотя он и отстоит на один уровень от настоящего.

Предупреждение «НЕ КУРИТЬ», висящее в кинотеатре, не относится к актерам, играющим в фильме: реальный мир не проникает в фантастический мир фильмов. Однако в исчислении высказываний существует не только воздействие реального мира на фантазии, но и фантазий на вложенные в них более глубокие фантазии. Это свойство отражено в следующем правиле:

ПРАВИЛО ПЕРЕНОСА: В фантазию можно внести любую теорему из «реальности» одним уровнем выше и использовать ее там.

Это похоже на то, если бы табличка «НЕ КУРИТЬ» относилась не только к зрителям, но и ко всем актерам, и далее, к актерам «фильмов в фильме», если бы таковые имелись. (Внимание: переноса в обратном направлении не существует --- теоремы из фантазии не приложимы к реальному миру! Иначе мы могли бы выдумать любую первую строчку фантазии и «вынести» ее в реальный мир в качестве теоремы.)

Чтобы показать, как действует правило переноса и как правило фантазии может быть применено рекурсивно, приведу следующий вывод:

\textbf{{[}} проталкивание

\textbf{~ P} посылка внешней фантазии

\textbf{~~{[}} снова проталкивание

\textbf{~~~ Q} посылка внутренней фантазии

\textbf{~~~ P} перенос \textbf{P} во внутреннюю фантазию

~~~ \textless{}\textbf{P} \&\#923; \textbf{Q} \textgreater{} объединение

\textbf{~~{]}} выталкивание из внутренней фантазии во внешнюю

~ \textless{}\textbf{Q} э\textless{}\textbf{P} \&\#923; \textbf{Q} \textgreater\textgreater{} правило фантазии

\textbf{{]}} выталкивание из внешней фантазии в реальный мир!

\textless{}\textbf{P} э\textless{}\textbf{Q} э\textless{}\textbf{P} \&\#923; \textbf{Q} \textgreater\textgreater\textgreater{} правило фантазии

Обратите внимание на то, что для внешней фантазии я отступил на один абзац, в то время как для внутренней --- на два; этим подчеркивается природа вставленных один в другой «уровней реальности». О правиле фантазии можно сказать, что оно вводит суждение, сделанное о системе, \emph{внутрь} самой системы. Таким образом, можно сказать, что полученная нами теорема~\textless{}\emph{x} э \emph{y} \textgreater~--- отображение внутри системы суждения о ней самой: «Если~\emph{x} --- теорема, то \emph{у} --- также теорема». Более конкретно, \textless{}\textbf{P} э \textbf{Q} \textgreater~интерпретируется как «если \textbf{P} , то \textbf{Q} » или, что одно и то же, «из~\textbf{P} следует \textbf{Q} ».

Перевернутое правило фантазии

В Диалоге Льюиса Кэрролла шла речь о высказываниях типа «если\ldots{} то». В частности, Ахилл никак не мог убедить Черепаху принять за истинную вторую часть «если\ldots{} то» высказывания, даже когда она приняла за истинные как все высказывание целиком, так и его первую часть. Следующее правило позволяет вам вывести вторую часть строчки «э», в том случае, если сама эта строчка и ее первая часть обе являются теоремами.

ПРАВИЛО ОТДЕЛЕНИЯ: Если~\emph{x} и \textless{}\emph{x} э \emph{y} \textgreater{} --- теоремы, то \emph{у} --- также теорема.

Это правило часто зовется «Modus ponens», а правило фантазии --- «Теоремой дедукции».

Интерпретация символов

Довольно загадок! Пора вытащить кота из мешка и открыть «значение» всех остальных символов нашей системы, если это вам еще не ясно. Итак, символ~«\&\#923;»~действует в точности также, как обыкновенное «и». Символ «\textbf{\textasciitilde{}} » заменяет слово «не» в формальном отрицании. Уголки «\textless» и «\textgreater» являются группирующими скобками --- их функция весьма напоминает функцию обычных скобок в алгебре. Основное различие в том, что в алгебре мы свободны вводить или не вводить скобки, согласно нашему вкусу и стилю, в то время как в формальной системе подобная анархия не допускается. Символ «V»~ заменяет слово «или» (по латыни «Vel»). Имеется в виду так называемое включающее «или»; это означает, что \textless{}\emph{x} ~V \emph{y} \textgreater{} читается как «\emph{x} или \emph{у} --- или оба сразу».

Единственные символы, которые мы еще не интерпретировали, это атомы. У них нет единственной интерпретации --- их можно интерпретировать, как любое высказывание русского языка (если атом встречается несколько раз в одной и той же деривации, он должен быть интерпретирован всегда одинаково). Таким образом, например, правильно сформированная строчка \textless{}\textbf{P} \&\#923; \textbf{\textasciitilde P} \textgreater{} может быть интерпретирована следующим образом:

\emph{Этот разум --- Будда, и этот разум --- не Будда} .

Давайте теперь вернемся к теоремам, которые мы вывели до сих пор, и постараемся их интерпретировать. Первая теорема была \textless{}\textbf{P} э \textbf{\textasciitilde\textasciitilde P} \textgreater. Если интерпретировать~\textbf{P} всегда одинаково, то мы получим следующее высказывание:

\emph{Если этот разум --- Будда, то неверно, что этот разум --- не Будда} .

Обратите внимание, как я сформулировал двойное отрицание. В любом натуральном языке неловко повторять отрицание два раза --- мы обходим это препятствие, выражая отрицание по-разному. Вторая наша теорема была \textless\textless{}\textbf{P} \&\#923; \textbf{Q} \textgreater э\textless{}\textbf{Q} \&\#923; \textbf{P} \textgreater\textgreater. Пусть \textbf{Q} --- высказывание «Этот огурец весит полкило»; тогда наша теорема читается как:

\emph{Если этот разум --- Будда и этот огурец весит полкило, то этот огурец весит полкило и этот разум --- Будда.}

Третьей теоремой была \textless{}\textbf{P} ~э\textless{}\textbf{Q} э\textless{}\textbf{P} \&\#923; \textbf{Q} \textgreater\textgreater\textgreater. Она разворачивается в структуру «если~\ldots{} то» с вложением:

\emph{Если этот разум --- Будда то, если этот огурец весит полкило, то этот разум --- Будда и этот огурец весит полкило} .

Вы вероятно, заметили, что каждая теорема, будучи интерпретированной, выражает что-либо совершенно тривиальное и самоочевидное. (Иногда теоремы бывают \emph{настолько} самоочевидными, что кажутся бессмысленными --- и даже, как это ни парадоксально, ложными!) Может быть, это вас не впечатляет; но вспомните, сколько ложных высказываний, кишмя кишащих кругом, мы могли бы вывести --- но не вывели. Система исчисления высказываний аккуратно ступает от истины к истины, осторожно избегая всех ложных высказываний, подобно человеку, который, переходя ручей и желая остаться сухим, осторожно ступает с камня на камень, следуя выложенной «тропинке», как бы извилиста она не была. Удивительно то, что в исчислении высказываний все делается исключительно \emph{типографским путем} . «Внутри» системы нет никого, кто бы думал о \emph{значении} строчек. Здесь все делается строго механически и бездумно.

Полный список правил

Мы еще не привели всех правил исчисления высказываний. Их полный список, включая три новые правила, приведен ниже.

ПРАВИЛО ОБЪЕДИНЕНИЯ: Если~\emph{x} и \emph{у} --- теоремы системы, то строчка \textless{}\emph{x} \&\#923; \emph{y} \textgreater{} --- также теорема.

ПРАВИЛО РАЗДЕЛЕНИЯ: Если \textless{}\emph{x} \&\#923; \emph{y} \textgreater{} --- теорема, то и~\emph{x} и \emph{у} --- также теоремы.

ПРАВИЛО ДВОЙНОЙ ТИЛЬДЫ: Строчка~«\textbf{\textasciitilde\textasciitilde{}} » может быть выброшена из любой теоремы. Она также может быть вставлена в любую теорему, если при этом получается правильно сформированная строчка.

ПРАВИЛО ФАНТАЗИИ: Если, принимая~\emph{x} за теорему, можно вывести \emph{у} , то \textless{}\emph{x} э \emph{y} \textgreater{} является теоремой.

ПРАВИЛО ПЕРЕНОСА: В фантазию можно внести любую теорему из «реальности» одним уровнем выше и использовать ее там.

ПРАВИЛО ОТДЕЛЕНИЯ: Если~\emph{x} и \textless{}\emph{x} э \emph{y} \textgreater~--- теоремы, то \emph{у} --- также теорема.

ПРАВИЛО КОНТРАПОЗИЦИИ:~\textless{}\emph{x} э \emph{y} \textgreater{} и~ \textless\textasciitilde{}\emph{y} э \textasciitilde{}\emph{x} \textgreater{} взаимозаменяемы.

ПРАВИЛО ДЕ МОРГАНА: \textless\textasciitilde{}\emph{x} \&\#923; \textasciitilde{}\emph{y} \textgreater{} и \textasciitilde\textless{}\emph{x} V \emph{y} \textgreater{} взаимозаменяемы.

ПРАВИЛО ЗАМЕНЫ: \textless{}\emph{x} V \emph{y} \textgreater~и \textless\textasciitilde{}\emph{x} э \emph{y} \textgreater{} взаимозаменяемы.

Под «взаимозаменяемостью» здесь понимается следующее: если одно из двух выражений встречается в виде теоремы или части теоремы, оно может быть заменено на второе, и результат также будет теоремой.

Объяснение правил

Прежде чем рассматривать эти правила в действии, я хочу их коротко пояснить. Вы, вероятно, можете придумать примеры получше; поэтому я ограничусь только несколькими.

Правило контрапозиции показывает то, каким образом мы перевертываем условные предложения (обычно мы делаем это бессознательно). Например, буддистское изречение о Тропе --- дороге к Мудрости:

Если вы изучаете ее, то вы далеко от Тропы,

означает то же самое, что

Если вы близко к Тропе, то вы ее не изучаете.

Правило Де Моргана может быть проиллюстрировано на примере хорошо знакомого нам высказывания «Флаг не движется и ветер не движется». Если~\textbf{P} означает «флаг движется» и \textbf{Q} --- «ветер движется», то комбинированное высказывание будет \textless{}\textbf{\textasciitilde P} \&\#923; \textbf{\textasciitilde Q} \textgreater, которое, согласно правилу Де Моргана, может быть заменено на \textbf{\textasciitilde{}} \textless{}\textbf{P} ~V \textbf{Q} \textgreater: «Неверно, что флаг или ветер движутся». Никто не станет оспаривать, что это весьма осмысленное дзен-ключение\ldots{}

Для иллюстрации правила замены возьмем высказывание «Либо туча зависла над горой, либо лунный луч проникает сквозь волны озера» --- фраза, которую мог бы произнести дзен-буддистский мастер, пытаясь мысленно увидеть любимое озеро. Теперь держитесь крепче: правило замены утверждает, что это высказывание может быть заменено на мысль «Если туча не зависла над горой, то лунный луч проникает сквозь волны озера.» Это, может быть, и не Просветление, но это большее, что исчисление высказываний может нам предложить.

Игра с системой

Теперь давайте приложим эти правила к одной из предыдущих теорем и посмотрим, что у нас выйдет. Возьмем, к примеру, теорему \textless{}\textbf{P} э \textbf{\textasciitilde\textasciitilde P} \textgreater:

\textless{}\textbf{P} э \textbf{\textasciitilde\textasciitilde P} \textgreater{} старая теорема

\textless{}\textbf{\textasciitilde\textasciitilde\textasciitilde P} э \textbf{\textasciitilde P} \textgreater{} контрапозиция

\textless{}\textbf{\textasciitilde P} э \textbf{\textasciitilde P} \textgreater{} двойная тильда

\textless{}\textbf{P} V \textbf{\textasciitilde P} \textgreater{} замена

Новая теорема в интерпретации утверждает, что:

\emph{Либо этот разум Будда, либо этот разум не Будда} .

Интерпретированная теорема снова оказалось истинным (хотя, может быть, и не таким уж удивительным) высказыванием.

Частичная интерпретация

Читая вслух теоремы исчисления высказываний, кажется естественным интерпретировать все, кроме атомов. Я называю это \emph{частичной интерпретацией} . Например, частичной интерпретацией\textless{}\textbf{P} ~V \textbf{\textasciitilde{} P} \textgreater~ было бы:

\textbf{P} ~или не~\textbf{P}

Хотя~\textbf{P} здесь и не высказывание, приведенное полувысказывание все же звучит как истинное, поскольку мы можем легко вообразить на месте~\textbf{P} любое предложение --- и форма этой частичной интерпретации уверяет нас, что, независимо от нашего выбора, результатом будет истинное высказывание. Именно это --- центральная идея исчисления высказываний: оно производит теоремы, которые, будучи частично интерпретированными, производят «универсально истинные полувысказывания». Независимо от того, как мы дополним интерпретацию, у нас получатся истинные суждения.

Топор Ганто

Теперь мы можем проделать более сложное упражнение, основанное на дзен-буддистстком коане под названием «Топор Ганто». Вот его начало:

Однажды Токусан сказал своему ученику Ганто «В нашем монастыре есть два монаха, которые прожили здесь много лет. Иди и проверь их». Ганто взял топор и пошел в хижину, где монахи занимались медитацией. Он поднял топор со словами. «Если вы скажете хоть одно слово, я отрублю вам головы; и если вы не скажете ни слова, я все равно отрублю вам головы».\footnote{Gyomay M. Kubose «Zen Koans» стр. 178}

Если вы скажете хоть одно слово, я прерву этот коан; и если вы не скажете ни слова, я все равно прерву этот коан --- поскольку хочу перевести его в нашу нотацию. Пусть «вы скажете слово» будет \textbf{P} , а «я отрублю вам головы» --- \textbf{Q} . Тогда угроза Ганто записывается как \textless\textless{}\textbf{P} э \textbf{Q} \textgreater\&\#923;\textless{}\textbf{\textasciitilde P} э \textbf{Q} \textgreater\textgreater. Что, если бы эта угроза являлась бы аксиомой? Ответом на этот вопрос служит следующая фантазия:

~ (1) \textbf{{[}} проталкивание

~ (2)~~ \textless\textless{}\textbf{P} э \textbf{Q} \textgreater\&\#923;\textless{}\textbf{\textasciitilde P} э \textbf{Q} \textgreater\textgreater{} аксиома Ганто

~ (3)~~ \textless{}\textbf{P} э \textbf{Q} \textgreater{} разделение

~ (4)~~ \textless{}\textbf{\textasciitilde Q} э \textbf{\textasciitilde P} \textgreater{} контрапозиция

~ (5)~~ \textless{}\textbf{\textasciitilde P} э \textbf{\textasciitilde Q} \textgreater{} разделение

~ (6)~~ \textless{}\textbf{\textasciitilde Q} э \textbf{\textasciitilde\textasciitilde P} \textgreater{} контрапозиция

~ (7)~~ \textbf{{[}} снова проталкивание

~ (8)~~~~ \textbf{\textasciitilde Q} посылка

~ (9)~~~~ \textless{}\textbf{\textasciitilde Q} э \textbf{\textasciitilde P} \textgreater{} перенос строки 4

(10)~~~ ~\textbf{\textasciitilde P} отделение

(11)~~~~ \textless{}\textbf{\textasciitilde Q} э \textbf{\textasciitilde\textasciitilde P} \textgreater{} перенос строки 6

(12)~~~~ \textbf{\textasciitilde\textasciitilde P} отделение (строки 8 и 11)

(13)~~~~ \textless{}\textbf{\textasciitilde P} \&\#923; \textbf{\textasciitilde\textasciitilde P} \textgreater{} объединение

(14)~~~~ \textbf{\textasciitilde{}} \textless{}\textbf{P} V \textbf{\textasciitilde P} \textgreater{} Де Морган

(15)~~ \textbf{{]}} выталкивание

(16)~~~\textless{}\textbf{\textasciitilde Q} э \textbf{\textasciitilde{}} \textless{}\textbf{P} V\textbf{\textasciitilde P} \textgreater\textgreater~правило фантазии

(17)~~ \textless\textless{}\textbf{P} V \textbf{\textasciitilde P} \textgreater{} э \textbf{Q} \textgreater~контрапозиция

(18)~~ \textbf{{[}} проталкивание

(19)~~~~ \textbf{\textasciitilde P} посылка (и результат!)

(20)~~~\textbf{{]}} выталкивание

(21)~~ \textless{}\textbf{\textasciitilde Р} э \textbf{\textasciitilde Р} \textgreater~правило фантазии

(22)~~ \textless{}\textbf{P} ~V \textbf{\textasciitilde P} \textgreater~правило замены

(23)~~ \textbf{Q} ~отделение (строки 22 и 17)

(24) \textbf{{]}} выталкивание

Этот пример показывает, насколько мощно исчисление высказываний. Всего лишь за 24 шага мы логически вывели, что \textbf{Q} --- иными словами, головы будут отрублены! (Зловещая примета: последнее использованное нами правило было правилом «отделения»\ldots) Теперь, скажете вы, нет смысла продолжать коан, так как исход уже известен. Однако я передумал и не буду его прерывать --- в конце концов, это настоящий дзен-коан! Итак, вот конец этого рассказа:

Оба монаха продолжали медитировать как ни в чем не бывало, словно они ничего не слышали. Тогда Ганто опустил топор и воскликнул: «Вы --- настоящие дзен-буддисты!» Затем он вернулся к Токусану~и рассказал о случившемся. «Я понимаю вашу идею», --- сказал тот, --- «но скажите мне, какова их идея?» «Тозан мог бы принять их в ученики, --- ответил Ганто, --- но они не должны быть приняты в ученики Токусаном».\footnote{Там же стр. 178}

Понимаете ли вы мою идею? А как насчет идеи дзена?

Имеется ли разрешающий алгоритм для теорем?

Исчисление высказываний дает нам набор правил для производства таких высказываний, которые были бы истинными в любом из возможных миров. Именно поэтому все его теоремы звучат так просто, кажется, что они совершенно лишены содержания! С такой точки зрения, исчисление высказываний должно казаться пустой тратой времени, поскольку оно сообщает нам абсолютно тривиальные вещи. С другой стороны, это делается путем определения \emph{формы} универсально истинных высказываний, что представляет основные истины вселенной в новом свете. Они не только фундаментальны, но и \emph{регулярны} : их можно произвести, используя определенный набор типографских правил. Иными словами, все они сделаны из одного теста. Можете поразмыслить над тем, возможно ли произвести также и дзен-буддисткие коаны, пользуясь набором типографских правил.

Весьма важным здесь является вопрос о разрешающей процедуре --- а именно, существует ли некий механический метод отличения теорем от не-нетеорем? Если да, то это будет означать, что теоремы исчисления высказываний не только рекурсивно перечислимы, но и рекурсивны. Оказывается, что алгоритм разрешения существует, и довольно интересный --- таблицы истинности. Изложение этого метода увело бы нас слишком далеко в сторону; вы можете найти его почти в любой книге по логике. А как насчет дзен-буддистских коанов? Может ли существовать такая механическая процедура разрешения, которая отличала бы настоящий дзен-коан от всех остальных вещей?

Откуда мы знаем, что система непротиворечива?

До сих пор, мы только \emph{предполагали} , что все теоремы, интерпретированные должным образом, производят истинные высказывания. Но знаем ли мы это \emph{наверняка} ? Можем ли мы это доказать? Иными словами, заслуживают ли наши интерпретации («и» для «\&\#923;» и так далее) того, чтобы именоваться «пассивными значениями» символов? На это существуют два различных взгляда, которые можно назвать «осторожным» и «неосторожным». Я представлю это взгляды так, как я их понимаю; пусть их выразителей зовут, соответствено, «Осторожность» и «Неосторожность».

\emph{Осторожность} : Мы будем знать наверняка, что при нашей интерпретации все теоремы получаются истинными, только тогда, когда сможем это доказать. Это вдумчивый и осторожный способ действия.

\emph{Неосторожность} : Напротив, ОЧЕВИДНО, что все теоремы получаются истинными. Если вы в этом сомневаетесь, взгляните еще раз на правила системы. Вы увидите, что каждое правило заставляет символ действовать точно также, как должно действовать слово, им представляемое. Например, правило объединения заставляет символ «\&\#923;» действовать как «и»; правило отделения заставляет «э» действовать также, как слова «если~\ldots{} то», и так далее. Если только вы не похожи в этом отношении на Черепаху, то легко узнаете в каждом правиле кодификацию схем, которыми пользуетесь в собственных мыслях. Поэтому, если вы доверяете собственным мыслям, вы ОБЯЗАНЫ верить в то, что все теоремы в интерпретации выходят истинными. Таково мое мнение. Я не нуждаюсь в дальнейших доказательствах. Если вы считаете, что какая-нибудь теорема может получиться ложной, значит вы думаете, что какое-то из правил неверно. В таком случае, покажите мне, какое именно?

\emph{Осторожность} : Не могу, поскольку я не знаю точно, что там есть неверные правила --- поэтому я не могу указать вам на одно из них Все же я могу вообразить себе следующую сцену. Следуя правилам, вы выводите теорему --- скажем, \emph{x} . Между тем, я, также следуя правилам, вывожу другую теорему --- и предположим, у меня вышло \emph{\textasciitilde x} . Можете ли вы представить себе такое?

\emph{Неосторожность} : Хорошо --- представим себе, что такое произошло. Чем это вам помешает? Скажем, мы обе играем с системой \textbf{MIU} ; у меня получилась теорема \emph{x} , а у вас --- \emph{x} \textbf{U} . Можете вы представить такое?

\emph{Осторожность} : Разумеется: и \textbf{MI} , и \textbf{MIU} --- теоремы.

\emph{Неосторожность} : И вас это не смущает?

\emph{Осторожность} : Конечно, нет. Ваш пример просто смешон, поскольку теоремы \textbf{MI} и \textbf{MIU} не ПРОТИВОРЕЧАТ одна другой, в то время как строчки~\emph{x} и \textbf{\textasciitilde{}} \emph{x} в исчислении высказываний противоречивы.

\emph{Неосторожность} : Хорошо --- если только вы решили интерпретировать «\textbf{\textasciitilde{}} » как «не». Но что заставляет вас думать, что «\textbf{\textasciitilde»} должно быть интерпретировано именно так?

\emph{Осторожность} : Сами правила. Их них видно, что единственной возможной интерпретацией для «\textbf{\textasciitilde{}} » является «не», единственной возможной интерпретацией для~«\&\#923;» --- «и» и так далее.

\emph{Неосторожность} : Иными словами, вы считаете, что правила описывают значения слов?

\emph{Осторожность} : Именно так.

\emph{Неосторожность} : И, несмотря на это, вы все еще цепляетесь за мысль, что обе~\emph{x} и \textbf{\textasciitilde{}} \emph{x} могут быть теоремами? Почему бы вам заодно не предположить, что ежи --- это жабы, или что 1 равняется 2, или что луна сделана из зеленого сыра? Я, со своей стороны, не хочу даже и думать, что основные ингредиенты моего мыслительного процесса могут быть ошибочными --- иначе мне пришлось бы усомниться в собственном анализе всего этого вопроса, и я бы совершенно запуталась.

\emph{Осторожность} : Ваши аргументы притянуты за уши. Все же мне хотелось бы увидеть ДОКАЗАТЕЛЬСТВО того, что все теоремы истинны, или того, что~\emph{x} и \textbf{\textasciitilde{}} \emph{x} не могут быть теоремами одновременно.

\emph{Неосторожность} : Желаете доказательства? По-моему, вы более хотите убедиться в непротиворечивости исчисления высказываний, чем в вашем собственном душевном здоровье. Любое мыслимое доказательство включало бы более сложные операции, чем те, что возможны в самом исчислении высказываний. И что бы это доказало? С вашим желанием доказать непротиворечивость исчисления высказываний вы напоминаете мне человека, который захотел выучить русский и потребовал для этого словарь, определяющий все простые слова через более сложные\ldots{}

Снова Кэрролловский Диалог

Этот небольшой спор показывает, как трудно использовать логику и рассуждеения для защиты самой логики. В какой-то момент вы упираетесь в стенку, и вам ничего не остается, кроме как выкрикивать: «Я знаю, что я прав!» Мы снова столкнулись с вопросом, который Льюис Кэрролл так ярко проиллюстрировал в своем Диалоге: продолжать защищать схему собственного мышления до бесконечности невозможно. Рано или поздно наступает момент, когда приходится в нее просто поверить.

Систему рассуждений можно сравнить с яйцом. Его внутренность защищена скорлупой --- но чтобы куда-то это яйцо послать, вы на нее не надеетесь. Вы упаковываете яйцо в контейнер, выбранный в соответствии с трудностью предстоящего путешествия. Если вы хотите действовать более осторожно, можете даже уложить яйцо в несколько вложенных одна в другую коробок. Однако сколько бы коробок вы не использовали, всегда можно вообразить себе, что происходит катастрофа и яйцо все же разбивается. Точно так же мы никогда не можем дать абсолютное, конечное доказательство того, что доказательства какой-либо системы истинны. Разумеется, мы можем представить доказательство доказательства, или доказательство доказательства доказательства --- но нам всегда приходится принимать на веру состоятельность самой внешней из систем. Всегда возможно вообразить, что некая тонкость разрушит каждое из наших доказательств --- и когда мы дойдем до «дна», то «доказанный» результат окажется вовсе не таким уж истинным. Это, однако, не означает, что математики и физики постоянно беспокоятся о том, что все здание математики может быть ложным. С другой стороны, когда люди сталкиваются с неординарными, или слишком длинными, или полученными на компьютере доказательствами, они начинают думать над тем, что же имеется в виду под этим почти святым понятием «доказательства».

Отличным упражнением для вас, читатель, было бы сейчас снова вернуться к Диалогу Кэрролла и попытаться закодировать весь спор с самого начала, используя нашу нотацию.

\emph{Ахилл} : Если у вас имеется \textless\textless{}\textbf{A} \&\#923; \textbf{B} \textgreater э \textbf{Z} \textgreater{} и \textless{}\textbf{A} \&\#923; \textbf{B} \textgreater, то у вас наверняка есть \textbf{Z} .

\emph{Черепаха} : Вы имеете в виду, что \textless\textless\textless\textless{}\textbf{A} \&\#923; \textbf{B} \textgreater{} э\textbf{Z} \textgreater\&\#923;\textless{}\textbf{A} \&\#923; \textbf{B} \textgreater\textgreater{} э\textbf{Z} \textgreater, не так ли?

(Подсказка: то, что Ахилл считает правилом вывода, Черепаха туг же превращает в простую строчку системы. Используя только буквы А, В и Z, вы получите непрерывно удлиняющуюся рекурсивную структуру.)

Кратчайший путь и выведенные правила

Выводя теоремы исчисления высказываний, мы обычно вскоре изобретаем различные сокращения пути, строго говоря, не являющиеся частью системы. Например, если бы в какой-то момент нам понадобилась бы строчка \textless{}\textbf{Q} ~V \textbf{\textasciitilde{} Q} \textgreater, и при этом у нас уже имелась бы ранее выведенная строчка \textless{}\textbf{P} ~V \textbf{\textasciitilde{} P} \textgreater, многие из нас действовали бы так, словно строчка \textless{}\textbf{Q} ~V \textbf{\textasciitilde{} Q} \textgreater{} уже выведена, так как мы знаем, что ее вывод в точности соответствует выводу \textless{}\textbf{P} ~V \textbf{\textasciitilde{} P} \textgreater. Выведенная теорема используется здесь как «схема теорем» --- форма для их отливки. Этот прием вполне допустим, поскольку он помогает нам выводить новые теоремы~--- но сам по себе он не является правилом исчисления высказываний. Скорее это вторичное, \emph{выведенное} правило, часть нашего знания о системе. Конечно, то, что это правило всегда оставляет нас в области теорем, еще надо доказать --- но тем не менее, это правило отличается от дериваций внутри системы. Оно является доказательством в ординарном, интуитивном значении этого слова --- цепочка рассуждений, проведенная по способу I. Теория об исчислении высказываний является «мета-теорией», и ее результаты можно назвать «мета-теоремами» --- Теоремами о теоремах. (Обратите внимание на заглавную букву в выражении «Теоремы о теоремах». Это --- следствие нашего соглашения: мета-теоремы являются Теоремами (доказанными результатами), касающимися теорем (выводимые строчки).)

В исчислении высказываний можно найти множество других мета-теорем, или вторичных правил вывода. Вот, например, вторичное правило Де Моргана:

\textless{}\textbf{\textasciitilde{}} \emph{x~} V \textbf{\textasciitilde{}} \emph{y} \textgreater{} и \textbf{\textasciitilde{}} \textless{}\emph{x} \&\#923; \emph{у} \textgreater{} взаимозаменяемы.

Если бы это было правилом системы, это значительно ускорило бы многие деривации. А что, если мы \emph{докажем} , что оно верно --- достаточно ли этого, чтобы использовать его в качестве еще одного правила вывода?

У нас нет причин сомневаться в истинности этого выведенного правила. Однако как только вы начинаете использовать выведенные правила в процедуре исчисления высказываний, формальность системы теряется, поскольку эти правила выведены неформально --- вне системы. Формальные системы были предложены, как способ проследить за каждым шагом доказательства внутри единой строгой системы, чтобы каждый математик мог механически проверить работу своих коллег. Однако если вы готовы при малейшей возможности выскочить за рамки системы, то зачем ее вообще было создавать? Как видите, у подобных правил есть и отрицательная сторона.

Формализация высших уровней

С другой стороны, возможен и иной выход. Почему бы нам не формализовать также и мета-теорию? Таким образом, выведенные правила (мета-теоремы) станут частью большей формальной системы и вывод новых, упрощающих деривацию теорем формализованной мета-теории станет законным. Эти теоремы затем могут быть использованы, чтобы облегчить вывод теорем исчисления высказываний. Это интересная идея, но как только мы начинаем ее обдумывать, то тут же сталкиваемся с мета-мета-теориями и так далее. Ясно, что сколько бы уровней мы не формализовали, всегда найдется кто-нибудь, кто захочет вывести упрощающие правила на высшем уровне.

Можно даже предположить, что теория логических рассуждений могла бы быть идентична своей мета-теории, если бы последняя была достаточно аккуратно разработана. Тогда, казалось бы, все уровни соединились бы в один единственный, и размышления \emph{о} системе стали бы аналогичны работе \emph{внутри} системы. Однако это не так просто. Даже если система и способна размышлять о самой себе, это еще не значит, что она \emph{выпрыгивает} из себя. Вы, находясь вне системы, воспринимаете ее по-другому, чем она воспринимает себя сама. Таким образом, мета-теория --- взгляд со стороны --- все равно существует, даже если теория и может «обдумывать себя саму», не выходя за пределы системы. В дальнейшем мы увидим, что существуют теории, способные на самоанализ. Более того, вскоре мы познакомимся с системой, где это происходит совершенно случайно, без малейшего нашего желания, и увидим, что из этого получается. Однако в нашей работе с исчислением высказываний мы постараемся придерживаться простейших идей и избегать смешения уровней.

Ошибки получаются, когда нам не удается четко разграничить работу внутри системы (способ \textbf{M} ) и размышления о системе (способ \textbf{I} ). Например, может показаться вполне разумным предположить, что, поскольку \textless{}\textbf{P} ~V \textbf{\textasciitilde{} P} \textgreater{} (частично интерпретируемое как~\textbf{P} или не \textbf{P} ) --- теорема, то одна из двух --- либо \textbf{P} , либо не \textbf{P} , должна также являться теоремой. Но это совершенно неверно; не один из членов этой пары не является теоремой. Опасно считать, что символы можно свободно передвигать между разными уровнями --- как, например, язык формальной системы и ее метаязык (русский).

Размышления о сильных и слабых сторонах системы

Мы только что познакомились с системой, предназначенной отразить часть архитектуры логического мышления. Эта система имеет дело с небольшим количеством простых и точных понятий. Именно простота и точность исчисления высказываний делает его таким привлекательным для математиков. Для этого есть две причины. (1) Его свойства можно изучать сами по себе (так геометрия изучает простые и неподвижные формы). Исчисление высказываний можно варьировать путем изменения различных символов, правил вывода, аксиом или схем аксиом и так далее. (Кстати, представленный здесь вариант исчисления высказываний был изобретен Г. Гентценом в начале 1930-х годов. Существуют другие версии, в которых используется единственное правило вывода --- обычно, отделение --- ив которых есть несколько аксиом или схем аксиом.) Изучение методов логического мышления при помощи элегантных формальных систем --- это весьма привлекательная ветвь чистой математики. (2) Исчисление высказываний может быть легко расширено до включения других фундаментальных аспектов мышления. Это будет частично показано в следующей главе, где исчисление высказываний целиком будет включено в намного большую и глубокую систему, способную на сложные рассуждения в области теории чисел.

Доказательства и деривации

Исчисление высказываний напоминает процесс мышления, но при этом мы не должны равнять его правила с правилами человеческой мысли. \emph{Доказательство} --- это нечто неформальное; иными словами --- это продукт нормального мышления, записанный на человеческом языке и предназначенный для человеческого потребления. В доказательствах могут использоваться всевозможные сложные мыслительные приемы и, хотя интуитивно они могут казаться верными, можно усомниться в том, возможно ли доказать их логически. Именно поэтому мы и нуждаемся в формализации. \emph{Деривация} , или \emph{вывод} --- это искусственное соответствие доказательства; ее назначение --- достичь той же цели, на этот раз с помощью логической структуры, методы которой не только ясно выражены, но и очень просты.

Обычно формальная деривация бывает крайне длинна по сравнению с соответствующей «естественной» мыслью. Это, конечно, плохо --- но это та цена, которую приходится платить за упрощение каждого шага. Часто бывает, что деривация и доказательство «просты» в дополнении друг к другу. Доказательство просто в том смысле, что каждый шаг «кажется правильным», даже если мы и не знаем точно, почему; деривация проста, потому что каждый из мириада ее шагов так прост, что к нему невозможно придраться и, поскольку вся деривация состоит из таких шагов, мы предполагаем, что она безошибочна. Каждый тип простоты, однако, привносит свой тип сложности. В случае доказательств, это сложность системы, на которую они опираются --- а именно, человеческого языка; в случае дериваций, это их астрономическая длина, делающая их почти невозможными для понимания.

Таким образом, мы считаем исчисление высказываний частью общего метода для составления искусственных структур, подобных доказательствам. Однако оно лишено гибкости или всеобщности, поскольку предназначено только для работы с математическими понятиями, которые, в свою очередь, жестко определенны. В качестве довольно интересного примера давайте рассмотрим деривацию, в которой посылкой фантазии является необычная строчка: \textless{}\textbf{Р} \&\#923; \textbf{\textasciitilde{} Р} \textgreater. По крайней мере, ее частичная интерпретация звучит странно. Исчисление высказываний, однако, не задумывается над интерпретациями --- вместо этого, оно просто манипулирует типографскими символами, а в типографском смысле в этой строчке нет ничего необычного.

Вот фантазия с данной строчкой в качестве посылки.

~ (1) \textbf{{[}} ~проталкивание

~ (2)~~~\textless{}\textbf{Р} ~\&\#923; \textbf{\textasciitilde Р} \textgreater{} посылка

~ (3)~~~\textbf{Р} ~разделение

~ (4)~~~\textbf{\textasciitilde Р} ~разделение

~ (5)~~~\textbf{{[}} ~проталкивание

~ (6)~~~~~\textbf{\textasciitilde Q} ~посылка

~ (7)~~~~~\textbf{Р} ~переход, строка 3

~ (8)~~~~~\textbf{\textasciitilde\textasciitilde Р} ~двойная тильда

~ (9)~~~\textbf{{]}} ~выталкивание

(10)~~~\textless{}\textbf{\textasciitilde Q} э \textbf{\textasciitilde\textasciitilde Р} \textgreater~фантазия

(11)~~~\textless{}\textbf{\textasciitilde P} э \textbf{Q} \textgreater~контрапозиция

(12)~~~\textbf{Q} ~отделение (строчки 4, 11)

(13)~\textbf{{]}} ~выталкивание

(14)~\textless\textless{}\textbf{P} \&\#923; \textbf{\textasciitilde P} \textgreater{} э \textbf{Q} \textgreater~фантазия

Эта теорема имеет очень странную частичную интерпретацию:

Из~\textbf{P} и не~\textbf{P} вместе взятых следует \textbf{Q} .

Поскольку \textbf{Q} можно интерпретировать любым предложением, мы можем считать, что эта теорема утверждает, что «из противоречия следует что угодно»! Поэтому системы, основанные на исчислении высказываний, не могут содержать противоречий --- противоречия мгновенно заражают всю систему, подобно глобальному раку.

Подход к разрешению противоречий

Это не похоже на человеческую мысль. Если вы обнаружите в своих рассуждениях противоречие; вряд ли это разрушит все здание вашего мышления. Вместо этого вы, скорее всего, попытаетесь найти те идеи или методы рассуждения, которые явились причиной противоречия. Иными словами, вы попытаетесь, насколько это возможно, выйти из ваших внутренних систем, приведших к противоречию, и попробуете их исправить. Маловероятно, что вы поднимете руки вверх и воскликнете: «Это показывает, что теперь я верю во все, что угодно!» --- разве что в шутку.

В действительности, противоречия --- это важный источник прогресса во всех областях жизни, и математика не является исключением. В прошлом, когда в математике обнаруживалось противоречие, математики тут же пытались найти виновную в этом систему, выйти из таковой, проанализировать ее и, если возможно, залатать прореху. Вместо того, чтобы делать математику слабее, нахождение и «починка» противоречивых систем только усиливали ее. Этот путь был долог и усеян ошибками, но в конце концов, он приносил плоды. Например, в средневековье предметом горячих споров была бесконечная последовательность

1-1 + 1-1 + 1-\ldots{}

Существовали «доказательства», что эта серия равняется 0, 1, 1/2 --- а может быть, и другим числам. Из подобных противоречивых результатов выросла более полная и глубокая теория бесконечных рядов. Более актуальный пример --- противоречие, с которым мы сталкиваемся в данный момент; это противоречие между тем, как мы действительно думаем, и тем, как исчисление высказываний имитирует наше мышление. Это продолжает быть источником дискомфорта для многих логиков; множество творческих усилий было приложено к тому, чтобы улучшить исчисление высказываний, чтобы оно не было таким жестким. Одна из попыток, изложенная в книге А. Р. Андерсона и Н. Белнапа «Следствие» (A.R. Anderson \& N.Belnap, «Entailment»),\footnote{A. R. Anderson and N. D. Belnap Jr. «Entailment» (Princeton N J Princeton University Press 1975)} включает «уместный подтекст», с тем, чтобы придать символу «если --- то» действительную причинность или, по крайней мере, некоторую связь со смыслом. Взгляните на следующие теоремы исчисления высказываний:

\textless{}\textbf{P} э \textless{}\textbf{Q} э \textbf{P} \textgreater\textgreater{}

\textless{}\textbf{P} э\textless{}\textbf{Q} V \textbf{\textasciitilde Q} \textgreater\textgreater{}

\textless\textless{}\textbf{Р} \&\#923; \textbf{\textasciitilde Р} \textgreater{} э \textbf{Q} \textgreater{}

\textless\textless{}\textbf{P} э \textbf{Q} \textgreater V\textless{}\textbf{Q} э \textbf{P} \textgreater\textgreater{}

Эти и другие подобные теоремы показывают, что первая и вторая части суждений типа «если\ldots{} то» вовсе не должны иметь никакой связи для того, чтобы быть доказанными в исчислении высказываний. С другой стороны, «уместный подтекст» ставит некоторые ограничения на контекст, в котором может действовать правило вывода. Опираясь на интуицию, он говорит нам, что «что-то может быть выведено из чего-то только в том случае, если эти части как-то соотносятся между собой». Например, строка 10 в деривации выше была бы невозможна в данной системе --- и это, в свою очередь, заблокировало бы вывод строчки \textless\textless{}\textbf{P} \&\#923;~\textbf{\textasciitilde P} \textgreater э\textbf{Q} \textgreater.

Более радикальные попытки полностью отказываются от поисков непротиворечивости и полноты, пытаясь взамен симулировать человеческое мышление со всеми его противоречиями. Подобные исследования уже не ставят своей целью дать математике прочный фундамент; они занимаются изучением процесса человеческой мысли.

Несмотря на некоторые странности, исчисление высказываний обладает многими положительными чертами. Если рассматривать его как часть большей системы (что мы и сделаем в следующей главе), и знать наверняка, что сама эта система свободна от противоречий (мы будем в этом уверены), то исчисление высказываний выполняет все, чего мы можем от него ожидать: оно производит все возможные правильные умозаключения. Даже если противоречие все-таки будет обнаружено, мы можем быть уверены, что виновница этого --- сама большая система, а не ее подсистема --- исчисление высказываний.

\emph{Рис. 42. М. К. Эшер «Крабий канон» (1965)}


% % \subsubsection{Крабий канон}
% \subsubsection{Крабий канон}

\emph{В один прекрасный день, Ахилл и Черепаха, прогуливаясь по парку, наталкиваются друг на друга.}

\emph{Черепаха} : Приветствую, г-н А.!

\emph{Ахилл} : И я вас тоже.

\emph{Черепаха} : Всегда рада вас видеть.

\emph{Ахилл} : Вы читаете мои мысли.

\emph{Черепаха} : В такой денек приятно пройтись; пожалуй, я пойду домой пешочком.

\emph{Ахилл} : Неужели? Гулять, знаете ли, весьма полезно для здоровья.

\emph{Черепаха} : Кстати, в последнее время вы выглядите как огурчик.

\emph{Ахилл} : О, благодарю вас.

\emph{Черепаха} : Не стоит. Не желаете ли угоститься моими сигарами?

\emph{Ахилл} : Да вы, как я погляжу, филистер. По моему мнению, голландский вклад в эту область --- значительно худшего вкуса, и я хочу попытаться вас в этом убедить.

\emph{Черепаха} : Наши мнения по этому вопросу расходятся. Кстати, говоря о вкусах: несколько дней назад я была на выставке, где, наконец, увидела «Крабий канон» вашего любимого художника, М. К. Эшера. Какая красота! Как ловко он переворачивает тему задом наперед! Но боюсь, что для меня Бах всегда останется выше Эшера.

\emph{Ахилл} : Не знаю, не знаю\ldots{} Я уверен только в том, что меня не волнуют споры о вкусах. De gustibus non est disputandum.

\emph{Черепаха} : Поговорим лучше о другом. Знаете ли вы, что я уже давно пытаюсь собрать полную коллекцию редких записей Баха --- хоть это и отнимает много времени, но я считаю, что лучшего хобби не найти.

\emph{Ахилл} : Ну и волокита! Не знаю, как кому-то могут нравиться такие вещи\ldots{}

\emph{(Вдруг, откуда ни возьмись, появляется Краб. Он стремительно подбегает к друзьям, указывая на огромный синяк под глазом.)}

\emph{Краб} : Приветик! Бонжурчик! Я сегодня как огурчик, только вот синяк --- кошмар, не правда ли? Мне его наставил этот поляк, ужасный, скажу вам, пошляк. Хо! Да еще в такой чудесный денек! Я себе по парку гулял, никого не задирал; вдруг слышу --- музыка небесная, полька расчудесная. Гляжу, а на скамье сидит девица, да такая, что нам с вами не пара; а в руках у нее --- гитара. Я и сам, знаете ли, из музыкальной семьи: мой кузен рак --- мужик не дурак! --- всегда зимовал ничуть не ближе, чем в Париже. Он был придворным музыкантом короля --- услаждал его величество художественным свистом, когда тот сидел с придворными за вистом. Любовь к музыке у нас, ракообразных, в крови\ldots{} Понимаете теперь, почему я не удержался, на скамейку взобрался, и говорю на ушко девице: «Щипать струны вы, гляжу, мастерица! Позвольте мне, как музыканту, сделать вам комплимент --- а также предложить свой аккомпанемент. Чтоб польке дать полнее звук, сыграем-ка в двенадцать рук!» Она как вскочит, да как завопит, что есть мочи! Тут откуда ни возьмись, явился этот здоровяк, этот поляк\ldots{} Бах! Трах! Прямо в глаз попал --- вот откуда этот фингал! Не думайте, что я трус --- атаковать я не боюсь. Но по давней семейной традиции, крабы --- мастера защитной диспозиции\ldots{} Ведь мы, когда идем вперед, движемся назад. Это у нас в генах --- переворачивать все задом наперед. Кстати, это мне напоминает\ldots{} Я всегда спрашивал себя: «Что было раньше, Краб или Ген?» То есть, я хочу сказать: «Что было позже, Ген или Краб»? Я всегда переворачиваю все задом наперед, знаете ли --- это у нас в генах. Ведь мы, когда идем вперед, движемся назад\ldots{} Ох, и заболтался же я, друзья! Да еще в такой чудный денек, хо! Поползу себе, пожалуй. Приветик!

\emph{(И он исчезает так же внезапно, как и появился.)}

\emph{Рис. 43. Кусочек одного из Крабьих Генов. Если спирали ДНК развернуть и положить рядом, то получится следующая картина: TTTTTTTCGAAAAAAA ... AAAAAAAGTTTTTTTT... Обратите внимание на то, что спирали одинаковы - разница только в том, что одна из них идет в обратном порядке. Эта черта определяет также музыкальную форму под названием ракоход, или «крабий канон.» Очень похожи на это и палиндромы --- предложения, которые при прочтении задом наперед дают точно тот же результат. В молекулярной биологии подобные сегменты ДНК называются «палиндромами» --- ко самом деле, более точным названием было бы «крабий канон». Этот сегмент ДНК не только «крабо-каноничен» --- в его основной структуре также закодирована структура Диалога. Присмотритесь повнимательней!}

\emph{Черепаха} : Ну и волокита! Не знаю, как кому-то могут нравиться такие вещи\ldots{}

\emph{Ахилл} : Поговорим лучше о другом. Знаете ли вы, что я уже давно пытаюсь собрать полную коллекцию редких гравюр Эшера --- хоть это и отнимает много времени, но я считаю, что лучшего хобби не найти.

\emph{Черепаха} : Не знаю, не знаю\ldots{} Я уверена только в том, что меня не волнуют споры о вкусах. Disputandum non est de gustibus.

\emph{Ахилл} : Наши мнения по этому вопросу расходятся. Кстати, говоря о вкусах: несколько дней назад я был на концерте, где наконец, услышал «Крабий канон» вашего любимого композитора, И. С. Баха. Какая красота! Как ловко он переворачивает тему задом наперед! Но боюсь, что для меня Эшер всегда останется выше Баха.

\emph{Черепаха} : Да вы, как я погляжу, филистер. По моему мнению, голландский вклад в эту область --- значительно худшего вкуса, и я хочу попытаться вас в этом убедить.

\emph{Ахилл} : Не стоит. Не желаете ли угоститься моими сигарами?

\emph{Черепаха} : О, благодарю вас.

\emph{Ахилл} : Кстати, в последнее время вы выглядите как огурчик.

\emph{Черепаха} : Неужели? Гулять, знаете ли, весьма полезно для здоровья.

\emph{Ахилл} : В такой денек приятно пройтись; пожалуй, я пойду домой пешочком.

\emph{Черепаха} : Вы читаете мои мысли.

\emph{Ахилл} : Всегда рад вас видеть.

\emph{Черепаха} : И я вас тоже.

\emph{Ахилл} : Приветствую, г-жа Ч.

\emph{РИС. 44. «Крабий канон»~из «Музыкального приношения» И. С. Баха.}


% % \subsubsection{ГЛАВА VIII: Типографская теория чисел}
% \subsubsection{ГЛАВА VIII: Типографская теори чисел}

«Крабий Канон» и косвенная автореференция

В «КРАБЬЕМ КАНОНЕ» есть три примера косвенной автореференции. Ахилл и Черепаха описывают известные им произведения искусства --- и по случайному совпадению оказывается, что эти произведения построены по той же схеме, как и диалог, в котором они упоминаются. Вообразите мое удивление, когда я, автор, сам это заметил! Более того, краб описывает биологическую структуру, которая тоже имеет подобные свойства. Разумеется, можно прочитать и понять диалог, не заметив при этом, что он сделан в форме ракохода --- но это было бы пониманием диалога только на одном уровне. Чтобы увидеть автореференцию, надо обратить внимание как на содержание, так и на форму диалога.

Построение Гёделя состоит из описания как формы, так и содержания строчек формальной системы, которую мы опишем в этой главе --- \emph{Типографской Теории Чисел} . Неожиданный поворот состоит в том, что при помощи хитроумного отображения, открытого Гёделем, форма строчек может быть описана в самой формальной системе. Давайте же познакомимся с этой странной системой, способной взглянуть сама на себя.

Что мы хотим выразить в ТТЧ

Для начала приведем некоторые высказывания, типичные для теории чисел; затем постараемся найти основные понятия, в терминах которых эти высказывания могут быть перефразированы. Далее эти понятия будут заменены индивидуальными символами. Необходимо заметить, что, говоря о теории чисел, мы имеем в виду только свойства положительных целых чисел и нуля (и множеств подобных чисел). Эти числа называются \emph{натуральными числами} . Отрицательные числа не играют в этой теории никакой роли. Таким образом, слово «число» будет относиться исключительно к натуральным числам. Очень важно для вас, читатель, помнить о разнице между формальной системой (ТТЧ) и удобной, хотя и не очень строго определенной, старой ветвью математики --- самой теорией чисел; я буду называть последнюю «Ч».

Вот некоторые типичные высказывания Ч --- теории чисел:

(1) 5 --- простое число.

(2) 2 не является квадратом другого числа.

(3) 1729 --- сумма двух кубов.

(4) Сумма двух положительных кубов сама не является кубом.

(5) Существует бесконечное множество простых чисел.

(6) 6 --- четное число.

Кажется, что нам понадобится символ для каждого из таких понятий, как «простое число», «куб» или «положительное число» --- однако эти понятия, на самом деле, не примитивны. Например, «простота» числа зависит от его множителей, которые, в свою очередь, зависят от умножения. Кубы также определяются в терминах умножения. Давайте постараемся перефразировать те же высказывания в более элементарных терминах.

(1) Не существует чисел \emph{а} и \emph{b} ~больших единицы, таких, что 5 равнялось бы \emph{а\&\#215;b}

(2) Не существует такого числа \emph{b} , что~\emph{b\&\#215;b} равнялось бы 2.

(3) Существуют такие числа~\emph{b} и \emph{с} , что \emph{b\&\#215;b\&\#215;b + с\&\#215;с\&\#215;с} равняется 1729.

(4) Для любых чисел~\emph{b} и \emph{с} больше нуля не существует такого числа \emph{а} , что \emph{а\&\#215;а\&\#215;а = b\&\#215;b\&\#215;b + с\&\#215;с\&\#215;с} .

(5) Для каждого \emph{а} существует \emph{b} , большее, чем \emph{а} , такое, что не существует чисел \emph{c} и \emph{d} , больших 1 и таких, что \emph{b} равнялось бы \emph{c\&\#215;d} .

(6) Существует число~\emph{e} такое, что 2\&\#215;\emph{e} равняется 6.

Этот анализ продвинул нас на пути к основным элементам языка теории чисел. Очевидно, что некоторые фразы повторяются снова и снова:

для всех чисел \emph{b} существует число \emph{b} , такое, что больше чем равняется умноженное на О, 1, 2,\ldots{}

Большинство таких фраз получат индивидуальные символы. Исключением является «больше чем», которое может быть упрощено еще. Действительно, высказывание «\emph{а} больше \emph{b} » становится:

существует число с отличное от 0, такое, что \emph{а = b + с} .

Символы чисел

Мы не будем вводить отдельного символа для каждого из натуральных чисел. Вместо этого у нас будет очень простой способ приписать каждому натуральному числу составной символ, так, как мы делали это в системе \textbf{pr} . Вот наше обозначение натуральных чисел.

нуль 0

один S0

два SS0

три SSS0

и т. д.

Символ S интерпретируется как «следующий за.» Таким образом, строчка SS0 интерпретируется буквально как «следующий за следующим за нулем.» Подобные строчки называются \emph{символами чисел} .

Переменные и термины

Ясно, что нам нужен способ говорить о неопределенных, или переменных числах. Для этого мы будем использовать буквы \emph{а, b, с, d, e} . Однако пяти букв будет недостаточно Так же, как для атомов в исчислении высказывании, нам требуется их неограниченное количество Мы используем похожий метод для получения большего количества переменных --- добавление любого количества штрихов. Например:

e

d'

с"

b'''

a''''

все являются переменными.

В каком-то смысле, использовать целых пять букв алфавита --- это слишком большая роскошь, так как мы могли бы легко обойтись просто буквой \emph{а} и штрихами. Впоследствии я действительно опущу буквы \emph{b} ,\emph{c} ,\emph{d} , и \emph{e} --- результатом будет более строгая версия \textbf{ТТЧ} , сложные формулы которой будет немного труднее расшифровать. Но пока давайте позволим себе некоторую роскошь! Как насчет сложения и умножения? Очень просто: мы будем использовать обычные символы «+» и «*». Однако мы также введем требование скобок (мы мало помалу углубляемся в правила, определяющие правильно построенные строчки \textbf{ТТЧ} ). Например, чтобы записать «\emph{b} плюс \emph{с} » и «\emph{b} , умноженное на \emph{с} », мы будем использовать строчки:

(\emph{b} + \emph{с} )

(\emph{b*с} )

В отношении скобок послабления быть не может; опустить их --- значит произвести неправильно сформированную формулу. («Формула?» Я использую этот термин вместо слова «строчка» лишь для удобства. \emph{Формула} --- это просто строчка \textbf{ТТЧ} .)

Кстати, сложение и умножение всегда будут рассматриваться как бинарные операции, то есть операции, объединяющие не более, чем два числа. Таким образом, если вы хотите записать~«1+2+3», вы должны решить, какое из двух выражений использовать:

(S0+(SS0+SSS0))

((S0+SS0)+SSS0)

Теперь давайте символизируем понятие \emph{равенства} . Для этого мы просто используем «=». Преимущество этого символа, принадлежащего \textbf{Ч} --- неформальной теории чисел --- очевидно: его весьма легко прочесть. Неудобство же при его использовании напоминает проблему, возникавшую при использовании слов «точка» и «линия» в формальном описании геометрии: если ослабить внимание, то легко спутать обыденное значение этих слов с поведением символов, подчиняющихся строгим правилам. Обсуждая проблемы геометрии, я различал между обыденными словами и терминами ---~последние печатались заглавными буквами. Так, в эллиптической геометрии ТОЧКОЙ было объединение двух точек. Здесь такого различия не будет, поэтому читатель должен постараться не спутать символ с многочисленными ассоциациями, которые он вызывает. Как я сказал ранее о системе \textbf{pr} , строчка \textbf{-\/-\/-} не является числом 3; вместо этого она действует изоморфно с числом 3, по крайней мере, при сложении. То же самое можно сказать и о строчке SSS0.

Атомы и символы высказываний

Все символы исчисления высказываний, кроме букв, с помощью которых мы получали атомы (\textbf{P} , \textbf{Q} , \textbf{R} ), будут использованы в \textbf{ТТЧ} ; при этом они сохранят ту же интерпретацию. Роль атомов будут играть строчки, которые, будучи интерпретированы, дадут равенства, такие как S0=SS0 или (S0\&\#215;S0) = S0. Теперь у нас есть достаточно данных, чтобы перевести несколько простых суждений в запись ТТЧ:

2+3 равняется 4: (SS0+SSS0)=SSSS0

2+2 не равняется 3: \textasciitilde(SS0+SS0)=SSS0

Если 1 равняется 0, то~0 равняется 1:~\textless S0=0э0=S0\textgreater{}

Первая из этих строчек --- атом; остальные --- составные формулы. (Внимание: «и» во фразе~«1 и 1 будет 2» --- всего лишь еще одно обозначение «плюса» и должно быть представлено «+» (и необходимыми скобками).

Свободные переменные и кванторы

Все правильно сформированные строчки, приведенные выше, обладают следующим свойством: их интерпретация --- либо истинное, либо ложное высказывание. Однако существуют правильно сформированные формулы, не обладающие этим свойством, такие, например, как:

(b+S0)=SS0

Ее интерпретация --- «\emph{b} плюс 1 равняется 2». Поскольку~\emph{b} не определено, то невозможно сказать, истинно ли данное высказывание. Это напоминает высказывание с местоимением, взятое отдельно от контекста, такое, как «Она неуклюжа.» Это высказывание не истинно и не ложно --- оно просто ждет, чтобы его поставили в контекст. Поскольку она не истинна и не ложна, подобная формула зовется \emph{открытой} , а переменная b называется \emph{свободной переменной} .

Одним из способов превратить открытую формулу в замкнутую формулу или высказывание является добавление \emph{квантора} --- фразы «существует число b такое, что\ldots» или фразы «для всех \emph{b} ». В первом случае, у вас получается высказывание:

Существует число~\emph{b} такое, что~\emph{b} плюс 1 равняется 2.

Ясно, что это истинно. Во втором случае, вы получите:

Для всех чисел \emph{b} ,~\emph{b} плюс 1 равняется 2.

Ясно, что это ложно. Теперь мы введем символы для обоих кванторов. Два высказывания, приведенные выше, в ТТЧ будут выглядеть как:

Eb:(b+S0)=SS0~(~\textbf{E} ~означает «существует»)

Ab:(b+S0)=SS0~(~\textbf{A} ~означает «все»)

Важно отметить, что речь идет уже не о неопределенных числах; первое высказывание --- это утверждение существования, второе --- утверждение общности. Их значение не изменится, даже если мы заменим~\emph{b} на~\emph{c:}

Ec:(c+S0)=SS0

Ac:(c+S0)=SS0

Переменная, управляемая квантором, называется \emph{квантифицированной переменной} . Две следующие формулы иллюстрируют разницу между свободной и квантифицированной переменной.

(b*b)=SS0~~~(открытая)

\textasciitilde Eb:(b*b)=SS0~~~(замкнутая - высказывание ТТЧ)

Первая формула выражает свойство, которое может быть присуще какому-либо натуральному числу. Разумеется, такого числа не существует. Этот факт выражен второй формулой. Очень важно понять разницу между строчкой со \emph{свободной} переменной и строчкой, в которой переменная \emph{квантифицирована} . Последняя строчка --- либо \emph{истинна} , либо \emph{ложна} . В переводе на русский язык, строчка, где есть по крайней мере одна свободная переменная, называется предикатом. Предикат --- это высказывание без подлежащего (или с подлежащим, выраженным местоимением, лишенным контекста). Например, высказывания:

«является предложением без подлежащего»

«было бы аномалией»

«читается вперед и назад одновременно»

«сымпровизировал по требованию шестиголосную фугу»

являются неарифметическими предикатами. Они выражают \emph{свойства} , которыми обладают или не обладают определенные предметы или существа. С тем же успехом мы могли бы добавить «подлежащее-пустышку», например, «такой-то». Строчка со свободными переменными подобна такому предикату с подлежащим-пустышкой. Например:

(S0+S0)=\emph{b}

означает~«1 плюс 1 равняется чему-то». Это предикат с переменной \emph{b} . Он выражает свойство, которым может обладать число \emph{b} . Заменяя~\emph{b} на различные числа, мы получили бы последовательность формул, большинство которых выражало бы ошибочные суждения. Вот еще один пример разницы между открытыми формулами и высказываниями:

A\emph{b} :A\emph{c} :(\emph{b} +\emph{c} )=(\emph{c} +\emph{b} )

Эта формула, разумеется, выражает коммутативность сложения. С другой стороны:

A\emph{c} :(\emph{b} +\emph{c} )=(\emph{c} +\emph{b} )

---~это открытая формула, поскольку \emph{b} здесь свободно. Она выражает свойство, которым может обладать или не обладать число \emph{b} , а именно --- коммутативность по отношению ко всем числам \emph{с} .

Примеры перевода высказываний

Теперь наш словарь, с помощью которого мы сможем выразить все высказывания теории чисел, полон. Чтобы научиться выражать сложные утверждения Ч и, наоборот, понимать значение правильно сформированных формул, необходимо много практиковаться. Поэтому мы обратимся к шести простым высказываниям, данным в начале, и попробуем перевести их на язык ТТЧ. Кстати, не думайте, что переводы, которые вы найдете далее, единственно возможные. На самом деле, существует бесконечное множество способов выразить каждое высказывание в ТТЧ.

Начнем с последнего высказывания:~«6 --- четное число». Переведем его в

более примитивные понятия: «Существует число \emph{e} , такое, что 2, умноженное на \emph{e} , равняется 6.» Это легко перевести в нотацию ТТЧ:

Ee:(SS0*e)=SSSSSS0

Обратите внимание на необходимость квантора; недостаточно было бы написать просто:

(SS0*e)=SSSSSS0

Интерпретация последней строчки не была бы ни истинной, ни ложной; она выражает только свойство, которое может иметь число \emph{e} .

Интересно, что, поскольку мы знаем, что умножение коммутативно, то вместо нашей строчки мы могли бы написать:

Ee:(e*SS0)=SSSSSS0

Таким же образом, зная, что равенство симметрично, мы могли бы поменять местами стороны этого равенства:

Ee:SSSSSS0=(SS0*e)

Эти три перевода высказывания~«6 --- четное число» дают три различные строчки; при этом совершенно не очевидно, что теоремность каждой из них связана с теоремностью остальных. (Совершенно так же тот факт, что строчка \textbf{-p-\/-r-\/-\/-} была теоремой, почти не соотносился с фактом, что ее «эквивалентная» строчка~\textbf{-\/-p-r-\/-\/-} также была теоремой. Эквивалентность --- у нас в голове, так как мы, люди, автоматически думаем об интерпретациях формул, а не об их структурных особенностях.)

С высказыванием 2:~«2 не является квадратом» мы расправимся быстро:

\textasciitilde E\emph{b} :(\emph{b} *\emph{b} )=SS0

Однако здесь мы снова сталкиваемся с двусмысленностью. А что, если бы мы записали эту формулу по-другому?

A\emph{b} :\textasciitilde(b*b)=SS0

Интерпретация первой строчки --- «Не существует такого числа \emph{b} , квадрат которого равнялся бы 2»; вторая строчка читается как «Для всех чисел~\emph{b} неверно, что квадрат~\emph{b} равняется 2». Для нас эти строчки представляют одно и то же понятие --- однако для ТТЧ это совершенно разные строчки.

Посмотрим теперь на высказывание 3: «1729 --- сумма двух кубов». Тут нам понадобятся два квантора один за другим, вот так:

Eb:Ec:SSSSSS.....SSSSS0=(((b*b)*b)+((c*c)*c))

.~~~~~~~~~ \textbar-\/-1729 раза-\/-\textbar{}

Есть несколько альтернатив этой записи: можно переменить порядок кванторов --- сменить переменные на \emph{d} и \emph{e} ; переменить порядок слагаемых; записать умножение по-иному и т. д., и т. п. Однако я предпочитаю следующие два варианта перевода:

Eb:Ec:(((SSSSSSSSSS0*SSSSSSSSSS0)*SSSSSSSSSS0)+((SSSSSSSSS0*SSSSSSSSS0)*SSSSSSSSS0))=(((b*b)*b)+((c*с)*с))

и

Eb:Ec:(((SSSSSSSSSSSS0*SSSSSSSSSSSS0)*SSSSSSSSSSSS0)+((S0*S0)*S0))=(((b*b)*b)+((c*c)*c))

~Понимаете, почему?

Трюки ремесла

Давайте попробуем перевести теперь высказывание 4: «Сумма двух положительных кубов сама не является кубом». Предположим, что мы хотим сказать, что 7 не является суммой двух положительных кубов. Легче всего сделать это, \emph{отрицая} формулу, утверждающую обратное. Эта формула будет почти как предыдущая, касавшаяся 1729, только теперь нам надо добавить, что кубы должны быть положительными. Мы можем сделать это при помощи следующего трюка: добавим к каждой переменной префикс S:

Eb:Ec:SSSSSSS0=(((Sb*Sb)*Sb)+((Sc*Sc)*Sc))

Как видите, мы возводим в куб не сами~\emph{b} и \emph{c} , а следующие за ними числа, которые должны быть положительными, поскольку минимальная величина~\emph{b} и~\emph{c} --- 0. Таким образом, правая сторона представляет сумму двух положительных кубов. Кстати, обратите внимание, что перевод высказывания «существуют числа~\emph{b} и \emph{c} , такие, что\ldots» не включает символа «\&\#923;», заменяющего «и». Этот символ используется для соединения целых правильно сформированных строчек, а не для соединения двух кванторов.

Итак, мы перевели высказывание~«7 --- сумма двух положительных кубов»; теперь нам нужно записать его отрицание. Для этого мы должны только поставить тильду слева от него. (Заметьте, что не требуется отрицать каждый квантор в отдельности, хотя нам и надо получить высказывание «Не существует чисел~\emph{b} и \emph{c} , таких, что\ldots») Таким образом, мы получим:

\textasciitilde Eb:Ec:SSSSSSS0=(((Sb*Sb)*Sb)+((Sc*Sc)*Sc))

Однако нашей первоначальной целью было выразить свойства всех чисел, а не только 7. Для этого давайте заменим символ числа SSSSSSSO строчкой ((\emph{а*а} )*\emph{а} ), являющейся переводом «\emph{а} в кубе».

\textasciitilde Eb:Ec:((a*a)*a)=(((Sb*Sb)*Sb)+((Sc*Sc)*Sc))

Ha этом этапе у нас имеется открытая формула, так как а все еще свободно. Эта формула выражает свойство, которым может обладать или не обладать а --- однако мы хотим сказать, что все числа обладают этим свойством. Это просто --- надо только добавить к имеющейся у нас формуле квантор общности:

Aa:\textasciitilde Eb:Ec:((a*a)*a)=(((Sb*Sb)*Sb)+((Sc*Sc)*Sc))

Таким же правильным переводом было бы:

\textasciitilde Eа:Eb:Eс:((а*a)*a)=(((Sb*Sb)*Sb)+((Sc*Sc)*Sc))

В \emph{строгом} ТТЧ мы могли бы использовать \emph{a'} вместо \emph{b} и \emph{a''} вместо \emph{c} ; таким образом, формула приобрела бы вид:

\textasciitilde Ea:Ea':Ea'':((a*a)*a)=(((Sa'*Sa')*Sa')+((Sa''*Sa'')*Sa''))

Как насчет высказывания 1:~«5 --- простое число»? Мы перефразировали его следующим образом: «Не существует чисел~\emph{a} и~\emph{b} больших 1, таких, что 5 равнялось бы~\emph{a} умноженному на \emph{b} .» Теперь мы можем это немного изменить: «Не существует чисел \emph{а} и~\emph{b} таких, что 5 равнялось бы \emph{а} плюс 2 умноженному на~\emph{b} плюс 2.» Это еще один трюк- поскольку \emph{а} и~\emph{b} здесь --- натуральные числа, эта формулировка кажется более адекватной. Далее, «\emph{b} +2» может быть переведено как (\emph{b} +SS0), но есть и более короткий способ записать то же самое: SS\emph{b} . Точно так же, «\emph{c} плюс 2» может быть записано как SS\emph{c} . Теперь наш перевод становится совсем коротким:

\textasciitilde Eb:Ec:SSSSS0=(SSb*SSc)

Без тильды в начале это было бы утверждением того, что существуют два натуральных числа, которые, если их увеличить на два, дают при умножении 5. Тильда в начале это отрицает; таким образом, мы получаем утверждение того, что 5 --- простое число.

Если бы вместо 5 мы хотели бы сказать то же самое про \emph{d} плюс~\emph{e} плюс 1, самым экономным способом было бы заменить символ числа 5 на строчку (\emph{d} +S\emph{e} ):

\textasciitilde Eb:Ec:(d+Se)=(SSb*SSc)

Мы снова получили открытую формулу; ее интерпретация --- не истина и не ложь, а лишь некое утверждение о каких-то двух числах \emph{d} и \emph{e} . Обратите внимание, что число, выраженное строчкой (\emph{d} +S\emph{e} ), больше \emph{d} , так как мы добавили к \emph{d} хотя и неопределенную, но положительную величину. Таким образом, если мы добавим к переменной~\emph{e} квантор существования, мы получим формулу, утверждающую, что

Существует некое простое число, большее \emph{d} .

Ee:\textasciitilde Eb:Ec:(d+Se)=(SSb*SSc)

Осталось только добавить, что это свойство верно всегда, вне зависимости от \emph{d} . Для этого мы должны добавить квантор общности для \emph{d} :

Ad:Ee:\textasciitilde Eb:Ec:(d+Se)=(SSb*SSc)

Перед нами ---~перевод высказывания 5!

Несколько задачек на перевод

Мы завершили упражнение на перевод шести типичных высказываний теории чисел. Однако это еще не гарантирует, что вы стали экспертом в нотации ТТЧ. Остается усвоить несколько тонкостей. Следующие шесть правильно сформированных формул послужат проверкой того, насколько вы овладели нотацией ТТЧ. Что эти формулы означают? Является ли их интерпретация истинными или ложными высказываниями? (Подсказка читателю: при работе с этим упражнением лучше всего двигаться справа налево. Сначала переведите атомы; затем подумайте, что получится, если добавить квантор или тильду; затем, двигаясь налево, добавьте еще один квантор или тильду; снова продвиньтесь налево и опять повторите этот процесс.)

\textasciitilde Ac:Eb:(SS0*b)=c

Ac:\textasciitilde Eb:(SS0*b)=c

Ac:Eb:\textasciitilde(SS0*b)=c

\textasciitilde Eb:Ac:(SS0*b)=c

Eb:\textasciitilde Ac:(SS0*b)=c

Eb:Ac:\textasciitilde(SS0*b)=c

(Еще одна подсказка, либо четыре из них истинны и два ложны, либо, наоборот, два истинны и четыре ложны.)

Как отличить истинное от ложного?

Теперь давайте на минуту остановимся и переведем дыхание --- а заодно подумаем, что означало бы иметь такую формальную систему, которая могла бы отличить все истинные высказывания от ложных. Для такой системы все эти строчки были бы просто некими формальными конструкциями, лишенными содержания (в то время как мы видим в них высказывания). Эта система была бы словно решето, сквозь которое проходили бы только конструкции определенного стиля --- «стиля истины». Если вы сами имели дело с шестью формулами выше и отделили истинные от ложных, размышляя об их значении, вы сможете оценить, насколько тонкой должна быть система, которая сможет проделать то же самое, но чисто типографским путем! Граница, отделяющая истинные высказывания от ложных (записанных нотацией ТТЧ) вовсе не пряма --- это граница со множеством предательских извилин (вспомните рис. 18). Математики смогли установить некоторые отрезки этой границы, работая над этим сотни лет. Представьте себе, как было бы здорово иметь типографский метод, который с гарантией мог бы поставить любую формулу по правильную сторону границы!

Правила для правильно-сформированности

Полезно иметь таблицу Правил Образования для правильно сформированных формул Такая таблица приведена ниже. На подготовительных этапах определяются \emph{символы чисел} , \emph{переменные} и \emph{термы} . Эти три класса строчек являются ингредиентами правильно сформированных формул, но сами они не являются правильно сформированными. Минимальные правильно сформированные формулы --- это атомы; существуют способы для соединения атомов. Многие из этих правил --- рекурсивные и удлиняющие: в качестве вводных данных они берут элемент определенного класса и производят более длинный элемент того же класса. В этой таблице я использую «\emph{x} » и «\emph{у} » как символы для правильно сформированных формул и «\emph{s} », «\emph{t} » и «\emph{u} » --- как символы для всех остальных строчек ТТЧ. Нет нужды говорить, что никакой из этих пяти символов сам по себе не является символом ТТЧ.

СИМВОЛЫ ЧИСЕЛ

0~--- это символ числа.

Символ числа, слева от которого стоит S --- также символ числа.

Примеры:~0 S0 SS0 SSS0 SSSS0 SSSSS0

ПЕРЕМЕННЫЕ

\emph{a} ~--- это переменная Если забыть об аскетизме, то \emph{b} , \emph{c} , \emph{d} , и~\emph{e} --- тоже переменные. Переменная со штрихом справа --- также переменная.

Примеры: \emph{а b' c" d''' e''''}

ТЕРМЫ

Термами являются символы чисел и переменные. Терм, слева от которого стоит S --- это также терм.

Если \emph{s} и \emph{t} --- термы, то (\emph{s} +\emph{t} ) и (\emph{s*t} ) --- также термы.

Примеры:~0~ \emph{b} ~SSa' ~(S0*(SS0)+c))~ S(Sa*(SbSc))

ТЕРМЫ могут быть подразделены на две категории:

(1) ОПРЕДЕЛЕННЫЕ термы. В них нет переменных.

Примеры:~0 ~(S0+S0) ~SS((S0*SS0)+(S0*S0))

(2) НЕОПРЕДЕЛЕННЫЕ термы. В них есть переменные.

Примеры: \emph{b} ~ Sa(b+S0) ~(((S0+S0)+S0)+e)

Приведенные выше правила объясняют нам, как получить части правильно сформированных формул; остальные правила говорят нам, как получить полные правильно сформированные формулы.

АТОМЫ

Если \emph{s} ~и \emph{t} --- термы, то \emph{s} +\emph{t} --- атом.

Примеры: S0=0~ (SS0+SS0)=SSSS0~ S(\emph{b} +\emph{c} )=((\emph{c*d} )*\emph{e} )

Если атом содержит переменную \emph{u} , то~\emph{u} в нем свободна.

Таким образом, в последнем примере есть четыре свободных переменных.

ОТРИЦАНИЯ.

Правильно сформированная формула перед которой стоит тильда также правильно сформирована.

Примеры:~\textasciitilde S0=0 ~ \textasciitilde Eb:(b+b)=S0 ~ \textasciitilde\textless0=0эS0=0\textgreater{} ~ \textasciitilde b=S0

\emph{Кванторный статус} переменной (говорящий нам, свободна или квантифицирована эта переменная) не меняется при отрицании.

СОСТАВНЫЕ.

Если \emph{x} и \emph{у} --- правильно сформированные формулы и при этом ни одна переменная, свободная в одной из них, не квантифицирована в другой, тогда все следующие формулы правильно сформированы:~\textless{}\emph{x} \&\#923; \emph{y} \textgreater, \textless{}\emph{x} V \emph{y} \textgreater,\textless{}\emph{x} э \emph{y} \textgreater{}

Примеры:~\textless0=0э\textasciitilde0=0\textgreater{} ~~ \textless b=bV\textasciitilde Ec:c=b\textgreater{} ~~ \textless S0=0эAc:\textasciitilde Еb:(b+b)=c\textgreater{}

Кванторный статус переменной здесь не меняется.

КВАНТИФИКАЦИЯ.

Если~\emph{u} --- переменная и~\emph{x} --- правильно сформированная формула, в которой и свободна, то следующие строчки --- также правильно сформированные формулы:Eu:\emph{x} и Au:\emph{x}

Примеры: Ab:\textless b=bV\textasciitilde Ec:c=b\textgreater{} ~~ Ac:\textasciitilde Eb:(b+b)=c ~~ \textasciitilde Еc:Sc=d

ОТКРЫТЫЕ ФОРМУЛЫ содержат по крайней мере одну свободную переменную.

Примеры: \textasciitilde c=c ~b=b~ \textless Ab:b=b\&\#923;\textasciitilde c=c\textgreater{}

ЗАМКНУТАЯ ФОРМУЛА (суждение) не содержит свободных переменных.

Примеры: S0=0~ \textasciitilde Ad:d=0~ Ec:\textless Ab:b=b\&\#923;\textasciitilde c=c\textgreater{}

Это дает нам полную таблицу Правил Образования для правильно сформированных формул ТТЧ.

Еще несколько упражнений на перевод

Вот еще несколько упражнений для вас, чтобы проверить, насколько вы поняли нотацию ТТЧ. Попробуйте перевести первые четыре из приведенных ниже высказываний Ч в высказывания ТТЧ, а последнее --- в открытую правильно сформированную формулу.

Все натуральные числа равны 4.

Ни одно натуральное число не равно собственному квадрату.

Различные натуральные числа имеют различные последующие элементы.

Если 1 равняется 0, то любое число нечетно.

\emph{b} --- это степень 2.

Последнее может показаться вам трудным. Однако это еще цветочки по сравнению со следующим:

\emph{b} ~--- это степень 10.

Как это ни странно, чтобы записать это выражение в нашей нотации, требуется большая ловкость. Приступайте к нему только в том случае, если вы готовы просидеть над ним несколько часов --- и если при этом вы уже хорошо знакомы с теорией чисел.

Нетипографская система

Мы изложили нотацию ТТЧ; остается только превратить ТТЧ в ту амбициозную систему, которую мы только что описали. Если нам это удастся, это будет значить, что интерпретация, которую мы дали символам, была правильна. До тех пор, однако, наши интерпретации не более оправданы, чем интерпретация «лошадь --- яблоко --- счастливая» для символов системы \textbf{pr} .

Можно было бы предложить следующий способ для построения ТТЧ: (1) Не использовать никаких правил вывода --- они не нужны, так как (2) мы будем считать за аксиомы все истинные суждения теории чисел (записанные нотацией ТТЧ). Какой простой рецепт! К несчастью, он начисто лишен смысла, как нам и подсказывает наша первая реакция. Часть (2), разумеется, не является типографским описанием строчек, в то время как целью ТТЧ является именно типографское описание истинных высказываний.

Пять аксиом и первое правило ТТЧ

Таким образом, нам придется отказаться от простого рецепта, предложенного выше, и пойти по более сложному пути: у нас будут аксиомы и правила вывода. Прежде всего, как было обещано, \emph{все правила исчисления высказываний будут использованы} в ТТЧ. Итак, первой теоремой ТТЧ будет следующая:

\textless S0 =~0~V \textasciitilde{} S0 = 0\textgreater{}

Она может быть выведена так же, как \textless{}\textbf{P} ~V \textbf{\textasciitilde{} P} \textgreater. Прежде чем приводить правила, давайте запишем пять аксиом. ТТЧ:

АКСИОМА 1: Aa:\textasciitilde Sa=0

АКСИОМА 2: Aa:(a+0)=a

АКСИОМА 3: Aa:Ab:(a+Sb)=S(a+b)

АКСИОМА 4: Aa:(a*0)=0

АКСИОМА 5: Aa:Ab:(a*Sb)=((a*b)+a)

(В строгой версии вместо b используйте a'.) Все они очень просты. Аксиома 1 сообщает что-то о числе 0; аксиомы 2 и 3 говорят о свойствах сложения; аксиомы 4 и 5 говорят о свойствах умножения и о его отношении к сложению.

Пять постулатов Пеано

Интерпретация первой аксиомы --- «Нуль не следует ни за каким натуральным числом» --- это одно из пяти знаменитых свойств натуральных чисел, впервые выраженных математиком и логиком Джузеппе Пеано в 1889 году. Излагая свои постулаты, Пеано следовал за Эвклидом в том смысле, что он не пытался формализовать принципы логических рассуждений. Вместо этого он хотел дать небольшой набор свойств натуральных чисел, из которого можно было бы вывести все остальные путем логических рассуждений. Таким образом, попытка Пеано может быть названа «полуформальной.» Работа Пеано оказала на математиков большое влияние, поэтому я приведу здесь его постулаты. Поскольку Пеано пытался определить именно «натуральное число», мы не будем использовать знакомый и вызывающий ассоциации термин «натуральное число» ---~вместо него мы будем пользоваться неопределенным термином \emph{гений} --- словом свежим и свободным от математических ассоциаций. Итак, пять постулатов Пеано устанавливают пять ограничений для гениев. Другие неопределенные термины, которыми мы будем пользоваться --- \emph{джинн} и \emph{мета} . Читатель может догадаться сам, какие знакомые понятия скрываются за этими двумя терминами. Далее следуют пять постулатов Пеано:

(1) Джинн --- это Гений.

(2) Каждый Гений имеет мету (которая тоже является Гением).

(3) Джинн не является метой никакого Гения.

(4) Различные Гении имеют различные меты.

(5) Если джинн имеет X и каждый Гений передает X своей мете, тогда все Гении получают X.

В свете ламп «Маленького гармонического лабиринта» мы должны наименовать множество всех Гениев «БОГом». Это напоминает нам о знаменитом высказывании немецкого математика и логика Леопольда Кроникера, архиврага Георга Кантора: «Бог сотворил натуральные числа; все остальное --- работа человека.»

Вы можете узнать в пятом постулате Пеано принцип математической индукции --- другой термин для «наследственного» доказательства. Пеано надеялся, что его ограничения понятий «джинна», «Гения» и «меты» были так сильны, что эти понятия были бы идентичны для всех людей и формировали бы у них в сознании совершенно \emph{изоморфные структуры} . Например, для любого человека существовало бы бесконечное число различных Гениев. И, предположительно, каждый согласился бы с тем, что ни один Гений не совпадает со своей метой или мета-метой\ldots{} и т. д.

В своих пяти постулатах Пеано хотел выразить сущность натуральных чисел. Математики обычно считают, что ему это удалось; однако это не уменьшает важности вопроса «каким образом можно отличить истинное высказывание о натуральных числах от ложного?» Ответа на этот вопрос математики ищут в формальных системах, подобных ТТЧ. Вы найдете в ТТЧ влияние Пеано, поскольку все его постулаты так или иначе вошли в эту систему.

Новые правила ТТЧ: спецификация и обобщение

Мы подошли к новым правилам ТТЧ. Многие из них позволят нам забраться внутрь этой системы и поменять внутреннюю структуру ее атомов. В этом смысле эти правила имеют дело с «микроскопическими» особенностями строчек в большей степени, чем правила исчисления высказываний, обращающиеся с атомами как с неделимыми. Например, было бы хорошо, если бы мы могли выделить строчку \textasciitilde S0=0 из первой аксиомы. Для этого нам понадобилось бы правило, позволяющее опустить общий квантор и при необходимости одновременно поменять внутреннюю структуру остающейся строчки. Вот это правило:

ПРАВИЛО СПЕЦИФИКАЦИИ. Предположим, что~\emph{u} --- переменная, встречающаяся внутри строчки \emph{x} . Если строчка~Au:\emph{x} ~ --- теорема, то~\emph{x} --- тоже теорема, как и все строчки, получающиеся из~\emph{x} путем замены и на любой (один и тот же) терм.

(Ограничение: Терм, заменяющий и, не должен содержать никакой переменной, квалифицированной в \emph{x} .)

Правило спецификации позволяет нам выделить нужную строчку из Аксиомы

1. Это одноступенчатая деривация:

Aa:\textasciitilde Sa=0~ аксиома 1

\textasciitilde S0=0~ спецификация

Обратите внимание, что правило спецификации позволяет некоторым формулам, содержащим свободные переменные (то есть, открытым формулам), стать теоремами. Например, следующие строчки также могут быть выведены из аксиомы 1 при помощи спецификации:

\textasciitilde Sa=0

\textasciitilde S(c+SS0)=0

Существует еще одно правило, правило обобщения, которое позволяет нам снова ввести квантор общности в теоремы с переменными, ставшими свободными в результате спецификации. Например, действуя на строчку низшего порядка, обобщение дало бы:

Ac:\textasciitilde S(c+SS0)=0

Обобщение уничтожает сделанное спецификацией, и наоборот. Обычно обобщение применяется после того, как были сделаны несколько промежуточных шагов, трансформировавших открытую формулу разными способами. Далее следует точная формулировка этого правила:

ПРАВИЛО ОБОБЩЕНИЯ: Предположим, что~\emph{x} --- теорема, в которой встречается свободная переменная \emph{u} . Тогда Au:\emph{x} --- тоже теорема.

(Ограничение: к переменным, которые встречаются в свободном виде в посылках фантазий, обобщение неприложимо.)

Вскоре я ясно покажу, почему эти два правила нуждаются в ограничениях. Кстати, это обобщение --- то же самое, о котором я упомянул в главе II в Эвклидовом доказательстве бесконечного количества простых чисел. Уже отсюда видно, как правила, манипулирующие символами, начинают приближаться к типу рассуждений, используемых математиками.

Квантор существования

Два предыдущих правила объяснили нам, как можно убрать квантор общности и вернуть его на место; следующие два правила покажут, как обращаться с кванторами существования.

ПРАВИЛО ОБМЕНА: Представьте, что~\emph{u} --- переменная. Тогда строчки Au:\textasciitilde{} и \textasciitilde Eu: взаимозаменимы везде внутри системы.

Давайте, например, применим это правило к аксиоме 1:

Aa:\textasciitilde Sa=0~ аксиома 1

\textasciitilde Ea:Sa=0~ обмен

Кстати, вы можете заметить, что обе эти строчки --- естественный перифраз в ТТЧ высказывания «Нуль не следует ни за одним из натуральных чисел.» Следовательно, хорошо, что они могут быть с легкостью превращены одна в другую.

Следующее правило еще более интуитивно. Оно соответствует весьма простому типу рассуждений, который мы употребляем, переходя от утверждения~«2 --- простое число» к утверждению «существует простое число.» Название этого правила говорит само за себя:

ПРАВИЛО СУЩЕСТВОВАНИЯ: Предположим, что некий терм (могущий содержать свободные переменные), появляется один или много раз в теореме. Тогда каждый (или несколько, или все) из этих термов может быть заменен на переменную, которая больше нигде в теореме не встречается, и предварен соответствующим квантором существования.

Давайте применим, как обычно, это правило к аксиоме 1:

Aa:\textasciitilde Sa=0~~аксиома 1~

Eb:Aa:\textasciitilde Sa=b ~существование

Вы можете поиграть с символами и при помощи данных правил получить теорему: \textasciitilde Ab:Ea:Sa=b

Правила равенства и следствия

Мы привели правила, объясняющие, как обращаться с кванторами --- но пока ни одно из них не сказало нам, как обращаться с символами «=» и «S». Сейчас мы это сделаем; в следующих правилах \emph{r} , \emph{s} и \emph{t} --- произвольные термы.

ПРАВИЛА РАВЕНСТВА:

СИММЕТРИЯ: Если~\emph{r} = \emph{s} --- теорема, то \emph{s} =~\emph{r} также является теоремой.

ТРАНЗИТИВНОСТЬ: Если~\emph{r} = \emph{s} ~и \emph{s} = \emph{t} --- теоремы, то~\emph{r} = \emph{t} также является теоремой.

ПРАВИЛА СЛЕДОВАНИЯ:

ДОБАВЛЕНИЕ S: Если \emph{r} = \emph{t} --- теорема, то S\emph{r} = S\emph{t} также является теоремой.

ВЫЧИТАНИЕ S: Если S\emph{r} = S\emph{t} --- теорема, то~\emph{r} = \emph{t} также является теоремой.

Теперь у нас есть правила, которые могут дать нам фантастическое разнообразие теорем. Например, результатом следующих дериваций являются фундаментальные теоремы:

(1) Aa:Ab:(a+Sb)=S(a+b)~~~~ аксиома 3

(2) Ab:(S0+Sb)=S(S0+b)~~~~~~спецификация (S0 для а)

(3) (S0+S0)=S(S0+0) спецификация (0 для b)

(4) Aa:(a+0)=a ~~~ аксиома 2

(5) (S0+0)=S0 ~~~ спецификация (S0 для а)

(6) S(S0+0)=SS0 ~~ добавление S

(7) (S0+S0)=SS0 ~~ транзитивность (строчки 3,6)

*****

(1) Aa:Ab:(a*Sb)=((a*b)+a) ~~ аксиома 5

(2) Ab:(S0*Sb)=((S0*b)+S0) ~~ спецификация (S0 для а)

(3) (S0*S0)=((S0*0)+S0) ~~~~ спецификация (0 для b)

(4) Aa:Ab:(a+Sb)=S(a+b) ~~ аксиома 3

(5) Ab:((S0*0)+Sb)=S((S0*0)+b спецификация ((S0*0) для а)

(6) ((S0*0)+S0)=S((S0*0)+0) ~~ спецификация (0 для b)

(7) Aa:(a+0)=a ~~~~~~~ аксиома 2

(8) ((S0*0)+0)=(S0*0) ~~~~~ спецификация ((S0*0) для а)

(9) Aa:(a*0)=0 ~~~~~~~ аксиома 4

(10) (S0*0)=0 ~~~~~~~ спецификация (S0 для а)

(11) ((S0*0)+0)=0~~~~~~~~~~~~~~~~~~~транзитивность (строчки 8,10)

(12)~ S((S0*0)+0)=S0 ~~~~ добавление S

(13)~ ((S0*0)+S0)=S0 ~~~~ транзитивность (строчки 6,12)

(14)~ (S0*S0)=S0~~~~~~~~~~~~~~~~~~~~~ транзитивность (строчки 3,13)

Нелегальные упрощения

Возникает интересный вопрос: «Каким образом мы можем вывести строчку 0=0?» Кажется, что очевидным способом было бы сначала вывести строчку Aa:a=a и затем использовать спецификацию. Как вы думаете, где ошибка в нижеследующем «выводе» Aa:a=a... Можете ли вы ее исправить?

(1) Aa:(a+0)=a ~~аксиома 2

(2) Aa:a=(a+0)~~~симметрия

(3) Aa:a=a транзитивность (строчки 2,1)

Я привел это маленькое упражнение, чтобы указать на следующий простой факт: при манипуляции хорошо знакомыми символами, такими, как «=», мы не должны торопиться. Мы должны следовать правилам, а не нашему знанию пассивных значений символов. (Тем не менее, это знание весьма ценно, чтобы помочь нам направить вывод по верному пути.)

Почему спецификация и общность ограничены

Давайте выясним, почему и спецификация, и общность нуждаются в ограничениях Взгляните на следующие две деривации; в каждой из них одно из ограничений нарушено. Обратите внимание, к каким печальным последствиям это привело.

(1)~ {[}~~~~~~~~~~~~~~~~~~~ проталкивание

(2)~~~~ a=0~~~~~~~~~~~~ посылка

(3)~~~~ Aa:a=0~~~~~~~ обобщение (ложно!)

(4)~~~~ Sa=0~~~~~~~~~~ спецификация

(5)~ {]}~~~~~~~~~~~~~~~~~~ выталкивание

(6)~ \textless a=0эSa=0\textgreater~~~~~ правило фантазии

(7)~ Aa:\textless a=0эSa=0 обобщение

(8)~ \textless0=0эS0=0\textgreater~~~~~ спецификация

(9)~ 0=0~~~~~~~~~~~~~~ предыдущая теорема

(10) S0=0~~~~~~~~~~~~ отделение (строчки 9,8)

Это первое из печальных последствий. Другое получается из неверной спецификации.

(1) Aa:a=a предыдущая теорема

(2) Sa=Sa спецификация

(3) Eb:b=Sa существование

(4) Aa:Eb:b=Sa обобщение

(5) Eb:b=Sb спецификация (ложно!)

Теперь вы убедились, почему необходимы ограничения. Вот простая задачка: переведите (если вы этого уже не сделали раньше) четвертый постулат Пеано в нотацию ТТЧ, и затем выведите эту строчку как теорему.

Чего-то не хватает

Если вы поэкспериментируете с теми правилами и аксиомами ТТЧ, которые я привел до сих пор, вы обнаружите, что возможно вывести следующую пирамидальную семью теорем (множество строчек, отлитых из одной формы и отличающихся только тем, что символы чисел 0, S0, SS0, и так далее, идут по нарастающей):

(0+0)=0

(0+S0)=S0

(0+SS0)=SS0

(0+SSS0)=SSS0

(0+SSSS0)=SSSS0

и так далее.

Каждая из теорем этой семьи может быть выведена из предыдущей теоремы с помощью коротенькой, всего лишь в пару строчек, деривации. Результатом является нечто вроде каскада теорем, каждая из которых вызывает к жизни следующую. (Эти теоремы напоминают теоремы \textbf{pr} , где средняя и правая группы тире возрастали одновременно.)

Существует одна строчка, которую легко записать и которая суммирует пассивное значение всех этих строчек, вместе взятых. Вот эта универсально квантифицированная суммирующая строчка:

Aa:(0+a)=a

Однако при помощи правил, данных до сих пор, эту строчку вывести нельзя. Попробуйте сами, и вы в этом убедитесь!

Вы можете подумать, что ситуацию легко исправить, используя следующее:

(ПРЕДЛАГАЕМОЕ) ВСЕОБЩЕЕ ПРАВИЛО: Если все строчки в пирамидальной семье --- теоремы, то универсально квалифицированная строчка, их суммирующая, также является теоремой.

Недостаток этого правила в том, что оно не может быть использовано при работе по способу \textbf{M} . Только люди, думающие о системе, могут знать, что каждая из бесконечного множества строчек --- теорема. Следовательно, это правило не может являться частью формальной системы.

\&\#969;-неполные системы и неразрешимые строчки

Мы очутились в странной ситуации, в которой возможно типографским путем производить теоремы о сложении любых \emph{конкретных} чисел, но даже такая простая строчка, как приведенная выше, выражающая свойство сложения в \emph{общем} , не является теоремой. Вы, возможно, найдете это не таким уж странным, поскольку мы уже были в похожей ситуации с системой \textbf{pr} . Однако система~\textbf{pr} не имела претензий по поводу своих возможностей; на самом деле, там было невозможно даже \emph{выразить} общие свойства, а тем более, доказать их. В той системе просто не было соответствующего «оборудования» --- при этом нам и в голову не приходило, что система была дефектна. Однако у ТТЧ возможностей гораздо больше; соответственно, мы ожидаем от нее большего, чем от системы \textbf{pr} . Если приведенная выше строчка --- не теорема, то у нас есть основания подозревать, что в ТТЧ есть какой-то дефект. На самом деле, существует даже название для систем с подобным дефектом --- они называются \&\#969;-\emph{неполными} . (Символ~\&\#969; --- «омега» --- выбран потому, что иногда все множество натуральных чисел обозначается этой буквой.) Далее следует точное определение:

Система является \&\#969;-неполной, если все строчки в пирамидальной семье --- теоремы, но универсально квантифицированная строчка, их суммирующая, --- не теорема.

Кстати, отрицание приведенной суммирующей строчки ---

\textasciitilde Aa:(0+a)=a

--- тоже не-теорема ТТЧ. Это означает, что первоначальная строчка \emph{неразрешима внутри системы} . Если бы та или другая были теоремами, мы сказали бы, что они разрешимы. Хотя это и звучит как мистический термин, в неразрешимости внутри данной системы нет ничего таинственного. Это означает, что система может быть дополнена. Например, внутри абсолютной геометрии пятый постулат Эвклида неразрешим. Чтобы получить геометрию Эвклида, его необходимо добавить; а отрицание пятого постулата, наоборот, производит не-эвклидову геометрию. Поскольку мы обратились к геометрии, давайте вспомним, почему это происходит. Дело в том, что четыре постулата не определяют термины «точка» и «линия» с достаточной точностью, так что остается возможность для различных интерпретаций этих понятий. Точки и линии Эвклидовой геометрии представляют собой лишь одну из возможных интерпретаций понятий «точка» и «линия» --- ТОЧКИ и ЛИНИИ неэвклидовой геометрии --- другая интерпретация. Однако то, что люди в течение тысячелетий пользовались такими хорошо известными словами как «точка» и «линия», заставило их думать, что эти слова могут иметь лишь одно-единственное значение.

Неэвклидова ТТЧ

С подобной же ситуацией мы сталкиваемся в ТТЧ Мы приняли нотацию, которая способствует созданию у нас некоторых предрассудков Например, использование символа «+» создает у нас впечатление, что любая теорема, использующая этот знак, сообщает нам что-то значимое о хорошо нам знакомой операции, под названием «сложение» Поэтому нам кажется, что предложить «шестую аксиому»

\textasciitilde Ea:(0+a)=a

было бы неверным. Она не совпадает с нашими знаниями о сложении Однако это одна из возможностей расширить ту ТТЧ, что мы сформулировали до сих пор Система, использующая данную строчку в качестве шестой аксиомы, последовательна в том смысле, что в ней нет двух теорем типа~\emph{x} и~\textasciitilde{}\emph{x.} Однако если вы наложите эту «шестую аксиому» на пирамидальную семью теорем, вы, возможно, будете поражены кажущимся несоответствием теорем этой семьи с новой аксиомой Этот тип непоследовательности, однако, не так вреден, как другой (где и~\emph{x} и~\textasciitilde{}\emph{x} --- теоремы). На самом деле это даже нельзя назвать непоследовательностью, так как существует такая интерпретация символов ТТЧ, в которой все получается хорошо.

\&\#969;-противоречивость не то же самое, что просто противоречивость

Этот тип противоречивости, созданный наложением (1) пирамидальной семьи теорем, которые, вместе взятые, утверждают, что все натуральные числа имеют определенное свойство, и (2) одной теоремы, утверждающей, что не все числа обладают этим свойством, называется \&\#969;-\emph{противоречивостью} . \&\#969;-\emph{противоречивая} система похожа на сначала-раздражающую-но-в-конце-концов-приемлемую неэвклидову геометрию. Чтобы построить мысленную модель того, что происходит, вообразите себе, что имеются некоторые дополнительные числа --- давайте будем называть их не натуральными, а \emph{экстранатуральными} --- у которых нет численных символов Поэтому их свойства не могут быть представлены в пирамидальной семье (Это немного напоминает представление Ахилла о БОГе --- что-то вроде «супергения», существа, находящегося выше всех гениев. Хотя это представление и было опровергнуто Гением, тем не менее это хороший образ, и может помочь вам вообразить экстранатуральные числа ).

Все это говорит нам о том, что аксиомы и правила ТТЧ, как мы до сих пор ее представляли, не описывают с достаточной полнотой интерпретации символов этой системы В нашей воображаемой модели понятий, которые эти символы представляют, еще остается место для вариантов Каждый из возможных вариантов системы опишет эти понятия немного полнее, но сделает это по-своему. Какие из символов приобретут «раздражающие» пассивные значения, если мы добавим приведенную выше «шестую аксиому»? Все ли символы окажутся «испорченными», или некоторые из них сохранят то значение, которые мы имели в виду? Предлагаю вам над этим поразмыслить. В главе XIV мы снова встретимся с подобным вопросом; там мы обсудим его подробнее. В любом случае, не будем здесь останавливаться на этом дополнении системы; вместо этого мы попытаемся исправить \&\#969;-неполноту ТТЧ.

Последнее правило

Недостаток обобщающего правила был в том, что оно требовало знания того факта, что все строчки бесконечной пирамидальной семьи --- теоремы; это слишком много для конечного существа. Однако представьте себе, что каждая строчка пирамиды может быть выведена из своей предшественницы регулярным путем. Тогда у нас оказалась бы конечное основание на то, чтобы считать все строчки пирамиды теоремами. Таким образом, трюк состоит в том, чтобы найти ту схему, которая порождает пирамиду, и показать, что сама эта схема является теоремой. Это подобно доказательству того, что каждый гений передает послание своему Мета-гению, как в детской игре в телефончик. Остается только доказать, что эта цепочка посланий начинается с гения --- то есть установить, что первая строчка пирамиды --- теорема. Тогда мы можем быть уверены, что послание дойдет до БОГа!

В конкретной пирамиде, которую мы рассматривали, такая схема существует; она представлена строчками 4-9 данной ниже деривации.

(1) Aa:Ab:(a+Sb)=S(a+b)~~аксиома 3

(2) Ab:(0+Sb)=S(0+b)~~~~~~~спецификация

(3) (0+Sb)=S(0+b) спецификация

(4) {[} ~~~~~~ проталкивание

(5) (0+b)=b ~~ посылка

(6) S(0+b)=Sb ~ добавление S

(7) (0+Sb)=S(0+b) перенос строки 3

(8) (0+Sb)=Sb ~ транзитивность

(9) {]} ~~~~~~ выталкивание

Посылка здесь --- (0+b)=b; результат --- (0+Sb)=Sb.

Первая строка пирамиды --- также теорема; это прямо следует из аксиомы 2. Все, что теперь требуется, это правило, позволяющее нам заключить, что строчка, суммирующая всю пирамиду в целом, тоже является теоремой. Такое правило будет формализованным пятым постулатом Пеано.

Чтобы выразить это правило, нам необходимо ввести кое-какую нотацию.

Давайте запишем правильно сформированную формулу, в которой переменная а свободна:

X\{a\}

(Там могут встречаться и другие свободные переменные, но нам это неважно.) Тогда запись X\{Sa/a\} будет обозначать то же самое, с той разницей, что все а будут заменены на Sa. Таким же образом, X\{0/а\} будет обозначать ту же строку, в которой все а заменены на 0.

Приведем конкретный пример. Пусть X\{a\} обозначает строчку (0+а)=а. Тогда X\{Sa/a\} представляет строчку (0+Sa)=Sa, a X\{0/a\} --- строчку (0+0)=0.

(Внимание: эта нотация не является частью ТТЧ; она служит нам лишь для того, чтобы говорить о ТТЧ.)

С помощью этой новой нотации мы можем выразить последнее правило ТТЧ весьма точно:

ПРАВИЛО ИНДУКЦИИ. Предположим, что~\emph{u} --- переменная и X\{\emph{u} \} --- правильно сформированная формула, в которой и свободно. Если и Au\textless X\{u\}эX\{Su/u\} и X\{0/u\} --- теоремы, то Au:X\{u\}~также является теоремой.

Мы подошли так близко, как возможно, к введению пятого постулата Пеано в ТТЧ. Давайте теперь используем его, чтобы показать, что Aa:(0+a)=a действительно является теоремой ТТЧ Выходя из области фантазии в приведенной выше деривации, мы можем использовать правило фантазии, чтобы получить

(10) \textless(0+b)=bэ(0+Sb)=Sb\textgreater{} правило фантазии

(11) Ab:\textless(0+b)=bэ(0+Sb)=Sb\textgreater~~ обобщение

Это --- первая из двух вводных теорем, требующихся для правила индукции другая --- первая строка пирамиды --- у нас уже имелась Следовательно мы можем применить правило индукции и получить то, что нам требуется

Ab:(0+b)=b

Спецификация и обобщение позволят нам изменить переменную с~\emph{b} на \emph{a} , таким образом,~Aa:(0+a)=a перестает быть неразрешимой строчкой ТТЧ.

Длинная деривация

Я хочу представить здесь более длинную деривацию ТТЧ с тем, чтобы читатель посмотрел, как она выглядит; эта деривация доказывает простой, но важный факт теории чисел.

(1) Aa:Ab:(a+Sb)=S(a+b) аксиома 3

(2) Ab:(d+Sb)=S(d+b) спецификация

(3) (d+SSc)=S(d+Sc) спецификация

(4) Ab:(Sd+Sb)=S(Sd+b) спецификация (строка 4)

(5) (Sd+Sc)=S(Sd+c) спецификация

(6) S(Sd+c)=(Sd+Sc) симметрия

(7) {[} проталкивание

(8) Ad:(d+Sc)=(Sd+c) посылка

(9) (d+Sc)=(Sd+c) спецификация

(10) S(d+Sc)=S(Sd+c) добавление S

(11) (d+SSc)=S(d+Sc) перенос 3

(12) (d+SSc)=S(Sd+c) транзитивность

(13) S(Sd+c)=(Sd+Sc) перенос 6

(14) (d+SSc)=(Sd+Sc) транзитивность

(15) Ad:(d+SSc)=(Sd+Sc) обобщение

(16) {]} выталкивание

(17) \textless Ad:(d+Sc)=(Sd+c)эAd:(d+SSc)=(Sd+Sc)\textgreater{} правило фантазии

(18) Ac:\textless Ad:(d+Sc)=(Sd+c)эAd:(d+SSc)=(Sd+Sc)\textgreater{} обобщение

*****

(19) (d+S0)=S(d+0) спецификация (строчка 2)

(20) Aa:(a+0)=a аксиома 1

(21) (d+0)=d спецификация

(22) S(d+0)=Sd добавление S

(23) (d+S0)=Sd транзитивность (строки 19,22)

(24) (Sd+0)=Sd спецификация (строка 20)

(25) Sd=(Sd+0) симметрия

(26) (d+S0)=(Sd+0) транзитивность (строчки 23, 25)

(27) Ad:(d+S0)=(Sd+0) обобщение

*****

(28) Ac:Ad:(d+Sc)=(Sd+c) индукция (строчки 18,27)

{[}В сложении S может быть передвинуто вперед или назад.{]}

(29) Ab:(c+Sb)=S(c+b) спецификация (строчка 1)

(30) (c+Sd)=S(c+d) спецификация

(31) Ab:(d+Sb)=S(d+b) спецификация (строчка 1)

(32) (d+Sc)=S(d+c) спецификация

(33) S(d+c)=(d+Sc) симметрия

(34) Ad:(d+Sc)=(Sd+c) спецификация (строчка 28)

(35) (d+Sc)=(Sd+c) спецификация

(36) {[} проталкивание

(37) Ac:(c+d)=(d+c) посылка

(38) (c+d)=(d+c) спецификация

(39) S(c+d)=S(d+c) добавление S

(40) (c+Sd)=S(c+d) перенос 30

(41) (c+Sd)=S(d+c) транзитивность

(42) S(d+c)=(d+Sc) перенос

(43) (c+Sd)=(d+Sc) транзитивность

(44) (d+Sc)=(Sd+c) перенос 35

(45) (c+Sd)=(Sd+c) транзитивность

(46) Ac:(c+Sd)=(Sd+c) обобщение

(47) {]} выталкивание

(48) \textless Ac:(c+d)=(d+c)эAc:(c+Sd)=(Sd+c)\textgreater{} правило фантазии

(49) Ad:\textless Ac:(c+d)=(d+c)эAc:(c+Sd)=(Sd+c)\textgreater{} обобщение

{[}Если d коммутирует с любым с,~то Sd обладает таким же свойством.{]}

*****

(50) (с+0)=с спецификация (строка 20)

(51) Aa:(0+a)=a предыдущая теорема

(52) (0+с)=с спецификация

(53) с=(0+с) симметрия

(54) (с+0)=(0+с) транзитивность (строчки 50, 53)

(55) Ac:(c+0)=(0+c) обобщение

{[}О коммутирует с любым с{]}

*****

(56) Ad:Ac:(c+d)=(d+c) индукция (строчки 49,55)

{[}Таким образом, любое d коммутирует с любым с{]}

Напряжение и разрешение в ТТЧ

ТТЧ доказала коммутативность сложения. Даже если вы не следили за всеми деталями этой деривации, важно понять, что, так же как и музыкальная пьеса, она имеет свой собственный естественный «ритм». Это вовсе не случайная про гулка, во время которой мы вдруг наткнулись на нужную строчку. Я ввел «паузы для дыхания», чтобы показать «артикуляцию» этой деривации. В частности, строчка 28 является переломным моментом в деривации~--- что-то вроде середины в пьесе типа~\emph{ААББ} , где происходит временное разрешение, хотя и не в ключевую тональность. Подобные важные промежуточные моменты часто называют «леммами».

Легко вообразить себе читателя, который начинает со строки 1 этой деривации, не зная, где он закончит, и постепенно, с каждой новой строкой, понимающего, куда он направляется. Это порождает внутреннее напряжение, очень похожее на то, которое порождает в музыке прогрессия аккордов, указывающая на тонику, но не разрешающаяся в нее. Прибытие к строке 28 подтверждает интуицию читателя и дает ему некое чувство удовлетворения; в то же время, это усиливает его желание дойти до предполагаемой конечной цели.

Строчка 49~--- критически важный увеличитель напряжения, поскольку она вызывает ощущение «почти у цели». Прервать деривацию в этот момент было бы очень неприятно. С этого момента мы уже почти можем предсказать развитие событий. Однако вам не хотелось бы прервать музыкальную пьесу в том момент, когда вам уже очевидно, как она разрешится. Вам не хотелось бы~\emph{воображать~} финал~--- вам хотелось бы его услышать. Так же и здесь, мы должны закончить деривацию. Строка 55 Неизбежна и производит максимальное финальное напряжение, которое затем разрешается в строке 56.

Это типично не только для структуры формальных дериваций, но и для неформальных доказательств. Чувство напряжения, возникающее у математиков, тесно связано с восприятием красоты; это делает математику интересным и стоящим занятием. Обратите внимание, однако, что в самой ТТЧ это напряжение, по-видимому, не отражается. Иными словами, понятия напряжения и раз решения, цели и временной цели, «естественности» и «неизбежности» не формализованы в ТТЧ подобно тому, как музыкальная пьеса не является книгой о гармонии и ритме. Возможно ли создать более сложную формальную систему, которая осознавала бы напряжение и цель внутри дериваций?

Формальные и неформальные рассуждения

Я предпочел бы показать вам, как выводится теорема Эвклида (бесконечность простых чисел), но это, возможно, сделало бы книгу вдвое длиннее. Теперь, после доказательства теоремы, естественным продолжением было бы доказать ассоциативность сложения, коммутативность и ассоциативность умножения и дистрибутивность умножения со сложением. Это создало бы прочную базу для дальнейшей работы.

В нашей теперешней формулировке ТТЧ достигла «критической массы». Ее мощь сравнялась с мощью «Principia mathematica»~--- в ней стало возможным доказать любую теорему, которую можно найти в стандартном труде по теории чисел. Конечно, никто не стал бы утверждать, что вывод теорем в ТТЧ - это наилучший способ заниматься теорией чисел. Человек, так считающий, принадлежал бы к классу людей, которые думают, что лучший способ узнать, сколько будет 1000\&\#215;1000~--- это нарисовать решетку размером 1000x1000 и подсчитать в ней клеточки. На самом деле, после полной формализации остается единственный путь~--- дать формальной системе послабление. Иначе она становится такой громоздкой, что теряет всякую практическую пользу. Таким образом, необходимо внести ТТЧ в более широкий контекст, такой, который позволит нам получить правила вывода, ускоряющие деривацию. Для этого нам понадобится формализовать язык, на котором выражены эти правила вывода~--- то есть метаязык. Можно пойти еще намного дальше; однако никакие из этих трюков не сделают ТТЧ более~\emph{мощной~---} они лишь сделают ее более~\emph{удобной для пользования} . Дело в том, что мы выразили в ТТЧ все типы рассуждений, на которые опираются математики, занимающиеся теорией чисел. Введение ее в более широкий контекст не увеличит количества теорем; оно лишь сделает работу в ТТЧ~--- или в любой «улучшенной» ее версии~--- более похожей на работу в традиционной теории чисел.

Специалисты по теории чисел закрывают лавочки

Представьте себе, что вы не знали заранее, что ТТЧ окажется неполной~--- напротив, вы ожидали, что она полна, то есть, что любое истинное высказывание, которое можно выразить в нотации ТТЧ, является теоремой. В таком случае вы могли бы иметь разрешающую процедуру для всей теории чисел. Ваш метод был бы прост; если вы хотите знать, истинно ли высказывание~X~теории чисел, вы должны закодировать его в строчку~\emph{x} ~ТТЧ. Теперь, если~X~--- истинно, то полнота говорит нам, что~\emph{x} ~--- теорема. С другой стороны, если~\textbf{не-X} ~--- истинно, тогда~\emph{\textasciitilde x} ~--- теорема. Таким образом, либо~\emph{x} , либо~\emph{\textasciitilde x} ~должны оказаться теоремами, поскольку либо~\textbf{X} , либо~\textbf{не-X} ~истинны. Теперь вы должны систематически пронумеровать все теоремы ТТЧ, так же как мы сделали это для систем~\textbf{MIU} ~и~\textbf{pr} . Какое-то время спустя, вы наткнетесь либо на~\emph{x} , либо на~\emph{\textasciitilde x} , и, таким образом, узнаете, какое из двух высказываний~---~X~или~\textbf{не-X} ~--- истинно. (Следите ли вы за ходом рассуждения? Очень важно держать в голове разницу между формальной системой ТТЧ и ее неформальным соответствием~--- теорией чисел; читатель должен постараться хорошо понять эту разницу.) Так что в принципе, если бы ТТЧ была полной, специалисты по теории чисел остались бы без работы: со временем любую проблему можно было бы решить чисто механическим путем. Оказывается, однако, что это невозможно; по этому поводу можно радоваться или огорчаться, в зависимости от вашей точки зрения.

Программа Гильберта

Последний вопрос, который мы рассмотрим в этой главе, таков: должны ли мы так же верить в непротиворечивость ТТЧ, как мы верили в непротиворечивость исчисления высказываний? И если нет, то возможно ли укрепить нашу веру в ТТЧ,~\emph{доказав} , что она непротиворечива? Для начала можно утверждать, подобно тому, как Неосторожность утверждала об исчислении высказываний, что непротиворечивость ТТЧ «очевидна»~--- а именно, что каждое правило воплощает принцип логических рассуждений, в который мы верим безоговорочно; следовательно, ставить под вопрос непротиворечивость ТТЧ, это все равно, что сомневаться в собственном здравом уме. Этот аргумент все еще имеет некоторый вес, но уже не такой, как раньше. Дело в том, что теперь у нас слишком много правил вывода, и в какие-то из них могла вкрасться ошибка. Более того, откуда мы знаем, что наша мысленная модель неких абстрактных единиц под названием «натуральные числа» последовательна? Может быть, наши собственные мыслительные процессы, те неформальные процессы, которые мы пытались выразить в правилах формальной системы, сами по себе непоследовательны! Конечно, мы не ожидаем подобного подвоха. Тем не менее, можно представить, что чем сложнее объект нашей мысли, тем легче в нем запутаться; а натуральные числа~--- объект совсем не тривиальный. Так что в этом случае мы должны серьезнее воспринимать аргументы Осторожности, когда она требует~\emph{доказательства~} непротиворечивости. Не то, чтобы мы действительно сомневались в непротиворечивости ТТЧ~--- но у нас все же есть~\emph{малюсенькое~} сомнение, тень сомнения, и доказательство помогло бы эту тень рассеять.

Какой же метод доказательства нам бы хотелось использовать? Здесь мы снова сталкиваемся с проблемой порочного круга. Если мы будем использовать в доказательстве факта о системе те же инструменты, какие используются~\emph{внутри~} самой системы, то чего мы таким образом добьемся? Если бы нам удалось убедиться в непротиворечивости ТТЧ, используя более слабую систему рассуждений, чем сама ТТЧ, мы избежали бы этого порочного круга! Подумайте о том, как протягивают тяжелый канат между двумя кораблями (по крайней мере, я читал об этом, когда был мальчишкой): сначала с одного из кораблей пускается стрела, которая перетаскивает через промежуток между кораблями веревку, затем при помощи этой веревки перетягивается канат. Если бы нам удалось использовать «легкую» систему, Чтобы показать непротиворечивость «тяжелой» системы, тогда мы могли бы считать, что действительно чего-то добились.

С первого взгляда может показаться, что у нас есть такая веревка. Наша цель~--- доказать, что в ТТЧ есть некоторое типографское свойство (непротиворечивость): в ней не встречаются одновременно теоремы формы~\emph{x} ~и~\emph{\textasciitilde x} . Это похоже на доказательство того, что~\textbf{MU} ~не является теоремой системы~\textbf{MIU} . В обоих случаях мы имеем дело с утверждениями о типографских свойствах си стем, манипулирующих символами. Наше сравнение с веревкой основано на предположении о том, что~\emph{факты теории чисел не нужны} ~для доказательства некоего типографского свойства. Иными словами, если не использовать свойства целых чисел вообще~--- или использовать только несколько простейших свойств~--- мы можем доказать непротиворечивость ТТЧ, используя способы, более простые, чем наша внутренняя система рассуждений.

Именно на это надеялась школа математиков и логиков начала века; главой этой влиятельной школы был Давид Гильберт. Их целью было доказать непротиворечивость формализации теории чисел, подобных ТТЧ, используя весьма ограниченный набор логических принципов рассуждения, называемых~\emph{финитными} . Эти принципы были бы их «веревкой». Среди финитных методов~--- все методы исчисления высказываний, и некоторые методы численных рассуждений. Однако труды Гёделя показали, что любые усилия протащить через про пасть канат непротиворечивости ТТЧ, пользуясь веревкой финитных методов, обречены на провал. Гёдель показал, что для того, чтобы перетащить этот канат, невозможно пользоваться более легкой веревкой~--- просто нет настолько крепкой веревки, чтобы она выдержала такую нагрузку. Выражаясь менее метафорично, можно сказать:~\emph{любая система, достаточно мощная, чтобы доказать непротиворечивость ТТЧ, по крайней мере так же мощна, как сама ТТЧ} . Поэтому порочного круга здесь не избежать.


% % \subsubsection[Приношение «МУ»]{\texorpdfstring{Приношение «МУ»\footnote{Все коаны в этом Диалоге подлинны, они взят из следующих двух книг Paul Reps «Zen Flesh Zen Bones» и Gyomay M. Kubose «Zen Koans»}}{Приношение «МУ»}}
% \subsubsection[Приношение «МУ»]{\texorpdfstring{Приношение «МУ»\footnote{Все коаны в этом Диалоге подлинны, они взят из следующих двух книг Paul Reps «Zen Flesh Zen Bones» и Gyomay M. Kubose «Zen Koans»}}{Приношение «МУ»}}

\emph{Черепаха и Ахилл только что вернулись с лекции о происхождении Генетического Кода; они сидят у Ахилла и пьют чай.}

\emph{Ахилл} : Я должен кое в чем признаться, г-жа Ч.

\emph{Черепаха} : Что такое?

\emph{Ахилл} : Несмотря на интереснейшую тему, я пару раз задремал\ldots{} Но даже во сне я кое-что слышал. Вот какая странная мысль всплыла из глубины моего сознания: «А» и «Т» могут обозначать не «аденин» и «тимин», а мое и ваше имена! Ведь вас зовут Тортилла! Кроме того, в моем полусне вдоль остова двойной спирали ДНК были подвешены крохотные Ахиллы и Тортиллы, всегда в парах, как аденин и тимин. Правда, странный образ?

\emph{Черепаха} : Фу! Кто верит в подобные глупости? К тому же, что вы скажете о «С» и «G»?

\emph{Ахилл} : Что ж, цитозин мог бы обозначать г-на Краба --- ведь его имя пишется «Crab». Насчет «G» я не знаю, но уверен, что можно было бы что-нибудь придумать. Так или иначе, было забавно вообразить мою ДНК, полную ваших малюсеньких копий --- и моих, конечно. Только подумайте, к какой бесконечной регрессии это бы привело!

\emph{Черепаха} : Вижу, что вы не очень-то внимательно слушали лекцию.

\emph{Ахилл} : Неправда --- я старался изо всех сил. Просто было очень трудно отделить мои фантазии от фактов. В конце концов, молекулярные биологи изучают такой необыкновенный нижний мир\ldots{}

\emph{Черепаха} : Что вы имеете в виду?

\emph{Ахилл} : Молекулярная биология полна странных спиральных петель, которые я как следует не понимаю. Например, белки, закодированные в ДНК, могут «провернуться назад» и повлиять на саму ДНК --- даже разрушить ее. Подобные странные петли меня всегда запутывают. В них есть что-то пугающее.

\emph{Черепаха} : Я нахожу их весьма привлекательными.

\emph{Ахилл} : Разумеется --- они вполне в вашем вкусе. Но мне иногда хочется прекратить весь этот анализ и просто помедитировать немного, в качестве противоядия. Это очищает голову от путаницы странных петель и всех этих невероятных сложностей, о которых мы сегодня услышали.

\emph{Черепаха} : Удивительно! Никогда бы не подумала, что вы медитируете.

\emph{Ахилл} : Разве я никогда не говорил вам, что изучаю дзен-буддизм?

\emph{Черепаха} : Боже мой, как вы до этого додумались?

\emph{Ахилл} : Мне всегда казалось, что без инь и янь мое дело --- дрянь; знаете, все эти путешествия в восточный мистицизм, И-Чинг, гуру, и тому подобное. В одни прекрасный день я подумал: «Почему бы мне не заняться и дзеном?» Так это все и началось.

\emph{Черепаха} : Превосходно! Может быть и я, наконец, сподоблюсь просветиться.

\emph{Ахилл} : Ну-ну, не так быстро. Просветление --- совсем не первый шаг на пути к буддизму; скорее, это последний шаг. Просветление не для таких новичков, как вы, г-жа Ч!

\emph{Черепаха} : Вы меня не поняли. Я не имела в виду буддистское просветление --- мне просто хотелось узнать, что такое дзен-буддизм.

\emph{Ахилл} : Бог ты мой, что же вы сразу не сказали? Я буду очень рад рассказать вам все, что знаю о дзене. Может быть, вам даже захочется стать учеником буддизма, таким же, как и я.

\emph{Черепаха} : Что ж, нет ничего невозможного.

\emph{Ахилл} : Вы можете изучать буддизм вместе со мной у моего Мастера Оканисамы --- седьмого патриарха.

\emph{Черепаха} : Черт меня побери, если я что-нибудь понимаю!

\emph{Ахилл} : Чтобы это понять, необходимо знать историю дзен-буддизма.

\emph{Черепаха} : В таком случае, не расскажете ли вы мне немного об истории дзена?

\emph{Ахилл} : Отличная мысль. Дзен --- это тип буддизма; он был основан монахом по имени Бодхидхарма, который оставил Индию и поселился в Китае. Это было в шестом веке. Бодхидхарма был первым патриархом. Шестым патриархом был\ldots{} э-э-э\ldots{} проклятый склероз\ldots{} Энон! (Наконец-то вспомнил!)

\emph{Черепаха} : Неужели Зенон? Как странно, что именно он оказался замешанным в таком деле.

\emph{Ахилл} : Осмелюсь заметить, что вы недооцениваете значимость дзена. Послушайте еще немного и, может быть, вы будете относиться к нему с большим уважением. Так вот, как я говорил, примерно пятьсот лет спустя дзен пришел в Японию, где он прекрасно прижился. С того времени он стал одной из основных религий Японии.

\emph{Черепаха} : Кто такой этот Оканисама, «седьмой патриарх»?

\emph{Ахилл} : Он мой Мастер, и его учение прямо следует из учения шестого патриарха. Он научил меня тому, что действительность --- едина и неизменна; вся множественность, изменения и движение --- не более, чем иллюзии наших чувств.

\emph{Черепаха} : Точно --- это за километр пахнет дзеном. Но как же он впутался в дзен, бедняга?

\emph{Ахилл} : Что-о? Если КТО-ТО и запутался, то это\ldots{} Ну ладно, это уже другой разговор. Так или иначе, я не знаю ответа на ваш вопрос. Вместо этого я вам лучше расскажу еще что-нибудь из поучений моего Мастера. Я узнал, что в дзене человек ищет Просветления, или САТОРИ --- состояния «He-разума». В этом состоянии человек не думает о мире --- он просто СУЩЕСТВУЕТ. Я также узнал, что изучающий дзен не должен «привязывать» себя ни к какому объекту, или мысли, или человеку --- то есть, он не должен верить ни в какой абсолют и не должен зависеть от чего-либо, включая и саму эту философию не-привязанности.

\emph{Черепаха} : Г-мм\ldots{} Это уже КОЕ-ЧТО; дзен начинает мне нравиться.

\emph{Ахилл} : У меня было предчувствие, что вы сразу к нему привяжетесь.

\emph{Черепаха} : Но скажите мне: если дзен отрицает интеллектуальную деятельность вообще, то какой смысл размышлять о нем и усердно его изучать?

\emph{Ахилл} : Мне тоже не давала покоя эта мысль. Но думаю, что я, наконец, нашел ответ: к дзену можно подходить по любой дороге, даже если эта дорога кажется ведущей совершенно в другую сторону. По мере того, как вы к нему приближаетесь, вы учитесь отходить от дороги в сторону; и чем больше вы отходите в сторону, тем ближе вы подходите к дзену.

\emph{Черепаха} : Теперь все кажется совсем простым.

\emph{Ахилл} : Моя любимая дорога к дзену проходит через его короткие, интересные и странные притчи, под названием «коаны».

\emph{Черепаха} : Что это такое --- коан?

\emph{Ахилл} : Коан --- это история о Мастерах дзена и их учениках. Иногда он в форме загадки, иногда --- басни, а иногда коан совершенно не похож ни на что, слышанное вами раньше.

\emph{Черепаха} : Звучит интригующе. Вы думаете, что читать коаны и наслаждаться ими значит заниматься дзен-буддизмом?

\emph{Ахилл} : Сомневаюсь. Однако мне кажется, что получать удовольствие от коанов в миллион раз ближе к настоящему дзену, чем читать об этой религии том за томом, написанные на тяжелом философском жаргоне.

\emph{Черепаха} : Хотелось бы услышать какой-нибудь коан.

\emph{Ахилл} : С удовольствием расскажу вам парочку. Я должен, пожалуй, начать с самого знаменитого. Итак, много столетий тому назад жил Мастер дзен-буддизма по имени Джошу, который дожил до 119 лет.

\emph{Черепаха} : Просто юнец!

\emph{Ахилл} : С вашей точки зрения, конечно. Так вот, однажды, когда Джошу и другой монах стояли вместе в монастыре, мимо пробежала собака. Монах спросил Джошу: «У этого дога --- природа Будды?»

\emph{Черепаха} : Непонятно. Так что же ответил монах?

\emph{Ахилл} : МУ.

\emph{Черепаха} : МУ? Что это за «МУ» такое? А как же насчет собаки? И природы Будды? Как же ответ?

\emph{Ахилл} : Но ведь «МУ» и есть ответ Джошу! Говоря «МУ», Джошу дал понять другому монаху, что только воздерживаясь от подобных вопросов, можно получить на них ответ.

\emph{Черепаха} : Джошу «развопросил» этот вопрос.

\emph{Ахилл} : Именно!

\emph{Черепаха} : Это «МУ» --- весьма полезная штучка. Иногда мне тоже хочется развопросить кое-какие вопросы. Кажется, я начинаю ухватывать суть дзена\ldots{} Вы знаете еще какие-нибудь коаны, Ахилл? Мне хотелось бы услышать еще несколько.

\emph{Ахилл} : Охотно. Я знаю парочку коанов, которые всегда рассказываются вместе. Только\ldots{}

\emph{Черепаха} : Что такое?

\emph{Ахилл} : Дело в том, что мой Мастер предупреждал меня, что только один из них настоящий. Хуже того, он не знает, какой из них подлинный, а какой --- фальшивка.

\emph{Черепаха} : С ума сойти! Расскажите-ка их мне, чтобы мы могли наугадываться всласть!

\emph{Ахилл} : Хорошо. Один из коанов таков:

\emph{Один монах спросил Басо: «Что такое Будда?»}

\emph{Басо ответил: «Этот разум --- Будда.»}

\emph{Черепаха} : Гмм\ldots{} «Этот разум --- Будда»? Иногда мне трудно понять, что хотят сказать эти дзен-буддисты.

\emph{Ахилл} : Тогда второй коан может понравиться вам больше.

\emph{Черепаха} : Что это за коан?

\emph{Ахилл} : Вот он:

\emph{Один монах спросил Басо: «Что такое Будда?»}

\emph{Басо ответил: «Этот разум --- не Будда.»}

\emph{Черепаха} : Ну и ну! Как если бы мой панцирь был зеленый и не зеленый! Это мне нравится!

\emph{Ахилл} : Однако, г-жа Т, коаны совсем не предназначены для того, чтобы просто «нравиться».

\emph{Рис. 45. М. К. Эшер «Мечеть» (черные и белые мелки, 1936)}

\emph{Черепаха} : Отлично, в таком случае это мне не нравится.

\emph{Ахилл} : Так-то лучше. Так вот, как я говорил, мой мастер считает, что только один из них --- настоящий.

\emph{Черепаха} : Не могу себе представить, что заставило его так решить. Все равно этот вопрос чисто академический, поскольку невозможно узнать, какой из двух коанов --- оригинал, а какой --- подделка.

\emph{Ахилл} : Вы ошибаетесь: мой Мастер научил нас, как это сделать.

\emph{Черепаха} : Неужели? Разрешающий алгоритм для установления подлинности коанов? Хотелось бы мне услышать об ЭТОМ.

\emph{Ахилл} : Это довольно сложный ритуал: в нем два этапа. На первом этапе вы должны ТРАНСЛИРОВАТЬ данный коан в цепочку, уложенную спиралью в трех измерениях.

\emph{Черепаха} : Забавная штучка. А как насчет второго этапа?

\emph{Ахилл} : Ну, это совсем просто: надо всего-навсего определить, имеет цепочка природу Будды или нет! Если у нее --- природа Будды, то коан --- подлинный, а если нет, то он --- фальшивка.

\emph{Черепаха} : Гмм\ldots{} Это звучит так, словно вы только перенесли нужду в разрешающей процедуре в другую область. ТЕПЕРЬ вам нужна разрешающая процедура для определения природы Будды. Что же дальше? В конце концов, если вы не можете сказать даже того, буддистская ли природа у СОБАКИ,~~как же вы собираетесь определить это для любого кусочка цепочки трехмерной укладки?

\emph{Ахилл} : Мой мастер объяснил мне, что переход из одной области в другую может помочь. Это похоже на перемену точки зрения. Некоторые вещи выглядят сложными под одним углом, но простыми под другим. Он привел в пример сад: глядя на него с одной стороны, вы не видите никакого порядка, только под некоторыми углами перед вами возникает прекрасная упорядоченность. Вы организовали информацию иначе, взглянув на вещи с иной точки зрения.

\emph{Черепаха} : Понятно. В таком случае, может оказаться, что подлинность коана спрятана в нем где-то глубоко, но когда вам удается перевести его в цепочку, она каким-то образом всплывает на поверхность?

\emph{Ахилл} : Именно это и открыл мой Мастер.

\emph{Черепаха} : В таком случае, мне бы хотелось узнать об этой технике побольше. Но сперва скажите мне, как вы можете превратить коан (последовательность слов) в уложенную в пространстве цепочку (трехмерный объект)? Ведь это довольно разные классы предметов.

\emph{Ахилл} : Это как раз одна из наиболее таинственных вещей, которые я узнал, изучая дзен. Есть два шага: «транскрипция» и «трансляция». Сделать транскрипцию коана --- значит записать его фонетическим алфавитом, который содержит только четыре геометрических символа. Эта фонетическая транскрипция коана называется ПОСРЕДНИКОМ.

\emph{Черепаха} : Как выглядят эти геометрические символы?

\emph{Ахилл} : Они состоят из гексагонов и пентагонов; вот так (берет лежащую рядом салфетку и набрасывает следующие четыре фигуры):

\emph{Черепаха} : Выглядит загадочно.

\emph{Ахилл} : Только для непосвященных. Теперь, когда посредник готов, вы натирайте руки рибосом, и\ldots{}

\emph{Черепаха} : Рибосом? Это что, ритуальная мазь?

\emph{Ахилл} : Не совсем. Это специальный клейкий состав, который помогает цепочке сохранять форму, когда она уложена.

\emph{Черепаха} : Из чего он сделан?

\emph{Ахилл} : Точно не знаю, но он клейкий на ощупь и прекрасно работает. Так или иначе, когда вы натерли руки рибосом, вы можете транслировать последовательность символов в посреднике в некий тип укладки цепочки. Как видите, все очень просто.

\emph{Черепаха} : Подождите! Не так быстро! Как вы это делаете?

\emph{Ахилл} : Вы берете прямую цепочку и начинаете укладывать ее с одного конца, в соответствии с геометрическими символами посредника.

\emph{Черепаха} : Значит, каждый из этих символов обозначает особый тип укладки?

\emph{Ахилл} : Сам по себе нет. Они всегда берутся группами по три. Вы начинаете с одного конца цепочки и с одного конца посредника. Первая тройка символов определяет, что делать с первым дюймом цепочки. Следующие три символа говорят вам, как укладывать второй дюйм. Таким образом, вы шаг за шагом продвигаетесь вдоль цепочки и вдоль посредника, укладывая~~каждый крохотный сегмент цепочки, пока посредник не кончится Если вы хорошенько смазали все рибосом, цепочка сохранит свою укладку и у вас получится трансляция коана в цепочку.

\emph{Черепаха} : Эта процедура не лишена элегантности. Наверное, у вас получаются чертовски интересные цепочки.

\emph{Ахилл} : Еще бы! Коаны подлиннее транслируются в весьма причудливые структуры.

\emph{Черепаха} : Могу себе представить. Но чтобы транслировать посредник в цепочку вы должны знать, какой укладке соответствует каждая тройка геометрических символов. Откуда вы это знаете? У вас что, есть словарь?

\emph{Ахилл} : Да --- это замечательная книга, в которой приведен весь Геометрический Код. Если у вас этой книги нет, то, разумеется, вы не можете транслировать коаны в цепочки.

\emph{Черепаха} : Разумеется нет. Каково происхождение Геометрического Кода?

\emph{Ахилл} : Его начало восходит к древнему Мастеру по имени Великий Ментор, мой Мастер говорит, что он единственный, кто когда-либо достиг Архи-просветления.

\emph{Черепаха} : Ах ты батюшки! Словно одного уровня мало Что ж, обжоры бывают всех сортов --- почему бы не обжираться и просветлением?

\emph{Ахилл} : А что, если в слове Архи-просветление закодировано мое имя? А-Х-И-Л.

\emph{Черепаха} : По моему мнению, это маловероятно. Скорее, там можно найти намек на имя скромной ЧерепАХИ.

\emph{Ахилл} : При чем здесь вы? Вы даже не достигли ПЕРВОГО состояния просветления, и уж тем более\ldots{}

\emph{Черепаха} : Почем знать, почем знать. Может быть те, кто изучил всю подноготную просветления возвращаются в первоначальное, допросветленное состояние Я всегда считала, что дважды просветленный --- это снова непросветленный. Но вернемся же к нашему Великому Ментору.

\emph{Ахилл} : О нем известно очень мало --- пожалуй, только то, что он изобрел Искусство Дзен-Цепочек.

\emph{Черепаха} : Что это такое?

\emph{Ахилл} : Это искусство на котором основана разрешающая процедура для определения буддистской природы. Я могу рассказать вам об этом поподробнее.

\emph{Черепаха} : Буду счастлива. Новичкам вроде меня так много приходится выучить!

\emph{Ахилл} : Говорят, что был даже специальный коан, повествующий о том, с чего началось Искусство Дзен-Цепочек. Но, к несчастью, он уже давным-давно уплыл по течению реки времен --- а она, как известно, уносит навечно. Впрочем может быть это и неплохо --- а то нашлись бы имитаторы, которые стали бы всячески копировать Мастера, пользуясь его именем.

\emph{Черепаха} : Разве плохо, если бы все ученики дзен-будцизма стали бы копировать Великого Ментора --- самого просветленного Мастера всех времен?

\emph{Ахилл} : Позвольте вместо ответа рассказать вам коан об имитаторе.

\emph{Мастер дзена по имени Гутей всегда поднимал палец когда его спрашивали о дзене. Молоденький ученик стал его копировать. Когда Гутей услышал об имитаторе, он позвал ученика и спросил правда ли это. «Да» --- признался тот. Тогда Гутей спросил его понимает ли он, что делает. Вместо ответа ученик поднял указательный палец. Гутей быстро отрезал палец, вопя от боли ученик побежал к двери. Когда он достиг выхода Гутей позвал его: «Мальчик!» Ученик обернулся, и Гутей поднял свой указательный палец. В этот момент юноша достиг Просветления.}

\emph{Черепаха} : Кто бы мог подумать! Как раз когда я решил, что дзен --- весь о Джошу и его проказах, оказалось, что и Гутей приглашен на праздник. Кажется, у него порядочное чувство юмора.

\emph{Ахилл} : Этот коан совершенно серьезен; не знаю, откуда у вас появилась мысль, что в нем какой-то юмор.

\emph{Черепаха} : Может быть, дзен так поучителен именно потому, что в нем много юмора. Мне кажется, что если воспринимать эти истории на полном серьезе, то в половине случаев их смысл пройдет мимо вас.

\emph{Ахилл} : Может быть, в этом Черепашьем Дзене и есть какой-то смысл.

\emph{Черепаха} : Можете ли вы ответить мне на один вопрос? Я хочу знать, почему Бодхидхарма приехал из Индии в Китай.

\emph{Ахилл} : Ого! Хотите, я вам скажу, что ответил Джошу на точно такой же вопрос?

\emph{Черепаха} : О, да!

\emph{Ахилл} : Он ответил: «Дуб в саду.»

\emph{Черепаха} : Разумеется; я сказала бы то же самое. С той разницей, что в моем случае это был бы ответ на другой вопрос: «Какое место лучше всего подходит, чтобы укрыться от полуденного солнца?»

\emph{Ахилл} : Вы, сами того не подозревая, затронули сейчас один из основных вопросов дзена. Вопрос звучит безобидно: «Каков основной принцип дзена?»

\emph{Черепаха} : Удивительно! Я и понятия не имела, что основная цель дзен-буддизма --- в том, чтобы найти место в тенёчке.

\emph{Ахилл} : Да нет же, вы меня совершенно не поняли. Я не имел в виду ЭТОТ вопрос. Я думал о первом вашем вопросе --- почему Бодхидхарма приехал из Индии в Китай.

\emph{Черепаха} : Понятно. Я и не знала, что ныряю на такую глубину\ldots{} Но вернемся к этим странным отображениям. Значит, любой коан может быть превращен в уложенную цепочку, следуя этому методу. А как насчет обратного процесса? Можно ли прочитать любую цепочку так, чтобы получился коан?

\emph{Ахилл} : В некотором роде. Однако\ldots{}

\emph{Черепаха} : Что такое?

\emph{Ахилл} : Вы просто не должны читать ее таким образом. Это нарушило бы Центральную Догму Дзен-цепочек, которую можно нарисовать следующим образом (рисует на салфетке):

коан~~~~ ~=\textgreater~~~~~~ ~посредник~~~ =\textgreater~~~ ~ уложенная цепочка

.~~~~ транскрипция~~~~~~~~~ ~трансляция

Идти против стрелок нельзя --- особенно против второй стрелки.

\emph{Черепаха} : Скажите мне: у этой догмы --- природа Будды, или нет? Впрочем, если подумать, то я, пожалуй, могу развопросить этот вопрос. Если вы, конечно, не возражаете\ldots{}

\emph{Ахилл} : Буду только рад. Я хочу открыть вам один секрет --- поклянитесь, что никому не скажете!

\emph{Черепаха} : Слово Черепахи.

\emph{Ахилл} : Иногда я все-таки двигаюсь против стрелок. Запретный плод сладок, знаете ли\ldots{}

\emph{Черепаха} : Ай да Ахилл! Понятия не имела, что вы способны на такие непочтительные действия!

\emph{Ахилл} : Я никому в этом не признавался --- даже Оканисаме.

\emph{Черепаха} : Так скажите мне, что получается, когда вы двигаетесь против стрелок Центральной Догмы? Это значит, что вы начинаете с цепочки и кончаете коаном?

\emph{Ахилл} : Иногда --- но часто случаются всякие странные вещи.

\emph{Черепаха} : Более странные, чем производство коанов?

\emph{Ахилл} : Да\ldots{} Когда вы делаете трансляцию и транскрипцию наоборот, у вас получается НЕЧТО, что не всегда является коаном. Некоторые цепочки, когда их читаешь вслух таким образом, звучат сплошной бессмыслицей.

\emph{Черепаха} : Разве это не синоним коана?

\emph{Ахилл} : Вижу, моя дорогая, что вы еще не прониклись подлинным духом дзена.

\emph{Черепаха} : По крайней мере, у вас хотя бы получаются рассказы?

\emph{Ахилл} : Не всегда; иногда выходят бессмысленные слоги, иногда --- предложения-окрошка. Но иногда выходит что-то, похожее на коан.

\emph{Черепаха} : Только ПОХОЖЕЕ?

\emph{Ахилл} : Видите ли, это может оказаться подделкой.

\emph{Черепаха} : Ах, разумеется.

\emph{Ахилл} : Я называю такие цепочки, которые производят коаны, «правильно сформированными.»

\emph{Черепаха} : А как вы отличаете поддельные коаны от подлинных?

\emph{Ахилл} : К этому я и веду. Имея коан (или не-коан, как иногда случается), первое, что надо сделать, --- это транслировать его в трехмерную цепочку. Потом остается только выяснить, буддистская ли природа у этой цепочки.

\emph{Черепаха} : Как же можно ухитриться проделать подобное?

\emph{Ахилл} : Мой Мастер говорит, что Великий Ментор мог узнать это, просто взглянув на цепочку.

\emph{Черепаха} : А если вы еще не достигли Архи-просветления? Есть ли иной способ узнать, буддистская ли природа у данной цепочки?

\emph{Ахилл} : Да, есть. Здесь как раз вступает в игру Искусство Дзен-цепочек. Этот способ --- создание бесконечного множества цепочек с буддистской природой.

\emph{Черепаха} : Да что вы говорите! А есть ли способ произвести цепочки БЕЗ буддистской природы?

\emph{Ахилл} : Зачем это вам?

\emph{Черепаха} : Я просто думала --- а вдруг это может пригодиться\ldots{}

\emph{Ахилл} : У вас весьма странный вкус. Надо же! Ей интереснее вещи не-буддистской природы, чем вещи с природой Будды!

\emph{Черепаха} : Можете приписать это моему непросветленному состоянию.

\emph{Ахилл} : Итак, сначала вы вешаете петлю цепочек на руки в одной из пяти дозволенных начальных позиций; например, вот так\ldots{} \emph{(Снимает длинную цепочку, висящую у него на шее, и надевает ее на руки, набрасывая петли между пальцами.)}

\emph{Черепаха} : Что представляют собой дозволенные позиции?

\emph{Ахилл} : Каждая из них --- это позиция, считающаяся самоочевидным способом брать цепочку. Даже новички часто берут цепочки именно так. И все эти пять цепочек имеют природу Будды.

\emph{Черепаха} : Разумеется.

\emph{Ахилл} : Кроме того, имеются некоторые Правила Обращения с Цепочками, следуя которым, можно произвести из цепочек более сложные фигуры. В частности, позволено изменять форму вашей цепочки при помощи простейших движений рук. Например, вы можете взяться за эту цепочку здесь и потянуть вот так --- а теперь так перекрутить. Каждая операция меняет конфигурацию цепочки, надетой на ваши руки.

\emph{Черепаха} : Это выглядит, как игра в веревочку --- «колыбель для кошки» и прочие занимательные фигуры, которые можно сплести из веревки, надетой на пальцы.

\emph{Ахилл} : Верно. Смотрите, некоторые из этих правил усложняют цепочку, а некоторые упрощают. Но неважно, в каком порядке вы это делаете; пока вы следуете Правилам Обращения с Цепочками, любая ваша цепочка будет иметь природу Будды.

\emph{Черепаха} : Это чудесно. А как насчет коана, спрятанного в строчке, что вы только что сплели? Будет ли он подлинным?

\emph{Ахилл} : Согласно тому, что я выучил, именно так и будет. Поскольку я придерживался Правил и начал в одной из пяти самоочевидных позиций, цепочка должна иметь природу Будды и, следовательно, соответствовать подлинному коану.

\emph{Черепаха} : Знаете ли вы, какому именно?

\emph{Ахилл} : Вы хотите, чтобы я нарушил Центральную Догму? Ах вы, вредное создание!

\emph{(Ахилл раскрывает книгу Кода и, высунув от усердия язык, дюйм за дюймом продвигается вдоль цепочки, записывая каждый поворот с помощью тройки геометрических символов этого странного фонетического алфавита для коанов, пока салфетка не оказывается исписанной его каракулями)}

Готово!

\emph{Черепаха} : Здорово! Теперь давайте почитаем, что получилось.

\emph{Ахилл} : Хорошо.

\emph{Путешествующий монах спросил у старухи дорогу к Тайзаиу, известному храму, превращающему тех, кто в нем молится, в мудрецов. Старуха ответила: «Идите прямо». Когда тот удалился, старуха пробормотала себе под нос: «Еще один паломник». Кто-то рассказал об этом случае Джошу, и тот заметил: «Подождите, я сам проверю». На следующий день он отправился тем же путем и задал тот же вопрос. Старуха повторила свой ответ, и Джошу сказал: «Я проверил эту старую женщину».}

\emph{Черепаха} : С его страстью к расследованиям, жаль, что Джошу никогда не работал в ФБР. А скажите, я могла бы повторить то, что вы сейчас сделали, если бы следовала Правилам Искусства Дзен-цепочек, не правда ли?

\emph{Ахилл} : Совершенно верно.

\emph{Черепаха} : Я должна буду проделывать все операции в том же ПОРЯДКЕ, как и вы?

\emph{Ахилл} : Да нет, годится любой порядок.

\emph{Черепаха} : Разумеется, тогда я получу другую цепочку и, следовательно, другой коан. Теперь скажите мне, я должна буду повторить то же ЧИСЛО операций?

\emph{Ахилл} : Ни в коем случае. Вы можете делать любое число шагов.

\emph{Черепаха} : В таком случае, есть бесконечное множество цепочек с природой Будды --- а следовательно, бесконечное множество подлинных коанов! Но откуда вы знаете, есть ли какая-либо цепочка, которая НЕ МОЖЕТ быть получена при помощи ваших Правил?

\emph{Ахилл} : Ах, да --- вернемся к вещам, лишенным природы Будды. Получается так, что как только вы научитесь производить цепочки БУДДИСТСКОЙ природы, вы сразу же научитесь производить и HE-БУДДИСТСКИЕ цепочки. Это мой Мастер вдолбил в меня с самого начала.

\emph{Черепаха} : Прекрасно! Как же это получается?

\emph{Ахилл} : Очень просто. Вот, смотрите: сейчас я сделаю цепочку, у которой нет природы Будды\ldots{}

\emph{(Он берет цепочку, из которой был «извлечен» предыдущий коан, и завязывает на одном из концов неточку, затягивая ее большим и указательным пальцами.)}

Готово: в этой цепочке НЕТ никакой буддистской природы.

\emph{Черепаха} : Потрясающе! Я просвещаюсь с каждой минутой. И всего-то понадобилась какая-то ниточка? Откуда вы знаете, что у новой цепочки нет буддистской природы?

\emph{Ахилл} : Не ниточка, а НЕТОЧКА --- именно так указал мастер. Основное свойство природы Будды таково: если две правильно сформированные цепочки отличаются только тем, что одна из них имеет неточку на конце, то только ОДНА из этих цепочек может иметь буддистскую природу.

\emph{Черепаха} : А скажите: есть ли такие цепочки буддистской природы, которые НЕВОЗМОЖНО получить, в каком бы порядке мы не применяли Правила Дзен-цепочек?

\emph{Ахилл} : Стыдно признаться, но этого я сам точно не знаю. Сначала мой мастер говорил, что буддистская природа цепочки ОПРЕДЕЛЕНА тем, что мы начинаем с одной из пяти начальных позиций и затем строго следуем Правилам. Но позже он сказал что-то о какой-то «Теореме», как бишь его\ldots{} Гоголя?., или Де Голля? Боюсь, что я так этого и не понял; а может быть, просто не расслышал. Но так или иначе, у меня появилось сомнение, можно ли получить этим методом ВСЕ цепочки с природой Будды. До сих пор мне это удавалось, но ведь буддистская природа --- штука непростая, знаете ли\ldots{}

\emph{Черепаха} : Я так и думала, судя по «МУ» Джошу. Хотелось бы мне знать\ldots{}

\emph{Ахилл} : Что такое?

\emph{Черепаха} : Я думала о тех двух коанах\ldots{} Я имею в виду, коан и не-коан: «Этот разум --- Будда» и «Этот разум --- не Будда». Как они выглядят, если перевести их в цепочки по Геометрическому Коду?

\emph{Ахилл} : С удовольствием вам покажу.

\emph{(Он записывает фонетическую транскрипцию, достает из кармана пару цепочек и начинает аккуратно, дюйм за дюймом, складывать их, следуя тройкам символов, записанных странным алфавитом. Затем он кладет получившиеся цепочки рядом.)}

Видите, они различаются.

\emph{Черепаха} : На мой взгляд, они весьма схожи. О, теперь я вижу, в чем разница: на конце у одной из них --- неточка!

\emph{Ахилл} : Клянусь Джошу, вы правы.

\emph{Черепаха} : Ага! Я понимаю теперь, почему ваш Мастер не доверял этим коанам.

\emph{Ахилл} : Неужели?

\emph{Черепаха} : Согласно его указаниям, НЕ БОЛЕЕ, ЧЕМ ОДНА цепочка из этой пары может иметь природу Будды; так что сразу можно сказать, что один из коанов --- подделка.

\emph{Ахилл} : Но это еще не говорит нам, какой именно. Мы с моим Мастером давно пытаемся сложить эти цепочки, следуя Правилам; но у нас пока ничего не выходит. Это ужасно неприятно, и можно начать сомневаться\ldots{}

\emph{Черепаха} : В том, что у этих цепочек вообще есть природа Будды? Может быть, ее нет ни у одной цепочки, и оба коана поддельны?

\emph{Ахилл} : Я никогда не заходил так далеко --- но вы правы, в принципе это возможно. Однако вы не должны задавать так много вопросов о природе Будды. Мастер дзена Мумон всегда предупреждал своих учеников, что слишком много спрашивать опасно.

\emph{Черепаха} : Хорошо --- вопросов больше не будет. Но зато мне очень хочется самой уложить цепочку. Интересно посмотреть, получится ли она правильно сформированной.

\emph{Ахилл} : И правда, интересно. Вот, пожалуйста. \emph{(Передает цепочку Черепахе.)}

\emph{Черепаха} : Вы знаете, я понятия не имею, что с ней делать. Что ж, рискнем --- мое неуклюжее произведение, сделанное без Правил, как Бог на душу положит, будет, скорее всего, совершенно невозможно расшифровать. \emph{(Берет цепочку, делает из нее петлю, и несколькими движениями лап укладывает цепочку в сложный узор, который затем молча протягивает Ахиллу. В этот момент лицо воина освещается.)}

\emph{Ахилл} : Вот это да! Я должен попробовать этот метод сам. Никогда не видел подобной цепочки!

\emph{Черепаха} : Надеюсь, что она правильно сформирована.

\emph{Ахилл} : На одном конце у нее завязана неточка.

\emph{Черепаха} : Ох, погодите --- можно мне эту цепочку на минутку? Я хочу еще кое-что сделать.

\emph{Ахилл} : Почему бы~и нет --- пожалуйста.

\emph{(Снова протягивает ее Черепахе, та завязывает еще одну неточку на том же конце. После этого она встряхивает цепочку и внезапно обе неточки исчезают!)}

\emph{Ахилл} : Что случилось?

\emph{Черепаха} : Я просто хотела избавиться от той неточки.

\emph{Ахилл} : Но вместо того, чтобы ее развязать, вы завязали еще одну, и тут их как ножом отрезало, обе исчезли! Куда они подевались?

\emph{Черепаха} : В Лимбедламию, разумеется. Это Закон Двойного Отрезания.

\emph{(Вдруг обе неточки опять появляются ниоткуда --- то бишь, из Лимбедламии.)}

\emph{Ахилл} : Удивительно. К некоторым районам Лимбедламии, видно, существует легкий доступ, если эти неточки могут так запросто проталкиваться и выталкиваться. Или же вся Лимбедламия одинаково недоступна?

\emph{Черепаха} : Не могу вам сказать. Правда, я думаю, что если бы мы эту цепочку расплавили, то неточки вряд ли вернулись бы. В этом случае, мы считали бы, что они попали на более глубокий уровень Лимбедламии. Там, возможно, есть миллионы уровней. Но это для нас неважно. Меня сейчас интересует то, как эта цепочка зазвучит, если мы переведем ее обратно в фонетические символы.

\emph{Ахилл} : Я всегда чувствую себя виноватым, когда нарушаю Центральную Догму.

\emph{(Достает ручку и книгу Кода и аккуратно записывает тройные символы, соответствующие поворотам Черепашьей цепочки; когда все готово, он откашливается.)} Кхе-кхе. Послушаем, что у вас получилось\ldots{}

\emph{Черепаха} : Если вы готовы\ldots{}

\emph{Ахилл} : Отлично. Вот что тут написано:

\emph{Один монах постоянно приставал к Великой Чепупахе (единственной, которая когда-либо достигла Архи-просветлеиия), спрашивая у нее, имеют ли те или иные вещи природу Будды. Чепупаха отвечала на эти вопросы молчанием. Монах уже спросил о бобе, озере, и лунной ночи. Однажды он принес Чепупахе кусочек цепочки и задал тот же вопрос. В ответ Чепупаха взяла цепочку, сделала из нее петлю и несколькими движениями лап ---}

\emph{Черепаха} : Несколькими движениями лап? Как странно!

\emph{Ахилл} : Почему же именно Вы находите это странным?

\emph{Черепаха} : Ах да, конечно, вы правы. Продолжайте, прошу вас!

\emph{Ахилл} : Хорошо.

\emph{Несколькими движениями лап Чепупаха уложила цепочку в сложный узор, который затем молча протянула монаху. В этот момент монах достиг Просветления.}

\emph{Черепаха} : Что до меня, то я бы предпочла Архи-просветление.

\emph{Ахилл} : Далее тут описывается, как сделать цепочку Великой Чепупахи, если начать с петли, наброшенной на лапы. Эти скучные детали я пропущу\ldots{} А вот и конец:

\emph{С тех пор монах больше не приставал к Чепупахе. Вместо этого он укладывал цепочку за цепочкой по ее методу; он передал этот метод своим ученикам, а те --- своим.}

\emph{Черепаха} : Ну и хитросплетение! Трудно поверить, что все это было спрятано в моей цепочке.

\emph{Ахилл} : Так оно и есть. Удивительно, что вы смогли уложить правильно сформированную цепочку --- верно говорят, что новичкам везет!

\emph{Черепаха} : Но как же выглядела цепочка Великой Чепупахи? Мне кажется, в этом самая суть коана.

\emph{Ахилл} : Сомневаюсь. Мы не должны «привязываться» к таким мелочам. Главное не детали, а дух коана как целого. А знаете, что мне только что пришло в голову? Я думаю, что вы, как это ни удивительно, только что наткнулись на давно утерянный коан, описывающий происхождение Искусства Дзен-цепочек!

\emph{Черепаха} : О, это было бы слишком хорошо для того, чтобы иметь буддистскую природу!

\emph{Ахилл} : Но это бы значило, что великий Мастер, единственный, кто достиг мистического состояния Архи-просветления, звался не Ментором, а Чепупахой. Вот уж поистине странное имя!

\emph{Черепаха} : Я не согласна --- по-моему, это очень красивое имя. Но я все же хочу знать, как выглядела эта Чепупашья цепочка. Можете ли вы воссоздать ее по описанию, данному в коане?

\emph{Ахилл} : Я могу попытаться, хотя мне это будет очень трудно --- ведь у меня нет лап, а в коане все описывается с точки зрения движения именно лап. Это очень необычно, но я постараюсь. Попытка --- не пытка\ldots{}

\emph{(Он берет коан и кусочек цепочки и в течение нескольких минут, пыхтя от усердия, сгибает и складывает его самым невероятным образом, пока в его руках не оказывается готовый продукт.)}

Вот, пожалуйста. Странно, но это выглядит очень знакомо.

\emph{Черепаха} : И правда! Интересно, где я это видела?

\emph{Ахилл} : Я знаю! Это же ВАША цепочка, разве не так?

\emph{Черепаха} : Наверняка нет.

\emph{Ахилл} : Ну конечно: это ваша первая цепочка, которую вы мне дали до того, как завязали вторую неточку.

\emph{Черепаха} : Действительно, она самая. Надо же\ldots{} Интересно, что из этого следует?

\emph{Ахилл} : Все это очень странно, чтобы не сказать большего.

\emph{Черепаха} : Вы думаете, мой коан --- подлинный?

\emph{Ахилл} : Подождите-ка минутку\ldots{}

\emph{Черепаха} : А эта цепочка --- есть ли в ней природа Будды?

\emph{Ахилл} : Ваша цепочка кажется мне подозрительной\ldots{}

\emph{Черепаха (с предовольным видом, не обращая на Ахилла никакого внимания)} : А как насчет Чепупашьей цепочки? Есть ли в ней природа Будды? У меня столько вопросов!

\emph{Ахилл} : Я бы поостерегся задавать столько вопросов, г-жа Ч. Что-то здесь творится, и я совсем не уверен, что это мне нравится.

\emph{Черепаха} : Грустно слышать; но я не понимаю, что вас тревожит?

\emph{Ахилл} : Лучше всего это объясняет цитата из другого древнего Мастера дзен-буддизма по имени Киоген. Киоген сказал:

\emph{Дзен подобен человеку, удерживающемуся зубами за ветку растущего над пропастью дерева. Руки и ноги его, не имея опоры, болтаются в воздухе. Под деревом стоит другой человек и спрашивает его. «Почему Бодхидхарма пришел из Индии в Китай?». Если человек на дереве не ответит, он изменит дзену, а если он ответит, то упадет и погибнет. Что ему делать?}

\emph{Черепаха} : Ясно как день: ему надо оставить дзен и заняться молекулярной биологией.


% \subsubsection{ГЛАВА IX: Мумон и Гёдель}
% \subsubsection{ГЛАВА IX: Мумон и Гёдель}

Что такое дзен-буддизм?

Я НЕ УВЕРЕН В ТОМ, что знаю, что такое дзен. В каком-то смысле мне кажется, что я понимаю его очень хорошо; с другой стороны, иногда я думаю, что никогда не пойму в нем ничего. С тех пор, как на первом курсе университета профессор английской литературы прочитал нам «„МУ`` Джошу», я начал бороться с дзен-буддистскими аспектами жизни и, наверное, никогда не перестану. Для меня дзен подобен зыбучим пескам~--- анархия, темнота, бессмыслица, хаос. Он дразнит и приводит в бешенство. И в то же время дзен полон юмора, свежести и привлекательности. В нем есть собственный тип значения, блеска и ясности. Надеюсь, что, прочитав эту главу, вы это почувствуете. И эта тема, как ни странно может показаться, выведет нас прямо к Гёделю.

Одна из главных идей дзен-буддизма в том, что его невозможно определить. Как бы вы не пытались заключить его в словесные рамки, он сопротивляется и вырывается на свободу. Может показаться, что в таком случае все попытки объяснить дзен~--- это пустая трата времени. Но ученики и мастера дзена так не считают. Например, буддистские коаны~--- центральная часть изучения дзена, хотя они и состоят из слов. Коаны призваны служить «триггерами»~--- сами по себе они не содержат достаточно информации, чтобы вызвать Просветление, но могут привести в действие внутренние механизмы, ведущие к Просветлению. Но в общем дзен утверждает, что слова и истина несовместимы~--- словами невозможно уловить истину.

Мастер дзена Мумон

Возможно, чтобы лучше выразить эту идею, монах Мумон (что в переводе означает «Нет выхода»), живший в тринадцатом веке, собрал сорок восемь коанов, снабдив каждый из них комментарием и небольшим «стихотворением». Этот труд называется «Безвыходный выход» или «Мумонкан.» Интересно заметить, что даты жизни Мумона и Фибоначчи почти точно совпадают: Мумон жил в Китае с 1183 по 1260 год, а Фибоначчи - в Италии с 1180 по 1250 год. Те, кто попытаются «понять» коаны «Мумонкана», найти в них смысл, будут горько разочарованы, поскольку как сами коаны, так и комментарии к ним и стихотворения абсолютно туманны. Приведу несколько примеров:\footnote{Paul Reps «Zen Flesh Zen Bones» стр. 110-11}

\emph{Коан:}

Хоген из монастыря Сеирио собирался читать обычную лекцию перед ужином; вдруг он заметил, что бамбуковая занавесь, опущенная для медитации, еще не поднята. Он указал на нее; два монаха безмолвно встали и подняли занавесь. Хоген, наблюдая за этим физическим моментом, заметил: «Состояние первого монаха~--- хорошо, но не состояние второго».

\emph{Комментарий Мумона:}

Я спрашиваю вас: кто из этих двух монахов выиграл, а кто проиграл? Если у кого-то из вас~--- один глаз, тот заметит ошибку Учителя. Но я не обсуждаю выигрыша и проигрыша.

\emph{Рис. 46. М. К. ~Эшер. «Три мира» (литография, 1955).}

\emph{Стихотворение Мумона:}

Когда занавесь поднята,

открывается широкое небо

Но небо не созвучно дзену.

Лучше забыть о широком небе И спрятаться от любого ветра.

А вот ещё:\footnote{Там же стр. 119}

\emph{Коан:}

Госо сказал: «Когда бизон выходит из укрытия на край пропасти, его рога, и голова, и копыта проходят; но почему не может пройти его хвост?»

\emph{Комментарий Мумона:}

Если кто-нибудь сейчас может открыть один глаз и сказать слово дзена, тот готов к тому, чтобы отплатить за четыре награды~--- более того, он сможет спасти всех существ, стоящих ниже него. Но если он не может сказать слова дзена, то он должен повернуться обратно к своему хвосту.

\emph{Стихотворение Мумона} :

Если бизон побежит, он упадет в пропасть;

Если он повернет назад, его зарежут.

Очень странная штука

---~Этот хвост!

Я думаю, вы согласитесь с тем, что объяснения Мумона не многое проясняют. Можно сказать, что в данном случае метаязык (на котором пишет Мумон) не слишком отличается от предметного языка (языка коанов). Некоторые считают, что комментарии Мумона~--- намеренно идиотские, и что он хочет показать, насколько бесполезно тратить время на разговоры о дзене. Однако комментарии Мумона можно понять на нескольких уровнях. Например, давайте рассмотрим следующий:

\emph{Коан:}\footnote{Там же стр. 111-12}

Один монах спросил Нансена: «Есть ли поучение, которое не произнес ни один мастер?»

Нансен сказал: «Да, есть».

«Какое же оно?» - спросил монах.

Нансен ответил: «Это не разум, это не Будда, это не вещи.»

\emph{Комментарий Мумона} :

Старый Нансен раскрыл свои заветные слова. Наверняка, он был очень взволнован.

\emph{Стихотворение Мумона} :

Нансен был слишком добр и потерял свое сокровище.

Поистине, слова бессильны.

Даже если гора обратится в море,

Слова не могут открыть разум другого.

Кажется, что в этой поэме Мумон говорит нечто центральное для дзен-буддизма и не делает никаких дурацких заявлений. Любопытно, однако, что поэма автореферентна и, таким образом, комментирует не только слова Нансена, но и свою собственную безрезультатность. Подобные парадоксы характерны для дзена. Это попытка «сломить разум логики». То же парадоксальное качество вы можете увидеть и в коане. Говоря о комментарии Мумона --- как вы думаете, был ли Нансен так уверен в своем ответе? И важна ли «правильность» его ответа? Играет ли вообще правильность какую-либо роль в дзен-буддизме? Какая разница между правильностью и истинностью, и есть ли она вообще? Что, если бы Нансен сказал: «Нет, такого поучения нет»? Что бы это изменило? Был бы подобный ответ увековечен в коане?

\emph{Рис. 47. М. К. Эшер. «Капля росы» (глубокая печать, 1948).}

Вот еще один коан, направленный на то, чтобы сломить разум логики:\footnote{«Zen Buddhism» (Mount Vernon NY Peter Pauper Press 1959) стр. 22}

\emph{Ученик Доко пришел к мастеру дзена и сказал: «Я ищу истину. До какого состояния я должен натренировать свои разум, чтобы ее найти?»}

\emph{Мастер ответил: «Поскольку разума не существует, его невозможно привести ни в какое состояние. Поскольку истины не существует, невозможно натренировать себя для ее обретения.»}

\emph{«Если нет ни разума, чтобы тренировать его, ни истины, чтобы ее искать, то зачем же вы каждый день собираете перед собой монахов для тренировки и изучения дзена?»}

\emph{«Но у меня здесь нет ни дюйма места,»~--- сказал мастер, «как же здесь могут собираться монахи? У меня нет языка~--- как же я могу созывать или учить их?»}

\emph{«Как вы можете так лгать?»~--- спросил Доко.}

\emph{«У меня нет языка, чтобы разговаривать с другими - так как же я могу лгать тебе?»~---~спросил мастер.}

\emph{Тогда Доко грустно заметил: «Я не могу уследить за вашей мыслью. Я вас не понимаю.»}

\emph{«Я сам себя не понимаю».}

Если какой-либо коан и служит для того, чтобы запутать слушателя, то именно этот. И скорее всего, в этом и есть его прямое назначение, потому что когда наш разум заходит в тупик, он начинает оперировать до какой-то степени нелогично. Теория говорит нам, что только отходя от логики, человек может достичь Просветления. Но что же такого плохого в логике? Почему она не позволяет нам совершить скачок к Просветлению?

Борьба дзена против дуализма

Чтобы ответить на эти вопросы, необходимо знать кое-что о Просветлении. Возможно, что самым коротким его определением было бы следующее: выход за пределы дуализма. Что же такое дуализм? Это мысленное разделение мира на категории. Возможно ли преодолеть это естественное стремление? Предваряя слово «разделение» словом «мысленное», я мог создать у вас впечатление, что это --- интеллектуальное или сознательное усилие и, значит, дуализм можно преодолеть, просто остановив мысли (словно это так легко --- перестать думать!). На самом деле, разбиение мира на категории происходит гораздо глубже самого высокого уровня мышления: дуализм настолько же процесс \emph{восприятия} мира, как и его \emph{понимания.} Иными словами, человеческое восприятие по своей природе дуалистично --- что делает борьбу за просветление титанической, если не сказать большего.

\emph{Рис. 48. М. К. Эшер. «Иной мир» (гравюра на дереве, 1947)}

В сердце дуализма, согласно дзену, лежат слова --- простые, обыкновенные слова. Использование слов всегда дуалистично --- очевидно, что каждое слово представляет собой определенную умозрительную категорию. Отсюда следует, что большая часть дзена посвящена борьбе против доверия к словам. Одно из лучших оружий в этой борьбе --- коан, поскольку со словами он обращается настолько неуважительно, что наш разум тут же забуксует, если мы попытаемся воспринимать коан серьезно. Может быть, неверно говорить, что врагом Просветления является логика; скорее, это дуалистичное, словесное мышление. Или даже еще проще, восприятие. Воспринимая предмет, вы тут же отграничиваете его от всего остального мира; вы делите мир на части и, таким образом, отходите от Пути.

\emph{Рис. 49. М. К. Эшер. «День и ночь» (гравюра на дереве, 1938).}

Вот коан, демонстрирующий борьбу против слов:\footnote{Reps стр. 124}

\emph{Коан} :

Шузан протянул вперед свою короткую палку и сказал: «Если вы скажете, что это короткая палка, то согрешите против действительности. Если вы не скажете, что это короткая палка, то проигнорируете факт. Так что же вы скажете?»

\emph{Комментарий Мумона} :

Если вы скажете, что это короткая палка, то согрешите против действительности. Если вы не скажете, что это короткая палка, то проигнорируете факт. Это нельзя выразить словами, и это нельзя выразить без слов. Теперь быстро говорите, что это такое.

\emph{Стихотворение Мумона} :

Протягивая вперед короткую палку,

Он дал приказ о жизни и смерти.

Позитивное и негативное переплетены,

Даже Будды и патриархи не могут избежать этой атаки.

(Под «патриархами» здесь имеются в виду шесть почитаемых основателей дзен-буддизма, из которых первым был Бодхидхарма и шестым~--- Энон.)

Почему назвать это короткой палкой означало пойти против действительности? Может быть, потому, что подобная категоризация дает иллюзию углубления в действительность, в то время как на самом деле это утверждение даже не поцарапало ее поверхности. Можно сравнить его с утверждением «5 --- простое число.» Это утверждение оставляет без внимания огромное, бесконечное количество фактов. С другой стороны, не назвать ее короткой палкой --- означает проигнорировать этот факт, как бы незначителен он не был. Следовательно, слова ведут к частичной истинности --- и, возможно, к частичной ложности --- но, безусловно, не к полной истине. Надеяться на слова, чтобы найти истину --- все равно, что надеяться на неполную формальную систему, чтобы найти истину. Формальная система даст вам некоторые истины, но, как мы скоро увидим, формальная система, какой бы мощью она не обладала, не может привести ко всем истинным высказываниям. Дилемма математиков такова: на что еще можно опираться, кроме формальных систем? Дилемма последователей дзена такова: на что еще можно опираться, кроме слов? Мумон выражает эту дилемму с предельной ясностью: «Это нельзя выразить словами, и это нельзя выразить без слов.»

\emph{Рис. 50. М. К. Эшер. «Кожура» (гравюра на дереве, 1955).}

Вот еще один коан о Нансене:\footnote{«Zen Buddhism» стр. 38}

\emph{Джошу спросил учителя Нансена: «Какой Путь правилен?»}

\emph{Нансен ответил: «Правилен повседневный Путь».}

\emph{Джошу спросил: «Могу ли я его изучать?»}

\emph{Нансен ответил: «Чем больше вы его изучаете, тем больше вы удаляетесь от него».}

\emph{Джошу спросил: «Если я не буду его изучать, как же я его узнаю?»}

\emph{Нансен ответил: «Путь не принадлежит увиденным вещам и не принадлежит неувиденным вещам. Он не принадлежит известным вещам, и он не принадлежит неизвестным вещам. Не ищи его, не изучай его и не называй его. Чтобы оказаться на Пути, стань открытым и широким как небо.»} (См. рис. 50.)

Кажется, что это любопытное утверждение полно парадоксов. Оно немного напоминает следующее верное средство от икоты: «Обегите трижды вокруг дома, не думая о слове „волк``.» Дзен-буддизм --- это философия, которая, по-видимому, считает, что дорога к абсолютной истине, так же, как единственный верный способ против икоты, должна изобиловать парадоксами.

Изм, режим U и Унмон

Если слова --- плохи, и мышление --- плохо, то что же тогда хорошо? Разумеется, сам по себе такой вопрос весьма дуалистичен, но поскольку, обсуждая его, мы не претендуем на верность дзену, то попытаемся ответить на него серьезно. Назовем то, к чему стремится дзен, \emph{измом} . Изм --- это антифилософия, способ существования без мышления. Мастерами изма являются камни, деревья, моллюски. Существам же, стоящим на более высокой ступени развития, приходится за это бороться; при этом они никогда не достигнут полного изма. Все же нам иногда удается увидеть проблеск изма; возможно, следующий коан покажет вам такой проблеск:\footnote{Reps стр. 121}

\emph{Хиакуйо захотел послать монаха, чтобы открыть новый монастырь. Он сказал ученикам, что назначит того из них, кто сумеет лучше всех ответить на его вопрос. Поставив кувшин с водой на землю, он сказал: «Кто может сказать, что это такое, не называя при этом его имени?»}

\emph{Главный монах сказал: «Никто не может назвать это деревянным башмаком.»}

\emph{Повар Изан перевернул кувшин ногой и ушел.}

\emph{Хиакуйо улыбнулся и сказал: «Главный монах проиграл.» И Изан стал Мастером нового монастыря.}

Сущность дзена~--- и изма~--- в том, чтобы подавить восприятие, подавить логическое, словесное, дуалистичное мышление. Это и есть \emph{режим \textbf{U}} ~--- \emph{Ультра} ; не Интеллектуальный, не Механический, а просто «Ультра». Джошу действовал по способу \textbf{U} ; поэтому его МУ «развопросило» вопрос. Для Мастера дзена Унмона способ \textbf{U} был естественным:\footnote{Gyomay M. Kubose «Zen Koans» стр. 35}

\emph{Однажды Унмон сказал своим ученикам: «Моя палка превратилась в дракона и проглотила вселенную! Где же теперь реки, и горы, и великая Земля?»}

Дзен~--- это холизм, доведенный до логической крайности. Если холизм (от английского~\emph{whole} --- целое) утверждает, что вещи могут быть поняты только как целое, а не как сумма их частей, то дзен идет еще дальше, утверждая, что мир вообще не может быть разделен на части. Делить мир на части~--- это создавать иллюзии и терять возможность Просветления.

\emph{Один любопытный монах спросил Мастера : «Что такое Путь?»}

\emph{«Он у тебя перед глазами», - ответил Мастер.}

\emph{«Почему же я сам его не вижу?»}

\emph{«Потому что ты думаешь о себе».}

\emph{«А вы - вы его видите?»}

\emph{«До тех пор, пока ты все представляешь в двойном свете, говоря „я вижу``, „вы не видите`` и тому подобное, у тебя всегда будет туман перед глазами», - сказал Мастер.}

\emph{«А когда не станет ни „Я`` ни „Вы``, его можно будет увидеть?»}

\emph{«Когда не станет ни „Я`` ни „Вы``, кто будет тот, кто захочет его видеть?»} \footnote{«Zen Buddhism» стр. 31}

По-видимому, Мастер хочет сказать, что в состоянии Просветления границы между «Я» и остальным миром стираются.

Это было бы настоящим концом дуализма, поскольку тогда, как сказал Мастер, не осталось бы системы, жаждущей восприятия. Но что это такое, если не смерть? Как может живое человеческое существо стереть границы между собой и миром?

Дзен и Лимбедламия

Буддистский монах Бассуи написал письмо одному из своих учеников, который был при смерти; в письме он сказал: «Твой конец --- бесконечен; он подобен снежинке, таящей в чистом воздухе». Снежинка, бывшая вполне заметной подсистемой, теперь растворяется в более широкой системе, когда-то ее содержавшей. Хотя она больше и не присутствует как отдельная система, ее сущность все еще сохраняется. Она кружится в Лимбедламии, рядом с неначавшейся икотой и буквами непрочитанных историй\ldots{} Так я понимаю письмо Бассуи.

Дзен признает свои ограничения, точно так же, как математики научились признавать ограничения аксиоматического метода для нахождения истины. Дзен не знает ответа на то, что лежит за его пределами, так же, как у математиков нет ясного понимания форм рассуждений, лежащих за пределами формализации. Одно из самых ясных утверждений дзена о его границах дано в следующем странном, весьма в духе Нансена, коане:\footnote{Kubose стр. 110}

\emph{Тозан сказал своим монахам: «Вы, монахи, должны знать, что в буддизме есть еще высшее понимание.» Один монах вышел вперед и спросил: «Что такое высший буддизм?» Тозан ответил: «Это не Будда.»}

Путь не кончается никогда; Просветление не означает конца буддизма. Не существует рецепта, говорящего, как можно переступить пределы буддизма; единственное, в чем можно быть уверенным, это то, что Будда~--- \emph{не} путь. Дзен --- это система, и он не может быть своей собственной метасистемой; всегда есть что-то вне дзена, что не может быть полностью понято и описано в его терминах.

\emph{Рис. 51. М. К. Эшер. «Лужа» (гравюра на дереве, 1952).}

Эшер и дзен

В своих сомнениях относительно восприятия и своей любви к абсурдным загадкам без ответа дзен имеет единомышленника~--- М.К. Эшера. Взгляните на «\emph{День и ночь} » (рис: 49)~--- шедевр «переплетения негативного и позитивного» (говоря словами Мумона). Читатель может спросить: «Что это такое на самом деле, птицы или поля? Что это, день или ночь?» Однако все мы знаем, что подобные вопросы задавать незачем. Эта картина, подобно буддистскому коану, пытается разбить разум логики. Эшер также находит удовольствие в создании противоречивых картин, таких, как «\emph{Иной мир} » (рис. 48)~--- картин, которые играют с реальностью и нереальностью на манер дзена. Должны ли мы принимать Эшера всерьез? Должны ли мы принимать дзен всерьез?

Взгляните на изысканный, подобный хайку рисунок отражений в «\emph{Капле росы} » (рис. 47), на спокойную луну, отраженную в тихой воде «\emph{Лужи} » (рис. 51) и на «\emph{Рябь на воде} » (рис. 52). Отражение луны~--- тема, встречающаяся в нескольких коанах. Вот лишь один пример:\footnote{Там же стр. 120}

\emph{Чионо изучала дзен многие годы под руководством Букко из Энгаку. Все же ей не удавалось достичь плодов медитации. Однажды лунной ночью она несла воду в старой деревянной бадье, опоясанной бамбуком. Бамбуковый обруч сломался, и дно бадьи выпало. В этот момент Чионо освободилась. Тогда она сказала: «Нет воды в ведре --- нет и луны в воде.»}

«Три мира»~--- картина Эшера (рис. 46) и тема дзен-буддистского коана:\footnote{Там же стр. 180}

\emph{Один монах спросил Ганто: «Чте мне делать, когда мне угрожают три мира?» Ганто ответил: «Садись». «Я не понимаю»,~--- сказал монах. Ганто сказал: «Подними гору и принеси ее мне. Тогда я тебе объясню».}

\emph{Рис. 52. М. К. Эшер. «Рябь на воде» (гравюра на линолеуме, 1950)}

Гемиола и Эшер

В картине «\emph{Вербум} » (рис. 149) противоположности превращены в единство на нескольких уровнях. Двигаясь по кругу, мы видим постепенные превращения черных птиц~--- в белых птиц~--- в белых рыб~--- в черных жаб~--- в белых жаб~--- в черных птиц\ldots~После шести шагов мы оказались опять в начале! Не примирение ли это дихотомии белого и черного? Или «трихотомии» птиц, рыб и жаб? Или это~--- шестиступенчатое единство, сделанное из противопоставления четности двух и нечетности трех? В музыке шесть нот одинаковой длительности создают ритмическую двусмысленность: две группы по три ноты, или три группы по две? Эта двусмысленность имеет свое название: гемиола. Шопен был мастером гемиолы; см. его «Вальс» оп. 42, или «Этюд» оп. 25, номер 2. У Баха это «Темпо ди Менуетто» из клавишной партитуры номер 5 или удивительный финал соль минорной «Сонаты для скрипки соло».

Когда мы приближаемся к центру гравюры «Вербум», различия постепенно стираются, и к концу остается не три, не две, но одна единственная суть: Вербум~--- слово, сверкающее во всем блеске, возможно, символ Просветления. Ирония в том, что «вербум» не только \emph{является} словом, но и означает «слово»~--- не слишком-то совместимое с дзеном понятие. С другой стороны, «вербум»~--- единственное слово в картине. Мастер дзена Тозан однажды сказал «Вся „Трипитака`` может быть выражена в одной букве.» («Трипитака» или «Три корзины» - это полный текст священных книг буддизма.) Интересно, какой декодирующий механизм понадобился бы, чтобы извлечь три корзины из одной буквы? Может быть, механизм с двумя полушариями?

\emph{Рис. 53. М. К. Эшер. «Три сферы II» (литография, 1946).}

Сеть Индры

Наконец, давайте взглянем на «Три сферы II»; кажется, что каждая часть мира здесь содержит все остальные и содержится в них сама: письменный стол отражает шары на его поверхности, шары отражают друг друга, а также стол, рисунок, их изображающий, и самого художника. Эта литография лишь намекает на бесконечное взаимодействие всех вещей --- однако этого намека вполне достаточно. Буддистская аллегория «Сеть Индры» описывает бесконечную сеть, нити которой пронизывают всю вселенную: горизонтальные нити протянуты в пространстве, вертикальные --- во времени. Каждая точка пересечения --- это индивидуум, и каждый индивидуум --- это стеклянная сфера. Великий свет «Абсолютного существа» освещает каждую стеклянную сферу и проникает сквозь нее; более того, каждая сфера отражает не только свет каждой другой сферы в сети, но и каждое отражение каждого отражения во вселенной.

Этот образ напоминает мне о ренормализованных частицах: в каждом электроне заключены виртуальные фотоны, позитроны, нейтрино, муоны\ldots; в каждом фотоне --- виртуальные электроны, протоны, нейтроны, пионы\ldots; в каждом пионе\ldots{}

Затем на ум приходит другая картина: люди, каждый из которых отражен в голове многих других, которые, в свою очередь, отражены в уме кого-то другого, и так далее.

Обе эти картины могут быть представлены коротко и элегантно с помощью Увеличенных Схем Перехода. В случае частиц, у нас будет отдельная схема для каждой категории частиц; в случае людей --- отдельная схема для каждого человека. Каждая из них будет вызывать много других, таким образом создавая виртуальное облако УСП вокруг каждой УСП. Вызов одной из них приведет к вызову других, и этот процесс может идти как угодно долго, пока мы не достигнем поверхности.

Мумон о МУ

Завершим нашу короткую экскурсию в дзен еще одним обращением к Мумону. Вот его комментарий к МУ Джошу:\footnote{Reps стр. 89-90}

Чтобы понять дзен, надо преодолеть барьер патриархов. Просветление всегда приходит после того, как преграждается дорога мысли. Если вы не преодолели барьера патриархов или если дорога вашей мысли не преграждена, то что бы вы не думали и что бы вы не делали, это будет лишь призрачная путаница. Вы можете спросить: «Что такое барьер патриархов?» Это лишь одно слово: «МУ».

Это барьер дзена. Если вы его преодолеете, то встретитесь с Джошу лицом к лицу. Тогда вы сможете работать плечом к плечу со всеми патриархами. Не чудесно ли это?

Если вы хотите преодолеть этот барьер, вы должны до мозга костей проникнуться вопросом: «Что такое МУ?» и размышлять об этом день и ночь. Не думайте, что это --- обычное отрицание и означает ничто. Это не пустота, не противоположность существованию. Если вы действительно хотите преодолеть этот барьер, вы должны чувствовать себя так, словно ваш рот наполнен расплавленным металлом, который вы не можете не проглотить, ни выплюнуть.

Тогда исчезнет ваше предыдущее, меньшее знание. Как плод зреет по осени, так ваша объективность и субъективность естественно сольются в одно. Это похоже на немого, увидевшего сон: он знает о нем, но не может рассказать его.

Когда он достигнет этого состояния, скорлупа его эго разобьется и он сможет трясти небеса и двигать землю. Он станет подобен великому воину с острым мечом. Если Будда встанет на его дороге, он рассечет его своим мечом; если патриарх будет чинить ему препятствия, он убьет его; он будет свободен в своем рождении и смерти. Он сможет войти в любой мир, как в свой собственный дом. В этом коане я скажу вам, как этого добиться:

Сконцентрируйте всю вашу энергию на МУ и не отвлекайтесь ни на миг. Когда вы войдете в МУ, не позволяя себе останавливаться, вы станете словно свеча, своим пламенем освещающая всю вселенную.

От Мумона к головоломке MU

С головоломных высот МУ Джошу спустимся теперь к прозаическому MU Хофстадтера\ldots{} Я знаю, что вы уже пробовали сконцентрировать на нем всю вашу энергию, когда вы читали главу I. Сейчас я отвечу на поставленный в ней вопрос:

Обладает ли \textbf{MU} природой теоремы?

Ответ на этот вопрос~--- не ускользающее \textbf{MU} , но полновесное НЕТ. Чтобы показать это, мы воспользуемся привилегиями дуалистического, логического мышления.

В главе I мы сделали два важных наблюдения:

(1) что сложность головоломки \textbf{MU} зависит от взаимодействия удлиняющих и укорачивающих правил;

(2) что тем не менее есть надежда решить эту задачу, пользуясь достаточно сложным орудием: теорией чисел.

~В главе I мы не стали подробно анализировать головоломку \textbf{MU} с этой точки зрения; теперь пришло время это сделать. Скоро мы увидим, как второе наблюдение (вынесенное за пределы незначительной системы \textbf{MIU} ) стало одним из самых плодотворных открытий математики и как оно изменило взгляд математиков на их предмет.

Для удобства я повторю здесь основные положения системы MIU:

СИМВОЛЫ: \textbf{М} , \textbf{I} , \textbf{U} .

АКСИОМА: \textbf{MI}

ПРАВИЛА:

I. Если \emph{х} \textbf{I} ~--- теорема, то \emph{x} \textbf{IU} ~--- также теорема.

II. Если \textbf{M} \emph{x ---} ~теорема, то \textbf{M} \emph{хх} --- также теорема.

III. В любой теореме \textbf{III} может быть заменено на \textbf{U} .

IV. \textbf{UU} может быть вычеркнуто из любой теоремы.

Мумон показывает нам, как решить головоломку MU

Согласно приведенным выше наблюдениям, MU --- не более как головоломка о натуральных числах, одетая в типографский костюм. Переведя ее в область теории чисел, мы смогли бы найти ее решение. Давайте поразмыслим над словами Мумона, сказавшего: «Если у кого-нибудь из вас --- один глаз, тот заметит ошибку учителя.» Но почему важно иметь именно один глаз?

Если вы попробуйте подсчитать количество \textbf{I} в теоремах, то вскоре заметите, что оно никогда не равняется \textbf{0} . Иными словами, кажется, что сколько бы мы не удлиняли и не сокращали, нам никогда не удается избавиться от всех \textbf{I} . Будем называть количество \textbf{I} в каждой строчке \emph{\textbf{величиной I}} данной строчки. Заметьте, что \emph{величина I} аксиомы \textbf{MI} --- \textbf{1} . Можно доказать не только то, что \emph{величина I} не может равняться \textbf{0} , но и то, что \emph{величина I} не может делиться на \textbf{3} .

Для начала заметьте, что правила I и IV совершенно не затрагивают \emph{величины I} . Так что нам придется иметь дело только с правилами II и III. Правило III уменьшает величину \textbf{I} ровно на \textbf{3} . После приложения этого правила величина \textbf{I} результата могла бы делиться на \textbf{3} --- но только в том случае, если бы величина I \emph{изначальной строчки} тоже делилась на \textbf{3} . Короче, правило III никогда не создает числа, делящегося на \textbf{3} , «из воздуха». Оно может сделать это лишь тогда, когда такое число уже имеется в начале. То же самое верно для правила II, которое удваивает \emph{величину I} . Это происходит потому, что, если \textbf{2} \emph{n} делится на \textbf{3} , то, поскольку \textbf{2} не делится на \textbf{3} , то на \textbf{3} должно делиться \emph{n} (простой факт теории чисел). Ни правило II, ни правило III не могут создать делящегося на \textbf{3} числа из ничего.

Но это же ключ к головоломке \textbf{MU} ! Мы знаем следующее:

(1) \emph{Величина I} начинается с \textbf{1} (\textbf{1} не делится на \textbf{3} );

(2) Два правила вообще не влияют на \emph{величину I} ;

(3) Два оставшихся правила влияют на \emph{величину I} , но таким образом, что они не могут создать делимое на \textbf{3} число, если таковое не дано в начале.

Отсюда следует типично «наследственное» заключение: \emph{величина I} никогда не может стать делимой на \textbf{3} . В частности, \textbf{0} --- пример запрещенной \emph{величины I} . Таким образом, \textbf{MU} \emph{не является теоремой системы} \textbf{MIU} .

Обратите внимание, что даже в форме головоломки о \emph{величине I} , эта проблема все еще усложнена игрой удлиняющих и укорачивающих правил. Нашей целью было прийти к нулю; \emph{величина I} могла увеличиваться (правило II) или уменьшаться (правило III). До анализа ситуации мы могли считать, что применив эти два правила достаточное количество раз, когда-нибудь мы смогли бы получить \textbf{0} . Теперь, благодаря простому доказательству теории чисел, мы знаем, что это невозможно.

Гёделева нумерация для системы MIU

Не все проблемы подобного типа решаются так легко. Однако мы видели, что по крайней мере одна такая головоломка может быть введена в теорию чисел и решена там. Теперь мы увидим, что в теорию чисел возможно включить \emph{все} проблемы о \emph{любой} формальной системе. Это возможно сделать благодаря открытию Гёделем специального типа изоморфизма. Я проиллюстрирую его на примере системы MIU.

Рассмотрим для начала нотацию этой системы. Каждому ее символу будет соответствовать новый символ:

\textbf{M} \textless==\textgreater{} 3

\textbf{I} \textless==\textgreater{} 1

\textbf{U} \textless==\textgreater{} 0

Это соответствие~--- вполне произвольно; я выбрал его потому, что эти символы слегка похожи на те, которым они соответствуют. Каждый номер называется \emph{Гёделев номер} соответствующей буквы. Уверен, что вы можете легко догадаться, как будет выглядеть Гёделев номер строчки из нескольких букв:

\textbf{MU} \textless==\textgreater{} 30

\textbf{MIIU} \textless==\textgreater{} 3110

и т. д.

Это нетрудно. Ясно, что такое соответствие между двумя нотациями является превращением, сохраняющим информацию; это все равно, что одна и та же мелодия, исполненная на двух разных инструментах.

Теперь давайте посмотрим на типичную деривацию системы MIU, записанную одновременно в двух нотациях:

(1)~~~~~~~~~~~ \textbf{MI} -\/- аксиома~~~ -\/- 31

(2)~~~~~~~~~~ \textbf{МII} -\/- правило II~ -\/- 311

(3)~~~~~~~~ \textbf{MIIII} -\/- правило II~ -\/- 31111

(4)~~~~~~~~ \textbf{MUI} -\/- правило III -\/- 301

(5)~~~~~~ \textbf{MUIU} -\/- правило I~~ -\/- 3010

(6)~ \textbf{MUIUUIU} -\/- правило II~ -\/- 3010010

(7)~~~~~ \textbf{MUIIU} -\/-~ правило IV -\/- 30110

Левая колонка получается при помощи наших четырех формальных типографских правил. О правой колонке можно сказать, что она также получилась в результате применения подобных правил. Однако правая колонка --- дуалистична. Сейчас я объясню, чти это означает.

Восприятие вещей одновременно с типографской и с арифметической точки зрения

~О пятой строчке («3010») можно сказать, что она была сделана из четвертой добавлением «0» справа; с другой стороны, мы можем так же легко представить себе, что она была получена в результате \emph{арифметической} операции~--- а именно, умножения на 10. Когда натуральные числа записаны в десятичной системе, умножение на 10 и добавление справа «0» неотличимы друг от друга. Мы можем воспользоваться этим и записать \emph{арифметическое} правило, соответствующее типографскому правилу I:

АРИФМЕТИЧЕСКОЕ ПРАВИЛО Iа: Число, десятичное продолжение которого оканчивается справа на «1», может быть умножено на 10.

Мы можем избавиться от упоминания символов в десятичном продолжении, арифметически описав правую цифру:

АРИФМЕТИЧЕСКОЕ ПРАВИЛО Ib: Если при делении некоего числа на 10 в остатке получается «1», то это число может быть умножено на 10.

Можно было бы воспользоваться и чисто типографским правилом, как, например, следующее:

ТИПОГРАФСКОЕ ПРАВИЛО I: Из любой теоремы, которая кончается на «1», можно получить новую теорему, добавляя «0» справа от этой «1».

Все эти правила дают одинаковый эффект. Именно поэтому правая колонка дуалистична: ее можно рассматривать как серию типографских операций, превращающих одну схему символов в другую, или как серию арифметических операций, превращающих одну величину в другую. Существуют веские причины к тому, чтобы больше интересоваться арифметической версией. Переход из одной чисто типографской системы в другую, изоморфную типографскую систему --- это не слишком занимательно; с другой стороны, переход из типографской области в изоморфную ей часть теории чисел предоставляет интересные, ранее неиспользованные возможности. Словно кто-то всю жизнь имел дело только с нотной записью, и вдруг ему показали соответствие между нотами и звуками. Какой удивительное богатство открылось перед ним! Или, возвращаясь к Ахиллу и Черепахе, играющим с цепочками, представьте себе человека, который хорошо знаком с фигурами из цепочек, и которому вдруг открылось соответствие между цепочками и рассказами. Какое откровение! Открытие Геделевой нумерации сравнивают с открытием Декарта, установившего изоморфизм между линиями на плоскости и уравнениями с двумя переменными. Это кажется невероятно просто --- но это открывает дорогу в огромный новый мир.

Однако прежде чем придти к заключению, давайте рассмотрим подробнее этот высший уровень изоморфизма. Это очень хорошее упражнение. Наша цель --- придумать арифметические правила, действующие точно так же, как типографские правила системы MIU.

Ниже приведено решение. В этих правилах \emph{m} и \emph{k} - произвольные натуральные числа, и \emph{n} --- любое натуральное число, меньшее 10\emph{m} .

ПРАВИЛО 1: Если мы получили 10\emph{m} + 1, то мы можем получить 10 * (10\emph{m} + 1).

\emph{Пример} : Переход от строчки 4 к строчке 5. Здесь \emph{m} = 30

ПРАВИЛО 2: Если мы получили 3 * 10\textsuperscript{\emph{m}} + \emph{n} , то мы можем получить 10\textsuperscript{\emph{m}} * (3 * 10\textsuperscript{\emph{m}} + \emph{n} ) + \emph{n}

\emph{Пример} : Переход от строчки 1 к строчке 2, где \emph{n} и \emph{m} равняются 2.

ПРАВИЛО 3: Если мы получили \emph{k} *10\textsuperscript{\emph{m} +\textbf{3}} + 111 * 10\textsuperscript{\emph{m}} + \emph{n} , то мы можем получить \emph{k} * 10 \textsuperscript{\emph{m} +\textbf{1}} + \emph{n} .

\emph{Пример} : Переход от строчки 3 к строчке 4. Здесь \emph{m} и \emph{n} равняются 1 и \emph{k} равняется 3.

ПРАВИЛО 4: Если мы получили \emph{k} * 10\textsuperscript{\emph{m} +\textbf{2}} + \emph{n} , то мы можем получить \emph{k} * 10 \textsuperscript{\emph{m}} + \emph{n.}

\emph{Пример} : Переход от строчки 6 к строчке 7. Здесь \emph{m} =2, \emph{n} =10 и \emph{k} =301.

Не следует забывать нашу аксиому! Без нее мы как без рук, так что давайте запишем постулат.

Мы можем получить 31.

Теперь правую колонку можно рассматривать как арифметический процесс в новой арифметической системе, которую мы назовем \emph{системой 310} :

(1)~~~~~~~~ 31~~~~~~ аксиома

(2)~~~~~~~ 311~~~~~~ правило 2 (\emph{m} = 1, \emph{n} = 1)

(3)~~~~~ 31111~~~~~~ правило 2 (\emph{m} = 2, \emph{n} = 11)

(4)~~~~~~~ 301~~~~~~ правило 3 (\emph{m} = 1, \emph{n} = 1, \emph{k} = 3)

(5)~~~~~~ 3010~~~~~~ правило 1 (\emph{m} = 30)

(6)~~~ 3010010~~~~~~ правило 2 (\emph{m} = 3, \emph{n} = 10)

(7)~~~~~ 30110~~~~~~ правило 4 (\emph{m} = 2, \emph{n} = 10, \emph{k} = 301)

Обратите внимание на то, что удлиняющие и укорачивающие правила снова с нами и в системе 301; они просто переведены в область чисел таким образом, что Гёделевы номера в системе возрастают и уменьшаются. Если вы посмотрите внимательно на то, что происходит, то увидите, что правила основаны на простой идее, а именно: сдвиг цифр направо и налево в десятичной записи чисел имеет отношение к умножению на степени числа 10. Это простое наблюдение обобщено в следующем центральном предложении:

ЦЕНТРАЛЬНОЕ ПРЕДЛОЖЕНИЕ: Если у нас имеется некоторое правило, говорящее нам, как определенные цифры могут быть передвинуты, заменены, добавлены или опущены в в десятичной записи любого числа, то это правило также может быть представлено соответствующим арифметическим правилом при помощи арифметических операций со степенями числа 10, а также сложения, вычитания и так далее.

Или короче:

Типографские правила манипуляции с \emph{символами чисел} эквивалентны арифметическим правилам операций с \emph{числами} .

Это простое наблюдение находится в самом сердце Гёделева метода; оно будет иметь совершенно потрясающий эффект. Оно говорит нам, что если у нас есть Гёделева нумерация для любой формальной системы, мы можем тут же получить набор арифметических правил, дополняющих Гёделев изоморфизм. В результате оказывается возможным перевести изучение любой формальной системы~--- на самом деле, \emph{всех} формальных систем~--- в область теории чисел.

Числа, выводимые в MIU

Подобно тому, как набор типографских правил порождает набор теорем, в результате повторного применения арифметических правил получается соответствующее множество натуральных чисел. Эти \emph{выводимые числа} играют ту же роль в теории чисел, как теоремы~--- в любой формальной системе. Разумеется, набор выводимых чисел изменяется в зависимости от принятых правил. «Выводимые числа» выводимы только \emph{относительно данной системы} арифметических правил. Например, такие числа как \textbf{31} , \textbf{3010010} , \textbf{31111} и так далее могут быть названы выводимыми в системе \textbf{MIU} . Это неуклюжее название можно сократить до чисел \textbf{MIU} ; оно символизирует тот факт, что эти числа~--- результат перевода системы \textbf{MIU} в теорию чисел при помощи Гёделевой нумерации. Если бы мы захотели приложить Гёделеву нумерацию к системе \textbf{pr} и затем «арифметизировать» ее правила, мы могли бы называть полученные числа «числами \textbf{pr} »~--- и так далее.

Заметьте, что выводимые числа (в любой данной системе) определяются рекурсивным методом: нам даны числа, о которых мы знаем, что они выводимы, и набор правил, объясняющих, как получить другие выводимые числа. Таким образом, класс выводимых чисел постоянно расширяется, подобно списку чисел Фибоначчи или чисел Q. Множество выводимых чисел любой системы~--- это \emph{рекурсивно счетное множество} . А как насчет его дополнения~--- множества невыводимых чисел? Имеют ли они какую-либо общую арифметическую черту?

Подобные вопросы возникают тогда, когда изучение формальных систем переносится в область теории множеств. О каждой арифметизированной системе можно спросить: «Возможно охарактеризовать выводимые числа каким-либо простым способом?» «Возможно ли охарактеризовать невыводимые числа рекурсивно счетным способом?» Эти вопросы теории чисел весьма непросты, и, в зависимости от арифметизированной системы, могут оказаться для нас слишком трудными. Если и есть надежда найти на них ответ, то она лежит в методических логических рассуждениях, подобных тем, что обычно используются для изучения натуральных чисел. Суть этих рассуждений была изложена в предыдущей главе. По всей видимости, в ТТЧ нам удалось полностью представить все математические рассуждения в одной единственной компактной системе.

ТТЧ помогает ответить на вопросы о выводимых числах

Значит ли это, что одна-единственная формальная система~--- ТТЧ~--- предоставляет нам способ ответить на любой вопрос о любой формальной системе? Возможно. Возьмем например, такой вопрос:

Является ли \textbf{MU} теоремой системы MIU?

Найти ответ на этот вопрос означало бы определить, является ли \textbf{30} числом MIU. Поскольку это утверждение --- высказывание теории чисел, мы должны надеяться, что при достаточном усилии нам удастся перевести высказывание «30 --- число MIU» в нотацию ТТЧ, точно так же, как нам удалось перевести на язык ТТЧ другие высказывания теории чисел. Должен сразу предупредить читателя, что, хотя подобный перевод существует, он невероятно сложен. Если вы помните, в главе VIII я говорил, что даже такой простой арифметический предикат как «b --- степень 10» весьма непросто перевести в ТТЧ; предикат же «30 --- число MIU» перевести еще гораздо сложнее! Все же этот, перевод можно найти, и число SSSSSSSSSSSSSSSSSSSSSSSSSSSSSS0 может быть подставлено в него вместо любого \emph{b} . Результатом явилась бы МОНструозная строчка ТТЧ, говорящая о головоломке MU. Сдается мне, что подходящим названием для этой строчки было бы МУМОН. С помощью МУМОНа и подобных строчек ТТЧ теперь способна говорить в закодированной форме о системе MIU.

Дуалистическая природа МУМОНа

Чтобы извлечь какую-либо пользу из этой странной трансформации нашего первоначального вопроса, нам необходимо ответить еще на один вопрос:

Является ли МУМОН теоремой ТТЧ?

До сих пор мы всего лишь заменили короткую строчку (MU) на другую (монструозный МУМОН) и простую формальную систему (\textbf{MIU} ) --- на более сложную (ТТЧ). Хотя мы перефразировали, вопрос, маловероятно, что это приблизило нас к ответу. Действительно, в ТТЧ есть такая куча укорачивающих и удлиняющих правил, что перифраз вопроса, скорее всего, окажется гораздо труднее оригинала. Некоторые читатели, пожалуй, могли бы сказать, что анализировать \textbf{MU} пои помощи МУМОНа --- значит нарочно смотреть на вещи по-дурацки. Однако МУМОНа можно рассматривать более, чем на одном уровне.

Интересно то, что в МУМОНе есть два различных пассивных значения. Во-первых, приведенное выше:

30 --- число \textbf{MIU} .

Во-вторых, мы знаем, что это высказывание изоморфно следующему:

\textbf{MU} --- теорема системы \textbf{MIU} .

Следовательно, мы имеем право утверждать, что последнее высказывание --- второе пассивное значение МУМОНа. Это может показаться странным, поскольку МУМОН состоит всего лишь из плюсов, скобок и тому подобных символов ТТЧ. Как же он может выражать что-либо, кроме арифметических высказываний?

На самом деле, это возможно. Так же, как одна единственная музыкальная строчка может заключать в себе гармонию и мелодию, как слово~BACH может быть прочитано как имя и как мелодия, как одно и то же словосочетание может быть аккуратным описанием картины Эшера, структуры ДНК, произведения Баха или Диалога под тем же названием, МУМОН может быть понят, по крайней мере, двояко. Это происходит благодаря следующим фактам:

Факт 1. Высказывания типа «MU~--- теорема» могут быть закодированы в теории чисел при помощи Гёделевой нумерации.

Факт 2. Высказывания теории чисел могут быть переведены в ТТЧ.

Можно сказать, что (согласно Факту 1) МУМОН~--- это закодированное сообщение, в котором (согласно Факту 2) символы кода ---~не более, чем символы ТТЧ.

Коды и неявное значение

Вы можете возразить, что закодированное сообщение, в отличие от незакодированного, само по себе ничего не выражает --- чтобы его понять, необходимо знать код. Однако на самом деле незакодированных сообщений не существует Просто одни сообщения написаны на более знакомых кодах, а другие --- на менее знакомых. Чтобы раскрыть значение сообщения, его необходимо «извлечь» из кода при помощи некоего механизма, или изоморфизма Иногда открыть метод дешифровки бывает трудно, но, как только этот метод раскрыт, сообщение становится прозрачным, как стекло. Когда код становится достаточно знакомым, он перестает выглядеть как таковой, и мы забываем о существовании декодирующего .механизма. Сообщение сливается со значением.

Здесь мы сталкиваемся со случаем такого полного отождествления сообщения со значением, что мы с трудом можем вообразить, что данные символы могут иметь какое-то иное значение. Мы настолько привыкли считать, что символы ТТЧ придают строчкам этой системы теоретико-числовое значение (и только теоретико-числовое), что нам бывает трудно представить, что некоторые строчки ТТЧ могут быть интерпретированы, как высказывания о системе MIU. Однако Гёделев изоморфизм заставляет нас признать этот второй уровень значения у некоторых строчек ТТЧ.

МУМОН, декодированный в более знакомом нам виде, сообщает, что

30 --- число \textbf{МIU} .

Это высказывание теории чисел, полученное при интерпретации каждого знака обычным путем.

Открыв Гёделеву нумерацию и построенный на ее основе изоморфизм, мы в каком-то смысле расшифровали код, на котором высказывания о системе \textbf{MIU} записаны при помощи строчек ТТЧ. Гёделев изоморфизм --- это новый обнаружитель информации, в том же смысле, как дешифровки старинных текстов были обнаружителями заложенной в этих текстах информации.

Декодированное этим новым и менее знакомым нам способом, МУМОН сообщает, что

\textbf{MU} --- теорема системы \textbf{MIU} .

Мораль этой истории мы уже слышали: любой узнанный нами изоморфизм автоматически порождает значение; следовательно, у МУМОНа есть по крайней мере два пассивных значения, а может быть, и больше!

Бумеранг --- Гёделева нумерация ТТЧ

Разумеется, это еще не конец; мы только начали открывать возможности Гёделева изоморфизма. Естественным трюком было бы использовать возможность ТТЧ отображать другие формальные системы на себя саму, на манер того, как Черепаха повернула патефоны Краба против их самих, или как Бокал Г атаковал сам себя, разбившись. Чтобы это сделать, мы должны приложить Гёделеву нумерацию к самой ТТЧ, так же, как мы это сделали с системой \textbf{MIU} , и затем «арифметизировать» правила вывода. Это совсем нетрудно. Например, мы можем установить следующее соответствие:

~\emph{Символ~~~ Кодон~~ Мнемоническое обоснование}

0 .......~~ 666~~~~~~~Число Зверя для Таинственного Нуля

S .......~~ 123~~~~~~~последовательность: 1, 2, З\ldots{}

= .......~~ 111~~~~~~~зрительное сходство, в повернутом виде + ....... 112~~1+1=2

* .......~~~236~~~~~~~~2*3=6

( .......~~~362~~~~~~~~кончается на 2 \textbackslash{}

) .......~~~323~~~~~~~~кончается на 3~ \textbar~~ эти

\textless{} .......~~~212~~~~~~~кончается на 2~ \textbar~~ три пары

\textgreater{} .......~~ 213~~~~~~~кончается на 3~ \textbar~~ формируют

{[} .......~~ 312~~~~~~~ кончается на 2~ \textbar~~ схему

{]} .......~~ 313~~~~~~~ кончается на 3 /

а .......~~ 262~~~~~~ противоположно A (626)

' .......~~~ 163~~~~~~~163-простое число

\&\#923;~......~~~ 161~~~~~~~«\&\#923;»-«график» последовательности 1-6-1"

V~......~~~ 616~~~~~~~«V»-«график» последовательности 6-1-6

э ......~~~ 633~~~~~~~в некотором роде, из 6 следуют 3 и 3

\textasciitilde{} .......~~ 223~~~~~~~2+2 \emph{не} 3

E .......~~ 333~~~~~~~«E» выглядит как «3»

A .......~~ 626~~~~~~~противоположно «A»- также «график» 6-2-6

: .......~~ 636~~~~~~~ две точки, две шестерки

пунк ....~ 611~~~~~~ особенное число (именно потому,~что в нем нет ничего особенного)

Каждый символ ТТЧ соотнесен с трехзначным числом, составленным из цифр 1, 2, 3 и 6 таким образом, чтобы его было легче запомнить. Каждое такое трехзначное число я буду называть \emph{Геделев кодоном} , или, для краткости, \emph{кодоном} . Заметьте, что для b. с, d или е кодонов не дано, поскольку мы используем здесь строгую версию ТТЧ. Для этого есть причина, которую вы узнаете в главе XVI. Последняя строчка, «пунктуация», будет объяснена в главе XIV.

Теперь мы можем представить любую строчку или правило ТТЧ в новом наряде. Вот, например, Аксиома 1 в двух нотациях, новая над старой:

~626, 262, 636, 223, 123, 262, 111, 666

. A~~~~~ a~~~~ ~:~~~~~ \textasciitilde~~~~~S~~~~~a~~~~ =~~~~~ 0

Обычная условность~--- использование пунктуации после каждых трех цифр~--- очень кстати совпала с нашими кодонами, облегчая их чтение.

Вот Правило Отделения в новой записи:

ПРАВИЛО: Если~\emph{x} и 212\emph{x} 633y213 являются теоремами, то \emph{у} - также теорема.

Наконец, вот целая деривация, взятая из предыдущей главы; она дана в строгой версии ТТЧ и записана в новой нотации:

626,262,636,626.262,163,636,362,262,112,123,262,163,323,111,123,362,262,112,262,163,323 аксиома 3

. A~~~ a~~~~ :~ ~~ A~~~ ~a~~~~ '~~ ~ :~~~~ (~~~ ~a~~~ +~~~~ S~~~ a~~~~ '~~~~ )~~~=~~~ S~~~ (~~~~ a~~~~ +~~~ a~~~~~ '~~~ )

626,262,163,636,362,123,666,112,123,262,163,323,111,123,362,123,666,112,262,163,323 спецификация

. A~~~ a~~~~ '~~~~ :~~~~ (~~~~ S~~ ~0~~~~ +~~~ S~~~ a~~~~ '~~~~~ )~~~~=~~~ S~~~~ (~~~~ S~~~ 0~~~~ +~~~ a~~~~ '~~~~ )

362,123,666,112,123,666,323,111,123,362,123,666,112,666,323~спецификация

. (~~~~ S~~~ 0~~~~ +~~~ S~~~ 0~~~~~ )~~~ =~~~ S~~~~ ( ~~~ S~~~ 0~~~ +~~~~ 0~~~ )

626,262,636,362,262,112,666,323.111.262~аксиома 2

.~ A~~~ а~~ ~:~~~~ (~~~~ а~~~ +~~~ 0~~~ ~)~~~~ =~~~ а

362,123,666,112,666,323,111,123,666~~спецификация

. (~~~~ S~~~~ 0~~~ +~~ ~0~~~~ )~~~~ =~~~ S~~~ 0

123,362,123,666.112,666,323,111,123,123,666~~добавить «123»

. ~S~~ (~~~~~ S~~~ 0~~~~ +~~ ~0~~~ )~~~~ =~~~~ S~~~ S~~~ 0

362,123,666,112,123,666,323,111,123,123,666~~транзитивность

.~ (~~~ S~~~ 0 ~~~ +~~~ S~~~~ 0~~~~ )~~~~= ~~~ S~~ ~S~~~ 0

Обратите внимание, что я изменил название правила «добавить S» на «добавить 123», поскольку данное правило узаконивает именно эту типографскую операцию.

Новая нотация кажется весьма странной. Вы теряете всякое ощущение значения; однако, если потренироваться, вы сможете читать строчки в этой нотации так же легко, как вы читали строчки ТТЧ. Вы сможете отличать правильно сформированные формулы от неправильных с первого взгляда. Естественно, поскольку это настолько наглядно, вы будете думать об этом, как о типографской операции~--- но в то же время выбор правильно сформированных \emph{формул} в этой нотации эквивалентен выбору определенного класса \emph{чисел} , у которых есть также арифметическое определение.

А как же насчет «арифметизации» всех правил вывода? Они все еще остаются типографскими. Но погодите минутку! Согласно Центральному Предложению, типографское правило~--- все равно, что арифметическое правило. Ввод и перестановка цифр в числах десятичной записи~--- это \emph{арифметическая} операция, которая может быть осуществлена типографским путем. Подобно тому, как добавление «О» справа от числа эквивалентно умножению этого числа на 10, каждое правило представляет собой компактное описание длинного и сложного арифметического действия. Таким образом, нам не придется искать эквивалентных арифметических правил, поскольку все правила \emph{уже} арифметические!

Числа ТТЧ: рекурсивно счетное множество чисел

С такой точки зрения, приведенная выше деривация теоремы «362,123,666,112,123,666,323,111,123,123,666» представляет собой последовательность весьма сложных теоретико-численных трансформаций, каждая из которых действует на одно или более данных чисел. Результатом этих трансформаций является, как и ранее, \emph{выводимое число} , или, более точно, \emph{число ТТЧ} . Некоторые арифметические правила берут старое число ТТЧ и \emph{увеличивают} его определенным образом, чтобы получить новое число ТТЧ, некоторые \emph{уменьшают} старое число ТТЧ; другие правила берут два числа ТТЧ, воздействуют на них определенным образом и комбинируют результаты, получая новое число ТТЧ --- и так далее, и тому подобное. Вместо того, чтобы начинать с одного известного числа ТТЧ, мы начинаем с \emph{пяти} --- одно для каждой аксиомы (в строгой нотации). На самом деле, арифметизированная ТТЧ очень похожа на арифметизированную систему \textbf{MIU} --- только в ней больше аксиом и правил, и запись точных арифметических эквивалентов была бы титаническим и совершенно «непросветляющим» трудом. Если вы внимательно следили за тем, как это было сделано для системы \textbf{MIU} , у вас должно быть сомнений в том, что здесь это делается совершенно аналогично.

Эта «гёделизация» ТТЧ порождает новый теоретико-числовой предикат:

\emph{а} ~--- число ТТЧ.

Например, мы знаем из предыдущей деривации, что \textbf{362,123,666,112,123,666,323,111,123,123,666} \emph{является} числом ТТЧ, в то время как число \textbf{123,666,111,666} числом ТТЧ предположительно \emph{не} является.

Оказывается, что этот новый теоретико-численный предикат можно \emph{выразить} некоей строчкой ТТЧ с одной свободной переменной~--- скажем, \emph{а} . Мы могли бы поставить тильду впереди, и эта строчка выражала бы дополняющее понятие:

\emph{а} ~--- не число ТТЧ.

Теперь давайте заменим все \emph{а} в этой второй строчке на символ числа ТТЧ для \textbf{123,666,111,666} ~--- символ, содержащий ровно 123,666,111,666 \textbf{S} и слишком длинный, чтобы его здесь записывать. У нас получится строчка ТТЧ, которая, подобно МУМОНу, может быть интерпретирована на двух уровнях. Во-первых, она будет означать

123,666,111,666~--- не число ТТЧ.

Но, благодаря изоморфизму, связывающему числа, ТТЧ с теоремами ТТЧ, у этой строчки есть и второе значение:

S0=0 не теорема ТТЧ.

ТТЧ пытается проглотить саму себя

Это неожиданно двусмысленное толкование показывает, что ТТЧ содержит строчки, говорящие о других строчках ТТЧ. Иными словами, метаязык, на котором мы можем говорить о ТТЧ, берет начало, хотя бы частично, \emph{внутри} самой ТТЧ. И это не случайность; дело в том, что архитектура любой формальной системы может быть отражена в Ч (теории чисел). Это такая же неизбежная черта ТТЧ, как колебания, вызываемые в патефоне, проигрываемой на нем пластинкой. Кажется, что колебания должны вызываться внешними причинами, --- например, прыжками детей или ударами мяча; но побочный --- и неизбежный --- эффект произведения звуков заключается в том, что они заставляют колебаться сам механизм, их порождающий. Это не случайность, а закономерный и неизбежный побочный эффект. Он свойствен самой природе патефонов. И так же самой природе любой формализации теории чисел свойственно то, что ее метаязык содержится в ней самой.

Мы можем почтить это наблюдение, назвав его Центральной Догмой Математической Логики и изобразив его на двухступенчатой диаграмме.

ТТЧ ==\textgreater{} Ч ==\textgreater{} мета-ТТЧ

Иными словами, у строчки ТТЧ есть интерпретация в Ч, а у высказывания Ч может быть второе значение~--- оно может быть понято как высказывание о ТТЧ.

G: строчка, говорящая о себе самой на коде

Эти интересные факты --- только половина истории. Другая половина --- интенсификация автореференции. Мы сейчас находимся в положении Черепахи, когда она обнаружила, что можно создать пластинку, разбивающую проигрывающий ее патефон. Вопрос только в том, какую именно запись надо ставить на данный патефон. Выяснить это непросто.

Для этого нужно найти строчку ТТЧ --- мы будем называть ее «G» --- которая говорит о себе самой, в том смысле, что --- одно из ее пассивных значений --- это высказывание о G.

В частности, этим пассивным значением окажется

«G- не теорема ТТЧ»

Я должен добавить, что у G есть и другое пассивное значение, являющееся \emph{высказыванием теории чисел} ; подобно тому, как МУМОН мог быть интерпретирован двояко. Важно то, что каждое пассивное значение --- действительно и полезно, и никоим образом не бросает тень сомнения на второе значение. (Тот факт, что играющий патефон может вызывать колебания в самом себе и в пластинке, не отрицает того, что эти колебания --- музыкальные звуки!)

В неполноте ТТЧ виновато существование G

Об изобретательном методе создания G и о некоторых важных понятиях ТТЧ мы поговорим в главах XIII и XIV; пока же давайте заглянем вперед и постараемся увидеть, какие последствия будет иметь нахождение автореферентной часта ТТЧ. Кто знает --- может быть, это будет подобно взрыву! В некотором роде, это так и есть. Как вы думаете,

Является ли G теоремой ТТЧ, или нет?

Постарайтесь сформировать собственное мнение по этому поводу, не опираясь на мнение G о себе самой. В конце концов, G может понимать себя не лучше, чем понимает себя какой-нибудь мастер дзен-буддизма. Подобно МУМОНу, G может быть ложным утверждением. Подобно \textbf{MU} , G может быть не-теоремой. Мы не обязаны верить в любую возможную строчку ТТЧ, а только в ее теоремы. Давайте используем наше умение рассуждать логически и постараемся разъяснить этот вопрос.

Предположим, как обычно, что ТТЧ включает правильные методы рассуждения и что, следовательно, ложные утверждения не могут являться ее теоремами. Иными словами, любая теорема ТТЧ выражает истину. Таким образом, если бы строчка G была теоремой, она выражала бы истину, а именно: «G --- не теорема.» Вся сила ее автореферентности видна здесь в действии. Будучи теоремой, G должна быть ложна. Опираясь на наше предположение, что ТТЧ не имеет ложных теорем, мы должны теперь заключить, что G~--- \emph{не теорема} . Это не так страшно, но оставляет нас с меньшей проблемой. Зная, что G --- не теорема, мы должны согласиться с тем, что она выражает истину\ldots{} В этой ситуации ТТЧ не оправдывает наших ожиданий --- мы нашли строчку, выражающую истинное высказывание, которая в то же время не является теоремой! И, как бы мы не удивлялись, мы не должны упускать из виду тот факт, что у G есть также и арифметическая интерпретация. Это позволяет нам подвести итог нашим наблюдениям:

\emph{Найдена такая строчка ТТЧ, которая является недвусмысленным высказыванием о некоторых арифметических свойствах натуральных чисел; более того, рассуждая вне системы, мы можем определить не только то, что это высказывание истинно, но и то, что эта строчка не является теоремой ТТЧ. Таким образом, если мы спросим у ТТЧ, истинно ли это высказывание, она не сможет ответить ни да, ни нет.}

Аналогична ли G Черепашья цепочка в «Приношении MU»? Не совсем. Аналогичней с Черепашьей цепочкой будет \textasciitilde G. Почему это так? Давайте подумаем! Что говорит \textasciitilde G? Она должна утверждать обратное строчке G. G говорит: «G --- не теорема ТТЧ»; следовательно, \textasciitilde G должно читаться «G --- теорема ТТЧ». Мы можем перефразировать обе эти строчки следующим образом:

G: «Я не теорема (ТТЧ)»

\textasciitilde G: «Мое отрицание~--- теорема (ТТЧ)»

Именно \textasciitilde G параллельна Черепашьей цепочке, так как она говорит не о себе самой, но о той цепочке, что Черепаха дала Ахиллу сначала~--- цепочке, на которой была завязана дополнительная неточка (или на одну неточку меньше, чем надо~--- это зависит от точки зрения).

Последнее слово~--- за Мумоном

В своем коротком стихотворении о MU Джошу, Мумон проник в Мистерию Ультранеразрешимости глубже всех:

\emph{Есть ли у собаки природа Будды?}

\emph{Это самый серьезный вопрос из всех.}

\emph{Если вы ответите да или нет,}

\emph{Вы утратите собственную природу Будды.}



\part{Часть~II}

\emph{Триплеты «GEB» и «EGB»}

% \subsubsection{Прелюдия и\ldots{}}
% \subsubsection{Прелюдия и\ldots{}}

\emph{Рис. 54. М. К. Эшер. «Лист Мёбиуса II» (гравюра на дереве, 1963).}

\emph{Ахилл и Черепаха пришли в гости к Крабу, чтобы познакомиться с его другом Муравьедом. После того, как новые знакомые представлены друг другу, вся четверка садится за чай.}

\emph{Черепаха} : Мы вам кое-что принесли, мистер Краб.

\emph{Краб} : Очень любезно с вашей стороны, но зачем же было утруждаться?

\emph{Черепаха} : О это так, мелочь --- в знак нашего уважения. Ахилл, отдайте, пожалуйста, подарок м-ру К.

\emph{Ахилл} : С удовольствием. С наилучшими пожеланиями, м-р К. Надеюсь, что вам понравится.

\emph{(Ахилл протягивает Крабу элегантно завернутый пакет, квадратный и плоский Краб начинает его разворачивать).}

\emph{Муравьед} : Интересно, что это такое?

\emph{Краб} : Сейчас узнаем (Кончает разворачивать и вытаскивает подарок). Две пластинки! Прекрасно! Но погодите-ка здесь нет этикетки. Неужели это снова ваши «особые» записи, г-жа Ч?

\emph{Черепаха} : Если вы имеете в виду разбивальную музыку, на этот раз нет. Но эти записи действительно уникальны, так как они сделаны по персональному заказу. На самом деле, их еще никто никогда не слышал --- кроме, конечно Баха, когда тот их играл.

\emph{Краб} : Когда Бах их играл? Что вы имеете в виду?

\emph{Ахилл} : Вы будете вне себя от счастья, м-р Краб, когда г-жа Ч объяснит вам, что это за пластинки.

\emph{Черепаха} : Почему бы вам самому этого не рассказать, Ахилл? Не стеснятесь, говорите!

\emph{Ахилл} : Можно? Вот здорово! Но я лучше загляну сначала в свои записи (Вытаскивает бумажку и откашливается ) Кхе-кхе. Желаете послушать рассказ о замечательных новых результатах в математике --- результатах, которым ваши пластинки обязаны своим существованием?

\emph{Краб} : Мои пластинки восходят к каким-то математическим выкладкам? Как интересно! Что ж, теперь, когда вы задели мое любопытство, я просто обязан об этом узнать.

\emph{Ахилл} : Отлично! (Делает паузу, чтобы отхлебнуть чай, затем продолжает) Кто-нибудь из вас слышал о печально известной «Последней Теореме» Ферма?

\emph{Муравьед} : Не уверен. Звучит знакомо, но не могу припомнить.

\emph{Рис. 55. Пьер Де Ферма}

\emph{Ахилл} : Идея очень проста. Пьер де Ферма, адвокат по профессии и математик по призванию, однажды, читая классический текст Диофанта «Арифметика», наткнулся на следующее уравнение:

a\&\#178; + b\&\#178; = c\&\#178;

Он тут же понял, что это уравнение имеет бесконечно много решений для \emph{а} , \emph{b} , и \emph{с,} и написал на полях свою знаменитую поправку:

Уравнение:

а \textsuperscript{\emph{n}} + b \textsuperscript{\emph{n}} = с \textsuperscript{\emph{n}}

имеет решение в положительных целых числах \emph{а} , \emph{b} , \emph{с} , и~\emph{n} только при~\emph{n} = 2 (и в таком случае имеется бесконечное множество \emph{a} , \emph{b} , и \emph{c} , удовлетворяющих этому уравнению), но для \emph{n} \textgreater2 решений не существует. Я нашел замечательное доказательство этого, которое, к несчастью, не помещается на полях.

С того дня и в течение почти трехсот лет математики безуспешно пытаются сделать одно из двух: либо доказать утверждение Ферма и таким образом очистить его репутацию, в последнее время слегка подпорченную скептиками, не верящими, что он действительно нашел доказательство --- либо опровергнуть его утверждение, найдя контрпример: множество четырех целых чисел \emph{а} , \emph{b} , \emph{с} , и \emph{n} , где~\emph{n} \textgreater{} 2, которое удовлетворяло бы этому уравнению. До недавнего времени все попытки в любом из этих двух направлений проваливались. Точнее, теорема доказана лишь для определенных значений~\emph{n} --- в частности, для всех~\emph{n} до 125 000.

\emph{Ахилл} : Не лучше ли тогда называть это Гипотезой вместо Теоремы, поскольку настоящее доказательство еще не найдено?

\emph{Ахилл} : Строго говоря, вы правы, но по традиции это зовется именно так.

\emph{Краб} : Удалось ли кому-нибудь в конце концов разрешить этот знаменитый вопрос?

\emph{Ахилл} : Представьте себе, да: это сделала г-жа Черепаха, как всегда, в момент гениального озарения. Она не только нашла ДОКАЗАТЕЛЬСТВО Последней Теоремы Ферма (оправдав, таким образом, ее название и очистив репутацию Ферма), но и КОНТРПРИМЕР, показав, что интуиция скептиков их не подвела!

\emph{Краб} : Вот это да! Поистине революционное открытие.

\emph{Муравьед} : Прошу вас, не тяните: что это за магические числа, удовлетворяющие уравнению Ферма? Мне особенно любопытно узнать значение \emph{n} .

\emph{Ахилл} : Ах, какой ужас! Какой стыд! Верите ли, я оставил все выкладки дома на громаднейшем листе бумаги. К несчастью, он был слишком велик, чтобы принести его с собой. Хотел бы я, чтобы он был сейчас здесь и чтобы можно было вам все показать. Но кое-что я все же помню: величина~\emph{n} --- единственное положительное число, которое нигде не встречается в непрерывной дроби числа \&\#960;.

\emph{Краб} : Какая жалость, что у вас нет с собой ваших записей. Так или иначе, у нас нет оснований сомневаться, что все, что вы нам сказали --- чистая правда.

\emph{Муравьед} : Да и кому, в конце концов, нужно видеть~\emph{n} в десятичной записи? Ахилл же объяснил нам, как найти это число. Что ж, г-жа Черепаха, примите мои сердечные поздравления по поводу вашего эпохального открытия!

\emph{Черепаха} : Благодарю вас. Однако практическая польза, которую немедленно принес мой результат, кажется мне еще важнее теоретического открытия.

\emph{Краб} : Смерть как хочется услышать об этом --- ведь я всегда считал, что теория чисел --- Царица Чистой Математики, единственная ветвь математики, не имеющая НИКАКОГО практического приложения.

\emph{Черепаха} : Вы не единственный, кто так думает; однако на деле почти невозможно предсказать, когда и каким образом какая-либо ветвь чистой математики --- или даже какая-либо индивидуальная Теорема --- повлияет на другие науки. Это происходит совершенно неожиданно, и данный случай --- хороший тому пример.

\emph{Ахилл} : Обоюдоострый результат г-жи Черепахи прорубил дверь в область акусто-поиска.

\emph{Муравьед} : Что такое акусто-поиск?

\emph{Ахилл} : Название говорит само за себя: это поиск и извлечение акустической информации из сложных источников. Например, типичная задача акусто-поиска~--- восстановить звук, произведенный упавшим в воду камнем, по форме расходящихся по воде кругов.

\emph{Краб} : Но это невозможно!

\emph{Ахилл} : Почему же? Это весьма похоже на то, что делает наш мозг, когда он восстанавливает звук, произведенный голосовыми связками другого человека, по колебаниям, переданным барабанной перепонкой далее по лабиринту ушной раковины.

\emph{Краб} : Ясно. Но я все еще не вижу связи этого ни с теорией чисел, ни с моими новыми пластинками.

\emph{Ахилл} : Видите ли, в математике акусто-поиска часто возникают вопросы, связанные с числом решений неких Диофантовых уравнений. А г-жа Ч годами занималась тем, что пыталась восстановить звуки игры Баха на клавесине (что происходило более двухсот лет тому назад), основываясь на расчетах движения всех молекул в атмосфере в настоящее время.

\emph{Муравьед} : Но это же совершенно невозможно! Эти звуки утрачены навсегда, утеряны невозратимо!

\emph{Ахилл} : Так думают непосвященные --- но г-жа Ч посвятила много лет этой проблеме и пришла к выводу, что все зависит от количества решений уравнения:

а \textsuperscript{\emph{n}} + b \textsuperscript{\emph{n}} = с \textsuperscript{\emph{n}}

в положительных числах, при~\emph{n} \textgreater{} 2.

\emph{Черепаха} : Я могла бы объяснить, при чем здесь это уравнение, но не хочу наскучить присутствующим.

\emph{Ахилл} : Оказалось, что теория акусто-поиска предсказывает, что звуки Баховского клавесина могут быть восстановлены по движению всех молекул атмосферы при одном из двух условий ЛИБО у этого уравнения есть хотя бы одно решение.

\emph{Краб} : Удивительно!

\emph{Муравьед} : Фантастика да и только!

\emph{Черепаха} : Кто бы мог подумать!

\emph{Ахилл} : Я хотел сказать, «ЛИБО такое решение существует, ЛИБО существует доказательство, что уравнение НЕ имеет решений!» Итак, г-жа Ч начала кропотливую работу с обоих концов проблемы одновременно Оказалось, что нахождение контрпримера было ключом к нахождению доказательства, так что одно прямо вело к другому.

\emph{Краб} : Как же это возможно?

\emph{Черепаха} : Видите ли, мне удалось показать, что структуру любого доказательства Последней Теоремы Ферма --- если таковое существует --- возможно описать с помощью элегантной формулы, которая зависела бы от величин решения некоего уравнения. Когда я нашла это второе уравнение, к моему удивлению оно оказалось не чем иным как уравнением Ферма. Забавное случайное соотношение между формой и содержанием. Так что, когда я нашла контрпример, мне осталось только использовать эти числа как план для построения доказательства того, что это уравнение не имеет решения. Замечательно просто, если подумать. Не знаю, почему никто не нашел этого результата раньше.

\emph{Ахилл} : В результате этого неожиданного блестящего математического успеха, г-же Ч удалось провести акусто-поиск о котором она столько лет мечтала. Подарок полученный м-ром Крабом представляет собой осязаемую реализацию этой абстрактной работы.

\emph{Краб} : Не говорите мне пожалуйста что это запись Баха, играющего на клавесине собственные сочинения!

\emph{Ахилл} : Сожалею, но приходится поскольку это именно она и есть! Это набор из двух записей Себастиана Баха исполняющего весь Хорошо Темперированный Клавир. На каждой пластинке записана одна из двух его частей, это значит что каждая запись состоит из 24 прелюдий и фуг по одной в каждом мажорном и минорном ключе.

\emph{Краб} : В таком случае мы должны немедленно прослушать эти бесценные пластинки! Как я смогу вас отблагодарить?

\emph{Черепаха} : Вы уже нас отблагодарили сполна этим превосходным чаем, который вы для нас приготовили.

(Краб вынимает одну из пластинок из конверта и ставит ее на свой патефон. Комната наполняется звуками потрясающей, мастерской игры на клавесине, при этом качество записи самое высокое, какое можно вообразить. Можно даже разобрать --- или это только воображение слушателя? ---~тихий голос Баха, подпевающего собственной игре)

\emph{Краб} : Хотите следить по нотам? У меня есть уникальное издание Хорошо Темперированною Клавира, проиллюстрированное одним из моим учителей, который также был необыкновенным каллиграфом.

\emph{Черепаха} : Это было бы чудесно.

\emph{(Краб подходит к элегантному книжному шкафу с застекленными дверцами, открывает его и достает два больших тома.)}

\emph{Краб} : Вот, пожалуйста, г-жа Черепаха. Я сам еще не видел всех прекрасных иллюстраций в этом издании, все никак клешни не доходят. Может быть, ваш подарок меня наконец на это подвигнет.

\emph{Черепаха} : Надеюсь.

\emph{Муравьед} : Вы заметили, что во всех этих произведениях прелюдия точно определяет настроение следующей фуги?

\emph{Краб} : О, да. Хотя это трудно объяснить, но между ними всегда есть некая таинственная связь. Даже если у прелюдии и фуги нет общей музыкальной темы, в них всегда присутствует неуловимое абстрактное нечто, которое их прочно связывает.

\emph{Черепаха} : И в кратких моментах тишины, которые отделяют прелюдию от фуги, есть что-то необыкновенно драматическое. Это тот момент, когда тема фуги готова вступить в свои права, сначала в отдельных голосах, которые потом сплетаются, создавая все более сложные уровни странной, изысканной гармонии.

\emph{Ахилл} : Я знаю, что вы имеете в виду. Я слышал еще далеко не все прелюдии и фуги, но меня очень волнует этот момент тишины; в это время я всегда пытаюсь угадать, что старик Бах задумал на этот раз. Например, я всегда спрашиваю себя, в каком темпе будет следующая фуга? Будет ли это аллегро или адажио? Будет ли она на 6/8 или на 4/4? Будет ли в ней три голоса, или пять --- или четыре? И вот звучит первый голос\ldots{} Потрясающий момент!

\emph{Краб} : Да, я помню давно ушедшие дни моей юности, дни, когда я трепетал от счастья, слушая эти прелюдии и фуги, возбужденный их новизной и красотой, и теми сюрпризами, которые они скрывают.

\emph{Ахилл} : А теперь? Неужели это счастливый трепет прошел?

\emph{Краб} : Он перешел в привычку, как всегда и бывает с подобными чувствами. Но в привычке также есть своя глубина, и это приносит определенное удовлетворение. Кроме того, я всегда обнаруживаю какие-нибудь новые сюрпризы, которых раньше не замечал.

\emph{Ахилл} : Повторения темы, которых вы раньше не слышали?

\emph{Краб} : Может быть; в особенности, когда эта тема проходит в обращении, спрятанная среди нескольких других голосов, или когда она поднимается на поверхность, словно возникая из ничего. Есть там также и удивительные модуляции, которые приятно слушать снова и снова, спрашивая себя, как это старик Бах смог создать подобное.

\emph{Ахилл} : Приятно слышать, что все эти радости останутся у меня после того, как пройдет моя первая влюбленность в Хорошо Темперированный Клавир, жаль, однако, что это блаженное состояние не может длиться вечно.

\emph{Краб} : Не бойтесь, влюбленность не пройдет бесследно. Эта юношеская влюбленность хороша тем, что ее всегда можно оживить именно тогда, когда вы считаете, что она уже умерла. Для этого необходим лишь толчок извне в нужном направлении.

\emph{Ахилл} : Правда? Что же это за толчок?

\emph{Краб} : Например, прослушивание этой музыки ушами того, кто слушает ее в первый раз; такой человек здесь вы, Ахилл. Каким-то образом, ваш трепет передается мне, и я снова полон блаженного восторга!

\emph{Ахилл} : Звучит интригующе. Восторг спит где-то внутри вас, но сами вы не в состоянии вытащить его из глубин подсознания.

\emph{Краб} : Именно так. Возможность оживить это чувство «закодирована» каким-то образом в структуре моего мозга, но я не могу осуществить это по желанию; я должен ждать счастливого случая, который запустит этот механизм.

\emph{Ахилл} : У меня вопрос насчет фуг; мне стыдно об этом спрашивать, но, поскольку я новичок в искусстве слушания фуг, не может ли кто-нибудь из вас, матерых слушателей, научить меня кое-чему?

\emph{Черепаха} : Я с удовольствием поделюсь с вами своими скудными познаниями, если это может вам чем-то помочь.

\emph{Ахилл} : О, благодарю вас. Позвольте мне начать издалека. Знакомы ли вы с гравюрой М. К. Эшера под названием «Куб с магическими лентами»?

\emph{Рис. 56. М. К. Эшер «Куб с магическим лентами» (литография, 1957)}

\emph{Черепаха} : На которой изображены изогнутые ленты с искривлениями в виде пузырей, которые кажутся попеременно то выпуклыми, то вогнутыми?

\emph{Ахилл} : Она самая.

\emph{Краб} : Я помню эту картину. Кажется, что пузыри на ней все время перескакивают из одного состояния в другое: они то выпуклые, то вогнутые, в зависимости от того, с какого угла на них посмотреть. Невозможно одновременно увидеть их и выпуклыми, и вогнутыми --- почему-то мозг этого просто не позволяет. У нас просто есть два разных способа воспринять эти пузыри.

\emph{Ахилл} : Вы совершенно правы Знаете, мне кажется, что я открыл два способа слушать фугу, в чем-то аналогичных этому Вот они: либо следить лишь за одним отдельным голосом в каждый момент, либо слушать общее звучание, не пытаясь распутать голоса. Я пробовал оба эти способа и, к моему разочарованию, оказалось, что каждый из них исключает другой. Это просто не в моей власти слушать каждый индивидуальный голос и в то же время слышать общий эффект. Я все время перескакиваю с одного способа на другой, более или менее спонтанно и непроизвольно.

\emph{Муравьед} : Так же, как когда вы смотрите на магические ленты?

\emph{Ахилл} : Да. Но скажите\ldots{} мое описание двух способов слушания фуги безошибочно указывает на меня, как на наивного, неопытного слушателя, не способного уловить более глубокие уровни восприятия?

\emph{Черепаха} : Вовсе нет, Ахилл Я могу говорить только за себя, но я тоже постоянно перепрыгиваю с одного способа на другой, не контролируя этот процесс и не пытаясь сознательно решить, какой из двух способов должен господствовать. Не знаю, испытывали ли остальные наши друзья что-нибудь подобное.

\emph{Краб} : Безусловно. Это весьма мучительное состояние, поскольку вы чувствуете, что дух фуги витает где-то близко --- но вы не можете охватить его полностью, так как не в состоянии слушать сразу двумя способами.

\emph{Муравьед} : У фуг есть интересная особенность: каждый из голосов является музыкальной пьесой сам по себе, так что фугу можно рассматривать как набор нескольких различных музыкальных произведений, основанных на одной и той же теме и исполняемых одновременно. И слушатель (или его подсознание) должен сам решать, воспринимать ли фугу как целое или как набор отдельных частей, гармонирующих друг с другом.

\emph{Ахилл} : Вы говорите, что эти части «независимы», однако это не может быть совершенно верным. Между ними должна существовать какая-то координация, иначе, когда они исполняются вместе, мы слышали бы беспорядочное столкновение звуков --- а это далеко не так!

\emph{Муравьед} :~Наверное, лучше сказать так: если бы вы слушали каждый голос в отдельности, вы обнаружили бы, что он имеет смысл сам по себе. Он может быть исполнен в одиночку, и именно это я имел в виду, говоря, что голоса независимы. Но вы совершенно правы, указывая, что каждая из этих индивидуальных мелодий соединяется с остальными совсем не случайным образом, сливаясь в изящное целое. Искусство создания прекрасных фуг заключается именно в умении соединять несколько линий, каждая из которых кажется написанной ради своей собственной красоты --- но когда они взяты все~вместе, целое звучит вполне естественно. Между прочим, двойственность между слушанием фуги как целого и слушанием составляющих её голосов --- это частный пример более общей двойственности, приложимой к разным структурам, построенным, начиная с нижних уровней.

\emph{Ахилл} : Правда?~Вы хотите сказать, что мои два «способа»~приложимы не только к ситуации со слушанием фуг?

\emph{Муравьед} : Совершенно верно.

\emph{Ахилл} : Интересно, как это может быть? Наверное, это связано с попеременным восприятием чего-либо как целого, или как собрания его частей. Но я сталкивался с этой дихотомией только слушая фуги

\emph{Черепаха} : Вот это да! Посмотрите-ка! Я только что перевернула страницу следя за музыкой, и нашла великолепную иллюстрацию на странице перед титульным листом.

\emph{Краб} : Я раньше никогда не видел этой иллюстрации. Будьте добры, передайте книгу по кругу.

\emph{(Черепаха передает книгу. Каждый из четырех приятелей рассматривает книгу по-своему --- кто издалека, кто поднося прямо к глазам, при этом каждый из них качает головой в удивлении. Наконец, книга обходит всех и возвращается к Черепахе, которая смотрит в нее очень внимательно)}

\emph{Ахилл} : Мне кажется, прелюдия почти кончилась. Хотелось бы знать, удастся ли мне, слушая фугу, найти ответ на этот опрос «как нужно слушать фугу --- как целое или как сумму частей?»

\emph{Черепаха} : Слушайте внимательно и вы поймете!

\emph{(Прелюдия заканчивается. Следует пауза, и затем\ldots{} )}


% \subsubsection{ГЛАВА X: Уровни описания и компьютерны системы}
% \subsubsection{ГЛАВА X: Уровни описания и компьютерны системы}

Уровни описания

У ГЁДЕЛЕВОЙ СТРОЧКИ G и у фуги Баха есть одно и то же свойство: их можно понять на нескольких уровнях. Все мы знакомы с подобным явлением; иногда оно нас озадачивает, а иногда мы не видим в нем ничего особенного. Например, все мы знаем, что человеческие существа сделаны из огромного количества (около 25 триллионов) клеток и, следовательно, все, что мы делаем, может быть в принципе описано на клеточном --- или даже на молекулярном --- уровне. Большинство из нас воспринимает этот факт как нечто само собой разумеющееся. Когда мы идем к доктору, он смотрит на нас на более низком уровне, чем воспринимаем себя мы сами. Мы читаем о ДНК и «генетической инженерии», попивая при этом кофе. По-видимому, нам удалось примирить эти два несовместимых восприятия нас самих, просто разъединив их в сознании. Для нас практически невозможно соотнести собственное микроскопическое описание с восприятием себя как личности, и поэтому мы храним эти две разные картины в разных «отделениях» мозга. Изредка мы пытаемся соотнести эти два восприятия, спрашивая себя: «Как это так, что эти две совершенно разные вещи --- не что иное, как один и тот же человек?»

Возьмите, например, последовательность образов на экране телевизора, показывающего улыбающуюся Мэрилин Монро. Глядя на эту последовательность, мы знаем, что на самом деле видим не женщину, а множество мерцающих точек на плоской поверхности. Однако в данный момент это нас совершенно не волнует. У нас в голове совмещаются две абсолютно разные картины того, что мы видим на экране, но это нас не смущает. Мы можем легко «выключить» одну из них и начать следить за другой и делаем это постоянно. Какая из них «реальнее»? Это зависит от того, кто вы такой: человек, собака, компьютер или телевизионный аппарат.

Блоки и шахматное мастерство

Одна из самых трудных задач, стоящих перед исследователями искусственного интеллекта --- найти способ соединить эти два описания и создать систему, которая могла бы принимать один уровень описания и производить другой. Эта проблема хорошо иллюстрируется прогрессом в создании компьютерных программ, играющих в шахматы. В 1950-х и 1960-х годах считалось, что ключом к созданию хорошо играющей машины является ее умение заглянуть вперед в разветвляющуюся сеть возможных продолжений игры дальше, чем любой шахматный мастер. Однако, когда программы стали мало-помалу приближаться к этой цели, обнаружилось, что никакого скачка в качестве игры шахматных компьютеров не произошло, и они не обогнали человеческих экспертов. Фактом остается то, что по сегодняшний день шахматные мастера-люди все еще регулярно обыгрывают самые лучшие программы.

Объяснение этого факта давно уже опубликовано В 1940 году датский психолог Адриан де Грот исследовал то, как шахматные мастера, в отличие от новичков, оценивают позицию. Его исследования показали, что мастера воспринимают расположение фигур \emph{блоками} . Существует более высокий уровень описания доски, чем прямолинейное «белая пешка на е5, черная ладья на д6», и мастер каким-то образом создает мысленный образ доски на высшем уровне. Это доказывается тем, как быстро, по сравнению с новичком, мастер может восстановить какую-либо позицию из партии, после того, как обоим показали доску в течение пяти секунд. Весьма показателен тот факт, что ошибки мастера касались целых групп фигур, которые он ставил в неправильное место; при этом стратегически позиция оставалась почти той же самой --- но не на взгляд новичка! Окончательным доказательством этого факта послужил тот же эксперимент, в котором на этот раз вместо настоящих позиций фигуры были расставлены как попало. В реконструкции таких случайных позиций мастера показали себя ничуть не лучше новичков.

Из этого следует, что в шахматных партиях повторяются некие типы ситуаций, некие определенные схемы и что именно эти схемы высшего уровня воспринимаются мастером. Он думает на \emph{ином уровне} , чем новичок, и оперирует другим набором понятий. Почти все бывают удивлены, узнав, что во время партии мастер редко заглядывает вперед дальше, чем новичок --- более того, мастер обычно рассматривает всего лишь горстку возможных ходов. Трюк заключается в том, что его восприятие доски подобно фильтру, глядя на позицию, он буквально \emph{не видит плохих ходов} , подобно тому, как любители не видят ходов, \emph{противоречащих правилам} . Любой, кто хотя бы немного играл в шахматы, организует свое восприятие таким образом, что диагональные ходы ладьей, вертикальное взятие пешками и тому подобное просто не приходят ему в голову. Подобно этому, мастера создали высшие уровни организации в их восприятии позиции; в результате, рассматривать плохие ходы для них так же маловероятно, как для большинства людей --- рассматривать незаконные ходы. Это можно назвать \emph{явной обрезкой} гигантского разветвленного дерева возможностей. С другой стороны, неявная обрезка включает рассмотрение хода и, после поверхностного анализа, решение этот ход больше не анализировать.

Это различие приложимо также и к другим видам интеллектуальной деятельности --- например, к занятиям математикой. Способный математик обычно не обдумывает всяческие ложные пути к доказательству нужной теоремы, как это могли бы делать менее одаренные люди; скорее, он «нюхом чувствует» многообещающие пути и сразу направляется по ним.

Компьютерные шахматные программы, основанные на заглядывании далеко вперед, не научены думать на высшем уровне; стратегией таких машин была «грубая сила» просчета вариантов, в надежде таким образом сокрушить любое сопротивление. Однако оказалось, что эта стратегия не работает. Может быть, когда-нибудь и удастся создать такую программу, которая, основываясь только на грубой силе --- умению считать варианты --- действительно сможет обыгрывать лучших человеческих игроков. Однако это будет небольшим выигрышем в области интеллекта, по сравнению с открытием того, что важнейшей составляющей разума является его умение создавать многоуровневые описания сложных схем, таких, как шахматные доски, телевизионные экраны, печатные страницы или картины.

Похожие уровни

Обычно нам не приходится держать в уме больше одного уровня понимания ситуации. Более того, как мы уже сказали ранее, различные описания одной и той же системы бывают настолько далеки друг от друга концептуально, что у нас не возникает проблемы одновременного восприятия обоих; они просто хранятся в разных мысленных отделениях. Трудности возникают тогда, когда одна и та же система допускает два или более описаний, в чем-то похожих друг на друга. Тогда нам бывает трудно, думая о системе, не смешивать различные уровни --- и при этом мы легко можем запутаться.

Вне сомнения, это происходит, когда мы думаем о нашей собственной психологии~--- скажем, когда мы пытаемся понять мотивы различных человеческих поступков. В структуре человеческого разума есть множество уровней --- безусловно, это система, которую мы пока понимаем недостаточно хорошо. Существуют сотни соперничающих друг с другом теорий, объясняющих различное поведение; они основаны на предположениях о том, насколько глубоко в~этой иерархии уровней расположены те или иные психологические «силы». Поскольку в настоящее время мы используем почти один и тот же язык для описания различных уровней, это приводит к немалой путанице и, наверняка, к рождению множества ложных теорий. Например, мы говорим о стимулах --- сексе, власти, славе, любви --- понятия при этом не имея, где именно в структуре человеческого интеллекта они зарождаются. Я не буду останавливаться на этом подробно; скажу лишь, что наше непонимание того, кто мы есть, безусловно связано с тем фактом, что мы состоим из большого количества уровней и используем один и тот же язык для описания нас самих на разных уровнях.

Компьютерные системы

Существует еще одно место, где многие уровни описания сосуществуют в единой системе и все уровни концептуально близки один к другому. Я имею в виду компьютерные системы. Работающую компьютерную программу можно рассматривать на нескольких уровнях. На каждом уровне описание дается на языке вычислительных машин, что делает все описания в какой-то мере схожими --- в то же время между нашим восприятием разных уровней есть крайне важные различия.~На низшем уровне описание настолько сложно, что его можно сравнить с описанием образа на экране телевизора в виде набора точек; однако для определенных целей нужен именно такой взгляд на вещи. На высшем уровне описание представлено в форме крупных \emph{блоков} и воспринимается совершенно по-другому, несмотря на то, что многие понятия повторяются как на низшем так и на высшем уровнях. Блоки описания на высшем уровне можно сравнить с блоками шахматного мастера и с блочным описанием образа на экране: они суммируют в сжатой форме те вещи, которые на низших уровнях представлены как отдельные. (См. рис. 57.)

\emph{Рис. 57. Идея «укрупнения» группа предметов воспринимается как единый «блок» Граница этого блока работает как клеточная мембрана или национальная граница, она устанавливает индивидуальность группы предметов внутри нее. В зависимости от контекста, внутренняя структура блока может приниматься во внимание или игнорироваться.}

Чтобы предмет нашего разговора не стал слишком абстрактным, обратимся к конкретным фактам из области вычислительной техники; для начала бросим взгляд на~то, что представляет собой компьютерная система на низшем уровне. Низший уровень? Не совсем, конечно --- но я не буду здесь говорить об элементарных частицах, так что для нас это будет низшим уровнем.

В основании компьютерной системы находится \emph{память} , \emph{центральный процессор (ЦП)} , и некоторые \emph{вводно-выводные устройства} . Сначала давайте опишем память. Она состоит из отдельных физических единиц, называемых словами. Для конкретности скажем что в памяти есть 65 536 слов (это типичное число --- 2 в 16-ой степени). Слово далее подразделяется на то что мы будем считать атомами информатики ---~\emph{биты} . В типичном слове --- около тридцати шести битов. Физически бит представляет собой магнитный «выключатель» который может быть в одном из двух положений.

00X0XXX0X00XX00X0XXXXXX0XX00XXX0000

--- слово из 36 битов ---

Вы можете называть эти положения «вверх» и «вниз», или «x» и «o», или «1» и «0». Последнее --- общепринятое название, оно вполне адекватно, но может запутать читателя, заставив его думать, что на самом деле в памяти компьютера хранятся числа. Это неверно. У нас столько же оснований думать о наборе из тридцати шести битов, как о числе, как и считать, что два четвертака --- это цена мороженого. Так же, как деньги могут быть использованы по-разному так и слово в памяти может выполнять разные функции. Строго говоря, иногда эти тридцать шесть битов действительно могут представлять число в двоичной записи. В другой раз они могут представлять тридцать шесть точек на экране телевизора, или же несколько букв текста. Наша интерпретация слова в памяти целиком зависит от той роли, которую это слово играет в использующей его программе. Разумеется, оно может играть несколько ролей --- как нота в каноне.

Команды и данные

Существует еще одна интерпретация слова, о которой я пока не упоминал слово может интерпретироваться как \emph{команда} . Слова памяти содержат не только данные, на основании которых действует компьютер, но и программу, действующую на эти данные. Существует ограниченное количество операций, которые могут быть выполнены центральным процессором --- ЦП --- и часть некоего слова (обычно несколько первых битов) интерпретируются как название типа команды, которая должна быть выполнена. Что же означают остальные биты в слове-команде? Чаще всего, они говорят, на какие другие слова памяти надо воздействовать. Иными словами, остальные биты являются указателем на какое-либо другое слово (или слова) памяти. Каждое слово в памяти имеет свое расположение, как дом на улице; это расположение называется \emph{адресом} . Память может иметь одну «улицу» или много «улиц» --- они называются страницами. Таким образом, адрес любого слова --- это номер страницы (если память подразделена на страницы) и его расположение на этой странице. Итак, «указатель» --- это часть команды, содержащая числовой адрес какого-либо слова (или слов) в памяти. На указатель нет никаких ограничений, так что команда может даже указывать сама на себя --- в этом случае, когда она действует, она изменяет саму себя.

Откуда компьютер знает, в какой момент надо выполнять ту или иную команду? Об этом заботится ЦП. В нем есть специальный указатель, который указывает (то есть хранит соответствующий адрес) на следующее слово-команду. ЦП извлекает это слово из памяти и копирует его на специальное слово в самом ЦП. (Слова в ЦП обыкновенно называют не словами, а \emph{регистрами} .) После этого ЦП выполняет эту команду. Команда может вызывать любую из большого количества возможных операций; типичные операции включают:

ДОБАВИТЬ слово, указанное в команде, к регистру. (В этом случае данное слово интерпретируется как число.)

НАПЕЧАТАТЬ слово, указанное в команде, в виде букв. (В этом случае данное слово интерпретируется не как число, а как строчка букв.)

ПЕРЕЙТИ к слову, указанному в команде. (В этом случае ЦП интерпретирует данное слово, как следующую команду.)

Если первоначальная команда не содержит явного указания поступить иначе, ЦП просто обращается к следующему слову и интерпретирует его, как команду. Иными словами, ЦП предполагает, что он должен двигаться вдоль по «улице» последовательно, как почтальон, интерпретируя слово за словом как команды. Однако это последовательное движение может быть прервано некоторыми командами, такими как, например, ПЕРЕХОД.

Язык машины и язык ассемблера

Вы только что прочитали очень краткий обзор \emph{машинного} языка. В этом языке типы существующих операций составляют конечный репертуар, который не может быть расширен. Таким образом любая программа, какой бы большой и сложной она не была, должна состоять из этих типов команд. Рассматривать программу, написанную на машинном языке, это все равно что рассматривать молекулу ДНК атом за атомом. Если вы вернетесь к рис. 41, где изображена последовательность нуклеотидов молекулы ДНК (и имейте в виду, что в каждом нуклеотиде около двух дюжин атомов) и представите себе, что вам надо записать, атом за атомом, ДНК крохотного вируса (уж не говоря о человеке!), то вы получите представление о том, что такое создание сложной программы на машинном языке и каково пытаться понять, что происходит в этой программе, если у вас есть доступ только к ее описанию на машинном языке.

Надо сказать, что первоначально программирование делалось на еще более низком уровне, чем машинный язык: соединялись определенные провода, так что нужные операции как бы «телеграфировались» машине. Этот процесс настолько примитивен по современным понятиям, что теперь его трудно себе вообразить. И все же люди, впервые это сделавшие, безусловно испытали такую же радость, какую когда-либо чувствовали создатели современных компьютеров\ldots{}

Перейдем теперь на более высокую ступень иерархии уровней описания программ --- уровень \emph{языка ассемблера} . Между машинным языком и языком ассемблера дистанция не так уж велика; скорее, это маленький шажок. Главное здесь то, что между командами на языке машины и командами на языке ассемблера существует взаимно однозначное соответствие. Язык ассемблера представляет отдельные команды машинного языка в виде «блоков», так что, желая например, записать команду сложения, вместо последовательности битов «010111000» вы пишете просто ДОБАВИТЬ, и вместо того, чтобы давать адрес в двоичном коде, вы можете указать на слово в памяти, назвав его по имени. Следовательно, программа на языке ассемблера --- это что-то вроде программы на машинном языке, сделанной более удобной для людского чтения. Машинную версию программы можно сравнить с деривацией ТТЧ, записанной в туманной нотации Гёделевых номеров, в то время как версия на языке ассемблера сравнима с изоморфной деривацией ТТЧ, записанной в более легкой для понимания первоначальной нотации самой ТТЧ. Или, возвращаясь к образу ДНК: разница, существующая между машинным языком и языком ассемблера подобна разнице между определением нуклеотидов при помощи их кропотливого, атом за атомом, описания и определением нуклеотидов по именам (как, например, «A», «G», «С» или «Т»). Подобная операция «превращения в блоки» представляет собой огромную экономию труда, хотя концептуально почти ничего при этом не меняется.

Программы, переводящие программы

Возможно, что самое важное в языке ассемблера --- не его отличие от машинного языка, которое не столь уж велико, но сама идея того, что программы вообще могут быть написаны на различных уровнях. Ведь компьютерная аппаратура построена так, чтобы «понимать» программы на машинном языке --- последовательности битов --- а не буквы и не числа в десятичной записи! Что происходит, когда в эту аппаратуру вводится программа на языке ассемблера? Это напоминает попытку заставить клетку узнать бумажку с записанном буквами нуклеотидом, вместо самого нуклеотида со всеми его химическими компонентами. Что делать клетке с этой бумажкой? Что делать компьютеру с программой на языке ассемблера?

Здесь мы подошли к главному: возможно написать на машинном языке \emph{программу-переводчик} . Эта программа, под названием \emph{ассемблер} , берет имена, десятичные числа и другие сокращения, которые программист может легко запомнить, и превращает их в монотонные, но необходимые последовательности битов. После того, как программа на языке ассемблера \emph{собрана} (то есть переведена), она --- точнее, ее эквивалент на машинном языке --- выполняется компьютером. Однако здесь это лишь вопрос терминологии; программа какого уровня выполняется машиной? Вы не ошибетесь, сказав, что выполняется программа~на машинном языке поскольку в выполнении любой программы всегда задействована аппаратура --- но вполне разумно также предположить, что выполняется программа на языке ассемблера. Например, вполне можно сказать: «В данный момент ЦП выполняет команду „ПЕРЕХОД``», вместо того, чтобы говорить «В данный момент ЦП выполняет команду „111010000``». Пианист, играющий ноты G-E B-E-B-G. в то же время играет арпеджио в ми миноре. Нет~причин отказываться от описания вещей с точки зрения высших уровней. Таким образом, можно считать, что программа на языке ассемблера выполняется одновременно с программой на машинном языке то, что происходит в ЦП, можно описать двумя способами.

Языки высших уровней, компиляторы и интерпретаторы

На следующем уровне иерархии крайне важная идея о том, что сами компьютеры можно заставить переводить программы с высших на низшие уровни, развивается еще далее. В начале 1950-х годов, когда программа ассемблера уже использовалась в течение нескольких лет, было подмечено, что существуют несколько характерных структур, появляющихся в программе за программой. По-видимому, так же как и в шахматах, это были некие характерные структуры, естественно возникающие тогда, когда люди пытаются найти алгоритмы --- точные описания процессов, которые они хотят осуществить. Иными словами, кажется, что в алгоритмах есть некие компоненты высшего уровня, при помощи которых они могут быть описаны с большей легкостью и эстетизмом, нежели на весьма ограниченном машинном языке или языке ассемблера. Обычно такой компонент высшего уровня в алгоритме представляет собой не одну-две машинных команды, но целый конгломерат; при этом эти команды не обязательно соседствуют в памяти. Подобный компонент может быть представлен на языке высшего уровня как некое единство, или блок.

Оказывается, что кроме стандартных блоков (только что открытых компонентов, из которых могут быть построены все алгоритмы), почти все программы содержат еще большие блоки --- так сказать, суперблоки. Эти суперблоки меняются от программы к программе, в зависимости от типа задания на высшем уровне, которое данная программа должна выполнить. Мы уже говорили о суперблоках в главе V, употребляя общепринятые названия, «подпрограммы» и «процедуры». Ясно, что если бы удалось \emph{определить} новые единицы высшего уровня в терминах уже известных единиц и затем \emph{вызывать} их по имени, это было бы важнейшим дополнением к любому языку программирования. Таким образом разделение на блоки оказалось бы включено в сам язык. Вместо определенного репертуара команд, из которых пришлось бы кропотливо собирать каждую программу, программист смог бы создавать свои собственные модули. Каждый из них имел бы собственное имя и мог бы использоваться в любом месте программы, как если бы он был неотъемлемой характеристикой языка. Разумеется, невозможно избежать того, что внизу, на уровне машинного языка, все будет состоять из прежних команд на этом языке; но они будут неявны и не будут видны программисту, работающему на высшем уровне. Новые языки, основанные на этих идеях, были названы \emph{языки-компиляторы} . Один из самых первых и элегантных получил имя «АЛГОЛ» (от английского Algorithmic Language --- алгоритмический язык). В отличие от языка ассемблера, здесь нет взаимно-однозначного соответствия между высказываниями на АЛГОЛе и командами на машинном языке. Однако некий тип соответствия между АЛГОЛом и машинным языком все же существует, хотя оно гораздо более запутано, чем соответствие между языком ассемблера и машинным языком. Грубо говоря, перевод программы на АЛГОЛе в ее машинный эквивалент сравним с переводом словесного выражения алгебраической проблемы на язык формул. (На самом деле, переход от словесного выражения задачи к~её выражению в формулах гораздо более сложен, но он дает определенное представление о типе «распутывания», необходимом при переводе с языка высшего уровня на язык низшего уровня ) В середине 1950-х годов были созданы удачные программы под названием \emph{компиляторы} , функцией которых был перевод с языка-компилятора на машинный язык.

Были изобретены также \emph{интерпретаторы.} Подобно компиляторам, интерпретаторы переводят с языков высших уровней на машинный язык, но вместо того, чтобы сначала переводить все высказывания и затем выполнять машинный код, они считывают одну строчку и тут же ее выполняют. Преимущество здесь в том, что для использования интерпретатора не обязательно иметь полную программу. Программист может придумывать свою программу строчка за строчкой и проверять ее в процессе создания. Интерпретатор по сравнению с компилятором --- то же, что устный переводчик по сравнению с переводчиком письменных текстов. Один из самых интересных и важных языков программирования --- это ЛИСП (от английского List Processing --- обработка списка), изобретенный Джоном Маккарти примерно тогда же когда был изобретен АЛГОЛ. Впоследствии ЛИСП приобрел большую популярность среди специалистов по искусственному интеллекту.

Между принципом работы интерпретаторов и компиляторов есть одно интересное различие. Компилятор берет входные данные (к примеру, законченную программу на Алголе) и производит некий результат (длинную последовательность команд на машинном языке). На этом его работа закончена и результат вводится в компьютер для обработки. Интерпретатор, напротив работает непрерывно, пока программист вводит одно за другим высказывания ЛИСПа, каждое из них немедленно выполняется. Однако это не означает что каждое высказывание сначала переводится и затем выполняется~--- тогда интерпретатор был бы всего лишь построчным компилятором. Вместо этого в интерпретаторе операции считки новой строчки, ее «понимания» и выполнения переплетены --- они происходят одновременно.

Поясню эту идею немного подробнее. Как только новая строчка ЛИСПа вводится в интерпретатор, он пытается ее обработать. Это означает, что интерпретатор начинает действовать, и в нем выполняются некие машинные команды. Какие именно --- это зависит, разумеется, от данного высказывания ЛИСПа. Внутри интерпретатора много команд типа ПЕРЕХОД, так что новая строчка ЛИСПа может заставить контроль двигаться довольно сложным путем вперед, назад, затем опять вперед и т. д. Таким образом, каждое высказывание ЛИСПа превращается в некий «маршрут» внутри интерпретатора, и следование по этому маршруту приносит нужный эффект.

Иногда бывает полезно интерпретировать высказывания ЛИСПа как некие блоки данных, которые постепенно вводятся в непрерывно действующую программу машинного языка (интерпретатор ЛИСПа). Думая об этом таким образом, вы по-иному видите отношения между программой, написанной на языке высшего уровня, и исполняющей эту программу машиной.

Самонастройка

Разумеется, компилятор, будучи программой, должен быть сам написан на каком-либо языке. Первые компиляторы были созданы на языке ассемблера вместо машинного языка; таким образом полностью использовались преимущества подъема на одну ступеньку над машинным языком. Краткое описание этих довольно сложных понятий представлено на рис. 58.

\emph{РИС. 58. Как ассемблеры, так и компиляторы~--- это переводчики на машинный язык. Это указано прямыми линиями. Более того, поскольку они сами являются программами, они первоначально также создаются на каком-либо языке программирования. Волнистые линии указы указывают на то, что компилятор может быть написан на языке ассемблера, а ассемблер~--- на машинном языке.}

По мере того, как программирование становилось более изощренным, было замечено, что частично законченный компилятор может быть использован для того, чтобы компилировать собственные продолжения. Иными словами, когда создано определенное минимальное ядро компилятора, это минимальное ядро может переводить большие компиляторы на машинный язык, пока таким образом не создастся окончательный, полный компилятор. Этот процесс известен под именем «самонастройки»; он несколько напоминает достижение ребенком критического уровня владения своим родным языком, после чего его словарь и грамматическое мастерство растут как снежный ком, так как для изучения языка он может \emph{использовать} сам язык.

Уровни описания работающих программ

Языки компиляторов обычно не отражают структуры машин, на которых будут выполняться написанные на этих языках программы. Это одно из их основных преимуществ по сравнению с весьма специализированными языками ассемблера и машинным языком. Разумеется, когда программа на языке компилятора переводится на язык машины, получается программа, зависящая от машины. Таким образом, возможно описать программу, которая исполняется либо зависящим от машины путем, либо не зависящим, подобно тому, как мы можем описать абзац в книге по его содержанию (описание, не зависящее от издания) или по номеру страницы и его расположению на ней (описание, зависящее от издания).

Пока программа работает хорошо, то, как мы ее описываем и что мы о ней думаем, не столь важно. Но как только возникают неполадки, становится важным умение увидеть программу на разных уровнях. Если, например, компьютеру дана задача в какой-то момент разделить на нуль, он остановится и сообщит пользователю о возникшей проблеме, указав при этом, в каком месте программы произошло это неприятное событие. Однако эти детали часто сообщаются на более низком уровне, чем тот, на котором написана сама программа. Вот три параллельных описания забуксовавшей программы:

Уровень машинного языка:

«Выполнение программы прекратилось по адресу 1110010101110111»

Уровень языка ассемблера:

«Выполнение программы прекратилось, когда она дошла до команды РАЗДЕЛИТЬ».

Уровень языка компилятора:

«Выполнение программы прекратилось в момент оценки алгебраического выражения „(А + B)/Z``».

Одна из основных задач программистов (людей, которые создают компиляторы, интерпретаторы, ассемблеры и другие программы, которые затем используются многими людьми) --- это создание находящих ошибки подпрограмм. Необходимо, чтобы информация, которую эти подпрограммы выдают пользователю, в чьей программе обнаружен дефект, представляла бы описание проблемы на высшем, а не на низшем, уровне. Интересно, что если сбой обнаруживается в генетической «программе» (например, мутация), то происходит обратное, ошибка бывает заметна только на \emph{высшем} уровне, то есть на уровне фенотипа, а не генотипа. На самом деле, современная биология использует мутации, как одно из основных окон в мир генетических процессов, поскольку они могут быть прослежены на многих уровнях.

Микропрограммирование и операционные системы

В современных компьютерных системах есть несколько других уровней иерархии. Например, некоторые системы --- часто называемые «микрокомпьютерами» --- используют еще более рудиментарные команды на машинном языке, чем добавка числа в памяти к числу в регистре. Пользователь должен сам решать, какой тип команд на обычном машинном языке он хочет запрограммировать; он «микропрограммирует» эти команды в терминах имеющихся у него «микрокоманд» После этого разработанные им команды на языке высшего уровня могут быть включены в схему компьютера и стать частью аппаратуры, хотя это и не обязательно. Подобное микропрограммирование позволяет пользователю спуститься немного ниже уровня обычного машинного языка. Одним из следствий этого является то, что какой-либо компьютер одной фирмы может, путем микропрограммирования, быть снабжен такой аппаратурой, что она повторяет машинные команды другого компьютера той же (или даже иной) фирмы. При этом говорится, что компьютер с микропрограммой имитирует другой компьютер.

Далее, у нас имеется уровень \emph{операционной системы} , который расположен между уровнями программы на машинном языке и следующим уровнем, на котором программирует пользователь. Операционная система --- это программа, предотвращающая доступ пользователей к самой машине (и таким образом защищающая систему); эта программа избавляет пользователя от многих сложных и запутанных проблем, таких, как прочтение программы, вызов программы-переводчика, выполнение переведенной программы, направление результата по нужным каналам в нужное время и передача контроля следующему пользователю. В случае, когда с ЦП говорят сразу несколько пользователей, операционная система переключает внимание ЦП в определенном порядке. Операционные системы удивительно сложны; здесь я только намекну на эти сложности при помощи следующей аналогии.

Рассмотрим первую телефонную систему. Александр Грэхем Белл мог позвонить своему ассистенту в соседнюю комнату: электронная передача голоса! Это сравнимо с простым компьютером без операционной системы: электронные вычисления!

Рассмотрим теперь современную телефонную систему. У вас есть выбор, с каким телефоном соединиться; к тому же, можно отвечать на многие звонки одновременно. Вы можете добавить код и соединиться с другими районами. Вы можете позвонить прямо или через оператора; так, что звонок будет оплачен вашим собеседником или по вашей кредитной карточке. Можно говорить с одним человеком или сразу с несколькими; можно «перенаправить» или проследить звонок. Существует сигнал «занято», сигнал, говорящий вам, что набранный номер не является «хорошо сформированным» и сигнал, говорящий вам, что вы набирали номер слишком долго. Вы можете установить местный коммутатор, соединяющий несколько телефонов, --- и так далее, и тому подобное. Это удивительный список, если подумать, сколько возможностей он представляет, в особенности, по сравнению с былым чудом~«голого» телефона. Вернемся теперь к компьютерам: сложные операционные системы выполняют примерно те же операции направления трафика и переключения уровней по отношению к пользователям и их программам. Мы можем быть практически уверены в том, что у нас в мозгу происходят некие параллельные процессы, одновременная обработка многих стимулов; решения о том, что должно выйти на первый план и на какое время; мгновенные «перерывы» из-за неожиданных событий и критических положений и так далее.

Забота о пользователе и защита системы

Многие уровни сложной компьютерной системы, взятые вместе, облегчают пользователям их работу, позволяя им не думать о процессах, происходящих на низших уровнях (которые, скорее всего, для них совершенно неважны). Пассажир в самолете обычно не интересуется уровнем горючего в баках, скоростью ветра, количеством куриных крылышек, которые будут поданы на ужин пассажирам, или воздушным трафиком около места назначения. Все это --- дело служащих на разных уровнях иерархии авиакомпании; пассажир же хочет только одного: чтобы его доставили из одного места в другое. Только когда случается что-нибудь непредвиденное, например, потеря багажа, пассажир понимает, с какой запутанной системой уровней он имеет дело.

Компьютеры --- супергибкость или супержесткость?

Одной из основной целей в нашем стремлении к высшим уровням всегда было желание сообщать компьютеру о том. чего мы от него хотим, самым естественным для нас образом. Безусловно, конструкции высшего уровня в языках компиляторах ближе к категориям, в которых обычно думают люди, чем конструкции низшего уровня, такие, как в машинных языках. Но в этом стремлении к легкости общения с компьютерами мы обычно забываем об одном из аспектов «естественности», --- а именно, том факте, что общение между людьми имеет намного меньше ограничений, чем общение между человеком и машиной. Например, мы зачастую произносим бессмысленные словосочетания, ища, как бы получше выразить свою мысль, кашляем в середине фразы, перебиваем друг друга, используем двусмысленные описания и «неправильный» синтаксис, придумываем выражения и искажаем смысл --- но наши сообщения обычно все же достигают цели. В языках программирования, напротив синтаксис должен быть стопроцентно строгим, в них не должно быть двусмысленных выражений и конструкций. Интересно, что печатный эквивалент кашля разрешен, но только если он предварен условным знаком (например, словом КОММЕНТАРИЙ), после него также должен иметься условный знак (например, точка с запятой). Ирония в том, что эта небольшая уступка гибкости создает свои проблемы: если точка с запятой (или любой другой условный знак, отмечающий конец комментария) встречается \emph{внутри} комментария, программа переводчик интерпретирует ее, как сигнал окончания комментария, после чего следует полная неразбериха.

Представьте, что в программе определена процедура под названием ПОНИМАНИЕ, и эта процедура затем вызвана семнадцать раз. Если в восемнадцатый раз это слово ошибочно написано ПОМИНАНИЕ, горе программисту! Компилятор взбунтуется и напечатает весьма неприятное послание ОШИБКА, сообщая, что он никогда не слыхал ни о каком ПОМИНАНИИ. Часто, когда компилятор обнаруживает подобную ошибку, он пытается продолжить работу, но из-за отсутствия у него поминания, он не может понять, что имел в виду программист. На самом деле, он может даже вообразить, что тот имел в виду нечто совершенно другое, и начать действовать согласно этой ошибочной интерпретации. В результате, остальная программа будет усеяна посланиями «ошибка», потому что компилятор --- а не программист --- запутался. Вообразите, какая путаница получится, если, переводя с английского на русский, переводчик услышит фразу по-французски и попытается переводить остальной английский текст, как французский! Компиляторы часто запутываются таким жалким образом. \emph{C'est la vie.}

Может быть, это звучит как приговор компьютерам, --- но я вовсе не имел это в виду. В некотором смысле, такое положение вещей необходимо. Если подумать, для чего обычно используются компьютеры, становится ясно, что они выполняют весьма определенные и точные задания, которые слишком сложны для людей. Чтобы мы могли доверять компьютерам, необходимо, чтобы они совершенно точно, без следа двусмысленности, понимали, что от них требуется. Необходимо также, чтобы компьютер делал не больше и не меньше того, что ему приказано. Если между компьютером и программистом стоит программа, предназначенная угадывать, чего тот хочет или имеет~в виду, то весьма вероятно, что, когда программист попытается сообщить машине задачу, она будет понята совершенно неверно. Таким образом важно, чтобы программы высшего уровня хотя и удобные для людей, тем не менее были бы недвусмысленными и точными.

Как предвосхитить желания пользователя

Несмотря на это, возможно создать язык программирования который допускает некоторый тип неточности и программу, переводящую его на низшие уровни. Можно сказать, что программа-переводчик при этом будет пытаться интерпретировать нечто, сделанное «вне правил языка». Но если в языке допускаются некие «нарушения» правил, подобные нарушения уже нельзя назвать настоящими нарушениями, поскольку они включены в правила! Если программисту разрешено допускать определенный тип ошибок, он может использовать эту черту, зная, что при этом он оперирует строго в рамках правил, несмотря на видимость обратного. Иными словами, если пользователь знает о всех трюках, делающих программу-переводчика более гибкой и удобной для пользования, то он знает и предел, который он не может перейти; следовательно, ему эта программа все равно кажется жесткой и негибкой, хотя она и дает ему гораздо большую свободу по сравнению с ранними версиями, не включавшими «автоматическую компенсацию человеческих ошибок.»

По отношению к «эластичным» языкам подобного типа может быть две альтернативы: (1) пользователь знает о встроенных в язык и в программу-переводчика уступках; (2) пользователь о них не знает. В первом случае, язык может быть использован для точного сообщения программ, поскольку программист может предсказать, как компьютер будет интерпретировать программы, написанные на этом языке. Во втором случае, в языке есть скрытые черты, могущие выкинуть что-нибудь непредсказуемое с точки зрения пользователя, не знающего о том, как работает программа-переводчик. Результатом этого могут быть грубые ошибки в интерпретации программы, поэтому такой язык не годится для использования компьютеров за их быстроту и надежность.

На самом деле, есть и третья альтернатива; (3) пользователь знает о встроенных в язык и в программу-переводчик отклонениях от правил, но их так много и они взаимодействуют таких сложным путем, что что он не может предсказать, как будут интерпретированы программы. Это может быть сказано о человеке, написавшем переводящую программу; он, разумеется, знает ее структуру как никто другой --- и все же он не может предсказать того, как она будет реагировать на данный тип необычной конструкции.

Одна из основных областей исследования в сегодняшней науке об искусственном интеллекте называется \emph{автоматическим программированием} ; автоматическое программирование работает над созданием языков еще более высоких уровней, языков, переводящие программы которых смогут проделывать хотя бы некоторые из следующих удивительных вещей: обобщать на основе примеров, исправлять типографские или грамматические ошибки, пытаться понять двусмысленные описания, при помощи упрощенной модели пытаться угадывать, что на уме у пользователя, задавать вопросы, когда машине что-то непонятно, использовать человеческий язык и т. д. Может быть, со временем удастся найти компромисс между гибкостью и строгостью.

Прогресс искусственного интеллекта --- это прогресс языка

Удивительно, насколько прогресс в исследовании компьютерной техники (и в частности, искусственного интеллекта) связан с развитием новых языков. В последнее десятилетие возникла ясная тенденция: воплощать новые открытия в новых языках. Один из ключей к пониманию и созданию интеллекта лежит в постоянном развитии и улучшении языков, описывающих процессы манипуляции символами. На сегодняшний день имеется около трех-четырех дюжин экспериментальных языков, созданных исключительно для исследований по искусственному интеллекту. Важно понимать, что любая программа, написанная на одном из этих языков, в принципе может быть переведена на языки низших уровней, хотя это и потребовало бы от людей огромных усилий; получившаяся программа была бы такой длинной, что она оказалась бы за пределами человеческого понимания. Это не означает, что каждый высший уровень увеличивает потенциал компьютера; весь этот потенциал уже существует в наборе команд машинного языка. Просто новые понятия на языках высшего уровня по самой своей природе наводят на мысль о новых путях и перспективах.

«Пространство» всех новых программ настолько обширно, что никто не может представить себе всех возможностей. Каждый язык высшего уровня предназначен для исследования определенных районов «программного пространства», таким образом, используя данный язык, программист оказывается в соответствующем районе. Язык не \emph{заставляет} его писать программы именно такого типа, но \emph{облегчает} для него выполнение определенных задач. Близость к понятию и небольшой толчок --- вот все, что обычно требуется здесь для крупного открытия; именно поэтому исследователи стремятся к языкам еще более высоких уровней.

Программирование на разных языках подобно сочинению музыкальных произведений в различных тональностях, особенно если вы работаете на клавиатуре. Если вы уже выучили или написали произведения во многих ключах, каждая клавиша будет иметь для вас свою собственную эмоциональную окраску. Некоторые мелодии кажутся естественными в одном ключе, но неловкими в другом. Таким образом, ваше направление определяется выбором тональности. В некотором роде, даже энгармонические тональности, такие как до-диез и ре-бемоль, весьма отличаются по настроению. Это говорит о том, что система нотации может играть важную роль в том, как будет выглядеть конечный продукт.

«Стратифицированная» схема искусственного интеллекта показана на рис. 59; внизу лежат компоненты аппаратуры, такие, как транзисторы, а на вершине расположены «думающие программы». Эта иллюстрация взята из книги «Искусственный интеллект» Патрика Генри Винстона (Patrick Henry Winston, «Artificial Intelligence»), она представляет собой общепринятый среди специалистов взгляд на искусственный интеллект. Хотя я согласен с идеей, что ИИ должен быть стратифицирован подобным образом, мне не кажется, что с таким небольшим количеством уровней возможно получить думающие программы.

\emph{РИС. 59. Для создания думающих программ необходимо построить серию уровней аппаратуры и программного обеспечения, чтобы избежать мучений работы со всеми процессами только на низшем уровне. Описания одного и того же процесса на разных уровнях весьма отличаются друг от друга, и только самый высший уровень представлен в блоках, достаточно крупных для нашего понимания. {[}Взято из книги «Искусственный интеллект» П. Г. Винстона (Р.Н. Winston, «Artificial Intelligence»).{]}}

Между уровнем машинного языка и уровнем, на котором может быть достигнут настоящий интеллект, должна, по-моему убеждению, лежать еще дюжина (или даже несколько дюжин!) уровней, каждый следующий из которых, базируясь на предыдущем, в то же время был бы более гибким. Сейчас мы с трудом можем себе вообразить, как это будет выглядеть\ldots{}

Параноик и операционная система

Схожесть уровней в компьютерной системе может привести к странному их смешению. Однажды я был свидетелем того, как пара моих приятелей --- оба новички в компьютерном деле --- забавлялись на терминале с программой «PARRY». PARRY --- это печально известная программа, симулирующая параноика весьма простеньким образом: она выдает заранее заготовленные английские фразы, выбранные из широкого репертуара. Правдоподобие достигается тем, что программа может определить, какие именно «заготовки» могут звучать разумно в ответ на фразы, введенные в компьютер человеком.

В какой-то момент PARRY надолго задумалась, и я объяснил друзьям, что задержка, скорее всего, связана с большой нагрузкой на систему разделения времени. Я сказал им, что они могут узнать, сколько человек подключены к системе в данный момент; для этого нужно напечатать специальный символ «контроль», который пойдет прямо в операционную систему, минуя PARRY. Один из моих приятелей нажал на соответствующую клавишу, после чего некие данные о статусе операционной системы отпечатались на экране поверх фраз PARRY. При этом PARRY об этом понятия не имела, поскольку это программа, «понимающая» только в скачках и пари, но ничего не знающая об операционных системах, терминалах и специальных символах контроля. Для моих друзей, однако, PARRY и операционная система были одним и тем же --- «компьютером», загадочным, аморфным, далеким существом, которое отвечало на то, что они печатали. Так что для них было вполне естественно, когда один из них с улыбкой напечатал «Почему вы пишете поверх того, что на экране?» Идея о том, что PARRY может ничего знать об операционной системе, при помощи которой она действует, была непонятна моим друзьям. Идея что «мы» знаем все о «нас самих» казалась им настолько естественной из их опыта людских контактов, что они просто распространили ту же идею на компьютер --- в конце концов, он же был достаточно умен, чтобы «разговаривать» с ними по-английски! Их вопрос похож на то, как если бы вы спросили кого-нибудь: «Почему вы сегодня производите так мало красных кровяных шариков?» Люди не знают об этом уровне --- «уровне операционных систем» --- их тела.

Главная причина этого смешения уровней была в том, что общение со всеми уровнями компьютерной системы происходило на одном и том же экране, на одном и том же терминале. Хотя наивность моих друзей может показаться преувеличенной, даже опытные компьютерные специалисты часто допускают подобные ошибки, когда несколько уровней сложной системы бывают одновременно представлены на одном и том же экране. Они забывают, кто их «собеседник» и печатают что-то, не имеющее смысла на данном уровне, хотя и вполне приемлемое на другом. Может показаться, что хорошо было бы заставить саму систему сортировать уровни --- интерпретировать команды согласно тому, какая из них «имеет смысл». К несчастью, подобная интерпретация требует от машины немалой толики здравого смысла и совершенного знания намерений программиста, а это требует настолько развитого искусственного интеллекта, которого на сегодняшний день у компьютеров нет.

Граница между аппаратурой и программным обеспечением

Путаница может также возникнуть из-за того, что одни уровни гибки, а другие --- строги и жестки. Например, в некоторых компьютерах есть замечательные программы-редакторы, которые позволяют переводить куски текста из одного формата в другой, почти так же, как жидкость может быть перелита из одного сосуда в другой. Узкую страницу можно превратить в широкую, и наоборот. При такой мощи можно ожидать, что поменять шрифт, скажем, на курсив, также не представит никакого труда. Однако у программы может быть только один шрифт, что делает подобные изменения невозможными. Бывает также, что нужный шрифт можно получить на экране, но не на принтере --- или наоборот. Долго работая с компьютерами, легко избаловаться и считать, что программированию должно поддаваться все; не должно быть негибких принтеров, имеющих только один шрифт, или даже конечный набор шрифтов, шрифты должны определяться пользователем! Но достигнув этой степени гибкости, мы начинаем расстраиваться, что принтер не печатает разноцветными чернилами на бумаге любой формы и размера или что он не чинит сам себя\ldots{}

Проблема в том, что в какой-то момент вся эта гибкость должна, используя фразу из главы V, «коснуться дна». Должен существовать жесткий уровень аппаратуры, на котором основано все остальное. Он может быть спрятан под гибкими уровнями так глубоко, что немногие пользователи чувствуют ограничения, налагаемые аппаратурой --- однако эти ограничения неизбежны.

В чем именно состоит разница между \emph{программным обеспечением} и \emph{аппаратурой} ? Это разница между программами и машинами --- между длинными и сложными последовательностями команд и физическими аппаратами, которые эти команды выполняют. Я сказал бы, что программное обеспечение --- это «то, что можно передать по телефону», а аппаратура --- это «все остальное». Пианино --- это аппаратура, а ноты --- программное обеспечение. Телефонный аппарат --- это аппаратура, а телефонный номер --- программное обеспечение. К сожалению, это полезное различие далеко не всегда так ясно.

В нас, человеческих существах, тоже есть аспекты «аппаратуры» и «программного обеспечения» и разница между ними для нас настолько естественна, что мы перестаем ее замечать. Мы привыкли к негибкости нашей физиологии: то, что мы не можем усилием воли вылечить себя от всех болезней или заставить расти у нас на голове волосы любого цвета --- лишь два простых примера. Однако мы можем «перепрограммировать» наш мозг, чтобы оперировать в рамках новых понятий. Удивительная гибкость интеллекта кажется почти несовместимой с тем фактом, что наш мозг сделан из «аппаратуры», подчиняющейся строгим правилам, аппаратуры, которую невозможно изменить. Мы не можем заставить наши нейроны реагировать быстрее или медленнее, не можем «поменять проводку» у себя в мозгу, не можем изменить внутренность нейрона --- короче, у нас нет \emph{никакого} выбора относительно нашей «аппаратуры» --- и тем не менее, мы можем контролировать собственные мысли.

Однако существуют аспекты нашего мышления, не поддающиеся контролю. Мы не можем, по желанию, стать сообразительнее; не можем выучить новый язык так быстро, как бы нам хотелось; не можем заставить себя думать о нескольких вещах сразу и так далее. Это знание о нашей природе столь изначально, что его даже трудно заметить; это все равно, что постоянно сознавать, что вокруг нас --- воздух. Мы никогда не думаем о возможной причине подобных «дефектов» нашего интеллекта --- устройстве нашего мозга. Основная цель этой книги --- предложить пути примирения между аппаратурой --- мозгом и программным обеспечением --- интеллектом.

Промежуточные уровни и погода

Мы видели, что в компьютерных системах есть множество довольно четко определенных уровней, и что работающая программа может быть описана в терминах любого из них. Таким образом, существуют не только низший и высший уровни --- есть самые различные степени низкого и высокого. Типичны ли промежуточные ступени для всех систем с низшими и высшими уровнями? Рассмотрим для примера систему, аппаратурой которой является земная атмосфера, а программным обеспечением --- погода. Проследить за движением всех молекул одновременно было бы способом «понимания» природы на весьма низком уровне --- что-то вроде работы с огромной сложной программой на машинном языке. Ясно, что эта задача лежит далеко за пределами человеческих возможностей. Однако у нас есть наш особый, человеческий способ наблюдения за погодными явлениями и их описания. Мы воспринимаем природные явления на высоком уровне --- крупными блоками, такими, как дождь, снег, туман, ураганы, холодные фронты, времена года, атмосферное давление, ветры, течения, кучевые облака, грозы, уровни инверсии и так далее. Во всех этих явлениях участвует астрономическое число молекул, которые каким-то образом действуют вместе, давая крупномасштабный эффект. Этот метод сравним с использованием для анализа погоды языка компилятора.

Существует ли аналог исследованию погоды при помощи промежуточных языков, таких, как язык ассемблера? Бывают ли, к примеру, очень маленькие местные «мини-штормы», как те крохотные смерчи, крутящие пыльные столбы максимум пару метров в диаметре? Является ли порыв ветра блоком промежуточного уровня, играющим роль в создании погодных явлений более крупного масштаба? Или же не существует практического способа использовать наши знания о подобных явлениях с тем, чтобы получить более полное объяснение погоды?

Тут возникают еще два вопроса. Первый такой: «Может ли быть, что погодные явления, воспринимаемые нами как смерчи и засухи, на самом деле --- лишь явления промежуточных уровней, составляющие часть каких-то более общих, медленно протекающих явлений?» В таком случае, погодные явления настоящего высшего уровня были бы глобальными, и их время измерялось бы по геологической шкале. Ледниковый период был бы погодным событием такого высшего уровня. Второй вопрос: «Есть ли такие погодные явления промежуточного уровня, которых люди до сих пор не замечали, но которые могли бы дать нам более глубокое понимание погоды?»

От смерчей к кваркам

Последнее предположение может звучать, как чистая фантазия, но это не совсем так. Стоит только взглянуть на точнейшую из точных наук, физику, чтобы найти необычные примеры систем, описанных в терминах взаимодействия таких «частей», которые сами по себе невидимы. В физике, как и в любой другой дисциплине, \emph{системой} считается группа взаимодействующих \emph{частей} . В большинстве известных нам систем части сохраняют свою индивидуальность при взаимодействии, так что мы можем различить их внутри системы. Например, когда собирается футбольная команда, ее игроки продолжают быть отдельными личностями, они не сливаются, теряя свою индивидуальность, в какое-то составное существо. И все же --- и это очень важно --- определенные процессы в их мозгу вызваны именно контекстом команды, вне которого эти процессы не происходили бы. Таким образом, в некотором смысле индивидуальность игроков меняется, когда они становятся частью большей системы --- команды. Такой тип системы называется \emph{почти разложимой системой} (термин взят из статьи Г. А. Саймона «Архитектура сложности» (H. А. Simon, «Architecture of complexity»). Подобная система состоит из слабо взаимодействующих модулей, которые сохраняют свою собственную индивидуальность во время взаимодействия, но, слегка меняясь по сравнению с тем, какими они бывают вне системы, тем самым способствуют связному поведению целой системы. Изучаемые в физике системы обычно принадлежат именно к такому типу. Считается, например, что атом состоит из ядра, положительный заряд которого удерживает на орбите, или в связанном состоянии, некоторое количество электронов. Связанные электроны весьма похожи на свободные электроны, несмотря на то, что они находятся внутри сложной системы.

Некоторые системы, изучаемые в физике, представляют собой контраст по сравнению с относительно простым атомом. В таких системах взаимодействие частей необычайно сильно, в результате чего они проглатываются большей системой и частично или полностью теряют свою индивидуальность. Примером является ядро атома, которое обычно описывается как «набор протонов и нейтронов». Но силы, удерживающие вместе частицы, составляющие ядро, так велики, что эти частицы становятся совершенно непохожи на самих себя в «свободной» форме (то есть когда они находятся вне ядра). На самом деле, ядро во многих смыслах более похоже на единую частицу, чем на набор взаимодействующих частиц. Когда ядро расщепляется, при этом обычно освобождаются протоны и нейтроны, но также и другие частицы, такие как пи-мезоны и гамма-лучи. Находятся ли все эти частицы внутри ядра до его расщепления, или же они --- что-то вроде «искр», летящих при расщеплении ядра? Возможно, что искать ответа на подобный вопрос не имеет смысла. На уровне физики частиц разница между возможностью «высекать искры» и действительным наличием субчастиц не столь ясна.

Таким образом, ядро --- это система, «части» которой, хотя они и невидимы внутри системы, могут быть извлечены и сделаны видимыми. Однако есть и более патологические случаи, такие, как протон и нейтрон, взятые как системы. Существует предположение, что каждый из них состоит из тройки «кварков» --- гипотетических частиц, которые могут соединяться по две или по три, образуя при этом многие из известных основных частиц. Однако взаимодействие между кварками настолько сильно, что их не только невозможно увидеть внутри протонов и нейтронов, но и невозможно извлечь оттуда! Таким образом, хотя кварки помогают теоретически объяснить некоторые свойства протонов и нейтронов, их собственное существование, возможно, никогда не будет установлено с достоверностью. Здесь перед нами --- антипод «почти разложимой системы», система, которую скорее можно назвать «почти неразложимой». Интересно, однако, что теория протонов и нейтронов (и других частиц), основанная на «модели кварков», дает хорошее количественное объяснение многих экспериментальных результатов, касающихся частиц, предположительно составленных из кварков.

Сверхпроводимость: «парадокс» ренормализации

В главе V мы обсуждали то, как ренормализованные частицы возникают из своих голых центров в результате рекурсивно накапливающихся взаимодействий с виртуальными частицами. Ренормализованную частицу можно рассматривать либо как это сложное математическое построение, либо как некий «бугорок», чем она и является физически. Одно из самых странных и впечатляющих последствий этого способа описания частиц --- это объяснение, которое оно дает знаменитому явлению сверхпроводимости (свободному от сопротивления течению электронов в некоторых твердых телах при очень низких температурах).

Оказывается, что электроны в твердых телах ренормализованы в результате их взаимодействия с некими странными квантами вибраций, называемыми \emph{фононами} (которые, в свою очередь, ренормализованы!). Такие ренормализованные электроны называются \emph{поляронами} . Вычисления показывают, что при низких температурах два полярона с противоположным спином начинают притягивать друг друга и могут стать определенным образом связанными. При некоторых условиях все поляроны, переносящие ток, связываются по два, образуя так называемые \emph{куперовы пары} . Парадоксально то, что образование этих пар происходит именно потому, что электроны --- голые центры спаренных поляронов --- электрически отталкиваются друг от друга. В отличие от электронов, куперовы пары не притягиваются и не отталкиваются; поэтому они могут свободно перемещаться в металле, словно в вакууме. Изменив математическое описание подобного металла с такого, чьими основными единицами являются поляроны, на такое, чьи основные единицы --- куперовы пары, вы получите значительно упрощенный набор уравнений. Эта математическая простота указывает на то, что деление на «блоки» куперовых пар --- естественный взгляд на сверхпроводимость.

Здесь есть несколько уровней частиц: сама куперова пара, пара составляющих ее поляронов с противоположным спином, электроны и фотоны из которых составлены поляроны; внутри электронов --- виртуальные фононы и позитроны\ldots{} и так далее, и тому подобное. Мы можем смотреть на каждый уровень и воспринимать происходящие там явления согласно нашему пониманию лежащих ниже уровней.

«Запечатывание»

Точно так же, к счастью, нам не нужно знать о кварках всего, чтобы понимать многое в поведении частиц, составной частью которых они могут быть. Специалист по ядерной физике может разрабатывать теории о ядрах, основанные на протонах и нейтронах, и игнорировать как теории о кварках, так и теории, оспаривающие последние. Ядерный физик работает с \emph{блочной} картиной протонов и нейтронов --- описанием, основанным на теориях низших уровней, которое при этом не требует понимания этих теорий. Подобно этому, атомный физик работает с блочной картиной атомного ядра, основанной на теории ядра. Химик работает с блочной картиной электронов и их орбит, создавая на этом основании теории небольших молекул, которые, в свою очередь, могут быть использованы как блоки специалистом по молекулярной биологии, который интуитивно понимает, как соединяются маленькие молекулы, но специализируется в области крупных молекул и их взаимодействий. Далее, специалист по биологии клеток берет блочную картину единиц, усердно изучаемых молекулярным биологом, и пытается использовать ее для объяснения клеточного взаимодействия. Думаю, что вам понятно, к чему я веду. Каждый уровень в каком-то смысле «запечатан» --- изолирован от уровней, находящихся ниже его. Это еще один из выразительных терминов Саймона, напоминающий о том, как подводная лодка делится на секции; если одна из частей разгерметизируется, проблема не распространяется на остальные секции, так как двери испорченной секции закрываются, изолируя ее от соседних помещений.

Хотя некоторая «утечка» между иерархическими уровнями наук присутствует всегда, и химик не может полностью игнорировать низшие уровни физики, или биолог --- полностью игнорировать химию, утечки между далекими уровнями почти не происходит. Именно поэтому мы можем понимать других людей, не имея при этом глубокого понимания модели кварков, структуры ядра, природы орбит электронов, химических связей, структуры белков, органоидов в клетках, путей межклеточного сообщения, физиологии различных органов человеческого тела, или сложных взаимодействий между органами. Все, что нам необходимо, --- это блочная модель действия высших уровней; и, как мы все знаем, подобные модели весьма реалистичны и успешны.

Разделение на блоки и детерминизм

Однако у блочной модели есть и значительная негативная сторона; обычно она не дает точных предсказаний. Это значит, что хотя блочные модели спасают нас от невыполнимой задачи воспринимать людей как набор кварков (или того, что в них имеется на низшем уровне), они дают нам только вероятностные оценки того, как другие люди чувствуют, реагируют на наши слова и поступки и так далее. Короче, используя блочную модель, мы приносим в жертву детерминизм и выигрываем в простоте. Несмотря на то, что мы не знаем, как люди среагируют на наш анекдот, мы все же рассказываем его; при этом мы скорее ожидаем, что они засмеются (или не засмеются), чем, скажем, полезут на ближайший столб. (Конечно, мастер дзена запросто мог бы сделать именно это!) Блочная модель определяет «интервал» возможного поведения и вероятность того, что определенное поведение будет лежать в той или иной области этого интервала.

«Компьютеры могут делать только то, что им приказано»

Эти идеи могут быть приложены не только к сложным физическим системам, но и к компьютерам. Известно высказывание: «Компьютеры могут делать только то, что им приказано». В каком-то смысле это верно, но при этом не учитывается следующий факт: последствия ваших инструкций неизвестны вам заранее, поэтому поведение компьютера может быть так же удивительно и непредсказуемо для вас, как и поведение человека. Обычно вам заранее известен тот приблизительный \emph{промежуток} , в рамки которого уложится результат, но неизвестны детали того, где именно этот результат будет расположен. Например, вы можете написать программу для того, чтобы вычислить первый миллион цифр числа \&\#960;. Ваша программа начнет выдавать эти цифры гораздо быстрее, чем могли бы это сделать вы сами --- но то, что компьютер обгоняет программиста, не удивительно. Вы знаете заранее, в каком интервале будет лежать результат --- а именно, цифры от 0 до 9; то есть у вас имеется блочная модель поведения программы; если бы вы знали все остальное, вам не нужно было бы писать программу.

Это старое высказывание неверно и в другом смысле. Дело в том, что, программируя на языках все высших уровней, вы все с меньшей и меньшей точностью можете сказать, что именно вы приказываете компьютеру! Многие прослойки переводов могут отделять «передний конец» сложной программы от действительных команд на машинном языке. На уровне, на котором вы думаете и программируете, ваши высказывания могут быть более похожи на утверждения и предложения, чем на команды. При этом внутренняя «возня», вызванная вводом высказывания высшего уровня, обычно остается для вас невидима, так же, как, когда вы едите бутерброд, вы не думаете о пищеварительных процессах, которые при этом начинаются у вас внутри.

Так или иначе, мнение, что «компьютеры могут делать только то, что им приказано», впервые высказанное лэди Лавлэйс в ее знаменитых мемуарах, настолько распространено и так связано с мнением о том, что «компьютеры не могут думать», что мы вернемся к нему в следующих главах, когда сможем обсудить этот вопрос на более высоком уровне.

Два типа системы

Системы, построенные из многих частей, бывают двух типов. Первый их них характеризуется тем, что поведение одних частей аннулирует поведение других. В подобных системах не столь важно, что делается на низшем уровне, поскольку результатом любых происходящих там событий будет почти одинаковое поведение высшего уровня. Примером такой системы может служить баллон с газом, молекулы которого сталкиваются друг с другом в результате множества сложных микроскопических процессов; однако макроскопическое целое --- это стабильная система в спокойном состоянии, в которой определены температура, давление и объем. В системах второго типа микроскопические изменения на низшем уровне могут возрасти до такой степени, что в результате заметно изменится макроскопический уровень. Примером такой системы является сборочный конвейер. Если один из сборщиков ошибется, с конвейера сойдет бракованная деталь.

Компьютер --- это сложная комбинация систем обоих типов. Его провода представляют собой предсказуемую систему: они проводят электричество в соответствии с законом Ома. Этот весьма точный закон похож на законы, описывающие поведение газа в баллоне, поскольку он зависит от статистических эффектов: хаотическое поведение биллионов частиц дает в результате предсказуемое общее поведение системы. Компьютер также содержит макроскопические части, такие как печатающее устройство, чье поведение задается определенными электрическими импульсами. То, что печатает это устройство, ни в коей мере не является результатом мириад взаимоуничтожающих микроскопических эффектов. В большинстве компьютерных программ значение каждого бита играет важную роль в том, что напечатает компьютер. От изменения любого бита информации значительно изменяется и конечный результат.

Системы, состоящие только из «надежных» подсистем, --- то есть таких подсистем, чье поведение может быть с уверенностью предсказано на основании описания их частей, --- играют важнейшую роль в нашей повседневной жизни, поскольку они являются оплотом стабильности. Мы можем быть уверены, что стены не упадут нам на голову, что тротуар окажется сегодня там же, где вчера, что солнце не исчезнет с небосвода, что часы показывают правильное время и так далее. Блочные модели подобных систем практически полностью детерминисткие. Разумеется, другой тип системы, играющей важную роль в нашей жизни, это система, чье поведение варьируется в зависимости от внутренних микроскопических параметров, --- зачастую огромного множества таких параметров, --- которые не поддаются прямому наблюдению. Наша блочная модель подобной системы будет выражаться в терминах некоего «пространства» ее действия и будет включать вероятностные оценки того, в каком месте этого пространства «приземлится» система в данный момент.

Баллон с газом, который, как я уже сказал, является надежной системой в результате множества взаимоуничтожающих микроскопических эффектов, подчиняется точным, детерминистким законам физики. Это \emph{блочные законы} , поскольку они рассматривают газ как единое целое, игнорируя его составляющие части. Более того, микроскопическое и макроскопическое описания газа используют совершенно разные термины. Первое требует определения положения и скорости каждой из молекул газа; второе требует определения только трех новых величин температуры, давления и объема. Две первые величины вообще не имеют соответствия на микроскопическом уровне. Математическое соотношение этих трех величин, выраженное в следующем простом уравнении: pV=cT, где~\emph{с} --- постоянная, --- это закон, который одновременно зависит и не зависит от событий на низшем уровне. Если говорить менее парадоксально, этот закон может быть выведен из законов, управляющих молекулярным уровнем, в этом смысле он зависит от низшего уровня. С другой стороны, этот закон позволяет, при желании, полностью игнорировать низший уровень; в этом смысле он от него не зависит.

Важно иметь в виду, что закон высшего уровня не может быть выражен в терминах низших уровней. «Давление» и «температура» --- новые термины, которые не могут быть поняты только на основании низшего уровня. Мы, люди, прямо воспринимаем температуру и давление, поскольку мы так устроены, не удивительно, что мы открыли этот закон. Но существа, которые воспринимали бы газы как абстрактные математические конструкции, должны были бы обладать умением выводить новые понятия, чтобы открыть подобный закон.

Эпифеномены

В завершение этой главы я хотел бы рассказать забавную историю о сложных системах. Однажды я беседовал с двумя программистами, работавшими с операционной системой компьютера, который я использовал. Они сказали, что она запросто справляется со своей задачей, когда к ней подключено менее тридцати пяти человек; но когда это число достигает тридцати пяти, время ответа внезапно замедляется настолько, что с таким же успехом можно отключиться от системы, пойти домой и вернуться попозже. Шутя, я сказал: «Эту проблему решить ничего не стоит --- для этого нужно только отыскать то место в операционной системе, где записано число „35``, и поменять его на „60``!» Все рассмеялись. Дело, разумеется, в том, что такого места просто не существует. Откуда же, в таком случае, появляется это критическое число --- 35 пользователей? \emph{Это видимое следствие общей организации системы --- так называемый~«эпифеномен».}

Так же вы можете спросить о бегуне. «Где в нем содержится число „10``, позволяющее ему пробегать 100 метров за 10 секунд?» Ясно, что оно не содержится ни в каком специальном месте. Время, которое бегун показывает на стометровке, --- результат его физического состоянии, быстроты его реакций, и миллиона других факторов, взаимодействующих между собой, когда он бежит. Это время вполне воспроизводимо, но оно не записано нигде в его теле. Оно распределено по всем клеткам его тела и проявляется только во время бега.

\emph{Рис. 60. Картина «МУ» (Рисунок автора. Русский графический вариант выполнен переводчиком.)}





\end{document}
