\documentclass[a4paper,12pt,oneside,openany]{memoir}
\usepackage{sty/preamble}

%% Frame around the text area
% \usepackage{showframe}
% \renewcommand\ShowFrameLinethickness{0.1pt}
% \renewcommand*\ShowFrameColor{\color{lightgray}}

\setcounter{tocdepth}{1}
\setcounter{secnumdepth}{1}


\title{ГЁДЕЛЬ, ЭШЕР, БАХ: \\ эта бесконечная гирлянда}
\author{Дуглас Ричард Хофштадтер}


\begin{document}

\frontmatter

\maketitle

\mainmatter

% Праздничное предисловие автора к русскому изданию книги «Гёдель, Эшер Бах»
\subfileinclude{parts/preface}

\begingroup
\hypersetup{hidelinks}
\clearpage
% \begin{KeepFromToc}
\tableofcontents
% \end{KeepFromToc}
\clearpage
\endgroup

% Обзор
\subfileinclude{parts/overview}
% Список иллюстраций
\subfileinclude{parts/illustrations}
% Благодарность
\subfileinclude{parts/acknowledgment}

\part{ГЭБ}

% Интродукция: музыко-логическое приношение
\subfileinclude{parts/introduction}
%
% Трехголосная инвенция
\subfileinclude{parts/dial01}
% Глава I: Головоломка MU
\subfileinclude{parts/ch01}
%
% Двухголосная инвенция
\subfileinclude{parts/dial02}
% Глава II: Содержание и форма в математике
\subfileinclude{parts/ch02}
%
% Соната для Ахилла соло
\subfileinclude{parts/dial03}
% Глава III: Рисунок и фон
\subfileinclude{parts/ch03}
%
% Акростиконтрапунктус
\subfileinclude{parts/dial04}
% Глава IV: Непротиворечивость, полнота и геометрия
\subfileinclude{parts/ch04}
%
% Маленький гармонический лабиринт
\subfileinclude{parts/dial05}
% Глава V: Рекурсивные структуры и процессы
\subfileinclude{parts/ch05}
%
% Канон с интервальным увеличением
\subfileinclude{parts/dial06}
% Глава VI: Местонахождение значения
\subfileinclude{parts/ch06}
%
% Хроматическая фантазия и фига
\subfileinclude{parts/dial07}
% Глава VII: Исчисление Высказываний
\subfileinclude{parts/ch07}
%
% Крабий канон
\subfileinclude{parts/dial08}
% Глава VIII: Типографская теория чисел
\subfileinclude{parts/ch08}
%
% Приношение «МУ»
\subfileinclude{parts/dial09}
% Глава IX: Мумон и Гёдель
\subfileinclude{parts/ch09}

\part{ЭГБ}

% Прелюдия и...
\subfileinclude{parts/dial10}
% Глава X: Уровни описания и компьютерные системы
\subfileinclude{parts/ch10}
%
% ...и Муравьиная фуга
\subfileinclude{parts/dial11}
% Глава XI: Мозг и мысль
\subfileinclude{parts/ch11}
%
% Англо-франко-германо-русская сюита
\subfileinclude{parts/dial12}
% Глава XII: Разум и мысль
\subfileinclude{parts/ch12}
%
% Ария с разнообразными вариациями
\subfileinclude{parts/dial13}
% Глава XIII: Блуп, Флуп и Глуп
\subfileinclude{parts/ch13}
%
% Ария в ключе G
\subfileinclude{parts/dial14}
% Глава XIV: О формально неразрешимых суждениях ТТЧ и родственных систем
\subfileinclude{parts/ch14}
%
% Праздничная Кантататата...
\subfileinclude{parts/dial15}
% Глава XV: Прыжок из системы
\subfileinclude{parts/ch15}
%
% Благочестивые размышления курильщика табака
\subfileinclude{parts/dial16}
% Глава XVI: Авто-реф и авто-реп
\subfileinclude{parts/ch16}
%
% Магнификраб в пирожоре
\subfileinclude{parts/dial17}
% Глава XVII: Чёрч, Тюринг, Тарский и другие
\subfileinclude{parts/ch17}
%
% ШРДЛУ
\subfileinclude{parts/dial18}
% Глава XVIII: Искусственный интеллект: взгляд в прошлое
\subfileinclude{parts/ch18}
%
% Контрафактус
\subfileinclude{parts/dial19}
% Глава XIX: Искусственный Интеллект: виды на будущее
\subfileinclude{parts/ch19}
%
% Канон Ленивца
\subfileinclude{parts/dial20}
% Глава XX: Странные Петли или Запутанные Иерархии
\subfileinclude{parts/ch20}
%
% Шестиголосый Ричеркар
\subfileinclude{parts/dial21}

\end{document}
