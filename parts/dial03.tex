\documentclass[../main.tex]{subfiles}
\begin{document}

\section{Соната для Ахилла соло}

\centerblock{%
    \emph{Звонит телефон \--- Ахилл берет трубку.}
}

\begin{dialogue}

\speak{Ахилл} Алло, Ахилл слушает.

\speak{Ахилл} А, здравствуйте, г-жа Черепаха. Как дела?

\speak{Ахилл} Кривошея и чихиллит? Что такое чихи\ldots \--- а, теперь понимаю. Будьте здоровы!\ldots{} Что и говорить, неприятная комбинация. Как это вы ухитрились такое подцепить?

\speak{Ахилл} И долго вы её так продержали?

\speak{Ахилл} Еще на самом сквозняке \--- не удивительно, что вам в шею надуло!

\speak{Ахилл} Что же вас заставило так долго там проторчать?

\speak{Ахилл} Многие из них удивительные? Какие, например?

\speak{Ахилл} Фантасмагорические чудища? Что вы имеете в виду?

\speak{Ахилл} И вам не страшно было в такой компании?

\speak{Ахилл} Гитара? Вот странно \--- откуда взялась гитара среди этих диковинных созданий. Кстати, вы играете на гитаре?

\speak{Ахилл} Ах, для меня это одно и то же.

\speak{Ахилл} Вы правы удивительно, как это я сам до сих пор не заметил, в чем разница между гитарой и скрипкой. Кстати о скрипках: не хотите ли вы заглянуть ко мне и послушать сонату для скрипки соло вашего любимого композитора, И.С.~Баха? Я только что купил отличную запись. Поразительно, как это Баху удалось, используя одну-единственную скрипку, создать такую интересную вещь.

\speak{Ахилл} Головная боль тоже? Бедняжка\ldots{} Пожалуй, вам лучше лечь в постель и постараться заснуть.

\speak{Ахилл} Понятно. Овец считать уже пробовали? Где-то у меня была целая картотека подобных трюков \--- говорят, они здорово помогают от бессоницы.

\speak{Ахилл} Ах, да. Я отлично понимаю, что вы имеете в виду \--- я это тоже пробовал. Может быть, если уж эта задачка так застряла у вас в голове, вы поделитесь ею со мной, чтоб и я мог попробовать свои силы?

\speak{Ахилл} Слово, внутри которого встречаются подряд буквы «Р», «Т», «О», «Т»,~«Е»\ldots{} Г-м-м\ldots{} Как насчет «ретотра»?

\speak{Ахилл} Ах, какой стыд\ldots{} Конечно вы правы \--- я опять все перепутал. К тому же в слове «реторта» эти буквы все равно идут задом наперед.

\speak{Ахилл} Уже несколько часов? Хорошенькую вы мне задали задачку\ldots{} Где вы откопали такую дьявольскую головоломку?

\speak{Ахилл} Вы имеете в виду, что он только делал вид, что размышляет над эзотерическими буддистскими проблемами, когда на самом деле он пытался придумать сложные словесные головоломки?

\speak{Ахилл} Ага! Улитка знала, чем он занимается. Как же вам удалось с ней переговорить?

\speak{Ахилл} Вы знаете, я как-то слышал похожую головоломку. Хотите, я вам её задам? Или это ещё хуже вас отвлечет?

\speak{Ахилл} Согласен \--- хуже уже вряд ли будет. Так вот: какое слово начинается с «КА» и кончается на «КА»?

\speak{Ахилл} Очень остроумно \--- но это нечестно. Я совершенно не это имел в виду!

\speak{Ахилл} Согласен, это слово выполняет условие; но все равно это какое-то дегенеративное решение.

\speak{Ахилл} Абсолютно верно! Как вам удалось так быстро найти ответ?

\speak{Ахилл} Это \--- ещё один пример того, какой полезной может оказаться картотека трюков от бессоницы. Прекрасно! Но я все ещё блуждаю в потемках с вашей задачкой о «PTOTE».

\speak{Ахилл} Поздравляю \--- теперь вам, может быть, удастся заснуть. Скажите же мне решение!

\speak{Ахилл} Вообще-то я не люблю подсказок, но на этот раз ладно, валяйте.

\speak{Ахилл} Не понимаю. Что вы имеете в виду под «рисунком» и «фоном»?

\adjustimage{
    max width=\textwidth,
    max totalheight=\textheight-\baselineskip-\abovecaptionskip-\belowcaptionskip-3pt,
    center,
    caption={М.К.~Эшер. «Мозаика~II»},
    label={fig:escher-mosaic},
    figure=!b,
}{img/escher-mosaic-ii.png}

\speak{Ахилл} Разумеется, я знаком с «Мозаикой~II». Я знаю ВСЕ работы Эшера. В конце концов, это мой любимый художник! Кстати, репродукция «Мозаики II» висит прямо у меня перед носом.

\speak{Ахилл} Всех черных зверей? Конечно, вижу!

\speak{Ахилл} Верно: их «негативное пространство» \--- то, что остается свободным \--- определяет белых зверей.

\speak{Ахилл} А, так вот что вы называете «рисунком» и «фоном»! Но какое отношение это имеет к головоломке о «Р-Т-О-Т-Е»?

\speak{Ахилл} Это для меня слишком сложно\ldots{} Теперь и у меня начинает болеть голова; пойду, пожалуй, поищу мою спасительную картотеку, может быть она мне поможет забыться сном.

\speak{Ахилл} Вы хотите зайти сейчас? Но я думал\ldots{}

\speak{Ахилл} Ну что ж, хорошо. А я пока постараюсь решить эту задачку с помощью вашей подсказки о рисунке и фоне и моей головоломки.

\speak{Ахилл} С удовольствием сыграю их для вас.

\speak{Ахилл} Вы изобрели о них теорию?

\speak{Ахилл} В сопровождении какого инструмента?

\speak{Ахилл} В таком случае, как странно, что он не записал также и партию клавесина, и не опубликовал их в таком виде.

\speak{Ахилл} А, понимаю \--- нам предоставляется выбор: слушать её с аккомпанементом или без оного. Но откуда мы знаем, как он должен звучать?

\speak{Ахилл} Да, вы правы \--- наверное, лучше всего оставить эту работу воображению слушателя. Согласен \--- может быть, у Баха в мыслях вообще не было никакого аккомпанемента. Действительно, эти сонаты и так звучат замечательно.

\speak{Ахилл} Точно. Ну, до скорого.

\speak{Ахилл} Пока, г-жа Ч.

\end{dialogue}

\end{document}
