\documentclass[../main.tex]{subfiles}
\begin{document}

\DialogueChapter{ШРДЛУ}

\centerblock{
    \emph{Однажды Ета Ойн заходит в лабораторию искусственного интеллекта Массачусетского Института Технологии, где она встречается с блестящей молодой программой по имени ШРДЛУ. Оказывается, что ШРДЛУ сгорает от желания найти кого нибудь кто согласился бы опробовать недавно созданное человеческое существо под названием «д-р Тире-Рвинога». ШРДЛУ объясняет, что д-р Тире-Рвинога довольно сообразителен в ограниченной области анализа бесед об игрушечной мире --- мире, состоящем из кубиков разных размеров, форм и цветов. Эти кубики лежат на столе, и их можно брать и перекладывать с места на место. Ета Ойн, заинтересовавшись, начинает печатать команды для ШРДЛУ, в то время как д-р Тире-Рвинога, стоя за ее спиной, комментирует происходящее, как и было обещано.}\footnote{Этот Диалог взят из статьи Terry Winograd. «A Procedural Model of Language Understanding» в сборнике R Schank and К Colby, eds «Computer Models of Thought and Language», стр. 155-66. Изменены только имена собеседников.}
}

\begin{dialogue}

% TODO: fix
\EmptyItem

1. Ета Ойн: Возьми большой красный кубик (См. рис. 110)

ШРДЛУ: ХОРОШО

\emph{Д-р Тире-Рвинога: ШРДЛУ отвечает «хорошо», когда она выполняет команду. Чтобы взять красный кубик, ей пришлось его освободить, найдя место для зеленого кубика и положив его туда.}

2. Ета Ойн: Возьми эту пирамиду.

ШРДЛУ: Я НЕ ПОНИМАЮ, КОТОРУЮ ПИРАМИДУ ВЫ ИМЕЕТЕ В ВИДУ.

\emph{Д-р Тире-Рвинога: На столе стоят три пирамиды, а ШРДЛУ знает, что предложения со словом «этот» относятся к определенному предмету, который говорящий имеет в виду.}

\emph{Рис. 110. «Возьми большой красный кубик» (Terri Winograd «Understanding Natural Language», cтp. 8)}

3. Eтa Ойн: найди кубик выше, чем тот, что ты держишь, и положи его в коробку. (См. рис. 111.)

\emph{Д-р Тире-Рвинога: ШРДЛУ должна сообразить, к чему относятся слова «тот» и «его», а также путем умозаключений вывести, какой именно кубик она держит.}

ШРДЛУ: ПОД «ЕГО» ВЫ, ВЕРОЯТНО, ИМЕЕТЕ В ВИДУ КУБИК, КОТОРЫЙ ВЫШЕ, ЧЕМ ТОТ, ЧТО Я ДЕРЖУ.

\emph{Д-р Тире-Рвинога: «Его» могло относиться к одному из двух кубиков: к выбранному кубику или же к кубику, который держит ШРДЛУ. В ШРДЛУ запрограммированны методы для выбора между ними.}

ШРДЛУ: ХОРОШО.

\emph{Д-р Тире-Рвинога: ШРДЛУ приходится положить красный кубик, чтобы взять более высокий голубой кубик.}

\emph{Рис. 111. «Найди кубик выше, чем тот, что ты держишь, и положи его в коробку». (Terri Winograd, «Understanding Natural Language», стр. 9)}

4. Ета Ойн: Что находится в коробке?

ШРДЛУ: ГОЛУБАЯ ПИРАМИДА И ГОЛУБОЙ КУБ.

5. Ета Ойн: На что опирается эта пирамида?

\emph{Д-р Тире-Рвинога: Раньше ШРДЛУ ответила на слова «эта пирамида», сказав «Я НЕ ПОНИМАЮ, КОТОРУЮ ПИРАМИДУ ВЫ ИМЕЕТЕ В ВИДУ.» Теперь она подразумевает, что вы имеете в виду только что упомянутую пирамиду.}

ШРДЛУ: НА КОРОБКУ.

6. Ета Ойн: Сколько кубиков находится не в коробке?

\emph{Д-р Тире-Рвинога: ШРДЛУ понимает стандартные отрицания, такие как «нет» и «не», в каком бы контексте они не появлялись.}

ШРДЛУ: ЧЕТЫРЕ.

7. Ета Ойн: Правда ли, что по крайней мере один из них уже, чем тот, который я попросила тебя взять?

\emph{Д-р Тире-Рвинога: Выражение «по крайней мере один из них» должно быть проанализировано по отношению к предыдущему вопросу; «тот, который» должно также относиться к кубикам. Группы существительных могут иметь сложное число, как например «по крайней мере, один», «больше трех», «ровно семь» и так далее. Память о прошлых событиях используется, чтобы понять выражение «тот, который я попросила тебя взять». Обратите внимание на сложную синтаксическую структуру, в которой слово «тот» является дополнением придаточного предложения, находящегося двумя уровнями глубже. Программа синтаксического анализа ШРДЛУ включает грамматику, обобщенно решающую такие проблемы.}

ШРДЛУ: ДА. КРАСНЫЙ КУБ.

8. Ета Ойн: Он на что-то опирается?

ШРДЛУ: ДА. НА СТОЛ.

\emph{Д-р Тире-Рвинога: ШРДЛУ умеет определять, какая информация важна для собеседника, даже если тот не спрашивает об этом прямо.}

9. Ета Ойн: Может ли стол брать кубики?

ШРДЛУ: НЕТ.

\emph{Д-р Тире-Рвинога: Семантические правила ШРДЛУ включают знания о том, что «брать» что-либо может только некто одушевленный, и что стол --- предмет неодушевленный.}

10. Ета Ойн: Может ли пирамида стоять на кубике?

ШРДЛУ: ДА.

\emph{Д-р Тире-Рвинога: Дедуктивная система ШРДЛУ находит пример подобного положения, так что она знает, что это возможно.}

11. Ета Ойн: Может ли пирамида стоять на пирамиде?

ШРДЛУ: Я НЕ ЗНАЮ.

\emph{Д-р Тире-Рвинога: Этот вопрос имеет смысл, но такого примера перед ШРДЛУ нет. Она не может рассмотреть детали своих программ, несмотря на то, что от них в конечном счете зависят ее возможности.}

12. Ета Ойн: Положи одну пирамиду на другую.

ШРДЛУ: Я НЕ МОГУ.

\emph{Д-р Тире-Рвинога: ШРДЛУ попыталась это сделать, но у нее ничего не получилось.}

13. Ета Ойн: Голубая пирамида моя.

ШРДЛУ: Я ПОНИМАЮ.

\emph{Д-р Тире-Рвинога: ШРДЛУ заранее известна вся информация о цвете, форме, размере и расположении этих объектов. О принадлежности этих предметов ей надо сообщать. Она сохраняет эту информацию в своей базе данных.}

14. Ета Ойн: Мне принадлежат все не красные кубики, но мне не принадлежит ничего, что поддерживало бы пирамиду.

ШРДЛУ: Я ПОНИМАЮ.

\emph{Д-р Тире-Рвинога: Результатом понимания этих предложений является создание новых процедур в ШРДЛУ, которые выясняют, что вам принадлежит.}

15. Ета Ойн: Принадлежит ли мне эта коробка?

ШРДЛУ: НЕТ.

\emph{Д-р Тире-Рвинога: На коробку опирается все, что в ней находится, включая пирамиду.}

16. Ета Ойн: Принадлежит ли мне что-нибудь из находящегося в коробке?

ШРДЛУ: ДА. ДВЕ ВЕЩИ: ГОЛУБОЙ КУБИК И ГОЛУБАЯ ПИРАМИДА.

17. Ета Ойн: He поставишь ли ты друг на друга оба красных кубика и зеленый куб или пирамиду? (См. рис. 112)

ШРДЛУ: ХОРОШО.

\emph{Д-р Тире-Рвинога: Логические связки такие как~«и», «или» и так далее, понимаются как на уровне грамматики, так и на уровне семантики. Команда выполняется, даже если она выглядит как вопрос. Эта команда выполняется в несколько шагов. ШРДЛУ кладет зеленый куб на большой красный кубик (при этом она выбирает тот зеленый куб, на котором ничего не стоит), затем убирает маленькую пирамиду с маленького красного куба, чтобы иметь возможность использовать его для окончания постройки требуемой башни.}

\emph{Рис. 112. «Не поставишь ли ты друг на друга оба красных кубика и зеленый куб или пирамиду?» (Terri Winograd, «Understanding Natural Language», стр. 12)}

18. Ета Ойн: Какой куб лежит на столе?

ШРДЛУ: БОЛЬШОЙ ЗЕЛЕНЫЙ КУБ НА КОТОРОМ СТОИТ КРАСНАЯ ПИРАМИДА.

\emph{Д-р Тире-Рвинога: Обычно ШРДЛУ удовлетворилась бы фразой «БОЛЬШОЙ ЗЕЛЕНЫЙ КУБ», но поскольку вы спросили «какой куб», она должна быть более точной, называя его. В ней имеются также процедуры для опускания существительных или замены их место имениями, чтобы сделать ее речь более естественной.}

19. Ета Ойн: Есть ли большой кубик за пирамидой?

ШРДЛУ: ДА. ТРИ ИЗ НИХ. БОЛЬШОЙ КРАСНЫЙ, БОЛЬШОЙ ЗЕЛЕНЫЙ и БОЛЬШОЙ ГОЛУБОЙ.

20. Ета Ойн: Положи маленький на зеленый кубик, на котором стоит пирамида.

\emph{Д-р Тире-Рвинога:«Маленький» должно соотноситься с вашей предыдущей фразой «большой кубик», так что это значит «маленький кубик».}

ШРДЛУ: ХОРОШО.

21. Ета Ойн: Поставь на него наименьшую пирамиду.

\emph{Д-р Тире-Рвинога: Слова типа «наименьший» отсутствуют в словаре ШРДЛУ, но она выводит их значение из корней других слов, как, например, меньше. Чтобы исполнить эту команду, ШРДЛУ сравнивает размер всех пирамид.}

ШРДЛУ: ХОРОШО.

22. Ета Ойн: Сколько вещей стоит на зеленых кубах?

ШРДЛУ: Я НЕ СОВСЕМ ПОНИМАЮ, ЧТО ВЫ ИМЕЕТЕ В ВИДУ ПОД «НА» ВО ФРАЗЕ «НА ЗЕЛЕНЫХ КУБАХ». ВЫ ХОТИТЕ СКАЗАТЬ:

1~--- ПРЯМО НА ПОВЕРХНОСТИ,

2 --- ГДЕ ЛИБО НА НИХ?

\emph{Д-р Тире-Рвинога: В предыдущем предложении это выражение было интерпретировано без проблем, так как его смысл был ясен из контекста. В этом случае эвристика для ликвидации двусмысленности не находит четкого решения и ШРДЛУ приходится просить у вас дополнительную информацию. При этом она использует перифразы словарных определений слова «на». Подобного разъяснения она может попросить также тогда, когда ей неясно значение местоимений, как, например, «это» или «они».}

23. Eтa Ойн: Прикасалась ли ты к какой-нибудь еще пирамиде, прежде чем поставить зеленую на маленький куб?

ШРДЛУ: ДА, К ЗЕЛЕНОЙ.

\emph{Д-р Тире-Рвинога: ШРДЛУ «вспоминает», какое событие вы имеете в виду; она может соотносить события во времени. Для этого она анализирует глагольные времена и употребление наречий времени --- специальные процедуры имеют дело с неправильными глаголами.}

24. Ета Ойн: Когда ты ее брала?

ШРДЛУ: КОГДА Я СТРОИЛА БАШНЮ ИЗ КРАСНОГО КУБА И БОЛЬШИХ КУБИКОВ, КРАСНОГО И ЗЕЛЕНОГО.

\emph{Д-р Тире-Рвинога: Время здесь описано в терминах задачи, над которой ШРДЛУ работала в тот момент.}

25. Ета Ойн: Почему?

\emph{Д-р Тире-Рвинога: Чтобы понять подобные вопросы, ШРДЛУ вспоминает предыдущие предложения.}

ШРДЛУ: ЧТОБЫ ЕЕ УБРАТЬ.

\emph{Д-р Тире-Рвинога: Поскольку ШРДЛУ следит за первоначальным деревом подзадач, она в какой-то мере понимает мотивы собственных действий.}

26. Ета Ойн: Почему тебе надо было это сделать?

ШРДЛУ: ЧТОБЫ ОСВОБОДИТЬ КРАСНЫЙ КУБ.

\emph{Д-р Тире-Рвинога: «Это» относится к действию, которое ШРДЛУ описывала в своем последнем ответе.}

27. Ета Ойн: Зачем ты это освободила?

\emph{Д-р Тире-Рвинога: В данном случае «это» означает только что упомянутый куб.}

ШРДЛУ: ЧТОБЫ ПОСТАВИТЬ ЕГО НА БОЛЬШОЙ ЗЕЛЕНЫЙ КУБ.

28. Ета Ойн: Зачем ты это сделала?

ШРДЛУ: ЧТОБЫ ПОСТАВИТЬ ДРУГ НА ДРУГА КРАСНЫЙ КУБ, БОЛЬШОЙ ЗЕЛЕНЫЙ КУБИК И БОЛЬШОЙ ЗЕЛЕНЫЙ КУБ.

29. Ета Ойн: Зачем ты это сделала?

ШРДЛУ: ПОТОМУ ЧТО ТЫ МЕНЯ ПОПРОСИЛА.

\emph{Д-р Тире-Рвинога: ШРДЛУ помнит, что она делала, и как была выражена команда.}

30. Ета Ойн: Как ты это сделала?

\emph{Д-р Тире-Рвинога: «Это» относится к действию, упомянутому в последнем вопросе, а не к ответу на него.}

ШРДЛУ: ПОЛОЖИВ БОЛЬШОЙ КРАСНЫЙ КУБИК НА СТОЛ; ЗАТЕМ ОТПУСТИВ ЕГО; ЗАТЕМ ПОЛОЖИВ НА НЕГО БОЛЬШОЙ ЗЕЛЕНЫЙ КУБ, ЗАТЕМ ОТПУСТИВ ЭТОТ КУБ, ЗАТЕМ ПОЛОЖИВ КРАСНЫЙ КУБ НА ЭТОТ КУБ; ЗАТЕМ ОТПУСТИВ ЭТОТ КУБ.

\emph{Д-р Тире-Рвинога: Это описание подзадач высшего уровня, использованных для достижения основной задачи.}

31. Eтаa Ойн: К скольким предметам ты притронулась, пока это делала?

\emph{Д-р Тире-Рвинога: Отрезок времени может быть определен с помощью таких слов как «пока».}

ШРДЛУ: К ЧЕТЫРЕМ.

32. Ета Ойн: Что стояло на красном кубе до того, как ты начала его освобождать?

ШРДЛУ: ЗЕЛЕНАЯ ПИРАМИДА.

\emph{Д-р Тире-Рвинога: Вспоминая, что она сделала, ШРДЛУ может делать выводы о предыдущей ситуации. Вы можете использовать сложные структуры с несколькими придаточными предложениями типа~«до того, как ты начала его освобождать».}

33. Ета Ойн: Слева от коробки находились тогда пять кубиков.

\emph{Д-р Тире-Рвинога: «Тогда» должно относиться ко времени, упомянутом в последней фразе.}

ШРДЛУ: НЕТ, ТОЛЬКО ЧЕТЫРЕ КРАСНЫЙ КУБ, ДВА БОЛЬШИХ ЗЕЛЕНЫХ КУБА И БОЛЬШОЙ КРАСНЫЙ КУБИК.

\emph{Д-р Тире-Рвинога: Раньше в разговоре информация о «принадлежности» была принята за чистую монету. Но ШРДЛУ знает, что в ее распоряжении --- полная информация об основных свойствах предметов. Она проверяет высказывания о предметах и может с ними не согласиться.}

34. Ета Ойн: Расположи голубую пирамиду на кубике в коробке.

\emph{Д-р Тире-Рвинога: Это предложение синтаксически двусмысленно, но оно понято без труда, поскольку разделение, соответствующее значению «расположи голубую пирамиду, которая стоит на кубике, в коробке» здесь бессмысленно, в то время как «расположи голубую пирамиду на кубике, который стоит в коробке» имеет смысл.}

ШРДЛУ: ХОРОШО.

35. Ета Ойн: Есть ли здесь что-нибудь что больше любой пирамиды, но не так широко, как то на чем оно стоит?

\emph{Д-р Тире-Рвинога: ШРДЛУ понимает такие слова как «каждый», «никакой»,~«что-нибудь» и так далее и выражения типа «больше, чем», «такой же широкий, как» и тому подобное. Здесь местоимение «оно» относится к группе существительных в том же предложении, но в программе синтаксического анализа оно находится двумя уровнями выше.}

ШРДЛУ: ДА ГОЛУБОЙ КУБИК.

36. Ета Ойн: Расположена ли колокольня.

ШРДЛУ: ПРОСТИТЕ, Я НЕ ЗНАЮ СЛОВА «КОЛОКОЛЬНЯ».

\emph{Д-р Тире-Рвинога: ШРДЛУ немедленно перебивает, если она видит не знакомое слово.}

37. Ета Ойн: «Колокольня» это башня, построенная из двух зеленых кубов и пирамиды.

\emph{Д-р Тире-Рвинога: Кавычки означают, что вы определяете новое слово. Определение добавляется к словарю ШРДЛУ в форме, совместимой с тем, что она уже знает.}

ШРДЛУ: Я ПОНИМАЮ.

38. Ета Ойн: Есть ли сейчас здесь колокольни?

ШРДЛУ; НЕТ.

\emph{Д-р Тире-Рвинога: Новое определение используется как описание, чтобы отличать колокольни.}

39. Ета Ойн: Построй ее.

ШРДЛУ: ХОРОШО.

\emph{Д-р Тире-Рвинога: Описания, данного в определении, достаточно для того, чтобы выработать план построения новых объектов, и для того, чтобы их узнавать. Слово «ее» означает «колокольню», и план приводится в исполнение. Сначала освобождается один из больших зеленых кубов --- с него снимается красный куб и две пирамиды. Затем два зеленых куба и высокая красная пирамида ставятся друг на друга соответствующим описанию образом.}

40. Ета Ойн: Называй наибольший кубик «суперкубиком».

ШРДЛУ: ХОРОШО.

\emph{Д-р Тире-Рвинога: Можно давать предметам имена.}

41. Ета Ойн: Ты брала суперкубик с тех пор, как мы начали?

ШРДЛУ: ДА.

42. Ета Ойн: Почему ты его уронила?

ШРДЛУ: Я ЗАКЛЮЧАЮ, ЧТО ПОД «НИМ» ВЫ ИМЕЕТЕ В ВИДУ СУПЕРКУБИК. Я НЕ МОГУ ОБЪЯСНИТЬ НЕСЛУЧИВШЕЕСЯ СОБЫТИЕ.

\emph{Д-р Тире-Рвинога: В памяти ШРДЛУ нет записи об уроненном блоке. Рутинные процедуры ответа могут распознавать подобные «вопросы на засыпку», предполагающие ложную информацию.}

43. Ета Ойн: Есть ли что-нибудь слева от красной пирамиды?

ШРДЛУ: ДА. ЧЕТЫРЕ ПРЕДМЕТА: КОРОБКА, КРАСНЫЙ КУБ, ГОЛУБАЯ ПИРАМИДА И БЛОК, НАЗЫВАЕМЫЙ СУПЕРКУБИКОМ.

\emph{Д-р Тире-Рвинога: Когда предмет получает название, оно затем используется в ответах для идентификации этого предмета.}

44. Ета Ойн: Спасибо.

ШРДЛУ: ПОЖАЛУЙСТА!

\emph{Д-р Тире-Рвинога: Грамматика ШРДЛУ дает ей возможность распознавать идиоматические выражения.}

\end{dialogue}

\end{document}
