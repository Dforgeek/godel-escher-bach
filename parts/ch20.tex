\documentclass[../main.tex]{subfiles}
\begin{document}

\Chapter{Странные Петли или Запутанные Иерархии}

\subsection{Могут ли машины быть оригинальными?}

В ВОСЕМНАДЦАТОЙ ГЛАВЕ я описал удачную шашечную программу Артура Самуэля, которая была способна обыграть своего создателя. Интересно послушать, что говорит по поводу компьютеров и оригинальности сам автор этой программы. Следующие отрывки взяты из опровержения на статью Норберта Винера, написанного Самуэлем в 1960 году.

Я убежден в том. что машины не могут быть оригинальными в том смысле, какой вложил в это понятие Винер, когда он написал: «Машины могут превосходить некоторые ограничения их создателей и уже это делают. Благодаря этому, они могут быть одновременно эффективными и опасными.»\ldots{} Машина --- это не вызванный из бутылки джинн; она не работает с помощью магии. У нее нет собственной воли, и, вопреки тому, что говорит Винер, из нее не выходит ничего, что не было бы в нее заложено (исключая, разумеется, редкие случаи поломок).

«Намерения», которые выказывает машина, на самом деле являются намерениями человека-программиста, оговоренными заранее, или же вторичными намерениями, выведенными из первых согласно правилам, установленным тем же программистом. Мы можем даже, подобно Винеру, предвидеть некий высший уровень абстракции, на котором программа не только будет модифицировать вторичные намерения, но и правила, используемые для их деривации --- или же будет варьировать способы самой этой модификации\ldots{} и так далее. Можно даже вообразить, что машина сумеет построить другую машину с улучшенными возможностями. Однако --- и это важно помнить --- машина не может и не будет проделывать ничего подобного, пока в нее не введут соответствующие инструкции. Существует и логически должен всегда существовать пробел между (\emph{i} ) любым конечным продуктом разработки человеческих желаний и (\emph{ii} ) развитием в машине её собственной воли. Думать иначе --- значит либо верить в магию или считать, что человеческая свободная воля --- это иллюзия и что человеческие действия так же механистичны, как и действия машины. Может быть, и статья Винера и мое опровержение были механистически предопределены --- но я отказываюсь в это верить.\footnote{A. L. Samuel, «Some Moral and Technical Consequences of Automation --- A Refutation», Science 132 (сентябрь 16, 1960), стр. 741-2.}

Это напоминает мне Диалог Кэрролла («Двухголосная инвенция») и вот почему. Самуэль основывает свой аргумент против машинного сознания (или воли) на понятии, что \emph{любое механическое проявление воли с необходимостью потребует бесконечного регресса} . Таким же образом, Кэрроллова Черепаха пытается доказать, что даже простейшее рассуждение не может быть проведено без обращения к неким правилам высшего уровня, разрешающим данное рассуждение. Но поскольку это, в свою очередь, тоже шаг рассуждения, то приходится обращаться к правилам еще более высокого уровня --- и так далее. Заключение: \emph{рассуждения приводят к бесконечному регрессу} .

Разумеется, в доводах Черепахи есть ошибка, и я думаю, что аналогичная ошибка имеется и в рассуждениях Самуэля. Чтобы показать, в чем эти ошибки похожи, я сыграю роль «адвоката дьявола», пытаясь доказать точку зрения, противоположную моей собственной. (Поскольку, как известно, Бог помогает тем, кто помогает себе сам, то дьявол, вероятно, помогает тем и только тем, кто сам себе \emph{не} помогает. Помогает ли Дьявол сам себе?) Вот мои «дьявольские» заключения, выведенные из Диалога Кэрролла:

Заключение о том, что рассуждения невозможны, неприложимо к людям, поскольку всем известно, что людям всё-таки удается проводить множество рассуждений, несмотря на все высшие уровни. Это показывает, что люди функционируют, \emph{не нуждаясь в правилах} : Люди --- это пример «неформальной системы». С другой стороны, этот аргумент действителен, когда мы применяем его против \emph{механических} рассуждающих систем, поскольку они всегда подчиняются правилам. Такие системы не смогут начать работать, пока у них не будет мета-правил, указывающих им, когда применять правила, мета-мета-правил, говорящих, когда применять мета-правила, и так далее. Таким образом, мы можем заключить, что умение рассуждать не может быть воплощено в машине, --- это исключительно человеческая черта.

Где в этих доводах ошибка? В предположении, что \emph{машина не способна начать действовать без правила, говорящего ей, как это сделать} . В действительности, машины обходят глупые Черепахины возражения так же легко, как и люди, и по той же самой причине: как люди, так и машины сделаны из аппаратуры, которая действует сама по себе, согласно законам физики. Вовсе не надо опираться на «правила, говорящие, как использовать правила», поскольку правила \emph{низшего} уровня --- не имеющие никаких «мета» перед ними --- встроены в саму аппаратуру и действуют самостоятельно. Вывод: Диалог Кэрролла ничего не говорит нам о разнице между людьми и машинами. (На самом деле рассуждения \emph{возможно} механизировать.)

Теперь перейдем к доводам Самуэля. В карикатурном изображении, его точка зрения сводится к следующему:

Нельзя сказать, что компьютер «хочет» что-либо сделать, поскольку он был запрограммирован кем-то другим. Только в том случае, если бы он мог запрограммировать сам себя, начиная с нуля, мы могли бы сказать, что компьютер обладает собственной волей.

В своих доводах Самуэль встает на позицию Черепахи, заменяя «рассуждения» на «волю». Он хочет сказать, что за любым механизмом желания должен стоять либо бесконечный регресс, либо, что еще хуже, закрытая петля. Если у компьютеров нет собственной воли именно по этой причине, то что можно сказать о людях? Тот же самый критерий позволяет заключить, что:

человек обладает собственной волей только тогда, когда он сделал себя сам и выбрал собственные желания (а также выбрал выбор собственных желаний и так далее).

Это заставляет нас хорошенько подумать над тем. откуда появляются наши желания. Если вы не верите в наличие души, то. возможно, скажете, что они зарождаются в вашем мозгу --- аппаратуре, которую вы не создавали и не выбирали. Тем не менее, от этого ваше чувство, что вы желаете чего-то определенного, не становится слабее. Вы --- вовсе не «само-программирующий объект» (что бы это ни значило); тем не менее у вас всё же есть собственная воля, зарождающаяся на физическом уровне вашего интеллекта. Таким же образом, у машин когда-нибудь будет собственная воля, несмотря на тот факт, что никакая магическая «само-программирующая» программа не появляется в их памяти из ничего, словно по мановению волшебной палочки. У них будет воля по той же причине, что и у людей --- как следствие организации и структуры многих уровней аппаратуры и программного обеспечения. Вывод: доводы Самуэля ничего не говорят нам о разнице между людьми и машинами. (На самом деле, волю \emph{возможно} механизировать.)

Любая запутанная иерархия основана на неизменном уровне

Сразу после «Двухголосной инвенции» я написал, что центральной темой этой книги будет обсуждение вопроса «подчиняются ли слова и мысли формальным правилам?» Одной из моих основных задач было показать многоуровневость интеллекта и мозга и объяснить, почему конечным ответом на этот вопрос является «да, если спуститься на низший уровень --- уровень аппаратуры --- и найти там правила.»

Точка зрения Самуэля затронула некую тему, которую я хочу обсудить подробнее. Вот она: думая, мы, безусловно, меняем наши мысленные правила, а также правила, меняющие правила, и так далее --- но это, образно говоря, правила «программ». При этом \emph{фундаментальные} правила, правила «аппаратуры», остаются неизменными. Нейроны всегда действуют одинаково. Мы не можем уговорить нейроны повести себя «ненейронным» образом; все, что нам удается сделать, это поменять тему или стиль наших мыслей. Подобно Ахиллу в «Прелюдии» и «Муравьиной фуге», мы имеем доступ только к нашим мыслям, а не к нейронам. Правила программ могут варьироваться на разных уровнях --- правила аппаратуры всегда остаются одними и теми же. Именно этим фактом и объясняется гибкость программ! Это вовсе не парадокс, а фундаментальный, простой факт, касающийся механизмов разума.

Именно различие между само-модифицирующимися программами и неизменной аппаратурой будет темой последней главы этой книги. Некоторые из последующих вариаций на это тему могут показаться довольно надуманными; однако надеюсь, что к тому моменту, когда я завершу цикл, вернувшись к мозгу, интеллекту и чувству самосознания, вы сможете увидеть неизменную основу в каждой из этих вариаций.

В этой главе я хочу поделиться с читателем теми образами, которые помогают мне понять, каким образом сознание вырастает из джунглей нейронов. Надеюсь, что эти интуитивные образы окажутся полезными и немного помогут читателям в определении их собственных представлений о том, что заставляет функционировать разум. Может быть, возникающие в моем мозгу туманные образы мозга и образов послужат катализатором для образования более четких образов мозга и образов в мозгу моих читателей.

Модифицирующаяся игра

Итак, первая вариация: игры, в которых очередной игрок может изменять правила. Представьте себе шахматы. Ясно, что правила здесь остаются неизменными, а меняется только позиция на доске после каждого хода. Но давайте теперь рассмотрим такой вариант шахмат, в котором очередной игрок имеет право либо сделать ход, либо поменять правила. Каким образом? Произвольно? Можно ли превратить шахматы, скажем, в шашки? Понятно, что подобная анархия была бы бессмысленна --- должны существовать некоторые ограничения. Например, в одной из версий будет позволено изменять ход коня: вместо~«1 и затем 2» конь будет передвигаться на «\emph{m} » и затем «\emph{n} » клеток, где \emph{m} и \emph{n} --- любые натуральные числа; очередной игрок сможет увеличивать или уменьшать на 1 либо~\emph{m} либо \emph{n} . Таким образом, ход коня сможет меняться от 1-2 до 1-3, до 0-3, до 0-4, до 0-5. до 1-5, до 2-5\ldots{} Вместо этого могут существовать правила, модифицирующие ход слона и других фигур. Другие правила могут добавлять новые клетки к доске, или стирать старые\ldots{}

У нас будет два уровня правил, одни говорят нам, как ходят фигуры, и другие --- как изменяются правила. Таким образом, у нас есть правила и мета-правила. Следующий шаг очевиден: введение мета-мета-правил, говорящих нам, как менять мета-правила. Однако вовсе не очевидно, как именно это сделать. Правила, меняющие ходы фигур, придумать легко, поскольку фигуры двигаются в формализованном пространстве шахматной доски. Если бы нам удалось придумать простую формальную запись для правил и мета-правил, тогда обращаться с ними стало бы так же легко, как с цепочками формул или даже с шахматными фигурами. Доводя эту идею до логической крайности, мы могли бы представить, что правила и мета-правила могут быть изображены в виде позиций на вспомогательной шахматной доске. Тогда каждая позиция сможет, в зависимости от вашей интерпретации, быть понята как момент игры, набор правил или набор мета-правил. Разумеется, оба игрока должны будут заранее договориться о том, как интерпретировать нотацию.

В этой игре у нас может быть любое количество дополнительных досок: доска для игры, для правил, для мета-правил, для мета-мета-правил и так далее, пока нам не надоест. Очередной игрок может ходить на любой из этих досок, кроме доски самого высшего уровня. Правила при этом определены доской «ступенькой выше». Несомненно, оба игрока вскоре запутаются из-за того, что почти все --- но не всё! --- может меняться. По определению, доска высшего уровня должна оставаться неприкасаемой, поскольку у вас нет правил, говорящих вам, как её менять. Это --- \emph{неизменный} уровень. Неизменны также условия, по которым изменяются другие доски, соглашение играть по очереди, условие, что очередной игрок может менять что-то только на одной из досок --- вы найдете здесь и другие неизменные элементы, если рассмотрите эту идею более подробно.

Возможно пойти гораздо дальше, если убрать опорные ориентиры. Начнем действовать постепенно\ldots{} Сначала сведем весь набор досок к одной-единственной доске. Что это означает? Что эту доску можно будет интерпретировать двояко как (1) фигуры, которые надо двигать и (2) правила ходов. Игроки, двигающие фигуры, тем самым меняют правила! Таким образом, правила постоянно меняют сами себя. Здесь слышен отголосок типогенетики (и настоящей генетики!) Различие между игрой, правилами, мета-правилами и мета-мета-правилами оказывается стерто. То, что когда-то было четкой иерархической системой, превратилось в Странную Петлю или Запутанную Иерархию. Ходы меняют правила, правила определяют ходы --- и так далее, по кругу. Здесь все еще есть различные уровни, но разница между «высшими» и «низшими» уровнями уже исчезла.

При этом часть того, что раньше было неприкасаемым, стало возможно модифицировать. Но в системе все еще осталось множество неизменных вещей. Так же как и раньше, между вами и вашим противником существуют некие соглашения, при помощи которых вы интерпретируете доску как определенный набор правил, соглашение играть по очереди и другие негласные условия. Заметьте, что теперь понятие различных уровней изменилось довольно неожиданным образом. У нас есть Неизменный уровень --- давайте назовем его \emph{уровень Н} --- на котором находятся соглашения, касающиеся интерпретации, и Запутанный уровень --- \emph{уровень~З} --- на котором находится Запутанная Иерархия. Эти два уровня все еще иерархичны: уровень Н управляет тем, что происходит на уровне З, в то время как уровень~З не затрагивает и не может затронуть уровня Н. Несмотря на то, что сам уровень~З представляет из себя Запутанную Иерархию, он всё же подчиняется набору правил, находящихся за его пределами. Это очень важный момент.

Как вы, несомненно, уже предположили, ничто не мешает нам сделать «невозможное» --- а именно, соединить уровень Н с уровнем 3. Для этого надо только поставить сами условия интерпретации в зависимость от положения на шахматной доске. Однако для того, чтобы провести подобное «сверх-соединение», вам и вашему противнику придется выработать некие новые соглашения, соотносящие два уровня --- и это создаст новый неизменный уровень сверху «сверхсмешанного» (или под ним, если вам так больше нравится). И это может продолжаться до бесконечности. «Скачки», которые при этом совершаются, напоминают те, что были описаны в Диалоге «Праздничная Кантатата» и в повторной Гёделизации, примененной к разнообразным улучшенным вариантам ТТЧ\@. Каждый раз, когда вам кажется, что вы подошли к концу, возникает новый вариант выхода из системы; чтобы его заметить, нужно некоторое творческое воображение.

Снова авторский треугольник

Я не собираюсь здесь прослеживать эту странную тему усложняющихся комбинаций систем, которые могут возникнуть в само-изменяющихся шахматах. Моей целью было показать читателю графически, что в каждой системе есть некий «защищенный» уровень, на который не действуют правила других уровней, какими бы запутанными не были их взаимодействия между собой. Забавная загадка из главы IV иллюстрирует эту мысль в немного ином контексте. Может быть, она застанет вас врасплох:

\emph{Рис. 134. «Авторский треугольник».}

Перед нами три автора: З, Ч и Э.~З существует только в романе, написанном Ч. Аналогично, Ч --- только герой романа, написанного Э. Что удивительно, Э --- тоже не более как персонаж романа --- чей автор, естественно, З. Может ли существовать такой авторский треугольник?

Разумеется, может! Но для этого все трое должны быть персонажами четвертого романа, написанного X. Можно сказать, что З-Ч-Э представляет из себя Странную Петлю или Запутанную Иерархию, а автор X находится в неизменном пространстве, вне той системы, в которой происходит эта путаница. Хотя З, Ч и Э имеют прямой или косвенный доступ друг к другу и могут напакостить один другому в своих романах, ни один из них не может затронуть жизнь X. Они даже не могут вообразить его, так же, как вы не в состоянии представить себе автора того романа, который выдумал в качестве своего героя \emph{вас} . Если бы я хотел ввести в схему автора X, я нарисовал бы его вне страницы. Разумеется, это было бы проблематично, поскольку изображение предмета с необходимостью помещает его на странице\ldots{} Так или иначе, X в действительности находится вне мира, в котором обитают З, Ч и Э, и должен быть представлен соответствующим образом.

\emph{Рис. 135. М. К. Эшер. Рисующие руки, (литография, 1948).}

Эшеровы «Рисующие руки»

Другая классическая вариация на эту тему --- картина Эшера «Рисующие руки» (Рис. 135). Здесь левая рука (ЛР) рисует правую руку (ПР), в то время как ПР рисует ЛР. Снова уровни, обычно понимаемые как иерархические --- рисующее и рисуемое --- замыкаются друг на друга, создавая Запутанную Иерархию. Этот пример, разумеется, подтверждает идею данной главы, поскольку за ним стоит ненарисованная, но рисующая рука самого Эшера --- создателя как ЛР, так и ПР. Эшер стоит вне пространства этих рук, и это хорошо видно на рис. 136. В верхней части этого схематического варианта картины Эшера вы видите Странную Петлю или Запутанную Иерархию, а в нижней --- Неизменный уровень, позволяющий её существование. Мы могли бы еще раз «Эшеризировать» картину Эшера. сфотографировав рисующую её руку\ldots{} и так далее.

\emph{Рис. 136. Абстрактная диаграмма, представляющая картину Эшера «Рисующие руки». Внизу приведено её решение.}

Мозг и разум: переплетение нейронов, лежащее в основе переплетения символов

Теперь мы можем соотнести эту картину с мозгом, а также с программами ИИ\@. Когда мы думаем, символы в нашем мозгу активируют другие символы, и все они взаимодействуют гетерархически. Более того, символы могут заставить друг друга измениться внутренне и стать чем-то вроде программ, действующих на другие программы. Благодаря Запутанной Иерархии символов, у нас создается иллюзия, что \emph{неизменяемого уровня в мозгу не существует} . Мы думаем, что подобного уровня нет, потому что он для нас невидим.

Если бы было возможно изобразить это схематически, получился бы гигантский лес символов, соединенных друг с другом перепутанными линиями, вроде лиан в джунглях. Это --- высший уровень, где рождаются и развиваются мысли, тот ускользающий уровень \emph{разума} , который аналогичен рисующим друг друга рукам. Внизу на схеме помещалось бы изображение мириад нейронов --- «неизменного субстрата,» лежащего в основе переплетения символов и аналогичного «движущей силе» --- Эшеру. Интересно, что в буквальном смысле сам этот нижний уровень тоже представляет из себя переплетение: миллиарды клеток и сотни миллиардов аксонов, соединяющих клетки между собой.

В этом интересном случае сложное переплетение на уровне программ основано на переплетении на уровне самой аппаратуры --- нейронов. Но Запутанной Иерархией можно назвать лишь переплетение символов. Переплетение нейронов --- это «простое» переплетение. Это различие подобно разнице между Странными Петлями и обратной связью, которое я описал в главе XVI. Запутанная Иерархия получается тогда, когда строго иерархичные на первый взгляд уровни внезапно начинают действовать друг на друга в нарушение всех правил иерархии. Элемент неожиданности здесь очень важен; именно поэтому я называю Странные Петли «странными». Простое переплетение, такое, как обратная связь, не нарушает установленных различий между уровнями. Например, когда вы стоите под душем и моете правую руку левой рукой и наоборот, это в порядке вещей. Эшер не случайно решил нарисовать руки, рисующие руки!

События, подобные моющим друг друга рукам, случаются в мире очень часто, и мы их обычно не замечаем. Я говорю что-то вам, а вы в ответ говорите что-то мне. Парадокс? Вовсе нет; наше восприятие друг друга с самого начала не включает никакой иерархии, поэтому здесь нет ничего странного.

С другой стороны, в языке получаются странные петли тогда, когда он прямо или косвенно говорит сам о себе. При этом нечто, лежащее \emph{внутри} системы, выходит из нее и воздействует \emph{на} систему так, словно оно находится вовне. Возможно, что нас смущает некое неопределенное чувство топологической неправильности: стирание различия между внутренним и внешним, как в знаменитой «бутыли Клейна». Хотя система абстрактна, наш мозг создает для нее пространственный образ с некоторой мысленной топологией. Вернемся к путанице символов. Если глядеть только на нее и игнорировать нейронный фундамент, то в ней можно увидеть самопрограммирующий объект --- точно так же, как глядя на «Рисующие руки», мы видим саморисующую картину и на мгновение верим этой иллюзии, забывая об Эшере. В случае картины эта иллюзия рассеивается мгновенно --- но в случае человеческого разума она оказывается весьма стабильной. Мы \emph{чувствуем} , что мы самопрограммирующие. Более того, мы и не можем чувствовать иначе, поскольку мы защищены от низшего уровня, уровня нейронных сплетений. Нам кажется, что наши мысли живут в своем собственном пространстве, создавая новые мысли и изменяя старые; мы не замечаем помогающих этому нейронов! Но так и должно быть. Мы просто не можем их заметить.

Аналогичная двусмысленность может произойти с программами ЛИСПа, которые умеют действовать на самих себя, изменяя собственную структуру. Посмотрев на них на уровне ЛИСПА, вы можете сказать, что они меняют сами себя; но, сменив уровни и представив программы ЛИСПА как данные для интерпретатора ЛИСПа (см. главу X), вы увидите, что единственная работающая программа здесь --- интерпретатор и что все изменения --- не более как изменения неких данных. Сам интерпретатор ЛИСПа защищен от изменений.

То, каким образом вы описываете подобные запутанные ситуации, зависит от того, насколько далеко вы при этом отходите в сторону от системы. Глядя издалека, часто можно увидеть разгадку, позволяющую разобраться в путанице.

Странные Петли в правительстве

Интереснейшее поле, где перекрещиваются различные иерархии --- это правительство и, в особенности, суд. Обычно судящиеся стороны представляют свои аргументы, и судья решает, кто прав. Судья находится на более высоком уровне, чем судящиеся. Но когда в тяжбе оказываются замешаны сами суды, могут происходить странные вещи. Как правило, существует некий высший суд, находящийся вне данного дела. Скажем, если два районных суда начнут борьбу друг с другом, требуя справедливости, всегда найдется какая-нибудь высшая инстанция (в этом случае, областной суд), аналогичный неприкасаемому уровню интерпретации соглашений, который мы обсуждали в нашей вариации шахмат.

Но что произойдет, если замешанным в неприятности с законом окажется сам Верховный суд? Подобное чуть не случилось в Соединенных Штатах в период Уотергэйта. Президент пригрозил, что он подчинится только «окончательному» решению Верховного суда, --- а затем сказал, что никто, кроме него самого, не имеет права решать, что является «окончательным». Эта угроза так и не была приведена в исполнение; но если бы такое произошло, результатом было бы монументальное столкновение двух уровней правительства, каждый из которых с некоторым правом может считать себя «выше» другого, --- и кто должен был бы решать, кто здесь прав? Конгресс тут помочь не мог, поскольку, хотя Конгресс и может приказать Президенту подчиниться Верховному суду, Президент может отказаться, ссылаясь на свое право не подчиняться Верховному суду (и Конгрессу!) в определенных обстоятельствах. Это создало бы беспрецедентный тип судебной тяжбы, нарушающий всю систему, поскольку это было бы так неожиданно --- так Запутанно --- так Странно!

Ирония здесь в том, что когда ваша голова упирается в потолок и вы уже не можете выпрыгнуть из системы, обратившись к высшей инстанции, ваша единственная надежда заключается в силах, которые, не будучи так четко определены правилами, сами являются единственным источником правил высшего уровня. Я имею в виду правила низшего уровня --- в данном случае, общественное мнение. Необходимо помнить, что в американском обществе юридическая система в некотором смысле представляет из себя вежливый жест, одобренный миллионами людей, и что её можно иногда обойти с такой же легкостью, с какой река выходит из берегов. Результат на первый взгляд кажется анархией --- но у анархии не меньше правил, чем у любого цивилизованного общества. Разница только в том, что они действуют снизу вверх, а не сверху вниз. Те, кто интересуется анархией, могут попытаться найти правила, по которым развиваются анархические ситуации; скорее всего, подобные правила существуют.

Здесь уместна аналогия из области физики. В этой книге я уже упоминал о том, что газы, находящиеся в равновесии, повинуются простым законам, соотносящим их температуру, давление и объем. Однако газ может нарушить эти законы (так же, как Президент может нарушить законы), когда он выходит из состояния равновесия. Описывая систему, не находящуюся в равновесии, физики могут опираться только на статистическую механику, то есть немакроскопический уровень описания. Окончательное объяснение поведения газа всегда лежит на молекулярном уровне, так же, как окончательное объяснение политического поведения общества всегда лежит на уровне народа. Изучение неравновесных процессов --- это поиск макроскопических законов для описания поведения газов (и других систем), которые не находятся в равновесии. Оно аналогично ветви социологии, изучающей законы анархических обществ.

Вот другие интересные примеры переплетения уровней в американском правительстве: ФБР, расследующее собственные преступления; шериф, угодивший в тюрьму, самоприложение парламентарного закона процедур и т. д. Один из самых интересных случаев на моей памяти касается человека, утверждавшего, что он --- экстрасенс. Он говорил, что может использовать свое умение для определения черт характера людей и, таким образом, может помогать судьям в выборе жюри. Однако что случилось бы, если в один прекрасный день этот «экстрасенс» сам оказался под судом? Какое влияние оказало бы это на тех членов жюри, которые верят в существование экстрасенсорных способностей? Насколько они окажутся под влиянием экстрасенса (вне зависимости от подлинности его способностей)? Это поле созрело для эксплуатации, на его почве могут отлично произрастать самоисполняющиеся пророчества.

Путаница, касающаяся науки и мистики

Кстати об экстрасенсах и мистических способностях, псевдонаука --- это еще одна сфера жизни, где кишмя кишат странные петли. Псевдонаука ставит под сомнение многие стандартные процедуры или мнения ортодоксальной науки и, таким образом, бросает вызов её объективности. Псевдонаука предлагает иные пути интерпретации очевидного. Но как вообще можно оценить интерпретацию очевидного? Ведь это та же проблема объективности, только перенесенная уровнем выше! Парадокс Льюиса Кэрролла --- бесконечный регресс --- появляется здесь в новом обличье. Черепаха утверждала бы, что если вы хотите доказать, что А --- это факт, то вам нужно доказательство --- Б. Но как можно с уверенностью сказать, что Б доказывает А? Для этого нужно мета-доказательство ---~В. Но для доказательства действительности мета-доказательства вам понадобится мета-мета-доказательство --- и так далее, пока не надоест. Несмотря на этот довод, у людей есть врожденное чувство очевидности. Как я уже говорил, это происходит потому, что в человеческий мозг встроена некая аппаратура, включающая рудиментарные способы интерпретации данных. Основываясь на этом, мы можем развить новые способы интерпретации и даже научиться иногда подавлять основные механизмы интерпретации фактов --- как приходится делать, скажем, для объяснения трюков фокусника.

Конкретные примеры проблем с очевидным возникают во множестве явлений псевдонауки. Например, экстрасенсорные способности часто проявляются вне лаборатории, но таинственным образом исчезают, как только экстрасенс попадает внутрь. Стандартное научное объяснение заключается в том, что эти способности --- явление недействительное, не выдерживающее строгой проверки. Некоторые (но не все) сторонники экстрасенсов, однако, находят интересный способ опровержения этих доводов. Они говорят: «Нет, экстрасенсорные способности реальны, но они исчезают, когда кто-то пытается исследовать их научными методами, потому что они несовместимы с таким подходом». Это довольно бесстыдная техника, которую можно назвать «отфутболиванием проблемы этажом выше.» Это значит, что вместо анализа самой проблемы они сомневаются в теориях, принадлежащих к высшему уровню правдоподобия. Защитники экстрасенсов утверждают, что проблема не в \emph{их} идеях, а в системе научных взглядов. Это весьма смелое утверждение; за отсутствием неопровержимых доказательств стоит усомниться в его правильности. Однако мы говорим здесь о «неопровержимых доказательствах» так, словно все согласны, что это означает!

Природа очевидного

Запутанная ситуация Сагредо-Симплицио-Салвиати, упомянутая в главах XIII и XV, --- это еще один пример того, как сложно оценить очевидное. Сагредо пытается найти некий объективный компромисс между противоположными точками зрения Симплицио и Салвиати. Но компромисс оказывается не всегда возможным. Как можно «справедливо» примирить правильное и неправильное? Справедливое и несправедливое? Компромисс и не компромисс? Эти вопросы возникают снова и снова, в разных формах, при обсуждении самых повседневных вещей.

Возможно ли дать определение очевидному? Можно ли перечислить законы, объясняющие, как находить смысл в различных ситуациях? Скорее всего, нет, поскольку у любых жестких правил несомненно будут исключения, а нежесткие правила --- уже не правила. Думающая программа также не поможет делу, поскольку в качестве процессора очевидного она будет ничуть не надежнее людей. Так если очевидное настолько неуловимо, почему же я протестую против новых путей его интерпретации? Не противоречу ли я сам себе? В данном случае, не думаю. Мне кажется, что здесь есть определенные ориентиры, при помощи которых можно добиться органического синтеза. Но в этом с неизбежностью будет некоторая доля интуиции и субъективности --- а они различны в каждом отдельном человеке. Они будут различны и в разных программах ИИ\@. Существуют сложные критерии для решения того, хорош ли данный метод оценки очевидности. Один из них касается «полезности» идей, полученных данным методом. Рассуждения, приводящие к жизненно полезным идеям, считаются в каком-то смысле правильными. Однако слово «полезный» здесь очень субъективно\ldots{}

Я считаю, что процесс, с помощью которого мы решаем, что истинно и действительно, --- это вид искусства. Он опирается на чувство красоты и простоты не менее, чем на железные принципы логических рассуждений или чего-либо иного, что может быть объективно формализовано. Я \emph{не} утверждаю, что (1) истина --- химера или что (2) человеческий интеллект в принципе невозможно запрограммировать. Я \emph{утверждаю} то, что (1) истина слишком неуловима, чтобы человек или группа людей могли её полностью понять и (2) когда ИИ достигнет уровня человеческого интеллекта --- или превзойдет его --- он все еще будет бороться с проблемами искусства, красоты и простоты и постоянно наталкиваться на эти вопросы в своем стремлении к знанию. «Что такое очевидность?» --- это не только философский вопрос, поскольку он часто вторгается в обыденную жизнь. В каждую минуту перед нами огромный выбор способов интерпретации очевидного. Сегодня трудно найти книжный магазин, где бы вы не увидели книг об астрологии, хиромантии, мистицизме, телекинезе, НЛО, Бермудском треугольнике, черных дырах, биологической обратной связи, телепатии, трансцендентальной медитации, новых теориях психологии\ldots{} В науке идут яростные споры о теории катастрофы, теории элементарных частиц, черных дырах, истине и существовании в математике, свободе воли, Искусственном Интеллекте, редукционизме и холизме\ldots{} На более практическом уровне идут споры о том, что полезнее витамин С или летрил (новое средство против рака), о размере нефтяных запасов (под землей или в хранилищах), о том то является причиной инфляции и безработицы --- и так далее, и тому подобное. Не будем забывать и о Дзен-буддизме, парадоксах Зенона, психоанализе и т. п. Способы оценки действительности играют важнейшую роль, будь то в тривиальном вопросе размещения книг на полках в книжном магазине или в вопросе о том, какие идеи должны преподаваться в школах.

Самовосприятие

Одна из самых сложных проблем интерпретации действительности --- это истолкование множества беспорядочных внешних сигналов, говорящих нам кто мы такие. Здесь очень высока возможность конфликтов внутри уровней и между ними. Психическим механизмам приходится одновременно иметь дело с внутренней потребностью человека в самоуважении и непрерывным потоком информации извне, атакующим его представление о самом себе. В результате информация течет между разными уровнями личности по сложному руслу. Пока она крутится в этом водовороте, какие-то её части разрастаются а какие-то уменьшаются, что-то отрицается вообще, а что-то меняется почти до неузнаваемости. Затем результат снова попадает в водоворот и этот процесс но повторяется снова и снова в попытке примирить то, что есть, с тем чего бы нам хотелось (См. рис. 81).

В результате этого необычайно сложного процесса общее представление о том, «кто я такой» интегрируется с остальной мысленной структурой и содержит, для каждого из нас, большое количество нерешенных и, возможно неразрешимых противоречий. Безусловно, это один из основных источников того динамического напряжения, которое так свойственно человеческим существам. Из этого противоречия между внутренним и внешним образами того кто мы такие, рождаются те стремления и цели, которые делают каждого из нас единственным в своем роде. Парадоксальным образом, именно наша общность --- то, что все мы обладаем самосознанием --- ведет к удивительному разнообразию того, как мы усваиваем информацию о самых разных вещах и является ведущей силой в создании разных индивидуальностей.

Теорема Гёделя и другие дисциплины

Кажется естественным проводить параллели между людьми и достаточно сложными формальными системами, которые, как и люди, обладают неким «самосознанием». Теорема Геделя показывает, что в непротиворечивых формальных системах, способных к автореференции, есть фундаментальные ограничения. Можно ли обобщить этот вывод? Существует ли например «Теорема Геделя в психологии»?

Если использовать теорему Геделя как метафору и источник вдохновения вместо того, чтобы пытаться дословно перевести её на язык психологии, возможно, что она сможет подсказать новые истины в психологии и других областях. Но пытаться прямолинейно переводить её на язык других дисциплин и считать что она действительна и там, было бы ошибкой. Неверно было бы думать что то, что было тщательнейшим образом разработано в математической логике можно без изменений пересадить на совершенно иную почву других дисциплин.

Интроспекция и душевные заболевания: проблема типа Гёделевой

Мне кажется, что перевод теоремы Гёделя в другие области может навести нас на новые идеи, если мы договоримся заранее о том, что переводы --- только метафоры и не должны пониматься дословно. С подобной оговоркой я вижу две основных аналогии, соотносящие Теорему Гёделя с человеческим мышлением. Одна из них касается проблемы размышлений о собственной нормальности. Каким образом вы можете решить, что вы не сумасшедший? Это --- настоящая Странная Петля. Как только вы начинаете сомневаться в собственном душевном здоровье, то можете оказаться во все убыстряющемся водовороте самоисполняющихся пророчеств (хотя этот процесс вовсе не неизбежен). Известно, что сумасшедшие интерпретируют мир с помощью странной, но последовательной логики; откуда вы знаете, «странная» ваша собственная логика или нет, если можете судить об этом только с помощью той же самой логики? Я не знаю ответа на этот вопрос. Это напоминает мне о второй Теореме Гёделя, из которой следует, что противоречивы только те версии формальной теории чисел, которые утверждают собственную непротиворечивость.

Можем ли мы понять собственный разум или мозг?

Другая аналогия с Теоремой Гёделя, которая кажется мне интересной, намекает на то, что мы не можем полностью понять собственного разума/мозга. Эта идея настолько сложна и нагружена ассоциациями на многих уровнях, что обсуждать её надо с осторожностью. Что означает «понять собственный разум/мозг»? Это может означать общее ощущение того, как они работают, подобно тому, как автомеханик интуитивно ощущает, как работает мотор. Это может означать полное объяснение того, почему люди поступают так, а не иначе. Это может означать полное понимание физической структуры собственного мозга на всех его уровнях. Это может означать, что в некоей книге или на компьютере у нас есть подробная диаграмма устройства мозга. Это может означать точное знание того, что происходит в нашем мозгу на нейронном уровне в любой момент --- возбуждение каждого нейрона, синаптические изменения и так далее. Это может означать создание программы, способной пройти тест Тюринга. Это может означать такое полное знание себя, что понятия подсознательного и интуитивного теряют смысл, поскольку становятся открыты взгляду. Это может означать также любую комбинацию из вышеприведенных вещей.

Какое из этих самоотражений более всего напоминает Теорему Гёделя? Затрудняюсь ответить. Некоторые из них довольно глупы. Например, идея детального наблюдения за состоянием собственного мозга --- ни что иное, как беспочвенная фантазия, абсурдное и неинтересное предположение; когда Теорема Гёделя говорит нам, что это невозможно, это отнюдь не является для нас неожиданностью. С другой стороны, извечное стремление человека глубже познать самого себя (назовем это «пониманием собственной психической структуры») кажется естественным. Но нет ли и здесь некоей Гёделевой петли, ограничивающей глубину возможного проникновения в собственную психику? Мы не можем увидеть своими глазами собственное лицо; не разумно ли ожидать, что мы также окажемся не в состоянии полностью отразить собственную психическую структуру в тех символах, которые её составляют?

Как ограничительные Теории метаматематики, так и теория вычислений говорят, что, как только возможность представлять собственную структуру достигает некоей критической точки, то пиши пропало --- это гарантия того, что вы никогда не сможете представить себя полностью. Теорема Гёделя о неполноте, Теорема Черча о неразрешимости, Теорема остановки Тюринга, Теорема Тарского об истине --- все они чем-то напоминают старинные сказки, предупреждающие читателя о том, что «поиск самопознания --- это путешествие, которое\ldots{} обречено быть неполным, не может быть изображено ни на каких картах, никогда не остановится и не сможет быть описано.»

Но имеют ли эти ограничительные теоремы какое-нибудь отношение к людям? Об этом можно рассуждать так. Либо я непротиворечив, либо я противоречив. (Последнее гораздо вероятнее, но, для полноты картины, я рассмотрю обе возможности.) Если я непротиворечив, этому могут быть два объяснения. (1) Я подобен патефону «низкого качества»: мое понимание самого себя находится ниже некоего критического порога. В данном случае, я неполон по определению. (2) Я подобен патефону «высокого качества», мое понимание себя самого достигло критического порога, за которым становится приложима метафорическая аналогия ограничительных Теорем; таким образом, мое самопознание саморазрушается Гёделевым способом, и поэтому я неполон. Случаи (1) и (2) основаны на предположении, что я стопроцентно непротиворечив --- маловероятное положение дел. Скорее всего, я противоречив, --- но это еще хуже, поскольку означает, что во мне есть противоречия; как я смогу когда-либо это понять?

Противоречив человек или нет, он обречен вечно размышлять над загадкой собственной личности. Скорее всего, мы все противоречивы. Мир слишком сложен, чтобы позволить человеку роскошь примирить между собой все его убеждения. Напряжение и неразбериха важны в мире, где часто приходится быстро принимать решения Мигель де Унамуно однажды сказал: «Если кто-то никогда не противоречит сам себе, скорее всего, это потому, что он вообще никогда ничего не говорит.» Я сказал бы, что мы все находимся в том же положении, как тот мастер дзена, который, высказав подряд несколько противоречивых суждений, сказал сбитому с толку Доко: «Я сам себя не понимаю.»

Теорема Гёделя и личное несуществование

Наверное, самое большое противоречие нашей жизни, то, которое труднее всего понять, --- это знание того, что было время, когда нас не было, и придет время, когда нас не будет. На одном уровне, когда мы «выходим из себя» и видим себя как «одно из человеческих существ», этот факт имеет смысл. Но на другом, более глубоком уровне, личное несуществование совершенно бессмысленно. Все, что мы знаем, находится у нас в мозгу, и мы не можем понять, как это все может отсутствовать во вселенной. Это основная и неоспоримая тайна жизни; возможно, что это --- лучшая метафорическая аналогия Теоремы Гёделя. Когда мы пытаемся вообразить собственное несуществование, нам приходится выйти из себя и отобразить себя на кого-то другого. Мы пытаемся убедить самих себя, что можем внести внутрь чужие представления о нас, примерно так же, как ТТЧ «верит», что ей удается отразить внутри себя собственную мета-теорию. Однако ТТЧ содержит собственную мета-теорию не целиком, а только до определенного предела. Что касается нас, то мы можем только воображать, что вышли из себя, мы никогда не в состоянии действительно это сделать, так же, как Эшеровский дракон не может вырваться из своей родной двухмерной плоскости в трехмерный мир. В любом случае, это противоречие настолько велико, что обычно мы просто-напросто игнорируем всю эту путаницу, поскольку попытки в ней разобраться ни к чему не приводят.

Последователи дзена, с другой стороны, упиваются этой противоречивостью. Снова и снова они встречаются с конфликтом между восточным представлением о том, что «мир и я --- одно целое, поэтому понятие моего несуществования само по себе противоречиво» (моя формулировка, наверняка, слишком западная --- прошу прощения у дзен-буддистов) и западным представлением: «Я --- только часть мира; я умру, но мир будет жить и после меня.»

Наука и дуализм

Науку часто критикуют за то, что она слишком «западна» или «дуалистична» --- то есть проникнута дихотомией между субъектом и объектом, наблюдателем и наблюдаемым. Действительно, вплоть до нашего столетия наука занималась только вещами, которые могли быть легко отличимы от человека, --- например, кислород, углерод, свет и тепло, ускорение и орбиты и так далее. Эта фаза развития была необходимой прелюдией к более современной фазе, в которой объектом исследований явилась сама жизнь. Шаг за шагом «западная» наука неизбежно движется к изучению человеческого разума --- иными словами, разума самого наблюдателя. В настоящий момент в этом лидируют исследования по Искусственному Интеллекту. До появления ИИ в науке произошли два события, позволяющие до некоторой степени предвидеть последствия смешения субъекта и объекта. Одним из них была революция в квантовой механике; она породила эпистемиологические проблемы, касающиеся влияния наблюдателя на наблюдаемое. Другим было смешение объекта и субъекта в метаматематике, начавшееся с Теоремы Гёделя и присутствующее во всех ограничительных Теоремах, о которых мы говорили. Возможно, что после ИИ наступит очередь самоприложения науки --- она начнет изучать саму себя. Это иной способ смешения субъекта и объекта, может быть, даже более запутанный, чем люди, изучающие собственный мозг.

Кстати, интересно заметить, что все результаты, зависящие от слияния субъекта с объектом, оказываются ограничительными. Кроме ограничительных Теорем, сюда относится принцип неопределенности Хайзенберга, утверждающий, что измерение некоей величины делает невозможным измерение другой величины, связанной с первой. Я не знаю, почему все эти результаты получаются ограничительными. Читатель может понимать это, как хочет.

Символ и объект в современной музыке и живописи

Дихотомия субъекта и объекта --- близкая родственница дихотомии символа и объекта, которая была глубоко изучена Людвигом Витгенштейном в начале этого столетия. Позже для обозначения этого различия были приняты термины «использование» и «упоминание». Квайн и другие подробно описали отношение между знаками и тем, что они обозначают. Но эта глубокая и абстрактная тема занимала не только философов. В нашем столетии как музыка, так и изобразительное искусство испытали кризис, отразивший глубокий интерес к этой проблеме. Музыка и живопись традиционно выражали идеи с помощью некоего набора «символов» (зрительные образы, аккорды, ритмы и тому подобное), сейчас, однако, появилась тенденция исследовать способность искусства не \emph{выражать} , а просто \emph{быть} . Например, быть пятнами краски или чистыми звуками, лишенными всякого символического значения.

В частности, на музыку оказал большое влияние Джон Кэйдж со своим новым, напоминающим дзен-буддизм, подходом к звуку. Многие из его сочинений показывают презрение к «использованию» звуков (то есть использованию звуков для передачи эмоциональных состояний) и удовольствие от «упоминания» звуков (то есть создания произвольных комбинаций звуков, не пользуясь заранее установленным кодом, с помощью которого слушатель мог бы расшифровать некое послание). Типичным примером такой композиции является «Воображаемый пейзаж \# 4», пьеса для нескольких радио, которую я описал в главе VI. Возможно, что я несправедлив к Кэйджу но мне кажется что его основной целью было привнесение в музыку бессмысленности и наделение значением самой этой бессмысленности. Алеаторная музыка --- типичный шаг в этом направлении. Многие современные композиторы последовали за Кэйджем но немногие из них были так же оригинальны. В пьесе Анны Локвуд под названием «Горящий рояль» имитируется звук лопающихся струн, для чего они натягиваются как можно туже, в пьесе Ламонте Юнга источником шума является рояль, который возят туда-сюда по сцене и сталкивают с препятствиями.

В искусстве нашего столетия было множество подобных судорог. Сперва художники отказались от представления действительности, что было по-настоящему революционным шагом --- началом абстрактного искусства. Постепенный переход от реалистического представления к чисто абстрактным схемам можно видеть в работах Пьета Мондриана. После того, как мир привык к нерепрезентативному искусству, родился сюрреализм Это был странный поворот, что-то вроде нео-классицизма в музыке, крайне репрезентативное искусство было здесь перевернуто с ног на голову и использовано с совершенно иной целью, чтобы шокировать, сбить с толку и удивить. Школа сюрреализма была основана Андрэ Бретоном и находилась, в основном, во Франции, среди самых влиятельных её последователей были Дали, Магритт, де Чирико и Тангуй.

Семантические иллюзии Магритта

Из этих художников наиболее чувствующим загадку субъекта и объекта был Магритт (для меня эта загадка является продолжением различия между использованием и упоминанием). Его картины поражают именно этим, хотя зрители обычно не выражают своих впечатлений в таких терминах. Взгляните, например, на странную вариацию на тему натюрморта под названием «Здравый смысл» (рис. 137).

\emph{Рис. 137. Рене Магритт. «Здравый смысл» (1945-1946).}

Блюдо, полное фруктов --- то, что обычно изображается на натюрморте, --- здесь стоит на чистом холсте. Конфликт между символом и реальностью велик. Но ирония на этом не кончается, поскольку все это, разумеется, всего лишь картина, --- а именно, натюрморт с нестандартным сюжетом.

Серия картин Магритта, представляющих трубку, одновременно очаровывает и приводит в замешательство. Взгляните, например, на «Две тайны» (рис. 138). Внутренний фрагмент картины говорит вам, что символы и трубки различны. Затем ваш взгляд переходит к «настоящей» трубке, плавающей в воздухе. Вы воспринимаете её, как настоящую, в то время как другая трубка --- только символ. Но, разумеется, это совершенно неверно: обе они написаны на плоской поверхности. Идея, что одна из трубок --- «картина с двойным вложением» и поэтому в каком-то смысле «менее реальна,» совершенно ошибочна. Вы были одурачены уже в тот момент, когда, приняв изображение за реальность, решили «войти в комнату». Будучи последовательным в вашей доверчивости, вы должны теперь спуститься еще одним уровнем ниже и спутать с реальностью изображение-внутри-изображения. Единственный способ не быть затянутым внутрь иллюзии заключается в том, чтобы видеть обе трубки лишь как цветные пятна на поверхности, отстоящей от вашего носа на насколько сантиметров. Только тогда вы сможете по-настоящему оценить полное значение послания «Ceci n'est pas une pipe» (Это не трубка) --- но, к несчастью, в тот самый момент, когда трубки превращаются в цветные пятна на холсте, то же самое происходит с надписью, которая, таким образом, теряет смысл! Иными словами, в этот момент словесное сообщение на картине саморазрушается самым что ни на есть Гёделевым образом.

\emph{Рис. 138. Рене Магритт. «Две тайны» (1966).}

Картина «Воздух и песня» достигает того же эффекта, как и «Две тайны», но делает это на одном уровне вместо двух. Мои рисунки «Дымовой сигнал» и «Сон о трубке» (рис. 139 и 140) --- вариации на тему Магритта. Попытайтесь смотреть на «Дымовой сигнал» в течение некоторого времени. Вскоре вы различите скрытое послание «Ceci n'est pas un message» (Это не сообщение). Таким образом, если вы находите сообщение, оно отрицает само себя --- а если вы его не находите, то вообще не понимаете картины. Благодаря своему косвенному «саморазрушению», оба мои рисунка могут быть приблизительно отображены на Гёделево высказывание G.

\emph{Рис. 139. Дымовой сигнал. (Рисунок автора.)}

Классическим примером смешения «использования» с «упоминанием» может служить изображение на картине палитры. В то время как эта нарисованная палитра --- иллюзия, созданная искусством художника, краски на ней --- самые настоящие мазки краски с его палитры. Краска здесь представляет саму себя и ничего больше. В «Доне Джованни» Моцарт исследовал родственный прием, включив в партитуру звуки настраивающегося оркестра. Таким же образом, если я хочу, чтобы буква 'я' играла роль самой себя (а не символизировала меня), то включаю 'я' в свой текст; в таком случае, я заключаю 'я' в кавычки. У меня получается ``я`` (не "я" и не '''я'''). Понимаете?

\emph{Рис. 140. Сон о трубке (Рисунок автора.)}

Код современного искусства

Множество влияний, которые вряд ли возможно охарактеризовать полностью, привели к дальнейшему исследованию искусством дуализма между символом и объектом. Нет сомнения в том, что Джон Кэйдж с его интересом к дзен-буддизму оказал большое влияние не только на музыку, но и на живопись. Его друзья Джаспер Джонс и Роберт Раушенберг исследовали различие между символами и объектами, используя для этого в качестве символов сами объекты, --- или, наоборот, используя символы как объекты сами по себе. Все эти усилия, возможно, были направлены на то, чтобы опровергнуть мнение, что искусство стоит в стороне от действительности и говорит на «коде», который зритель должен затем интерпретировать. Идея заключалась в том, чтобы исключить интерпретацию и позволить обнаженному предмету просто быть --- и точка. (Эта «точка» --- забавный пример смешения различия между использованием и упоминанием.) Однако если их намерение было таково, то можно считать, что оно с треском провалилось.

Когда некий предмет находится на выставке или именуется «произведением искусства», он приобретает ореол глубокого внутреннего значения, даже если при этом зрителей \emph{попросили} этого значения не доискиваться. Более того, чем настойчивее зрителей просят не искать в произведениях никакого скрытого смысла, тем больше смысла они там находят. В конце концов, если деревянный ящик, стоящий на полу музея, всего-навсего деревянный ящик на полу музея, то почему уборщица не вынесет его на помойку? Почему к нему привязана этикетка с именем художника? Почему этот художник хочет удалить из искусства всякую тайну? Почему пятно грязи на передней стенке ящика не несет подписи художника? Не розыгрыш ли все это? Интересно, кто сошел с ума я или художники? Все новые и новые вопросы приходят в голову зрителю --- он не может этого избежать Это так называемый «эффект рамы», который автоматически создается Искусством. Рождение вопросов в голове любопытного зрителя предотвратить невозможно.

Разумеется, если его целью является постепенное внушение дзен-буддистского восприятия мира как свободного от категорий и значений, то такое искусство --- как и рассуждения по поводу дзена --- пытается послужить катализатором, вдохновляющим зрителя на более глубокое ознакомление с философией, отрицающей «внутренние значения» и объемлющей мир как одно целое. В таком случае, оно не достигает этой цели немедленно, так как зрители все равно размышляют о его значении; но, в конце концов, некоторые из них могут обратиться к источникам этого искусства, и тогда его цель будет достигнута. Но в любом случае неверно, что здесь нет никакого кода, с помощью которого идеи передаются зрителю.~На самом деле, этот код весьма сложен и включает сведения об отсутствии кодов и тому подобное --- то есть он является отчасти кодом, отчасти мета-кодом и так далее. Сообщения, которые передают самые «дзен-буддистские» предметы искусства, представляют из себя Запутанную Иерархию; может быть, поэтому многие находят современное искусство таким непонятным.

Еще раз об изме

Во главе движения, пытавшегося стереть границы между искусством и природой, стоял Кэйдж. Он считал, что в музыке все звуки равны --- нечто вроде акустической демократии. Тишина точно так же важна, как и звук, и случайные звуки ничем не хуже организованных. Леонард Мейер в своей книге «Музыка, искусство и идеи» (Leonard В. Meyer. «Music, Art and Ideas») называет это движение в музыке «трансцендентализмом» и утверждает:

Если различие между искусством и природой ошибочно, то эстетическая оценка неважна. Фортепианная соната достойна оценки не более, чем камень, буря или морская звезда. «Категорические суждения, такие, как правильно и неправильно прекрасно или уродливо, типичные для рационалистского мышления тональной эстетики» --- пишет Люциано Берио (современный композитор), --- «уже не годятся для понимания того, как сегодняшний композитор работает над слышимыми формами и музыкальным действием».

Затем Мейер продолжает, описывая философскую позицию трансцендентализма:

все вещи во времени и пространстве сложнейшим образом переплетены друг с другом. Любые деления, классификации или типы организации, открытые нами во вселенной, чисто случайны. Мир --- это сложное, непрерывное, единое событие.\footnote{Leonard В. Meyer, «Music, The Arts, and Ideas», стр. 161, 167.} (Эхо дзена!)

~Мне кажется, что «трансцендентализм» --- слишком громоздкое название для этого движения. Я предпочитаю называть его просто «измом». Будучи суффиксом без корня, это напоминает идеологию без идей, --- что, скорее всего, так и есть, как бы мы её не интерпретировали. Поскольку «изм» включает в себя все, что угодно, это название сюда отлично подходит. В «изме» слово «is» (есть) наполовину используется, наполовину упоминается; что может быть более подходящим? Изм --- это дух дзена в искусстве. Так же, как основная задача дзена --- сорвать маску с самого себя, основная задача искусства нашего столетия, как кажется, --- это найти ответ на вопрос, что такое искусство. Все его метания --- поиски самого себя.

Итак, конфликт между использованием и упоминанием, доведенный до крайности, превращается в философскую проблему дуализма символа и объекта, что связывает его с тайной разума. Магритт писал о своей картине «Человеческое состояние I» (Рис. 141):

Я расположил перед окном картину, видимую из комнаты, на которой была изображена именно часть пейзажа, скрытая картиной. Таким образом дерево на картине скрывало от взгляда дерево, расположенное за ним, вне комнаты. При этом в голове зрителя дерево существовало одновременно в комнате (как часть картины) и снаружи (как часть настоящего пейзажа). Именно так мы видим мир мы думаем, что он вне нас, хотя он --- только наше мысленное представление о нем, возникающее внутри нас.\footnote{Suzi Gablik, «Magritte», стр. 97.}

\emph{Рис. 141. Рене Магритт. Человеческое состояние I (1933).}

Понимание разума

Сначала многозначительными образами своего рисунка и затем прямым текстом Магритт говорит о связи между двумя вопросами: «Как работают символы?» и «Как работает наш разум?» Кроме того, он возвращает нас к поставленному ранее вопросу. «Можем ли мы надеяться когда-либо понять собственный мозг и разум?»

Или же какое-то удивительное и дьявольское построение, подобное Гёделеву, не позволит нам проникнуть в эту тайну? Если принять достаточно разумное определение того, что такое «понимание», то я не вижу никаких Геделевых препятствий к постепенному пониманию сути нашего разума. Например, мне кажется вполне разумным желание понять общий принцип работы мозга, так же, как мы понимаем общий принцип работы автомобильного мотора. Это совсем не то, что пытаться понять любой отдельный мозг во всех деталях, --- и, тем более, пытаться проделать это с собственным мозгом! Я не вижу никакой связи между Теоремой Гёделя, даже в самой приблизительной интерпретации, и возможностью выполнения этого проекта. Мне кажется, что Теорема Гёделя не накладывает никаких ограничений на нашу способность формулировать и проверять общие механизмы мыслительных процессов, происходящих в нервных клетках. По моему мнению, Теорема Гёделя не противоречит созданию компьютеров (или их преемников), которые смогут манипулировать символами примерно с тем же успехом, как и мозг. Совершенно иное дело --- пытаться воспроизвести в программе определенный человеческий мозг, однако создание разумных программ вообще --- это более скромная цель Теорема Гёделя запрещает воспроизводство нашего уровня разума с помощью программ не более, чем она запрещает воспроизводство нашего уровня разума с помощью передачи наследственной информации в ДНК\@. В главе XVI мы видели, как именно замечательный Гёделев механизм --- Странная Петля белков и ДНК --- делает возможной передачу разума.

Значит ли это, что Теорема Гёделя не привносит ничего нового в наши размышления о собственном разуме? Мне кажется, что это не так, --- некая связь здесь есть, но не в том мистическом и ограничительном смысле, как считают некоторые. Думаю, что процесс понимания Гёделева доказательства с его произвольными кодами, сложными изоморфизмами, высоким и низким уровнями интерпретации и способностью к самоотражению может обогатить наше представление о символах и их обработке, что, в свою очередь, может развить наше интуитивное понимание мыслительных структур на разных уровнях.

Случайная необъяснимость разума?

Прежде чем предложить философски интригующее «приложение» Гёделева доказательства, я хочу упомянуть об идее «случайной необъяснимости» разума. Вот в чем она состоит. Может быть, наши мозги, в отличие от автомобильных моторов, представляют собой упрямые и необъяснимые системы, разложить которые никак невозможно. В данной момент мы не знаем, уступит ли мозг нашим усилиям разделить его на уровни, каждый из которых сможет быть объяснен в терминах низших уровней, или же он сорвет все наши попытки его проанализировать.

Но даже если мы и потерпим неудачу в попытке понять самих себя, за этим вовсе не обязательно должна стоять теорема Гёделя. Может быть, наш мозг по чистой случайности слишком слаб для этого. Подумайте, например, о скромном жирафе. Очевидно, что его мозг --- намного ниже уровня, необходимого для понимания себя. Тем не менее, он очень похож на наш мозг! Действительно, мозги горилл, эму и бабуинов --- и даже мозги черепах или неизвестных существ, намного умнее нас, --- действуют, скорее всего, по примерно одинаковому принципу. Жирафы могут находиться намного ниже уровня, необходимого для понимания того, как эти правила сочетаются, чтобы произвести качества разума. Люди могут стоять ближе к этому уровню --- чуть-чуть ниже или даже чуть-чуть выше критического порога понимания. Но в этом может не быть никакой принципиальной причины типа Гёделевой, по которой качества разума были бы необъяснимы, --- они могут быть вполне понятны существам, стоящим на более высокой ступени развития.

Неразрешимость неотделима от точки зрения высшего уровня

Исключив пессимистическое понятие о врожденной необъяснимости нашего мозга, посмотрим, какие идеи может нам предложить доказательство Гёделя в отношении объяснения нашего мозга/разума. Оно дает нам понять, что взгляд на систему с точки зрения высшего уровня может позволить понять то, что на низших уровнях кажется совершенно необъяснимым. Я имею в виду следующее. Предположим, что в качестве строчки ТТЧ вам дали высказывание Гёделя G. Представьте, что вам при этом ничего не известно о Гёделевой нумерации. Вы должны ответить на вопрос: «Почему эта строчка --- не теорема ТТЧ?»

~Вы уже хорошо знакомы с подобными вопросами; например, если бы такой вопрос был задан вам о строчке S0=0, вы ответили бы без труда: «\emph{Потому что теоремой является её отрицание} , \textasciitilde S0=0.» Этот факт вместе с вашим знанием о непротиворечивости ТТЧ объясняет, почему данная строчка --- не теорема. Это то, что я называю объяснением «на уровне ТТЧ». Обратите внимание, насколько оно отличается от объяснения того, почему MU --- не теорема системы MIU, первое объяснение дано в режиме М, второе --- в режиме I.

А как насчет G? Объяснение на уровне ТТЧ, сработавшее для строчки S0=0, для G не работает, поскольку \textasciitilde G теоремой \emph{не} является. Человек, не имеющий общего представления о ТТЧ, не поймет, почему он не может вывести G, следуя правилам, --- ведь в G, как в арифметическом высказывании, нет никаких ошибок! Когда G превращено в универсально квантифицированную строчку, в ТТЧ может быть выведено любое высказывание, полученное из него путем подстановки символов чисел вместо переменных. Единственный способ объяснить нетеоремность G заключается в использовании Гёделевой нумерации и взгляде на ТТЧ с совершенно иного уровня. Дело тут не в том, что в ТТЧ объяснение написать слишком сложно, --- это просто невозможно. Подобного объяснения в ТТЧ в принципе не существует. На высшем уровне есть некие возможности, которыми ТТЧ не обладает. Нетереомность ТТЧ, если можно так выразиться, является \emph{фактом высшего уровня} . У меня есть подозрение, что это верно для \emph{всех} неразрешимых суждений --- иными словами, любое неразрешимое суждение является ни чем иным, как Гёделевым высказыванием, утверждающим собственную нетеоремность в некоей системе с помощью какого-либа кода.

Сознание как явление высшего уровня

В этом смысле, Гёделево доказательство наводит на мысль --- хотя ни в коем случае её не доказывает! --- что может существовать некий высший уровень, на котором можно рассматривать разум/мозг. На этом уровне могут существовать понятия, отсутствующие на низших уровнях. Это значит, что там можно было бы легко объяснить те факты, которые на низшем уровне объяснить \emph{невозможно} . Какими бы длинными и громоздкими ни были высказывания низшего уровня, они не смогут объяснить данного явления. Это аналогично тому факту, что, выводя одну за другой деривации в ТТЧ, какими бы длинными и громоздкими они ни получались, вы никогда не сможете вывести G, несмотря на то, что на высшем уровне вы легко замечаете, что G истинно.

В чем могут заключаться эти понятия высшего уровня? Ученые и гуманисты, сторонники холизма и наличия души, давно уже предположили, что \emph{сознание} невозможно объяснить в терминах составляющих мозга, --- так что это, по крайней мере, один кандидат. Кроме того, существует загадочное понятие \emph{свободной воли} . Возможно, что эти качества появляются «неожиданно», в том смысле, что психология не в состоянии объяснить их возникновения. Но важно понять, что, руководствуясь доказательством Гёделя в формировании этих смелых гипотез, мы должны довести аналогию до конца. В частности, необходимо помнить, что нетеоремность G \emph{имеет} объяснение, --- это вовсе не тайна! Это объяснение опирается не только на понимание отдельного уровня, но и того, как этот уровень отражает свой мета-уровень и какие от этого получаются последствия. Если наша аналогия правильна, то «неожиданные» явления могут быть объяснены в терминах отношений между различными уровнями в разумных системах.

Странные Петли в сердце разума

Я убежден в том, что объяснение «неожиданно» возникающих в наших мозгах явлений --- идей, надежд, образов, аналогий и, наконец, сознания и свободной воли --- основаны на некоем типе Странных Петель, то есть такого взаимодействия между уровнями, при котором высший уровень воздействует на низший уровень, будучи в то же время сам определен этим низшим уровнем. Иными словами, это самоусиливающий «резонанс» между различными уровнями --- нечто вроде суждения Хенкина, которое становится доказуемым, только утверждая свою доказуемость. Индивидуальность рождается в тот момент, когда она становится способна отразить саму себя.

Это не должно быть понято как антиредукционистское утверждение. Я хочу сказать лишь то, что редукционистское объяснение разума, \emph{чтобы быть понятым} , должно содержать такие «гибкие» понятия как уровни, отображение и значение. В принципе, я не сомневаюсь, что теоретически может существовать полностью редукционистское, но непостижимое объяснение мозга; проблема заключается в том, как перевести его на понятный нам язык. Безусловно, нам не нужно описания в терминах позиций и моментов частиц: мы хотим иметь описание, соотносящее нейронную активность с «сигналами» (явлениями промежуточного уровня), а сигналы, в свою очередь, --- с «символами» и «подсистемами», включая предполагаемый «само-символ». Перевод с языка низших уровней физиологической аппаратуры на язык высших уровней психологических программ аналогичен переводу численно-теоретических суждений в суждения метаматематики. Вспомните, что именно скрещение уровней, возникающее в момент перевода, является причиной Гёделевой неполноты и самодоказующего характера суждения Хенкина. Я утверждаю, что именно это скрещение порождает наше почти неподдающееся анализу чувство индивидуальности.

Чтобы понять мозг и разум во всей полноте, мы должны быть способны с легкостью переходить от одного уровня к другому. Кроме того, мы должны будем принять существование нескольких типов «причинности», то есть того, как явления на одном уровне описания могут быть причиной явлений на других уровнях. Иногда мы будем говорить, что явление А является «причиной» явления Б просто потому, что одно из них --- перевод второго в термины иного уровня. Иногда слово «причина» будет употребляться в обычном смысле --- физическая причина. Оба типа причинности --- и, возможно, какие-либо еще --- должны быть приняты в любом объяснении разума, поскольку мы должны будем согласиться с тем, что в Запутанной Иерархии разума причины могут распространяться как снизу вверх, так и сверху вниз --- так же, как и в схеме Центральной Догмы.

В моей гипотетической модели мозга сознание представлено как весьма реальная действующая сила, влияющая на события. Оно занимает важное место в причинно-следственной связи событий и в цепи команд, управляющих мозговыми процессами, где сознание появляется в качестве активной силы\ldots{} Выражаясь проще, все сводится к тому, кто главенствует среди множества причинных сил, населяющих наш мозг. Иными словами, дело идет об установлении иерархии внутричерепных сил контроля. Под черепной коробкой живет множество различных причинных сил; более того, там существуют силы внутри сил внутри сил. как ни в каком другом известном нам пространстве размером в половину кубического фута вселенной.

Короче говоря, продолжая взбираться наверх в иерархии команд в мозгу, на самом верху мы находим общие организующие силы и динамические качества крупных возбужденных структур мозга, соответствующих мысленным состояниям или психической активности. Близко к вершине этой системы команд в мозгу мы находим идеи. В отличие от шимпанзе, у человека есть идеи и идеалы. В этой модели сила причинности которой обладает идея или идеал, так же реальна как молекула, клетка или нервный импульс. Одни идеи порождают другие и помогают их эволюции. Они взаимодействуют между собой~и с другими мысленными силами в одном и том же мозгу, в соседних мозгах и, благодаря глобальной системе коммуникаций в далеких, иностранных мозгах. Кроме этого, они также взаимодействуют с внешним миром; общим результатом всех этих взаимодействий является гигантский скачок в эволюции, подобного которому история еще не знала, включая сюда возникновение живой клетки.\footnote{Roger Sperry, «Mind, Brain, and Humanist Values», стр. 78-83.}

Известно, что между двумя языками, субъективным и объективным, большая разница. Например, «субъективное» чувство красного и «объективная» длина волны, соответствующая красному цвету. Многим людям эти языки кажутся в принципе несовместимыми. Я так не считаю. Мне кажется, они не более несовместимы, чем два восприятия Эшеровских рисующих рук. «изнутри системы», где руки рисуют одна другую, и извне, где Эшер рисует обе руки. Субъективное ощущение красного появляется благодаря самосознанию в мозгу; объективная длина волны соответствует взгляду извне системы. Хотя никому не удастся выйти из системы настолько, чтобы увидеть «всю картину разом», мы не должны забывать, что такая картина существует. Необходимо помнить, что все это вызвано к жизни физическими законами, глубоко-глубоко в нейронных закоулках и трещинках, куда не достигают наши «зонды», запущенные с высшего уровня наблюдения.

Символ самого себя и свободная воля

В главе XII была высказана мысль, что то, что мы называем свободной волей, --- это результат взаимодействия символа (или подсистемы) самого себя с другими символами в мозгу. Если согласиться с тем, что символы --- это явления высшего порядка, которые наделяются значениями, то можно попытаться объяснить связь между символом «Я» и остальными символами мозга. Чтобы рассмотреть вопрос о свободной воле в перспективе, его можно заменить вопросом, по моему мнению, эквивалентным, но выраженным в более нейтральных терминах. Вместо того, чтобы спрашивать: «Обладает ли система X свободной волей?» мы можем спросить: «Есть ли в системе X понятие выбора?» Думаю, что выяснение того, что мы имеем в виду, говоря, что некая механическая или биологическая система способна «выбирать», может многое прояснить в вопросе о свободной воле. Рассмотрим несколько различных систем, которые в разных обстоятельствах классифицируются нами как «способные к выбору» Из этих примеров станет ясно, что мы имеем в виду под этим выражением.

В качестве парадигмы давайте возьмем следующие системы: шарик, скатывающийся с горки; карманный калькулятор, вычисляющий десятичную часть квадратного корня из двух; сложная компьютерная программа, отлично играющая в шахматы; робот в Т-образном лабиринте (лабиринт в форме буквы «Т», в одном из концов которого находится награда), человеческое существо перед сложной задачей.

Прежде всего, рассмотрим скатывающийся с горы шарик. Выбирает ли он свой путь? Думаю, что все мы единогласно скажем, что нет, хотя никто из нас не способен предсказать даже короткий отрезок его пути. Нам кажется, что он не мог бы катиться по иному пути, поскольку его путь предопределен жесткими законами природы. В нашем мысленном блочном представлении о физике мы, разумеется, можем вообразить множество «возможных» путей шарика, по одному из которых шарик катится в действительности. Отсюда следует, что на некоем уровне нашего разума мы считаем, что шарик «выбрал» один из мириад мысленных путей; в то же время, на другом уровне мы инстинктивно понимаем, что мысленная физика --- всего лишь вспомогательное средство для формирования нашего внутреннего представления о мире. Механизмы, вызывающие к жизни те или иные события действительности, не нуждаются в том, чтобы природа проходила через аналогичный процесс разработки возможных мысленных вариантов в некоей гипотетической вселенной («мозг Бога») и затем выбирала между ними. Таким образом, мы не должны называть этот процесс «выбором», хотя с практической точки зрения этот термин удобен, поскольку он вызывает множество ассоциаций.

Как насчет калькулятора, запрограммированного на вычисление десятичной дроби корня из двух? Или шахматной программы? Можно сказать, что здесь мы имеем дело всего лишь с усложненными «шариками,» катящимися с усложненных горок. На деле, аргументы против выбора здесь еще сильнее, чем в предыдущем случае. Если вы попробуете повторить эксперимент с шариком, то, без сомнения, получите иные результаты: шарик покатится по новой дорожке. В то же время, сколько бы раз вы не включали калькулятор, вычисляющий квадратный корень из двух, результат всегда будет одинаковым. Кажется, что шарик выбирает иной путь, как бы аккуратно вы ни повторяли условия первого спуска, в то время как программа действует совершенно одинаково каждый раз.

В случае сложных шахматных программ есть несколько возможностей. Если вы начнете вторую партию теми же ходами, что и первую, некоторые программы будут просто повторять свои ходы. Незаметно, чтобы они чему-нибудь учились или стремились к разнообразию. Другие программы имеют устройства, обеспечивающие некоторое разнообразие, но это делается чисто механически, а не по желанию программы. Параметры такой программы можно вернуть в начальное состояние, словно она играет в первый раз, и она опять будет повторять точно те же ходы. Существуют также программы, которые учатся на своих ошибках и меняют стратегию в зависимости от результата партии. Они не будут повторять ходов, если в первый раз эти ходы привели к проигрышу. Разумеется, и здесь можно «перевести часы назад», стерев все изменения в памяти, представляющие новое знание, так же, как можно было вернуть к нулю генератор произвольных чисел в предыдущем случае, --- однако это было бы довольно недружелюбным поступком по отношению к машине. Кроме того, можно ли считать, что \emph{вы} смогли бы изменить любое из \emph{ваших} прошлых решений, если бы каждая деталь --- включая, разумеется, ваш мозг --- была бы возвращена к начальному состоянию их принятия?

Но вернемся к вопросу о том, применимо ли сюда слово «выбор». Если программы --- не более, чем «сложные шарики, скатывающиеся со сложных горок», то есть ли у них выбор? Конечно, ответ всегда будет субъективен, но я бы сказал, что сюда подходят те же соображения, как и в случае шарика. Однако должен добавить, что использование слова «выбор» здесь весьма привлекательно, хотя это слово и является только удобным сокращением. То, что шахматная программа, в отличие от шарика, заглядывает вперед и выбирает одну из ветвей сложного дерева возможностей, делает её более похожей на одушевленное существо, чем на программу, вычисляющую квадратный корень из двойки. И~всё~же здесь еще нет ни глубокого самосознания, ни чувства свободной воли.

Теперь давайте вообразим робота, снабженного набором символов. Он помещается в Т-образный лабиринт. Вместо того, чтобы идти за поощрением, расположенным в одном из концов Т, робот запрограммирован таким образом, что он идет налево, когда следующая цифра корня из двойки четная, и направо, когда она нечетная. Робот умеет изменять ситуацию в своих символах таким образом, что может наблюдать за процессом решения. Если каждый раз, когда он приближается к развилке, спрашивать его: «Знаешь ли ты, куда ты сейчас повернешь?», --- он будет отвечать «Нет.» Затем он должен будет включить процедуру «решение», вычисляющую следующую цифру квадратного корня из двойки, и затем принять решение. О внутреннем механизме принятия решения роботу ничего не известно --- в его системе символов этот механизм выглядит как черный ящик, таинственным и, по-видимому, произвольным образом выдающий команды «направо» или «налево.» Если символы робота не способны установить связи между его решениями и чередованием четных и нечетных цифр в корне из двойки, бедняга будет недоумевать перед своим «выбором». Но можно ли сказать, что этот робот на самом деле что-либо выбирает? Поставьте себя на его место. Если бы вы находились в шарике, катящемся с горы, и могли бы наблюдать его путь, не имея никакой возможности на него повлиять, сказали бы вы, что шарик выбирает дорогу? Разумеется, нет. Если вы не можете повлиять на выбор пути, то совершенно все равно, существуют ли символы.

Теперь мы модифицируем нашего робота, позволив символам --- в том числе, символу его самого --- влиять на его решения. Перед нами оказывается пример действующей по законам физики программы, которая гораздо ближе подходит к сути проблемы выбора, чем предыдущие примеры. Когда на сцену выходит блочное самовосприятие робота, мы можем идентифицировать себя с ним, поскольку сами действуем подобным образом. Это больше не похоже на вычисление квадратного корня из двойки, где никакие символы не влияли на результат. Однако, если бы мы взглянули на программу нашего робота на низшем уровне, то обнаружили бы, что она выглядит почти так же, как и программа для вычисления корня из двойки. Она выполняет команду за командой и результатом является «налево» или «направо». Но на высшем уровне мы видим, что в оценке ситуации и в принятии решения участвуют символы. Это коренным образом меняет наше восприятие программы. На этом этапе на сцену выходит \emph{значение} , похожее на то, с каким имеет дело человеческий разум.

Водоворот Гёделя, где скрещиваются все уровни

Если некая внешняя сила теперь предложит роботу пойти налево («Л»), это предложение будет направлено в крутящуюся массу взаимодействующих символов. Там, как лодка, затянутая в водоворот, оно неизбежно окажется втянутым во взаимодействие с символом, представляющим самого робота. Здесь «Л» попадает в Запутанную Иерархию символов, где оно передается наверх и вниз. Само-символ не способен наблюдать за всеми внутренними процессами; таким образом, когда принято конечное решение --- «Л»,~«П» или что-либо вне системы, --- система не способна сказать, откуда оно взялось. В отличие от стандартной шахматной программы, которая не следит за собой и не знает, почему она выбирает тот или иной ход, эта программа имеет некоторое понятие о собственных идеях; однако она не может уследить за всеми деталями идущих в ней процессов. Не понимая их полностью, она воспринимает эти процессы интуитивно. Из этого равновесия между само-пониманием и само-непониманием рождается чувство свободной воли.

Представьте, например, писателя, старающегося передать некие идеи, представленные набором образов у него в голове. Он не уверен, как эти образы ухитряются гармонично сочетаться в его воображении, и начинает экспериментировать, выражая вещи по-разному, пока не остановится на окончательном варианте. Знает ли он, почему выбрал именно этот вариант? Только приблизительно. Большая часть источников его решения, подобно айсбергу, находится глубоко под водой, невидимая глазу, --- и он об этом знает. Или представьте себе программу-композитора. Ранее мы уже это обсуждали, спрашивая, когда можно будет назвать эту программу композитором, а не простым инструментом человеческого сочинителя. Возможно, что мы сможем согласиться с её самостоятельностью, когда в программе появится самосознание, основанное на взаимодействии символов, и она достигнет равновесия между само-пониманием и само-непониманием. Неважно, если система действует по детерминистским законам, мы говорим, что она делает выбор, когда можем \emph{идентифицировать себя с описанием процессов, происходящих на высшем уровне работающей программы} .

На низшем уровне, уровне машинного языка, эта программа будет выглядеть точно так же, как любая другая, только на высшем, «блочном» уровне могут возникнуть такие качества, как «воля», «интуиция» и «творческие способности».

Идея в том, что именно «водоворот» само-символа порождает запутанность и «Гёдельность» мышления Меня иногда спрашивают:«Автореферентность --- очень интересная и забавная штука, но действительно ли вы считаете, что в этом есть что-то серьезное?» Безусловно. Я думаю, что именно это окажется в сердце Искусственного Интеллекта и в фокусе всех усилий направленных на понимание того, как работает человеческий разум. Именно поэтому Гёдель так органично вплетен в ткань моей книги.

Водоворот Эшера, где скрещиваются все уровни

\emph{Рис. 142. М. К. Эшер. Картинная галерея (литография, 1956).}

Поразительно красивая и в то же время странно тревожащая иллюстрация «глаза» циклона, порожденного Запутанной Иерархией, дана нам Эшером в его «Картинной галерее» (рис. 142). На этой литографии изображена картинная галерея где стоит молодой человек, глядя на картину корабля в гавани небольшого городка, может быть, мальтийского, судя по архитектуре, с его башенками, куполами и плоскими каменными крышами, на одной из которых сидит на солнце мальчишка, а двумя этажами ниже какая-то женщина --- может быть, мать этого мальчишки --- глядит из окна квартиры, расположенной прямо над картинной галереей, где стоит молодой человек, глядя на картину корабля в гавани небольшого городка, может быть, мальтийского --- Но что это!? Мы вернулись к тому же уровню, с которого начинали, хотя логически этого никак не могло случиться. Давайте нарисуем диаграмму того, что мы видим на этой картине (рис 143).

\emph{Рис. 143. Абстрактная диаграмма «Картинной галереи» М. К. Эшера.}

На этой диаграмме показаны три вида включения. Галерея \emph{физически} включена в город («включение»); город \emph{художественно} включен в картину («изображение»); картина \emph{мысленно} включена в человека («представление»). Хотя эта диаграмма может показаться точной, на самом деле она произвольна, поскольку произвольно количество показанных на ней уровней. Ниже представлен другой вариант верхней половины диаграммы (рис. 144):

\emph{Рис. 144. Сокращенная версия предыдущей диаграммы.}

Мы убрали уровень «города»; хотя концептуально он полезен, без него можно вполне обойтись. Рис. 144 выглядит так же, как диаграмма «Рисующих рук»: это двухступенчатая Странная Петля. Разделительные знаки произвольны, хотя и кажутся нам естественными. Это видно яснее из еще более сокращенной диаграммы «Картинной галереи»:

\emph{Рис. 145. Дальнейшее сокращение рис. 143.}

Парадокс картины выражен здесь в крайней форме. Но если картина «включена в саму себя», то молодой человек тоже включен сам в себя? На этот вопрос отвечает рис. 146.

\emph{Рис. 146. Другой способ сокращения рис. 143.}

Здесь мы видим молодого человека «внутри самого себя», в том смысле, какой получается от соединения трех аспектов «внутренности». Эта диаграмма напоминает нам о парадоксе Эпименида с его одноступенчатой автореференцией, в то время как двухступенчатая диаграмма похожа на пару утверждений, каждое из которых ссылается на другое. Затянуть Петлю туже не удается, но можно её ослабить, вводя любое количество промежуточных уровней, таких как «рама картины», «аркада» и «здание». Сделав так, мы получим многоступенчатые Странные Петли, диаграммы которых изоморфны «Водопаду» (рис. 5) или «Спуску и подъему» (рис. 6) Количество ступеней определяется нашим чувством того, что «естественно», что может варьироваться в зависимости от контекста, цели, или нашего настроения. В конечном итоге, восприятие уровней --- это вопрос интуиции и художественного вкуса.

Оказываются ли зрители, глядящие на «Картинную галерею,» затянутыми «в самих себя»? На самом деле, этого не происходит. Нам удается избежать этого водоворота благодаря тому, что мы находимся вне системы. Глядя на картину, мы видим то, что незаметно молодому человеку, --- например, подпись Эшера «МСЕ» в центральном «слепом пятне». Хотя это пятно кажется дефектом, скорее всего, дефект заключается в наших ожиданиях, поскольку Эшер не мог бы закончить этот фрагмент картины без того, чтобы не вступить в противоречие с правилами, по которым он её создавал. Центр водоворота остается --- и должен оставаться --- неполным. Эшер мог бы сделать его сколь угодно малым, но избавиться от него совсем он не мог. Таким образом мы, глядя снаружи, видим, что «Картинная галерея» неполна, чего молодой человек на картине заметить не в состоянии. Здесь Эшер дал художественную метафору Теоремы Геделя о неполноте. Поэтому Эшер и Гёдель так тесно переплетены в моей книге.

Водоворот Баха, где скрещиваются все уровни

Глядя на диаграммы Странных Петель, мы не можем не вспомнить о Естественно Растущем Каноне из «Музыкального приношения». Его диаграмма состояла бы из шести ступеней, как показано на рис. 147. К сожалению, когда канон возвращается к до, он оказывается на октаву выше, чем в начале.

\emph{Рис. 147. Схема гексагональной модуляции Баховского Естественно Растущего Канона выглядит как настоящая Странная Петля, если использовать тональную систему Шепарда.}

Однако возможно сделать так, что Канон вернется точно к началу, если использовать так называемую \emph{тональную систему Шепарда} , названную в честь её автора, психолога Роджера Шепарда. Принцип тонов Шепарда показан на рис. 148. Он заключается в том что параллельные гаммы играются в нескольких различных октавах. Каждая нота имеет собственную независимую интенсивность, по мере того, как мелодия становится выше эта интенсивность меняется. Таким образом вы добиваетесь того что высшая октава постепенно переходит в низшую. Как раз в тот момент, когда вы ожидаете оказаться на октаву выше, интенсивности изменились так, что вы оказываетесь в точности там же, где начали. Так можно «бесконечно подниматься», никогда не оказываясь выше! Можете попробовать сыграть это на пианино. Еще лучше получается, когда тона точно воспроизводятся с помощью компьютера. При этом достигается удивительно полная иллюзия.

Это замечательное музыкальное открытие позволяет сыграть Естественно Растущий Канон так что, «поднявшись» на октаву, он сливается сам с собой. Эта идея, принадлежащая мне и Скотту Киму, была приведена в исполнение с помощью компьютерной музыкальной системы и результат был записан на магнитофон. Получившийся эффект едва различим, но вполне реален. Интересно то, что сам Бах, по-видимому, в некотором роде осознавал возможность подобных гамм, поскольку в его музыке можно найти пассажи разрабатывающие приблизительно такую же идею --- например в середине «Фантазии из органной \enquote*{Фантазии и фуги в соль миноре}».

Ханс Теодор Давид своей книге «\enquote*{Музыкальное приношение} И. С. Баха» (Hans Theodore David «J.S. Bach's \enquote*{Musical Offering}») пишет:

На всем протяжении Музыкального приношения читатель, исполнитель или слушатель должен искать Королевскую тему во всех её формах. Таким образом все это произведение ---~ricercar в первоначальном буквальном смысле слова.\footnote{H. T. David, «J. S. Bach's \enquote*{Musical Offering}», стр. 43.}

Я думаю, что это верно, --- мы никогда не можем достаточно глубоко заглянуть в «Музыкальное приношение». Когда мы думаем, что поняли его полностью, мы обнаруживаем в нем нечто новое. Например, в конце того самого «Шестиголосного ричеркара», который Бах отказался импровизировать, он искусно запрятал собственное имя, разделенное между двумя верхними голосами. В «Музыкальном приношении» множество уровней, там можно найти игру с нотами и буквами, хитроумные вариации на Королевскую тему, оригинальные типы канонов, удивительно сложные фуги, красоту и крайнюю глубину чувства, в нем даже присутствует наслаждение многоуровневостью произведения. «Музыкальное приношение» --- это фуга фуг, Запутанная Иерархия, подобная Запутанным Иерархиям Эшера и Геделя интеллектуальная конструкция, напоминающая мне о прекрасной многоголосной фуге человеческого разума. Именно поэтому Гёдель, Эшер и Бах сплетены в моей книге в эту Бесконечную Гирлянду.

\emph{Рис. 148. Два полных цикла тональных гамм Шепарда в нотации для рояля. Громкость каждой ноты пропорциональна её местонахождению: в тот момент, когда верхний голос сходит на нет, очень тихо вступает новый нижний голос. (Напечатано с помощью программы Дональда Бирна «СМУТ»).}

\end{document}
