\documentclass[../main.tex]{subfiles}
\begin{document}

\DialogueChapter{Праздничная Кантататата\ldots{}}

\centerblock{
    \emph{В один прекрасный майский день Черепаха и Ахилл встречаются, прогуливаясь по лесу. Ахилл, разодетый в пух и прах, пританцовывает под звуки мелодии, которую он сам себе напевает под нос. На его пиджаке прицеплен огромный круглый значок со словами «Сегодня~--- мой день рождения!»}
}

\begin{dialogue}

\speak{Черепаха} Приветствую вас, Ахилл! Что это вы сияете, как начищенный пяпятак? У вас, случайно, не день рождения?

\emph{Ахилл} : Да, да! Да, сегодня у меня день рождения!

\emph{Черепаха} : Я так и думала, из-за значка на вашем пиджаке. Кроме того, вы напеваете тему Баховской «Праздничной кантаты», написанной в 1727 году на день рождения Саксонского короля Августа, которому тогда исполнилось 57 лет.

\emph{Ахилл} : Вы правы. Мы с королем родились в один день, поэтому ЭТА «Праздничная кантата» имеет двойное значение. Однако я вам не скажу, сколько мне лет.

\emph{Черепаха} : Хорошо, но мне бы хотелось узнать вот что: могу ли я заключить из того, что вы мне до сих пор сообщили, что сегодня ваш день рождения?

\emph{Ахилл} : Конечно, можете. Сегодня ДЕЙСТВИТЕЛЬНО мой день рождения.

\emph{Черепаха} : Прекрасно. Я так и подозревала. Так что теперь я заключу, что сегодня ваш день рождения, если только это не\ldots{}

\emph{Ахилл} : Если только это не --- что?

\emph{Черепаха} : Если только это не будет слишком поспешным заключением. Знаете ли, черепахи не любят делать поспешных заключений. (Мы вообще не любим спешить, и особенно в наших заключениях.) Так что позвольте мне вас спросить, зная вашу любовь к логическому мышлению, разумно ли заключить из ваших предыдущих высказываний, что сегодня ваш день рождения?

\emph{Ахилл} : Мне кажется, я улавливаю некую схему в ваших вопросах, г-жа Черепаха. Но вместо того, чтобы делать поспешные заключения, я постараюсь понять ваш вопрос буквально и ответить на него прямо: ДА.

\emph{Черепаха} : Чудно! Чудно! Мне нужно знать только еще одну вещь, чтобы быть вполне уверенной в том, что сегодня ---

\emph{Ахилл} : Да, да, да, да\ldots{} Я уже представляю себе, что вы сейчас спросите. Я покажу вам, что я уже не так прост, как тогда, когда мы обсуждали Эвклидово доказательство.

\emph{Черепаха} : Кто когда-либо считал вас простаком? Как раз наоборот --- я считаю вас экспертом в логическом мышлении, знатоком науки верных заключений, кладезем знаний о правильных методах рассуждения\ldots{} По правде говоря, Ахилл, по моему мнению вы --- просто гигант мысли, титан искусства рациональных размышлений И только лишь поэтому я хочу вас спросить «Дают ли ваши предыдущие высказывания достаточно оснований для того, чтобы я без дальнейших колебаний могла заключить, что сегодня ваш день рождения?»

\emph{Ахилл} : Вы меня совсем раздавили своей тяжеловесной похвалой --- подавили, я имею в виду. Но я удивлен повторяющимся характером ваших вопросов --- по-моему, вы и сами могли ответить «да» на каждый из них.

\emph{Черепаха} : Разумеется, могла бы, Ахилл. Но это было бы Тыканием Пальцем В Небо --- а Черепахи этого терпеть не могут. Черепахи допускают только Разумные Догадки. О, мощь Разумных Догадок! Вы не представляете себе, сколько людей забывает учитывать все Важные Факторы, когда они строят свои предположения.

\emph{Ахилл} : Мне кажется, что во всей этой белиберде был только один Важный Фактор --- мое первое утверждение.

\emph{Черепаха} : Точнее, это по меньшей мере ОДИН из факторов, который мне необходимо учесть --- но неужели вы хотите, чтобы я упускала из вида Логику, эту славную науку древних? Логика всегда являлась Важным Фактором при построении Разумных Догадок, и, поскольку я имею счастье находиться в компании известного эксперта по Логике, думаю, что будет только логично этим воспользоваться и подтвердить мою интуицию, прямо спросив у него, права ли я. Так что позвольте мне, наконец, обратиться к вам с прямым вопросом «Позволяют ли предыдущие суждения заключить, что сегодня ваш день рождения?»

\emph{Ахилл} : И еще раз, ДА! Но честно говоря, у меня складывается впечатление, что вы сами могли ответить на этот вопрос, как и на все предыдущие.

\emph{Черепаха} : О, что за удивительные слова! Желала бы я быть такой мудрой, как вы предполагаете. Но будучи только простой смертной Черепахой, глубоко невежественной и желающей принять во внимание все Важные Факторы, я нуждалась в ваших ответах на все эти вопросы.

\emph{Ахилл} : В таком случае, позвольте мне прояснить ситуацию раз и навсегда ответом на этот и на все последующие подобные вопросы является ДА.

\emph{Черепаха} : Великолепно! Одним ударом, со свойственным вам блеском, вам удалось разобраться во всей этой путанице. Надеюсь вы не возражаете, если я назову этот изобретательный трюк СХЕМОЙ ОТВЕТОВ. Она превращает положительные ответы на первый, второй, третий и так далее вопросы в один единственный положительный ответ. На самом деле, поскольку эта схема является завершающим аккордом наших рассуждений, она заслуживает называться Схемой Ответов Омега, поскольку~«\&\#969;» --- последняя буква греческого алфавита (Бог мой, кому я это объясняю!)

\emph{Ахилл} : Мне не важно как вы это назовете, но какое облегчение, что вы наконец согласились с тем, что сегодня мой день рождения, и мы можем поговорить о чем-нибудь другом --- например, о том что вы мне подарите.

\emph{Черепаха} : Погодите --- не так быстро! Я СОГЛАШУСЬ, что сегодня ваш день рождения --- при одном условии.

\emph{Ахилл} : Каком? Не просить подарка?

\emph{Черепаха} : Вовсе нет. Наоборот, я собиралась пригласить вас на шикарный праздничный ужин, после того, как я буду убеждена, что одновременное знание всех этих положительных ответов (утверждаемое схемой \&\#969;) позволит мне прямо и без дальнейших экивоков заключить, что сегодня ваш день рождения. Так оно и есть, не правда ли?

\emph{Ахилл} : Разумеется.

\emph{Черепаха} : Хорошо. Предположим, что я получила ответ \&\#969; + 1. Вооруженная им, я могу приступить к принятию гипотезы, что сегодня ваш день рождения, если только это позволено сделать. Что вы мне посоветуете, Ахилл?

\emph{Ахилл} : Что такое? Я-то думал, что мне удалось вырваться из ваших бесконечных сетей. Почему же вас не удовлетворяет ответ \&\#969; + 1? Ну, хорошо: я дам вам не только положительный ответ \&\#969; + 2, но и \&\#969; + 3, \&\#969; + 4, и так далее.

\emph{Черепаха} : Как это щедро с вашей стороны, Ахилл. А ведь сегодня как раз ваш день рождения, когда это Я должна преподносить ВАМ подарки, а не наоборот. Скорее, я ПОДОЗРЕВАЮ, что сегодня ваш день рождения. Наверное, теперь, когда я вооружена новой Схемой Ответов, которую я назову «Схемой Ответов 2\&\#969;», я могу заключить, что сегодня ваш день рождения. Но скажите мне, пожалуйста, Ахилл: действительно ли Схема Ответов 2\&\#969; позволяет мне совершить этот огромный скачок, или же я что-то пропускаю?

\emph{Ахилл} : Больше вы меня не проведете, г-жа Черепаха. Как я погляжу, этой глупой игре конца нет! Я решил покончить с этим раз и навсегда и дать вам такую Схему Ответов, которая одним ударом расправится со всеми предыдущими Схемами. Я дам вам одновременно Схему Ответов \&\#969;, 2\&\#969;, З\&\#969;, 5\&\#969; и~т.\,д. С этой Мета-Схемой-Ответов мне уж наверняка удастся ВЫСКОЧИТЬ из системы, перехитрить эту глупую игру, в сети которой вы думали меня уловить --- теперь-то вам ПРИДЕТСЯ в этом признаться!

\emph{Черепаха} : О, Боже мой! Какая честь для меня --- оказаться обладательницей такой мощной Схемы Ответов! Мне кажется, что человеческая мысль редко изобретала что-либо подобное. Я восхищена её гигантской мощью! Вы не возражаете, если я дам имя вашему подарку?

\emph{Ахилл} : Конечно, нет.

Черепаха: Тогда я назову его «Схемой Ответов \&\#969;\textsuperscript{2}.» И мы сможем перейти к другим темам --- как только вы скажете мне, что обладание Схемой Ответов \&\#969;\textsuperscript{2} позволит мне заключить, что сегодня ваш день рождения.

\emph{Ахилл} : Увы мне, увы!.. Кончатся ли когда-нибудь эти мученья? Что еще меня ожидает?

\emph{Черепаха} : С удовольствием скажу вам. Дело в том, что после вашей Схемы Ответов~\&\#969;\textsuperscript{2} идет ответ \&\#969;\textsuperscript{2} + 1, затем \&\#969;\textsuperscript{2} + 2\ldots{} Разумеется, вы можете собрать их в кучу под названием Схема Ответов~\&\#969;\textsuperscript{2} + \&\#969;, после чего могут последовать несколько других «куч», как, например,~\&\#969;\textsuperscript{2} + 2\&\#969;, \&\#969;\textsuperscript{2} + З\&\#969; и так далее. Рано или поздно вы придете к Схеме Ответов 2\&\#969;\textsuperscript{2}, затем З\&\#969;\textsuperscript{2}, 4\&\#969;\textsuperscript{2} и так далее. Существуют также дальнейшие Схемы Ответов, такие, как \&\#969;\textsuperscript{3}, \&\#969;\textsuperscript{4}, \&\#969;\textsuperscript{5} Так может продолжаться довольно долго.

\emph{Ахилл} : Могу себе представить. Наверное, через некоторое время так можно дойти до Схемы Ответов \&\#969;\textsuperscript{\&\#969;}.

\emph{Черепаха} : Разумеется.

\emph{Ахилл} : А затем~\&\#969;\textsuperscript{\&\#969;\&\#969;} и так далее?

Черепаха: Вы довольно быстро ухватили мою идею. Если не возражаете, хочу вам кое-что предложить. Почему бы вам не соединить их все в одну-единственную Схему Ответов?

\emph{Ахилл} : Хорошо, хотя я начинаю сомневаться, есть ли от этого какая-нибудь польза.

\emph{Черепаха} : Мне кажется, что нам будет трудненько найти имя для этой Схемы. Может быть, нам придется просто назвать её Схема Ответов \&\#949;\textsubscript{0}.

\emph{Ахилл} : Черт побери! Каждый раз, когда вы даете очередной Схеме Ответов, имя это разбивает мои надежды на то, что мой ответ вас, наконец, удовлетворит. Почему бы нам просто не оставить Схему безымянной?

\emph{Черепаха} : Никак невозможно, Ахилл. Как же мы будем говорить об этой схеме, если у нее не будет имени? Кроме того, именно в этой Схеме есть что-то особенно завершенное и прекрасное. Было бы некрасиво оставить её безымянной! А вы не хотели бы совершать некрасивых поступков, особенно в день вашего рождения не правда ли? Неужели сегодня ваш день рождения? Кстати о днях рождения, сегодня мой день рождения!

\emph{Ахилл} : Неужели?

\emph{Черепаха} : Да. Вообще-то, на самом деле, сегодня день рождения моего дяди, но это почти одно и то же. Как насчет того, чтобы пригласить меня на шикарный праздничный ужин?

\emph{Ахилл} : Подождите минутку г-жа Ч! Сегодня МОЙ день рождения, и это Вы должны меня приглашать!

\emph{Черепаха} : Но вам так и не удалось убедить меня в том, что вы говорите правду! Вы развели страшную путаницу, выдумали какие-то Схемы Ответов\ldots{} Я всего-навсего хотела узнать, не день рождения ли у вас сегодня, но вам удалось меня совершенно сбить с толку. Как вам только не стыдно? Так или иначе я была бы счастлива, если бы вы пригласили меня на ужин сегодня вечером.

\emph{Ахилл} : Ну что ж. Я знаю одно местечко. Там готовят самые экзотические супы, и я точно знаю какого супчика мне бы сейчас хотелось!

\end{dialogue}

\end{document}
