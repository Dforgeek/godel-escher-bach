\documentclass[../main.tex]{subfiles}
\begin{document}

\DialogueChapter{Двухголосная инвенция}

\par\begingroup\centering\Large%
    или

    \emph{Что Черепаха сказала Ахиллу}

    \large%
    (записано Льюисом Кэрроллом\footnote{Lewis Carroll «What the Tortoise Said to Achilles» в журнале «Mind» n s 4 1895, стр. 255 6})
\par\endgroup

\centerblock{%
    \emph{Ахилл перегнал Черепаху и с комфортом уселся отдыхать на её широкой спине.}
}

\begin{Dialogue}

«Так вам всё же удалось добежать до финиша?» \--- сказала Черепаха. «Несмотря на то, что дистанция состояла из бесконечного ряда отрезков? Я-то думала, какой-то умник доказал, что это невозможно сделать?»

«Это ВОЗМОЖНО сделать», \--- сказал Ахилл: «И я это СДЕЛАЛ! \emph{Solvitur ambulando}. Видите ли, дистанции постоянно УМЕНЬШАЛИСЬ\ldots»

«А если бы они постоянно УВЕЛИЧИВАЛИСЬ?» \--- перебила Черепаха, \--- «Что тогда?»

«Тогда бы меня здесь ещё не было,» \--- скромно ответил Ахилл, \--- «А Вы уже успели бы обежать несколько раз вокруг света.»

«Вы весьма великодушны, Ахилл. Вы меня просто подавили\ldots{} я хочу сказать, придавили, поскольку вы нешуточный тяжеловес. А теперь, не угодно ли вам послушать про такую беговую дорожку, о которой большинство людей воображают, что могут преодолеть её в два-три шага, когда на самом деле она состоит из бесконечного числа расстояний, где каждое последующее больше предыдущего?»

«С превеликим удовольствием,» \--- ответствовал греческий воин, доставая из шлема (в те дни мало кто из греческих воинов мог похвастаться карманами) огромный блокнот с карандашом. «Приступайте к своему рассказу, да говорите, пожалуйста, помедленнее \--- ведь стенография ещё не изобретена!»

«Этот прекрасный Первый Постулат Эвклида\ldots» \--- пробормотала мечтательно Черепаха, \--- «вы восхищаетесь Эвклидом?»

«Страстно! Постольку, конечно, поскольку можно восхищаться трудом, который будет опубликован лишь через несколько столетий\ldots»

«Давайте, в таком случае, рассмотрим первые два пункта его доводов, и выводы, которые из них следуют. Будьте так любезны, запишите их к себе в блокнот \--- для удобства обозначим их А, В и Z:

\begin{statements}
    \item[(A)] Вещи, равные одному и тому же, равны между собой.
    \item[(B)] Две стороны этого треугольника суть вещи, равные одному и тому~же.
    \item[(Z)] Две стороны этого треугольника равны между собой.
\end{statements}

Читатели Эвклида согласятся, я думаю, что Z логически следует из А и В, так что тот, кто согласен с истинностью А и В, ДОЛЖЕН считать истинным и Z?»

«Несомненно! Уж с ЭТИМ-то легко согласится любой старшеклассник \--- как только старшие классы будут изобретены, каких-нибудь пару тысяч лет спустя.»

«И если какой-нибудь читатель не принимает А и В за истинные, он, тем не менее, должен согласиться с тем, что ВЗЯТАЯ ЦЕЛИКОМ, эта последовательность имеет смысл?»

«Без сомнения, такого читателя можно вообразить. Он мог бы сказать: \enquote*{Я~принимаю за истинное Гипотетическое Утверждение, что ЕСЛИ А~и~В истинны, то Z должно быть тоже истинно.} Такой читатель поступил бы мудро, если бы он оставил Эвклида и занялся футболом».

«А что, если какой-нибудь другой читатель сказал бы: \enquote*{Я принимаю за истинные А~и~В, но НЕ Гипотетическое Утверждение}?»

«Наверное, и такой читатель мог бы существовать. Ему, впрочем, тоже было бы лучше заняться футболом.»

«И никакой из этих читателей ПОКА не обязан соглашаться с тем, что логически Z должно быть истинно?»

«Совершенно верно,» \--- кивнул Ахилл.

«Теперь представьте на минуту, что я \--- тот второй читатель, и попробуйте логически заставить меня признать, что Z истинно.»

«Черепаха, играющая в футбол, была бы\ldots» \--- начал Ахилл.

«\ldots{} совершеннейшей аномалией, конечно,» \--- торопливо перебила Черепаха. «Не будем отвлекаться; сначала давайте разберемся с Z, а потом уж поговорим о футболе!»

«Я должен заставить вас принять Z, не так ли?» \--- задумчиво пробормотал Ахилл. «И вы утверждаете, что принимаете А и В, но тем не менее не принимаете Гипотетическое Утверждение\ldots»

«Назовем его С», \--- вставила Черепаха.

«Но вы не принимаете

\begin{statements}
    \item[(С)] Если А и В истинны, следовательно Z должно быть истинно.»
\end{statements}

«Именно это я и утверждаю,» \--- сказала Черепаха.

«В таком случае я должен попросить вас согласиться с С.»

«Я, пожалуй, уважу вашу просьбу, как только вы занесете её в свой блокнот. Кстати, что у вас там ещё записано?»

«Только несколько заметок на память,» \--- сказал Ахилл, нервно шурша страницами: «несколько заметок о\ldots{} о сражениях в которых я отличился!»

«Здесь полно чистых страниц, как я погляжу!» \--- радостно заметила Черепаха. «Нам понадобятся ВСЕ они, до последней странички!» (Ахилл содрогнулся.) «Теперь пишите за мной:

\begin{statements}
    \item[(A)] Вещи, равные одному и тому же, равны между собой.
    \item[(B)] Две стороны этого треугольника суть вещи, равные одному и тому же.
    \item[(C)] Если А и В истинны, следовательно Z должно быть истинно.
    \item[(Z)] Две стороны этого треугольника равны между собой.»
\end{statements}

«Вы должны бы называть последнее утверждение D, а не Z, поскольку оно прямо следует за первыми тремя. Если вы принимаете А, В, и С, вам ПРИДЕТСЯ принять~Z.»

«Почему это мне \enquote*{придется}?»

«Потому что Z ЛОГИЧЕСКИ следует из них. Если А, и В, и С истинны, Z~ДОЛЖНО быть истинно. С этим-то вы, надеюсь, не станете спорить?»

«Если А, и В, и С истинны, Z ДОЛЖНО быть истинно,» \--- в раздумьи повторила Черепаха. «Это ещё одно Гипотетическое Утверждение, не правда ли? И если я его не приму, я все ещё могу считать истинными А, В и С, но не принимать Z, не так ли, мой друг?»

«Пожалуй, что и так,» \--- согласился простодушный герой, \--- «хотя такое упрямство было бы просто феноменально. Все же, это событие ВОЗМОЖНО\@. А раз так, я должен попросить вас принять ещё одно Гипотетическое Утверждение.»

«Прекрасно! Я согласен принять и это Утверждение, как только вы его запишете. Мы назовем его D.

\begin{statements}
    \item[(D)] Если А, и В, и С истинны, Z ДОЛЖНО быть истинно.
\end{statements}

Уже записали?»

«Записал, записал!» \--- радостно воскликнул Ахилл, вкладывая карандаш в футляр. «Наконец-то мы пришли к концу нашей воображаемой беговой дорожки! Теперь, когда вы принимаете А, и В, и С, и D, вы, КОНЕЧНО, принимаете и Z.»

«Неужели?» \--- спросила Черепаха с невинным видом. «Давайте-ка это выясним. Я принимаю А, и В, и С, и D. Что, если я ВСЕ ЕЩЁ отказываюсь принять~Z?»

«Тогда госпожа Логика возьмет вас за горло и ЗАСТАВИТ!» \--- торжествующе ответил Ахилл. «Логика скажет вам: \enquote*{У вас нет выхода. Теперь, когда вы согласились с А, и В, и С и D, вы ОБЯЗАНЫ согласиться с Z!} Так что у вас нет выбора, как видите.»

«То, что произносит госпожа Логика, уж конечно стоит того, чтобы быть ЗАПИСАНО,» \--- сказала Черепаха. «Так что, пожалуйста занесите и это в ваш блокнот. Мы назовем это

\begin{statements}
    \item[(E)] Если А и В и С и D истинны, Z должно быть истинным.
\end{statements}

«До тех пор, пока я не согласилась с ЭТИМ утверждением, я не обязана принимать Z за истинное. Теперь вы видите, что это совершенно НЕОБХОДИМЫЙ шаг?»

«Вижу, вижу\ldots» \--- сказал Ахилл, и в его голосе явственно послышались грустные нотки.

В этот момент рассказчику пришлось покинуть счастливую парочку, так как ему срочно нужно было в банк. Он снова попал в те места только через несколько месяцев. Доблестный герой Ахилл все ещё восседал на спине долготерпеливой Черепахи и писал в своем блокноте, который уже почти заполнился, а Черепаха говорила: «Записали последний шаг? Если я не сбилась со счета, у нас набралось уже 1001. Осталось всего каких-нибудь несколько миллионов\ldots{} Зато подумайте только, какую ОГРОМНУЮ пользу наша беседа принесет Логикам Девятнадцатого Века!»

«Не думаю, что кто-нибудь из них сможет разобраться во всей этой чепухе», \--- отвечал усталый воин, в отчаянии пряча лицо в ладонях. «Сделайте милость, разрешите мне позаимствовать каламбур, который в девятнадцатом столетии придумает знакомая Алисы, ваша кузина Черепаха Квази, и переименовать вас в г-жу Чепупаху.»

«Ахиллес, бедняга, вы видно совсем устали, такую вы несете ахиллею\ldots{} по этому поводу, я, пожалуй, позволю себе каламбур, до которого моя кузина Черепаха Квази не додумается, и переименую вас в Ахинесса.»

\end{Dialogue}

\end{document}
