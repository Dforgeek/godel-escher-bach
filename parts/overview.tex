\documentclass[../main.tex]{subfiles}
\begin{document}

\ifSubfilesClassLoaded{% subfile
    \frontmatter
    \chapterstyle{FrontMatterChapterStyle}
}{% main file
}

\chapter{Обзор}

\begin{center}
    \Large
    Часть I: ГЭБ
\end{center}

\textbf{Интродукция: Музыко-логическое приношение.}
Книга начинается с истории Баховского «Музыкального приношения». Бах неожиданно посетил короля Пруссии Фридриха Великого. Король предложил Баху тему для импровизации; результат явился основой этого великого творения. «Музыкальное приношение» и история его создания являются той темой, на которую я «импровизирую» в этой книге, создавая, таким образом, нечто вроде «Метамузыкального приношения». В интродукции обсуждается автореферентность и взаимодействие между различными уровнями у Баха; затем я перехожу к параллельным идеям в рисунках Эшера и Теореме Гёделя. Чтобы поместить последнюю в исторический контекст, дана краткая история логики и парадоксов. Это ведет к обсуждению механистической философии и компьютеров и спора о возможности создания искусственного интеллекта. В заключение я объясняю, как возникла идея этой книги и, в особенности, Диалогов.

\emph{Трехголосная инвенция.}
Бах написал пятнадцать трехголосных инвенций. В этом трехголосном Диалоге Черепаха и Ахилл \--- главные действующие лица моих Диалогов \--- «изобретаются» Зеноном (как на самом деле и произошло, для иллюстрации парадоксов Зенона о движении). Этот Диалог совсем коротенький; он дает читателю почувствовать дух последующих Диалогов.

\textbf{Глава I: Головоломка MU.}
Представлена простая формальная система, \textbf{MIU}; чтобы ближе ознакомиться с формальными системами, читателю предлагается найти решение некоей головоломки. Вводится несколько основных понятий: строчка, теорема, аксиома, правило вывода, деривация, формальная система, разрешающая процедура, работа внутри и вне системы.

\emph{Двухголосная инвенция.}
Бах написал также пятнадцать двухголосных инвенций. Этот двухголосный Диалог был написан не мной, а Люисом Кэрроллом в 1895 году. Кэрролл позаимствовал Ахилла и Черепаху у Зенона, а~я, в свою очередь, позаимствовал их у Кэрролла. Тема Диалога \--- отношения между рассуждениями, рассуждениями о рассуждениях, рассуждениями о рассуждениях о рассуждениях и так далее. В каком-то смысле парадокс Кэрролла параллелен парадоксу Зенона о невозможности движения, путем бесконечного регресса доказывая, что рассуждения невозможны. Этот парадокс очень красив; он упоминается в книге несколько раз.

\textbf{Глава II: Значение и форма в математике.}
Вводится новая формальная система (система~\textbf{pr}), ещё более простая, чем система MIU предыдущей главы. Её символы, вначале кажущиеся бессмысленными, приобретают значение благодаря форме тех теорем, в которых они находятся. Глубокая связь значения с изоморфизмом \--- наше первое важное открытие. В~этой главе обсуждаются многие темы, связанные со значением: истина, доказательство, манипуляция символами, а также само ускользающее понятие «формы».

\emph{Соната для Ахилла соло.}
Диалог, имитирующий сонату Баха для скрипки соло. Ахилл \--- единственный собеседник, поскольку это запись его реплик в телефонном разговоре с Черепахой. Речь идет о «рисунке» и «фоне» в разных контекстах \--- например, рисунки Эшера. Сам Диалог \--- пример такого различия, поскольку реплики Ахилла представляют «рисунок», а соответствующие воображаемые ответы Черепахи \--- «фон».

\textbf{Глава III: Рисунок и фон.}
Различие между рисунком и фоном в изобразительном искусстве сравнивается с различием между теоремами и не-теоремами в формальных системах. Вопрос «содержит ли рисунок ту же информацию, что и фон?» ведет к различию между рекурсивно перечислимыми и рекурсивными множествами.

\emph{Акростиконтрапунктус.}
Это центральный Диалог книги, поскольку он содержит множество перифразов Гёделева автореферентного построения и теоремы о неполноте. Один из них утверждает: «Для каждого патефона существует запись, которую он не может воспроизвести». Название Диалога \--- комбинация слов «акростих» и «контрапунктус» \--- латинское слово, использованное Бахом для названия многих фуг и канонов, составляющих «Искусство фуги». «Искусство фуги» несколько раз упоминается в Диалоге. Сам Диалог содержит хитрые трюки типа акростихов.

\textbf{Глава IV: Непротиворечивость, полнота и геометрия.}
Предыдущий Диалог разъясняется настолько, насколько это возможно на данном этапе. Это снова приводит к вопросу, когда и каким образом символы в формальных системах приобретают значение. Для иллюстрации труднообъяснимого понятия «неопределенных термов» используется история эвклидовой и неэвклидовой геометрии. Это ведет к идеям о непротиворечивости различных и, возможно, «соперничающих» геометрий. Это обсуждение разъясняет понятие неопределенных термов и их отношение к восприятию и мыслительным процессам.

\emph{Маленький гармонический лабиринт.}
Этот Диалог основан на органной пьесе Баха того же названия. Это забавное введение в понятие рекурсивных \--- то есть вложенных одна в другую \--- структур. Основная история, вместо того, чтобы закончиться, обрывается на полпути, так что читатель зависает в воздухе. Одна из историй-матрешек касается модуляций в музыке и, в особенности, в одной органной пьесе, заканчивающейся в неправильной тональности, так что слушатель зависает в воздухе.

\textbf{Глава V: Рекурсивные структуры и процессы.}
Идея рекурсии представлена в разных контекстах: музыкальные, лингвистические и геометрические структуры, математические функции, физические теории, компьютерные программы и~т.\,д.

\emph{Канон с интервальным увеличением.}
Ахилл и Черепаха пытаются ответить на вопрос: «Где содержится больше информации \--- в пластинке или в патефоне?» Этот странный вопрос возникает, когда Черепаха описывает пластинку с некоей оригинальной записью. Будучи проиграна на разных патефонах, эта запись воспроизводит две различные мелодии: В-А-С-H и C-A-G-E. Однако оказывается, что, в некотором смысле, эти две мелодии \--- «одно и то же».

\textbf{Глава VI: Местонахождение значения.}
Подробное обсуждение того, каким образом значение разделено между закодированным сообщением, дешифрующим механизмом и получателем этого сообщения. В качестве примеров приводятся цепочки ДНК, нерасшифрованные старинные надписи и пластинки, затерянные в космосе. Предполагается связь разума с «абсолютным» значением.

\emph{Хроматическая фантазия и фига.}
Короткий Диалог, почти ничем, кроме названия, не похожий на Баховскую «Хроматическую фантазию и фугу». Речь здесь идет о том, как правильно манипулировать высказываниями, чтобы они оставались истинными; в частности, обсуждается вопрос, существуют ли правила обращения с союзом~«и».

\textbf{Глава VII: Исчисление высказываний.}
Обсуждается, как слова, подобные~«и», могут управляться формальными правилами. Снова используются идеи изоморфима и автоматического приобретения значения символами в подобной системе. Между прочим, все примеры в этой главе \--- «дзентенции», суждения, взятые из коанов дзена. Это сделано специально; ирония в том, что коаны дзена намеренно нелогичны.

\emph{Крабий канон.}
Диалог, основанный на одноименной пьесе из «Музыкального приношения». Оба названы так, поскольку крабы (предположительно) ходят, пятясь. Краб впервые выходит на сцену в этом Диалоге. Возможно, что это самый насыщенный словесными трюками и игрой разных уровней Диалог в книге. Гёдель, Эшер и Бах тесно переплетены в этом коротеньком Диалоге.

\textbf{Глава VIII: Типографская теория чисел.}
Представляет расширенный вариант исчисления высказываний, так называемую «\acs{tnt}»\@. В~\acs{tnt} теоретико-численные рассуждения могут быть сведены к строгой манипуляции символами. Рассматриваются различия между формальными рассуждениями и человеческой мыслью.

\emph{Приношение МУ.}
В этом Диалоге вводятся несколько новых тем книги. Хотя, на первый взгляд, в нем обсуждаются дзен-буддизм и коаны, на самом деле это тонко завуалированное обсуждение теоремности и нетеоремности, истинности и ложности строчек теории чисел. Упоминается молекулярная биология \--- в особенности, Генетический Код. Сходство с «Музыкальным приношением» здесь только в названии и в автореферентных играх.

\textbf{Глава IX: Мумон и Гёдель.}
Разговор идет о странных идеях дзен-буддизма. Центральная фигура \--- монах Мумон, автор знаменитых комментариев к коанам. В~метафорическом смысле, идеи дзена напоминают определенные идеи в современной философии математики. После этого обсуждения вводится основная идея Гёделя \--- Геделева нумерация, и затем Теорема Гёделя впервые приводится целиком.


\begin{center}
    \Large
    Часть II: ЭГБ
\end{center}

\emph{Прелюдия\ldots}
Этот Диалог связан со следующим Оба они основаны на прелюдиях и фугах из Баховского «Хорошо темперированного клавира». Ахилл и Черепаха приносят подарок Крабу, у которого в это время в гостях Муравьед. Подарок оказывается записью «ХТК», и друзья решают сразу же её прослушать. Во время прелюдии они обсуждают строение прелюдий и фуг, Ахилл спрашивает, каким образом лучше слушать фугу: как одно целое или как сумму разных голосов? Этот спор между холизмом и редукционизмом затем продолжается в «Муравьиной фуге».

\textbf{Глава X: Уровни описания и компьютерные системы.}
Обсуждаются разные уровни восприятия картин, шахматных позиций и компьютерных систем. Последние затем объясняются подробно; это включает описание машинных языков, языков ассемблера, языков компилятора, операционных систем и так далее. Далее разговор переходит к другим типам сложных систем, таких как спортивные команды, ядра, атомы, погода и так далее. Возникает вопрос, как много существует промежуточных уровней, и существуют ли они вообще.

\emph{\ldots и Муравьиная фуга.}
Имитация музыкальной фуги: каждый голос вступает с одним и тем же замечанием. Рекурсивный рисунок вводит тему Диалога \--- холизм и редукционизм. Рисунок составлен из слов, которые, в свою очередь, состоят из меньших слов и так далее На четырех уровнях этой странной картинки появляются слова «ХОЛИЗМ», «РЕДУКЦИОНИЗМ» и «МУ». Затем разговор переходит к знакомой Муравьеда; мадам Мура Вейник \--- разумная муравьиная колония. Обсуждаются разные уровни её мыслительных процессов. В этом Диалоге есть множество приемов фуги, для подсказки читателю упоминаются те же самые приемы, звучащие в фуге, которую слушает четверка друзей. В конце «Муравьиной фуги», значительно измененные, появляются темы «Прелюдии».

\textbf{Глава XI: Мозг и мысль.}
Тема этой главы \--- «Как физическая аппаратура мозга может порождать мысли?» Сначала описываются крупномасштабные и мелкомасштабные структуры мозга. Затем выдвигается несколько гипотез об отношении понятий к нейронной деятельности.

\emph{Англо-франко-немецко-русская сюита.}
Интерлюдия, состоящая из трех переводов знаменитого стихотворения «Jabberwocky» Льюиса Кэрролла.

\textbf{Глава XII: Разум и мысль.}
Предыдущие стихотворения естественно подводят к вопросу: «Могут ли языки \--- или даже сам разум разноязычных людей \--- быть \enquote*{отображены} один на другой?» Как вообще возможна коммуникация между мозгами двух разных людей? Что между ними общего? Может ли мозг, в некоем объективном смысле, быть понят другим мозгом? Для возможного ответа используется географическая аналогия.

\emph{Ария с различными вариациями.}
Форма этого Диалога основана на «Гольдберг-вариациях» Баха, а его содержание имеет отношение к теоретико-численным задачам, подобным Гипотезе Гольдбаха. Основная цель этого гибрида \--- показать, как гибкость теории чисел опирается на тот факт, что поиски в бесконечном пространстве имеют множество вариантов. Некоторые из них оказываются бесконечными, некоторые \--- конечными, а другие находятся где-то посередке.

\textbf{Глава XIII: Блуп, Флуп и Глуп.}
Это названия трех компьютерных языков. Программы Блупа могут осуществлять только предсказуемо конечный поиск, в то время как программы Флупа способны на непредсказуемый или даже бесконечный поиск. В этой главе я стараюсь объяснить понятие примитивно рекурсивных и общерекурсивных функций в теории чисел, поскольку они очень важны для доказательства Теоремы Гёделя.

\emph{Ария в ключе G.}
В этом Диалоге словесно отражена автореферентная конструкция Гёделя. Эта идея принадлежит У.\,Я.\,О.~Квайну. Диалог служит прототипом следующей главы.

\textbf{Глава XIV: О формально неразрешимых суждениях \acs{tnt} и родственных систем.}
Название этой главы \--- адаптация заглавия статьи Гёделя 1931 года, где впервые появилась его теорема о неполноте. Тщательно рассматриваются две основные части доказательства. Показано, как из предположения о непротиворечивости \acs{tnt} вытекает то, что она (или любая похожая система) неполна. Обсуждаются отношения \acs{tnt} к эвклидовой и неэвклидовой геометрии, и значение теоремы Гёделя для философии математики.

\emph{Праздничная кантатата\ldots{}}
В которой Ахилл не может убедить скептически настроенную Черепаху в том, что сегодня его день рождения. Его повторные неудачные попытки предвосхищают повторяемость Гёделева аргумента.

\textbf{Глава XV: Прыжок из системы.}
Обсуждается повторяемость Гёделева аргумента, из чего вытекает, что \acs{tnt} не только неполна, но и в принципе непополнима. Анализируется и опровергается интересный аргумент Лукаса, использующего Теорему Гёделя для доказательства того, что человеческая мысль не может быть механизирована.

\emph{Благочестивые размышления курильщика табака.}
В этом Диалоге затрагиваются многие темы, относящиеся к автореферентности и самовоспроизводству. Среди примеров \--- телевизионные камеры, снимающие сами себя, а также вирусы (и другие подклеточные существа), способные на самосборку. Название Диалога происходит из стихотворения самого Баха, которое цитируется в тексте.

\textbf{Глава XVI: Авто-реф и Авто-реп.}
В этой главе обсуждается связь между разными типами автореференции и самовоспроизводящимися объектами (такими, как компьютерные программы или молекулы ДНК). Объясняются отношения между самовоспроизводящимся объектом и внешними механизмами, помогающими этому воспроизводству; особое внимание уделяется отсутствию между ними четкой границы. Тема этой главы \--- передача информации между различными уровнями подобных систем.

\emph{Магнификраб в пирожоре.}
Это название \--- игра слов; имеется в виду Баховский «Magnificat в ре-мажоре». Речь идет о Крабе, который, по-видимости, обладает магической способностью различать между истиннными и ложными высказываниями теории чисел. Читая их как музыкальные пьесы, он проигрывает их на флейте и определяет, «красивы» ли они.

\textbf{Глава XVII: Чёрч, Тюринг, Тарский и другие.}
Фантастический Краб предыдущего Диалога заменен здесь несколькими реальными людьми с удивительными математическими способностями. Тезис Чёрча-Тюринга, связывающий мозговую деятельность с вычислениями, представлен в нескольких версиях. Все они анализируются с точки зрения их последствий для возможности механического подражания мышлению и программирования на компьютере умения чувствовать и создавать прекрасное. Тема связи мозговой деятельности с вычислениями приводит к таким вопросам как Тюрингова Проблема Остановки или Теорема Истинности Тарского.

\emph{ШРДЛУ.}
Этот Диалог основан на статье Т.~Винограда о его программе ШРДЛУ; я изменил только несколько имен. В Диалоге некая компьютерная программа, на довольно впечатляющем языке, беседует с человеком о так называемом «мире кубиков». Кажется, что программа на самом деле понимает тот ограниченный мир, о котором говорит.

\textbf{Глава XVIII: Искусственный интеллект: краткий обзор.}
Эта глава начинается с обсуждения знаменитого «теста Тюринга» \--- предложенного пионером компьютеров Аланом Тюрингом способа определить, «думает» ли машина. Далее мы переходим к краткому обзору истории искусственного интеллекта. Обсуждаются программы, до какой-то степени умеющие играть в различные игры, доказывать теоремы, решать задачи, сочинять музыку, заниматься математикой и пользоваться естественным языком (английским).

\emph{Контрафактус.}
О том, как мы организуем наши мысли, воображая гипотетические варианты реальности. Это умение приобретает иногда странные формы, \--- как например, в характере Ленивца, этого страстного любителя блинчиков и ненавистника воображаемых ситуаций.

\textbf{Глава XIX: Искусственный интеллект: виды на будущее.}
Предыдущий Диалог затрагивает вопрос о том, как информация представлена на различных уровнях контекста. Это приводит к современной идее «фреймов». Для конкретности дан пример того, как зрительные головоломки решаются «методом фреймов». Затем обсуждается важный вопрос взаимодействия понятий вообще, что приводит к разговору о творческих способностях. В заключение дан список моих собственных предположительных «Вопросов и Ответов» на тему ИИ и разума в общем.

\emph{Канон Ленивца.}
Этот Диалог имитирует Баховский канон, в котором один голос повторяет ту же мелодию, что и другой, только «вверх ногами» и вдвое медленнее. Третий голос свободен. Ленивец произносит те же реплики, как и Черепаха, при этом отрицая (с свободном смысле слова) все, \textbf{что} она говорит, и говоря вдвое медленнее. Свободный голос \--- Ахилл.

\textbf{Глава XX: Странные Петли или Запутанные Иерархии.}
Грандиозный водоворот множества идей о иерархических системах и автореферентности. Речь идет о странной «путанице», возникающей, когда система начинает действовать сама на себя, \--- например, наука, изучающая науку, правительство, исследующее правительственные преступления, искусство, нарушающее законы искусства и, наконец, люди, размышляющие о собственном мозге и разуме. Имеет ли Теорема Гёделя какое-нибудь отношение к этой последней «путанице»? Связаны ли с этой Теоремой свободная воля и самосознание? В заключение Гёдель, Эшер и Бах снова связываются в одно целое.

\emph{Шестиголосный ричеркар.}
Этот Диалог \--- игра, изобилующая многими идеями, которыми проникнута эта книга. Он является повторением истории «Музыкального приношения», с которой начинается книга. В то же время это «перевод» в слова самой сложной части «Музыкального приношения» \--- «Шестиголосного ричеркара». Подобная двойственность наделяет «Ричеркар» таким количеством уровней значения, какого нет ни в каком другом Диалоге книги. Фридрих Великий заменен здесь Крабом, фортепиано \--- компьютерами и так далее. Читателя ожидает множество сюрпризов. В Диалоге снова затрагиваются проблемы разума, сознания, свободной воли, искусственного интеллекта, теста Тюринга и так далее. Он заканчивается косвенной ссылкой на начало книги, таким образом превращая её в гигантскую автороферентную Петлю, одновременно символизирующую музыку Баха, рисунки Эшера и Теорему Гёделя.

\end{document}
