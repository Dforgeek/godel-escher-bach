\documentclass[../main.tex]{subfiles}
\begin{document}

\ifSubfilesClassLoaded{% subfile
    \frontmatter
    \chapterstyle{FrontMatterChapterStyle}
}{% main file
    %
}

\chapter{Благодарность}

Эта книга зрела у меня в голове около двадцати лет \--- с тех пор, как в тринадцать лет я задумался над тем, как я думаю по-английски и по-французски. Даже раньше по некоторым признакам уже можно было понять, в какой области лежат мои основные интересы. Помню, что когда я был совсем ребенком, для меня не было ничего интереснее, чем идея трех 3: операция, проводимая над тройкой с помощью её самой! Я был убежден, что это тонкое наблюдение не могло прийти в голову никому другому; но однажды я все же осмелился спросить мать, что из этого получится, и она ответила: «Девять». Однако я не был уверен, что она поняла, что я имел в виду. Позже мой отец посвятил меня в тайны квадратных корней и мнимой единицы.

Я обязан моим родителям больше, чем любому другому. Они были для меня столпами, на которые я мог опереться в любое время. Они направляли, вдохновляли, поощряли и поддерживали меня. Более того, родители всегда в меня верили. Им посвящена эта книга.

Особая благодарность двум моим старым друзьям \--- Роберту Бёнингеру и Питеру Джонсу; они помогли сформировать мое мышление. Их влиянием и идеями проникнута вся книга.

Я многим обязан Чарльзу Бреннеру, научившему меня программированию, когда мы оба были молоды; благодарю его за постоянное подталкивание и стимулирование, которое на самом деле равнялось завуалированной похвале \--- а также за иногда случавшуюся критику.

Рад отдать должное Эрнесту Нагелю, моему многолетнему другу и учителю, оказавшему на меня огромное влияние. «Доказательство теоремы Гёделя» Нагеля и Ньюмана \--- одна из моих любимых книг, и я многое вынес из наших бесед много лет назад в Вермонте и не так давно в Нью-Йорке.

Ховард Делонг своей книгой пробудил давно дремавший во мне интерес к темам этой книги. Я поистине многим ему обязан.

Давид Джонатан Джастман научил меня, что значит быть Черепахой \--- изобретательным, настойчивым и ироничным существом, любительницей парадоксов и противоречий. Надеюсь, что он прочтет эту книгу, которой я ему во многом обязан, и что она его развлечет.

Скотт Ким оказал на меня огромное влияние. С тех пор как мы с ним встретились около двух с половиной лет тому назад, между нами всегда был невероятный резонанс. Его идеи о музыке и изобразительном искусстве, его юмор и аналогии, его добровольная бескорыстная помощь в критические минуты внесли значительный вклад в книгу; кроме того, Скотту я обязан новой перспективой, благодаря чему мой взгляд на стоявшую передо мной задачу менялся по мере того, как книга продвигалась вперед. Если кто-то и понимает эту книгу, то это Скотт.

За крупномасштабными и мелкомасштабными советами я неоднократно обращался к Дону Бирду, знающему эту книгу вдоль и поперек. Он безошибочно чувствует её структуру и цель и много раз подавал мне отличные идеи, которые я с удовольствием включал в книгу. Я сожалею только о том, что уже не смогу включить будущие идеи Дона, когда книга выйдет из печати. И не дайте мне забыть поблагодарить Дона за чудесную гибкость-в-негибкости его нотной программы СМУТ\@. Многих длинных дней и трудных ночей стоило ему уговорить СМУТ исполнить необходимые причудливые трюки. Некоторые из его результатов включены в иллюстрации книги; однако его влияние, к моему вящему удовольствию, распространено в ней повсюду.

Я не смог бы написать эту книгу без помощи технического оборудования Института математических исследований в общественных науках Стэнфордского университета. Пат Суппс, директор Института и мой давний друг, был очень великодушен, поселив меня в Вентура Холле, дав мне допуск к превосходной компьютерной программе и обеспечив мне великолепную рабочую обстановку в течение двух лет.

Это приводит меня к Пентти Канерва, автору программы-редактора, которой эта книга обязана своим существованием. Я многим говорил, что написание этой книги отняло бы у меня вдвое больше времени, если бы не «TV-Edit», удобная и настолько простая по духу программа, что только Пентти мог написать подобное. Благодаря ему я сумел сделать то, что удается мало кому из авторов, \--- сверстать свою собственную книгу. Пентти был главной двигающей силой исследований по компьютерной верстке в упомянутом выше Институте. Очень важным для меня было также редкое качество Пентти \--- его чувство стиля. Если эта книга выглядит хорошо, это во многом заслуга Пентти Канерва.

Эта книга родилась в типографии Стэнфордского университета. Мне хотелось бы высказать сердечную благодарность директору типографии, Беверли Хендрикс, и её сотрудникам за помощь в минуты особой нужды и за их ровное хорошее настроение несмотря на многие неудачи. Я хотел бы также поблагодарить Сесиль Тэйлор и Барбару Ладдага, проделавших большую часть печатания гранок.

За многие годы моя сестра Лаура Хофштадтер во многом способствовала формированию моих взглядов. Ее влияние присутствует как в форме, так и в содержании этой книги.

Я признателен моим новым и старым друзьям Мари Антони, Сидни Арковиц, Бенгту Олле Бенгтссону, Феликсу Блоху, Франсуа Вануччи, Терри Винограду, Бобу Вольфу, Эрику Гамбургу, Майклу Голдхаберу, Пранабу Гошу, Авриль Гринберг, Дэйву Дженнигсу, Перси Диаконису, Най-Хуа Дуан, Уилфреду Зигу, Дианне Канерва, Лори Канерва, Инге Карлингер, Джонатану и Эллен Кинг, Франциско Кларо, Гэйл Ландт, Биллу Льюису, Джиму Макдональду, Джону Маккарти, Джое Марлоу, Луису Менделовицу, Майку Мюллеру, Розмари Нельсон, Стиву Омохундро, Полю Оппенгеймеру, Питеру Е.~Парксу, Давиду Поликански, Питу Римбею, Кэти Россер, Гаю Стилу, Ларри Теслеру, Филу Уадлеру, Робину Фрееману, Дану Фридману, Роберту Херману, Рэю Химану и Джону Эллису за их «резонанс» со мной в критические минуты моей жизни; каждый по-своему, они помогли мне написать эту книгу.

Я написал эту книгу дважды. Закончив её в первый раз, я начал сначала и все переделал. Первый вариант был закончен, когда я был аспирантом-физиком в Орегонском университете; четверо из профессоров отнеслись чрезвычайно снисходительно к моим странностям: Майк Моравчик, Грегори Ванниер, Руди Хва и Пауль Чонка. Я ценю их понимающее отношение. К тому же, Пауль Чонка прочел всю первую версию и сделал множество ценных замечаний.

Спасибо Е.О.~Вильсону за прочтение и комментарии по поводу раннего варианта «Прелюдии» и «Муравьиной фуги».

Спасибо Марше Мередит за то, что она была мета-автором забавного коана.

Спасибо Марвину Мински за памятную беседу у него дома как-то мартовским днем; часть её читатель найдет в этой книге.

Спасибо Биллу Кауфману за советы по издательской части, а также Джереми Бернштейну и Алексу Джорджу за их поддержку в нужные минуты.

Горячая благодарность Мартину Кесслеру, Морин Бишоф, Винсенту Торре, Леону Дорину и другим работникам издательства «Бэйсик Букс» за то, что они взялись за эту издательскую задачу, во многом необычную.

Спасибо Фиби Хосс за отличное редактирование и Ларри Бриду за корректирование в последнюю минуту.

Спасибо многим соседям по «Imlac», которые столько раз за эти годы записывали для меня телефонные сообщения, а также работникам Пайн Холла, создавшим аппаратуру и программы, от которых зависело существование этой книги.

Спасибо Деннису Дэвису из Стэнфордского института телевизионных сетей за его помощь в установке «самопоглощающих экранов», которые я фотографировал в течение нескольких часов.

Спасибо Джерри Прайку, Бобу Парксу, Теду Брадшоу и Винни Авени из лаборатории физики высоких энергий в Стэнфорде за их помощь в изготовлении триплетов

Спасибо моим дяде и тете, Джимми и Бетти Гиван, за рождественский подарок; они не подозревали, какое удовольствие я от него получил!
Это был «черный ящик», единственная функция которого состояла в самовыключении.

Наконец, я хочу выразить особую благодарность моему профессору английской литературы, Бренту Гарольду, который открыл для меня дзен-буддизм, когда я был первокурсником; Кесу Гужелоту, давным-давно, в грустный ноябрьский день, подарившему мне пластинку с «Музыкальным приношением», а также Отто Фришу, в чьем кабинете я впервые познакомился с магией Эшера.

Особая благодарность автора издателю Михаилу Бахраху и специалисту по компьютерной верстке Павлу Иванникову за понимание и подлинный профессионализм в работе над русским изданием книги.

Я попытался вспомнить всех людей, имевших отношение к созданию этой книги, но список, несомненно, оказался неполон.

В каком-то смысле эта книга \--- символ моей веры. Я надеюсь, что мои читатели это поймут и что мой энтузиазм и поклонение перед определенными идеями проникнут в чье-нибудь сердце и разум. Это большее, на что я могу надеяться.

\fancybreak{$\ast$ $\ast$ $\ast$}

Переводчик выражает глубокую благодарность автору за его ценные советы: Ариадне Соловьевой за бескорыстное редактирование русского варианта книги; Дэвиду Риду за советы в области математической логики; Мику Армбрустеру за любезно предоставленный персональный компьютер и Наталье Эскиной за редактирование Диалогов и за прекрасный перевод Баховского стихотворения.

\end{document}
