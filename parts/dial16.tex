\documentclass[../main.tex]{subfiles}
\begin{document}

\section{Благочестивые размышления курильщика табака}

\centerblock{
    \emph{Ахилл пришел в гости к крабу.}
}

\begin{dialogue}

\speak{Ахилл} У вас, я гляжу, появилось несколько новых приобретений, м-р Краб. Особенно хороши вот эти картины.

\emph{Краб} : Благодарю вас. Мне нравятся многие художники, а больше всех --- Рене Магритт. Многие картины в моем доме написаны им.

\emph{Ахилл} : Должен признаться, что эти образы довольно загадочны. Эти картины Магритта напоминают мне о работах МОЕГО любимого художника, Эшера.

\emph{Краб} : Я вас понимаю. Оба они весьма реалистичны в исследовании парадоксальных и иллюзорных миров; оба тонко чувствуют, какие образы особенно стимулируют мысль. Кроме того, они оба --- мастера грациозных линий, что часто упускают из виду даже их горячие поклонники.

\emph{Ахилл} : Однако в чем-то они очень различны. Как вы думаете, как можно охарактеризовать эту разницу?

\emph{Краб} : Было бы замечательно интересно сравнить их в подробностях.

\emph{Ахилл} : Я нахожу, что Магритт --- удивительный мастер реализма. Например, мне очень нравится рисунок дерева с гигантской трубкой за ним.

\emph{Краб} : Вы хотите сказать, нормальной трубки с крохотным деревцем перед ней?

\emph{Ахилл} : Ах, так вот что это такое! Так или иначе, когда я ее увидел в первый раз, мне почудился запах табачного дыма. Представляете, как глупо я себя почувствовал?

\emph{Краб} : Я отлично вас понимаю. Эта картина уже обманула многих из моих гостей.

\emph{Рис. 77. Рене Магритт. «Тени» (1966)}

\emph{(С этими словами он дотягивается до картины на стене, вынимает трубку из-за дерева, выбивает ее об стол --- при этом комната наполняется запахом застарелого табака --- и начинает набивать ее снова.)}

Это превосходная старая трубка, Ахилл. Верите ли, внутри она обита медью, отчего становится только лучше с годами.

\emph{Ахилл} : Неужели медью? Невероятно!

\emph{Краб (доставая коробок спичек и зажигая трубку)} : Не желаете ли, Ахилл?

\emph{Ахилл} : Нет, спасибо. Я курю редко, и только сигары.

\emph{Краб} : Никаких проблем --- у меня и сигара найдется!

\emph{(Дотягивается до другого рисунка Магритта, на котором изображен велосипед, стоящий на дымящейся сигаре.)}

\emph{Рис. 78. Репе Магритт. «Грация» (1959)}

\emph{Ахилл} : Гм-м\ldots{} Нет, благодарю вас, сейчас мне что-то не хочется.

\emph{Краб} : Ну, как хотите. Сам-то я --- неисправимый курильщик. Кстати, это мне кое-что напоминает --- вы, конечно, знаете, что старик Бах был любителем выкурить трубочку?

\emph{Ахилл} : Что-то не припоминаю.

\emph{Краб} : Старик Бах любил заниматься стихоплетством, философствованием, курением трубки и сочинением музыки (не обязательно в этом порядке). Он соединил все эти пристрастия в одном забавном стихотворении, которое положил на музыку. Вы можете найти его в знаменитой записной книжке, которая хранилась у его жены, Анны Магдалены. Оно называется:

«Благочестивые размышления курильщика табака»\protect\hyperlink{c_2}{\textsuperscript{\uline{\{2\}}}}

Когда я трубку набиваю

Отборным, крепким табаком

И кольца дыма выпускаю,

И тает дым под потолком ---

Каким печальным поученьем

Приходит мне на ум сравненье:

Подобна дыму жизнь земная,

И трубка как судьба хрупка.

Ее как жизнь оберегаю,

Трясусь над ней --- дрожит рука.

Адам из глины сотворен ---

Вернется в прах и глину он.

Вот необкуренная трубка~---

Бледна как смерть, бела как мел.

И я улягусь бледным трупом,

В гробу как трубка буду бел.

Но почернеет скоро тело ---

От дыма трубка почернела.

Сияньем ярким ослепила ---

Но гаснет огонек во мгле.

Так угасают власть и сила.

Ищи-свищи огонь в золе.

Зола и пепел --- и тела,

И честь, и слава, и дела.

Горячий пепел обжигает,

Но раскаленный уголек

Через минуту остывает,

Смягчится боль, пройдет ожог.

Земная боль --- одна услада

В сравненьи с вечной мукой ада.

Кто с трубкой пишет и гуляет,

Тому дымок над головой,

Как добрый пастырь помогает

Пройти достойно путь земной.

Куда б ни вел тебя твой путь,

В дорогу трубку не забудь.

Очаровательная философия, не правда ли?

\emph{Ахилл} : Действительно. Старик Бах был, как видно, поэтом и любителем искусств.

\emph{Краб} : Вы просто читаете мои мысли! Знаете в свое время и я забавлялся стихоплетством. Но боюсь что результаты получались довольно бледными. У меня нет особого дара слова.

\emph{Ахилл} : Зато, м-р Краб, я уверен, что ваши стихи были не без чувства. Я был бы счастлив услышать одно из ваших творений.

\emph{Краб} : Польщен вашим вниманием. Хотите услышать мою песню в исполнении автора? Не помню где я ее написал, она называется «Рассказ ни к селу, ни к городу».

\emph{Ахилл} : Как поэтично!

\emph{(Краб достает пластинку из пакета и подходит к огромному сложному на вид аппарату. Он открывает крышку и вкладывает диск в зловещую механическую пасть. Внезапно по поверхности диска скользит яркий зеленоватый луч секунду спустя, диск исчезает во внутренностях этого фантастического сооружения, и в комнате раздаются звуки Крабьего голоса)}

Всех смешил наш поэт до упаду.

Смастерил не одну он шараду,

Но в последней строке,

В его каждом стишке

Не бывало совершенно никакого смысла.

\emph{Ахилл} : Прелестно! Только я не понимаю одного мне кажется, что в конце этой песни ---

\emph{Краб} : Нет совершенно никакого смысла?

\emph{Ахилл} : Нет. Я хочу сказать что там нет ни складу, ни ладу.

\emph{Краб} : Вы вероятно правы.

\emph{Ахилл} : В остальном это прекрасная песня, но я должен признаться, что я больше всего поражен этим чудовищно сложным сооружением. Это гигантский патефон да?

\emph{Краб} : О нет, это гораздо больше чем просто патефон. Это Черепахоуничтожающий Патефон!

\emph{Ахилл} : О Боже мой!

\emph{Краб} : Не волнуйтесь, он уничтожает не черепах, а всего лишь пластинки, сделанные г-жой Черепахой.

\emph{Ахилл} : Что ж, это уже не так страшно. Так это часть той странной музыкальной войны, которая с недавнего времени идет между вами и г-жой Черепахой?

\emph{Краб} : В некотором роде, позвольте мне объяснить поподробнее. Видите ли г-жа Черепаха так навострилась, что какой бы патефон я ни купил, она ухитрялась его разбить.

\emph{Ахилл} : Но я слышал, что в конце концов, у вас появился непобедимый патефон со встроенной телекамерой микрокомпьютером и так далее, который мог разобрать и снова собрать сам себя так, что он становился неуязвимым для Черепашьих атак.

\emph{Краб} : Увы мне, увы! Мой план провалился, поскольку Черепаха воспользовалась одной тонкостью, которую я проглядел ---~подсистема, осуществлявшая разборку и сборку, сама оставалась фиксированной в течение всего процесса. Ясно, что она не могла разбирать и собирать саму себя, так что она оставалась неизменной.

\emph{Ахилл} : Ну и что?

\emph{Краб} : О, это привело к самым печальным последствиям! На этот раз г-жа Черепаха направила свою атаку прямо против этой подсистемы.

\emph{Ахилл} : Как это?

\emph{Краб} : Она просто сделала запись, вызывающую роковые колебания в единственной детали, которая не менялась --- в сборочно-разборочной подсистеме!

\emph{Ахилл} : А, понимаю\ldots{} Хитро, ничего не скажешь! ---

\emph{Краб} : Так я и подумал. И ее стратегия сработала --- правда, не с первого раза. Я думал, что мне удалось ее перехитрить, когда мой патефон выдержал ее первый натиск. Я смеялся и торжествовал! Но когда она явилась в следующий раз, я увидел в ее глазах стальной блеск и понял, что на этот раз она подготовилась не на шутку. Я поставил ее новую пластинку на мой патефон. В тишине мы оба смотрели, как компьютер анализирует звуковые дорожки, затем снимает пластинку, разбирает патефон, снова собирает его, ставит пластинку обратно --- и затем медленно опускает иглу.

\emph{Ахилл} : Ах!

\emph{Краб} : Как только прозвучал первый звук, в комнате раздалось оглушительное БА-БАХ! Вся моя конструкция развалилась, причем хуже всего была повреждена сборочно-разборочная подсистема. В этот страшный момент, стоя над останками моего детища, я наконец осознал, что Черепаха ВСЕГДА сможет направить свою атаку против --- простите за выражение --- Ахиллесовой пяты системы.

\emph{Ахилл} : О, какой ужас! Я вас очень хорошо понимаю --- это бывает страшно неприятно.

\emph{Краб} : Да, некоторое время я был в отчаяньи и уже собирался махнуть на все клешней. Но, к счастью, это не конец истории. У нее есть продолжение\ldots{} Из всего этого я извлек ценный урок, которым хочу с вами поделиться. По рекомендации Черепахи, я ознакомился с интересной книгой, полной странных Диалогов на самые разные темы, включая молекулярную биологию, фуги, дзен-буддизм, и Бог знает, что еще.

\emph{Ахилл} : Небось, какой-нибудь псих насочинял. Как эта книга называется?

\emph{Краб} : Если меня не обманывает память, она называется «Медь, серебро, золото --- этот неразрушимый сплав».

\emph{Ахилл} : О, я о ней уже слышал от г-жи Черепахи. Книгу написал один ее знакомый, который, как кажется, без ума от металл-логики.

\emph{Краб} : Интересно, что за знакомый такой. Так или иначе, в одном из Диалогов я наткнулся на некие Благочестивые Размышления Вируса Табачной Мозаики, на рибосомы и другие странные штуки, о которых раньше никогда не слыхал.

\emph{Ахилл} : Что такое Вирус Табачной Мозаики? Кто такие рибосомы?

\emph{Краб} : Я точно не знаю --- во всем, что касается биологии, я полный профан. Все, что мне известно, я узнал из этого Диалога. Там говорится, что вирус табачной мозаики --- это крохотные сигаретообразные штучки, причина болезни табачного растения.

\emph{Ахилл} : Рак?

\emph{Краб} : Нет, не совсем ---

\emph{Ахилл} : Чего же еще? Табачное растение, которое курит сигареты и заболевает раком! Поделом ему!

\emph{Рис. 79. Вирус табачной мозаики. (Lehninger A. «Biochemistry»)}

\emph{Краб} : Мне кажется, вы делаете слишком поспешные выводы, Ахилл. Табачные растения НЕ КУРЯТ эти «сигареты». Эти противные маленькие «сигареты» --- незваные гости, атакующие растения.

\emph{Ахилл} : А-а, понятно. Ладно, теперь, когда я знаю, что такое вирус табачной мозаики, объясните мне, пожалуйста, что такое рибосомы.

\emph{Краб} : Рибосомы --- это, кажется, что-то вроде существ, меньших по размеру, чем клетка; они переводят информацию в другую форму.

\emph{Ахилл} : Вроде крохотного магнитофона или патефона?

\emph{Краб} : Метафорически, пожалуй. Я обратил внимание на строчку, в которой некий престранный персонаж упоминает о том, что рибосомы --- как и вирус табачной мозаики и некоторые другие удивительные биологические структуры --- обладают «поразительной способностью к спонтанной само-сборке». Именно так он и сказал.

\emph{Ахилл} : Я думаю, что это было самое странное из его высказываний.

\emph{Краб} : Так и подумал другой персонаж Диалога. Но мне кажется, что это --- превратная интерпретация. \emph{(Краб глубоко затягивается своей трубочкой и выпускает несколько клубов дыма.)}

\emph{Ахилл} : Что же такое эта «спонтанная само-сборка»?

\emph{Краб} : Идея состоит в том, что когда некие биологические системы внутри клетки распадаются, они могут спонтанно собраться снова --- и ими при этом не управляет никакая другая система. Они просто приближаются друг к другу и --- раз! --- снова «склеиваются» в одно целое.

\emph{Ахилл} : Волшебство да и только! Хорошо бы патефоны обладали подобной способностью\ldots{} Я хочу сказать, что если миниатюрный «патефон»-рибосома на такое способен, то почему бы не сделать то же самое и настоящему, большому патефону. Это позволило бы вам создать неразрушимый патефон, не так ли? Каждый раз, когда он разваливался на куски, они бы снова собирались вместе, и патефон через несколько минут был бы снова цел!

\emph{Краб} : Так я и подумал. Я тут же написал на фабрику патефонов, объясняя им понятие само-сборки, и спросил их, не могут ли они построить мне такой патефон, который бы сам разбирался и затем собирался в другой форме.

\emph{Ахилл} : Задачка не из легких!

\emph{Краб} : Верно; но через несколько месяцев они написали мне, что им удалось выполнить мою просьбу, --- и прислали мне такой счет, заплатить по которому оказалось, действительно, задачкой не из легких. Хо! И вот в один прекрасный денек пришла посылка с моим новым Самособирающимся Патефоном. На этот раз я был полностью уверен в победе. Я позвонил Черепахе и пригласил ее опробовать мою новую модель патефона.

\emph{Ахилл} : Значит, эта великолепная машина перед нами и есть тот самый патефон?

\emph{Краб} : Боюсь, что нет, Ахилл.

\emph{Ахилл} : Неужели это случилось снова?

\emph{Краб} : К несчастью, вы правы. Я уже не пытаюсь понять, почему это произошло. Мне слишком тяжело об этом вспоминать. Все эти пружины и провода, валяющиеся кучей на полу, и клубы дыма\ldots{} Ах\ldots{} Боже мой\ldots{}

\emph{Ахилл} : Полноте, успокойтесь, мистер Краб, не принимайте этого так близко к сердцу.

\emph{Краб} : Не волнуйтесь, со мной все в порядке --- просто у меня время от времени пошаливают нервы. Так вот, после того, как Черепаха вдоволь нарадовалась победе, она, наконец, заметила мое жалкое состояние и попыталась меня утешить, объясняя, что с этим ничего нельзя поделать. Оказалось, что мои патефоны имели отношение к чьей-то Теореме, в которой я не понял ни слова. Как бишь она называлась ---«Теорема Гоголя»?

\emph{Ахилл} : Может быть, «Теорема Гёделя»? Она мне как-то раз пыталась ее объяснить, и звучало это довольно мрачно.

\emph{Краб} : Возможно, я точно не помню.

\emph{Ахилл} : Уверяю вас, мистер Краб, что я выслушал этот рассказ, испытывая глубочайшее сочувствие к вашему положению. Оно, действительно, очень грустно. Но помните, вы сказали, что из всего этого вам удалось извлечь полезный урок? Не поделитесь ли вы со мной вашим опытом? Хотя и говорят, что молчание --- золото, а слово --- всего лишь серебро, но прошу вас, говорите --- я сгораю от любопытства!

\emph{Краб} : Да уж, серебро\ldots{} Лучше бы вы о нем не упоминали, после того, что мне пришлось заплатить за новый патефон! Так вот, в конце концов я отказался от поисков совершенного патефона и решил заняться усовершенствованием защитных механизмов от Черепашьих пластинок. Я заключил, что мне придется оставить мечту о совершенном патефоне, вместо этого мне нужен патефон, который сможет ВЫЖИТЬ и не разлетится на куски, даже если это и будет означать, что он сможет проигрывать только несколько записей.

\emph{Ахилл} : Так вы решили построить сложные анти-Черепашьи механизмы, пожертвовав для этого способностью патефона воспроизводить любые звуки?

\emph{Краб} : Я бы не сказал, что я это «решил» Я был к этому ПРИНУЖДЕН.

\emph{Ахилл} : Да, я понимаю.

\emph{Краб} : Моя новая идея заключалась в том, чтобы предотвратить любую «чужую» запись от проигрывания на моем патефоне. Я знаю, что мои собственные пластинки безопасны. Если бы я мог не позволить другим ставить ИХ пластинки на мой патефон, это бы его защитило, а мне позволило бы спокойно наслаждаться моими записями.

\emph{Ахилл} : Отличная стратегия. И что же, эта гигантская штуковина представляет собой последний плод ваших усилий?

\emph{Краб} : Точно. Г-жа Черепаха, разумеется, поняла, что ей также придется изменить стратегию. Теперь у нее новая цель --- она пытается сделать такую запись, которой удастся проскользнуть незамеченной мимо моих «цензоров».

\emph{Ахилл} : Но скажите как же вам удается не допускать к машине ее записи и все остальные «чужие» пластинки?

\emph{Краб} : Сначала поклянитесь, что не выдадите моего секрета Черепахе.

\emph{Ахилл} : Честное Черепашье!

\emph{Краб} : Что?!

\emph{Ахилл} : О, это просто присказка, которой я выучился у г-жи Черепахи Не волнуйтесь --- ваш секрет умрет со мною!

\emph{Краб} : Ну, хорошо. Мой план заключался в использовании ЯРЛЫКОВ. Я снабжу каждую свою запись секретным ярлыком. Патефон перед вами, как и его предшественники оборудован телекамерой для сканирования записей и компьютером для анализа полученных данных и контроля дальнейших операций. Моя идея --- разбивать любую пластинку, у которой не будет надлежащего ярлыка.

\emph{Ахилл} : О, сладкая месть! Но мне кажется, что ваш план довольно легко расстроить. Г-же Черепахе надо только заполучить одну из ваших пластинок и скопировать ее ярлык!

\emph{Краб} : Это не так-то просто, Ахилл. Почему вы думаете, что она сумеет его найти? Ярлычок будет надежно запрятан!

\emph{Ахилл} : Вы хотите сказать, что он будет каким-то образом смешан с музыкой на пластинке?

\emph{Краб} : Именно! Но существует способ их различить. Для этого надо считать данные с пластинки и затем ---

\emph{Ахилл} : А, теперь я понимаю, что это был за зеленый свет!

\emph{Краб} : Верно, это была телекамера, сканирующая звуковые дорожки. Схема звуковых дорожек потом была направлена в микрокомпьютер, тот стал анализировать музыкальный стиль записи и все в полной тишине! Еще ничего пока не проигрывалось.

\emph{Ахилл} : Значит сканнер забраковывает записи, не обладающие определенным стилем?

\emph{Краб} : Вы ухватили самую суть моей идеи. Единственные записи, способные пройти эту проверку --- записи моего стиля. Боюсь, что имитация этого стиля окажется для Черепахи орешком не по зубам. Так что на этот раз я уверен в победе. Однако, справедливости ради, должен сказать, что Черепаха со своей стороны уверена, что ей удастся так или иначе одурачить мои цензорные устройства и ввести в патефон одну из ее записей.

\emph{Ахилл} : И снова превратить вашу прекрасную машину в груду обломков?

\emph{Краб} : О, нет --- она уже удовлетворила свои кровожадные инстинкты против моих патефонов. Теперь она всего-навсего хочет доказать, что ее пластинка сможет проникнуть в мой патефон, как бы я не старался это предотвратить. Она в последнее время часто бормочет себе под нос что-то о песнях с весьма странными названиями, вроде «Меня можно проиграть на патефоне X». Но МЕНЯ ей не запугать! Правда, меня немного тревожит то, что она, как прежде, приводит какие-то непонятные доводы, которые\ldots{} которые\ldots{} (Краб умолкает и с задумчивым видом снова затягивается своей трубкой.)

\emph{Ахилл} : Гм-м-м\ldots{} Полагаю, что теперь Черепаха столкнулась с неразрешимой задачей. На это раз нашла-таки коса на камень!

\emph{Краб} : Интересно, что вам так кажется\ldots{} Я не думаю, что вы хорошо знакомы с Теоремой Хенкина --- или я ошибаюсь?

\emph{Ахилл} : Хорошо знаком с ЧЬЕЙ Теоремой? Никогда не слыхал ничего похожего. Я уверен, что это должно быть захватывающе интересно, но сейчас я предпочитаю послушать о «проникальной» музыке. Это забавная история\ldots{} Мне кажется, что я мог бы угадать ее конец. Ясно, что г-жа Черепаха поймет, что ее дело безнадежно, и ей придется трусливо капитулировать --- этим все и кончится! Не так ли?

\emph{Краб} : Вашими бы устами да мед пить. Не желаете ли теперь ознакомиться с внутренностями моего защитного патефона?

\emph{Ахилл} : С радостью. Мне всегда хотелось поглядеть на работающую телекамеру.

\emph{Краб} : Сказано --- сделано, друг мой. (Засовывает клешню в разинутую «пасть» огромного патефона, отвинчивает там что-то и вытаскивает аккуратно упакованный механизм.) Видите ли, весь патефон собран из автономных частей --- они могут использоваться отдельно. Эта телекамера, например, прекрасно работает сама по себе! Смотрите на тот экран, около картины с пылающей тубой. (Он направляет камеру на Ахилла, чье лицо тут же появляется на большом экране.)

\emph{Рис. 80. Рене Магритт. «Прекрасный пленник»~(1947)}

\emph{Ахилл} : Здорово! Можно попробовать?

\emph{Краб} : Разумеется.

\emph{Ахилл (направляя камеру на Краба)} : Теперь на экране вы, м-р Краб.

\emph{Краб} : Да, вижу.

\emph{Ахилл} : А что, если направить камеру на картину с горящей тубой? О, теперь и она на экране!

\emph{Краб} : Попробуйте использовать объектив --- вы можете получить крупный план.

\emph{Ахилл} : Потрясающе! Сейчас я сфокусирую камеру на самой верхушке пламени, там, где оно почти касается рамы картины\ldots{} Какое странное чувство --- я могу мгновенно «скопировать» что угодно в комнате --- все, что пожелаю --- на этом экране. Я только должен направить на мой объект камеру, и, словно по волшебству, он появится на экране!

\emph{Краб} : ВСЕ ЧТО УГОДНО? Вы уверены?

\emph{Ахилл} : Разумеется --- все, что я здесь вижу.

\emph{Краб} : Но что получится, если вы направите камеру на пламя на экране?

\emph{(Ахилл двигает камеру --- теперь она направлена точно на то место на экране, где только что было изображение пламени.)}

\emph{Ахилл} : О, как интересно! Само это действие заставляет пламя ИСЧЕЗНУТЬ! Куда оно делось?

\emph{Краб} : Вы не можете одновременно сохранять образ на экране и передвигать камеру.

\emph{Ахилл} : А, ясно. Но я совсем не понимаю, что теперь на экране. Там какой-то странный коридор. Но я же направляю камеру вовсе не на коридор, а на экран телевизора!

\emph{Краб} : Посмотрите повнимательней, Ахилл. Вы уверены, что это, действительно, коридор?

\emph{Ахилл} : О, теперь я вижу. Это изображения самого экрана, вложенные одно в другое, и они становятся все меньше, и меньше, и меньше\ldots{} Ну, разумеется! Изображение пламени ДОЛЖНО БЫЛО исчезнуть, потому что оно появлялось тогда, когда я направлял камеру на картину. Когда камера направлена на экран, то там появляется изображение самого экрана с тем, что там в данный момент изображено, --- то есть сам экран с тем, что там в данный момент изображено, --- то есть сам экран с тем~---

\emph{Краб} : Благодарю вас, Ахилл, дальше я могу и сам догадаться. Попробуйте теперь вращать камеру.

\emph{Ахилл} : О! Теперь у меня получился прелестный «спиральный коридор»! Каждый экран поворачивается внутри экрана-рамки, так что чем меньше они становятся, тем больше они повернуты по отношению к самому внешнему экрану. Эта идея телевизионного экрана, «поглощающего самого себя», кажется мне очень странной.

\emph{Краб} : Что значит «поглощающий самого себя», Ахилл?

\emph{Ахилл} : Это то, что получается, когда я направляю камеру на экран.

\emph{Рис. 81. Двенадцать «самопоглощающих» телевизионных экранов. Я включил бы сюда еще один, если бы тринадцать не было простым числом.}

\emph{Краб} : Вы не возражаете, если мы исследуем это явление еще немного? Ваша идея меня заинтриговала.

\emph{Ахилл} : И меня тоже.

\emph{Краб} : Хорошо; тогда скажите мне: если вы направите камеру НА УГОЛ экрана, это все равно будет «самопоглощением»?

\emph{Ахилл} : Сейчас попробую. Гм-м\ldots{} Теперь коридор экранов смещается, и бесконечного самовложения больше не выходит. Красиво, но мне кажется, это уже нельзя назвать самопоглощением --- разве что неудавшимся.

\emph{Краб} : Если вы направите камеру обратно в центр экрана, то, может быть, сумеете поправить дело.

\emph{Ахилл (осторожно поворачивая камеру)} : Да! Коридор опять удлиняется\ldots{} Вот! Все в порядке! Теперь коридор уходит так далеко, куда достигает взгляд. Он стал бесконечным, когда камера опять оказалась направленной на ВЕСЬ экран. Гм-м\_. Это мне что-то напоминает. Несколько дней назад г-жа Черепаха сказала мне, что автореференция случается только тогда, когда высказывание относится к себе самому ЦЕЛИКОМ\ldots{}

\emph{Краб} : Что, простите?

\emph{Ахилл} : Да так, я говорил сам с собой.

\emph{(Ахилл забавляется с линзами и кнопками камеры и на экране возникают различные самопоглощающие изображения: крутящиеся спирали, похожие на галактики, калейдоскопические цветы и другие интересные узоры\ldots)}

\emph{Краб} : Вы, как я погляжу, нашли себе игрушку по душе.

\emph{Ахилл (отрываясь от камеры)} : Точно! Что за богатство образов может породить эта простая идея! (Он снова смотрит на экран и в удивлении восклицает): Вот это да, м-р Краб! Смотрите, на экране получился узор, похожий на пульсирующие лепестки! Откуда взялась эта пульсация? И телевизор, и камера неподвижны.

\emph{Краб} : Иногда можно получить изображение, которое постепенно меняется. Это объясняется тем, что между моментом, когда камера что-то «видит», и моментом, когда изображение появляется на экране, всегда проходит какое-то время --- около сотой доли секунды. Так что если у вас на экране пятьдесят вложенных друг в друга картинок, то опоздание будет около полминуты. Если на экране появляется движущийся предмет --- например, ваш палец, который вы держите перед камерой, --- то на расположенных в глубине экранах он появится не сразу. Это опоздание --- нечто вроде зрительного эха, которое потом распространяется на всю систему. И если вам удастся добиться того, чтобы эхо не умирало сразу, то у вас получится пульсирующее изображение.

\emph{Ахилл} : Удивительно! А что, если попытаться получить ПОЛНОЕ «самопоглощение»?

\emph{Краб} : Что вы хотите этим сказать?

\emph{Ахилл} : Видите ли, все эти экраны внутри экранов, конечно, интересны, но мне бы хотелось получить на экране ОДНОВРЕМЕННО изображение телекамеры и самого экрана. Только тогда можно считать, что система полностью самопоглощается. Ведь экран --- это только часть системы.

\emph{Краб} : А, понятно. Может быть, при помощи этого зеркала вам удастся получить нужный эффект.

\emph{(Краб протягивает Ахиллу зеркало, и тот манипулирует зеркалом и камерой так, что на экране появляется изображение камеры и самого экрана)}

\emph{Ахилл} : Готово! Я получил ПОЛНОЕ самопоглощение!

\emph{Краб} : Но у вас видна только одна сторона зеркала --- а как же насчет обратной стороны? Ведь именно благодаря ей на экране видна камера.

\emph{Ахилл} : Вы правы. Но, чтобы показать на экране обратную сторону зеркала, мне понадобилось бы еще одно зеркало.

\emph{Краб} : Но тогда вам пришлось бы показать и обратную сторону второго зеркала. А как же насчет обратной стороны телевизора и его внутренностей и проводов, и~---

\emph{Ахилл} : Ой! Я вижу, что мой «проект полного самопоглощения» гораздо труднее, чем я думал. У меня даже голова закружилась!

\emph{Краб} : Я вас отлично понимаю. Знаете что, присядьте-ка лучше вот сюда и перестаньте думать о самопоглощении. Не нервничайте! Посмотрите на мои картины и успокойтесь.

\emph{(Ахилл ложится на диван и начинает размеренно дышать)}

Может быть вас раздражает дым от моей трубки? Я ее сейчас потушу. \emph{(Вынимает трубку изо рта и аккуратно кладет ее под надписью внутри очередной картины Магритта)} Ну вот. Как, получше стало?

\emph{Ахилл} : Все еще малость подташнивает. \emph{(Показывает на картину)} Интересная работа. Особенно мне нравится этот блестящий ободок внутри деревянной рамы.

\emph{Краб} : Благодарю вас. Ободок был сделан по спецзаказу --- это золото.

\emph{Ахилл} : Золотой ободок? Вот это да! А что означают эти слова под трубкой? Что это за язык, английский?

\emph{Краб} : Нет, это французский. «Ceci n'est pas une pipe» означает «это не трубка» --- что совершенно верно.

\emph{Ахилл} : Но ведь это же ТРУБКА! Вы ее только что курили!

\emph{Краб} : Вы, кажется, не поняли. Слово «ceci» относится к рисунку, а не к самой трубке. Разумеется, трубка --- это трубка, но рисунок --- это не трубка.

\emph{Ахилл} : Интересно, относится ли «ceci» ко ВСЕЙ картине или же только к трубке внутри картины? Ах, Боже мой! Ведь это было бы еще одним самопоглощением! Мистер Краб, мне стало совсем нехорошо. По-моему, я заболеваю\ldots{}

\emph{Рис. 82. Рене Магритт «Воздух и песня» (1964)}

\end{dialogue}

\end{document}
