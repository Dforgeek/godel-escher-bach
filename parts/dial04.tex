\documentclass[../main.tex]{subfiles}
\begin{document}

\DialogueChapter{Акростиконтрапунктус}

\begin{dialogue}

\speak{Ахилл} {\Large Х}орошая у вас коллекция бумерангов, я такой нигде не видал!

\speak{Черепаха} Обыкновенная, не преувеличивайте, пожалуйста. У любой Черепахи можно увидеть коллекцию ничуть не хуже.

\speak{Ахилл} Феноменально! Вы, Черепахи, никогда не перестанете удивлять меня своей любовью к собиранию бумерангов.

\speak{Черепаха} Шутить изволите? Да страсть к коллекционированию этого оружия у нас в крови. А сейчас, не угодно ли пройти в гостиную?

\speak{Ахилл} Только после Вас, как обычно, госпожа Черепаха. \emph{(Следуя за Черепахой, Ахилл входит в гостиную и направляется в угол комнаты.)} Я вижу, что у вас также неплохое собрание пластинок. Какую музыку вы предпочитаете?

\speak{Черепаха} Актуальный вопрос. Видите ли, хотя я всегда была и остаюсь поклонницей Баха, должна признаться, что сейчас я увлекаюсь довольно необычной музыкой.

\speak{Ахилл} Да? Что же это за музыка?

\speak{Черепаха} Такая, о которой вы, скорее всего, никогда не слыхали. Я называю её «разбивальная музыка».

\speak{Ахилл} Едва ли не самая поразительная вещь, которую я слыхал от вас за последнее время. Что значит это необычное название?

\speak{Черепаха} Рада удовлетворить ваше любопытство. Эта музыка \--- для разбивания патефонов.

\speak{Ахилл} О ужас!

\speak{Черепаха} Вы полагаете?

\speak{Ахилл} С ума сойти! Воображаю, как вы, пританцовывая с кувалдой в руке, сокрушаете один патефон за другим под звуки «Битвы при Виттории» Бетховена.

\speak{Черепаха} Какое у вас образное мышление! Должна вас разочаровать, эта музыка не совсем то, что вы предполагаете. Однако её истинная природа тоже любопытна. Могу дать вам кое-какие разъяснения\ldots{}

\speak{Ахилл} Интересно\ldots{} Я весь внимание!

\speak{Черепаха} Йоркширский мой приятель, старый Краб (вы с ним, часом, не знакомы?) пришел ко мне однажды с визитом\ldots{}

\speak{Ахилл} {\Large А}рхибольшая умница, этот Краб. Я много о нем наслышан, но сам с ним никогда не встречался. Уверен, что знакомство со стариком принесло бы мне немалое удовольствие.

\speak{Черепаха} Конечно, он личность незаурядная. Надо бы мне устроить вашу встречу; может быть, мы все как-нибудь увидимся в парке на прогулке. Думаю, что вы понравитесь друг другу!

\speak{Ахилл} Расчудесная идея! Буду ждать этого с нетерпением\ldots{} Однако мы отклонились от темы вы, кажется, хотели объяснить мне, что такое разбивальная музыка?

\speak{Черепаха} Ох, да, чуть не забыла. Так вот, пришел, значит, Краб ко мне в гости. Вы, наверное, слыхали, что у него всегда была страсть ко всяческим машинкам и приспособлениям; в то время он прямо-таки сходил с ума по патефонам. Он тогда только что приобрел свой первый патефон и, будучи наивным и доверчивым покупателем, поверил во всю ту белиберду, что нам обычно говорят усердные клерки в надежде сбыть свой товар. На этот раз клерк объявил, что понравившийся Крабу патефон может верно воспроизвести любой звук. Короче говоря, Краб уверился в том, что он купил Идеальный Патефон.

\speak{Ахилл} Само собой разумеется, вы с этим не согласились.

\speak{Черепаха} Точно, но он заупрямился и твердил, что его проигрыватель может воспроизвести какие угодно мелодии. Спорить не было толку, и каждый остался при своем мнении. Вскоре, однако, я опять пришла к Крабу в гости, на этот раз не с пустыми руками: я принесла с собой запись песни моего собственного сочинения. Песня называется «Меня нельзя воспроизвести на Патефоне №1».

\speak{Ахилл} Идея неординарная, ничего не скажешь! Это вы ему в подарок принесли?

\speak{Черепаха} Конечно. Я предложила ему прослушать мое сочинение на его новом патефоне, и он с радостью согласился. Он поставил пластинку и включил патефон; но после первых же тактов бедный аппарат завибрировал, затрясся и вдруг \--- БА-БАХ! \--- разбился на мельчайшие кусочки, разлетевшиеся по всей комнате. Натурально, пластинка тоже разбилась вдребезги\ldots{}

\speak{Ахилл} О, Боже!\ldots{} Какой удар для бедняги. Что-то было не в порядке с патефоном?

\speak{Черепаха} Ничего. Абсолютно ничего. Просто он не мог воспроизвести мелодию моей песни \--- эти звуки вызвали в нем такую сильную вибрацию, что он разбился.

\speak{Ахилл} Так значит, это всё же был не Идеальный Патефон. А ведь клерк ему такого наговорил\ldots{}

\speak{Черепаха} Разве вы, Ахилл, верите всему тому, что говорят продавцы? Неужели вы так же наивны, как старый Краб?

\speak{Ахилл} Абсолютно нет! Краб гораздо наивнее. Я-то знаю, что все торговцы \--- известные пройдохи и надувалы. Поверьте, я не вчера родился!

\speak{Черепаха} Представьте себе тогда, что тот клерк мог несколько преувеличить выдающиеся качества нового приобретения Краба. Скорее всего, его патефон вовсе не идеальный, а значит, не может воспроизвести любые звуки.

\speak{Ахилл} Увы, кажется, так оно и есть\ldots{} Но как вы объясняете тот удивительный факт, что именно ваша запись оказалась той самой «невоспроизводимой» мелодией?

\speak{Черепаха} Ничего удивительного; я сделала это специально. Перед тем, как снова отправиться к Крабу, я пошла в магазин, продавший ему патефон и спросила, где эта модель была сделана. Узнав адрес, я послала на фабрику запрос и со следующей почтой получила полное описание патефона Краба. Я работала, не покладая лап, проанализировала всю конструкцию, и мне удалось найти именно ту мелодию, которая, если её сыграть вблизи от этого патефона, разобьет его вдребезги!

\speak{Ахилл} Какое коварство! Зачем вы мне всё это выложили\ldots{} Значит, вы сами записали эту музыку, да ещё и принесли эту подлую штуку ему в подарок!

\speak{Черепаха} Точно, вы угадали, мой проницательный друг! Однако это ещё не конец. Краб не поверил, что его патефон оказался не Идеальным\ldots{}

\speak{Ахилл} Упрямец!

\speak{Черепаха} Совершенно верно! Он отправился в магазин, где приобрел себе ещё один патефон, значительно дороже. На этот раз клерк пообещал вернуть ему деньги в удвоенном размере, если тот найдет хотя бы один звук, который новый патефон не сможет воспроизвести.

\speak{Ахилл} {\Large Б}лестящая идея! Выходит, Краб ничем не рисковал\ldots{}

\speak{Черепаха} Ловкий трюк, это верно. Так вот, Краб тут же похвастался мне своим приобретением; старик был вне себя от радости, и я пообещала ему придти в гости и посмотреть его очередное любимое детище.

\speak{Ахилл} Естественно, перед тем как выполнить обещание, вы снова написали на фабрику и с учетом конструкции нового патефона Краба скомпоновали ещё одну вредительскую мелодию, на этот раз под названием «Меня нельзя воспроизвести на патефоне №2»?

\speak{Черепаха} Совершенно верно! Вижу, что вы вполне прониклись моей идеей\ldots{}

\speak{Ахилл} Так что же случилось на этот раз?

\speak{Черепаха} Я поставила мою запись и, как вы сами можете догадаться, история повторилась и патефон, и пластинка разлетелись вдребезги.

\speak{Ахилл} Щелчок по Крабьему самолюбию изрядный! Тут уж ему, конечно, пришлось признать, что Идеальных Патефонов в природе не существует?

\speak{Черепаха} Если бы. На самом деле он решил, что следующий патефон наверняка окажется «выигрышным билетом», а поскольку у него теперь была куча денег, он.

\speak{Ахилл (перебивает)} Еще раз пошел в магазин\ldots{} Постойте-ка: ведь он бы мог вас запросто перехитрить, купив посредственный патефон, не воспроизводящий с достаточной точностью никакую, в том числе и разбивальную, музыку. Тогда вам пришлось бы спасовать\ldots{}

\speak{Черепаха} Соблазнительная мысль. Однако она противоречит первоначальной идее иметь патефон, на котором можно воспроизвести даже его собственную разбивальную мелодию (что, естественно, невозможно).

\speak{Ахилл} Конечно Теперь я понимаю, в чем здесь загвоздка. Любой достаточно качественный патефон (назовем его X), который сможет воспроизвести разбивальную музыку, от нее же и погибнет! Значит, патефон X не совершенный. Избежать подобной участи может только какой-нибудь плохонький патефон, который, однако, уже по определению не будет Идеальным! Любой патефон будет непременно «увечен» в том или ином смысле, а значит, все они дефектны!

\speak{Черепаха} Разумеется, они не идеальны, но почему вы называете их «дефектными»? Никакой патефон не способен сделать всё то, чего бы нам от него хотелось. Уж если говорить о дефектах, то изъян не в самих патефонах, а в наших представлениях о том, на что они способны. Краб, к примеру, был полон самых фантастических надежд

\speak{Ахилл} Искать Идеальный патефон \--- неблагодарное занятие. Купит ли Краб высококачественный или посредственный аппарат, он всё равно проигрывает. Бедняга, мне его искренно жаль?\ldots{}

\speak{Черепаха} В таком духе наш «поединок» с Крабом продолжался ещё несколько раундов, пока Краб не раскусил принципа моих композиций. Тогда старик попытался меня перехитрить. Он послал фабрикантам описание патефона своего изобретения, который они и изготовили по его чертежам. Краб назвал свое детище «Патефон Омега» \--- этот аппарат был намного сложнее чем все предыдущие.

\speak{Ахилл} А, понимаю: у него вообще не было движущихся частей\ldots{} Может быть, он был сделан из ваты? Или\ldots{}

\speak{Черепаха} Если вы будете пытаться угадать, то мы просидим здесь до завтра. Позвольте вам помочь: «Омега» имела встроенную телекамеру, сканирующую любую пластинку, перед тем как поставить её на патефон. Эта камера была подключена к компьютеру, который, в свою очередь, устанавливал по форме дорожек, что за музыка записана на данной пластинке.

\speak{Ахилл} Тривиальной эту конструкцию не назовешь, но пока мне всё понятно. Однако как же Омега использовала полученную информацию?

\speak{Черепаха} {\Large И}нтереснейшим образом: компьютер при помощи сложных вычислений устанавливал, какой эффект данная мелодия произведет на патефон. Если музыка оказывалась «опасной», Омега делала что-то поистине удивительное: она меняла структуру частей патефона, перестраиваясь на ходу! Только сделавшись неуязвимой для данной разбивальной мелодии, Омега включала свой патефон и проигрывала пластинку.

\speak{Ахилл} Могу себе представить, как вы разочаровались: ведь это означало, что вашим проделкам пришел конец!

\speak{Черепаха} Я удивлена, Ахилл, что вы так считаете. Видимо, вы не слишком хорошо знакомы с теоремой Гёделя о неполноте.

\speak{Ахилл} \ldots{} гммм\ldots{} Чьей теоремой?

\speak{Черепаха} {\Large И}мя её создателя \--- Гёдель. Суть теоремы заключается в том, что\ldots{}

\speak{Ахилл (перебивает)} Гёдель? Не слыхал\ldots{} Послушайте, я уверен, что всё это захватывающе интересно, но я, право, предпочел бы услыхать продолжение истории о разбивальной музыке. Мне думается, что я могу сам угадать её конец\ldots{}

\speak{Черепаха} Рада вашей проницательности. Вы, вероятно, думаете, что Краб победил?

\speak{Ахилл} А как же! Признайтесь, что вам пришлось трусливо капитулировать. Не так ли?

\speak{Черепаха} Ей-Богу, Ахилл, ну и засиделись мы с вами! Уж полночь близится\ldots{} Я с удовольствием пообщалась бы с вами ещё, но у меня уже глаза слипаются.

\speak{Ахилл} То-то я чувствую, что и меня в сон клонит\ldots{} Пойду я, пожалуй. \emph{(Направляется к двери, но внезапно поворачивает обратно.)} Однако какой я забывчивый! Принес вам маленький презент и чуть не унес его обратно домой. \emph{(Протягивает Черепахе небольшой аккуратный сверток.)}

\speak{Черепаха} {\Large С}тоило ли беспокоиться\ldots{} Благодарю! \emph{(Нетерпеливо распаковывает пакет.)}

\speak{Ахилл} {\Large Б}езделушка, право слово\ldots{}

\speak{Черепаха} Ах\ldots{} \emph{(Срывает последнюю обертку и на свет появляется изящный стеклянный бокал.)} Какая прелесть! Как вы узнали, что я прямо-таки с ума схожу по стеклянным бокалам?

\speak{Ахилл} Разве? Не имел ни малейшего понятия, но я рад, что вам понравилось.

\speak{Черепаха} Обворожительно! Послушайте, если вы умеете хранить секреты, я вам кое-что расскажу. Я пытаюсь найти Идеальный Бокал, так сказать, Генерал-бокал, Гроссмейстер-бокал, чья форма не имела бы ни малейшего изъяна. Представляете, если бы ваш подарок, назовем его Бокал Г, оказался бы искомым сокровищем! Сделайте милость, поделитесь: где вы отыскали это чудо?

\speak{Ахилл} Частная коллекция, друг мой, частная коллекция \--- а больше того, не обессудьте, я вам открыть не могу: секрет! Могу, ежели желаете, сообщить, кому принадлежал сей бокальчик.

\speak{Черепаха} Не томите душу, говорите!

\speak{Ахилл} Имейте терпенье, друг мой. Слыхали ли вы когда-нибудь о знаменитом коллекционере бокалов по имени И.\,С.~Бах?

\speak{Черепаха} Мало кто не слышал хотя бы однажды этого блестящего имени; но позвольте, я впервые слышу, что И.\,С.~Бах занимался коллекционированием!

\speak{Ахилл} {\Large А}ртистичные натуры часто бывают весьма разносторонни. Конечно, Бах был в первую очередь известен не как коллекционер, однако это занятие было его излюбленным хобби, хотя почти ни одна душа об этом не знает. Этот бокальчик \--- его последнее приобретение.

\speak{Черепаха} Клянусь небом, это удивительно! Последнее приобретение? Если это так, то ему цены нет! Но почему вы так уверены, что бокал Г действительно принадлежал Баху?

\speak{Ахилл} Рассмотрите-ка его на свет: видите, внутри выгравирована надпись \mbox{B-A-C-H}?

\speak{Черепаха} О, вижу, вижу. Убедительно, ничего не скажешь. Поразительная вещь\ldots{} (аккуратно ставит Бокал Г на полку). Кстати, знаете ли вы, что каждая буква в имени BACH \--- это также название музыкальной ноты?

\speak{Ахилл} Странно, как же это возможно? Я знаю, что во многих языках ноты обозначаются буквами, но там используются буквы только от А до G.

\speak{Черепаха} Точно, в большинстве стран так оно и есть. Однако на родине Баха, в Германии, система немного другая. Например, нота~«си» будет по-немецки~«H», а «си бемоль» \--- «B». Так, си-бемоль минорная месса Баха по-немецки называется «H-moll Mess». Понимаете?

\speak{Ахилл} Изрядная путаница\ldots{} Подождите-ка. Нота~«си» по-немецки «H», а «си бемоль» \--- «В»\ldots{} Значит, само имя Баха \--- мелодия?

\speak{Черепаха} Хотя это и странно, но так оно и есть! На самом деле, Бах незаметно включил эту мелодию в одну из сложнейших композиций «Искусства фуги», финальный «Контрапункт». Это была последняя фуга, написанная Бахом.

\speak{Ахилл} О, какое совпадение! Последний бокал, последняя фуга\ldots{} Продолжайте, друг мой, прошу вас\ldots{}

\speak{Черепаха} Милейший Ахилл, наберитесь терпения, берите пример с нас, Черепах\ldots{} На чем, бишь, я остановилась? Когда я слушала «Контрапункт» впервые, я понятия не имела, какой будет финал. Внезапно, без малейшего предупреждения, музыка оборвалась. Затем \--- мертвая тишина\ldots{} Я тут же поняла, что как раз в тот момент композитор умер. Этот миг, неописуемо печальный, так на меня подействовал, что я почувствовала себя совершенно разбитой. Так или иначе, \mbox{B-A-C-H} \--- последняя тема этой фуги и она спрятана внутри произведения. Бах никому не сказал об этом, но, зная эту мелодию, её можно найти без труда. Ах, Ахилл, сколько существует ловких способов спрятать тайные послания в музыке\ldots{}

\speak{Ахилл} {\Large Х}итроумные уловки для этого есть и в поэзии. Поэты часто прибегали к похожим трюкам; теперь, к сожалению, это вышло из моды. Скажем, Льюис Кэрролл частенько прятал слова и имена в первых буквах строк своих стихов. Поэма, скрывающая таким образом какое-нибудь послание, называется «акростих».

\speak{Черепаха} Осведомлены ли вы, Ахилл, о том, что Бах тоже иногда писал акростихи?

\speak{Ахилл} Фантастическая разносторонность!

\speak{Черепаха} Широкие интересы у него были, ничего не скажешь! Но странного тут ничего нет: ведь контрапункт и акростих, с их скрытым смыслом, имеют очень много общего. Большинство акростихов прячут только одно послание, однако может существовать и акростих, так сказать, «с двойным дном», где первое послание, в свою очередь, является акростихом для второго. Можно представить себе и «контракростих», где секретное послание надо читать справа налево. Бог мой, да эта форма представляет почти неограниченные возможности! Более того, кто сказал, что акростихи \--- область исключительно поэтов? Их может сочинять кто угодно, даже диалогики.

\speak{Ахилл} Так, так\ldots{} Дело логики? Значит, мне это будет трудновато. Я с Госпожой Логикой не в ладах.

\speak{Черепаха} Ахилл, вы опять всё перепутали. Я сказала не «дело логики», а «диалогики», то есть сочинители диалогов. Гммм\ldots{} \emph{(Чешет лапой за ухом с задумчивым видом.)}

\speak{Ахилл} Друг мой, я по глазам вижу, что вы ещё что-то замышляете\ldots{}

\speak{Черепаха} Так, пустяки\ldots{} Я подумала: а что если какой-нибудь диалогик задумает написать один из своих диалогов в форме акростического контрапункта, в честь И.\,С.~Баха? Маловероятно, конечно, чтобы такая странная идея пришла кому-нибудь в голову\ldots{} Всё же, в таком случае, какое имя будет правильнее зашифровать: его собственное или Баховское? Впрочем, зачем нам волноваться о таких пустячных материях, пусть этот вопрос решает тот, кто задумает написать подобный диалог!\ldots{} Вернемся лучше к нашему «музыкальному» имени: знаете ли вы, что мелодия \mbox{B-A-C-H}, если её сыграть снизу вверх и задом наперед, звучит точно также, как оригинал?

\speak{Ахилл} Если её сыграть снизу вверх? Не понимаю. Задом наперед, это ясно: \mbox{H-C-A-B} \--- но снизу вверх? Вы, вероятно, меня разыгрываете?

\speak{Черепаха} Разрешите вам продемонстрировать; сейчас, только принесу скрипку\ldots{} \emph{(Идет в соседнюю комнату и возвращается со старинным инструментом.)} Сейчас я вам, скептику, сыграю эту мелодию задом наперед, вверх тормашками, шиворот навыворот и в любом виде, в каком вашей душеньке будет угодно\ldots{} Ну что ж, начнем\ldots{} \emph{(Кладет на пюпитр ноты «Искусства фуги» и открывает их на последней странице.)} Вот он, последний «Контрапунктус» и вот она, последняя тема.

\stage{\emph{(Черепаха начинает играть: \mbox{B-A-C-H\ldots{}} но когда она пытается взять финальное «H», внезапно, без малейшего предупреждения, резкий звук бьющегося стекла грубо прерывает её игру.Черепаха и Ахилл оборачиваются как раз вовремя, чтобы успеть увидеть, как крохотные блестящие осколки осыпаются дождем с полки, где только что стоял Бокал Г. Затем \--- мертвая тишина\ldots)}}

% TODO: illustration 19
\emph{Рис. 19. Последняя страница «Искусства фуги» Баха. На подлиннике рукой сына композитора, Карла Филиппа Эммануэля, написано: «NB: Во время исполнения этой фуги, в тот момент когда прозвучала мелодия \mbox{B-A-C-H}, композитор скончался». (На рисунке мелодия \mbox{B-A-C-H} взята в рамку) Пусть последняя страница Баховского «Контрапункта» послужит здесь как эпитафия. (Ноты отпечатаны при помощи компьютерной программы СМУТ, разработанной Дональдом Бирдом в Индианском университете США.)}

\end{dialogue}

\end{document}
