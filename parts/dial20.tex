\documentclass[../main.tex]{subfiles}
\begin{document}

\DialogueChapter{Канон Ленивца}

\centerblock{
    \emph{На этот раз мы находим Ахилла и Черепаху Тортиллу в гостях у их нового приятеля, Ленивца Сплюшки.}
}

\begin{dialogue}

\speak{Ахилл} Хотите послушать про забавное соревнование по бегу, которое мы однажды устроили с Черепахой?

\emph{Ленивец} : О, да, прошу вас!

\emph{Ахилл} : Это соревнование стало довольно известным --- я слышал, что оно даже было описано неким Зеноном.

\emph{Ленивец} : Это интересно, и я всегда настроен вас послушать.

\emph{Ахилл} : Это, и правда, было интересно. Видите ли, г-жа Тортилла стартовала первой. У нее была огромная фора, и тем не менее ---

\emph{Ленивец} : Вы ее нагнали, не правда ли?

\emph{Ахилл} : Разумеется --- поскольку я такой быстроногий, я сокращал расстояние между нами с постоянной скоростью, и вскоре обогнал ее.

\emph{Ленивец} : Поскольку расстояние между вами становилось все меньше и меньше, вам удалось это сделать.

\emph{Ахилл} : Именно. О, смотрите --- г-жа Черепаха принесла свою скрипку. Можно мне попробовать что-нибудь сыграть, г-жа Черепаха?

\emph{Черепаха} : О, нет, прошу вас. Это будет неинтересно --- она расстроена так, что невозможно слушать.

\emph{Ахилл} : Ну ладно. Но у меня сегодня почему-то музыкальное настроение.

\emph{Ленивец} : Можете поиграть на пианино, Ахилл.

\emph{Ахилл} : Благодарю вас. Немного погодя я так и сделаю. Но сначала доскажу, что потом мы с Черепахой бегали наперегонки еще раз. К несчастью, в этом соревновании ---

\emph{Черепаха} : Вы меня не нагнали, не правда ли? Расстояние между нами становилось все больше и больше, так что вам не удалось этого сделать.

\emph{Ахилл} : Это верно. Мне кажется, что и ЭТО соревнование было описано --- неким Льюисом Кэрроллом. А теперь, м-р Сплюшка, я готов принять ваше любезное приглашение и сыграть что-нибудь на пианино. Но предупреждаю: я очень плохой пианист --- даже не знаю, стоит ли мне пытаться.

\emph{Ленивец} : Вы должны попытаться.

\emph{(Ахилл садится за пианино и начинает играть простенькую мелодию.)}

\emph{Ахилл} : Ой, как странно звучит! Это совершенно не то, что я хотел сыграть! Что-то здесь не в порядке.

\emph{Черепаха} : Вы не можете играть на пианино, Ахилл. Вы не должны и пытаться.

\emph{Ахилл} : Это похоже на отражение пианино в зеркале. Высокие ноты находятся слева, а низкие --- справа. Каждая мелодия получается перевернутой с ног на голову. Интересно, кто это придумал такую дурацкую шутку?

\emph{Черепаха} : Этим отличаются ленивцы. Они висят ---

\emph{Ахилл} : Да, я знаю: на ветвях деревьев --- головой вниз, разумеется. Это пианино-ленивец годится, чтобы играть на нем перевернутые мелодии, которые встречаются в канонах и фугах. Но научиться играть на пианино, свисая с дерева, нелегко --- это требует немалого трудолюбия.

\emph{Ленивец} : Этим ленивцы не слишком отличаются.

\emph{Ахилл} : Да, мне кажется, что ленивцы не любят утруждать себя. Они делают все вдвое медленнее, чем все остальные. И кроме того, вверх ногами. Какой своеобразный жизненный уклад! Кстати, о замедленных и перевернутых вещах --- в «Музыкальном приношении» есть канон под названием «Canon per augmentationem contrario motu», что значит «увеличенный и перевернутый канон». В моем издании «Приношения» перед тремя нотными строчками стоят три буквы --- «S», «А» и «T» . Непонятно, что они значат\ldots{} Так или иначе, Бах сделал все это очень ловко. Как вы считаете, г-жа Тортилла?

\emph{Черепаха} : Он превзошел самого себя. Что касается букв «S», «А» и «T», то я догадываюсь, что они означают.

\emph{Ахилл} : «Сопрано», «Альт» и «Тенор», скорее всего. Трехчастные пьесы часто пишутся для этой комбинации голосов. Как вы думаете, м-р Сплюшка?

\emph{Ленивец} : Они означают ---

\emph{Ахилл} : О, подождите минутку! Г-жа Черепаха, почему вы одеваетесь? Надеюсь, вы не хотите нас покинуть? Мы только что собирались приготовить блинчики\ldots{} Вы выглядите очень усталой. Как вы себя чувствуете?

\emph{Черепаха} : Обессиленной. Пока! \emph{(Устало ползет к двери.)}

\emph{Ахилл} : Бедняжка --- она, действительно, выглядит измученной. Она пробегала все утро -\/--- тренировалась для следующего соревнования со мной.

\emph{Ленивец} : Видно, она превозмогала саму себя.

\emph{Ахилл} : И совершенно зря. Может быть, она могла бы перегнать Сплюшку\ldots{} но меня? Никогда! Да, вы, кажется, хотели сказать, что означают буквы «S», «А»,~«T» ?

\emph{Ленивец} : Вам в жизни не догадаться!

\emph{Ахилл} : Неужели они могут значит что-то еще? Вы меня заинтриговали. Я еще над этим поразмыслю. Скажите, а на каком молоке вы делаете тесто для блинчиков?

\emph{Ленивец} : На обезжиренном.

\emph{Ахилл} : Хорошо. Я ставлю сковородку на огонь.

\emph{Ленивец} : Уже?

\emph{Ахилл} : Ну ладно, тогда сначала размешаю тесто. Какие блинчики будут --- пальчики оближешь! Жаль, что Черепахе не удастся их отведать.

\emph{Рис. 133. «Канон Ленивца», из «Музыкального Приношения» И. С. Баха. (Ноты напечатаны с помощью компьютерной программы СМУТ Дональда Бирда.)}

\end{dialogue}

\end{document}
