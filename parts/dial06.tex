\documentclass[../main.tex]{subfiles}
\begin{document}

\DialogueChapter{Канон с интервальным увеличением}

\centerblock{%
    \emph{Ахилл и Черепаха только что доели превосходный ужин на двоих в лучшем китайском ресторане города.}
}

\begin{dialogue}

\speak{Ахилл} Здорово вы управляетесь с палочками, г-жа Ч\@.

\speak{Черепаха} Приходится \--- я с детства питаю слабость к восточной кухню. Как насчет вас, Ахилл \--- вам понравилось?

\speak{Ахилл} Еще как! Я никогда раньше не пробовал китайской еды, и сегодняшний ужин был приятным знакомством с ней. А сейчас, если вы не торопитесь мы можем ещё немного посидеть и поболтать.

\speak{Черепаха} Что ж, с удовольствием побеседую с вами, пока мы пьем чай. Официант!

\stage{\emph{(Подходит официант.)}}

Пожалуйста, принесите наш счет. И ещё немного чая!

\stage{\emph{(Официант торопливо уходит.)}}

\speak{Ахилл} Вы можете понимать больше меня в китайской кухне, г-жа Ч, но могу поспорить, что о японской поэзии я знаю побольше вас. Читали ли вы когда-нибудь хайку?

\speak{Черепаха} Боюсь, что нет. Что это такое?

\speak{Ахилл} Хайку \--- это японская поэма, в которой семнадцать слогов. Правильнее сказать, что это мини-поэма, наводящая на размышление так,же, как благоуханный розовый лепесток или покрытые росой кувшинки в пруду. Обычно хайку состоит из группы пяти слогов, затем \--- семи, и затем \--- снова пяти.

\speak{Черепаха} Такая краткость \--- всего семнадцать слогов \--- но где же здесь смысл?

\speak{Ахилл} Смысл живет также в голове читателя \--- не только в хайку.

\speak{Черепаха} Гм-м-м\ldots{} Это утверждение наводит на размышления.

\stage{\emph{(Подходит официант со счетом, чайничком, полным чая, и парой печений «с сюрпризом» \--- бумажкой, на которой написана судьба едока.)}}

Премного благодарна. Еще чайку не желаете, Ахилл?

\speak{Ахилл} Пожалуй. Эти печеньица выглядят весьма аппетитно. \emph{(Берет печенье, откусывает кусочек и начинает жевать.)} Эй \--- что эта за штуковина тут внутри? Клочок бумаги?

\speak{Черепаха} Это ваша судьба, Ахилл. Во многих китайских ресторанах вместе со счетом подают печенья с судьбой-сюрпризом, чтобы смягчить удар. Завсегдатаи китайских ресторанов обычно считают их не за печенья, а за посланцев судьбы. К несчастью, вы, кажется, проглотили кусочек своей судьбы. Что там написано, на оставшемся клочке?

\speak{Ахилл} Странно \--- все буквы сгрудились в кучу, нет никакого деления на слова. Может быть, это надо расшифровать? О, я понял если расставить промежутки там, где надо, получится: «НИС КЛАДУН ИЛ АДУ». Поистине, адская бессмыслица! Может быть, это что-то вроде хайку, от которого я отъел большинство слогов.

\speak{Черепаха} В таком случае, ваша судьба теперь всего лишь 6/17 хайку. Веселенькие ассоциации всё это вызывает. Колдуны, болота, черти, клады\ldots{} Что и говорить, картинка унилая\ldots{} унылая, я имею в виду. Это звучит как комментарий к новой форме искусства \--- 6/17 хайку. Можно мне взглянуть?

\speak{Ахилл (протягивая Черепахе узкий клочок бумаги)} Конечно.

\speak{Черепаха} Но, Ахилл, в моей «расшифровке» получается нечто совершенно другое! Это вовсе не 6/17 хайку, а шестисложное послание \--- и вот что в нем написано «НИ СКЛАДУ НИ ЛАДУ». Поистине, глубокий комментарий к этой новой форме искусства \--- 6/17 хайку!

\speak{Ахилл} Вы правы. Удивительно, что это послание содержит комментарий о самом себе!

\speak{Черепаха} Я только передвинула рамку чтения на единицу \--- сдвинула все промежутки между словами на один интервал.

\speak{Ахилл} Посмотрим, какая судьба выпала сегодня вам.

\speak{Черепаха (ловко разламывая печенье, читает)} «Судьбу едока не печенье содержит, а его рука».

\speak{Ахилл} Ваша «судьба» тоже хайку, г-жа Черепаха \--- по крайней мере, в ней семнадцать слогов. \mbox{5-7-5}.

\speak{Черепаха} Потрясающе! Я бы сама этого ни за что не заметила, Ахилл \--- такие вещи только вы подмечаете. То, что меня больше всего удивило, это сам текст послания; разумеется, его можно интерпретировать по-разному.

\speak{Ахилл} Наверное, мы все интерпретируем послания по-своему, когда с ними сталкиваемся\ldots{}

\stage{\emph{(Лениво рассматривает чаинки на дне чашки.)}}

\speak{Черепаха} Подлить вам чаю?

\speak{Ахилл} Да, спасибо. Кстати, как поживает ваш товарищ, старый Краб? Я частенько о нем вспоминаю, с тех пор, как вы рассказали мне о его диковиной патефонной войне.

\speak{Черепаха} Я ему о вас кое-что рассказала, и ему тоже не терпится с вами встретиться. У него всё в порядке;на днях он приобрел новую штуковину из серии проигрывателей, какой-то странный проигрыватель-автомат.

\speak{Ахилл} Расскажите-ка мне об этом поподробнее. Обожаю эти автоматы \--- кругом разноцветные огоньки, и когда опустишь монетку, машина играет глупые песни, которые так и окунают тебя в старое доброе прошлое\ldots{}

\speak{Черепаха} Этот проигрыватель слишком велик, чтобы держать его дома, и Краб построил для него во дворе специальный навес.

\speak{Ахилл} Не представляю себе, почему он такой большой? Может, в нем огромная коллекция пластинок?

\speak{Черепаха} На самом деле, в нем всего одна запись.

\speak{Ахилл} Что? Проигрыватель-автомат с одной пластинкой? Это уже само по себе противоречие! Почему же он так велик? Может, его единственная пластинка \--- гигант двадцати футов в диаметре?

\speak{Черепаха} Да нет, пластинка самая обыкновенная.

\speak{Ахилл} Ах, г-жа Черепаха, не иначе как вы надо мной смеетесь. Ну скажите на милость, что это за автомат с единственной песней?

\speak{Черепаха} Кто сказал хотя бы слово о единственной песне?

\speak{Ахилл} Любой проигрыватель-автомат, с которым я когда-либо сталкивался, подчинялся фундаментальной аксиоме этих аппаратов: «одна пластинка, одна песня».

\speak{Черепаха} Этот автомат не таков, Ахилл. Единственная пластинка в нем расположена вертикально, и за ней находится небольшая, но сложная система рельсов, на которых подвешены проигрыватели. Когда вы нажимаете на пару кнопок, скажем, \mbox{В-1}, вы выбираете один из проигрывателей. Это пускает в действие механизм, и проигрыватель со скрипом отправляется по ржавым рельсам. Вскоре он прибывает к краю пластинки, и \--- щелк! \--- устанавливается в нужную позицию.

\speak{Ахилл} И тогда пластинка начинает вращаться, и раздается музыка, правда?

\speak{Черепаха} Не совсем. Пластинка остается неподвижной \--- вращается сам проигрыватель.

\speak{Ахилл} Я мог бы догадаться. Но каким же образом, если у вас только одна пластинка, вы можете выудить из этой сумасшедшей конструкции больше одной песни?

\speak{Черепаха} Я и сама спрашивала Краба об этом. Он посоветовал мне попробовать самой. Я нашла в кармане монетку (её хватало на три песни), засунула её в щель и нажала наугад: \mbox{В-1}, \mbox{С-3}, и \mbox{V-10}.

\speak{Ахилл} Значит, патефон В-1 поехал по рельсам, подкатился к вертикальной пластинке и стал вращаться?

\speak{Черепаха} Точно. Получилась довольно приятная музыка, основанная на знаменитой старой мелодии \mbox{B-A-C-H}, которую, я полагаю, вы ещё помните\ldots{}

\speak{Ахилл} Могу ли я её забыть?

\speak{Черепаха} Это был патефон \mbox{В-1}. Когда мелодия закончилась, он отъехал назад, чтобы дать место патефону \mbox{С-3}.

\speak{Ахилл} Неужели \mbox{С-3} заиграл другую мелодию?

\speak{Черепаха} Именно так.

\speak{Ахилл} А, понимаю. Он проиграл другую сторону пластинки, или, может быть, другую полосу на этой стороне.

\speak{Черепаха} Нет, на этой пластинке дорожки только с одной стороны и на ней только одна полоса.

\speak{Ахилл} Ничего не понимаю. Получить разные песни из одной записи НЕВОЗМОЖНО!

\speak{Черепаха} Я тоже так думала, пока не увидела проигрыватель м-ра Краба.

\speak{Ахилл} Как звучала эта вторая песня?

\speak{Черепаха} Это-то как раз интересно: она была основана на мелодии \mbox{C-A-G-E}.

\speak{Ахилл} Но это совершенно иная мелодия!

\speak{Черепаха} Верно.

\speak{Ахилл} Кажется, Джон Кэйдж \--- это композитор, создатель авангардистской музыки? Мне кажется, я читал о нем в одной из моих книг хайку.

\speak{Черепаха} Точно. Многие его творения довольно известны, например, $4'33''$ \--- трехчастная пьеса, состоящая из безмолвий разной длины. Она необыкновенно выразительна \--- если у вас есть вкус к подобным вещам.

\speak{Ахилл} Что ж, если бы я находился в шумном ресторане, я с удовольствием поставил бы $4'33''$ Кэйджа на музыкальном автомате. Это могло бы быть некоторым облегчением!

\speak{Черепаха} Правильно \--- кому хочется слушать звон тарелок и стук ножей? Эта пьеса пришлась бы весьма кстати ещё в одном месте, в Павильоне~Гигантских Кошек, во время кормления.

\speak{Ахилл} Вы намекаете на то, что Кэйджу место в зверинце? Что ж, если учесть, что его фамилия в переводе с английского значит «клетка»\ldots{} Но вернемся к крабьему музыкальному автомату \--- я ничего не понимаю. Как могут на одной и той же записи быть сразу \mbox{B-A-C-H} и \mbox{C-A-G-E}?

\speak{Черепаха} Если вы посмотрите повнимательней, Ахилл, вы можете подметить, что между ними есть некоторая связь. Вот, взгляните: что у вас получится, если вы последовательно запишете интервалы мелодии \mbox{B-A-C-H}?

\speak{Ахилл} Ну-ка, посмотрим\ldots{} Сначала она понижается на полтона, от~B~до~A (я имею в виду немецкое~B); затем поднимается на три полутона до~C, и, наконец, опускается на полутон, до~H. Получается следующая схема:
\[
    -1, +3, -1
\]

\speak{Черепаха} Совершенно верно. А как насчет \mbox{C-A-G-E}?

\speak{Ахилл} Здесь мелодия сначала идет на три полутона вниз, потом поднимается на десять полутонов, и снова опускается на три полутона. Получается:
\[
    -3, +10, -3
\]
Очень похоже на первую мелодию, правда?

\speak{Черепаха} Действительно, похоже. В некотором смысле, у этих двух мелодий совершенно одинаковый «скелет». Вы можете получить \mbox{C-A-G-E} из \mbox{B-A-C-H}, умножив все интервалы на $3.5$ и беря ближайшее целое число.

\speak{Ахилл} Вот это да! Это значит, что на звуковых дорожках записан только некий основной код, который разные проигрыватели интерпретируют по-разному?

\speak{Черепаха} Я не уверена \--- этот уклончивый Краб не посвятил меня во все детали. Но мне удалось услышать третью песню, произведенную на проигрывателе~\mbox{В-10}.

\speak{Ахилл} И как она звучала?

\speak{Черепаха} её мелодия состояла из огромных интервалов: \mbox{B-C-A-H}.

Схема в полутонах была такая:
\[
    -10, +33, -10
\]
Эта мелодия получается из \mbox{C-A-G-E}, если снова умножить интервалы на $3.3$ и округлить результаты до ближайшего целого числа.

\speak{Ахилл} Есть ли какое-то название у такого умножения интервалов?

\speak{Черепаха} Его можно назвать «интервальным увеличением». Оно похоже на прием ритмического увеличения темы канона. При этом длительность всех нот мелодии умножается на какое-либо постоянное число. В результате мелодия замедляется. Здесь же интересным образом расширяется диапазон мелодии.

\speak{Ахилл} Удивительно. Так что все три мелодии, что вы услышали, были интервальными увеличениями одной и той же схемы звуковых дорожек?

\speak{Черепаха} Таково мое заключение.

\speak{Ахилл} Забавно, когда мы увеличиваем \mbox{B-A-C-H}, у нас получается \mbox{C-A-G-E}, a когда мы опять увеличиваем \mbox{C-A-G-E}, то снова получаем \mbox{B-A-C-H}, только теперь он весь перевернут, словно \mbox{B-A-C-H} разнервничался, проходя через промежуточный этап \mbox{C-A-G-E}.

\speak{Черепаха} Поистине, глубокий комментарий к этой новой форме искусства \--- музыке Кэйджа.

\end{dialogue}

\end{document}
