\documentclass[../main.tex]{subfiles}
\begin{document}
\begingroup

\newtcolorbox{koan}[1][]{% [<style>]
    boxrule=0.3mm,
    fontupper=\small,
    left skip=\parindent,
    right skip=2pc,
    enlarge top initially by=2pt,
    enlarge bottom finally by=2pt,
    colback=black!3,
    #1
}

\NewEnviron{mumonverse}{%
    \begin{adjustwidth}{4pc}{}
        \BODY\par
    \end{adjustwidth}
}

\Chapter{Мумон и Гёдель}

\subsection{Что такое дзен-буддизм?}

Я НЕ УВЕРЕН В ТОМ, что знаю, что такое дзен. В каком-то смысле мне кажется, что я понимаю его очень хорошо; с другой стороны, иногда я думаю, что никогда не пойму в нём ничего. С тех пор, как на первом курсе университета профессор английской литературы прочитал нам «\enquote*{МУ} Джошу», я начал бороться с дзен-буддистскими аспектами жизни и, наверное, никогда не перестану. Для меня дзен подобен зыбучим пескам \--- анархия, темнота, бессмыслица, хаос. Он дразнит и приводит в бешенство. И в то же время дзен полон юмора, свежести и привлекательности. В нём есть собственный тип значения, блеска и ясности. Надеюсь, что, прочитав эту главу, вы это почувствуете. И эта тема, как ни странно может показаться, выведет нас прямо к Гёделю.

Одна из главных идей дзен-буддизма в том, что его невозможно определить. Как бы вы не пытались заключить его в словесные рамки, он сопротивляется и вырывается на свободу. Может показаться, что в таком случае все попытки объяснить дзен \--- это пустая трата времени. Но ученики и мастера дзена так не считают. Например, буддистские коаны \--- центральная часть изучения дзена, хотя они и состоят из слов. Коаны призваны служить «триггерами» \--- сами по себе они не содержат достаточно информации, чтобы вызвать Просветление, но могут привести в действие внутренние механизмы, ведущие к Просветлению. Но в общем дзен утверждает, что слова и истина несовместимы \--- словами невозможно уловить истину.


\subsection{Мастер дзена Мумон}

Возможно, чтобы лучше выразить эту идею, монах Мумон (что в переводе означает «Нет выхода»), живший в тринадцатом веке, собрал сорок восемь коанов, снабдив каждый из них комментарием и небольшим «стихотворением». Этот труд называется «Безвыходный выход» или «Мумонкан». Интересно заметить, что даты жизни Мумона и Фибоначчи почти точно совпадают: Мумон жил в Китае с~1183 по~1260 год, а Фибоначчи \--- в Италии с~1180 по~1250 год. Те, кто попытаются «понять» коаны «Мумонкана», найти в них смысл, будут горько разочарованы, поскольку как сами коаны, так и комментарии к ним и стихотворения абсолютно туманны. Приведу несколько примеров:\footnote{Paul Reps «Zen Flesh Zen Bones» стр.~110\==111}

\begin{koan}
    \emph{\normalsize Коан:}
    \vspace{2pt}

    Хоген из монастыря Сеирио собирался читать обычную лекцию перед ужином; вдруг он заметил, что бамбуковая занавесь, опущенная для медитации, еще не поднята. Он указал на нее; два монаха безмолвно встали и подняли занавесь. Хоген, наблюдая за этим физическим моментом, заметил: «Состояние первого монаха \--- хорошо, но не состояние второго».
\end{koan}

\begin{koan}
    \emph{\normalsize Комментарий Мумона:}
    \vspace{2pt}

    Я спрашиваю вас: кто из этих двух монахов выиграл, а кто проиграл? Если у кого-то из вас \--- один глаз, тот заметит ошибку Учителя. Но я не обсуждаю выигрыша и проигрыша.
\end{koan}

\begin{koan}
    \emph{\normalsize Стихотворение Мумона:}

    \begin{mumonverse}
        Когда занавесь поднята, открывается широкое небо \--- \\
        Но небо не созвучно дзену. \\
        Лучше забыть о широком небе \\
        И спрятаться от любого ветра.
    \end{mumonverse}
\end{koan}

% TODO: illustration 46
\emph{Рис. 46. М.\,К.~Эшер. «Три мира» (литография, 1955)}

А вот ещё:\footnote{Там же стр.~119}

\begin{koan}
    \emph{\normalsize Коан:}
    \vspace{2pt}

    Госо сказал: «Когда бизон выходит из укрытия на край пропасти, его рога, и голова, и копыта проходят; но почему не может пройти его хвост?»
\end{koan}

\begin{koan}
    \emph{\normalsize Комментарий Мумона:}
    \vspace{2pt}

    Если кто-нибудь сейчас может открыть один глаз и сказать слово дзена, тот готов к тому, чтобы отплатить за четыре награды \--- более того, он сможет спасти всех существ, стоящих ниже него. Но если он не может сказать слова дзена, то он должен повернуться обратно к своему хвосту.
\end{koan}

\begin{koan}
    \emph{\normalsize Стихотворение Мумона:}

    \begin{mumonverse}
        Если бизон побежит, он упадет в пропасть; \\
        Если он повернет назад, его зарежут. \\
        Очень странная штука \--- \\
        Этот хвост!
    \end{mumonverse}
\end{koan}

Я думаю, вы согласитесь с тем, что объяснения Мумона не многое проясняют. Можно сказать, что в данном случае метаязык (на котором пишет Мумон) не слишком отличается от предметного языка (языка коанов). Некоторые считают, что комментарии Мумона \--- намеренно идиотские, и что он хочет показать, насколько бесполезно тратить время на разговоры о дзене. Однако комментарии Мумона можно понять на нескольких уровнях. Например, давайте рассмотрим следующий:

\begin{koan}
    \emph{\normalsize Коан:}\footnote{Там же стр. 111\==112}
    \vspace{2pt}

    Один монах спросил Нансена: «Есть ли поучение, которое не произнес ни один мастер?» \\
    Нансен сказал: «Да, есть». \\
    «Какое же оно?» \--- спросил монах. \\
    Нансен ответил: «Это не разум, это не Будда, это не вещи».
\end{koan}

\begin{koan}
    \emph{\normalsize Комментарий Мумона:}
    \vspace{2pt}

    Старый Нансен раскрыл свои заветные слова. \\
    Наверняка, он был очень взволнован.
\end{koan}

\begin{koan}
    \emph{\normalsize Стихотворение Мумона:}

    \begin{mumonverse}
        Нансен был слишком добр и потерял свое сокровище. \\
        Поистине, слова бессильны. \\
        Даже если гора обратится в море, \\
        Слова не могут открыть разум другого.
    \end{mumonverse}
\end{koan}

Кажется, что в этой поэме Мумон говорит нечто центральное для дзен-буддизма и не делает никаких дурацких заявлений. Любопытно, однако, что поэма автореферентна и, таким образом, комментирует не только слова Нансена, но и свою собственную безрезультатность. Подобные парадоксы характерны для дзена. Это попытка «сломить разум логики». То же парадоксальное качество вы можете увидеть и в коане. Говоря о комментарии Мумона \--- как вы думаете, был ли Нансен так уверен в своем ответе? И важна ли «правильность» его ответа? Играет ли вообще правильность какую-либо роль в дзен-буддизме? Какая разница между правильностью и истинностью, и есть ли она вообще? Что, если бы Нансен сказал: «Нет, такого поучения нет»? Что бы это изменило? Был бы подобный ответ увековечен в коане?

\emph{Рис. 47. М.\,К.~Эшер. «Капля росы» (глубокая печать, 1948).}

Вот еще один коан, направленный на то, чтобы сломить разум логики:\footnote{«Zen Buddhism» (Mount Vernon NY Peter Pauper Press 1959) стр. 22}

\begin{koan}
    Ученик Доко пришел к мастеру дзена и сказал: «Я ищу истину. До какого состояния я должен натренировать свои разум, чтобы ее найти?»

    Мастер ответил: «Поскольку разума не существует, его невозможно привести ни в какое состояние. Поскольку истины не существует, невозможно натренировать себя для ее обретения».

    «Если нет ни разума, чтобы тренировать его, ни истины, чтобы ее искать, то зачем же вы каждый день собираете перед собой монахов для тренировки и изучения дзена?»

    «Но у меня здесь нет ни дюйма места», \--- сказал мастер, «как же здесь могут собираться монахи? У меня нет языка \--- как же я могу созывать или учить их?»

    «Как вы можете так лгать?» \--- спросил Доко.

    «У меня нет языка, чтобы разговаривать с другими \--- так как же я могу лгать тебе?» \--- спросил мастер.

    Тогда Доко грустно заметил: «Я не могу уследить за вашей мыслью. Я вас не понимаю».

    «Я сам себя не понимаю».
\end{koan}

Если какой-либо коан и служит для того, чтобы запутать слушателя, то именно этот. И скорее всего, в этом и есть его прямое назначение, потому что когда наш разум заходит в тупик, он начинает оперировать до какой-то степени нелогично. Теория говорит нам, что только отходя от логики, человек может достичь Просветления. Но что же такого плохого в логике? Почему она не позволяет нам совершить скачок к Просветлению?


\subsection{Борьба дзена против дуализма}

Чтобы ответить на эти вопросы, необходимо знать кое-что о Просветлении. Возможно, что самым коротким его определением было бы следующее: выход за пределы дуализма. Что же такое дуализм? Это мысленное разделение мира на категории. Возможно ли преодолеть это естественное стремление? Предваряя слово «разделение» словом «мысленное», я мог создать у вас впечатление, что это \--- интеллектуальное или сознательное усилие и, значит, дуализм можно преодолеть, просто остановив мысли (словно это так легко \--- перестать думать!). На самом деле, разбиение мира на категории происходит гораздо глубже самого высокого уровня мышления: дуализм настолько же процесс \emph{восприятия} мира, как и его \emph{понимания}. Иными словами, человеческое восприятие по своей природе дуалистично \--- что делает борьбу за просветление титанической, если не сказать большего.

% TODO: illustration 48
\emph{Рис. 48. М.\,К.~Эшер. «Иной мир» (гравюра на дереве, 1947)}

В сердце дуализма, согласно дзену, лежат слова \--- простые, обыкновенные слова. Использование слов всегда дуалистично \--- очевидно, что каждое слово представляет собой определенную умозрительную категорию. Отсюда следует, что большая часть дзена посвящена борьбе против доверия к словам. Одно из лучших оружий в этой борьбе \--- коан, поскольку со словами он обращается настолько неуважительно, что наш разум тут же забуксует, если мы попытаемся воспринимать коан серьезно. Может быть, неверно говорить, что врагом Просветления является логика; скорее, это дуалистичное, словесное мышление. Или даже еще проще, восприятие. Воспринимая предмет, вы тут же отграничиваете его от всего остального мира; вы делите мир на части и, таким образом, отходите от Пути.

% TODO: illustration 49
\emph{Рис. 49. М.\,К.~Эшер. «День и ночь» (гравюра на дереве, 1938).}

Вот коан, демонстрирующий борьбу против слов:\footnote{Reps стр. 124}

\begin{koan}
    \emph{\normalsize Коан:}
    \vspace{2pt}

    Шузан протянул вперед свою короткую палку и сказал: «Если вы скажете, что это короткая палка, то согрешите против действительности. Если вы не скажете, что это короткая палка, то проигнорируете факт. Так что же вы скажете?»
\end{koan}

\begin{koan}
    \emph{\normalsize Комментарий Мумона:}
    \vspace{2pt}

    Если вы скажете, что это короткая палка, то согрешите против действительности. Если вы не скажете, что это короткая палка, то проигнорируете факт. Это нельзя выразить словами, и это нельзя выразить без слов. Теперь быстро говорите, что это такое.
\end{koan}

\begin{koan}
    \emph{\normalsize Стихотворение Мумона:}

    \begin{mumonverse}
        Протягивая вперед короткую палку, \\
        Он дал приказ о жизни и смерти. \\
        Позитивное и негативное переплетены, \\
        Даже Будды и патриархи не могут избежать этой атаки.
    \end{mumonverse}
\end{koan}

(Под «патриархами» здесь имеются в виду шесть почитаемых основателей дзен-буддизма, из которых первым был Бодхидхарма и шестым \--- Энон.)

Почему назвать это короткой палкой означало пойти против действительности? Может быть, потому, что подобная категоризация дает иллюзию углубления в действительность, в то время как на самом деле это утверждение даже не поцарапало ее поверхности. Можно сравнить его с утверждением «5 \--- простое число». Это утверждение оставляет без внимания огромное, бесконечное количество фактов. С другой стороны, не назвать ее короткой палкой \--- означает проигнорировать этот факт, как бы незначителен он не был. Следовательно, слова ведут к частичной истинности \--- и, возможно, к частичной ложности \--- но, безусловно, не к полной истине. Надеяться на слова, чтобы найти истину \--- всё равно, что надеяться на неполную формальную систему, чтобы найти истину. Формальная система даст вам некоторые истины, но, как мы скоро увидим, формальная система, какой бы мощью она не обладала, не может привести ко всем истинным высказываниям. Дилемма математиков такова: на что еще можно опираться, кроме формальных систем? Дилемма последователей дзена такова: на что еще можно опираться, кроме слов? Мумон выражает эту дилемму с предельной ясностью: «Это нельзя выразить словами, и это нельзя выразить без слов».

% TODO: illustration 50
\emph{Рис. 50. М.\,К.~Эшер. «Кожура» (гравюра на дереве, 1955).}

Вот еще один коан о Нансене:\footnote{«Zen Buddhism» стр. 38}

\begin{koan}
    Джошу спросил учителя Нансена: «Какой Путь правилен?»

    Нансен ответил: «Правилен повседневный Путь».

    Джошу спросил: «Могу ли я его изучать?»

    Нансен ответил: «Чем больше вы его изучаете, тем больше вы удаляетесь от него».

    Джошу спросил: «Если я не буду его изучать, как же я его узнаю?»

    Нансен ответил: «Путь не принадлежит увиденным вещам и не принадлежит неувиденным вещам. Он не принадлежит известным вещам, и он не принадлежит неизвестным вещам. Не ищи его, не изучай его и не называй его. Чтобы оказаться на Пути, стань открытым и широким как небо». (См.~рис.~50.)
\end{koan}

Кажется, что это любопытное утверждение полно парадоксов. Оно немного напоминает следующее верное средство от икоты: «Обегите трижды вокруг дома, не думая о слове \enquote*{волк}». Дзен-буддизм \--- это философия, которая, по-видимому, считает, что дорога к абсолютной истине, так же, как единственный верный способ против икоты, должна изобиловать парадоксами.


\subsection{Изм, режим U и Унмон}

Если слова \--- плохи, и мышление \--- плохо, то что же тогда хорошо? Разумеется, сам по себе такой вопрос весьма дуалистичен, но поскольку, обсуждая его, мы не претендуем на верность дзену, то попытаемся ответить на него серьезно. Назовем то, к чему стремится дзен, \emph{измом}. Изм \--- это антифилософия, способ существования без мышления. Мастерами изма являются камни, деревья, моллюски. Существам же, стоящим на более высокой ступени развития, приходится за это бороться; при этом они никогда не достигнут полного изма. Всё же нам иногда удается увидеть проблеск изма; возможно, следующий коан покажет вам такой проблеск:\footnote{Reps стр. 121}

\begin{koan}
    Хиакуйо захотел послать монаха, чтобы открыть новый монастырь. Он сказал ученикам, что назначит того из них, кто сумеет лучше всех ответить на его вопрос. Поставив кувшин с водой на землю, он сказал: «Кто может сказать, что это такое, не называя при этом его имени?»

    Главный монах сказал: «Никто не может назвать это деревянным башмаком».

    Повар Изан перевернул кувшин ногой и ушел.

    Хиакуйо улыбнулся и сказал: «Главный монах проиграл». И Изан стал Мастером нового монастыря.
\end{koan}

Сущность дзена \--- и изма \--- в том, чтобы подавить восприятие, подавить логическое, словесное, дуалистичное мышление. Это и есть \emph{режим \textbf{U}} \--- \emph{Ультра}; не Интеллектуальный, не Механический, а просто «Ультра». Джошу действовал по способу \textbf{U}; поэтому его МУ «развопросило» вопрос. Для Мастера дзена Унмона способ \textbf{U} был естественным:\footnote{Gyomay M.~Kubose «Zen koans» стр. 35}

\begin{koan}
    Однажды Унмон сказал своим ученикам: «Моя палка превратилась в дракона и проглотила вселенную! Где же теперь реки, и горы, и великая Земля?»
\end{koan}

Дзен \--- это холизм, доведенный до логической крайности. Если холизм (от английского \emph{whole} \--- целое) утверждает, что вещи могут быть поняты только как целое, а не как сумма их частей, то дзен идет еще дальше, утверждая, что мир вообще не может быть разделен на части. Делить мир на части \--- это создавать иллюзии и терять возможность Просветления.

\begin{koan}
    Один любопытный монах спросил Мастера: «Что такое Путь?»

    «Он у тебя перед глазами», \--- ответил Мастер.

    «Почему же я сам его не вижу?»

    «Потому что ты думаешь о себе».

    «А вы \--- вы его видите?»

    «До тех пор, пока ты всё представляешь в двойном свете, говоря \enquote*{я вижу}, \enquote*{вы не видите} и тому подобное, у тебя всегда будет туман перед глазами», \--- сказал Мастер.

    «А когда не станет ни \enquote*{Я} ни \enquote*{Вы}, его можно будет увидеть?»

    «Когда не станет ни \enquote*{Я} ни \enquote*{Вы}, кто будет тот, кто захочет его видеть?»\footnote{«Zen Buddhism» стр.~31}
\end{koan}

По-видимому, Мастер хочет сказать, что в состоянии Просветления границы между «Я» и остальным миром стираются.

Это было бы настоящим концом дуализма, поскольку тогда, как сказал Мастер, не осталось бы системы, жаждущей восприятия. Но что это такое, если не смерть? Как может живое человеческое существо стереть границы между собой и миром?


\subsection{Дзен и Лимбедламия}

Буддистский монах Бассуи написал письмо одному из своих учеников, который был при смерти; в письме он сказал: «Твой конец \--- бесконечен; он подобен снежинке, таящей в чистом воздухе». Снежинка, бывшая вполне заметной подсистемой, теперь растворяется в более широкой системе, когда-то ее содержавшей. Хотя она больше и не присутствует как отдельная система, ее сущность всё ещё сохраняется. Она кружится в Лимбедламии, рядом с неначавшейся икотой и буквами непрочитанных историй\ldots{} Так я понимаю письмо Бассуи.

Дзен признает свои ограничения, точно так же, как математики научились признавать ограничения аксиоматического метода для нахождения истины. Дзен не знает ответа на то, что лежит за его пределами, так же, как у математиков нет ясного понимания форм рассуждений, лежащих за пределами формализации. Одно из самых ясных утверждений дзена о его границах дано в следующем странном, весьма в духе Нансена, коане:\footnote{Kubose стр. 110}

\emph{Тозан сказал своим монахам: «Вы, монахи, должны знать, что в буддизме есть еще высшее понимание». Один монах вышел вперед и спросил: «Что такое высший буддизм?» Тозан ответил: «Это не Будда».}

Путь не кончается никогда; Просветление не означает конца буддизма. Не существует рецепта, говорящего, как можно переступить пределы буддизма; единственное, в чем можно быть уверенным, это то, что Будда \--- \emph{не} путь. Дзен \--- это система, и он не может быть своей собственной метасистемой; всегда есть что-то вне дзена, что не может быть полностью понято и описано в его терминах.

% TODO: illustration 51
\emph{Рис. 51. М.\,К.~Эшер. «Лужа» (гравюра на дереве, 1952).}


\subsection{Эшер и дзен}

В своих сомнениях относительно восприятия и своей любви к абсурдным загадкам без ответа дзен имеет единомышленника \--- М.\,К.~Эшера. Взгляните на «\emph{День и ночь} » (рис.~49) \--- шедевр «переплетения негативного и позитивного» (говоря словами Мумона). Читатель может спросить: «Что это такое на самом деле, птицы или поля? Что это, день или ночь?» Однако все мы знаем, что подобные вопросы задавать незачем. Эта картина, подобно буддистскому коану, пытается разбить разум логики. Эшер также находит удовольствие в создании противоречивых картин, таких, как «\emph{Иной мир} » (рис.~48) \--- картин, которые играют с реальностью и нереальностью на манер дзена. Должны ли мы принимать Эшера всерьез? Должны ли мы принимать дзен всерьез?

Взгляните на изысканный, подобный хайку рисунок отражений в «\emph{Капле росы} » (рис. 47), на спокойную луну, отраженную в тихой воде «\emph{Лужи} » (рис. 51) и на «\emph{Рябь на воде} » (рис. 52). Отражение луны \--- тема, встречающаяся в нескольких коанах. Вот лишь один пример:\footnote{Там же стр. 120}

\begin{koan}
    Чионо изучала дзен многие годы под руководством Букко из Энгаку. Всё же ей не удавалось достичь плодов медитации. Однажды лунной ночью она несла воду в старой деревянной бадье, опоясанной бамбуком. Бамбуковый обруч сломался, и дно бадьи выпало. В этот момент Чионо освободилась. Тогда она сказала: «Нет воды в ведре \--- нет и луны в воде».
\end{koan}

«Три мира» \--- картина Эшера (рис. 46) и тема дзен-буддистского коана:\footnote{Там же стр. 180}

\begin{koan}
    Один монах спросил Ганто: «Чте мне делать, когда мне угрожают три мира?» Ганто ответил: «Садись». «Я не понимаю», \--- сказал монах. Ганто сказал: «Подними гору и принеси ее мне. Тогда я тебе объясню».
\end{koan}

% TODO: illustration 52
\emph{Рис. 52. М.\,К.~Эшер. «Рябь на воде» (гравюра на линолеуме, 1950)}


\subsection{Гемиола и Эшер}

В картине «\emph{Вербум} » (рис. 149) противоположности превращены в единство на нескольких уровнях. Двигаясь по кругу, мы видим постепенные превращения черных птиц \--- в белых птиц \--- в белых рыб \--- в черных жаб \--- в белых жаб \--- в черных птиц\ldots После шести шагов мы оказались опять в начале! Не примирение ли это дихотомии белого и черного? Или «трихотомии» птиц, рыб и жаб? Или это \--- шестиступенчатое единство, сделанное из противопоставления четности двух и нечетности трех? В музыке шесть нот одинаковой длительности создают ритмическую двусмысленность: две группы по три ноты, или три группы по две? Эта двусмысленность имеет свое название: гемиола. Шопен был мастером гемиолы; см. его «Вальс» оп. 42, или «Этюд» оп. 25, номер 2. У Баха это «Темпо ди Менуетто» из клавишной партитуры номер 5 или удивительный финал соль минорной «Сонаты для скрипки соло».

Когда мы приближаемся к центру гравюры «Вербум», различия постепенно стираются, и к концу остается не три, не две, но одна единственная суть: Вербум \--- слово, сверкающее во всем блеске, возможно, символ Просветления. Ирония в том, что «вербум» не только \emph{является} словом, но и означает «слово» \--- не слишком-то совместимое с дзеном понятие. С другой стороны, «вербум» \--- единственное слово в картине. Мастер дзена Тозан однажды сказал «Вся \enquote*{Трипитака} может быть выражена в одной букве». («Трипитака» или «Три корзины» \--- это полный текст священных книг буддизма.) Интересно, какой декодирующий механизм понадобился бы, чтобы извлечь три корзины из одной буквы? Может быть, механизм с двумя полушариями?

% TODO: illustration 53
% TODO: move this image into the following section
\emph{Рис. 53. М.\,К.~Эшер. «Три сферы II» (литография, 1946).}


\subsection{Сеть Индры}

Наконец, давайте взглянем на «Три сферы II»; кажется, что каждая часть мира здесь содержит всё остальные и содержится в них сама: письменный стол отражает шары на его поверхности, шары отражают друг друга, а также стол, рисунок, их изображающий, и самого художника. Эта литография лишь намекает на бесконечное взаимодействие всех вещей \--- однако этого намека вполне достаточно. Буддистская аллегория «Сеть Индры» описывает бесконечную сеть, нити которой пронизывают всю вселенную: горизонтальные нити протянуты в пространстве, вертикальные \--- во времени. Каждая точка пересечения \--- это индивидуум, и каждый индивидуум \--- это стеклянная сфера. Великий свет «Абсолютного существа» освещает каждую стеклянную сферу и проникает сквозь нее; более того, каждая сфера отражает не только свет каждой другой сферы в сети, но и каждое отражение каждого отражения во вселенной.

Этот образ напоминает мне о ренормализованных частицах: в каждом электроне заключены виртуальные фотоны, позитроны, нейтрино, муоны\ldots; в каждом фотоне \--- виртуальные электроны, протоны, нейтроны, пионы\ldots; в каждом пионе\ldots{}

Затем на ум приходит другая картина: люди, каждый из которых отражен в голове многих других, которые, в свою очередь, отражены в уме кого-то другого, и так далее.

Обе эти картины могут быть представлены коротко и элегантно с помощью Увеличенных Схем Перехода. В случае частиц, у нас будет отдельная схема для каждой категории частиц; в случае людей \--- отдельная схема для каждого человека. Каждая из них будет вызывать много других, таким образом создавая виртуальное облако УСП вокруг каждой УСП\@. Вызов одной из них приведет к вызову других, и этот процесс может идти как угодно долго, пока мы не достигнем поверхности.


\subsection{Мумон о МУ}

Завершим нашу короткую экскурсию в дзен еще одним обращением к Мумону. Вот его комментарий к МУ Джошу:\footnote{Reps стр. 89-90}

Чтобы понять дзен, надо преодолеть барьер патриархов. Просветление всегда приходит после того, как преграждается дорога мысли. Если вы не преодолели барьера патриархов или если дорога вашей мысли не преграждена, то что бы вы не думали и что бы вы не делали, это будет лишь призрачная путаница. Вы можете спросить: «Что такое барьер патриархов?» Это лишь одно слово: «МУ».

Это барьер дзена. Если вы его преодолеете, то встретитесь с Джошу лицом к лицу. Тогда вы сможете работать плечом к плечу со всеми патриархами. Не чудесно ли это?

Если вы хотите преодолеть этот барьер, вы должны до мозга костей проникнуться вопросом: «Что такое МУ?» и размышлять об этом день и ночь. Не думайте, что это \--- обычное отрицание и означает ничто. Это не пустота, не противоположность существованию. Если вы действительно хотите преодолеть этот барьер, вы должны чувствовать себя так, словно ваш рот наполнен расплавленным металлом, который вы не можете не проглотить, ни выплюнуть.

Тогда исчезнет ваше предыдущее, меньшее знание. Как плод зреет по осени, так ваша объективность и субъективность естественно сольются в одно. Это похоже на немого, увидевшего сон: он знает о нём, но не может рассказать его.

Когда он достигнет этого состояния, скорлупа его эго разобьется и он сможет трясти небеса и двигать землю. Он станет подобен великому воину с острым мечом. Если Будда встанет на его дороге, он рассечет его своим мечом; если патриарх будет чинить ему препятствия, он убьет его; он будет свободен в своем рождении и смерти. Он сможет войти в любой мир, как в свой собственный дом. В этом коане я скажу вам, как этого добиться:

Сконцентрируйте всю вашу энергию на МУ и не отвлекайтесь ни на миг. Когда вы войдете в МУ, не позволяя себе останавливаться, вы станете словно свеча, своим пламенем освещающая всю вселенную.


\subsection{От Мумона к головоломке MU}

С головоломных высот МУ Джошу спустимся теперь к прозаическому MU Хофштадтера\ldots{} Я знаю, что вы уже пробовали сконцентрировать на нём всю вашу энергию, когда вы читали главу~I\@. Сейчас я отвечу на поставленный в ней вопрос:

Обладает ли \textbf{MU} природой теоремы?

Ответ на этот вопрос \--- не ускользающее \textbf{MU}, но полновесное НЕТ\@. Чтобы показать это, мы воспользуемся привилегиями дуалистического, логического мышления.

В главе I мы сделали два важных наблюдения:

(1) что сложность головоломки \textbf{MU} зависит от взаимодействия удлиняющих и укорачивающих правил;

(2) что тем не менее есть надежда решить эту задачу, пользуясь достаточно сложным орудием: теорией чисел.

В главе I мы не стали подробно анализировать головоломку \textbf{MU} с этой точки зрения; теперь пришло время это сделать. Скоро мы увидим, как второе наблюдение (вынесенное за пределы незначительной системы \textbf{MIU} ) стало одним из самых плодотворных открытий математики и как оно изменило взгляд математиков на их предмет.

Для удобства я повторю здесь основные положения системы MIU:

СИМВОЛЫ: \textbf{М} , \textbf{I} , \textbf{U} .

АКСИОМА: \textbf{MI}

ПРАВИЛА:

I. Если \emph{х} \textbf{I} \--- теорема, то \emph{x} \textbf{IU} \--- также теорема.

II. Если \textbf{M} \emph{x \---} теорема, то \textbf{M} \emph{хх} \--- также теорема.

III. В любой теореме \textbf{III} может быть заменено на \textbf{U} .

IV. \textbf{UU} может быть вычеркнуто из любой теоремы.


\subsection{Мумон показывает нам, как решить головоломку MU}

Согласно приведенным выше наблюдениям, MU \--- не более как головоломка о натуральных числах, одетая в типографский костюм. Переведя ее в область теории чисел, мы смогли бы найти ее решение. Давайте поразмыслим над словами Мумона, сказавшего: «Если у кого-нибудь из вас \--- один глаз, тот заметит ошибку учителя». Но почему важно иметь именно один глаз?

Если вы попробуйте подсчитать количество \textbf{I} в теоремах, то вскоре заметите, что оно никогда не равняется \textbf{0} . Иными словами, кажется, что сколько бы мы не удлиняли и не сокращали, нам никогда не удается избавиться от всех \textbf{I} . Будем называть количество \textbf{I} в каждой строчке \emph{\textbf{величиной I}} данной строчки. Заметьте, что \emph{величина I} аксиомы \textbf{MI} \--- \textbf{1} . Можно доказать не только то, что \emph{величина I} не может равняться \textbf{0} , но и то, что \emph{величина I} не может делиться на \textbf{3} .

Для начала заметьте, что правила I и IV совершенно не затрагивают \emph{величины I} . Так что нам придется иметь дело только с правилами II и III\@. Правило III уменьшает величину \textbf{I} ровно на \textbf{3} . После приложения этого правила величина \textbf{I} результата могла бы делиться на \textbf{3} \--- но только в том случае, если бы величина I \emph{изначальной строчки} тоже делилась на \textbf{3} . Короче, правило III никогда не создает числа, делящегося на \textbf{3} , «из воздуха». Оно может сделать это лишь тогда, когда такое число уже имеется в начале. То же самое верно для правила II, которое удваивает \emph{величину I} . Это происходит потому, что, если \textbf{2} \emph{n} делится на \textbf{3} , то, поскольку \textbf{2} не делится на \textbf{3} , то на \textbf{3} должно делиться \emph{n} (простой факт теории чисел). Ни правило II, ни правило III не могут создать делящегося на \textbf{3} числа из ничего.

Но это же ключ к головоломке \textbf{MU} ! Мы знаем следующее:

(1) \emph{Величина I} начинается с \textbf{1} (\textbf{1} не делится на \textbf{3} );

(2) Два правила вообще не влияют на \emph{величину I} ;

(3) Два оставшихся правила влияют на \emph{величину I} , но таким образом, что они не могут создать делимое на \textbf{3} число, если таковое не дано в начале.

Отсюда следует типично «наследственное» заключение: \emph{величина I} никогда не может стать делимой на \textbf{3} . В частности, \textbf{0} \--- пример запрещенной \emph{величины I} . Таким образом, \textbf{MU} \emph{не является теоремой системы} \textbf{MIU} .

Обратите внимание, что даже в форме головоломки о \emph{величине I} , эта проблема всё ещё усложнена игрой удлиняющих и укорачивающих правил. Нашей целью было прийти к нулю; \emph{величина I} могла увеличиваться (правило II) или уменьшаться (правило III). До анализа ситуации мы могли считать, что применив эти два правила достаточное количество раз, когда-нибудь мы смогли бы получить \textbf{0} . Теперь, благодаря простому доказательству теории чисел, мы знаем, что это невозможно.


\subsection{Гёделева нумерация для системы MIU}

Не все проблемы подобного типа решаются так легко. Однако мы видели, что по крайней мере одна такая головоломка может быть введена в теорию чисел и решена там. Теперь мы увидим, что в теорию чисел возможно включить \emph{все} проблемы о \emph{любой} формальной системе. Это возможно сделать благодаря открытию Гёделем специального типа изоморфизма. Я проиллюстрирую его на примере системы MIU.

Рассмотрим для начала нотацию этой системы. Каждому ее символу будет соответствовать новый символ:

\textbf{M} \textless==\textgreater{} 3

\textbf{I} \textless==\textgreater{} 1

\textbf{U} \textless==\textgreater{} 0

Это соответствие \--- вполне произвольно; я выбрал его потому, что эти символы слегка похожи на те, которым они соответствуют. Каждый номер называется \emph{Гёделев номер} соответствующей буквы. Уверен, что вы можете легко догадаться, как будет выглядеть Гёделев номер строчки из нескольких букв:

\textbf{MU} \textless==\textgreater{} 30

\textbf{MIIU} \textless==\textgreater{} 3110

и~т.\,д.

Это нетрудно. Ясно, что такое соответствие между двумя нотациями является превращением, сохраняющим информацию; это всё равно, что одна и та же мелодия, исполненная на двух разных инструментах.

Теперь давайте посмотрим на типичную деривацию системы MIU, записанную одновременно в двух нотациях:

(1)~~~~~~~~~~~ \textbf{MI} -\/- аксиома~~~ -\/- 31

(2)~~~~~~~~~~ \textbf{МII} -\/- правило II~ -\/- 311

(3)~~~~~~~~ \textbf{MIIII} -\/- правило II~ -\/- 31111

(4)~~~~~~~~ \textbf{MUI} -\/- правило III -\/- 301

(5)~~~~~~ \textbf{MUIU} -\/- правило I~~ -\/- 3010

(6)~ \textbf{MUIUUIU} -\/- правило II~ -\/- 3010010

(7)~~~~~ \textbf{MUIIU} -\/-~ правило IV -\/- 30110

Левая колонка получается при помощи наших четырех формальных типографских правил. О правой колонке можно сказать, что она также получилась в результате применения подобных правил. Однако правая колонка \--- дуалистична. Сейчас я объясню, чти это означает.

Восприятие вещей одновременно с типографской и с арифметической точки зрения

~О пятой строчке («3010») можно сказать, что она была сделана из четвертой добавлением «0» справа; с другой стороны, мы можем так же легко представить себе, что она была получена в результате \emph{арифметической} операции \--- а именно, умножения на 10. Когда натуральные числа записаны в десятичной системе, умножение на 10 и добавление справа «0» неотличимы друг от друга. Мы можем воспользоваться этим и записать \emph{арифметическое} правило, соответствующее типографскому правилу I:

АРИФМЕТИЧЕСКОЕ ПРАВИЛО Iа: Число, десятичное продолжение которого оканчивается справа на «1», может быть умножено на 10.

Мы можем избавиться от упоминания символов в десятичном продолжении, арифметически описав правую цифру:

АРИФМЕТИЧЕСКОЕ ПРАВИЛО Ib: Если при делении некоего числа на 10 в остатке получается «1», то это число может быть умножено на 10.

Можно было бы воспользоваться и чисто типографским правилом, как, например, следующее:

ТИПОГРАФСКОЕ ПРАВИЛО I: Из любой теоремы, которая кончается на «1», можно получить новую теорему, добавляя «0» справа от этой~«1».

Все эти правила дают одинаковый эффект. Именно поэтому правая колонка дуалистична: ее можно рассматривать как серию типографских операций, превращающих одну схему символов в другую, или как серию арифметических операций, превращающих одну величину в другую. Существуют веские причины к тому, чтобы больше интересоваться арифметической версией. Переход из одной чисто типографской системы в другую, изоморфную типографскую систему \--- это не слишком занимательно; с другой стороны, переход из типографской области в изоморфную ей часть теории чисел предоставляет интересные, ранее неиспользованные возможности. Словно кто-то всю жизнь имел дело только с нотной записью, и вдруг ему показали соответствие между нотами и звуками. Какой удивительное богатство открылось перед ним! Или, возвращаясь к Ахиллу и Черепахе, играющим с цепочками, представьте себе человека, который хорошо знаком с фигурами из цепочек, и которому вдруг открылось соответствие между цепочками и рассказами. Какое откровение! Открытие Геделевой нумерации сравнивают с открытием Декарта, установившего изоморфизм между линиями на плоскости и уравнениями с двумя переменными. Это кажется невероятно просто \--- но это открывает дорогу в огромный новый мир.

Однако прежде чем придти к заключению, давайте рассмотрим подробнее этот высший уровень изоморфизма. Это очень хорошее упражнение. Наша цель \--- придумать арифметические правила, действующие точно так же, как типографские правила системы MIU.

Ниже приведено решение. В этих правилах \emph{m} и \emph{k} \--- произвольные натуральные числа, и \emph{n} \--- любое натуральное число, меньшее 10\emph{m} .

ПРАВИЛО 1: Если мы получили 10\emph{m} + 1, то мы можем получить 10 * (10\emph{m} + 1).

\emph{Пример} : Переход от строчки 4 к строчке 5. Здесь \emph{m} = 30

ПРАВИЛО 2: Если мы получили 3 * 10\textsuperscript{\emph{m}} + \emph{n} , то мы можем получить 10\textsuperscript{\emph{m}} * (3 * 10\textsuperscript{\emph{m}} + \emph{n} ) + \emph{n}

\emph{Пример} : Переход от строчки 1 к строчке 2, где \emph{n} и \emph{m} равняются 2.

ПРАВИЛО 3: Если мы получили \emph{k} *10\textsuperscript{\emph{m} +\textbf{3}} + 111 * 10\textsuperscript{\emph{m}} + \emph{n} , то мы можем получить \emph{k} * 10 \textsuperscript{\emph{m} +\textbf{1}} + \emph{n} .

\emph{Пример} : Переход от строчки 3 к строчке 4. Здесь \emph{m} и \emph{n} равняются 1 и \emph{k} равняется 3.

ПРАВИЛО 4: Если мы получили \emph{k} * 10\textsuperscript{\emph{m} +\textbf{2}} + \emph{n} , то мы можем получить \emph{k} * 10 \textsuperscript{\emph{m}} + \emph{n}.

\emph{Пример} : Переход от строчки 6 к строчке 7. Здесь \emph{m} =2, \emph{n} =10 и \emph{k} =301.

Не следует забывать нашу аксиому! Без нее мы как без рук, так что давайте запишем постулат.

Мы можем получить 31.

Теперь правую колонку можно рассматривать как арифметический процесс в новой арифметической системе, которую мы назовем \emph{системой 310} :

(1)~~~~~~~~ 31~~~~~~ аксиома

(2)~~~~~~~ 311~~~~~~ правило 2 (\emph{m} = 1, \emph{n} = 1)

(3)~~~~~ 31111~~~~~~ правило 2 (\emph{m} = 2, \emph{n} = 11)

(4)~~~~~~~ 301~~~~~~ правило 3 (\emph{m} = 1, \emph{n} = 1, \emph{k} = 3)

(5)~~~~~~ 3010~~~~~~ правило 1 (\emph{m} = 30)

(6)~~~ 3010010~~~~~~ правило 2 (\emph{m} = 3, \emph{n} = 10)

(7)~~~~~ 30110~~~~~~ правило 4 (\emph{m} = 2, \emph{n} = 10, \emph{k} = 301)

Обратите внимание на то, что удлиняющие и укорачивающие правила снова с нами и в системе 301; они просто переведены в область чисел таким образом, что Гёделевы номера в системе возрастают и уменьшаются. Если вы посмотрите внимательно на то, что происходит, то увидите, что правила основаны на простой идее, а именно: сдвиг цифр направо и налево в десятичной записи чисел имеет отношение к умножению на степени числа 10. Это простое наблюдение обобщено в следующем центральном предложении:

ЦЕНТРАЛЬНОЕ ПРЕДЛОЖЕНИЕ: Если у нас имеется некоторое правило, говорящее нам, как определенные цифры могут быть передвинуты, заменены, добавлены или опущены в в десятичной записи любого числа, то это правило также может быть представлено соответствующим арифметическим правилом при помощи арифметических операций со степенями числа 10, а также сложения, вычитания и так далее.

Или короче:

Типографские правила манипуляции с \emph{символами чисел} эквивалентны арифметическим правилам операций с \emph{числами} .

Это простое наблюдение находится в самом сердце Гёделева метода; оно будет иметь совершенно потрясающий эффект. Оно говорит нам, что если у нас есть Гёделева нумерация для любой формальной системы, мы можем тут же получить набор арифметических правил, дополняющих Гёделев изоморфизм. В результате оказывается возможным перевести изучение любой формальной системы \--- на самом деле, \emph{всех} формальных систем \--- в область теории чисел.


\subsection{Числа, выводимые в MIU}

Подобно тому, как набор типографских правил порождает набор теорем, в результате повторного применения арифметических правил получается соответствующее множество натуральных чисел. Эти \emph{выводимые числа} играют ту же роль в теории чисел, как теоремы \--- в любой формальной системе. Разумеется, набор выводимых чисел изменяется в зависимости от принятых правил. «Выводимые числа» выводимы только \emph{относительно данной системы} арифметических правил. Например, такие числа как \textbf{31} , \textbf{3010010} , \textbf{31111} и так далее могут быть названы выводимыми в системе \textbf{MIU} . Это неуклюжее название можно сократить до чисел \textbf{MIU} ; оно символизирует тот факт, что эти числа \--- результат перевода системы \textbf{MIU} в теорию чисел при помощи Гёделевой нумерации. Если бы мы захотели приложить Гёделеву нумерацию к системе \textbf{pr} и затем «арифметизировать» ее правила, мы могли бы называть полученные числа «числами \textbf{pr} » \--- и так далее.

Заметьте, что выводимые числа (в любой данной системе) определяются рекурсивным методом: нам даны числа, о которых мы знаем, что они выводимы, и набор правил, объясняющих, как получить другие выводимые числа. Таким образом, класс выводимых чисел постоянно расширяется, подобно списку чисел Фибоначчи или чисел Q\@. Множество выводимых чисел любой системы \--- это \emph{рекурсивно счетное множество} . А как насчет его дополнения \--- множества невыводимых чисел? Имеют ли они какую-либо общую арифметическую черту?

Подобные вопросы возникают тогда, когда изучение формальных систем переносится в область теории множеств. О каждой арифметизированной системе можно спросить: «Возможно охарактеризовать выводимые числа каким-либо простым способом?» «Возможно ли охарактеризовать невыводимые числа рекурсивно счетным способом?» Эти вопросы теории чисел весьма непросты, и, в зависимости от арифметизированной системы, могут оказаться для нас слишком трудными. Если и есть надежда найти на них ответ, то она лежит в методических логических рассуждениях, подобных тем, что обычно используются для изучения натуральных чисел. Суть этих рассуждений была изложена в предыдущей главе. По всей видимости, в \acs{tnt} нам удалось полностью представить все математические рассуждения в одной единственной компактной системе.


\subsection{\acs{tnt} помогает ответить на вопросы о выводимых числах}

Значит ли это, что одна-единственная формальная система \--- \acs{tnt} \--- предоставляет нам способ ответить на любой вопрос о любой формальной системе? Возможно. Возьмем например, такой вопрос:

Является ли \textbf{MU} теоремой системы MIU?

Найти ответ на этот вопрос означало бы определить, является ли \textbf{30} числом MIU\@. Поскольку это утверждение \--- высказывание теории чисел, мы должны надеяться, что при достаточном усилии нам удастся перевести высказывание «30 \--- число MIU» в нотацию \acs{tnt}, точно так же, как нам удалось перевести на язык \acs{tnt} другие высказывания теории чисел. Должен сразу предупредить читателя, что, хотя подобный перевод существует, он невероятно сложен. Если вы помните, в главе VIII я говорил, что даже такой простой арифметический предикат как «b \--- степень 10» весьма непросто перевести в \acs{tnt}; предикат же «30 \--- число MIU» перевести еще гораздо сложнее! Всё же этот, перевод можно найти, и число SSSSSSSSSSSSSSSSSSSSSSSSSSSSSS0 может быть подставлено в него вместо любого \emph{b} . Результатом явилась бы МОНструозная строчка \acs{tnt}, говорящая о головоломке MU\@. Сдается мне, что подходящим названием для этой строчки было бы МУМОН\@. С помощью МУМОНа и подобных строчек \acs{tnt} теперь способна говорить в закодированной форме о системе MIU.


\subsection{Дуалистическая природа МУМОНа}

Чтобы извлечь какую-либо пользу из этой странной трансформации нашего первоначального вопроса, нам необходимо ответить еще на один вопрос:

Является ли МУМОН теоремой \acs{tnt}?

До сих пор мы всего лишь заменили короткую строчку (MU) на другую (монструозный МУМОН) и простую формальную систему (\textbf{MIU} ) \--- на более сложную (\acs{tnt}). Хотя мы перефразировали, вопрос, маловероятно, что это приблизило нас к ответу. Действительно, в \acs{tnt} есть такая куча укорачивающих и удлиняющих правил, что перифраз вопроса, скорее всего, окажется гораздо труднее оригинала. Некоторые читатели, пожалуй, могли бы сказать, что анализировать \textbf{MU} пои помощи МУМОНа \--- значит нарочно смотреть на вещи по-дурацки. Однако МУМОНа можно рассматривать более, чем на одном уровне.

Интересно то, что в МУМОНе есть два различных пассивных значения. Во-первых, приведенное выше:

% TODO: block (?)
30 \--- число \textbf{MIU} .

Во-вторых, мы знаем, что это высказывание изоморфно следующему:

\textbf{MU} \--- теорема системы \textbf{MIU} .

Следовательно, мы имеем право утверждать, что последнее высказывание \--- второе пассивное значение МУМОНа. Это может показаться странным, поскольку МУМОН состоит всего лишь из плюсов, скобок и тому подобных символов \acs{tnt}\@. Как же он может выражать что-либо, кроме арифметических высказываний?

На самом деле, это возможно. Так же, как одна единственная музыкальная строчка может заключать в себе гармонию и мелодию, как слово~BACH может быть прочитано как имя и как мелодия, как одно и то же словосочетание может быть аккуратным описанием картины Эшера, структуры ДНК, произведения Баха или Диалога под тем же названием, МУМОН может быть понят, по крайней мере, двояко. Это происходит благодаря следующим фактам:

% TODO: block
Факт 1. Высказывания типа «MU \--- теорема» могут быть закодированы в теории чисел при помощи Гёделевой нумерации.

Факт 2. Высказывания теории чисел могут быть переведены в \acs{tnt}.

Можно сказать, что (согласно Факту 1) МУМОН \--- это закодированное сообщение, в котором (согласно Факту 2) символы кода \--- не более, чем символы \acs{tnt}.


\subsection{Коды и неявное значение}

Вы можете возразить, что закодированное сообщение, в отличие от незакодированного, само по себе ничего не выражает \--- чтобы его понять, необходимо знать код. Однако на самом деле незакодированных сообщений не существует Просто одни сообщения написаны на более знакомых кодах, а другие \--- на менее знакомых. Чтобы раскрыть значение сообщения, его необходимо «извлечь» из кода при помощи некоего механизма, или изоморфизма Иногда открыть метод дешифровки бывает трудно, но, как только этот метод раскрыт, сообщение становится прозрачным, как стекло. Когда код становится достаточно знакомым, он перестает выглядеть как таковой, и мы забываем о существовании декодирующего .механизма. Сообщение сливается со значением.

Здесь мы сталкиваемся со случаем такого полного отождествления сообщения со значением, что мы с трудом можем вообразить, что данные символы могут иметь какое-то иное значение. Мы настолько привыкли считать, что символы \acs{tnt} придают строчкам этой системы теоретико-числовое значение (и только теоретико-числовое), что нам бывает трудно представить, что некоторые строчки \acs{tnt} могут быть интерпретированы, как высказывания о системе MIU\@. Однако Гёделев изоморфизм заставляет нас признать этот второй уровень значения у некоторых строчек \acs{tnt}.

МУМОН, декодированный в более знакомом нам виде, сообщает, что

% TODO: block (?)
30 \--- число \textbf{МIU} .

Это высказывание теории чисел, полученное при интерпретации каждого знака обычным путем.

Открыв Гёделеву нумерацию и построенный на ее основе изоморфизм, мы в каком-то смысле расшифровали код, на котором высказывания о системе \textbf{MIU} записаны при помощи строчек \acs{tnt}\@. Гёделев изоморфизм \--- это новый обнаружитель информации, в том же смысле, как дешифровки старинных текстов были обнаружителями заложенной в этих текстах информации.

Декодированное этим новым и менее знакомым нам способом, МУМОН сообщает, что

% TODO: block
\textbf{MU} \--- теорема системы \textbf{MIU} .

Мораль этой истории мы уже слышали: любой узнанный нами изоморфизм автоматически порождает значение; следовательно, у МУМОНа есть по крайней мере два пассивных значения, а может быть, и больше!


\subsection{Бумеранг \--- Гёделева нумерация \acs{tnt}}

Разумеется, это еще не конец; мы только начали открывать возможности Гёделева изоморфизма. Естественным трюком было бы использовать возможность \acs{tnt} отображать другие формальные системы на себя саму, на манер того, как Черепаха повернула патефоны Краба против их самих, или как Бокал Г атаковал сам себя, разбившись. Чтобы это сделать, мы должны приложить Гёделеву нумерацию к самой \acs{tnt}, так же, как мы это сделали с системой \textbf{MIU} , и затем «арифметизировать» правила вывода. Это совсем нетрудно. Например, мы можем установить следующее соответствие:

~\emph{Символ~~~ Кодон~~ Мнемоническое обоснование}

0 .......~~ 666~~~~~~~Число Зверя для Таинственного Нуля

S .......~~ 123~~~~~~~последовательность: 1, 2, З\ldots{}

= .......~~ 111~~~~~~~зрительное сходство, в повернутом виде + ....... 112~~1+1=2

* .......~~~236~~~~~~~~2*3=6

( .......~~~362~~~~~~~~кончается на 2 \textbackslash{}

) .......~~~323~~~~~~~~кончается на 3~ \textbar~~ эти

\textless{} .......~~~212~~~~~~~кончается на 2~ \textbar~~ три пары

\textgreater{} .......~~ 213~~~~~~~кончается на 3~ \textbar~~ формируют

{[} .......~~ 312~~~~~~~ кончается на 2~ \textbar~~ схему

{]} .......~~ 313~~~~~~~ кончается на 3 /

а .......~~ 262~~~~~~ противоположно A (626)

' .......~~~ 163~~~~~~~163-простое число

\&\#923;~......~~~ 161~~~~~~~«\&\#923;»-«график» последовательности 1-6-1"

V~......~~~ 616~~~~~~~«V»-«график» последовательности 6-1-6

э ......~~~ 633~~~~~~~в некотором роде, из 6 следуют 3 и 3

\textasciitilde{} .......~~ 223~~~~~~~2+2 \emph{не} 3

E .......~~ 333~~~~~~~«E» выглядит как «3»

A .......~~ 626~~~~~~~противоположно «A»- также «график» 6-2-6

: .......~~ 636~~~~~~~ две точки, две шестерки

пунк ....~ 611~~~~~~ особенное число (именно потому, что в нём нет ничего особенного)

Каждый символ \acs{tnt} соотнесен с трехзначным числом, составленным из цифр 1, 2, 3 и 6 таким образом, чтобы его было легче запомнить. Каждое такое трехзначное число я буду называть \emph{Геделев кодоном} , или, для краткости, \emph{кодоном} . Заметьте, что для b, с, d или е кодонов не дано, поскольку мы используем здесь строгую версию \acs{tnt}\@. Для этого есть причина, которую вы узнаете в главе XVI\@. Последняя строчка, «пунктуация», будет объяснена в главе XIV.

Теперь мы можем представить любую строчку или правило \acs{tnt} в новом наряде. Вот, например, Аксиома 1 в двух нотациях, новая над старой:

~626, 262, 636, 223, 123, 262, 111, 666

. A~~~~~ a~~~~ ~:~~~~~ \textasciitilde~~~~~S~~~~~a~~~~ =~~~~~ 0

Обычная условность \--- использование пунктуации после каждых трех цифр \--- очень кстати совпала с нашими кодонами, облегчая их чтение.

Вот Правило Отделения в новой записи:

ПРАВИЛО: Если~\emph{x} и 212\emph{x} 633y213 являются теоремами, то \emph{у} \--- также теорема.

Наконец, вот целая деривация, взятая из предыдущей главы; она дана в строгой версии \acs{tnt} и записана в новой нотации:

626,262,636,626.262,163,636,362,262,112,123,262,163,323,111,123,362,262,112,262,163,323 аксиома 3

. A~~~ a~~~~ :~ ~~ A~~~ ~a~~~~ '~~ ~ :~~~~ (~~~ ~a~~~ +~~~~ S~~~ a~~~~ '~~~~ )~~~=~~~ S~~~ (~~~~ a~~~~ +~~~ a~~~~~ '~~~ )

626,262,163,636,362,123,666,112,123,262,163,323,111,123,362,123,666,112,262,163,323 спецификация

. A~~~ a~~~~ '~~~~ :~~~~ (~~~~ S~~ ~0~~~~ +~~~ S~~~ a~~~~ '~~~~~ )~~~~=~~~ S~~~~ (~~~~ S~~~ 0~~~~ +~~~ a~~~~ '~~~~ )

362,123,666,112,123,666,323,111,123,362,123,666,112,666,323~спецификация

. (~~~~ S~~~ 0~~~~ +~~~ S~~~ 0~~~~~ )~~~ =~~~ S~~~~ ( ~~~ S~~~ 0~~~ +~~~~ 0~~~ )

626,262,636,362,262,112,666,323.111.262~аксиома 2

.~ A~~~ а~~ ~:~~~~ (~~~~ а~~~ +~~~ 0~~~ ~)~~~~ =~~~ а

362,123,666,112,666,323,111,123,666~~спецификация

. (~~~~ S~~~~ 0~~~ +~~ ~0~~~~ )~~~~ =~~~ S~~~ 0

123,362,123,666.112,666,323,111,123,123,666~~добавить «123»

. ~S~~ (~~~~~ S~~~ 0~~~~ +~~ ~0~~~ )~~~~ =~~~~ S~~~ S~~~ 0

362,123,666,112,123,666,323,111,123,123,666~~транзитивность

.~ (~~~ S~~~ 0 ~~~ +~~~ S~~~~ 0~~~~ )~~~~= ~~~ S~~ ~S~~~ 0

Обратите внимание, что я изменил название правила «добавить S» на «добавить 123», поскольку данное правило узаконивает именно эту типографскую операцию.

Новая нотация кажется весьма странной. Вы теряете всякое ощущение значения; однако, если потренироваться, вы сможете читать строчки в этой нотации так же легко, как вы читали строчки \acs{tnt}\@. Вы сможете отличать правильно сформированные формулы от неправильных с первого взгляда. Естественно, поскольку это настолько наглядно, вы будете думать об этом, как о типографской операции \--- но в то же время выбор правильно сформированных \emph{формул} в этой нотации эквивалентен выбору определенного класса \emph{чисел} , у которых есть также арифметическое определение.

А как же насчет «арифметизации» всех правил вывода? Они всё ещё остаются типографскими. Но погодите минутку! Согласно Центральному Предложению, типографское правило \--- всё равно, что арифметическое правило. Ввод и перестановка цифр в числах десятичной записи \--- это \emph{арифметическая} операция, которая может быть осуществлена типографским путем. Подобно тому, как добавление «О» справа от числа эквивалентно умножению этого числа на 10, каждое правило представляет собой компактное описание длинного и сложного арифметического действия. Таким образом, нам не придется искать эквивалентных арифметических правил, поскольку все правила \emph{уже} арифметические!


\subsection{Числа \acs{tnt}: рекурсивно счетное множество чисел}

С такой точки зрения, приведенная выше деривация теоремы «362,123,666,112,123,666,323,111,123,123,666» представляет собой последовательность весьма сложных теоретико-численных трансформаций, каждая из которых действует на одно или более данных чисел. Результатом этих трансформаций является, как и ранее, \emph{выводимое число} , или, более точно, \emph{число \acs{tnt}} . Некоторые арифметические правила берут старое число \acs{tnt} и \emph{увеличивают} его определенным образом, чтобы получить новое число \acs{tnt}, некоторые \emph{уменьшают} старое число \acs{tnt}; другие правила берут два числа \acs{tnt}, воздействуют на них определенным образом и комбинируют результаты, получая новое число \acs{tnt} \--- и так далее, и тому подобное. Вместо того, чтобы начинать с одного известного числа \acs{tnt}, мы начинаем с \emph{пяти} \--- одно для каждой аксиомы (в строгой нотации). На самом деле, арифметизированная \acs{tnt} очень похожа на арифметизированную систему \textbf{MIU} \--- только в ней больше аксиом и правил, и запись точных арифметических эквивалентов была бы титаническим и совершенно «непросветляющим» трудом. Если вы внимательно следили за тем, как это было сделано для системы \textbf{MIU} , у вас должно быть сомнений в том, что здесь это делается совершенно аналогично.

Эта «гёделизация» \acs{tnt} порождает новый теоретико-числовой предикат:

\emph{а} \--- число \acs{tnt}.

Например, мы знаем из предыдущей деривации, что \textbf{362,123,""666,""112,""123,""666,""323,""111,""123,""123,666} \emph{является} числом \acs{tnt}, в то время как число \textbf{123,666,111,666} числом \acs{tnt} предположительно \emph{не} является.

Оказывается, что этот новый теоретико-численный предикат можно \emph{выразить} некоей строчкой \acs{tnt} с одной свободной переменной \--- скажем, \emph{а} . Мы могли бы поставить тильду впереди, и эта строчка выражала бы дополняющее понятие:

\emph{а} \--- не число \acs{tnt}.

Теперь давайте заменим все \emph{а} в этой второй строчке на символ числа \acs{tnt} для \textbf{123,666,111,666} \--- символ, содержащий ровно \num{123 666 111 666}~\textbf{S} и слишком длинный, чтобы его здесь записывать. У нас получится строчка \acs{tnt}, которая, подобно МУМОНу, может быть интерпретирована на двух уровнях. Во-первых, она будет означать

123,666,111,666 \--- не число \acs{tnt}.

Но, благодаря изоморфизму, связывающему числа, \acs{tnt} с теоремами \acs{tnt}, у этой строчки есть и второе значение:

S0=0 не теорема \acs{tnt}.


\subsection{\acs{tnt} пытается проглотить саму себя}

Это неожиданно двусмысленное толкование показывает, что \acs{tnt} содержит строчки, говорящие о других строчках \acs{tnt}\@. Иными словами, метаязык, на котором мы можем говорить о \acs{tnt}, берет начало, хотя бы частично, \emph{внутри} самой \acs{tnt}\@. И это не случайность; дело в том, что архитектура любой формальной системы может быть отражена в Ч (теории чисел). Это такая же неизбежная черта \acs{tnt}, как колебания, вызываемые в патефоне, проигрываемой на нём пластинкой. Кажется, что колебания должны вызываться внешними причинами, \--- например, прыжками детей или ударами мяча; но побочный \--- и неизбежный \--- эффект произведения звуков заключается в том, что они заставляют колебаться сам механизм, их порождающий. Это не случайность, а закономерный и неизбежный побочный эффект. Он свойствен самой природе патефонов. И так же самой природе любой формализации теории чисел свойственно то, что ее метаязык содержится в ней самой.

Мы можем почтить это наблюдение, назвав его Центральной Догмой Математической Логики и изобразив его на двухступенчатой диаграмме.

\acs{tnt} ==\textgreater{} Ч ==\textgreater{} мета-\acs{tnt}

Иными словами, у строчки \acs{tnt} есть интерпретация в Ч, а у высказывания Ч может быть второе значение \--- оно может быть понято как высказывание о \acs{tnt}.


\subsection{G: строчка, говорящая о себе самой на коде}

Эти интересные факты \--- только половина истории. Другая половина \--- интенсификация автореференции. Мы сейчас находимся в положении Черепахи, когда она обнаружила, что можно создать пластинку, разбивающую проигрывающий ее патефон. Вопрос только в том, какую именно запись надо ставить на данный патефон. Выяснить это непросто.

Для этого нужно найти строчку \acs{tnt} \--- мы будем называть ее «G» \--- которая говорит о себе самой, в том смысле, что \--- одно из ее пассивных значений \--- это высказывание о G.

В частности, этим пассивным значением окажется

«G- не теорема \acs{tnt}»

Я должен добавить, что у G есть и другое пассивное значение, являющееся \emph{высказыванием теории чисел} ; подобно тому, как МУМОН мог быть интерпретирован двояко. Важно то, что каждое пассивное значение \--- действительно и полезно, и никоим образом не бросает тень сомнения на второе значение. (Тот факт, что играющий патефон может вызывать колебания в самом себе и в пластинке, не отрицает того, что эти колебания \--- музыкальные звуки!)


\subsection{В неполноте \acs{tnt} виновато существование G}

Об изобретательном методе создания G и о некоторых важных понятиях \acs{tnt} мы поговорим в главах XIII и XIV; пока же давайте заглянем вперед и постараемся увидеть, какие последствия будет иметь нахождение автореферентной часта \acs{tnt}\@. Кто знает \--- может быть, это будет подобно взрыву! В некотором роде, это так и есть. Как вы думаете,

Является ли G теоремой \acs{tnt}, или нет?

Постарайтесь сформировать собственное мнение по этому поводу, не опираясь на мнение G о себе самой. В конце концов, G может понимать себя не лучше, чем понимает себя какой-нибудь мастер дзен-буддизма. Подобно МУМОНу, G может быть ложным утверждением. Подобно \textbf{MU} , G может быть не-теоремой. Мы не обязаны верить в любую возможную строчку \acs{tnt}, а только в ее теоремы. Давайте используем наше умение рассуждать логически и постараемся разъяснить этот вопрос.

Предположим, как обычно, что \acs{tnt} включает правильные методы рассуждения и что, следовательно, ложные утверждения не могут являться ее теоремами. Иными словами, любая теорема \acs{tnt} выражает истину. Таким образом, если бы строчка G была теоремой, она выражала бы истину, а именно: «G \--- не теорема». Вся сила ее автореферентности видна здесь в действии. Будучи теоремой, G должна быть ложна. Опираясь на наше предположение, что \acs{tnt} не имеет ложных теорем, мы должны теперь заключить, что G \--- \emph{не теорема} . Это не так страшно, но оставляет нас с меньшей проблемой. Зная, что G \--- не теорема, мы должны согласиться с тем, что она выражает истину\ldots{} В этой ситуации \acs{tnt} не оправдывает наших ожиданий \--- мы нашли строчку, выражающую истинное высказывание, которая в то же время не является теоремой! И, как бы мы не удивлялись, мы не должны упускать из виду тот факт, что у G есть также и арифметическая интерпретация. Это позволяет нам подвести итог нашим наблюдениям:

\emph{Найдена такая строчка \acs{tnt}, которая является недвусмысленным высказыванием о некоторых арифметических свойствах натуральных чисел; более того, рассуждая вне системы, мы можем определить не только то, что это высказывание истинно, но и то, что эта строчка не является теоремой \acs{tnt}\@. Таким образом, если мы спросим у \acs{tnt}, истинно ли это высказывание, она не сможет ответить ни да, ни нет.}

Аналогична ли G Черепашья цепочка в «Приношении MU»? Не совсем. Аналогичней с Черепашьей цепочкой будет \textasciitilde G\@. Почему это так? Давайте подумаем! Что говорит \textasciitilde G? Она должна утверждать обратное строчке~G\@. G~говорит: «G \--- не теорема \acs{tnt}»; следовательно, \textasciitilde G должно читаться «G \--- теорема \acs{tnt}». Мы можем перефразировать обе эти строчки следующим образом:

G: «Я не теорема (\acs{tnt})»

\textasciitilde G: «Мое отрицание \--- теорема (\acs{tnt})»

Именно \textasciitilde G параллельна Черепашьей цепочке, так как она говорит не о себе самой, но о той цепочке, что Черепаха дала Ахиллу сначала \--- цепочке, на которой была завязана дополнительная неточка (или на одну неточку меньше, чем надо \--- это зависит от точки зрения).


\subsection{Последнее слово \--- за Мумоном}

В своем коротком стихотворении о MU Джошу, Мумон проник в Мистерию Ультранеразрешимости глубже всех:

\settowidth{\dimen42}{Вы утратите собственную природу Будды.}
\begin{koan}[center, text width=\dimen42, fontupper=\normalsize]
    Есть ли у собаки природа Будды? \\
    Это самый серьезный вопрос из всех. \\
    Если вы ответите да или нет, \\
    Вы утратите собственную природу Будды.
\end{koan}

\endgroup
\end{document}
