\documentclass[../main.tex]{subfiles}
\begin{document}

\Chapter{Местонахождение значения}

\subsection{Когда одна и та же вещь не не похожа сама на себя?}

В ПОСЛЕДНЕЙ ГЛАВЕ, мы сформулировали вопрос. «Когда две вещи похожи друг на друга?» В этой главе мы рассмотрим оборотную сторону этого вопроса. «Когда одна и та же вещь не похожа сама на себя?» Мы попытаемся выяснить, присуще ли значение самому сообщению или же оно всегда порождается взаимодействием разума (или механизма) с этим сообщением \--- как в предыдущем Диалоге. В последнем случае нельзя было бы сказать ни что значение находится в каком-то одном месте, ни что сообщение имеет некое универсальное или объективное значение \--- поскольку каждый наблюдатель привносил бы в каждое сообщение свое собственное значение. Но в первом случае значение имело бы постоянное место и было бы универсально. В этой главе я постараюсь показать универсальность по крайней мере некоторых сообщений, не утверждая этого для всех сообщений вообще. Как мы увидим, идея «объективности значения» некоего сообщения интересным образом соотносится с тем, насколько легко может быть описан разум.


\subsection{Носители информации и обнаружители информации}

Начну с моего любимого примера отношения между пластинками, музыкой и проигрывателями. Мы привыкли к мысли о том, что пластинка содержит ту же информацию, что и музыкальное произведение, так как существуют проигрыватели, которые способны «читать» записи и превращать структуру звуковых дорожек в звуки. Иными словами, между звуковыми дорожками и звуками существует изоморфизм, проигрыватель \--- механизм, осуществляющий этот изоморфизм физически. Таким образом, естественно думать о пластинках как о \emph{носителях информации} и о проигрывателях, как об \emph{обнаружителях информации} . Другой пример этих понятий \--- система \textbf{pr}. Там «носителями информации» являлись теоремы, а её «обнаружителем» была интерпретация, такая прозрачная, что для извлечения информации из теорем нам не понадобились никакие электронные машины.

Эти два примера наводят на мысль, что изоморфизмы и декодирующие механизмы (то есть, обнаружители информации) всего лишь «проявляют» информацию, уже имеющуюся в структуре сообщения и только ждущую своего часа, чтобы быть извлеченной. Отсюда следует, что в любой структуре есть некая информация, которую \emph{возможно} извлечь, так же как и информация, которую извлечь \emph{нельзя} . Но что именно означает фраза «извлечь информацию»? С какой силой нам позволено её «вытягивать»? В некоторых случаях, приложив достаточно усилий, удается извлечь очень глубоко запрятанную информацию. На самом деле, извлечение информации может потребовать настолько сложных операций, что вам может показаться, что вы вкладываете больше информации, чем извлекаете.


\subsection{Генотип и фенотип}

Рассмотрим пример генетической информации, содержащейся в двойной спирали дезоксирибонуклеиновой кислоты (ДНК). Молекула ДНК \--- генотип \--- превращается в физический организм \--- фенотип \--- путем весьма сложного процесса, включающего выработку белков, воспроизведение ДНК, воспроизведение клеток, постепенное различение типов клеток, и~т.\,д. Процесс превращения генотипа в фенотип \--- эпигенез \--- представляет собой пример наиболее запутанной из запутанных рекурсий; мы уделим ему всё внимание в главе~XVI. Эпигенез зависит от множества сложнейших химических реакций и петель обратной связи. К тому времени, когда создание организма закончено, его физические характеристики не имеют ни малейшего сходства с его генотипом.

Тем не менее, считается, что физическая структура организма восходит к его ДНК \--- и только к ней. Впервые это подтвердили эксперименты Освальда Авери, проведенные в 1944 году; с тех пор собрано много убедительных данных в пользу этой идеи. Эксперименты Авери показали, что из множества молекул только ДНК обладает свойством передавать наследственные качества. Можно изменить другие молекулы в организме, например, белки, но эти изменения не будут переданы последующим поколениям. Однако когда меняется ДНК, изменения наследуются всеми последующими поколениями. Эти эксперименты доказали, что единственный способ изменить инструкции по построению нового организма заключается в изменении его ДНК; из этого, в свою очередь, следует, что эти инструкции должны быть закодированы где-то в структуре ДНК\@.


\subsection{Изоморфизмы экзотические и прозаические}

По-видимому, приходится заключить, что, структура ДНК содержит информацию о структуре фенотипа \--- иными словами, эти структуры \emph{изоморфны} . Это пример \emph{экзотического} изоморфизма; я имею в виду, что разделить фенотип и генотип на «части», которые могут быть отображены друг в друге \--- весьма нетривиальная задача. Напротив, \emph{прозаическим} изоморфизмом являлся бы такой, в котором части двух структур отображались бы друг в друге без труда. Пример тому \--- изоморфизм между пластинкой и музыкальным произведением \---~мы знаем, что для каждого звука в произведении существует его точное «изображение» в структуре звуковых дорожек и что, если потребуется, его можно всегда аккуратно указать. Другой пример прозаического изоморфизма \--- изоморфизм между Графиком G и любой из составляющих его бабочек.

Изоморфизм между структурой ДНК и структурой фенотипа никак нельзя назвать прозаическим \--- физически осуществляющий его механизм необыкновенно сложен. Например, было бы весьма трудно найти ту часть ДНК, которая в ответе за форму вашего носа или кончиков пальцев. Это немного похоже на попытку найти ту \emph{единственную} ноту, которая создает эмоциональный настрой музыкального произведения в целом. Конечно, такой ноты не существует, поскольку эмоциональное значение создается на гораздо высшем уровне \--- не единственной нотой, а большими «кусками» произведения. Кстати, эти «куски»~~не обязательно состоят из нот, идущих подряд \--- могут существовать также отдельные фрагменты, которые, взятые вместе, создают определенный эмоциональный настрой.

Подобно этому, «генетическое значение» \--- то есть, информация о структуре фенотипа \--- рассеяно по нескольким крохотным частям молекулы ДНК\@. Пока никто ещё не понимает этого «языка». (Внимание: понять этот «язык» \--- вовсе не то же самое, что разгадать Генетический Код, последнее произошло в начале шестидесятых годов. Генетический Код объясняет, как «перевести» небольшие порции ДНК в различные аминокислоты. Таким образом, разгадка Генетического Кода сравнима с нахождением фонетических значений букв иностранного алфавита \--- при этом мы ещё не знаем ни грамматики данного языка, ни значений его слов. Разгадка Генетического Кода явилась важнейшим шагом на пути к извлечению значения из ДНК, но это всего лишь первый шаг по длинной дороге, лежащей перед нами.)


\subsection{Проигрыватели-автоматы и пусковые механизмы}

Генетическая информация, содержащаяся в ДНК, \--- это один из лучших примеров неявного значения Чтобы превратить генотип в фенотип, требуются механизмы гораздо более сложные, чем сам генотип. Некоторые части генотипа служат пусковыми механизмами для этих процессов. Проигрыватель-автомат \--- обыкновенный, не крабий \--- хорошо поясняет эту идею: пара кнопок определяет серию действий, которые предстоит выполнить механизму. В этом смысле можно сказать, что кнопки «пустили в ход» песню, играемую на проигрывателе. В процессе, превращающем генотип в фенотип, клеточные «проигрыватели-автоматы» приводятся в действие с помощью «кнопок», каковыми являются короткие отрезки спирали ДНК и полученные таким образом «песни» часто служат кирпичиками для построения дальнейших «проигрывателей». Это можно сравнить с настоящими проигрывателями-автоматами, которые вместо лирических песенок проигрывали бы песни, объясняющие, как построить более сложные проигрыватели. Части ДНК запускают создание белков, эти белки пускают в ход сотни новых реакций, которые, в свою очередь, запускают операцию воспроизводства, которая в несколько этапов повторяет структуру ДНК \--- и так далее, и тому подобное. Это дает понятие о том, насколько рекурсивен этот процесс. Конечным результатом работы этого много раз запущенного пускового механизма является фенотип \--- индивид. Мы говорим, что фенотип \--- это раскрытие информации, содержавшейся в ДНК в скрытом состоянии (Термин «раскрытие» в этом контексте принадлежит Жаку Моноду, одному из лучших и оригинальнейших специалистов двадцатого века по молекулярной биологии). Никто не сказал бы, что песня, выходящая из динамиков музыкального автомата \--- это раскрытие информации, содержавшейся в паре нажатых нами кнопок, они послужили всего лишь \emph{триггером} для пуска в действие содержащих информацию механизмов самого автомата. С другой стороны, естественно говорить об извлечении музыки из звукозаписи как о «раскрытии» содержащейся в данной записи информации по нескольким причинам:

(1) музыка не запрятана в механизмах самого проигрывателя;

(2) возможно сопоставить части ввода (запись) с частями вывода (музыка) с любой степенью аккуратности;

(3) можно проигрывать на одном и том же проигрывателе разные записи и получать различные мелодии;

(4) запись и проигрыватель легко отделить друг от друга.

Совершенно другим вопросом является тот, присуще ли значение частям \emph{разбитой} пластинки. Края разбитой пластинки можно сложить вместе и таким образом восстановить значение \--- но вопрос здесь гораздо сложнее. Есть ли собственное значение у неразборчивого телефонного разговора?\ldots{} Спектр степеней собственных значений весьма широк. Интересно попытаться найти в этом спектре место для эпигенеза. Когда организм развивается, можем ли мы сказать, что информация извлекается из ДНК? Там ли находится вся информация о структуре организма?


\subsection{ДНК и необходимость химического контекста}

Благодаря экспериментам, подобным экспериментам Авери, в определенном смысле кажется, что ответ на этот вопрос положителен. Но в другом смысле кажется, что ответом будет «нет», поскольку процесс извлечения информации здесь в большой степени зависит от сложнейших клеточных химических процессов, которые не закодированы в самой ДНК\@. ДНК «надеется» на то, что они произойдут, но, по всей видимости, не содержит никакого кода, который вызывал бы эти процессы. Таким образом, у нас имеются два противоречивых взгляда на природу информации в генотипе. Один из них утверждает, что, поскольку такое большое количество информации содержится вне ДНК, мы должны рассматривать ДНК не более как очень сложный набор пусковых механизмов, что-то вроде кнопок на музыкальном автомате; другой взгляд \--- что вся \emph{информация} содержится в ДНК, только в очень неявной форме.

Можно подумать, что эти две точки зрения \--- лишь разные формы выражения одной и той же идеи; однако это вовсе не обязательно верно. Одна точка зрения утверждает, что ДНК почти бесполезна вне контекста; другая \--- что даже вне контекста структура молекулы ДНК живого существа имеет настолько \emph{убедительную внутреннюю логику}, что извлечь из нее информацию возможно в любом случае. Выражая ту же мысль короче, первый взгляд утверждает, что для выяснения значения ДНК необходим \emph{химический} контекст; другая точка зрения утверждает, что для раскрытия присущего ДНК значения необходим только разум.


\subsection{Фантастический НЛО}

Чтобы взглянуть на этот спорный вопрос в перспективе, вообразим себе странное гипотетическое событие. Запись фа-минорной сонаты Баха для скрипки и клавира в исполнении Давида Ойстраха и Льва Оборина отправлена в пространство в спутнике. Затем запись выброшена из спутника и направлена за пределы солнечной системы, а, может быть, и всей галактики \--- просто пластмассовый диск с дыркой в середине, крутящийся в межгалактическом пространстве. Безусловно, запись потеряла свой контекст. Каково теперь её значение?

Если бы иная цивилизация нашла эту пластинку, она была бы удивлена её формой и весьма заинтересована её назначением. Форма, действуя как пусковой механизм, сообщила бы им что, возможно, речь идет об искусственно сделанном предмете, и что этот предмет, может быть, несет определенную информацию. Эта мысль, сообщенная или «пущенная в действие» самой пластинкой, \emph{создает теперь новый контекст}, в котором пластинка будет рассматриваться в дальнейшем. Сама расшифровка может отнять гораздо больше времени \--- но нам об этом трудно судить. Можно представить себе, что если бы подобная запись попала на землю во времена Баха, никто не знал бы, что с ней делать, и, скорее всего, она так и осталась бы нерасшифрованной. Однако это не уменьшает нашей уверенности в том, что информация \emph{была там} изначально; просто мы знаем, что в то время человеческие знания о хранении, трансформации и извлечении информации были недостаточны.


\subsection{Уровни понимания сообщения}

В наши дни идея расшифровки распространена весьма широко; дешифровка составляет значительную часть работы астрономов, лингвистов, археологов, военных специалистов, и так далее. Существует предположение, что мы плаваем в море радиопосланий из других цивилизаций \--- посланий, которые мы пока ещё не умеем расшифровывать. Технике расшифровки подобных посланий было посвящено немало серьезных исследований. Одной из главных проблем \--- может быть, даже самой трудной \--- является следующая: «Как распознать шифрованное сообщение и поместить его в определенный контекст?» Посылка пластинки кажется простым решением; её физическая структура сразу привлекает внимание, и у нас есть разумная надежда на то, что достаточно развитый интеллект попытается найти спрятанную в ней информацию. Однако по технологическим причинам пока не представляется возможным посылать твердые объекты в другие солнечные системы. Это, разумеется, не мешает нам размышлять на эту тему.

Теперь представьте себе, что наша гипотетическая цивилизация догадалась, что для расшифровки записи нужен механизм, превращающий структуру звуковых дорожек в звуки. Это все ещё весьма далеко от настоящей расшифровки. Что же было бы удачной расшифровкой записи? Ясно, что для этого цивилизация должна найти смысл в звуках. Простое производство звуков было бы бесполезным, если бы оно не вызывало соответствующей реакции в мозгах (если можно так выразиться) у инопланетян. А что мы имеем в виду под «соответствующей реакцией»? Пуск в действие механизмов, вызывающих в их мозгах такой же эмоциональный настрой, какой возникает при прослушивании этой пьесы у нас. На самом деле, весь звуковоспроизводящий процесс можно было бы опустить, если бы инопланетянам удалось использовать пластинку как-то иначе, тем не менее получив при этом нужный эмоциональный эффект. (Если бы мы, земляне, умели бы последовательно активировать нужные механизмы в нашем мозгу так, как это делает музыка, возможно, что мы предпочли бы обходиться без звуков. Однако кажется маловероятным, что это может быть достигнуто без помощи слуха. Глухие композиторы \--- Бетховен, Дворжак, Форе \--- или музыканты, способные «слышать» музыку, глядя на ноты, не являются опровержением, так как их умение основано на долгом предварительном опыте прямого слушания музыки.)

Здесь все становится весьма расплывчато и неясно. Испытывают ли вообще инопланетяне какие-либо эмоции? Могут ли их эмоции \--- предполагая, что они у них есть \--- быть сравнимы с нашими? Если их эмоции схожи с нашими, группируются ли они, подобно нашим? Поймут ли они такие комбинации, как трагическая красота или мужественное страдание? Если окажется, что существа других миров разделяют с нами познавательные структуры до такой степени, что даже их эмоции совпадают с нашими, то, в некотором смысле, запись никогда не может оказаться полностью вне контекста \--- контекст оказывается частью схемы самой природы. Если дело действительно обстоит таким образом, то вполне возможно, что наша бродяга-пластинка, если не сломается по дороге, попадет в конце концов к какому-нибудь существу или группе существ и будет удачно расшифрована.


\subsection{Воображаемый космопейзаж}

Рассуждая о значении молекулы ДНК, я употребил выражение «убедительная внутренняя логика»; это кажется мне ключевым понятием. В качестве иллюстрации возьмем нашу гипотетическую посылку пластинки в пространство, на этот раз заменив Баха «Воображаемым пейзажем \#4» Джона Кейджа. Эта пьеса \--- классический пример «случайной» музыки, в которой вместо того, чтобы пытаться сообщить определенные эмоции, звукосочетания выбираются путем различных случайных процессов. В этом случае, двадцать четыре исполнителя держатся за двадцать четыре ручки двенадцати радио. Во время пьесы они крутят эти ручки кто во что горазд, так что настройка и громкость каждого радио все время меняются. Совокупность всех этих звуков и есть пьеса Кейджа. Композитор выразил свое намерение лаконично: «Позволим звукам быть самими собой, вместо того, чтобы заставлять их выражать придуманные человеком теории о его чувствах.»

Вообразите теперь, что это и есть пьеса, посланная в пространство на пластинке. Инопланетянам было бы весьма нелегко, если не невозможно, разгадать значение такого объекта. Скорее всего, они были бы удивлены противоречием между «рамкой» послания, говорящей: «Я \--- сообщение; расшифруйте меня», и хаосом его внутренней структуры. В этой пьесе Кейджа есть несколько кусочков, за которые можно ухватиться при расшифровке. С другой стороны, в пьесе Баха есть множество структур, структур структур и так далее. Мы не можем знать, являются ли эти структуры универсально привлекательными. Мы не знаем достаточно о природе разуме, эмоций или музыки, чтобы судить, настолько ли привлекательна внутренняя логика пьес Баха, что их значение способно пересечь галактики.

Однако вопрос здесь не в том, достаточно ли внутренней логики в пьесах Баха; вопрос в том, достаточно ли в любом отдельно взятом сообщении внутренней логики для того, чтобы его контекст был восстановлен автоматически при контакте с любой достаточно развитой цивилизацией. Если бы какое-либо сообщение обладало такой внутренней логикой, то разумно было бы сказать, что значение такого сообщения является его внутренним свойством.


\subsection{Героические расшифровыватели}

Еще один блестящий пример подобных идей \--- расшифровка старинных текстов, написанных на неизвестных языках и алфавитах. Интуиция говорит нам, что в подобных сообщениях есть смысл, независимо от того, удается ли нам этот смысл извлечь. Это чувство так же сильно, как и наша вера в то, что в газете, написанной по-китайски, есть внутренний смысл, даже если мы и не понимаем по-китайски ни слова. После того, как письменность или язык текста оказываются расшифрованными, никто не сомневается, что значение лежит в самом тексте, а не в методах расшифровки \--- так же как музыка «живет» в \emph{записи}, а не в проигрывателе. Именно так мы и определяем декодирующие механизмы, они не \emph{добавляют} никакого значения к знакам или предметам, которые служат им вводом, они лишь \emph{выявляют} значение, присущее этим знакам или предметам. Музыкальный автомат не является декодирующим механизмом, поскольку он не выявляет никакого значения вводных символов, напротив, он привносит значение, лежащее внутри него самого

Расшифровка старинного текста может потребовать многолетней работы коллективов ученых, пользующихся материалами множества библиотек всего мира\ldots{} Не добавляет ли и этот процесс определенную информацию? Насколько внутренним является значение самого текста, если его расшифровка требует таких гигантских усилий? Вкладывается ли при этом значение в текст, или оно уже в нем находилось? Моя интуиция говорит, что значение там уже было и что весь грандиозный труд по расшифровке не привнес в текст ничего нового. Это чувство основано на факте, что расшифровка была неизбежна, если не этой группой ученых, то другой, и если не теперь, так позже \--- и что результат был бы одним и тем же.

Значение содержится в самом тексте именно потому, что его воздействие на разум предсказуемо. В итоге мы можем утверждать, что значение является частью самого предмета постольку, поскольку этот предмет воздействует на разум определенным предсказуемым способом.

На рис. 39 показан камень Розетты, одно из важнейших исторических открытий. Он явился ключом к расшифровке египетских иероглифов, поскольку он содержит параллельный текст, написанный тремя древними письменностями: иероглифической, демотической и греческой. Надпись на базальтовой пластине была впервые расшифрована Жаном Франсуа Шамполионом, «отцом египтологии»; это декрет Мемфисского собрания священников в поддержку Птолемея V Эпифания.

% TODO: illustration 39
\emph{Рис. 39. Камень Розетты (С разрешения Британского музея )}


\subsection{Три уровня любого сообщения}

В этих примерах расшифровки помещенных вне контекста сообщений можно ясно различить три уровня информации: (1) \emph{сообщение-рамка} ; (2) \emph{внешнее сообщение} ; (3) \emph{внутреннее сообщение} . Мы лучше всего знакомы с (3) \--- внутренним сообщением. Оно передается явно, как эмоциональные ощущения в музыке, фенотип в генетике, описание династий и ритуалов древних цивилизаций в старинных надписях, и так далее.

\emph{Понять внутреннее сообщение означает извлечь значение, вложенное в сообщение его отправителем} .

Сообщение-рамка гласит: «Я \--- сообщение; расшифруйте меня, если сможете!». Эта информация содержится в структурном аспекте предмета \--- носителя сообщения.

\emph{Понять сообщение-рамку означает признать необходимость декодирующего механизма} .

Если мы видим сообщение-рамку, то наше внимание направляется на уровень (2) \--- внешнее сообщение. Это информация, явно переданная с помощью схем символов и общей структуры сообщения; она сообщает, как расшифровать внутреннее сообщение.

\emph{Понять внешнее сообщение означает построить \--- или знать, как построить \--- правильный декодирующий механизм для внутреннего сообщения}.

Сообщение внешнего уровня всегда неявно, поскольку отправитель послания не может гарантировать, что оно будет понято. Пытаться послать инструкции по расшифровке внешнего послания было бы напрасным усилием, так как они являлись бы частью внутреннего сообщения \--- а его можно понять только после того, как найден декодирующий механизм. Поэтому внешнее сообщение всегда \emph{представляет собой скорее набор триггеров}, чем какое-либо послание, поддающееся расшифровке.

Выделение этих трех «уровней» \--- только самое начало анализа того, как значение содержится в сообщениях. Сообщения могут иметь не один, а множество внешних и внутренних уровней. Взгляните, например, на то, насколько сложны и связаны между собой внутренний и внешний уровни сообщения на камне Розетты. Чтобы полностью расшифровать это послание и понять отправителя в самом глубоком смысле, нам пришлось бы восстановить всю семантическую структуру, лежащую в основе его создания. После этого мы могли бы вообще выбросить внутреннее сообщение, так как полное понимание всех тонкостей внешнего сообщения позволило бы нам это внутреннее сообщение восстановить.

Подробное обсуждение отношения между внутренним и внешним сообщениями имеется в книге Джорджа Стайнера «После Вавилона» (George Steiner, «After Babel»), хотя автор не использует этой терминологии. Тон этой книги хорошо передает следующая цитата:

Обычно мы используем сокращенную запись, за которой просвечивает богатство подсознательных ассоциаций, иногда нарочно затемненных, а иногда явных \--- ассоциаций, таких глубоких и сложных, что, взятые в сумме, они, возможно, передают все своеобразие нашего статуса как индивидуума. \footnote{George Sterner «After Babel» стр. 172 3}

Подобные мысли можно также найти в книге Леонарда Б. Мейера «Музыка, искусство, идеи» (Leonard В. Mayer, «Music, Art, Ideas»):

Манера, в которой мы слушаем композиции Элиотта Картера, весьма отличается от манеры, в которой мы слушаем работы Джона Кейджа. Таким же образом, роман Беккета должен читаться по-иному, чем роман Беллоу Картина, написанная Виллемом де Кунингом нуждается в другом восприятии, чем картина, написанная Энди Вархолем. \footnote{Leonard В. Meyer «Music The Arts and Ideas» стр. 87 8}

Может быть, произведения искусства пытаются, прежде всего, передать некий стиль. В таком случае, если бы мы могли полностью понять и прочувствовать, что именно представляет собой тот или иной \emph{стиль}, мы могли бы обойтись без произведений, написанных в данном стиле. «Стиль», «внешнее сообщение», «декодирующий механизм» \--- все это только разные способы выражения одной и той же идеи.


\subsection{Апериодические кристаллы Шредингера}

Что заставляет нас замечать сообщение-рамку в некоторых предметах и не видеть её в других? Почему инопланетянин, поймавший заблудшую пластинку, должен решить, что в ней спрятано какое-то послание? Чем отличается пластинка от метеорита? Ясно, что её геометрическая форма является первым ключом к тому, что здесь «что-то не то». Следующий ключ \--- то, что на микроскопическом уровне она состоит из очень длинной последовательности апериодических структур, расположенных по спирали. Если расправить эту спираль, то мы получили бы гигантский (около 600 метров) ряд, состоящий из миниатюрных символов. Это не так уж отличается от молекулы ДНК, символы которой, записанные алфавитом из четырех различных оснований, расположены в одномерной последовательности, которая затем скручена в спираль. Еще до того, как Авери установил связь между генами и ДНК, физик Эрвин Шредингер в своем труде «Что такое жизнь?» (Ervin Schroedinger, «What is life?») предсказал, основываясь на чисто теоретических соображениях, что генетическая информация должна содержаться в «апериодических кристаллах». В действительности, сами книги представляют собой апериодические кристаллы, содержащиеся внутри аккуратных геометрических форм. Эти примеры наводят на мысль, что апериодические кристаллы, «упакованные» внутри регулярной геометрической структуры, могут скрывать внутреннее сообщение. (Я не хочу сказать, что это является исчерпывающей характеристикой сообщения-рамки; однако многие типичные сообщения имеют именно такие рамки. На рис.~40 приведены хорошие примеры этого.)

% TODO: illustration 40
\emph{Рис. 40. Коллаж из различных письменностей. В верхнем левом углу \--- надпись на ещё нерасшифрованной бустрофедонской системе с острова Пасхи, в которой каждая вторая строчка перевернута. Знаки вырезаны на деревянной табличке размером 9x89 см. Двигаясь по часовой стрелке, мы находим вертикально записанный монгольский; над ним \--- современный монгольский, а под ним \--- документ, датирующийся 1314 годом. В правом нижнем углу мы находим поэму Рабиндраната Тагора, написанную по-бенгальски. Рядом с ней \--- газетный заголовок на майаламе (язык западной Кералы, провинции в южной Индии), над которым \--- элегантно изогнутая письменность тамильского (восточная Керала). Самый маленький фрагмент \--- отрывок сказания на бугинезском, языке островов Селибеса в Индонезии. В центре \--- абзац на тайском языке; над ним \--- манускрипт, написанный руническим письмом (четырнадцатый век), содержащий пример законов провинции Скании (южная Швеция). Наконец, налево вклинен фрагмент законов Хаммураби, написанный ассирийской клинописью. Как сторонний наблюдатель, я чувствую очарование тайны, думая о том, как передается значение в странных изгибах и углах этих прекрасных апериодических кристаллов. В самой форме здесь присутствует содержание. (Из книги Ханса Йенсена «Знак, символ и письменность» (Нью-Йорк, 1969), стр. 89 (клинопись), 356 (остров Пасхи), 386, 417 (монгольский), 552 (руническое письмо); из книги Кеннета Катцнера «Языки мира» (Нью-Йорк, 1975), стр. 190 (бенгальский), 237 (бугинезский); из книги И.А.~Ричардса и Кристины Гибсон «Английский в картинках» (Нью-Йорк, 1960), стр. 73 (тамильский), 82 (тайский).)}


\subsection{Языки для трех уровней}

Идею трех уровней сообщения хорошо поясняет пример бутылки, выброшенной на берег прибоем. С первым уровнем, рамкой, мы сталкиваемся, когда видим, что бутылка запечатана и внутри нее \--- сухой листок бумаги. Даже не видя, написано ли там что-нибудь, мы знаем, что этот предмет \--- носитель информации. Чтобы отбросить бутылку, не попытавшись её открыть, понадобилось бы потрясающее \--- почти нечеловеческое \--- отсутствие любопытства. Итак, мы открываем бутылку и исследуем значки на бумаге. Может быть, они написаны по-японски; это можно установить, узнав символы, но при этом не поняв ничего из внутреннего сообщения. Внешнее сообщение может быть передано русской фразой «Я \--- сообщение, написанное по-японски». Как только этот факт установлен, мы можем обратиться к внутреннему сообщению, которое может оказаться чем угодно: призывом к помощи, стихотворением хайку, жалобой влюбленного\ldots{}

Было бы бесполезно включать в перевод внутреннего сообщения фразу «Это сообщение написано по-японски», поскольку человек, это читающий, должен был бы знать японский. До того, как прочесть внутреннее сообщение, он знал бы, что, поскольку оно написано по-японски, он сможет его прочесть. Можно было бы вывернуться, предложив перевод фразы «Это сообщение написано по-японски» на несколько различных языков. Практически это помогло бы; но теоретически остается та же трудность. Человек, говорящий по-русски, должен сначала узнать «русскость» сообщения \--- иначе толку все равно мало. Следовательно, мы не можем избежать проблемы расшифровки внутреннего сообщения \emph{снаружи} ; само внутреннее сообщение может дать нам подсказки и подтверждения, но они не более, чем пусковые механизмы, действующие на человека, нашедшего бутылку (или на его помощников).

С подобными проблемами встречается слушатель коротковолнового радио. Прежде всего, он должен решить, являются ли звуки, которые он слышит, сообщением или просто шумом. Звуки сами по себе не дают ответа на этот вопрос, даже в том маловероятном случае, когда внутреннее сообщение оказывается на языке слушателя и состоит из фразы «Эти звуки \--- не шум, а сообщение!» Если слушатель узнает в звуках сообщение-рамку, он пытается установить, на каком языке идет передача \--- и ясно, что он находится все ещё извне; он принимает \emph{пусковые механизмы}, исходящие из радио, но они не могут дать ему явного ответа.

В самой природе внешних сообщений заложено то, что они не могут быть выражены на явном языке. Найти такой явный язык, на котором можно было бы передать внешнее сообщение, не было бы шагом вперед \--- это было бы противоречием в терминах! Понять внешнее сообщение всегда остается заботой слушателя. Если ему это удается, он проникает внутрь, в каковом случае отношение пусковых механизмов к явным значениям сдвигается в пользу последних. По сравнению с предыдущими этапами, понимание внутреннего сообщения весьма нетрудно; оно словно бы входит в нас само собой.


\subsection{Теория значения «музыкальный автомат»}

Эти примеры могут показаться подтверждением идеи, что у сообщений нет присущего им значения \--- ведь для того, чтобы понять сколь угодно простое внутреннее сообщение, необходимо сначала понять его рамку и его внешнее сообщение, представляющие из себя пусковые механизмы (такие, как японский алфавит или звуковые дорожки на пластинке). Начинает казаться, что от теории «музыкального автомата» нам никуда не деться. Эта теория гласит, что \emph{никакое сообщение не имеет присущего ему значения}, поскольку, чтобы понять какое-либо сообщение, его надо сначала ввести в «музыкальный автомат»; это значит, что информация, содержащаяся в этом автомате должна быть добавлена к сообщению \--- только тогда у него появится значение.

Этот довод весьма похож на ловушку, в которую Черепаха поймала Ахилла в Диалоге Льюиса Кэрролла. Там идея состояла в том, что, прежде чем использовать какое-то правило, необходимо иметь правило, говорящее нам, как использовать первое правило; иными словами, что существует бесконечная иерархия уровней правил, которая не позволяет исполниться ни одному из них. Здесь идея в том, что, прежде чем понять любое сообщение, нам необходимо сообщение, говорящее нам, как понять это сообщение; иными словами, что существует бесконечная иерархия уровней сообщений, которая не позволяет понять ни одного из них. Однако все мы знаем, что эти парадоксы недействительны, поскольку правила все-таки используются и сообщения понимаются. Как же это происходит?


\subsection{Против теории «музыкального автомата»}

Это происходит потому, что наш разум не бестелесен; он расположен в физических объектах \--- в наших мозгах. Их структура сформировалась в процессе долгой эволюции, и их действие подчиняются законам физики. Поскольку они являются физическими телами, \emph{наши мозги действуют, не нуждаясь в инструкциях к действию} . Именно на том уровне, где, повинуясь физическим законам, рождаются мысли, парадокс Кэрролла перестает действовать. Точно так же на том уровне, где мозг интерпретирует входящую информацию как сообщение, перестает действовать «парадокс сообщения». По-видимому, в нашем мозгу уже есть встроенная «аппаратура», позволяющая нам распознавать сообщения в некоторых объектах \--- и затем эти сообщения декодировать. Эта минимальная врожденная способность извлекать внутренние сообщения делает возможным в высшей степени рекурсивный, подобный снежному кому, процесс усвоения языков Эта врожденная аппаратура \--- что-то вроде музыкального автомата она дает недостающую информацию, превращающую простые пусковые механизмы в целые сообщения.


\subsection{Значение врожденно, если разум естественен}

Если бы «музыкальные автоматы» разных людей содержали бы разные «песни» и по-разному отвечали бы на одни и те же пусковые механизмы, нам не пришло бы в голову говорить о том, что этим механизмам присуще определенное значение. Однако человеческие мозги устроены так, что при равенстве остальных условий, один мозг отвечает на данный пусковой механизм почти так же, как и другой. Именно поэтому ребенок может выучить любой язык: все дети одинаково реагируют на «пусковой механизм» разных языков. Это единообразие «человеческого музыкального автомата» устанавливает общий «язык», на котором передаются рамки и внешние сообщения. Более того, если считать, что человеческий разум является лишь одним из примеров общего явления природы \--- появления разумных существ в самых разных ситуациях \--- то можно предположить, что «язык» на котором передаются рамки и внешние сообщения среди людей, является «диалектом» \emph{универсального} языка, на котором могут договориться между собой любые разумные существа. В таком случае, некоторые пусковые механизмы обладали бы \emph{универсальной пусковой мощью} в том смысле, что любое разумное существо отвечало бы на них примерно так же, как и мы.

Сказанное позволяет нам изменить наше описание того, где находится значение. Мы можем приписать все значения (рамку, внешнее и внутреннее) самому сообщению, поскольку сами декодирующие механизмы универсальны \--- иными словами, они представляют собой универсальные формы природы, возникающие в различных контекстах. Приведу конкретный пример: предположим, что кнопки «А-5» запустили одну и ту же песню на всех автоматах \--- и представьте также, что автоматы эти сделаны не человеком, а встречаются в природе повсеместно, как галактики или атомы углерода. В этой ситуации, пожалуй, было бы уместно назвать универсальную пусковую мощь кнопок «А-5» «присущим им значением»; кроме того, «А-5» заслуживали бы называться «сообщением» вместо «пускового механизма», и песня была бы «выявлением» внутреннего \--- хотя и неявного \--- значения этих кнопок.


\subsection{Земной шовинизм}

Таким образом, значение приписывается сообщению в том случае, когда это сообщение понимается одинаково представителями любой, в том числе инопланетной, цивилизации. В этом смысле оно напоминает массу, приписываемую предметам. В древности вес должен был казаться свойством, присущим самим предметам. Но, по мере того, как были лучше поняты законы тяготения, стало ясно, что вес предметов меняется в зависимости от различных гравитационных полей, действующих на данный предмет. Однако существует родственное свойство \--- масса; оно не варьируется в зависимости от гравитационного поля. Из этой неизменности вытекает заключение, что масса является свойством, присущим самим предметам. Если окажется, что масса тоже зависит от контекста, то нам придется пересмотреть нашу уверенность в том, что масса \--- свойство самих предметов. Таким же образом допустимо, что могут существовать другие типы «музыкальных автоматов» \--- разумных существ \--- которые общаются между собой при помощи сообщений, которые мы никогда бы не распознали как таковые; с другой стороны, эти существа также не могли бы распознать природу \emph{наших сообщений} . В таком случае, нам пришлось бы пересмотреть наше заключение о том, что наборам символов присущее определенное значение. С другой стороны, как бы мы вообще узнали о существовании подобных созданий?

Интересно сравнить эти рассуждения о неотъемлемости значения с аналогичными рассуждениями о неотъемлемости веса. Предположим, что мы определяем вес тела как «сила, с которой тело давит вниз, находясь на планете Земля». Согласно этому определению, для силы, с которой тело давит вниз, находясь на планете Марс, мы должны использовать иной термин. Это определение делает вес неотъемлемым свойством предметов, но происходит это за счет геоцентризма \--- «земного шовинизма». Это что-то вроде «гринвичского шовинизма» \--- отказа признавать местное время на всем земном шаре, за исключением гринвичского меридиана.

Возможно, что мы, сами того не сознавая, отягощены подобным шовинизмом в отношении разума, а следовательно и в отношении значения. Будучи такими шовинистами, мы назвали бы «разумными» существа, чей мозг достаточно похож на наш собственный, и отказались бы признавать разум за иными типами объектов. Вот немного преувеличенный пример: представьте себе метеорит, который, вместо того, чтобы пытаться расшифровать Баховскую запись, с абсолютным безразличием протыкает её и весело устремляется дальше по своей орбите. В нашем понимании, его контакт с пластинкой не затронул её значения. Поэтому нам может захотеться обозвать метеорит «тупицей». Но что если мы ошибаемся, и метеорит обладает неким «высшим разумом», который мы в своем земном шовинизме не в состоянии обнаружить? В таком случае его взаимодействие с пластинкой могло бы быть проявлением этого высшего разума. Возможно, что пластинка обладает неким «высшим значением», совершенно отличным от того, который приписываем ей мы; может быть, её значение зависит от типа разума, её интерпретирующего. Может быть\ldots{}

Было бы прекрасно, если бы могли определить разум как-нибудь иначе, чем «то, что интерпретирует символы таким же образом, как и мы». Ведь если это \--- единственное определение, которое мы можем дать разуму, то наше доказательство неотъемлемости значения было бы круговым, а следовательно, свободным от содержания. Мы должны попытаться определить множество характеристик, заслуживающих имя «разума», независимым способом. Эти характеристики представляли бы собой эссенцию разума, которую мы, люди, разделяем с другими разумными существами. На сегодня у нас ещё нет полного списка подобных характеристик. Однако весьма вероятно, что в ближайшие десятилетия в попытках определения человеческого разума будет сделан большой прогресс. В частности, не исключено, что специалисты по психологии познания, искусственному разуму и неврологии сумеют совместить их результаты и объяснить, что такое разум. Это определение может все равно оставаться человеко-шовинистическим \--- с этим ничего не поделаешь. Но чтобы это уравновесить, может существовать некий элегантный и красивый \--- и, возможно, даже простой \--- способ дать абстрактную характеристику того, что лежит в сердце разума. Это может уменьшить нашу неловкость от того, что мы сформулировали антропоцентрическое понятие. И, разумеется, если бы мы вступили в контакт с представителями цивилизации из другой звездной системы, мы уверились бы в том, что наш разум \--- не счастливая случайность, а пример естественного явления, которое возникает в природе в различных контекстах, так же как звезды и урановые ядра. В свою очередь, это подтвердило бы идею о неотъемлемости значения.

В заключение рассмотрим некоторые новые и старые примеры и обсудим степень неотъемлемости значения в каждом из них, представив на минуту, что мы находимся в положении инопланетянина, нашедшего странный объект\ldots{}


\subsection{Две пластинки в пространстве}

Представьте себе прямоугольную пластинку, сделанную из неразрушимого металлического сплава, на которой выгравированы две точки, одна над другой: такую же картинку представляет только что напечатанное двоеточие. Несмотря на то, что форма этого объекта наводит на мысль, что он искусственный и может содержать некую информацию, двух точек недостаточно, чтобы что-либо сообщить. (Можете ли вы, прежде чем читать далее, поразмышлять над тем, что они могут значить?) Представьте теперь, что мы изготовили вторую пластинку с большим количеством точек, а именно:

~~~~~~~~~~~~~~~~~~~ .

~~~~~~~~~~~~~~~~~~~ .

~~~~~~~~~~~~~~~~~~ ..

~~~~~~~~~~~~~~~~~ ...

~~~~~~~~~~~~~~~~ .....

~~~~~~~~~~~~~~ ........

~~~~~~~~~~~ .............

~~~~~~ .....................

..................................

Теперь естественнее всего \--- по крайней мере, для земного разума \--- было бы посчитать точки в каждом из рядов и записать получившуюся последовательность:

1, 1, 2, 3, 5, 8, 13, 21, 34.

Очевидно, что существует правило, управляющее количеством точек при переходе с одной линии на следующую. На самом деле, из этого списка мы можем с некоторой степенью уверенностью вывести рекурсивную часть определения чисел Фибоначчи. Предположим, что мы принимаем начальную пару значений (1, 1) за «генотип», из которого при помощи рекурсивного правила производим «фенотип» \--- весь ряд чисел Фибоначчи. Посылая лишь один генотип \--- первую версию пластинки \--- мы опускаем информацию, позволяющую реконструировать фенотип. Таким образом, генотип не содержит полного определения фенотипа. С другой стороны, если мы примем за генотип вторую версию пластинки, у нас будет гораздо больше шансов на то, что фенотип будет восстановлен. Эта новая версия генотипа \--- «длинный генотип» \--- содержит столько информации, что \emph{механизм, производящий фенотип из генотипа может быть выведен разумными существами из самого генотипа.}

Как только этот механизм для производства фенотипа из генотипа твердо установлен, мы можем вернуться к использованию «краткого генотипа» \--- первой версии пластинки. Например, краткий генотип (1, 3) произвел бы фенотип

1, 3, 4, 7, 11, 18, 29, 47,~\ldots{}

--- последовательность Лукаса. Для любого набора двух начальных значений \--- то есть, для любого краткого генотипа \--- существует соответствующий фенотип. Однако краткие генотипы, в отличие от длинных, действуют только как пусковые механизмы \--- кнопки на музыкальном автомате, в который встроено рекурсивное правило. Длинные генотипы содержат достаточное количество информации, чтобы разумное существо могло бы определить, какой именно «музыкальный автомат» надо сконструировать. В этом смысле, длинные генотипы содержат информацию о фенотипе, в то время как краткие \--- нет. Иными словами, длинные генотипы передают не только внутреннее сообщение, но и то внешнее сообщение, которое позволяет нам это внутреннее сообщение понять. Кажется, что ясность внешнего сообщения здесь зависит лишь от его длины. Это вовсе не является неожиданностью: то же самое верно и в случае дешифровки старинных текстов. Очевидно, что возможность успеха находится в прямой зависимости от количества имеющегося текста.


\subsection{Снова Бах против Кейджа}

Однако одного длинного текста может оказаться недостаточно. Давайте снова обратимся к разнице между посылкой в космос пластинки с музыкой Баха и пластинки с музыкой Кэйджа. Посмотрим, какое значение имеет для нас музыка Кэйджа. Его произведения должны рассматриваться в широком культурном контексте \--- как протест против определенных традиций. Таким образом, если мы хотим передать это значение, мы должны посылать не только ноты данной пьесы, но и всю историю западной культуры. Справедливо будет заключить, что, взятая сама по себе, музыка Кэйджа \emph{не имеет внутреннего значения} . Для слушателя, который достаточно искушен в западной и восточной культурах и, в особенности, в тенденциях западной музыки за последние десятилетия, она \emph{имеет} смысл \--- но такой слушатель будет подобен музыкальному автомату, а пьеса Кэйджа \--- паре кнопок на нем. Смысл прежде всего находится в голове у слушателя, и музыка служит лишь пусковым механизмом. И этот «музыкальный автомат», в отличие от чистого разума, вовсе не универсален; он связан с земной культурой и зависит от серии событий, происходивших на земном шаре в течение долгого времени. Надеяться на то, что музыка Кэйджа была бы понята инопланетянами, все равно что ожидать, что любимый вами мотивчик зазвучал бы из лунного музыкального автомата при нажатии тех же кнопок, что и на музыкальном автомате в кафе вашего родного городка.

С другой стороны, понимание музыки Баха нуждается в гораздо меньшем знании земной культуры. Это может звучать парадоксально, поскольку Бах сложен и организован, в то время как Кэйдж полностью лишен интеллектуальности. Дело в том, что разум любит организованность и избегает случайности. Для большинства слушателей случайная музыка Кэйджа требует подробных объяснений, даже после которых им все ещё может казаться, что они её не понимают. С~другой стороны, большинство Баховских композиций не нуждаются в словах. В~этом смысле в музыке Баха больше значения, чем в музыке Кэйджа. И~всё~же мы не можем в точности сказать, в какой степени в Бахе отражена человеческая культура.

Например, в музыке есть три основных структуры (мелодия, гармония и ритм), каждая из которых может быть в свою очередь подразделена на основной, промежуточный и мелкомасштабный аспекты. В каждом из этих измерений есть определенный уровень сложности, который наш мозг способен усвоить, прежде чем начать путаться; очевидно, что композитор, создавая свои произведения, принимает это в расчет \--- скорее всего, бессознательно. Эти уровни «терпимой сложности» в различных измерениях, возможно, зависят от специфических условий эволюции человеческого рода; другие разумные существа могли развить музыкальную культуру с совершенно иными уровнями терпимой сложности. Таким образом, вполне возможно, что пьеса Баха должна была бы сопровождаться значительным количеством информации о человеческом роде, которая не может быть выведена лишь из самой музыкальной структуры. Если сравнить музыку Баха с генотипом, а производимые ею эмоции \--- с фенотипом, то вопрос заключается в том, содержит ли генотип всю информацию, необходимую для восстановления фенотипа.


\subsection{Насколько универсально сообщение, содержащееся в ДНК?}

Основная проблема, с которой мы сталкиваемся, и которая весьма напоминает проблему двух пластинок, формулируется следующим образом: «Какое количество контекста, необходимого для понимания данного сообщения, может быть восстановлено на основе этого сообщения?» Теперь мы можем вернуться к первоначальному, биологическому значению терминов «генотип» и «фенотип» \--- ДНК и живой организм \--- и задать те же вопросы. Является ли ДНК универсальным пусковым механизмом? Или ему необходим «био-музыкальный автомат», чтобы раскрыть свое значение? Может ли ДНК вызвать фенотип, не используя соответствующего химического контекста? Ответ на этот вопрос \--- нет; но это «нет» \--- относительное. Разумеется, молекула ДНК в вакууме не создаст ничего. Однако если бы молекула ДНК была послана «искать счастья» в космос, как пластинки Баха и Кэйджа в нашем воображаемом примере, её могли бы найти разумная цивилизация. Прежде всего, они могли бы узнать её сообщение-рамку. После этого, они могли бы попытаться заключить, основываясь на химической структуре ДНК, какой тип химической среды является для нее подходящим, и обеспечить именно этот тип. Постепенно усложняющиеся попытки такого рода могли бы в конце концов привести к полному восстановлению химического контекста, необходимого для выявления фенотипного значения \@ Это звучит довольно неправдоподобно, но если дать на эксперименты много миллионов лет, то возможно, что значение ДНК в конце концов было бы восстановлено.

С другой стороны, если бы последовательность основ, составляющих цепь ДНК, была бы послана в космос в виде абстрактных символов (как на рис.~41) вместо длинной спиральной молекулы, шансов на то, что такое внешнее послание пустило бы в действие механизм декодирования, способный восстановить фенотип из генотипа, почти не было бы. Это пример того, как внутреннее послание может быть «завернуто» в настолько абстрактное внешнее послание, что возможности последнего к восстановлению контекста теряются. Практически этот набор символов здесь не имеет собственного смысла. Если вы считаете, что все это звучит безнадежно абстрактно и заумно, имейте в виду, что точный момент, когда фенотип может быть получен из генотипа, является сегодня предметом ожесточенных споров во многих странах, это вопрос о допустимости аборта.

% TODO: illustration 41
\emph{Рис. 41. Этот громадный апериодический кристалл \--- последовательность оснований хромосомы бактериофага фX174. Это первый геном живого организма, который удалось полностью отобразить. Чтобы показать основную последовательность лишь одной клетки кишечной бактерии, понадобилось бы около 2000 таких бустрофедонических страниц; для описания же человеческой клетки потребовалось бы около миллиона страниц. Книга, которую вы держите в руках, содержит приблизительно такое же количество информации, как и молекулярный отпечаток одной-единственной клетки кишечной бактерии.}

\end{document}
