\documentclass[../main.tex]{subfiles}
\begin{document}

\section{Трехголосная инвенция}

\centerblock{%
    \emph{Ахилл (греческий воин, самый быстроногий из смертных) и Черепаха стоят рядом на пыльной беговой дорожке; жара, палит солнце. Далеко в конце дорожки на высоком флагштоке висит большой прямоугольный ярко-красный флаг. В центре флага вырезана дыра в форме кольца, сквозь которую видно небо.}
}

\begin{Dialogue}

\speak{Ахилл} Что~это за странный флаг там, на другом конце дорожки? Он чем-то напоминает мне гравюру моего любимого художника, Эшера.

\speak{Черепаха} Это флаг Зенона.

\speak{Ахилл} Не кажется ли вам, что дыра в нем похожа на отверстия в листе Мёбиуса на одной из картин Эшера? Могу поспорить, что с этим флагом что-то не в порядке.

\speak{Черепаха} В нем вырезано кольцо в форме нуля \--- любимого числа Зенона.

\speak{Ахилл} Но ведь в то время нуль ещё не был изобретен! Он будет придуман неким индусским математиком только несколько тысяч лет спустя. Это доказывает, дорогая г-жа Ч, что подобный флаг невозможен.

\speak{Черепаха} Ваши доводы убедительны, Ахилл, и я должна согласиться, что такой флаг, действительно, не может существовать. Но все равно он замечательно красив, не правда ли?

\speak{Ахилл} В этом я не сомневаюсь.

\speak{Черепаха} Интересно, не связана ли его красота с его невозможностью? Не знаю, не знаю.. У меня никогда не доходили лапы до анализа Красоты. Это Сущность с Большой Буквы, а у меня никогда не хватало времени на Сущности с Большой Буквы.

\speak{Ахилл} Кстати, о Сущностях с Большой Буквы \--- вы никогда не задавались вопросом о Смысле Жизни?

\speak{Черепаха} Бог мой, конечно же, нет!

\speak{Ахилл} Не спрашивали ли вы себя, зачем мы здесь и кто нас изобрел?

\speak{Черепаха} Ну, это совершенно другое дело. Нас изобрел Зенон (в чем вы сами скоро убедитесь); мы находимся здесь, чтобы бежать наперегонки.

\speak{Ахилл} Мы \--- наперегонки?. Это возмутительно! Я, самый быстроногий из смертных \--- и вы медлительная, как\ldots{} как\ldots{} как Черепаха!

\speak{Черепаха} Вы могли бы дать мне фору.

\speak{Ахилл} Это была бы огромная фора.

\speak{Черепаха} Ну что же, я не возражаю.

\speak{Ахилл} Все равно я вас нагоню, раньше или позже \--- скорее всего, раньше.

\speak{Черепаха} А вот и нет, если верить парадоксу Зенона. Зенон надеялся с помощью нашего маленького соревнования доказать, что движение невозможно. По Зенону, движение происходит только в нашем воображении. Это значит, что Мир Изменяется Исключительно Иллюзорно. Он доказывает этот постулат весьма элегантно.

\speak{Ахилл} Ах, да, теперь я припоминаю~ знаменитый коан мастера дзен-буддизма Дзенона\ldots{} тьфу!. Зенона, я имею в виду. Действительно, очень просто.

\speak{Черепаха} Дзен коан? Дзен мастер? О чем вы говорите?

\speak{Ахилл} Вот, послушайте\ldots{} Два монаха спорили о флаге Один сказал; «Этот флаг движется». Другой возразил: «Нет, это ветер движется». В это время мимо проходил шестой патриарх, Зенон, который сказал монахам: «Не флаг и не ветер \--- движется ваша мысль!»

% TODO: illustration 10
\emph{Рис. 10. М.К.~Эшер. «Лист Мёбиуса I» (гравюра на дереве, отпечатанная с четырех блоков, 1961).}

\speak{Черепаха} Что-то вы все путаете, Ахилл. Зенон вовсе не мастер дзен-буддизма. На самом деле, он греческий философ из города Элей, лежащего на полпути между точками А и Б. Спустя столетия, его все ещё будут славить как автора парадоксов движения. В центре одного из них \--- наше соревнование по бегу.

\speak{Ахилл} Вы меня совсем сбили с толку. Я отчетливо помню, как много раз повторял наизусть имена шести патриархов дзена: «Шестой патриарх \--- Зенон, шестой патриарх \--- Зенон...» (Внезапно поднимается теплый ветер.) Взгляните, госпожа Черепаха, как развевается флаг! Как приятно смотреть на волны, бегущие по его мягкой ткани. И кольцо, вырезанное в нем, развевается вместе с флагом!

\speak{Черепаха} Не смешите меня. Этот флаг в принципе невозможен, следовательно, он не может развеваться. Это движется ветер.

\direct{\emph{(В этот момент мимо идет Зенон.)}}

\speak{Зенон} День добрый! Приветствую вас! Что слышно?

\speak{Ахилл} Флаг движется!

\speak{Черепаха} Ветер движется!

\speak{Зенон} О мои дражайшие друзья! Прекратите ваши словопрения! Оставьте ваши разногласия! Поберегите ваше красноречие! Я разрешу ваш спор, не сходя с места. Эгей, и в такой чудный денек!

\speak{Ахилл} Этот тип явно дурака валяет.

\speak{Черепаха} Нет, подождите, Ахилл, давайте-ка его послушаем. О неизвестный господин, будьте так любезны поделиться с нами вашими соображениями по этому поводу.

\speak{Зенон} С превеликим удовольствием. Не ветер и не флаг \--- на самом деле, вообще ничто не движется, что следует из моей великой Теоремы. Она гласит: «Мир Изменяется Исключительно Иллюзорно». А из этой Теоремы вытекает ещё более великая Теорема, Теорема Зенона: «Мир Ультранеподвижен».

\speak{Ахилл} Теорема Зенона? Вы, случаем, уж не Зенон ли из Элей будете?

\speak{Зенон} Он самый, Ахилл.

\speak{Ахилл (чешет голову в замешательстве)} Откуда он знает, как меня зовут?

\speak{Зенон} Возможно ли убедить вас выслушать меня, чтобы вы поняли, почему это так? Я прошел сегодня от точки А до самой Элей, только затем, чтобы найти кого-нибудь, кто согласился бы послушать мои тщательно отточенные доводы. Но все встречные сразу разбегались. Им, видите ли, было некогда. Вы не представляете себе, как это разочаровывает, когда встречаешь отказ за отказом\ldots{} Однако простите меня \--- я совсем замучил вас пересказом моих неприятностей. Я~прошу вас только об одном: не согласитесь ли вы ублажить старика-философа и уделить несколько минут \--- обещаю вам, всего лишь несколько минут \--- его экстравагантным теориям?

\speak{Ахилл} О, без сомнения! Сделайте милость, просветите нас! Я~знаю, что говорю за обоих, так как моя приятельница, госпожа Черепаха, только что отзывалась о вас весьма уважительно и упоминала как раз о ваших парадоксах.

\speak{Зенон} Благодарю вас. Видите ли, мой Мастер, пятый патриарх, учил меня, что реальность всегда одна и та же, единая и неизменная. Все разнообразие, изменение и движение \--- не более, чем иллюзии наших органов чувств. Некоторые смеялись над его взглядами, но я могу доказать всю абсурдность их насмешек. Мои доводы весьма просты. Я покажу их на примере двух персонажей моего собственного изобретения: Ахилл (греческий воин, самый быстроногий из смертных) и Черепаха. В моем рассказе, прохожий убеждает их бежать наперегонки к флагу, развевающемуся на ветру в конце беговой дорожки. Предположим, что Черепаха, как гораздо более медленный бегун, получит фору, скажем, в пятьдесят локтей. Соревнование начинается. В несколько прыжков Ахилл добегает до того места, откуда стартовала Черепаха.

\speak{Ахилл} Ха!

\speak{Зенон} Теперь Черепаха впереди него лишь на пять метров. Ахилл вмиг достигает того места.

\speak{Ахилл} Хо-хо!

\speak{Зенон} Все же за этот миг Черепаха успела немного продвинуться вперед. В мгновение ока Ахилл покрывает и эту дистанцию.

\speak{Ахилл} Хи-хи-хи!

\speak{Зенон} Но и в это кратчайшее мгновение Черепаха чуточку продвинулась, и опять Ахилл оказался позади. Теперь вы видите, что если Ахилл хочет нагнать Черепаху, ему придется играть в эти «догонялки» БЕСКОНЕЧНО \--- а следовательно, он НИКОГДА её не догонит!

\speak{Черепаха} Хе-хе-хе-хе!

\speak{Ахилл} Хм\ldots{} хм\ldots{} хм\ldots{} хм\ldots{} хм\ldots{} Этот довод кажется мне неверным. Однако я никак не могу понять, в чем здесь ошибка.

\speak{Зенон} Хороша головоломочка? Это мой любимый парадокс.

\speak{Черепаха} Прошу прощения, Зенон, но мне кажется, что вы рассказали нам что-то не то. Через несколько веков этот ваш рассказ будет известен как парадокс Зенона «Ахилл и Черепаха»; он показывает \--- гм! \--- что Ахилл никогда не догонит Черепаху. Доказательство же того, что Мир Изменяется Исключительно Иллюзорно (а следовательно, Мир Ультранеподвижен) содержится в вашем «Дихотомическом Парадоксе», не так ли?

\speak{Зенон} Ах, какой стыд. Конечно же, вы правы. Это тот парадокс, где объясняется, что идя от А до Б, надо сначала пройти половину пути \--- но от этой половины также придется сначала пройти половину\ldots{} и так далее. Оба эти парадокса очень похожи; честно говоря, я просто обыгрывал мою Великую Идею с разных сторон.

\speak{Ахилл} Могу поклясться, что эти аргументы содержат ошибку. Хотя я не вижу, где в них ошибка, зато прекрасно понимаю, что они не могут быть верными.

\speak{Зенон} Так вы сомневаетесь в правильности моих парадоксов? Отчего же вам самим не попробовать? Видите тот красный флаг в конце дорожки?

\speak{Ахилл} Невозможный, сделанный по гравюре Эшера?

\speak{Зенон} Тот самый. Как насчет того, чтобы вам с Черепахой пробежаться к флагу наперегонки? Конечно, ей надо будет дать приличную фору, скажем\ldots{}

\speak{Черепаха} Как насчет пятидесяти локтей?

\speak{Зенон} Отлично \--- пусть будут пятьдесят локтей.

\speak{Ахилл} Я-то всегда готов.

\speak{Зенон} Вот и чудесно. Все это захватывающе интересно! Сейчас мы проверим мою строго доказанную Теорему на опыте! Госпожа Черепаха, будьте так добры, займите позицию на пятьдесят локтей впереди Ахилла.

\direct{\emph{(Черепаха продвигается на пятьдесят локтей ближе к флагу.)}}

Ну как, вы оба готовы?

\speak{Черепаха и Ахилл} Готовы!

\speak{Зенон} На старт\ldots{} Внимание\ldots{} Марш!

\end{Dialogue}

\end{document}
