\documentclass[../main.tex]{subfiles}
\begin{document}

\section{Список иллюстраций}

Суперобложка. Триплеты «ГЭБ» и «ЭГБ», подвешенные в пространстве, отбрасывают символические тени на три плоскости, встречающиеся в углу комнаты. (Триплетом я называю блок, сделанный таким образом, что его тени, отброшенные под прямым углом, являются тремя разными буквами. Эта идея родилась у меня внезапно, когда как-то вечером я ломал голову над тем, как лучше символизировать единство Геделя, Эшера и Баха, слив их имена неожиданным образом. Два триплета, показанные на суперобложке, сделаны мной самим. Я выпилил их из красного дерева ручной пилой, используя для отверстий торцевую фрезу; стороны каждого триплета около 10 см длиной.

Перед «Благодарностью»: начало «Книги Бытия» на древнееврейском.~XXX

Часть I Триплет «GEB», отбрасывающий три тени под прямым углом.

1.~Элиас Готтлиб Гауссманн. «Портрет Иоганна Себастиана Баха».

2.~Адольф фон Мензель. «Концерт флейтистов в Сансуси».

3.~Королевская Тема.

4.~Акростих Баха «РИЧЕРКАР».

4а. Канон «Добрый король Венсеслас».

5.~М.К.~Эшер. «Водопад».

6.~М.К.~Эшер. «Подъем и спуск».

7.~М.К.~Эшер. «Рука с зеркальным шаром».

8.~М.К.~Эшер. «Метаморфоза II».

9.~Курт Гедель.

10.~М.К.~Эшер. «Лист Мёбиуса I».

11.~«Дерево» всех теорем системы MIU.

12.~М.К.~Эшер. «Воздушный замок».

13.~М.К.~Эшер. «Освобождение».

14.~М.К.~Эшер «Мозаика II».

15.~«РИСУНОК»

16.~М.К.~Эшер. «Деление пространства при помощи птиц».

17.~Скотт Е. Ким Рисунок «РИСУНОК-РИСУНОК».

18.~Диаграмма отношений между разными классами строчек ТТЧ.

19.~Последняя страница «Искусства фуги» И.С.~Баха.

20.~Наглядное объяснение принципа, лежащего в основе Теоремы Геделя.

21.~М.К.~Эшер. «Вавилонская башня»

22.~М.К.~Эшер. «Относительность».

23.~М.К.~Эшер. «Выпуклое и вогнутое».

24.~М. К Эшер. «Рептилии».

25.~Критский лабиринт.

26.~Структура Диалога «Маленький гармонический лабиринт».

27.~Схема~рекурсивных переходов для УКРАШЕННОГО СУЩЕСТВИТЕЛЬНОГО и СВЕРХУКРАШЕННОГО СУЩЕСТВИТЕЛЬНОГО.

28.~СРП для СВЕРХУКРАШЕННОГО СУЩЕСТВИТЕЛЬНОГО с одним рекурсивно расширенным узлом.

29.~Диаграмма G и Н, расширенная и нерасширенная.

30.~Диаграмма G, расширенная далее.

31.~СРП для чисел Фибоначчи.

32.~График функции INT (х).

33.~Скелеты INT и График G.

34. Рекурсивный График G.

35. Сложная диаграмма Фейнмана.

36.~М.К.~Эшер «Рыбы и чешуйки».

37.~М.К.~Эшер «Бабочки».

38.~Дерево игры в «крестики нолики».

39.~Камень Розетты.

40.~Коллаж из письменностей.

41.~Последовательность оснований хромосомы бактериофага 0X174.

42.~М.К.~Эшер «Крабий канон».

43.~Фрагмент одного из крабьих генов.

44.~«Крабий канон» из «Музыкального приношения» И С Баха.

45.~М.К.~Эшер «Мечеть».

46.~М.К.~Эшер «Три мира».

47.~М.К.~Эшер «Капля росы».

48.~М.К.~Эшер «Другой мир».

49.~М.К.~Эшер «День и ночь».

50.~М.К.~Эшер «Кожура».

51.~М.К.~Эшер «Лужа».

52.~М.К.~Эшер «Рябь на воде».

53.~М.К.~Эшер «Три сферы II».

Часть II Триплет «EGB» отбрасывающий три тени под прямым углом.

54.~М.К.~Эшер «Лист Мебиуса II».

55.~Пьер де Ферма.

56.~М.К.~Эшер «Куб с магическими лентами».

57.~Идея разделения на блоки.

58.~Ассемблеры компиляторы и уровни компьютерных языков.

59.~Разум строится уровень за уровнем.

60.~Картина «МУ».

61.~М.К.~Эшер «Муравьиная фуга».

62.~«Скрещение» двух знаменитых имен.

63.~Фотография муравьиного моста.

64.~«Спираль» ХОЛИЗМ РЕДУКЦИОНИЗМ.

65. Схематическое изображение нейрона.

66.~Человеческий мозг вид сбоку.

67.~Ответы разных типов нейронов на различные стимулы.

68.~Пересекающиеся нейронные пути.

69.~Строительство моста термитами рабочими.

70.~Небольшой фрагмент «семантической сети» автора.

71.~М.К.~Эшер «Порядок и хаос».

72.~Структура безвызовной программы Блупа.

73.~Георг Кантор.

74.~М.К.~Эшер «Сверху и снизу».

75.~«Разветвление» ТТЧ.

76.~М.К.~Эшер «Дракон».

77.~Рене Магритт «Тени».

78.~Рене Магритт «Грация».

79.~Вирус табачной мозаики.

80.~Рене Магритт «Прекрасный пленник».

81.~Самопоглощающие экраны телевизора.

82.~Рене Магритт «Воздух и песня».

83.~Эпименид приводящий в исполнение собственный смертный приговор.

84.~Айсберг парадокса Эпименида.

85.~Квайново предложение в виде куска мыла.

86.~Самовоспроизводящаяся песня.

87.~Типогенетический Код.

88.~Третичная структура типоэнзима.

89.~Таблица «прикрепительных вкусов» типоэнзимов.

90.~Центральная Догма типогенетики.

91.~Четыре основания, составляющих ДНК.

92.~Лестничная структура ДНК.

93.~Молекулярная модель двойной спирали ДНК.

94.~Генетический Код.

95.~Вторичная и третичная структуры миоглобина.

96.~Кусок мРНК, проходящий сквозь рибосому.

97.~Полирибосома.

98.~Двухтретичный молекулярный канон.

99.~Центральная схема.

100.~Код Гёделя.

101.~Бактериальный вирус Т4.

102.~Заражение бактерии вирусом.

103.~Морфогенетический путь вируса Т4.

104.~М.К.~Эшер. «Кастровалва».

105.~Шриниваса Рамануян и одна из его странных индийских мелодий.

106.~Изоморфизмы между натуральными числами, калькуляторами и человеческими мозгами.

107.~Нейронная и символическая деятельность мозга.

108.~«Выделение» высшего уровня мозга.

109.~Конфликт между высокими и низкими уровнями мозга.

110.~Начальная сцена Диалога с ШРДЛУ.

111. Еще один момент Диалога с ШРДЛУ.

112.~Последняя сцена Диалога с ШРДЛУ.

113.~Алан Матисон Тюринг.

114.~Доказательство «Ослиного мостика».

115.~Бесконечное дерево целей Зенона.

116.~Осмысленный рассказ на арабском языке.

117.~Рене Магритт. «Мысленная арифметика».

118.~Процедурное представление «красного куба, на котором стоит пирамида».

119.~Задача Бонгарда \#51.

120.~Задача Бонгарда \#47.

121.~Задача Бонгарда \#91.

122.~Задача Бонгарда \#49.

123.~Небольшой фрагмент сети понятий для задач Бонгарда.

124.~Задача Бонгарда \#33.

125.~Задачи Бонгарда \#85-87.

126.~Задача Бонгарда \#55.

127.~Задача Боигарда \#22.

128.~Задача Бонгарда \#58.

129.~Задача Бонгарда \#61.

130.~Задача Бонгарда \#70-71.

131.~Схематическая диаграмма Диалога «Крабий канон».

132.~Две гомологичные хромосомы, соединенные в центре центомерой.

133.~«Канон Ленивца» из «Музыкального приношения» И.С.~Баха.

134.~Авторский треугольник.

135.~М.К.~Эшер. «Рисующие руки».

136.~Абстрактная схема «Рисующих рук» Эшера.

137.~Рене Магритт. «Здравый смысл».

138.~Рене Магритт. «Две тайны».

139.~«Дымовой сигнал». Рисунок автора.

140.~«Сон о трубке». Рисунок автора.

141.~Рене Магритт. «Человеческое состояние I».

142.~М.К.~Эшер. «Картинная галерея».

143.~Абстрактная схема «Картинной галереи» Эшера.

144.~Сокращенный вариант предыдущей схемы.

145.~Еще более сокращенный вариант рис. 143.

146.~Еще один способ сократить рис. 143.

147.~Баховский «Естественно растущий канон», играемый в тональной системе Шепарда, образует Странную Петлю.

148.~Два полных цикла тональной гаммы Шепарда, в записи для фортепиано.

149.~М.К.~Эшер. «Вербум».

150.~Чарлз Баббадж.

151.~Крабья Тема.

152.~Последняя страница «Шестиголосного ричеркара» из оригинала «Музыкального приношения» И.С.~Баха.

\end{document}
