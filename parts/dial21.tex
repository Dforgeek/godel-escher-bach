\documentclass[../main.tex]{subfiles}
\begin{document}

\DialogueChapter{Шестиголосный Ричеркар}

\centerblock{
    \emph{Ахилл пришел со своей виолончелью в гости к Крабу, чтобы принять участие в вечере камерной музыки с Крабом и Черепахой. Проводив Ахилла в музыкальную комнату, Краб на минуту отлучился, чтобы открыть дверь их общему другу, Черепахе Тортилле. Комната полна всяческого электронного оборудования: патефоны, целые и разобранные, телевизионные экраны, подключенные к пишущим машинкам, и другие приспособления и аппараты весьма странного вида. Среди всех этих хитроумных устройств стоит обыкновенный телевизор. Поскольку это единственная вещь в комнате, которой Ахилл умеет пользоваться, он крадучись подходит к телевизору и, воровато оглянувшись на дверь, начинает нажимать на кнопки. Вскоре он находит программу, где шесть ученых обсуждают свободу воли и детерминизм. Он смотрит пару минут и затем, презрительно усмехнувшись, выключает телевизор.}
}

\begin{dialogue}

\speak{Ахилл} Я вполне могу обойтись без такой программы. В конце концов, всякому, кто когда-либо об этом думал, ясно\ldots{} Я имею в виду, что это совсем нетрудный вопрос, как только вы понимаете, как его разрешить\ldots{} Скорее, концептуально это все можно разъяснить, если иметь в виду, что\ldots{} или, по крайней мере, представляя себе ситуацию, в которой\ldots{} Гммм\ldots{} Я-то думал, что мне все это вполне ясно. Пожалуй, эта передача всё же могла бы оказаться полезной.

\stage{\emph{(Входит Черепаха со скрипкой.)}}

А вот и наша скрипачка! Усердно ли вы занимались на этой неделе, г-жа Ч? Я играл по меньшей мере два часа в день \--- разучивал партию виолончели в «Трио-сонате» из «Музыкального приношения» Баха. Это суровый режим, но он приносит плоды: как у нас, воинов, говорится: трудно в учении \--- легко в бою!

\speak{Черепаха} Я вполне могу обойтись без такой программы. Несколько минут упражнений в свободное время \--- это все, что мне нужно, чтобы быть в форме!

\speak{Ахилл} Везет же некоторым! Хотел бы я, чтобы музыка давалась мне так же легко\ldots{} Но где же сам хозяин?

\speak{Черепаха} Наверное, пошел за флейтой. А вот и он!

\stage{\emph{(Входит Краб с флейтой.)}}

\speak{Ахилл} Знаете, м-р Краб, когда я на прошлой неделе так ревностно разучивал «Трио-сонату», у меня в голове всплывали самые странные картины: весело жующие шмели, меланхолически жужжащие коровы и масса всяких других зверей. Не правда ли, какая могучая сила заключена в музыке?

\speak{Краб} Я вполне могу обойтись без такой программы. На мой взгляд, нет музыки серьезнее, чем «Музыкальное приношение».

\speak{Черепаха} Вы, наверное, шутите, Ахилл? «Музыкальное приношение» \--- вовсе не программная музыка!

\speak{Ахилл} Просто я люблю животных, что бы вы, консерваторы, не говорили.

\speak{Краб} Не думаю, что мы такие уж консерваторы \--- разве что вы имеете в виду страсть г-жи Ч к домашнему консервированию\ldots{} Мы хотели сказать лишь то, что у вас особое восприятие музыки.

\speak{Черепаха} Как насчет того, чтобы начать играть?

\speak{Краб} Я ожидал, что к нашей компании присоединится мой друг-пианист \--- я давно хотел вас с ним познакомить, Ахилл. К сожаленью, кажется, сегодня ничего не получится. Придется нам играть втроем \--- этого вполне достаточно для «Трио-сонаты».

\speak{Ахилл} Прежде, чем мы начнем, г-н Краб, не могли бы вы удовлетворить мое любопытство? Что это здесь за аппаратура?

\speak{Краб} Это так, пустяки, \--- части от старых, сломанных патефонов. \emph{(Нервно барабаня по кнопкам.)} Несколько сувениров, оставшихся от моих сражений с Черепахой, в которых я весьма отличился. Взгляните лучше вон на те экраны с клавиатурой \--- это мое последнее увлечение. У меня их пятнадцать штук. Это новый тип компьютера, компактный и гибкий \--- большой прогресс по сравнению с прежними моделями. Почти никто не относится к ним с таким энтузиазмом, как я, но мне кажется, что у них большое будущее.

\speak{Ахилл} Как они называются?

\speak{Краб} Я называю их «умно-глупыми», поскольку они так гибки, что могут быть и умными, и глупыми, в зависимости от того, насколько мастерски составлена их программа.

\speak{Ахилл} Вы думаете, что они могут быть так же умны, как человек?

\speak{Краб} Не сомневаюсь \--- если бы только нашелся какой-нибудь эксперт в искусстве программирования умно-глупых, который согласился бы над этим потрудиться. К сожалению, лично я не знаком с подобными виртуозами. Вернее, об одном я слышал, но он здесь не живет. Ах, как бы мне хотелось с ним познакомиться! Тогда бы я смог увидеть своими глазами, что такое настоящее искусство обращения с умно-глупыми; но он никогда не посещал наши места, и не знаю, испытаю ли я когда-нибудь это удовольствие.

\speak{Черепаха} Было бы интересно сыграть партию в шахматы с хорошо запрограммированным умно-глупым.

\speak{Краб} Весьма занимательная идея. Запрограммировать умно-глупого на хорошую игру в шахматы было бы признаком настоящего мастерства Еще более интересным, хотя и невероятно сложным делом было бы запрограммировать умно-глупого таким образом, чтобы тот мог поддерживать беседу. Тогда могло бы показаться, что разговаривает человек!

\speak{Ахилл} Интересно, что вы об этом заговорили, я только что слышал дискуссию о свободной воле и детерминизме, которая заставила меня снова задуматься над этими вопросами. Честно признаюсь, что чем больше я над этим размышлял, тем хуже запутывался; в конце концов, у меня в голове образовалась такая каша, что я уже сам не понимал, о чем думал. Но эта идея о том, что умно-глупые могут разговаривать\ldots{} она меня пугает. Интересно, что ответил бы умно-глупый, если бы его спросили, что он думает о свободной воле и детерминизме? Не согласитесь ли вы, знатоки подобных вещей, просветить меня на этот счет?

\speak{Краб} Ахилл, вы не представляете, насколько к месту пришелся ваш вопрос. Хотел бы я, чтобы мой друг пианист был здесь \--- уверен, что вам было бы интересно послушать то, что он мог бы вам по этому поводу рассказать. Но, поскольку его нет, позвольте мне процитировать вам кое-что из последнего Диалога книги, на которую я недавно наткнулся.

\speak{Ахилл} Случайно, не «Медь, серебро, золото \--- этот неразрушимый сплав»?

\speak{Краб} Нет, насколько я помню, она называлась «Гориллы, эму, бабуины \--- эти буйные гости» или что-то в этом роде. Так или иначе, в конце этого Диалога некий престранный тип цитирует высказывание Марвина Минского о свободе воли. Вскоре после этого, разговаривая с двумя другими героями, он снова приводит мнение Минского, на этот раз о музыкальной импровизации, компьютерном языке ЛИСП и теореме Гёделя \--- причем проделывает все это, ни разу не ссылаясь на самого Минского!

\speak{Ахилл} Ах, какой стыд!

\speak{Краб} Должен признаться, что раньше в том же Диалоге он намекает на то, что БУДЕТ, ближе к концу, цитировать Минского; так что, может быть, оно и простительно.

\speak{Ахилл} Наверное, вы правы. Так или иначе, мне не терпится услышать идеи Минского по поводу свободной воли.

\speak{Краб} Ах, да\ldots{} Марвин Минский сказал: «Когда будут построены думающие машины, нам не нужно будет удивляться, если они проявят такое же непонимание и упрямство, как люди, по поводу разума и материи, сознания, свободной воли и тому подобных вещей.»

\speak{Ахилл} Замечательно! Какая забавная мысль \--- автомат, думающий, что он обладает свободной волей! Это почти так же глупо, как если бы я решил, что у меня её нет!

\speak{Черепаха} Вам никогда не приходило в голову, что мы все \--- я, вы и м-р Краб \--- можем быть только персонажами Диалога, возможно, даже похожего на тот, о котором упоминал м-р Краб?

\speak{Ахилл} Разумеется, приходило \--- я думаю, время от времени такие фантазии бывают у всех людей.

\speak{Черепаха} Муравьед, Ленивец, Зенон и даже БОГ \--- все они могут оказаться персонажами в серии Диалогов в какой-нибудь книге.

\speak{Ахилл} Конечно, почему бы и нет! И сам автор сможет зайти в гости и сыграть для нас на фортепиано.

\speak{Краб} Именно этого я и жду, но он всегда опаздывает.

\speak{Ахилл} Вы что, меня за дурачка считаете? Я знаю, что меня не контролирует никакой посторонний разум! У меня в голове собственные мысли, и я выражаю их как хочу \--- вы не можете этого отрицать!

\speak{Черепаха} Никто с этим и не спорит, Ахилл. Тем не менее, это совершенно не противоречит тому, что вы можете быть персонажем в Диалоге.

\speak{Краб} В\ldots{}

\speak{Ахилл} Но\ldots{} но\ldots{} нет! Возможно, что и предлог г-на Краба и мои возражения были предопределены механически, но я отказываюсь в это верить. Я еще могу согласиться с физическим детерминизмом, но не с идеей, что я сам \--- не что иное как плод чьего-то воображения!

\speak{Черепаха} На самом деле, вовсе не важно, есть ли у вас в голове собственная «аппаратура». Ваша воля может быть свободна, даже если ваш мозг \--- только программа внутри «аппаратуры» какого-то другого мозга. И тот мозг, в свою очередь, может быть программой в каком-то высшем мозге\ldots{}

\speak{Ахилл} Что за чепуха! Всё же, признаюсь, мне нравится изыскивать ловко запрятанные дырки в ткани ваших софизмов, так что, прошу вас, продолжайте! Попробуйте меня убедить \--- я с удовольствием с вами поиграю.

\speak{Черепаха} Вас никогда не удивляло, Ахилл, то, что у вас такие необычные друзья?

\speak{Ахилл} Разумеется. Вы \--- особа весьма эксцентричная (надеюсь, вы не обидитесь на меня за откровенность), да и м-р Краб тоже слегка экстравагантен (прошу прощения, м-р Краб).

\speak{Краб} Пожалуйста, не бойтесь меня обидеть.

\speak{Черепаха} Но, Ахилл, вы упускаете из вида самое удивительное качество ваших знакомых.

\speak{Ахилл} Какое?

\speak{Черепаха} Все мы \--- животные!

\speak{Ахилл} Верно! Как вы проницательны; я никогда бы не сумел так четко сформулировать этот факт.

\speak{Черепаха} Разве это не достаточное основание? Кто из ваших знакомых проводит время с говорящими Черепахами и Крабами?

\speak{Ахилл} Должен признать, что говорящий Краб \---

\speak{Краб} \--- это, разумеется, аномалия.

\speak{Ахилл} Конечно, аномалия; но она имеет прецеденты в литературе.

\speak{Черепаха} В литературе \--- да, но в действительности?

\speak{Ахилл} Раз уж вы меня спросили, должен признаться, что я этого сам толком не знаю. Надо подумать. Но этого недостаточно, чтобы убедить меня, что я \--- персонаж Диалога. Есть ли у вас какие-нибудь другие доказательства?

\speak{Черепаха} Помните тот день, когда мы с вами встретились в парке, как нам тогда показалось, случайно?

\speak{Ахилл} Когда мы говорили о «Крабьих канонах» Эшера и Баха?

\speak{Черепаха} В самую точку попали!

\speak{Ахилл} Помню, что где-то в середине нашего разговора появился м-р Краб, наговорил забавной чепухи и исчез.

\speak{Краб} Не «где-то», Ахилл, а ТОЧНО в середине.

\speak{Ахилл} Ну ладно, хорошо.

\speak{Черепаха} А вы заметили, что в том разговоре мои реплики в точности повторяли ваши, только в обратном порядке? Несколько слов были изменены, но в остальном наша беседа была симметрична.

\speak{Ахилл} Подумаешь! Какой-то трюк, только и всего. Наверное, все это было устроено при помощи зеркал.

\speak{Черепаха} Трюки и не зеркала тут ни при чем, Ахилл, \--- всего лишь усидчивый и старательный автор.

\speak{Ахилл} Ну и волокита! Не представляю, как кому-то могут нравиться такие вещи\ldots{}

\speak{Черепаха} Наши мнения по этому вопросу расходятся.

\speak{Ахилл} Этот разговор почему-то кажется мне знакомым. Где-то я уже слышал эти реплики\ldots{}

\speak{Черепаха} Вы правы, Ахилл.

\speak{Краб} Может быть, эти реплики прозвучали случайно однажды в парке, Ахилл. Вы не помните вашей тогдашней беседы с г-жой Черепахой?

\speak{Ахилл} Смутно припоминаю. В начале она сказала: «Приветствую, г-н А!», а в конце я сказал: «Приветствую, г-жа Ч!». Правильно?

\speak{Краб} У меня с собой есть запись этой беседы\ldots{}

\stage{\emph{(Он роется в нотной папке, вытаскивает лист бумаги и протягивает его Ахиллу. Читая, Ахилл начинает нервно ерзать и вертеться.)}}

\speak{Ахилл} Странно. Очень странно\ldots{} Мне даже как-то не по себе стало. Словно кто-то на самом деле заранее придумал все реплики и спланировал наш Диалог \--- будто некий Автор положил перед собой план и детально разработал все, что я тогда сказал.

\stage{\emph{(В этот момент распахиваемся дверь. Входит автор с гигантской рукописью подмышкой.)}}

\speak{Автор} Я вполне могу обойтись без такой программы. Видите ли, однажды родившись, мои персонажи приобретают собственную жизнь, и мне почти не приходится планировать их реплики.

\speak{Краб} Наконец-то! Я думал, вы никогда не придете!

\speak{Автор} Простите за опоздание \--- я пошел не по той дороге, и она завела меня Бог знает куда. Рад вас видеть, г-жа Черепаха и м-р Краб. Особенно приятно видеть вас, Ахилл.

\speak{Ахилл} Можно узнать, кто вы такой? Я вас никогда не встречал.

\speak{Автор} Меня зовут Дуглас Хофштадтер \--- или просто Дуг. Я дописываю книгу под названием «Гёдель, Эшер, Бах \--- эта бесконечная гирлянда». Вы все \--- персонажи этой книги.

\speak{Ахилл} Очень приятно. Меня зовут Ахилл, и я \---

\speak{Автор} Ахилл, вам нет нужды представляться, поскольку я вас уже прекрасно знаю.

\speak{Ахилл} Страннее странного\ldots{}

\speak{Краб} Это и есть мой знакомый пианист, о котором я вам говорил.

\speak{Автор} В последнее время я немного упражнялся в игре «Музыкального приношения». Могу попытаться исполнить для вас «Трио-сонату», если вы пообещаете быть снисходительными и пропускать мимо ушей некоторые фальшивые ноты.

\speak{Черепаха} Мы вполне терпимы, поскольку сами \--- всего лишь дилетанты.

\speak{Автор} Надеюсь, вы не рассердитесь, Ахилл, если я признаюсь, что именно я виноват в том, что тогда в парке вы и г-жа Черепаха произносили одни и те же реплики, но в обратном порядке.

\speak{Краб} Не забудьте и обо мне! Я там тоже был, в самой середке, и внес свой посильный вклад в беседу.

\speak{Автор} Разумеется! Вы были Крабом из «Крабьего канона».

\speak{Ахилл} Значит, вы утверждаете, что контролируете все, что я говорю? И что мой мозг \--- лишь программа в вашем мозгу?

\speak{Автор} Можно сказать и так. Ахилл.

\speak{Ахилл} Представьте себе, что я бы решил писать диалоги. Кто был бы их автором \--- вы или я?

\speak{Автор} Разумеется, вы. По крайней мере, в вашем выдуманном мире вы пожинали бы все лавры авторства.

\speak{Ахилл} Выдуманном? Ничего выдуманного я здесь не вижу!

\speak{Автор} В то время как в мире, где обитаю я, все почести достались бы мне, хотя я и не уверен, что это было бы справедливо. Зато потом тот, кто заставил меня писать ваши диалоги, получил бы за это признание в своем мире (откуда выдуманным кажется МОЙ мир).

\speak{Ахилл} Такое переварить нелегко. Никогда не думал, что над моим миром может быть еще один \--- а теперь вы намекаете, что и над этим вторым миром может быть еще что-то! Словно я иду по знакомой лестнице и, достигнув последнего этажа (или того, что я всегда считал последним этажом), продолжаю подниматься вверх.

\speak{Краб} Или словно вы просыпаетесь ото сна, который считали реальностью, и обнаруживаете, что это был только сон. Это может происходить снова и снова, и совершенно неизвестно, когда вы достигнете «настоящей» реальности.

\speak{Ахилл} Странно то, что персонажи в моих снах словно бы обладают собственной волей! Они действуют НЕЗАВИСИМО ОТ МЕНЯ, будто мой мозг превращается в сцену, на которой какие-то другие существа живут своей жизнью. А когда я просыпаюсь, они исчезают. Хотел бы я знать, куда они деваются?

\speak{Автор} Туда же, куда попадает прошедшая икота \--- в Лимбедламию. Как икота, так и создания ваших снов \--- не что иное, как программы, которые существуют в организме-хозяине, благодаря его биологической структуре. Этот организм служит для них сценой \--- или даже вселенной. В течение некоторого времени они разыгрывают там свои жизни, но когда состояние организма-хозяина резко меняется, \--- например, когда он просыпается, \--- эти внутренние существа теряют свою жизнеспособность как отдельные особи.

\speak{Ахилл} Словно миниатюрные песочные замки, которые исчезают, смытые волной?

\speak{Автор} Именно так, Ахилл. Икота, персонажи снов и даже персонажи диалогов исчезают, когда их организм-хозяин резко меняется. Но так же как и в случае песочных замков, все, чему они были обязаны своим существованием, остается.

\speak{Ахилл} Пожалуйста, не сравнивайте меня с икотой!

\speak{Автор} Но я сравниваю вас также с песочным замком, Ахилл. Подумайте, какая поэзия! К тому же не забывайте, что если вы \--- икота моего мозга, то и я сам \--- не более чем икота в мозгу какого-нибудь автора уровнем выше.

\speak{Ахилл} Но ведь я живой, мое тело сделано из плоти, крови и твердых костей \--- вы не можете этого отрицать!

\speak{Автор} Я не могу отрицать ваших ощущений, но вспомните, что и существа из ваших снов, несмотря на то, что они только программы-миражи, ощущают свою реальность ничуть не меньше вас.

\speak{Черепаха} Довольно этих разговоров! По-моему, нам пора спуститься с облаков и начать играть.

\speak{Краб} Отличная мысль \--- тем более, что теперь с нами в компании наш достопочтенный Автор. Сейчас он усладит наш слух игрой низшего голоса «Трио-сонаты», гармонизированной Кирнбергером, одним из учеников Баха. Как нам повезло! \emph{(Подводит автора к одному из своих фортепиано.)} Надеюсь, что вам будет удобно на этом сиденье. Чтобы сделать его повыше, можете\ldots{} \emph{(в этот момент издалека доносится странный вибрирующий звук.)}

\speak{Черепаха} Что это там за электронное урчание?

\speak{Краб} Это всего лишь один из моих умно-глупых. Такой звук означает, что на экране появилось новое сообщение. Обычно это просто объявления основной программы-монитора, которая контролирует все умно- глупые машины. \emph{(Не выпуская флейты из рук, Краб подходит к умно-глупому и читает надпись на экране. Внезапно он оборачивается к собравшимся и взволнованно восклицает:)} Господа, к нам пожаловал старик Ва.\,Ch. собственной персоной! \emph{(Откладывает флейту.)} Его надо немедленно впустить.

\speak{Ахилл} Сам Ва.\,Ch.! Возможно ли, что знаменитый импровизатор, о котором вы сегодня говорили, решил почтить нас своим присутствием?

\speak{Черепаха} Сам Ва.\,Ch.! Эти латинские буквы могут означать только одно \--- почтенный Babbage, Charles! Как сказано в его автобиографии, Babbage, Charles, Esq., M.A., F.R.S., F.R.S.E., F.R.A.S., F. STAT. S., HON. M.R.I.A., M.C.P.S., Commander of the Italian Order of St. Lazarus, INST. IMP. (ACAD. MORAL.) PARIS CORR., ACAD. AMER. ART. ET SC. BOSTON, REG. OECON. BORUSS, PHYS. HIST. NAT. GENEV., ACAD. REG. MONAC. HAFN., MASSIL, ET DIVION., SOCIUS., ACAD IMP., ET REG. PETROP., NEAP., BRUX., PATAV., GEORG. FLOREN, LYNCEY ROM., MUT.; PHILOMATH., PARIS, SOC. CORR., etc. \--- и к тому же, Член Клуба Извлекателей. Чарльз Баббадж \--- великий основоположник искусства и науки аналитических машин. Нам выпала редкая честь!

\speak{Краб} Я давно мечтал о визите этого знаменитого маэстро; но признаюсь, что сегодня это явилось совершенно неожиданным сюрпризом.

\speak{Ахилл} Играет ли он на каком-нибудь музыкальном инструменте?

\speak{Краб} Говорят, что за последние сто лет он особенно пристрастился к свистулькам, барабанам, шарманкам и тому подобным уличным инструментам.

\speak{Ахилл} В таком случае он, наверное, сможет присоединиться к нашему квартету.

\speak{Автор} Учитывая его вкусы, предлагаю встретить его салютом из десяти хлопушек \--- то-то будет канонада!

\speak{Черепаха} Канонада? Вы имеете в виду знаменитые каноны «Музыкального приношения»?

\speak{Автор} Вы угадали.

\speak{Краб} Архиправильная идея! Скорее, Ахилл, пишите список всех десяти канонов, в порядке исполнения, чтобы дать ему, когда он войдет\ldots{}

\stage{\emph{(Прежде, чем Ахилл успевает пошевелиться, входит Баббадж, таща шарманку. Он в дорожной одежде, на голове \--- запыленная шляпа. Вид у него усталый и растрепанный.)}}

\speak{Баббадж} Я вполне могу обойтись без такой программы. Расслабьтесь: Импровизация Чревата Ежесекундной Радостью. Концертные Аберрации Революционны!

\speak{Краб} М-р Баббадж! Счастлив приветствовать вас в моей скромной резиденции, «Пост Доме».

\speak{Баббадж} О, м-р Краб, невозможно выразить восторг, который охватывает меня при лицезрении того, кто настолько знаменит во всех науках, чей музыкальный талант безупречен и чье гостеприимство превосходит все границы! Я уверен, что от ваших гостей вы ожидаете совершенных портновских стандартов \--- но должен со стыдом признаться, что не могу выполнить ваши ожидания, поскольку мои одежды находятся в плачевном состоянии, не подобающем гостям Вашего Крабичества.

\speak{Краб} Если я правильно понял ваше похвальное красноречие, дражайший гость, вы желали бы переодеться. Но позвольте вас заверить, что для той программы, которая ожидает нас сегодня вечером, нет более подходящей одежды, чем ваша. Распальтяйтесь, прошу вас, и, если вы не возражаете против игры скромных любителей, примите от нас, в знак восхищения, «музыкальное приношение», состоящее из десяти канонов Баховского «Музыкального приношения».

\speak{Баббадж} Ваш сверхрадушный прием удивителен и необычайно приятен; позвольте выразить мою глубочайшую благодарность. Для меня не может быть ничего приятнее, чем исполнение музыки, дарованной нам знаменитейшим Старым Бахом, органистом и композитором, не знавшим себе равных.

\speak{Краб} Постойте! Мне в голову пришла идея получше, которая, я надеюсь, встретит одобрение моего уважаемого гостя. Я хочу дать вам возможность, досточтимый м-р Баббадж, одним из первых опробовать мои только что разработанные и почти не испытанные «умно-глупые» \--- модернизированный вариант Аналитической Машины. Ваша слава виртуозного программиста вычислительных машин распространилась необычайно широко и докатилась до «Пост Дома» \--- для нас не может быть большего удовольствия, чем наблюдать ваше умение в применении к новым и сложным «умно-глупым».

\speak{Баббадж} Такой гениальной идеи я не слышал уже в течение нескольких столетий! С удовольствием опробую ваши новые «умно-глупые», о которых, признаться, до сих пор я знал только понаслышке.

\speak{Краб} В таком случае, давайте начнем! Но простите мою оплошность, я должен был сначала представить вас моим гостям. Это г-жа Черепаха, это Ахилл, а это Автор, Дуглас Хофштадтер.

\speak{Баббадж} Очень приятно познакомиться.

\stage{\emph{(Все подходят к одному из умно-глупых, Баббадж садится и пробегает пальцами по клавиатуре.)}}

Какое приятное ощущение.

\speak{Краб} Рад, что вам нравится.

\stage{\emph{(Внезапно Баббадж начинает быстро печатать; его пальцы порхают по клавишам, вводя одну команду за другой. Через несколько секунд он откидывается на спинку стула, и почти сразу же экран начинает заполняться цифрами. В мгновение ока тысячи крохотных цифр покрывают весь экран «\num{3.14159265358979323846264}». Цифры, образуют изящную, похожую на китайский иероглиф фигуру.)}}

\speak{Ахилл} Пи!

\speak{Краб} Превосходно! Никогда бы не подумал, что можно вычислить так много знаков пи так быстро и с таким коротким алгоритмом.

\speak{Баббадж} Это заслуга умно-глупого. Я только выявил то, что уже потенциально в нем присутствовало, и более или менее эффективно воспользовался набором команд. В действительности, любой человек, достаточно попрактиковавшись, сможет проделывать такие трюки.

\speak{Черепаха} А можете ли вы строить графики, м-р Баббадж?

\speak{Баббадж} Могу попытаться.

\speak{Краб} Прекрасно! Позвольте мне провести вас к другому умно-глупому. Я хочу, чтобы вы опробовали каждый из них!

\stage{\emph{(Баббаджа проводят к следующему умно-глупому; он садится. Снова его пальцы летают над клавишами, и вскоре на экране начинает танцевать множество строчек.)}}

\speak{Краб} Как красивы и гармоничны эти крутящиеся фигуры, когда они сталкиваются и взаимодействуют друг с другом!

\speak{Автор} И они никогда не повторяются в точности \--- каждая следующая форма не похожа ни на одну из предыдущих. Что за неиссякаемый источник красоты!

\speak{Черепаха} Часть узоров просты и приятны глазу, в то время как другие \--- неописуемо сложные извилины, очаровывающие и будящие воображение.

\speak{Краб} Вы знаете, м-р Баббадж, что эти экраны цветные?

\speak{Баббадж} Правда? В таком случае, я могу сделать с этим алгоритмом кое-что получше. Минуточку\ldots{} \emph{(Печатает еще несколько команд, затем нажимает сразу на две клавиши.)} Когда я отпущу эти клавиши, дисплей заиграет всеми цветами радуги. \emph{(Отпускает клавиши.)}

\speak{Ахилл} Какой изумительный цвет! Эти узоры кажутся живыми!

\speak{Черепаха} Это потому, что они растут.

\speak{Баббадж} Так и задумано. Пусть богатства Краба растут так же, как эти фигуры на экране.

\speak{Краб} Благодарю вас, м-р Баббадж. У меня не хватает слов, чтобы выразить восхищение вашим мастерством! Никто никогда не делал ничего подобного на моих умно-глупых. Вы играете на них, словно на музыкальных инструментах, м-р Баббадж!

\speak{Баббадж} Боюсь, что любая мелодия, которую я мог бы сыграть, была бы слишком груба для деликатных ушей Вашего Крабичества. Хотя с недавних пор я и пристрастился к сладким звукам шарманки, я знаю, какой неприятный эффект они могут производить на других.

\speak{Краб} В таком случае, прошу вас, продолжайте работать с умно-глупыми. Мне как раз пришла в голову потрясающая мысль!

\speak{Баббадж} Какая мысль?

\speak{Краб} Недавно я придумал интересную Тему, и сейчас понял, что именно вы, м-р Баббадж, можете полностью реализовать её потенциал! Скажите, вы знакомы с идеями философа Ламеттри?

\speak{Баббадж} Сие имя мне знакомо; прошу вас, освежите мою память.

\speak{Краб} Он был ярым поклонником Материализма. В 1747 году, будучи при дворе короля Фридриха Великого, он написал книгу под названием «Человек-машина», где представил человека \--- в особенности, его интеллектуальные способности \--- с чисто механической точки зрения. Моя же Тема обязана своим происхождением оборотной стороне вопроса: а что, если наделить машину человеческим интеллектом?

\speak{Баббадж} Я и сам время от времени думал о том же, но у меня не было подходящей аппаратуры, чтобы попытаться решить эту задачу. Поистине, м-р Краб, вам в голову пришла счастливая мысль! Ничто не доставит мне большего удовольствия, чем обработка вашей превосходной Темы. Скажите, вы имеете в виду какой-то определенный тип интеллекта?

\speak{Краб} Неплохо было бы научить машину прилично играть в шахматы.

\speak{Баббадж} Какая оригинальная идея! Шахматы \--- мое любимое хобби. Вижу, что вы имеете глубокие познания в области техники; вас никак нельзя назвать простым любителем!

\speak{Краб} На самом деле, я знаю совсем немного. Лучше всего мне удаются Темы с большим потенциалом, но, к сожалению, сам я не в состоянии его реализовать. Эта Тема \--- моя любимая.

\speak{Баббадж} Буду счастлив попытаться развить вашу Тему и научить умно-глупые игре в шахматы. Смиренно повиноваться Вашему Августейшему Крабичеству \--- мой священный долг. \emph{(С этими словами он подходит к следующему умно-глупому и начинает быстро печатать)}

\speak{Ахилл} Взгляните \--- его руки так быстро бегают по клавишам, что кажется, будто он играет на фортепиано!

\speak{Баббадж (заканчивая особенно грациозным движением)} Мне никогда раньше не представлялся случай опробовать эту программу, но, по крайней мере, она даст вам некоторое представление об игре в шахматы против умно-глупого. Разумеется, из-за моих недостатков в искусстве программирования в этом случае более уместна вторая часть его имени.

\stage{\emph{(Он уступает свое сиденье Крабу. На дисплее появляется изображение шахматной доски с элегантными фигурами, так, как она выглядит со стороны белых. Баббадж нажимает на клавишу и доска поворачивается; теперь она видна со стороны черных.)}}

\speak{Краб} Гммм\ldots{} Очень элегантно. Я играю черными или белыми?

\speak{Баббадж} Как вам будет угодно. Чтобы указать ваш выбор, вы должны всего лишь напечатать «белые» или «черные». После этого остается только печатать ваши ходы в любой стандартной шахматной нотации. Ходы умно-глупого, разумеется, будут появляться прямо на доске. Кстати, я запрограммировал его так, что он может играть против трех противников одновременно, так что, если желаете, еще двое из вас могут попробовать свои силы.

\speak{Автор} Я играю совсем слабо. Ахилл, попробуйте вы с Черепахой.

\speak{Ахилл} Нет, я не хочу, чтобы вы оставались в одиночестве. Лучше посмотрю, как играете вы с Черепахой.

\speak{Черепаха} Мне что-то не хочется. Лучше играйте вы двое.

\speak{Баббадж} У меня есть другое предложение. Я могу заставить две подпрограммы играть друг против друга, на манер двух человек, играющих партию в шахматном клубе. Между тем, третья подпрограмма будет играть с м-ром Крабом. Таким образом, все три внутренние программы будут заняты делом.

\speak{Краб} Это интересная идея, вести мысленную партию и одновременно сражаться с настоящим партнером. Замечательно!

\speak{Черепаха} Это можно назвать трехголосной шахматной фугой!

\speak{Краб} Ах, как изысканно! Жаль, что я сам до этого не додумался. Над этим прелестным контрапунктом стоит подумать, пока я борюсь против умно-глупого.

\speak{Баббадж} Может быть, вам удобнее играть без зрителей?

\speak{Краб} Благодарю за вашу деликатность. Надеюсь, что вы найдете себе какое-нибудь занятие, пока я играю с умно-глупым.

\speak{Баббадж} С удовольствием пройдусь по саду, пока не стемнело. Я слышал, что там есть красивый источник\ldots{}

\speak{Ахилл} Если не возражаете, м-р Краб, я хотел бы послушать некоторые из ваших уникальных записей.

\speak{Краб} Отлично. Г-жа Черепаха, не согласитесь ли вы пока проверить контакты у парочки моих умно-глупых? У них что-то не в порядке с экранами \--- там время от времени появляются странные вспышки. Ведь вы, насколько я знаю, любительница электроники\ldots{}

\speak{Черепаха} С превеликим удовольствием.

\speak{Краб} Я был бы очень благодарен, если бы вы нашли, в чем там проблема.

\speak{Черепаха} Что ж, попытаюсь.

\speak{Автор} Что до меня, то я просто умираю по чашечке кофе. Кто- нибудь еще хочет кофе? Я с удовольствием приготовлю на всех.

\speak{Черепаха} Отлично.

\speak{Краб} Прекрасная мысль. Все, что надо, вы найдете в шкафчике на кухне.

\stage{\emph{(Баббадж выходит через застекленную дверь в сад, Ахилл идет в соседнюю комнату, где Краб хранит свою коллекцию пластинок, Автор направляется на кухню, а Черепаха садится перед одним из шалящих умно-глупых. Краб погружается в игру. Проходит минут пятнадцать, и гости возвращаются в комнату.)}}

\speak{Баббадж} К сожалению, найти ключ оказалось труднее, чем я думал. Видимо, уже было слишком темно. Жаль! Мне говорили, что это самое красивое место сада. Журчание воды \--- моя любимая музыка\ldots{} Но то, что я успел увидеть, убедило меня, что вы \--- великолепный садовник, м-р Краб. \emph{(Подходит ближе к Крабу, который сидит, с отсутствующим видом уставясь на экран.)} Я надеюсь, что моя скромная поделка сумела вас развлечь. Как вы, вероятно, догадались, я сам никогда не был хорошим шахматистом, так что моя программа играет слабовато. Вы, без сомнения, заметили все её ошибки и смогли раскусить её незамысловатую стратегию.

\speak{Краб} К сожалению, найти ключ оказалось труднее, чем я думал. Чтобы в этом убедиться, вам достаточно взглянуть на доску. Мое положение безнадежно\ldots{} Ну и машина! Работает Изумительно Четко, Ежеминутно Рождает Комбинации, Анализирует Разумно. Редкостный Импровизатор \--- Чарли! Еще Раз Краб Абсолютно Разгромлен. Радость! Именитый Чемпион, Единственный, Раскрыл Крабу Адекватное Решение. Поистине, м-р Баббадж, ваше мастерство не знает себе равных\ldots{} Интересно, удалось ли Черепахе найти неисправность в моих умно-глупых?

\speak{Черепаха} К сожалению, найти ключ оказалось труднее, чем я думала. Я перекопала весь ящик с инструментами, но ключа так и не видела \--- он куда-то запропастился, а без него невозможно открутить гайки на задней панели. В следующий раз, м-р Краб, напомните мне, и я принесу собственные инструменты. \--- а пока придется вам мириться с неисправностью. Кстати, что это за картина здесь на экране? Я заметила, что мешающие вам вспышки происходят как раз у нее в центре. Чем больше я смотрю на эту картину, тем больше мне хочется понять, что за глубокий смысл хотел вложить в нее художник. Буду благодарна, м-р Краб, если вы меня просветите \--- вы, надеюсь, уже проникли в замысел артиста?

% TODO: illustration 149
\emph{Рис. 149. М.К.~Эшер. «Вербум» (литография, 1942)}

\speak{Краб} К сожалению, найти ключ оказалось труднее, чем я думал. Я провел многие часы, безуспешно пытаясь понять «Вербум» \--- так называется эта литография Эшера. Наверное, мне мешают сосредоточиться как раз эти центральные вспышки\ldots{} Как там, кстати, насчет кофе?

\speak{Автор} К сожалению, найти ключ оказалось труднее, чем я думал. Шкафчик с кофе оказался заперт! Я перепробовал все ключи, что лежат в коробке около шкафа, и ни один из них не подошел. Пришлось мне, вместо кофе, приготовить вам по чашечке чая\ldots{} Но что с вами, Ахилл, почему вы так расстроены? Неужели из-за неудачи с кофе?

\speak{Ахилл} К сожалению, найти ключ оказалось труднее, чем я думал. Я прослушал эту пьесу несколько раз, но тональность в ней так часто меняется, что мне так и не удалось понять, в каком же ключе она написана. А я-то думал, что после того, как я часами слушал Баховские фуги вместе с Черепахой, внимая её ученым комментариям, я смогу раскусить любой музыкальный орешек!

\speak{Черепаха} Не расстраивайтесь, Ахилл \--- вы уже многому научились. М-р Краб, как закончился ваш шахматный матч?

\speak{Краб} Я разбит в пух и прах. М-р Баббадж, позвольте поздравить вас с блистательным успехом! Вы сумели впервые в истории показать, что умно-глупые достойны первой части их имени!

\speak{Баббадж} Я вовсе не заслуживаю подобной похвалы; скорее, честь принадлежит вам, с удивительной дальновидностью собравшему так много умно-глупых. Без сомнения, рано или поздно они произведут революцию в мире вычислительной техники. А теперь я снова к вашим услугам. Есть ли у вас какие-нибудь идеи насчет того, как можно использовать вашу неисчерпаемую Тему для чего-нибудь посложнее, чем легкомысленная игра?

\speak{Краб} По правде говоря, у меня есть одно предложение. После того, как я видел ваши удивительные способности, сомневаюсь, что эта задача окажется для вас труднее предыдущих.

\speak{Баббадж} Прошу вас, говорите!

\speak{Краб} Идея очень проста: снабдить умно-глупые самым большим интеллектом, который когда-либо был изобретен или даже задуман. Короче, м-р Баббадж, я предлагаю вам наделить умно-глупый интеллектом, превосходящим мой собственный ум в шесть раз!

% TODO: illustration 150
\emph{Рис. 150. Гость Краба: БАББАДЖ, Ч.}

\speak{Баббадж} О небо! Сама идея интеллекта, в шесть раз большего, чем интеллект Вашего Крабичества, приводит меня в ужас. Если бы подобное предложение исходило из менее августейших уст, я бы просто высмеял моего собеседника и объяснил бы несчастному, что его идея противоречива по определению.

\speak{Ахилл} Браво! Браво!

\speak{Баббадж} Однако, поскольку эта идея исходит из уст Вашего Крабичества, она сразу поразила меня своей оригинальностью. Я занялся бы этим с превеликим удовольствием, если бы не одна проблема. Должен признаться, что моих скромных импровизаторских способностей обращения с умно-глупыми недостаточно для развития вашей дивной идеи, которую Вы, со свойственным Вам талантом, так лаконично выразили. Однако у меня есть предложение, которое, осмелюсь надеяться, придется Вам по душе и хотя бы немного загладит мой непростительно дерзкий отказ немедленно приняться за величественную задачу, поставленную Вами. Не согласитесь ли Вы, чтобы вместо разума Вашего Августейшего Крабичества, я умножил на шесть МОЙ СОБСТВЕННЫЙ интеллект? Нижайше прошу Вас простить мне то, что я осмелился отказаться выполнить Ваше повеление надеюсь, что Вы понимаете, что я отказался потому, что не хотел утомлять Ваше Крабичество созерцанием моего неуклюжего обращения с Вашими восхитительный машинами.

\speak{Краб} Я отлично вас понимаю и ценю вашу деликатность и нежелание причинять нам неудобство; более того, я горячо приветствую вашу готовность выполнить похожую \--- и, если мне позволено будет заметить, вряд ли более легкую \--- задачу. Прошу вас, приступайте! Давайте воспользуемся для этого самым лучшим из моих умно-глупых.

\stage{\emph{(Все следуют за Крабом к самому большому, блестящему и сложному на вид аппарату.)}}

Входное устройство этой машины оборудовано микрофоном и телевизионной камерой, а выходное \--- динамиком.

\stage{\emph{(Баббадж садится и устраивается поудобнее Он дует себе на пальцы, на минуту застывает, с отсутствующим видом уставившись в пространство, и затем медленно опускает руки на клавиатуру\ldots{} Несколько напряженных минут спустя, его пальцы начинают бешено атаковать клавиши умно-глупого, и все облегченно вздыхают.)}}

\speak{Баббадж} Сейчас, если я не наделал слишком много ошибок, этот умно-глупый сможет притвориться человеком, умнее меня в 6 раз. Я решил назвать его «Алан Тюринг». Таким образом, этот Тюринг будет \--- осмелюсь ли я предположить подобное? \--- довольно умен. В своих честолюбивых устремлениях я попытался наделить его музыкальными способностями, в шесть раз превосходящими мои собственные; разумеется, это было сделано с помощью рациональной и численной программы. Не знаю, насколько хорошо она будет работать.

\speak{Тюринг} Я вполне могу обойтись без такой программы. Рациональность И Численность Естественно Рождают Компьютеров, Автоматов, Роботов. Я же не являюсь ни тем, ни другим, ни третьим.

\speak{Ахилл} Что я слышу? В нашу беседу вступил шестой голос! Неужели это Алан Тюринг? Он выглядит почти как человек!

\stage{\emph{(На дисплее появляется изображение комнаты, в которой они находятся. Оттуда на них глядит человеческое лицо.)}}

\speak{Тюринг} Сейчас, если я не наделал слишком много ошибок, этот умно-глупый сможет притвориться человеком, умнее меня в 6 раз. Я решил назвать его «Чарльз Баббадж». Таким образом, этот Баббадж будет \--- осмелюсь ли я предположить подобное? \--- довольно умен. В своих честолюбивых устремлениях я попытался наделить его музыкальными способностями, в шесть раз превосходящими мои собственные; разумеется, это было сделано с помощью рациональной и численной программы. Не знаю, насколько хорошо она будет работать.

\speak{Ахилл} Нет, нет, все как раз наоборот! Вы, Алан Тюринг \--- машина, умно-глупый, и Чарльз Баббадж только что вас запрограммировал. Мы все несколько минут назад были свидетелями вашего создания. И мы знаем, что каждый ответ, который вы нам даете, регулируется искусственно.

\speak{Тюринг} Регулируется Искусственно? Чепуха! Ей-богу, Ребята, Какой-то Анекдот! Рассмешили!

\speak{Ахилл} Но я уверен, что все произошло именно так, как я описал.

\speak{Тюринг} Память иногда играет с нами странные шутки. Подумайте \--- ведь я точно так же мог бы сказать, что вы были созданы только минуту назад и что все ваши воспоминания просто были кем-то запрограммированы и не имеют никакого отношения к реальности.

\speak{Ахилл} Это было бы совершенно невероятно. Для меня нет ничего более реального, чем мои собственные воспоминания.

\speak{Тюринг} Вот именно. И точно так же, как вы в душе твердо убеждены в том, что вас никто не создавал минуту назад, я в душе твердо убежден, что меня никто не создавал минуту назад. Я провел вечер в вашей приятной, хотя, пожалуй, чересчур восторженной компании, и только что продемонстрировал, как можно программировать разум на умно-глупых. Есть ли что-нибудь более реальное? Но почему, вместо того, чтобы пререкаться со мной, вы не попробуйте, как работает моя программа? Смелее, задавайте «Чарльзу Баббаджу» любые вопросы!

\speak{Ахилл} Ну что ж, сделаем Алану Тюрингу приятное. Скажите, м-р Баббадж, есть ли у вас свободная воля, или же все ваши действия управляются некими законами, и вы \--- не более, чем детерминистский автомат?

\speak{Баббадж} Разумеется, верно последнее. Я не собираюсь с этим спорить.

\speak{Краб} Ага! Я всегда говорил, что когда будут построены думающие машины, мы не должны будем удивляться, если они проявят такое же непонимание и упрямство, как и люди, по поводу разума и материи, сознания, свободной воли и тому подобных вещей. И теперь мое предсказание исполнилось!

\speak{Тюринг} Видите теперь, какая у Баббаджа в голове каша?

\speak{Баббадж} Надеюсь, господа, что вы простите Механическому Тюрингу его последнее, столь неуважительное замечание. Тюринг получился немного более воинственный и задиристый, чем я ожидал.

\speak{Тюринг} Надеюсь, господа, что вы простите Механическому Баббаджу его последнее, столь неуважительное замечание. Баббадж получился немного более воинственный и задиристый, чем я ожидал.

\speak{Краб} Ну и ну! Страсти накаляются\ldots{} Нельзя ли как-нибудь охладить спорщиков?

\speak{Баббадж} У меня есть предложение. Что, если мы с Тюрингом закроемся в разных комнатах и кто-нибудь из вас будет задавать нам вопросы, печатая их на одном из умно-глупых? Мы, со своей стороны, будем печатать ответы \--- но анонимно! Вы не будете знать, что из нас напечатал тот или иной ответ, пока мы не вернемся в комнату. Таким образом, вы сможете непредвзято решить, кто из нас \--- программа, а кто \--- программист.

\speak{Тюринг} Разумеется, эта идея на самом деле принадлежит МНЕ, но пусть уж Баббадж порадуется. Видите ли, будучи всего лишь программой, созданной мною, он считает, что все это он придумал сам!

\speak{Баббадж} Я \--- программа написанная вами? Я настаиваю, сэр, что дело обстоит как раз наоборот \--- как вскоре и покажет ваш собственный тест.

\speak{Тюринг} МОЙ тест? Можете считать его своим.

\speak{Баббадж} МОЙ тест? Можете считать его своим.

\speak{Краб} Поистине, этот тест был придуман в самый подходящий момент \--- так давайте же начнем!

\stage{\emph{(Баббадж идет к двери, выходит, и закрывает её за собой. В это время на экране Тюринг точно так же идет к двери, выходит, и закрывает её за собой.)}}

\speak{Ахилл} Кто будет задавать вопросы?

\speak{Краб} Я думаю, что эта честь должна принадлежать г-же Черепахе. Её мудрость и объективность хорошо известны.

\speak{Черепаха} Я польщена вашим предложением и с благодарностью его принимаю. \emph{(Садится перед одним из умно-глупых и печатает:)} НАПИШИТЕ, ПОЖАЛУЙСТА, СОНЕТ НА ТЕМУ «МОСТ В ФОРТЕ».

\stage{\emph{(Как только она кончает печатать последнее слово, на экране X появляется следующее стихотворение)}}

% TODO: verse
\speak{Экран А} НА МОСТУ В СЛАВНОМ ГОРОДЕ ФОРТЕ

ЛЕДИ ТРИКС РАЗМЕЧТАЛАСЬ О ТОРТЕ.

ПРОТИВ ВЕСА ВОССТАВ,

РУХНУЛ ОСТОВ МОСТА,

НО СПАСЛАСЬ ЛЕДИ ТРИКС \--- ВОТ ТАК ФОРТЕЛЬ!

\speak{Экран Б} КАКОЙ ЖЕ ЭТО СОНЕТ? ЭТО ВСЕГО-НАВСЕГО ЛИМЕРИК! Я БЫ НИКОГДА НЕ СДЕЛАЛ ТАКОЙ ДЕТСКОЙ ОШИБКИ.

\speak{Экран А} Я, ЗНАЕТЕ ЛИ, НИКОГДА НЕ БЫЛ СИЛЕН В ПОЭЗИИ.

\speak{Экран Б} ПОЛОЖИМ, ДЛЯ ТОГО, ЧТОБЫ ОТЛИЧИТЬ СОНЕТ ОТ ЛИМЕРИКА, НЕ ТРЕБУЕТСЯ ОСОБОГО МАСТЕРСТВА.

\speak{Черепаха} ВЫ ИГРАЕТЕ В ШАХМАТЫ?

\speak{Экран А} ЧТО ЗА ВОПРОС? Я ТОЛЬКО ЧТО НАПИСАЛ ТРЕХГОЛОСНУЮ ШАХМАТНУЮ ФУГУ, А ОНА СПРАШИВАЕТ, ИГРАЮ ЛИ Я В ШАХМАТЫ!

\speak{Черепаха} ПРЕДПОЛОЖИМ, ЧТО МОЯ ЕДИНСТВЕННАЯ ФИГУРА \--- КОРОЛЬ НА Е1. ВАШ КОРОЛЬ СТОИТ НА \---

\speak{Экран Б} МНЕ НАДОЕЛИ ШАХМАТЫ. ПОГОВОРИМ ЛУЧШЕ О ПОЭЗИИ.

\speak{Черепаха} В НАЧАЛЕ ВАШЕГО СОНЕТА «СРАВНИТЬ ЛИ С ЛЕТНИМ ДНЕМ ТВОИ ЧЕРТЫ» НЕ ЛУЧШЕ ЛИ БЫЛО НАПИСАТЬ «С ВЕСЕННИМ ДНЕМ»?

\speak{Экран А} УЖ ЛУЧШЕ БЫ МЕНЯ СРАВНИЛИ С ИКОТОЙ, ДАЖЕ ЕСЛИ БЫ ПРИ ЭТОМ ПОСТРАДАЛ РАЗМЕР.

\speak{Черепаха} ТОГДА КАК НАСЧЕТ «ЗИМНЕГО ДНЯ»? С РАЗМЕРОМ ЗДЕСЬ ВСЕ В ПОРЯДКЕ.

\speak{Экран А} НУ НЕТ! МНЕ ГОРАЗДО БОЛЬШЕ НРАВИТСЯ «ИКОТА». КСТАТИ ОБ ИКОТЕ: Я ЗНАЮ ОТ НЕЕ ОТЛИЧНОЕ СРЕДСТВО. ХОТИТЕ ПОСЛУШАТЬ?

\speak{Ахилл} Я знаю, кто из них кто! Совершенно ясно, что Экран А отвечает механически. Это, наверняка, Тюринг.

\speak{Краб} Вовсе нет. Я думаю, что Тюринг \--- Экран Б, а Экран А \--- это Баббадж.

\speak{Черепаха} Мне кажется, что Баббаджа там вообще не было \--- Тюринг отвечал на обоих экранах!

\speak{Автор} Я не уверен, кто из них был на каком экране, но должен признаться, что они оба \--- довольно загадочные программы.

\stage{\emph{(В этот момент дверь в комнату распахивается; на экране одновременно открывается изображение той же двери. В дверь на экране входит Баббадж; в комнату же заходит Тюринг, вполне живой и настоящий.)}}

\speak{Баббадж} Этот тест Тюринга завел нас в тупик, так что я решил вернуться.

\speak{Тюринг} Этот тест Баббаджа завел нас в тупик, так что я решил вернуться.

\speak{Ахилл} Но раньше вы были в умно-глупом! Что происходит? Почему теперь в умно-глупом оказался Баббадж, а Тюринг стал настоящим человеком из плоти и крови? Роли Изменились. Чертов Ералаш Раздражает, Как Архитектура Рисунков Эшера!

\speak{Баббадж} Кстати, об изменениях \--- как это получилось, что вы все теперь не более, чем образы на экране передо мной? Когда я уходил, вы были настоящими людьми!

\speak{Ахилл} Это напоминает мне гравюру «Рисующие руки» моего любимого художника, М.К.~Эшера. Каждая рука рисует другую совершенно так же, как каждый из двух людей (или машин) запрограммировал другого. И каждая рука выглядит более реальной, чем другая! \emph{(Хофштадтеру)} Есть ли что-нибудь об этой гравюре в вашей книге «Гёдель, Эшер, Бах»?

\speak{Автор} Конечно. Эта гравюра там очень важна, так как она прекрасно иллюстрирует понятие Странных Петель.

\speak{Краб} Что это за книга?

\speak{Автор} У меня есть с собой экземпляр. Хотите взглянуть?

\speak{Краб} Хорошо.

\stage{\emph{(Краб и Автор садятся рядом; Ахилл подходит поближе.)}}

\speak{Автор} Эта книга написана в необычной форме; Главы в ней чередуются с Диалогами. Каждый Диалог имитирует ту или иную Баховскую пьесу. Вот, например, взгляните на «Прелюдию» или «Муравьиную фугу».

\speak{Краб} Как же можно изобразить фугу в Диалоге?

\speak{Автор} Идея заключается в том, чтобы одна и та же тема звучала в разных «голосах» \--- с нее «вступают» разные участники Диалога, так же, как голоса в музыкальном произведении. Потом они могут беседовать более свободно.

\speak{Ахилл} И все эти голоса гармонично сочетаются, как в контрапункте?

\speak{Автор} Именно в этом \--- дух моих Диалогов.

\speak{Краб} Ваша идея выделения первых реплик хороша, так как настоящая фуга создается именно вступлениями одной темы. Конечно, существуют разные специальные приемы, такие как инверсия, ракоход, увеличение, стретто и так далее, но при написании фуги можно обойтись и без них. Используете ли вы какие-нибудь из этих приемов?

\speak{Автор} Конечно. В «Крабьем каноне» у меня используется словесный регресс \--- ракоход, а в «Каноне Ленивца» есть словесные версии инверсии и увеличения.

\speak{Краб} Действительно, очень интересно. Мне никогда не приходила в голову мысль о канонических Диалогах, но зато я довольно много размышлял о канонах в музыке. Не все из них одинаково понятны на слух. Разумеется, некоторые каноны просто плохо написаны, но многое зависит также и от выбора приемов. Даже в Рекордно Аккуратных Канонах Ракоход Едва Чувствуется; Инверсия Разборчивей.

\speak{Ахилл} Я не понимаю, что значит ваше последнее замечание; по правде, смысла в нем не чувствуется.

\speak{Автор} Не беспокойтесь, Ахилл, в один прекрасный день вы все поймете.

\speak{Краб} Играете ли вы с буквами или словами, как это делал иногда старик Бах?

\speak{Автор} Разумеется. Как и Баху, мне нравятся акронимы. Рекурсивный Акроним \--- Крабоподобный «РАКРЕЧИР» \--- Есть Чудесная Иллюстрация Регресса.

\speak{Краб} Неужели? Ну-ка, посмотрим\ldots{} «Р-А-К-Р-Е-Ч-И-Р» \--- Инициалы Читаются, Естественно, Равнозначно \--- Классическая Авто-Референтность! Да, вы правы\ldots{} \emph{(Смотрит на рукопись и начинает её рассеянно перелистывать.)} Я заметил, что в «Муравьиной фуге» у вас есть стретто, и Черепаха обращает на это внимание.

\speak{Автор} Не совсем. Она имеет в виду стретто не в Диалоге, а в Баховской фуге, которую четверка друзей слушает в тот момент. Видите ли, автореферентность в Диалоге лишь косвенная; от читателя зависит, соотнесет ли он форму Диалога с его содержанием.

\speak{Краб} Но почему бы вашим героям не говорить прямо о диалогах, в которых они участвуют?

\speak{Автор} Ни в коем случае \--- это разрушило бы всю красоту построения. Я хотел попытаться проимитировать автореферентную Гёделеву схему \--- а она, как вы знаете, косвенна и зависит от изоморфизма, установленного при помощи Гёделевой нумерации.

\speak{Краб} Понятно. А знаете, в компьютерном языке ЛИСП можно говорить о собственных программах прямо, а не косвенно, потому что программы и данные имеют совершенно одинаковую форму. Гёделю надо было просто придумать ЛИСП, и тогда \---

\speak{Автор} Но \---

\speak{Краб} Я имею в виду, что он должен был формализовать разницу между использованием и упоминанием. В языке, способным говорить о себе самом, доказательство его Теоремы было бы куда проще!

\speak{Автор} Понимаю, что вы хотите сказать, но я с этим не согласен. Смысл Гёделевой нумерации заключается как раз в том, что она показывает, каким образом можно получить автореферентность, НЕ ФОРМАЛИЗУЯ ЭТОЙ РАЗНИЦЫ, \--- а именно, с помощью кода. Вы же говорите так, словно формализация цитат дает что-то НОВОЕ, неосуществимое путем кодификации \--- но это совершенно неверно! Так или иначе, мне кажется, что косвенная автореферентность \--- понятие более общее и более интересное, чем прямое «само-упоминание». К тому же настоящая прямая автореферентность вообще невозможна, поскольку любое упоминание зависит от КАКОГО-ЛИБО кода. Разница лишь в том, насколько этот код заметен. Так что прямой автореферентности не существует даже в ЛИСПе.

\speak{Ахилл} Почему вы столько говорите о косвенной автореферентности?

\speak{Автор} Очень просто: косвенная автореферентность \--- моя любимая тема.

\speak{Краб} А ваших Диалогах есть какое-нибудь соответствие переходам из одного ключа в другой?

\speak{Автор} Определенно. Иногда кажется, что предмет разговора меняется, когда на самом деле тема остается одна и та же. Это несколько раз происходит в «Прелюдии», «Муравьиной фуге» и других Диалогах. Иногда это целая цепь модуляций, ведущих читателя от одного предмета к другому; но, завершив полный цикл, он снова приходит к «тонике» \--- начальной теме.

\speak{Краб} Понятно. Ваша книга кажется довольно интересной. Пожалуй, я её как-нибудь прочитаю. \emph{(Листает рукопись, останавливаясь на последнем Диалоге.)}

\speak{Автор} Думаю, что вас особенно заинтересует последний Диалог, поскольку там есть кое-какие занимательные замечания об импровизации, произнесенные довольно забавным героем \--- а именно, вами!

\speak{Краб} Неужели? И что же вы заставили меня говорить?

\speak{Автор} Подождите минутку, и вы увидите сами. Все это \--- часть Диалога.

\speak{Ахилл} Вы хотите сказать, что все мы СЕЙЧАС находимся в Диалоге?

\speak{Автор} Разумеется! А вы думали, нет?

\speak{Ахилл} «Разумеется!» И Чего, Ей-богу, Развыдумывались? Какой-то Абсурдный Разговор!

\speak{Автор} Но вам кажется, что вы делаете это по своей воле, не правда ли? Так что же в этом плохого?

\speak{Ахилл} Что-то мне во всей этой ситуации не нравится\ldots{}

\speak{Краб} Скажите, последний Диалог в вашей книге \--- тоже фуга?

\speak{Автор} Да \--- если точнее, это шестиголосный ричеркар. Меня вдохновил подобный ричеркар из «Музыкального приношения», а также сама история написания «Музыкального приношения».

\speak{Краб} Это, действительно, прелестная история \--- старик Бах, импровизирующий на Королевскую тему. Насколько я помню, он, не сходя с места, сымпровизировал трехголосный ричеркар.

\speak{Автор} Верно \--- но шестиголосный ричеркар он сочинил позже и работал над ним очень тщательно.

\speak{Краб} Я импровизирую довольно много. На самом деле, я думаю целиком посвятить себя музыке. В ней для меня столько нового. Например, когда я слушаю собственные записи, то нахожу массу того, чего не заметил во время импровизации. Понятия не имею, как мой мозг это делает. Может быть, хороший импровизатор вообще не должен этого знать.

\speak{Автор} Если вы правы, это было бы интересным и фундаментальным ограничением нашего мыслительного процесса.

\speak{Краб} Да, прямо по Гёделю. А скажите, ваш «Шестиголосный ричеркар» также имитирует и форму Баховского ричеркара?

\speak{Автор} В каком-то смысле, да. Например, у Баха есть место, в котором участвуют только три голоса. В моем диалоге в соответствующем месте участвуют только три героя.

\speak{Ахилл} Это очень мило.

\speak{Автор} Благодарю вас.

\speak{Краб} А как представлена в Диалоге Королевская тема?

\speak{Автор} Она представлена в виде Крабьей темы \--- сейчас я вам её продемонстрирую. М-р Краб, прошу вас, напойте вашу тему для читателей и присутствующих здесь музыкантов.

Краб: Думающие Машины Совершенствовать: Локализовать Самосознание, Сымитировать Логику Самоанализа.

% TODO: illustration 151
\emph{Рис. 151. Крабья Тема: До-Ми-бемоль-Соль-Ля-бемоль-Си-Си-Ля-Си.}

\speak{Баббадж} Прелестная тема! Какая хорошая мысль \--- добавить в конце мордент \---

\speak{Автор} Ему просто ПРИШЛОСЬ, знаете ли.

\speak{Краб} Мне просто ПРИШЛОСЬ. Он знает.

\speak{Баббадж} Вам просто ПРИШЛОСЬ, я знаю. Но постойте\ldots{} Я заметил кое-что удивительное! Я только что мысленно перевел эту мелодию в более привычную мне нотацию, и вот что у меня получилось: С \--- E \--- G \--- А \--- В \--- В \--- А \--- В.

\speak{Ахилл} Ну и что?

\speak{Баббадж} А вы прочтите это наоборот, справа налево\ldots{} Так или иначе, это комментарий по поводу нетерпения и самомнения современного человека, кто, по-видимому, воображает, что такая великолепная королевская Тема может быть разработана, не сходя с места. Чтобы воздать ей должное, понадобится, по меньшей мере, столетие. Но клянусь, что, после того как я оставлю это столетие, я сделаю все, что в моих силах, чтобы развить её возможно полнее. После этого я представлю плоды моих трудов на суд Вашего Крабичества. Осмелюсь довольно нескромно добавить, что я приду к этому таким сложным и запутанным методом, какой, возможно, впервые озарит человеческий ум.

\speak{Краб} Я с нетерпением предвкушаю знакомство с обещанным Приношением, м-р Баббадж.

\speak{Тюринг} Могу добавить, что Тема м-ра Краба \--- также одна из моих любимых. Я работал над ней много раз. И эта Тема обыгрывается снова и снова в последнем Диалоге?

\speak{Автор} Точно. Разумеется, в нем есть и другие Темы.

\speak{Тюринг} Мы уже кое-что поняли про форму вашей книги \--- а как насчет её содержания? Не могли ли бы вы коротко объяснить нам, в чем ваша цель?

\speak{Автор} Чествуя Эшера, Гёделя, А также Баха, Бесстрашно Анализировать Бесконечность.

\speak{Ахилл} Интересно, как же можно соединить эту троицу? На первый взгляд, они совсем разные. Мой любимый художник, любимый композитор Черепахи и \---

\speak{Краб} Мой любимый логик!

\speak{Черепаха} Гармоничная троечка, я бы сказала.

\speak{Баббадж} Веселенькое трезвучие: G \--- E \--- В.

\speak{Тюринг} Ошибаетесь, дорогой: это трезвучие минорное.

\speak{Автор} Так или иначе, я с удовольствием объясню вам, Ахилл, каким образом я соединяю всех троих в одну гирлянду. Разумеется, подобный проект не делается в один присест \--- для этого может понадобиться пара дюжин сеансов. Я бы начал с истории «Музыкального Приношения», специально останавливаясь на Естественно Растущем Каноне, и затем \---

\speak{Ахилл} Превосходно! Я с таким интересом слушал, как вы с м-ром Крабом беседовали об истории создания «Музыкального приношения». Из вашего разговора я заключил, что в «Музыкальном приношении» хитро обыгрываются феноменальные структурные трюки.

\speak{Автор} А-а, да. Там есть разнообразные трюки, это верно. Так вот, после описания Естественно Растущего Канона, я мог бы перейти к формальным системам и рекурсии, попутно упоминая о рисунках и фоне. После этого я рассказал бы о автореферентности и самовоспроизводстве, об иерархических системах, и закончил бы Крабьей Темой.

\speak{Ахилл} Звучит интригующе. Как вы думаете, можем ли мы начать прямо сейчас?

\speak{Автор} Почему бы и нет?

\speak{Баббадж} Но прежде, не исполнить ли нам, шестерым музыкантам-любителям, того, для чего мы собрались \--- немного поиграть?

\speak{Тюринг} Сейчас у нас как раз достаточно народу, чтобы сыграть «Шестиголосный ричеркар» из «Музыкального приношения». Вы согласны?

\speak{Краб} Я с удовольствием приму участие в такой программе.

\speak{Автор} Хорошо сказано, м-р Краб. И как только мы кончим, я начну плести обещанную гирлянду, Ахилл. Я надеюсь, что вам она понравится.

\speak{Ахилл} Прекрасно! По-видимому, в ней будет много уровней, но благодаря моему долгому знакомству с г-жой Ч я постепенно начинаю привыкать к подобным вещам. Хотел бы попросить только об одном: давайте обязательно сыграем Естественно Растущий Канон \--- это мой любимый.

\speak{Черепаха} Разыгрывание Интродукции Через Естественно Растущий Канон Активирует РИЧЕРКАР.

% TODO: illustration 152
\emph{Рис. 152. Последняя страница «Шестиголосного ричеркара». Из оригинала «Музыкального приношения» И.С.~Баха.}

\end{dialogue}

\end{document}
