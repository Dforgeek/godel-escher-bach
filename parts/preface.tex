\documentclass[../main.tex]{subfiles}
\begin{document}

\ifSubfilesClassLoaded{% subfile
    \frontmatter
    \chapterstyle{FrontMatterChapterStyle}
}{% main file
}

\chapter[Предисловие]{Праздничное предисловие автора к русскому изданию книги «Гёдель, Эшер, Бах»}

\subsection{Май MMI}

Позвольте мне начать с истории, случившейся со мной в раннем детстве, \--- по-моему, эта история довольно показательна. Когда мне было три или четыре года, меня внезапно поразила сияющая, таинственная красота того факта, что ДВЕ ДВОЙКИ \--- ЭТО ЧЕТЫРЕ\@. Только маленький ребенок может любить что-либо так глубоко, с таким самозабвением. Может быть, дело было в том, что маленький Дагги подсознательно почувствовал, что эта короткая фраза двусмысленна, что в ней одновременно заключены две различных истины, одна \--- о понятии «$2 + 2$», другая \--- о понятии «$2 \times 2$» (впрочем, сомневаюсь, что в те времена я знал что-либо об умножении). Другое возможное объяснение моей очарованности понятием «двух двоек» \--- то, что оно прилагало идею к себе самой \--- а именно, идею двойки к самой этой двойке «Давайте-ка возьмем двойку ДВА раза!»

Как бы мы ни старались выразить первозданную красоту этой (или какой-нибудь другой) идеи словами, вскоре очарование начинает таять и мы, разочарованные, умолкаем. Однако, жадный до развлечений ребенок, как и взрослый, интереса не теряет и желает заново испытать радость открытия с помощью какого-нибудь обобщения или аналогии. В своем нежном возрасте я не являлся исключением. Я попытался обобщить мою чудесную идею «двойки, действующей саму на себя», и у меня получилось\ldots{} Сказать вам по правде, я и сам не знаю, что у меня тогда получилось \--- и тут я подхожу к самому главному в этой истории.

Когда в 1979 году я писал предисловие к английской версии «Гёделя, Эшера, Баха», я думал, что понял, какое обобщение придумал малыш Дагги. Я написал, что малыш сформулировал идею «трех троек» и спросил маму (хотя искренно сомневался в том, что она \--- или кто-либо другой в целом мире \--- в состоянии мыслить на таком высоком уровне абстракции), что получится в результате этой немыслимой операции. Однако после того, как книга вышла в свет, я продолжал размышлять о том случае и пытаться вспомнить точнее, что же всё-таки произошло. В голове у меня всплывали разные полузабытые картинки, вроде нашего первого автомобиля, в котором мы как раз сидели, когда я задал свой вопрос, моего любимого розового одеяльца \--- оно тогда было в машине \--- и множества других не относящихся к делу подробностей. Чем больше я напрягал память, тем расплывчатее становился мой «синтетический бриллиант».

Я 2001 года, в отличие от меня 1979 года, нахожу маловероятным, чтобы маленький Дагги действительно считал свою мать неспособной понять идею «трех троек» \--- в конце концов мама была для него источником сверхъестественной мудрости! Сейчас я склонен полагать, что Дагги пытался вообразить и затем выразить своей маме \--- МОЕЙ маме! \--- гораздо более абстрактное понятие, чем то, которое смышленый ребенок может описать как «ТРИЖДЫ три тройки».

Трехлетним малышом я не мог додуматься до того, что эта идея может быть представлена геометрически и даже построена в виде кубика из трех 3x3 слоев. Я был ещё слишком мал для того, чтобы воплотить мое смутное прозрение в конкретные образы. Меня увлекало САМО ЭТО ВЫРАЖЕНИЕ \--- и в частности, содержащаяся в нем волшебная идея «самоприложения троичности».

Если бы я был поискушеннее, я мог бы понять, что на самом деле я искал третью бинарную операцию в натуральной (и бесконечной) последовательности «сложение, умножение, возведение в степень\ldots» С~другой стороны, если бы я был настолько искушен, то мог бы пойти дальше и обнаружить смертельный недостаток, заключающийся в слове «бинарная», означающее всего-навсего «двоичная». Этот недостаток бросается в глаза в краткой записи моего детского прозрения:
\[
    3^3
\]
Да, к сожалению, здесь только две копии тройки; возведение в степень \--- бинарная операция.

Увидев, что моя наклонная башня имеет только два этажа, я, разумеется, захотел бы пойти дальше и построить вот это трехэтажное сооружение, опасно смахивающее на Пизанскую башню.

С первого взгляда кажется, что здесь всё в порядке \--- но увы, я мог бы затем понять, что даже такая башня может быть сокращена до «$3 \text{\textasciicircum} 3$» (маленькая шапочка обозначает операцию \#4 в упомянутой последовательности). Таким образом, двоичность опять вползла бы в построение и мои надежды были бы обмануты.

Думаю, что на этом этапе я бы уже понял, в чем тут дело, и сообразил, что «самоприложение троичности» просто невозможно реализовать в такой совершенной и прекрасной форме, как это можно сделать с двоичностью \--- там-то всегда выходит четыре, что бы с двумя двойками ни проделывали. Неважно, какую из бесконечной последовательности двоичных операций вы проделаете \--- сложение, умножение, возведение в степень \--- вы всегда получите один и тот же результат: четыре. С другой стороны, «три плюс три» совсем не то же самое, что «трижды три» \--- а это, в свою очередь, не то же самое, что «три в третьей степени», или «три шапка три», или любая из последующих более сложных операций последовательности.

Нет нужды говорить, что всё это было намного выше понимания маленького Дагги \--- и~всё же своим детским умишком он пытался нащупать все эти глубокие математические понятия. И уже в этих его детских неуклюжих попытках постичь тайну самоприложения вы можете заметить \--- я могу заметить \--- первые ростки его увлечения (МОЕГО увлечения!) самоописывающими высказываниями и самоприложимыми мыслями, и главной тайной «самости» \--- той бесконечно ускользающей сущности, которая заключена в крохотном, всего из одной буквы, слове~«Я». Можно даже сказать, что книгу, которую вы держите в руках, \--- русский перевод моей книги «Гёдель, Эшер, Бах» \--- лучше всего охарактеризовать как большой трактат, основная цель которого \--- раскрыть тайну слова~«я». К несчастью, читатели, думающие, что заглавие должно быть кратким пересказом содержания, не воспринимают мою книгу таким образом.

% \begin{adjustwidth}{\parindent}{}
\begin{dialogue}
\speak{Читатель} «ГЭБ» \--- про математика, музыканта и художника!

\speak{Автор} Нет, вы не правы.

\speak{Читатель} «ГЭБ» \--- о том, что математика, музыка и искусство \--- одно и то~же!

\speak{Автор} Опять ошибаетесь.

\speak{Читатель} Про что же тогда эта книга?

\speak{Автор} Про тайные абстрактные структуры, лежащие в основе слова~«я».
\end{dialogue}
% \end{adjustwidth}

История, которую я вам рассказал, дает некоторое понятие о том, что говорит мне моя интуиция о природе этих тайных абстрактных структур. Эта связь станет вам яснее, когда вы дочитаете до главы XII и увидите, как загадочная идея Курта Гёделя, «арифмоквайнирование» (как я его называю), довольно странным образом прилагается сама к себе. В результате получается удивительная структура, которая (как «две двойки») устойчиво автореферентна и (в отличие от «трех троек») не указывает ни на что, кроме себя самой.

Моя влюбленность в Курта Гёделя с его центральным, основным примером абстрактного явления, которое я окрестил «странными петлями», была той искрой, из которой родилась «ГЭБ». Идея этой книги появилась в 1972 году, когда мой мозг был раскален до белого каления, и я, аспирант кафедры теоретической физики одного американского университета, с трудом продирался сквозь дебри науки. Тогда мне необычайно повезло \--- в мои руки попала изумительная книга по математической логике. Та книга заставила меня совершенно забросить теоретическую физику, которой я должен был заниматься. Написанная философом Говардом Делонгом, она называлась «Краткий очерк математической логики» и захватила меня настолько, как я не мог и предположить. Внезапно она оживила ту горячую любовь, что я подростком испытывал к идеям, имеющим очень отдаленное отношение к физике. Тогда, в начале шестидесятых, я был очарован математикой и иностранными языками и исследовал множество различных структур \--- структур, состоящих из чисел и других математических понятий, структур, сделанных из слов и символов, структур, построенных из самих мыслительных процессов. В~те годы, когда складывалась моя личность, я бесконечно раздумывал над связью между словами и идеями, символами и их значениями, мыслями и формальными правилами мышления. Но сильнее всего меня интересовала связь между физическим веществом человеческого мозга и неуловимой сущностью~«я».

Почему я был так увлечен всем этим? Разумеется, никогда нельзя с точностью указать на причину возникновения какой бы то ни было страсти; тем не менее, в моей жизни было несколько определенных факторов, которые в какой-то мере объясняют мой интерес к подобного рода темам. Во-первых, с раннего детства я любил не только числа, но и сложные, драгоценные узоры, построенные с помощью чисел. История про маленького Дагги это подтверждает.

Во-вторых, в 1958\--1959 я научился бегло говорить по-французски и жгуче заинтересовался загадкой невинной на первый взгляд фразы «думать по-французски» \--- фразы, которую окружающие употребляли простодушно и бездумно. Мне же казалось, что под поверхностью каких бы то ни было слов на любом языке лежат чистые МЫСЛИ, которые по определению должны быть глубже, чем слова, глубже, чем любая грамматика. Однако меня сбивало с толку то, что даже сами эти «чистые мысли», по-видимому, зависели от выбранного мной средства общения. Так я обнаружил, что, когда я «думаю по-французски» мне в голову приходят совсем иные мысли, чем когда я «думаю по-английски»! Мне захотелось понять, что же главнее, язык или мысли? Способ передачи сообщения или само сообщение? Форма или содержание? И кто же всем этим управляет? Есть ли в моем мозгу место для меня самого?

Последним и гораздо более печальным фактором было плачевное состояние моей младшей сестры Молли, чья загадочная неспособность научиться говорить и понимать речь приводила в отчаяние моих родителей, меня, и мою другую сестру, Лауру. Болезнь Молли подвигла меня на прочтение пары книг о мозге \--- и~я был поражен кажущейся бессмысленностью того, что неодушевленные молекулы, собранные вместе в некую сложную структуру, могут служить местонахождением самосознания, «внутреннего света». Эта глубоко личная, внутренняя искорка «самости» сознания казалась несовместимой с грубой материей \--- и~всё же я, выросший в семье ученых и в возрасте четырнадцати лет проглотивший блистательную, разоблачающую псевдонауку книгу Мартина Гарднера «Модные поветрия и заблуждения во имя науки», не терпел расхожего мистицизма или дуалистического философствования, типа «\'elаn vital» (витальный порыв). По-моему мнению, существование внутреннего света «я» было результатом неких структур, и не более того. Но каких именно структур? Трагическое состояние моей сестры только усилило мой жгучий интерес к этой загадке.

В то время в мою жизнь вошла другая ключевая книга. Шел 1959 год, я только что вернулся в Калифорнию после года, проведенного в Женеве (где я выучил французский), и по счастливой случайности мне в руки попала тоненькая книжица Эрнста Нагеля и Джеймса Ньюмана «Доказательство Гёделя». По случайному стечению обстоятельств, Нагель когда-то был учителем и другом моего отца; я~проглотил эту книгу за один присест. Поразительным образом я нашел там все мои интуитивные прозрения о сущности~«я». Важнейшими для доказательства Гёделя оказались все мои вопросы о символах, значении, правилах; важнейшим для этого доказательства было понятие «самоприложения», важнейшим для него было неизбежное переплетение сообщения и его носителя, порождающее новую, невиданную доселе никем структуру. Эта абстрактная структура, как мне казалось, и была ключом к загадке самосознания и возникновения~«я».

Странно, что прошло двенадцать лет, прежде чем я попытался выразить все эти интуитивные идеи сознательно и ясно; а виновата в этом была моя судьбоносная встреча с книгой Делонга в 1972 году. Если бы не эта книга, сомневаюсь, что «ГЭБ» появилась бы на свет. Тем не менее, этот клубок интуитивных знаний был порожден таким множеством других книг и идей, что было бы несправедливо указывать только на несколько из них.

Итак, как я уже говорил, «ГЭБ» \--- не о мистере Гёделе, мистере Эшере и мистере Бахе и не о близости между математикой, музыкой и искусством \--- и~всё же, в каком-то смысле, «ГЭБ», безусловно, и обо всем этом. Иначе зачем бы я назвал книгу именно так? Должен признаться, что в моем маленьком диалоге с читателем я был слишком категоричен, напрочь отрицая наиболее очевидные интерпретации содержания книги. Как всякое сложное создание, её можно увидеть под разными углами. На самом деле, если бы все читатели поняли «ГЭБ» как книгу о загадке «я» и ни о чем более, я был бы глубоко разочарован.

Я никогда не забуду чудесного мгновения летом 1981 года, когда я встретил О.\,Б.~Хардисона, в то время директора знаменитой Шекспировской библиотеки в Вашингтоне, и он, в ответ на мой недоуменный вопрос, почему меня пригласили участвовать в конференции, посвященной искусству литературного перевода, широко улыбнулся и сказал: «Нет ничего проще \--- ведь вся ваша книга о переводе. Поэтому она мне так и понравилась!»

Это замечание открыло мне, автору книги, глаза. Разумеется, на поверхностном уровне, в главах XII и~XVII прямо говорится о переводе; кроме того, в книге довольно много материала о «переводе» как механизме, при помощи которого живые клетки превращают химические вещества в белки. Но в этих отрывках слово «перевод» употребляется в его прямом значении. Однако, чем больше я думал о словах Хардисона, тем больше убеждался, что на более глубоком уровне он был совершенно прав. «ГЭБ» полна идей, переносимых из одной схемы в другую, аналогий между очень несхожими между собой областями \--- а это равносильно переводу. Более того, основная идея, вызвавшая к жизни эту книгу, идея, породившая изначально Странную Петлю Гёделя, связана с отображением одной системы на другую совершенно неожиданным, но изумительно точным способом. В этом смысле перевод \--- не~просто одна из многих переплетающихся тем «ГЭБ»; скорее всю эту книгу можно понять как исследование перевода в его метафорическом значении.

Случилось так, что в 1980\--1981 академическом году я потратил сотни часов, прокладывая пути для потенциальных переводов «ГЭБ» на другие языки. По правде сказать, ни о каком конкретном переводе тогда речь ещё не шла, но вскоре, воодушевленные успехом «ГЭБ» среди англоязычных читателей, издатели многих стран захотели, чтобы книга вышла на их языке. Я всю жизнь был влюблен в языки и меня заинтриговал вопрос о том, каким образом мои сложные многоуровневые каламбуры и структурные игры можно воспроизвести \--- или, по крайней мере, как можно верно передать их дух \--- в совершенно иной языковой среде. Пытаясь предусмотреть некоторые трудности будущих переводчиков, я, слово за словом, прошелся по книге с красной ручкой и отметил все каламбуры, и акростихи, все словесные перестановки, и переклички далеких отрывков текста: я объяснил трудноуловимые двойные (или тройные, или четверные, или пятерные) значения и указал отрывки, в которых форма отражает содержание; отметил те места книги, в которых сами особенности типографского набора передают важную информацию, посоветовал, какие затруднительные пассажи могут быть облегчены в переводе, а какие необходимо сохранить, и так далее. С этой кропотливой работой я провозился целый год, но делал её с любовью; так или иначе, она была необходима, чтобы предотвратить катастрофу.

Дело в том, что «ГЭБ» \--- не только книга, выражающая множество сложным образом переплетенных идей. Это ещё и книга, в которой крупномасштабные художественные структуры и замысловатые лингвистические и типографские приемы, выбранные для передачи этих структур, играют фундаментальную, центральную роль. Переводчикам очень редко приходится иметь дело с таким интимным переплетением формы и содержания, но мне было ясно, что если не передать в переводе все эти аспекты одновременно, то дух книги, её «изюминка» и очарование, над которыми я работал с такой страстью на английском, будут полностью утрачены. Короче говоря, «ГЭБ» на новом языке потеряет всю свою «ГЭБ»-ность, если она не будет реконструирована с таким же стараньем и артистизмом, какие были вложены в оригинал.

Позже мне довелось работать с несколькими переводчиками (или группами переводчиков) на разных уровнях творческого сотрудничества. Я был настолько близок к двум французским переводчикам, что мне временами казалось, что над книгой работает трио, а не дуэт. Участие в этом в высшей степени творческом процессе принесло мне редкостное интеллектуальное наслаждение. Мне также повезло принять участие, хотя и в гораздо меньшей степени, в испанском, немецком, голландском и китайском переводах «ГЭБ».

В 1985 году, в 300-ю годовщину рождения И.\,С.~Баха, французская, итальянская, голландская, немецкая, шведская и японская версии «ГЭБ» почти одновременно вышли из печати. Хотя многие сомневались в том, что эта книга вообще может быть переведена, каждая из этих версий излучала собственное очарование, искрилась своей собственной игрой слов \--- и в большинстве случаев, отдавала должное оригиналу. Некоторые отрывки оказались даже лучше, чем в оригинале! Во всех этих странах переводы «ГЭБ» были распроданы на удивление быстро, и мне доставило огромную радость видеть, как коллективные усилия творческих переводчиков и непредвзятых издателей сделали возможным это чудо.

Моей давней мечтой было увидеть перевод «ГЭБ» на русский язык \--- но разрыв между востоком и западом в те времена был настолько велик, что «ГЭБ», несмотря на её огромную популярность на западе, оставалась неизвестной подавляющему большинству русских читателей. Долгие годы эта ситуация оставалась без изменений, и я уже начал сомневаться, появится ли когда-нибудь «ГЭБ» на русском (или любом другом славянском языке). Однако в 1986 началась невероятно странная серия событий, которые после удивительных поворотов привели к тому, что через 15 лет русская версия моей книги появилась на свет. Позвольте мне вкратце рассказать эту историю.

Следуя одному из тех интуитивных прозрений, что бывают только у матерей, весной 1986 мама подарила мне только что вышедший роман «Золотые ворота». Написал его неизвестный индийский автор Викрам Сет, учившийся тогда в аспирантуре экономического факультета Стэнфордского университета, в городе, где я вырос. Когда я в первый раз открыл эту книгу, у меня отвисла челюсть от удивления: я увидел непрерывную цепь сонетов! Передо мной лежало произведение художественной литературы, во многом напоминающее «ГЭБ» \--- форма в нем была равноправным партнером содержания. Я никогда в жизни не слышал ни о чем подобном, и с энтузиазмом уселся за чтение «Золотых ворот». Чтение романа в стихах оказалось невероятно интересным занятием. Когда я в следующий раз навестил родной Стэнфорд, я связался с Викрамом Сетом и встретился с ним. Мы провели приятный вечер за чашкой кофе, и я спросил, что навело его на подобную необычную идею \--- написать роман в стихах. К моему удивлению, он ответил, что его вдохновил роман в стихах, написанный ранее \--- а именно, «Евгений Онегин» Александра Пушкина в английском переводе британского дипломата Чарлза Джонстона.

Я не предполагал, что творение Сета было основано на уже существовавшем труде; хотя я, разумеется, слышал название «Евгений Онегин», оно вызывало у меня единственную ассоциацию \--- с оперой Чайковского. Я был поражен. Более того, я узнал от Викрама, что он позаимствовал у Пушкина даже точную форму так называемой «онегинской строфы» и написал этой строфой весь свой роман. И вот венец этой истории: мы пили кофе не где-нибудь, а в кафетерии книжного магазина, и не какого-нибудь магазина, а именно того, где Викрам сочинил большую часть своей книги и который он блестяще описал в одной из строф (отступление совершенно в пушкинском духе!). И тут Викрам сделал мне замечательный подарок \--- купил для меня экземпляр перевода Джонстона, назвав его «светящимся» и «искрометным».

Вы, наверное, думаете, что получив подобную рекомендацию от автора, которым я там восхищался, я тут же засел за «Евгения Онегина» Джонстона и проглотил его с такой же жадностью, как раньше \--- «Золотые ворота»? Вовсе нет \--- почему-то я просто поставил его на полку, где он простоял шесть лет, с удовольствием собирая пыль. Понятия не имею, почему. Но однажды, когда я опять оказался в том же калифорнийском книжном магазине, я начал просматривать секцию поэзии и снова наткнулся на название «Евгений Онегин» \--- но этот томик был другого формата и его обложка была другого цвета. Я снял книгу с полки и увидел, что это был ещё один перевод, сделанный Джеймсом Фаленом, американским профессором-русистом. «Что?» \--- подумал я. «Как может кто-либо воображать, что он в состоянии переплюнуть Джонстона с его \enquote*{светящимся} и \enquote*{искрометным} переводом? Какая дерзость!». Тем не менее я перелистал книгу, прочитал наугад несколько строф и подумал: «На мой неискушенный слух, звучит вполне прилично. Почему бы мне её не купить?» Теперь я оказался гордым обладателем двух английских переводов «ЕО» \--- и что же с ними сталось? Разумеется, они простояли на моей полке, холодно игнорируя друг друга и собирая пыль, ещё в течение нескольких месяцев.

Однако в один прекрасный день 1993 года, они, безо всякой видимой причины, вдруг попались мне на глаза, и я внезапно сказал своей жене Кэроль: «Хочешь, почитаем вслух этот занятный русский роман в стихах, \enquote*{Евгений Онегин}? У меня есть две версии, и мы можем каждый читать свою и сравнивать их строфа за строфой». Она с энтузиазмом подхватила мою идею, и каждую ночь, уложив спать наших двух малышей, мы укладывались бок о бок, открывали двух «Онегиных» и читали друг другу, тщательно сравнивая обе версии. Кэроль совсем не знала русского, я знал его лишь чуть-чуть, так что у нас даже мысли не возникало заглядывать в оригинал \--- и тем не менее, сравнивая два прекрасно сделанных перевода во всех деталях, мы почувствовали, что понимаем, как пушкинский текст должен звучать по-русски.

И вот что интересно: мы оба вскоре убедились, что перевод Джеймса Фалена был на голову выше работы Чарлза Джонстона во всех возможных аспектах \--- течение стиха была более мелодичным, он был яснее и проще, ритм был более регулярным, рифмы \--- более точными. В целом, перевод Фалена был просто более артистичным. Мы с Кэроль просто влюбились в него и однажды сказали об этом няне наших детей.

Да, мы нашли няню для наших малышей, Дэнни и Моники; она приходила к нам несколько раз в неделю. К счастью, наша бэбиситтер оказалась замечательной. Марина была аспиранткой кафедры лингвистики Индианского университета, она была из России \--- и вскоре стала нашим другом. Мы быстро обнаружили, что Марина обладает весьма живым интеллектом. Она закончила филфак МГУ, чудесно говорила по-английски, знала испанский и французский, легко обыгрывала нас в шахматы, была остроумна и иронична и замечательно рисовала для детей фантастические сцены и сказочных зверей. Но вот что самое интересное: оказалось, что когда-то один из её друзей дал ей почитать несколько отрывков из «ГЭБ», после чего Марина стала большим поклонником этой книги. Однако она не подозревала, что её автор жил в том самом небольшом городке, куда она поступила в аспирантуру. Когда она обнаружила, что отец детей, к которым её взяли няней \--- автор «ГЭБ», она была в восторге. Поэтому нам показалось естественным поделиться с нашей умной и веселой бэбиситтер тем удовольствием, которое мы получали от чтения этого небольшого романа девятнадцатого столетия, написанного её соотечественником. Мы понятия не имели, читала ли Марина эту книгу, но надеялись, что она хотя бы слышала о ней. Как абсурдно мало знали мы о роли Пушкина в русской культуре!

В ответ на наши слова Марина спокойно и непринужденно заметила: «\enquote*{Евгений Онегин}? Я его в школе от начала до конца наизусть знала». «Как?» \--- воскликнули мы. «Разве это возможно?» \--- «А почему бы и нет?» \--- у нее это звучало как нечто само собой разумеющееся, \--- «Тогда голова у меня была пустой, так что это почти самой собой вышло. Да в этом и нет ничего особенного \--- стихи Пушкина у нас многие наизусть знают». Кэроль и я были поражены. Внезапно до нас дошло, что этот короткий, блистательный роман, который мы считали нашей собственной маленькой находкой, был, оказывается, любим миллионами людей на другой стороне планеты.

Через несколько месяцев мы с Кэроль и с детьми уехали в Италию, где я намеревался провести свой годовой академический отпуск. Мы надеялись, что для нашей семьи это будет чудесным годом, полным открытий, радости и красоты. К сожалению, случилось обратное. В декабре врачи нашли у Кэроль опухоль мозга, и на следующий день она впала в кому, из которой уже никогда не вышла. Через десять дней \--- всего лишь через три месяца после нашего прибытия в Италию \--- её не стало. Боль и отчаяние, испытанные миою и детьми, были, конечно, неописуемы. Однако, несмотря на эту трагедию, я поклялся провести год в Италии с детьми, как планировали мы с Кэроль. И мы сделали для этого всё от нас зависящее.

Летом 1994, когда мой академический отпуск подходил к концу, до меня дошли новости о Марине, также невеселые. Она переживала очень трудный период и была в глубокой депрессии. «Какой тяжелый год это был для всех нас», \--- подумал я. «Не могу ли я чем-нибудь помочь Марине?» И тут я вспомнил её увлечение «ГЭБ», её знание языков, любовь к литературе и, не в последнюю очередь, её врожденное чувство юмора \--- и внезапно меня осенило: почему бы не спросить Марину, не хочет ли она стать переводчиком «ГЭБ» на русский?

Эта идея пришла ко мне неожиданно и казалась совершенно сумасбродной: попросить няню своих детей перевести эту «непереводимую» книгу о математической логике, мозге, искусственном интеллекте, автореференции, молекулярной биологии и Бог знает, о чем ещё. Когда мы вернулись из Италии, и Марина пришла к нам в гости, я высказал ей свою безумную, взятую с потолка идею, и к моему удивлению она ответила: «Прекрасно. Я и сама хотела попросить тебя о том же». Таким образом, почва была подготовлена.

Я дал ей аннотированный экземпляр книги, и она с головой ушла в работу. В течение следующего года Марина самозабвенно трудилась над переводом, и мы иногда встречались, чтобы обсудить наиболее трудные места. Это было похоже на те чудесные беседы, которые я вел с французскими и другими переводчиками моей книги, беседы, полные увлекательных возможностей и творческой изобретательности. Именно тогда я полностью убедился в том, что моя интуиция меня не подвела и что я поступил мудро, попросив заняться этой сложнейшей работой Марину.

Далее, однако, мой рассказ становится ещё более запутанным, так как в следующие два года меня всё глубже затягивало в водоворот «Евгения Онегина». Сначала я прочитал ещё несколько переводов его на английский (ни один из них и близко не подходил к волшебному артистизму версии Джеймса Фалена). Затем я начал писать об этих переводах. Эти размышления позже стали двумя центральными главами в моей книге «Le Ton Beau de Marot» («Могила Maро»; в~оригинале игра слов. Французское «le ton beau» означает «прекрасное звучание», а фонетически это выражение эквивалентно слову «могила» \--- «lе tombeau». \--- \textbf{Прим.\ перев.}). Эта книга была посвящена искусству творческого литературного перевода; она была мотивирована, в значительной степени, моим участием в переводе «ГЭБ» на разные языки.

Может быть, в этот момент мое знакомство с оригиналом «ЕО» стало, наконец, неизбежным. Не знаю. Знаю только то, что уже подростком я был влюблен в русскую музыку и мне был близок дух русской культуры \--- я словно был настроен на ту же эмоциональную волну. Я всегда мечтал выучить русский, но всё не было подходящего момента. Несомненно, однако, что мое страстное увлечение «Евгением Онегиным» втягивало меня всё глубже в орбиту Пушкина и его родного языка.

Однажды в марте 1997, почти необъяснимо для меня самого, я взял мой русский экземпляр «ЕО» (я купил его много лет назад, но, как раньше переводы Фалена и Джонстона, он много лет простоял непрочитанный в моем шкафу), открыл страницу с письмом Татьяны и начал читать его вслух. Я тешил себя надеждой, что знаю, как произносятся слова; правда, большинства из них я не понимал. Оказалось, однако, что я читаю ужасно. С помощью нашей с Мариной общей подруги Ариадны Соловьевой я стал произносить слова более или менее правильно и вскоре, как самолет на взлетной полосе, мои занятия русским начали набирать скорость.

Я перечитывал письмо Татьяны вслух снова и снова, и сам не заметил, как стал запоминать целые куски. Я совершенно не собирался делать ничего подобного, но тут я вспомнил Марину, в юности заучившую всего «Онегина» наизусть, и сказал себе: «Самое меньшее, что ты можешь сделать \--- выучить наизусть хотя бы этот центральный кусок». И через две недели уже знал письмо Татьяны наизусть.

Но это было ещё не всё. Вновь вдохновившись Марининым достижением, я решил выучить мои любимые строфы «ЕО». Я разыскал их в переводе Фалена, потом в русском тексте, и начал читать их вслух много раз подряд. Таким образом, в течение нескольких месяцев, в моей памяти оседала строфа за строфой. В один прекрасный день я осознал, что наконец научусь говорить на этом прекрасном, давно манившем меня языке. Дорога, избранная мной, была непохожа на тот путь, которым обычно идут иностранцы. Я карабкался по крутым ступеням русского языка, заучивая большие куски самого почитаемого в русской литературе произведения!

К сентябрю 1997 года я выучил наизусть около пятидесяти строф «Евгения Онегина». Память у меня неважная, и это было для меня огромным усилием \--- и~всё же это было волшебно прекрасно. Однажды, охваченный внезапной любовью к трем строфам, над которыми я тогда работал (VII. 1-3). Я решил, просто ради забавы, попытаться перевести их на английский. Я не смотрел ни в Фалена, ни в Джонстона, ни в какой-другой из существующих переводов. Я просто сел и начал переводить их прямо с подлинника, и, к моему удивлению, стихи полились легко и непринужденно. Разумеется, мои первые попытки перевода не были отшлифованы как следует, но в них было некое обещание. Несколько недель спустя я попробовал перевести ещё пару строф. Вы можете догадаться, к чему шло дело \--- но сам я ни о чем не догадывался. Я не видел пророческих слов на стене, не подозревал, что скоро погружусь с головой в самые тесные отношения с «Евгением Онегиным», не считая самого Пушкина \--- иными словами, что я буду переводить этот роман с начала до конца.

Только в начале 1998 года у меня появилась мысль перевести весь роман.
«Зачем?» \--- можете вы спросить. «Зачем переводить книгу, которая уже была переведена так хорошо, как только возможно?» Мой ответ прост: это делается из любви. И любовь эта как раз и рождается из восхищения другими переводами. Таким образом, один переводчик вдохновляется другим на тот же самый труд не из-за соперничества, но из чистого восхищения. То же самое происходит и с музыкой вы слышите запись великого музыканта, играющего какое-либо произведение, и оно вам так нравится, что вы хотите сыграть его сами. Играя, вы отдаете должное тому исполнителю, чья игра заставила вас влюбиться в эту музыку. Так случилось и со мной, восхищенным слушателем пушкинского шедевра в гениальном исполнении Фалена.

Кульминация странной саги о моем переводе «ЕО» приходится на октябрь 1998, когда я впервые приехал в Россию. К тому времени я перевел уже всю книгу, кроме трех завершающих строф восьмой главы. Я планировал закончить эту работу в Петербурге. Две первые строфы я перевел в гостиничном номере, а затем, в восхитительно романтическом кульминационном пункте моего «романа» с «Евгением Онегиным», перевел последнюю строфу книги (Но~те, которым в дружной встрече\ldots) в~квартире самого Пушкина на Мойке, где мне любезно предоставили разрешение провести в его кабинете два часа в одиночестве. Это было незабвенной возможностью завершить мой труд любви, и в начале следующего года, как раз к двухсотлетию со дня рождения Пушкина, мой перевод вышел из печати. Думаю, что читатели этого предисловия оценят тот факт, что мой перевод открывался поэмой-посвящением Джеймсу Фалену и его жене Еве.

Почему я вам это рассказываю? Какое отношение имеет всё это к русскому «ГЭБ»? Думаю, что очень большое. Та самая Марина Эскина, чья детская любовь к Пушкину подвигла меня, четверть века спустя, на заучивание письма Татьяны и затем ещё десятков строф, стала переводчиком моей книги «Гёдель, Эшер, Бах» на русский язык. Моя книга разделяет с романом Александра Пушкина то же необычное художественное качество такого тесного переплетения формы и содержания, что многие считают эти книги классическим примером непереводимости. Я разделяю с Мариной то же утонченное удовольствие воссоздания \--- каждый на своем родном языке \--- некоей кристально точной структуры, первоначально созданной на родном языке другого. В Маринином случае это было движение от английского к русскому, в моем, разумеется, наоборот. Но в обоих случаях святым, нерушимым принципом оставалось внимание одновременно и к форме, и к содержанию. При этом мы оба были готовы изменить букве, чтобы сохранить дух.

Мне бы хотелось завершить это предисловие примером моего и Марининого стилей перевода, по одному примеру в каждом направлении. Сначала позвольте показать вам, как я справился с переводом строфы IV.~42~«ЕО».

\settowidth{\versewidth}{Читатель ждет уж рифмы «розы»,}
\begin{verse}[\versewidth]
    И вот уже трещат морозы \\
    И серебрятся средь полей... \\
    (Читатель ждет уж рифмы «розы», \\
    На, вот возьми её скорей!) \\
    Опрятней модного паркета \\
    Блистает речка, льдом одета. \\
    Мальчишек радостный народ \\
    Коньками звучно режет лед, \\
    На красных лапках гусь тяжелый \\
    Задумав плыть по лону вод, \\
    Ступает бережно на лед, \\
    Скользит и падает, веселый \\
    Мелькает, вьется первый снег, \\
    Звездами падая на брег.
\end{verse}

Я выбрал в качестве примера именно эту строфу из-за шутки, которую Пушкин обращает к читателям в строчках 3-4. Он отходит в сторону от описываемой сцены и прямо упоминает своего читателя и свою рифму. Что может сделать переводчик с этой шалостью автора? Я подумал, что если Пушкин отважился на такое, то почему бы и мне, переводчику, не сделать то же самое и не упомянуть не только моих читателей и мою рифму, но заодно и автора, и самого себя! Вот мое переложение этой строфы на английский:

%% For debugging paracol:
% \backgroundcolor{c[0]}[rgb]{1,0.8,1} % pink for colunmn-0
% \backgroundcolor{c[1]}[rgb]{1,1,0.8} % cream yellow for column-1

\columnratio{0.47}
\setlength\columnsep{1em}
\begin{paracol}{2}
\footnotesize

\setlength{\vleftmargin}{0pt}
\begin{verse}
    Frost's crackling, too, but still she's cozy \\
    Amidst the fields' light silv'ry dust\ldots \\
    (You're all supposing I'll write «rosy», \\
    As Pushkin did \--- and so I must!) \\
    Slick as a dance parquet swept nicely \\
    The brooklet glints and glistens icily. \\
    A joyous band of skate-shod boys \\
    Cuts graceful ruts to rowdy noise. \\
    A clumsy goose, by contrast, wishing \\
    To swim upon the glassy sheet, \\
    Lands stumbling \\>
        on its red webbed feet, \\
    And slips and tumbles. Swirling, swishing, \\
    Gay twinkling stars \--- \\>
        the snow's first try \--- \\
    Bedaub the creekside ere they die.
\end{verse}

\switchcolumn

\begin{verse}
    Ей всё ещё уютно, хоть трещат морозы, \\
    Поля покрыты легкой серебряной пылью\ldots \\
    (Вы всё ожидаете, что я напишу «розы», \\
    Как у Пушкина \--- придется так и сделать!) \\
    Гладкая, как подметенный для танцев паркет, \\
    Речка сверкает и искрится ледяным блеском. \\
    Радостная толпа мальчишек, надев коньки, \\
    Шумно режет изящные дорожки. \\
    Наоборот, неуклюжий гусь, \\
    Задумав плыть по ледяному полю, \\
    Приземляется, спотыкаясь, \\>
        на красные перепончатые лапы \\
    Скользит и шлепается. Кружась и шелестя, \\
    Веселые мерцающие звезды \--- \\>
        первая попытка снега \--- \\
    Украшают берег реки перед тем, как умереть.
\end{verse}

\end{paracol}

Вы, конечно, заметили, что для женской рифмы в третьей строке я использовал те же слоги, что и Пушкин. Я даже думал, не написать ли слово «РОЗЫ» кириллицей, чтобы подчеркнуть идентичность моей и пушкинской рифм, но отказался от этой мысли, поскольку мало кто из моих читателей знает кириллицу (и среди них нет почти никого, кто понял бы мою шутку).

В моем переводе есть одно необычное место, которое стоит прокомментировать \--- я имею в виду конец двух последних строк. Почему я говорю о снеге, который умирает, едва коснувшись земли, хотя в оригинале подобных образов недолговечности нет? Хотите верьте, хотите \--- нет, но я защищал честь Пушкина. Вы сомневаетесь? Тогда вспомните знаменитые начальные строки пятой главы:

\settowidth{\versewidth}{Зимы ждала, ждала природа,}
\begin{verse}[\versewidth]
    В тот год осенняя погода \\
    Стояла долго на дворе, \\
    Зимы ждала, ждала природа, \\
    Снег выпал только в январе \\
    На третье в ночь\ldots
\end{verse}

Теперь скажите мне, пожалуйста, когда же в тот год выпал самый первый снег? В пятой главе недвусмысленно говорится, что это произошло только в январе, в то время как действие 42 строфы четвертой главы происходит на месяц или два раньше. Неужели наш великий Александр Сергеевич сам себе противоречит? Кажется, так оно и есть! Как его верный почитатель и исполнительный служитель, я почувствовал, что должен поспешить ему на помощь и примирить эти две строфы. Думаю, что Пушкину понравился бы мой поступок. Вы согласны?

Почему я об этом пишу? В том числе и потому, что хочу показать, насколько непредсказуем может быть процесс творческого перевода, особенно в тех случаях, когда форма и содержание так интимно связаны, как в поэзии или в словесных играх. И эта мысль подводит меня к рассказу о творческой работе Марины над переводом «ГЭБ». Для примера я выбрал крохотный, но довольно забавный пассаж из «Маленького гармонического лабиринта», одного из 21 Диалогов Ахилла и Черепахи. (Этими шутливыми диалогами прослоены главы книги). Данный Диалог включает одновременно несколько историй, вложенных одна в другую, и действие постоянно перескакивает между ними. Среди прочего, это аллегория «рекурсии» в информатике, где слова «push» и «рор» используются как технические термины, обозначающие, соответственно, переход на один уровень вниз и возвращением на один уровень вверх.

Так вот, в самой «глубокой» истории есть один момент, когда Черепаха находит плошку с попкорном (POPcorn), и Ахилл торопится его съесть, надеясь, что это вытолкнет их из данной истории, в которой они ухитрились попасть в передрягу. За секунду перед тем, как герои глотают первую порцию, в «обрамляющей» истории уровнем выше Черепаха бросает каламбурную реплику, имея в виду некую гипотетическую пищу, похожую на попкорн, но обратную по свойствам: «Надеюсь, что это не пушкорн! (PUSHcorn)». Хотя теоретически действие на разных уровнях происходит в совершенно отдельных мирах (хотя персонажи в них одни и те же), в этом месте происходит небольшая утечка, и до Ахилла нижнего уровня долетает каламбур Черепахи высшего уровня. Он спрашивает свою спутницу: «Что вы сказали про Пушкина?» \--- на что Черепаха нижнего уровня с невинным видом отвечает: «Ничего \--- вам, наверное, послышалось».

Посмотрим, с какими проблемами здесь пришлось столкнуться переводчику. Прежде всего, здесь есть понятия «pushing» и «popping», которые Марина совершенно справедливо перевела как «проталкивание» и «выталкивание». Перейдем теперь к счастливой находке Черепахи \--- плошке с попкорном, каламбуру черепашьей тезки с высшего уровня, превращающему это слово в «PUSHcorn» и, наконец, ошибке Ахилла, услышавшего этот неологизм как «Пушкин». Как здесь быть переводчику? Начнем с того, что упомянутый каламбур зависит от слова «рор» как части названия популярного в Америке кушанья. В России нет никакой еды, в название которой входило бы существительное «выталкивание» или глагол «вытолкнуть». Казалось бы, Марина и сама попала тут в хорошую передрягу.

Тем не менее, Марина, со свойственной ей ловкостью, сумела выкрутиться. Она заменила «выталкивание» на близкое по смыслу «вытаскивание», а плошку с попкорном \--- на бутылочку косметического лосьона «Vitaskin». Герои Диалога произносят это непонятное английское название на русский лад \--- вытаскин. Заметьте, что Марине удалось-таки построить необходимый звуковой мостик между понятием «чего-то съедобного» и понятием «вытаскивания». Но разве лосьон можно пить? Ничего удивительного \--- такой неотесанный солдафон, как Ахилл, думает, что всё, что налито в бутылку, можно выпить! (Американский автор явно не знаком с классическим трудом Венички Ерофеева! \--- \textbf{Прим.\ перев.})

Таким образом, когда Черепаха находит бутылочку «вытаскина», Ахилл хочет её тут же выпить \--- не только для того, чтобы залить жажду, но и чтобы волшебным образом оказаться вытащенным из той опасной ситуации, в которой они находятся. В этот момент Черепаха с далекого высшего уровня роняет свой каламбур, имея в виду гипотетический напиток, похожий на вытаскин, но обратный по свойствам: «Надеюсь, что это не протолкин!». Тут она, точно так же, как и в английском варианте каламбура, переходит от понятия «вытаскивания» к идее «проталкивания».

И именно тут проявляется поразительное Маринино чутье: реплика Черепахи просачивается на нижний уровень к Ахиллу и тот, не расслышав, замечает: «Что вы сказали про Толкиена?» Внезапно в русском Диалоге неизвестно откуда появляется имя знаменитого английского автора, совершенно так же, как в английском Диалоге неизвестно откуда появляется имя знаменитого русского автора!
Это была поистине гениальная находка!

Разумеется, чтобы оценить, как органично это звучит в контексте, надо прочесть весь Диалог. В том же Диалоге вы найдете десятки других примеров игры слов, каждый из которых был творчески переведен Мариной. Тут не скажешь «реконструирован», поскольку зачастую ей приходилось придумывать совершенно иные, оригинальные каламбуры. Видимо, для того, чтобы верно передать особенности Марининого перевода моей книги, лучше всего подходят слова «вновь изобрела».

Сейчас я почти так же далек от Дага, написавшего «ГЭБ», как он сам был далек от малыша Дагги, ломавшего голову над загадкой трех троек \--- поэтому я чувствую, что в какой-то мере являюсь в этой книге незваным гостем. Конечно, это не совсем так, но всё же я не уверен, сколько драгоценного читательского времени я имею право занять. Скорей всего, я уже и так потратил его слишком много, так что пора предоставить слово серединному Дагу, находящемуся примерно на полпути между мной и малышом Дагги. Ура! Давно~б (не~правда~ли?) пора! (В~оригинале по-русски. \--- \textbf{Прим.\ перев.}).

Итак, я ретируюсь \--- но напоследок хочу рассказать вам замечательный эпизод, о котором мне напомнила недавно сама Марина. Это случилось несколько лет тому назад, вскоре после того, как она закончила перевод «ГЭБ». Мы стояли во дворике моего дома, и она говорила мне, какую важную роль эта работа сыграла в её жизни. Вот что она сказала: «I'm eternally grateful to you for this». («Я навечно благодарна тебе за это»). Тут она заметила, что я гляжу поверх её головы остекленевшим взором.

«Что случилось, Дуг?» \--- спросила Марина.

«Я ищу третье слово», \--- ответил я.

«Какое третье слово?»

«Ты сказала, что ты \enquote*{eternally grateful}. Это дает нам \enquote*{E} и \enquote*{G} \--- остается отыскать слово, начинающееся с \enquote*{B}, но мне почему-то ничего не приходит в голову».

«Нет ничего проще!» \--- улыбнулась Марина: «Babysitter!»

Итак, я отхожу в сторону, чтобы дать моим русским читателям возможность насладиться блестящей переводческой интуицией и живым юмором нашей «вечно благодарной няни» («Eternally Grateful Babysitter»), которые сверкают и искрятся, вдыхая жизнь в страницы русского «ГЭБ».

\medskip

Счастливого пути!

\end{document}
