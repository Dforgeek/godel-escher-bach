\documentclass[../main.tex]{subfiles}
\begin{document}

\DialogueChapter{\ldots~и Муравьиная фуга}

\centerblock{
    \emph{\ldots тут, один за другим, вступают четыре голоса фуги)}
}

\begin{dialogue}

\speak{Ахилл} Вы не поверите, но ответ на этот вопрос --- прямо у вас перед носом: он спрятан в картинке. Это всего лишь одно слово, но преважное: «МУ»!

\emph{Краб} : Вы не поверите, но ответ на этот вопрос --- прямо у вас перед носом: он спрятан в картинке. Это всего лишь одно слово, но преважное: «ХОЛИЗМ»!

\emph{Ахилл} : Погодите-ка\ldots{} вам, наверное, почудилось. Ясно, как день, что на картине написано «МУ», а не «ХОЛИЗМ».

\emph{Краб} : Прошу прощения, но у меня отличное зрение. Взгляните-ка еще раз, прежде чем говорить, что на картинке нет моего слова.

\emph{Муравьед} : Вы не поверите, но ответ на этот вопрос --- прямо у вас перед носом: он спрятан в картинке. Это всего лишь одно слово, но преважное: «РЕДУКЦИОНИЗМ»!

\emph{Краб} : Погодите-ка\ldots{} вам, наверное, почудилось. Ясно, как день, что на картине написано «ХОЛИЗМ», а не «РЕДУКЦИОНИЗМ».

\emph{Ахилл} : Еще один фантазер! На картинке написано не «ХОЛИЗМ» и не «РЕДУКЦИОНИЗМ», а «МУ» --- в этом я совершенно уверен!

\emph{Муравьед} : Прошу прощения, но у меня великолепное зрение. Взгляните-ка еще раз, прежде чем говорить, что на картинке нет моего слова.

\emph{Ахилл} : Вы что, не видите, что картинка состоит из двух частей, и каждая из них --- одна буква?

\emph{Краб} : Вы правы насчет двух частей, но в остальном вы ошибаетесь. Левая часть состоит из трех копий одного и того же слова --- «ХОЛИЗМ», а правая часть --- из многих маленьких копий того же слова. Не знаю, почему буквы в одной части больше, но то, что передо мной «ХОЛИЗМ», ясно как день!

\emph{Муравьед} : Вы правы насчет двух частей, но в остальном вы ошибаетесь. Левая часть состоит из многих маленьких копий одного и того же слова: «РЕДУКЦИОНИЗМ», а правая --- из того же слова, написанного большими буквами. Не знаю, почему буквы в одной части больше, но то, что передо мной «РЕДУКЦИОНИЗМ», ясно как день! Не понимаю, как здесь можно увидеть что-либо иное.

\emph{Ахилл} : Я понял, в чем здесь дело. Каждый из вас видит буквы, которые либо составляют другие буквы, либо сами из них состоят. В левой части действительно есть три «ХОЛИЗМА», но каждый из них состоит из маленьких копий слова «РЕДУКЦИОНИЗМ». И наоборот, «РЕДУКЦИОНИЗМ» в правой части составлен из маленьких копий слова «ХОЛИЗМ». Все это замечательно, но пока вы ссорились из-за пустяков, вы оба пропустили самое главное, не увидев за деревьями леса. Что толку спорить о том, что правильно, --- «ХОЛИЗМ» или «РЕДУКЦИОНИЗМ», --- когда гораздо лучше взглянуть на дело извне, ответив «МУ».

\emph{Краб} : Теперь я вижу картинку так, как вы её описали, Ахилл, --- но что вы подразумеваете под этим странным выражением «взглянуть на дело извне»?

\emph{Муравьед} : Теперь я вижу картинку так, как вы её описали, Ахилл, --- но что вы подразумеваете под этим странным выражением «МУ»?

\emph{Ахилл} : Буду счастлив вас просветить, если вы будете так любезны и скажете мне, что значат эти странные выражения, «ХОЛИЗМ» и «РЕДУКЦИОНИЗМ».

\emph{Краб} : Нет ничего проще ХОЛИЗМА Это всего-навсего означает, что целое больше, чем сумма его частей. Ни один человек, если он в здравом уме, не может отрицать холизма.

\emph{Муравьед} : Нет ничего проще РЕДУКЦИОНИЗМА. Это всего-навсего означает, что целое может быть полностью понято, если вы понимаете его части, и природу их «суммы». Ни один человек, если он в твердой памяти, не может отрицать редукционизма.

\emph{Краб} : Я отрицаю редукционизм. К примеру, можете ли вы объяснить мне, как понять мозг с помощью редукционизма? Любое редукционистское описание мозга неизбежно столкнется с трудностями, пытаясь объяснить, откуда в мозгу берется сознание.

\emph{Муравьед} : Я отрицаю холизм. К примеру, можете ли вы объяснить мне, как холистское описание муравьиной колонии может помочь понять её лучше, чем описание отдельных муравьев, их взаимоотношений и ролей внутри колонии. Любое холистское описание муравьиной колонии неизбежно столкнется с трудностями, пытаясь объяснить, откуда в ней берется сознание.

\emph{Ахилл} : О, нет! Я вовсе не хотел быть причиной еще одного спора. Я понимаю, в чем суть несогласия, но думаю, что мое объяснение «МУ» вам поможет. Видите ли, «МУ» --- это старинный ответ дзен-буддизма, «развопросивающий» вопрос. Нашим вопросом было: «Должны ли мы понимать мир холистским или редукционистским способом?» Ответ «МУ» отрицает самую постановку этого вопроса --- предположение, что необходимо выбрать лишь один из двух способов. Развопросивая этот вопрос, «МУ» открывает нам истину высшего порядка: существует более широкий контекст, куда вписываются и холистский и редукционистский подходы.

\emph{Муравьед} : Чепуха! В вашем «МУ» не больше смысла, чем в коровьем мычаньи. Я не собираюсь глотать эту буддистскую бурду.

\emph{Краб} : Чушь! В вашем «МУ» не больше смысла, чем в кошачьем мяукании. Я не намерен слушать эту буддистскую белиберду.

\emph{Ахилл} : Ах, боже мой. Так мы ни до чего не дойдем. Почему вы все молчите, г-жа Черепаха? Это меня нервирует. Вы-то наверняка знаете, как распутать этот клубок!

\emph{Черепаха} : Вы не поверите, но ответ на этот вопрос --- прямо у вас перед носом: он спрятан в картинке. Это всего лишь одно слово, но преважное: «МУ»!

\emph{(В этот момент вступает четвертый голос фуги, точно на октаву ниже первого голоса.)}

\emph{Ахилл} : Эх, г-жа Ч, на этот раз вы меня разочаровали. Я был уверен, что вы, с вашей проницательностью, сможете разрешить эту дилемму --- но, к сожалению, вы увидели ничуть не больше меня. Что же делать, --- наверное, я должен быть счастлив, что хотя бы один раз мне удалось увидеть столько же, сколько и г-же Черепахе.

\emph{Черепаха} : Прошу прощения, но у меня превосходное зрение. Взгляните-ка еще раз, прежде чем говорить, что на картинке нет моего слова.

\emph{Ахилл} : Разумеется --- вы просто повторили мою первоначальную идею.

\emph{Черепаха} : Может быть, «МУ» существует на картине на более глубоком уровне, чем вам кажется, Ахилл --- образно говоря, на октаву ниже. Но я сомневаюсь, что мы сумеем разрешить наш спор таким абстрактным способом. Я хотела бы увидеть обе точки зрения, и холистскую, и редукционистскую, выраженные более конкретно. Тогда у нас будет больше оснований для решения вопроса --- хотя бы на примере редукционистского описания муравьиной колонии.

\emph{Краб} : Может быть, д-р~Муравьед поделится своим опытом на этот счет? Благодаря своей профессии, он должен быть экспертом по этой теме.

\emph{Черепаха} : Я уверена, что нам есть чему у вас поучиться, д-р~Муравьед. Можете ли вы нас просветить, рассказав нам, что собой представляет муравьиная колония с редукционистской точки зрения?

\emph{Муравьед} : С удовольствием. Как уже говорил Краб, моя профессия позволила мне весьма глубоко понять муравьиные колонии.

\emph{Ахилл} : Представляю себе! Любой муравьед должен быть экспертом по муравьиным колониям.

\emph{Муравьед} : Прошу прощения: муравьед --- это не моя профессия, это мой класс. По профессии я колониальный хирург. Я специализируюсь в излечении нервных расстройств колоний путем хирургического вмешательства.

\emph{Ахилл} : Вот оно что\ldots{} Но что вы имеете в виду под «нервными расстройствами» в муравьиной колонии?

\emph{Муравьед} : Большинство моих пациентов страдает каким-либо расстройством речи. Представьте себе колонии, которым приходится каждый день мучиться в поисках нужного слова. Это может быть довольно трагично. Я пытаюсь исправить ситуацию путем\ldots{} э-э-э\ldots{} удаления пораженной части колонии. Эти операции иногда бывают очень сложными и приходится учиться годами прежде чем приняться за их выполнение.

\emph{Ахилл} : Но\ldots{} мне кажется, что чтобы страдать расстройством речи, сначала необходимо иметь дар речи?

\emph{Муравьед} : Совершенно верно.

\emph{Ахилл} : Поскольку у муравьиных колоний нет дара речи, должен признаться, что я слегка сбит с толка.

\emph{Краб} : Жаль, Ахилл, что вас здесь не было на прошлой неделе, когда у меня в гостях вместе с д-ром Муравьедом была г-жа Мура Вейник. Надо было пригласить и вас.

\emph{Ахилл} : Кто такая эта Мура Вейник?

\emph{Краб} : Писательница и моя старая знакомая.

\emph{Муравьед} : Она всегда настаивает, чтобы все звали её полным именем (и это вовсе не псевдоним!); это одна из её забавных причуд.

\emph{Краб} : Верно, Мура Вейник --- особа эксцентрическая, но при этом она так мила\ldots{} Какая жалость, что вы с ней не встретились.

\emph{Муравьед} : Она, безусловно, одна из самых образованных муравьиных колоний, с которыми я когда-либо имел счастье общаться. Мы с ней скоротали множество вечеров, беседуя на самые разнообразные темы.

\emph{Ахилл} : Я-то думал, муравьеды --- пожиратели муравьев, а не покровители колоний!

\emph{Муравьед} : На самом деле, одно не исключает другого. Я в самых лучших отношениях с муравьиными колониями. Я ем всего лишь МУРАВЬЕВ, и это приносит пользу как мне, так и колонии.

\emph{Ахилл} : Как это возможно, чтобы ---

\emph{Черепаха} : Как это возможно, чтобы ---

\emph{Ахилл} : --- пожирание её муравьев шло колонии на пользу?

\emph{Краб} : Как-это возможно, чтобы ---

\emph{Черепаха} : --- лесной пожар пошел лесу на пользу?

\emph{Ахилл} : Как это возможно, чтобы ---

\emph{Краб} : --- прореживание ветвей шло дереву на пользу?

\emph{Муравьед} : --- стрижка волос пошла Ахиллу на пользу?

\emph{Черепаха} : Наверное, вы были так поглощены спором, что пропустили мимо ушей прелестную стретту, только что прозвучавшую в Баховской фуге.

\emph{Ахилл} : Что такое стретта?

\emph{Черепаха} : Ох, извините пожалуйста: я думала, вы знакомы с этим термином. Стретта --- это когда голоса вступают почти сразу один за другим, исполняя одну и ту же тему.

\emph{Ахилл} : Если я буду слушать фуги часто, вскоре я буду знать все эти штуки и смогу услышать их сам, без подсказки.

\emph{Черепаха} : Простите, что перебила, друзья мои. Д-р~Муравьед как раз пытался объяснить, как, поедая муравьев, можно при этом оставаться другом муравьиной колонии.

\emph{Ахилл} : Что ж, я могу, пожалуй, кое-как понять, как подъедание некоторого ограниченного количества муравьев может быть полезным для здоровья всей колонии --- но что действительно приводит меня в замешательство, это рассказы о разговорах с муравьиными колониями. Это невозможно! Муравьиная колония --- это не более, чем куча муравьев, снующих туда и сюда в поисках еды и строящих себе гнезда.

\emph{Муравьед} : Можете считать так, если вы настаиваете на том, чтобы смотреть на деревья, но не видеть за ними леса. На самом деле, муравьиная колония, видимая как одно целое, --- это весьма определенная единица, с собственными качествами, иногда включающими владение речью.

\emph{Ахилл} : Трудно представить, что если я закричу где-нибудь в лесу, то в ответ до меня донесется голос муравьиной колонии.

\emph{Муравьед} : Не говорите глупостей, мой друг, это происходит совсем не так. Муравьиные колонии не говорят вслух, они общаются письменно. Вы, наверное, видели, как муравьи прокладывают тропы туда и сюда?

\emph{Ахилл} : Да, конечно; обычно они ведут из кухонной раковины прямиком в мой любимый торт «Птичье молоко».

\emph{Муравьед} : Оказывается, что некоторые такие тропы содержат закодированную информацию. Если вы знаете эту систему, вы можете читать то, что они говорят, словно книгу.

\emph{Ахилл} : Потрясающе. И вы можете, в свою очередь, что-нибудь им сообщить?

\emph{Муравьед} : Без проблем. Именно так Мура Вейник и я беседуем друг с другом часами. Я беру прутик, черчу на влажной земле тропинки и смотрю, как муравьи по ним направляются. Вдруг где-то начинает формироваться новая тропинка\ldots{} я получаю большое удовольствие, наблюдая за их появлением. Я пытаюсь предсказать, в каком направлении пойдет та или иная тропа (мои предсказания по большей части бывают ошибочны). Когда тропа заканчивается, я знаю, о чем думает Мура Вейник, и тогда я, в свою очередь, могу ей ответить.

\emph{Ахилл} : Ручаюсь, что в этой колонии есть необыкновенно умные муравьи.

\emph{Муравьед} : По-моему, вы еще не научились видеть различие между уровнями. Подобно тому, как вы не спутаете отдельное дерево с лесом, вы не должны принимать отдельного муравья за всю колонию. Видите ли, Мура Вейник состоит из массы муравьев, каждый из которых глуп как пробка. Они не смогли бы разговаривать даже ради спасения своих жалких хитиновых покровов!

\emph{Ахилл} : В таком случае, откуда берется это умение беседовать? Оно должно находиться где-то внутри колонии! Не понимаю, как это получается: все муравьи глупы как пробка, а Мура Вейник часами занимает вас своей остроумной беседой.

\emph{Черепаха} : Мне кажется, что эта ситуация напоминает человеческий мозг, состоящий из нейронов. Чтобы объяснить человеческую способность к разумной беседе, никто не стал бы утверждать, что отдельные нервные клетки --- разумные существа.

\emph{Ахилл} : Разумеется, нет. Тут вы совершенно правы. Но мне кажется, что муравьи --- это совершенно из другой оперы. Они снуют туда и сюда по собственному желанию, совершенно беспорядочно, иногда натыкаясь на съедобный кусочек\ldots{} Они вольны делать все, что им угодно. Из-за этой свободы я совершенно не понимаю, как может их поведение в целом порождать нечто осмысленное, сравнимое с поведением мозга, необходимым для беседы.

\emph{Краб} : Я думаю, что муравьи свободны только до определенных пределов. Например, они свободны бродить где угодно, трогать друг друга, строить тропинки, поднимать небольшие предметы и так далее. Но они никогда не выходят из этого ограниченного мирка, так сказать, мура-системы, в которой они находятся. Это им никогда не пришло бы в голову, так как у них для этого не хватает ума. Таким образом, муравьи --- весьма надежные компоненты, в том смысле, что они всегда делают определенные вещи определенным образом.

\emph{Ахилл} : И~всё~же, внутри этих пределов они остаются свободными и бегают без толку, не выказывая никакого уважения к мыслительным процессам существа высшего порядка, составными частями которого они, по утверждению д-ра Муравьеда, являются.

\emph{Муравьед} : Да, но вы, Ахилл, упускаете из вида одну вещь: регулярность статистики.

\emph{Ахилл} : Как это?

\emph{Муравьед} : Хотя отдельные муравьи снуют туда-сюда беспорядочно, тем не менее из этого хаоса можно выделить общие тропы, по которым идет большое количество муравьев.

\emph{Ахилл} : Понятно. Действительно, муравьиные тропы --- отличный пример этого явления. Хотя движения каждого отдельного муравья непредсказуемы, сама тропа выглядит весьма постоянной и определенной. Безусловно, это означает, что на самом деле муравьи движутся не так уж хаотично.

\emph{Муравьед} : Точно, Ахилл. Муравьи сообщаются между собой достаточно, чтобы внести в их движение некоторую упорядоченность. При помощи этой минимальной связи они напоминают друг другу, что они --- части одного целого и должны сотрудничать с товарищами по команде. Чтобы выполнить любую задачу, такую, например, как прокладывание тропинок, требуется множество муравьев, передающих то же сообщение друг другу в течении определенного времени. Хотя мое понимание того, что происходит в мозгу, весьма приблизительно, я предполагаю, что нечто подобное может происходить при сообщении нейронов. Не правда ли, м-р Краб, что необходимо несколько нервных клеток, передающих сигнал другому нейрону, чтобы тот, в свою очередь, передал тот же сигнал?

\emph{Краб} : Совершенно верно. Возьмем, к примеру, нейроны в мозгу у Ахилла. Каждый из них принимает сигналы от нейронов, присоединенных к их «входу», и если сумма этих сигналов в какой-то момент превышает критический порог, то нейрон посылает свой собственный сигнал, идущий к другим нейронам, которые в свою очередь, могут «возбудиться»\ldots{} и так далее, и тому подобное. Нейронный луч устремляется, неутомимый, по Ахиллесовой тропе, по маршруту более причудливому, чем погоня голодной ласточки за комаром. Каждый поворот и изгиб определяется нейронной структурой Ахиллова мозга, пока не вмешиваются новые послания от органов чувств.

\emph{Ахилл} : Я-то думал, что сам осуществляю контроль над своими мыслями --- но ваше объяснение ставит все с ног на голову, так что теперь мне кажется, что «Я» --- это лишь результат комбинации всей этой нейронной структуры с законами природы. Получается, что то, что я считал «СОБОЙ» --- это, в лучшем случае, побочный продукт организма, управляемого законами природы, а в худшем случае, искусственное понятие, порожденное неверной перспективой. Иными словами, после вашего объяснения я уже не уверен, кто я такой (или что я такое).

\emph{Черепаха} : Чем больше мы беседуем, тем лучше вы это будете понимать. Д-р~Муравьед, а что вы думаете об этом сходстве?

\emph{Муравьед} : Я подозревал, что в этих разных системах происходят похожие процессы; теперь я гораздо лучше понимаю, в чем дело. По-видимому, осмысленные групповые явления, такие, например, как прокладывание тропинок, начинают происходить только тогда, когда достигается определенное критическое количество муравьев. Когда несколько муравьев собираются вместе и начинают, может быть, чисто случайно, прокладку тропы, может произойти одно из двух: либо после короткого хаотического старта их деятельность быстро сойдет на нет ---

\emph{Ахилл} : Когда муравьев собирается недостаточно, чтобы продолжать тропу?

\emph{Муравьед} : Именно так. Однако может случиться и так, что количество муравьев достигнет критической массы и начнет расти, как снежный ком. В этом случае, возникает целая «команда», работающая над одним проектом. Это может быть прокладка тропы, или поиски пищи, или ремонт муравейника. Несмотря но то, что в малом масштабе эта схема чрезвычайно проста, в большом масштабе она может привести к весьма сложным последствиям.

\emph{Ахилл} : Я могу понять общую идею порядка, по вашим словам, возникающего из хаоса, но это еще очень далеко от умения беседовать. В конце концов, порядок возникает из хаоса и тогда, когда молекулы газа беспорядочно сталкиваются друг с другом --- и результатом этого бывает лишь аморфная масса, характеризуемая всего тремя параметрами: объем, давление и температура. Это очень далеко от умения понимать мир и о нем разговаривать!

\emph{Муравьед} : Это подчеркивает весьма важную разницу между объяснением поведения муравьиной колонии и поведения газа в контейнере. Поведение газа можно объяснить, рассчитав статистические особенности движения его молекул. При этом не требуется обсуждать никаких высших, чем молекулы, элементов его структуры, кроме самого газа целиком. С другой стороны, в случае муравьиной колонии невозможно понять происходящие там действия без анализа нескольких уровней её структуры.

\emph{Ахилл} : А, теперь понимаю. В случае газа, всего один шаг переносит нас с низшего уровня --- молекулы --- на высший уровень --- сам газ. Там нет промежуточных уровней. Но как возникают промежуточные уровни организованного действия в муравьиной колонии?

\emph{Муравьед} : Это связано с тем, что в колонии есть несколько разных типов муравьев.

\emph{Ахилл} : Кажется, я что-то об этом слышал. Это называется «касты», да?

\emph{Муравьед} : Верно. Кроме царицы-матки, там есть самцы-трутни, совершенно не занимающиеся работой по поддержанию муравейника, и еще~---

\emph{Ахилл} : И, разумеется, там есть воины --- Славные Борцы Против Коммунизма!

\emph{Краб} : Гм-м-м\ldots{} В этом-то я сомневаюсь, Ахилл. Муравьиная колония весьма коммунистична по своей структуре, так что её солдатам незачем бороться против коммунизма. Правильно, д-р~Муравьед?

\emph{Муравьед} : Да, насчет колоний вы правы, м-р Краб: они действительно основаны на принципах, смахивающих на коммунистические. Но Ахилл в своих представлениях о солдатах весьма наивен. На самом деле, так называемые «солдаты» едва умеют сражаться. Это медлительные неуклюжие муравьи с гигантскими головами; они могут цапнуть своими мощными челюстями, но прославлять их не стоит. Как в настоящих коммунистических государствах, прославлять надо, скорее, хороших работников. Именно они выполняют большинство работ: собирают пищу, охотятся и ухаживают за детишками. Даже сражаются в основном они.

\emph{Ахилл} : Гм-м. Это просто абсурд, солдаты, которые не сражаются!

\emph{Муравьед} : Как я только что говорил, на самом деле они не солдаты. Роль солдат выполняют работники, в то время как солдаты --- просто разжиревшие лентяи.

\emph{Ахилл} : О, какой позор! Если бы я был муравьем, я бы навел у них дисциплину! Я вдолбил бы её в головы этим бесстыдникам!

\emph{Черепаха} : Если бы вы были муравьем? Как вы могли бы стать муравьем? Нет никакой возможности спроецировать ваш мозг на мозг муравья, так что и нечего беспокоиться о таком пустом вопросе. Больше смысла имело бы попытаться отобразить ваш мозг на всю муравьиную колонию\ldots{} Но не будем отвлекаться; позволим д-ру Муравьеду продолжить его ученое объяснение каст и их роли в высших уровнях организации.

\emph{Муравьед} : С удовольствием. Есть множество работ, необходимых для жизни колонии, и муравьи развивают «специализацию». Обычно специальность муравья меняется с возрастом, но она также зависит и от его касты. В каждый данный момент в любой маленькой области колонии можно найти муравьев всех типов. Разумеется, в некоторых местах какая-либо каста может быть представлена всего несколькими муравьями, тогда как в другом месте муравьев той же касты может быть очень много.

\emph{Краб} : Скажите, а «плотность» касты или специальности случайна? Почему муравьев определенного типа собирается больше в одном месте и меньше в другом?

\emph{Муравьед} : Я рад, что вы об этом спросили, поскольку это очень важно для понимания того, как думает колония. Со временем в колонии развивается очень точное распределение каст. Именно это распределение отвечает за сложность колонии, необходимую, чтобы вести беседы.

\emph{Ахилл} : Мне кажется, что постоянное движение муравьев туда-сюда делает абсолютно невозможным какое бы то ни было точное распределение. Любое такое распределение было бы тут же нарушено беспорядочным движением муравьев, так же как любые сложные структуры молекул газа не живут больше мгновения из-за беспорядочной бомбардировки со всех сторон.

\emph{Муравьед} : В муравьиной колонии ситуация совершенно обратная. На самом деле, именно постоянное снованье муравьев туда-сюда сохраняет и регулирует распределение каст в различных ситуациях. Оно не может оставаться одним и тем же. Оно должно непрерывно меняться, в некотором смысле отражая реальную ситуацию, с которой имеет дело колония в данный момент. Именно передвижение муравьев внутри колонии помогает приспособить распределение каст к нужной ситуации.

\emph{Черепаха} : Приведите, пожалуйста, пример.

\emph{Муравьед} : С удовольствием. Когда я, муравьед, прихожу в гости к г-же М.~Вейник, глупенькие муравьи, учуяв мой запах, начинают паниковать --- а это значит, что они принимаются бегать кругами, как сумасшедшие, совершенно иначе, чем они двигались до моего прихода.

\emph{Ахилл} : Ну, это-то понятно, поскольку вы --- смертельный враг колонии.

\emph{Муравьед} : Ну нет. Повторяю, нет ничего дальше от истины. Мура Вейник и я --- лучшие друзья. Я её любимый собеседник, она --- моя любимая Мурочка. Конечно, вы правы --- муравьи по отдельности меня до смерти боятся. Но это совершенно другое дело! Так или иначе, как видите, в ответ на мой приход внутреннее распределение муравьев в колонии полностью изменяется.

\emph{Ахилл} : Ясно.

\emph{Муравьед} : Новая дистрибуция отражает новую ситуацию --- мое присутствие. Видите ли, все зависит от того, как вы решите описывать распределение каст. Если вы продолжаете думать в терминах низших уровней --- отдельных муравьев --- то не видите леса за деревьями. Это слишком микроскопический уровень, а мысля микроскопическими категориями, вы обязательно упустите из виду некоторые крупномасштабные явления. Вы должны найти подходящую систему крупномасштабного описания кастовой дистрибуции; только тогда вы поймете, каким образом в распределении каст закодировано так много информации.

\emph{Ахилл} : Как же найти единицы нужного размера для описания состояния колонии?

\emph{Муравьед} : Что ж, начнем с самого начала. Когда муравьям нужно что-то сделать, они составляют маленькие «команды», которые работают вместе. Маленькие группы муравьев постоянно разваливаются и снова образуются. Те, что держатся вместе дольше --- настоящие команды; они не разбегаются, так как у муравьев есть общее дело.

\emph{Ахилл} : Вы говорили, что группа остается вместе, если количество муравьев достигает некоторого критического порога. Теперь вы утверждаете, что группа держится вместе, если у них есть какое-то дело.

\emph{Муравьед} : Эти утверждения эквивалентны. Возьмем, например, собирание еды: если какой-то муравей находит небольшое количество пищи и в своем энтузиазме сообщает об этом товарищам, число муравьев, которые ответят на зов, будет пропорционально количеству найденной еды. Если еды мало, она не привлечет критического количества муравьев. Именно это я и имел в виду, говоря об общем деле: со слишком маленьким кусочком пищи нечего делать, его надо просто игнорировать.

\emph{Ахилл} : Понятно. Значит, эти команды --- промежуточный уровень структуры между отдельными муравьями и колонией в целом.

\emph{Муравьед} : Совершенно верно. Существует специальный тип команды, который я называю «сигналом». Все высшие уровни структуры основаны на сигналах. На самом деле, все высшие существа являются набором сигналов, действующих согласованно. На высшем уровне существуют команды, составленные не из муравьев, а из команд низших уровней. Рано или поздно вы достигаете команд низшего уровня, то есть сигналов, а затем --- отдельных муравьев.

\emph{Ахилл} : Чему же команды-сигналы обязаны своим именем?

\emph{Муравьед} : Своей функции. Задача сигналов --- переправлять муравьев разных «профессий» в нужные места колонии. Типичная история сигнала такова: он формируется, когда перейден критический порог, необходимый для выживания команды, затем мигрирует на какое-то расстояние внутри колонии, и в определенный момент распадается на индивидуальных членов, предоставляя их своей судьбе.

\emph{Ахилл} : Это похоже на волну, несущую издалека ракушки и водоросли и оставляющую их на берегу.

\emph{Муравьед} : Действительно, это в чем-то аналогично, поскольку команда на самом деле должна оставить что-то, принесенное издалека, но волна отходит обратно в море, в то время как в случае сигнала аналогичной переносящей субстанции не существует, поскольку его составляют сами же муравьи.

\emph{Черепаха} : И, вероятно, сигнал начинает распадаться именно в том месте колонии, где нужны муравьи данного типа.

\emph{Муравьед} : Естественно.

\emph{Ахилл} : Естественно? МНЕ вовсе не так ясно, почему сигнал должен направляться именно туда, где он требуется. И даже если он идет в нужном направлении, откуда он знает, когда надо расходиться? Откуда он знает, что он прибыл на место своего назначения?

\emph{Муравьед} : Это очень важный вопрос. Он касается целенаправленного поведения --- или того, что кажется целенаправленным поведением --- сигналов. Из моего описания следует, что поведение сигналов можно охарактеризовать как направленное на выполнение некой задачи, и назвать его «целенаправленным». Но можно посмотреть на это и иначе.

\emph{Ахилл} : Подождите-ка. Либо поведение целенаправленно, либо НЕТ. Не понимаю, как можно иметь сразу обе возможности.

\emph{Муравьед} : Позвольте мне объяснить мою позицию и, может быть, тогда вы со мной согласитесь. Видите ли, когда сигнал сформирован, он понятия не имеет о том, что должен идти в каком-то определенном направлении. Но здесь решающую роль играет точное распределение каст. Именно оно определяет движение сигналов по колонии и то, как долго сигнал будет существовать и когда ему придет время «раствориться».

\emph{Ахилл} : Так что все зависит от дистрибуции каст?

\emph{Муравьед} : Верно. Предположим, сигнал движется вперед. В это время составляющие его муравьи сообщаются, либо путем прямого контакта, либо путем обмена запахов, с муравьями тех областей, где они проходят. Контакты и запахи передают информацию о местных необходимостях, как, скажем, построение гнезда или уход за детьми. Сигнал будет держаться вместе и продвигаться вперед до тех пор, пока местные нужды будут отличны от того, что он способен дать; но если его помощь ВОЗМОЖНА, он распадается на отдельных муравьев, которые могут быть использованы как дополнительная рабочая сила. Понимаете, каким образом распределение каст «ведет» сигналы внутри колонии?

\emph{Ахилл} : Да, теперь вижу.

\emph{Муравьед} : Вы понимаете, что, рассматривая вещи с подобной точки зрения, нельзя приписать сигналу какую-либо цель?

\emph{Ахилл} : Думаю, вы правы. На самом деле, я начинаю видеть ситуацию с двух сторон. С точки зрения муравьев, у сигнала нет никакой цели. Типичный муравей в сигнале просто бродит по колонии, не ища ничего особенного, пока не почувствует, что надо бы остановиться. Обычно его товарищи по команде согласны, и тогда команда «разгружается», и муравьи начинают действовать сами по себе. Для этого не нужно ни планов, ни заглядывания вперед, ни поиска нужного направления. Но с точки зрения КОЛОНИИ команда только что ответила на сообщение, написанное на языке кастовой дистрибуции. С этой точки зрения деятельность команды кажется целенаправленной.

\emph{Краб} : Что произошло бы, если бы распределение каст было совершенно случайным? Команды все равно бы формировались и расформировывались?

\emph{Муравьед} : Безусловно. Но благодаря бессмысленному распределению каст, колония не просуществовала бы долго.

\emph{Краб} : Именно к этому я и веду. Колонии выживают потому, что их кастовая дистрибуция имеет смысл, и этот смысл --- холистский аспект, невидимый на низших уровнях. Существование колонии невозможно объяснить, не принимая в расчет высшего уровня.

\emph{Муравьед} : Я понимаю вас, но мне кажется, вы смотрите на вещи слишком узко.

\emph{Краб} : Почему же?

\emph{Муравьед} : Муравьиные колонии эволюционировали в течение биллионов лет. Несколько механизмов прошли отбор, но большинство было забраковано. Конечным результатом явился набор механизмов, позволяющих колониям функционировать так, как мы только что описали. Если бы мы могли увидеть весь этот процесс в виде фильма, ускоренного в биллионы раз, возникновение новых механизмов выглядело бы как естественная реакция на внешние стимулы, подобно тому, как пузыри в кипящей воде --- реакция на внешний источник тепла. Я сомневаюсь, что вы могли бы увидеть некий «смысл» и «цель» в пузырях в кипящей воде --- или я ошибаюсь?

\emph{Краб} : Нет, но ---

\emph{Муравьед} : Хорошо; а вот МОЯ точка зрения. Как бы ни был велик такой пузырь, он обязан своим существованием процессам на молекулярном уровне, и вы можете забыть о «законах высшего уровня». То же самое верно и в случае колоний и команд. Глядя на картину с точки зрения эволюции, вы можете лишить всю колонию смысла и цели существования. Эти понятия становятся лишними.

\emph{Ахилл} : В таком случае, д-р~Муравьед, почему же вы говорите мне, что беседуете с мадам Вейник? Теперь мне кажется, что вы отказываете ей в каком-либо умении мыслить или говорить.

\emph{Муравьед} : Здесь нет никакого противоречия, Ахилл. Видите ли, мне так же трудно, как и другим, видеть вещи в таком грандиозном временном масштабе. Для меня гораздо легче поменять угол зрения. Когда я забываю об эволюции и вижу вещи такими, какими они являются здесь и сейчас, термины телеологии вновь обретают смысл: ЗНАЧЕНИЕ кастовой дистрибуции и ЦЕЛЕНАПРАВЛЕННОСТЬ сигналов. Это происходит не только тогда, когда я думаю о муравьиных колониях, но и когда я думаю о моем собственном мозге. Однако, сделав небольшое усилие, я всегда могу вспомнить и о другой точке зрения и увидеть эти системы как лишенные смысла.

\emph{Краб} : Эволюция, безусловно, творит чудеса. Никогда не знаешь, какой новый фокус она выкинет. Например, я не удивился бы, если бы существовала теоретическая возможность того, что два или более «сигналов» могли бы пройти сквозь друг друга, понятия не имея, что другой при этом тоже является сигналом; каждый из них считает, что другой --- лишь часть местного населения.

\emph{Муравьед} : Это возможно не только теоретически; именно так обыкновенно и происходит!

\emph{Ахилл} : Гм-м\ldots{} Какая странная картина мне пришла в голову. Я вообразил муравьев, двигающихся в четырех разных направлениях; некоторые муравьи белые, некоторые --- черные; они перекрещиваются, образуя упорядоченный узор, почти как\ldots{} почти как\ldots{}

\emph{Черепаха} : Фуга, может быть?

\emph{Ахилл} : Ага! Именно: муравьиная фуга!

\emph{Краб} : Интересный образ, Ахилл. Кстати, все эти разговоры о кипятке навели меня на мысль о чае. Кто хочет еще чашечку?

\emph{Ахилл} : Не откажусь, м-р Краб.

\emph{Краб} : Отлично.

\emph{Ахилл} : Как вы думаете, можно ли выделить разные зрительные «голоса» в такой «муравьиной фуге»? Я знаю, как мне бывает трудно ---

\emph{Черепаха} : Мне не надо, благодарю вас.

\emph{Ахилл} : --- проследить отдельный голос ---

\emph{Муравьед} : Мне тоже немного, м-р Краб ---

\emph{Ахилл} : --- в музыкальной фуге ---

\emph{Муравьед} : --- если нетрудно.

\emph{Ахилл} : --- когда все они ---

\emph{Краб} : Нисколько. Четыре чашки чая ---

\emph{Черепаха} : Три!

\emph{Ахилл} : --- звучат одновременно.

\emph{Краб} : --- почти готовы!

\emph{Муравьед} : Это интересная мысль, Ахилл. Но мне кажется маловероятным, чтобы кто-нибудь мог создать таким образом убедительную картину.

\emph{Ахилл} : Очень жаль\ldots{}

\emph{Черепаха} : Может быть, вы могли бы мне ответить, д-р~Муравьед. Сигнал, от своего рождения и до роспуска, всегда состоит из одних и тех же муравьев?

\emph{Муравьед} : На самом деле, отдельные муравьи в сигнале иногда «откалываются» и заменяются на муравьев той же касты, если какие-либо из них оказываются поблизости в данный момент. Чаще всего, сигналы прибывают к месту своего «назначения», не сохранив ни одного из первоначальных муравьев.

\emph{Краб} : Я вижу, что сигналы постоянно воздействуют на распределение каст на всем протяжении колонии, и это происходит в ответ на внутренние нужды колонии --- которые, в свою очередь отражают внешние ситуации в жизни колонии. Таким образом, как вы сказали, д-р~Муравьед, кастовая дистрибуция постоянно изменяется в соответствии с нуждами момента, отражая внешний мир.

\emph{Рис. 61. М. К. Эшер «Муравьиная фуга» (1953)}

\emph{Ахилл} : Но как же насчет промежуточных уровней структуры? Вы говорили, что распределение каст лучше всего описывать не в терминах отдельных муравьев или сигналов, но в терминах команд, состоящих из меньших команд, которые, в свою очередь,~состоят из других команд --- и так далее, пока мы не спустимся до уровня муравьев. И вы утверждали, что это --- ключ к пониманию того, как распределение каст может заключать закодированную информацию о мире.

\emph{Муравьед} : Да, я к этому и веду. Я предпочитаю называть команды достаточно высокого уровня «символами». Но, видите ли, значение, которое я вкладываю в это слово, несколько отличается от обычного. Мои «символы» --- АКТИВНЫЕ ПОДСИСТЕМЫ сложной системы, и они состоят из активных подсистем низших уровней. Таким образом они совершенно отличны от пассивных символов, находящихся вне системы, таких, как буквы алфавита или музыкальные ноты, которые совершенно инертны и ждут, чтобы активная система их обработала.~

\emph{Ахилл} : Как все это сложно\ldots{} Я и не знал, что у муравьиных колоний такая абстрактная структура.

\emph{Муравьед} : О да, она весьма замечательна. Но все эти слои структуры необходимы для хранения того типа знаний, которые позволяют организму быть «разумным» в любом приемлемом значении этого слова. У любой системы, владеющей языком, уровни структуры примерно одинаковые.

\emph{Ахилл} : Ну-ка постойте минутку\ldots{} Черт побери! Вы что же, хотите сказать, что мой мозг по сути состоит из кучи муравьев, бегающих взад и вперед?

\emph{Муравьед} : Ни в коем случае. Вы поняли меня слишком буквально. Самые низшие уровни этих систем могут быть совершенно различны. Скажем, мозг муравьедов вовсе не сделан из муравьев. Но когда вы поднимаетесь на несколько ступеней, элементы нового уровня имеют точное соответствие в других системах той же интеллектуальной мощи, таких, например, как муравьиная колония.

\emph{Черепаха} : Именно поэтому имеет смысл, Ахилл, пытаться отобразить ваш мозг на муравьиную колонию, но не на мозг отдельных муравьев.

\emph{Ахилл} : Благодарю за комплимент. Но как же возможно осуществить подобное отображение? Например, что в моем мозгу соответствует командам низших уровней, которые вы называете сигналами?

\emph{Муравьед} : Я всего лишь любитель в области мозга и не могу описать для вас всех замечательных подробностей. Но --- поправьте меня, если я ошибаюсь, м-р Краб --- мне кажется, что мозговое соответствие сигналам муравьиных колоний --- это действие нейрона; или, может быть, в более крупном масштабе, это схема действия нескольких нейронов.

\emph{Краб} : С этим я бы согласился. Конечно, найти точные отображения было бы желательно. Но не думаете ли вы, что для нашей дискуссии это не столь важно? Мне кажется, что главная идея здесь та, что такое соответствие в принципе существует, даже если мы пока еще точно не знаем, как его описать. Из всего, что вы сказали, д-р~Муравьед, я не понимаю только одного: как мы можем быть уверены в том, что соответствие начинается именно на каком-то данном уровне? Вы, кажется, считаете, что сигналы имеют прямое соответствие в мозгу; однако мне думается, что соответствие должно существовать только на уровне ваших АКТИВНЫХ СИМВОЛОВ и выше.

\emph{Муравьед} : Ваша интерпретация вполне может оказаться аккуратнее моей, м-р Краб. Благодарю за то, что вы подметили эту тонкость.

\emph{Ахилл} : Что может сделать символ, чего не дано сделать сигналу?

\emph{Муравьед} : Это что-то вроде разницы между словом и буквой. Слова --- единицы смысла --- состоят из букв, которые сами по себе лишены смысла. Это является хорошим примером, помогающим понять разницу между символом и сигналом. Надо только помнить, что буквы и слова ПАССИВНЫ, в то время как символы и сигналы АКТИВНЫ.

\emph{Ахилл} : Боюсь, я не понимаю, почему так важна разница между активными и пассивными символами.

\emph{Муравьед} : Дело в том, что значение, которое мы приписываем любому пассивному символу, например, слову на странице, в действительности восходит к значению, создаваемому соответствующими активными символами у нас в мозгу. Таким образом, значение пассивных символов можно понять полностью, только соотнеся его со значением активных символов.

\emph{Ахилл} : Хорошо. Но что же придает значение СИМВОЛУ, если СИГНАЛ, сам по себе полноправная единица, лишен значения?

\emph{Муравьед} : Дело в том, что одни символы могут вызвать к жизни другие символы. Когда какой-либо символ активизируется, он при этом не изолирован, а двигается в среде, характеризующейся определенной кастовой дистрибуцией.

\emph{Краб} : Разумеется, в мозгу нет никакой кастовой дистрибуции --- ей соответствует «состояние мозга» --- то есть состояние всех нейронов, все взаимосвязи между ними и порог, достигнув которого, нейрон активизируется.

\emph{Муравьед} : Ну что ж, мы может объединить «кастовую дистрибуцию» и «состояние мозга» под одним и тем же названием и именовать их просто «состоянием». Состояние можно описать на низшем или на высшем уровне. Описанием состояния на низшем уровне в случае муравьиной колонии будет кропотливое определение положения каждого муравья, его возраста и касты, и тому подобная информация. Но такое подробное описание не проливало бы ни малейшего света на то, ПОЧЕМУ колония находится в данном состоянии. С другой стороны, описание на высшем уровне определяет, какие именно символы могут быть «пущены в ход», какие комбинации других символов действуют при этом как пусковой механизм, при каких условиях это происходит, и так далее.

\emph{Ахилл} : А как насчет описания на уровне сигналов или команд?

\emph{Муравьед} : Описание на этом уровне располагалось бы где-то между низшим уровнем и уровнем символов. Оно содержало бы довольно много информации о том, что происходит в разных местах колонии, хотя и меньше, чем содержало бы описание всех муравьев в отдельности, поскольку команды состоят из нескольких муравьев. Описание на уровне команд --- нечто вроде конспекта описания на уровне муравьев. Однако туда приходится добавить некоторые вещи, которых не было в описании отдельных муравьев --- такие, как отношение между командами и наличие различных каст в разных районах. Это усложнение --- та цена, которую мы платим за право давать суммированные описания.

\emph{Ахилл} : Интересно сравнить достоинства описаний на разных уровнях. Описание на высшем уровне, по-видимому, обладает наибольшей способностью к объяснению явлений в колонии, поскольку оно дает наиболее интуитивную картину того, что в ней происходит, хотя, как ни странно, оставляет в стороне самое, казалось бы, важное --- самих муравьев.

\emph{Муравьед} : На самом деле, как ни странно, муравьи --- не самое важное в колонии. Разумеется, если бы их не было, колония не существовала бы; но нечто эквивалентное ей, мозг, может существовать без единого муравья. Так что, по-крайней мере с точки зрения высших уровней, можно вполне обойтись без муравьев.

\emph{Ахилл} : Я уверен, что ни один муравей не поддержал бы эту теорию.

\emph{Муравьед} : Мне еще не приходилось встречать муравья, который смотрел бы на свой муравейник с точки зрения высших уровней.

\emph{Краб} : Картинка, которую вы нарисовали, д-р~Муравьед, противоречит здравому смыслу. Кажется, что, если вы правы, то чтобы понять структуру муравьиной колонии, приходится описывать её, не упоминая об основных её составляющих.

\emph{Муравьед} : Может быть, вам станет яснее моя точка зрения на примере аналогии. Представьте себе, что перед вами --- роман Диккенса.

\emph{Ахилл} : Как насчет «Пиквикского клуба»?

\emph{Муравьед} : Превосходно! Теперь вообразите следующую игру: вы должны отыскать соответствие между буквами и идеями так, чтобы каждой букве соответствовала какая-либо идея; таким образом, весь «Пиквикский клуб» имел бы смысл, если читать его буква за буквой.

\emph{Ахилл} : Гм-м\ldots{} Вы имеете в виду, что каждый раз, когда я вижу слово «что», я должен думать о трех различных понятиях, одно за другим, и что при этом у меня нет никакого выбора?

\emph{Муравьед} : Именно так. У вас будет «понятие \enquote*{ч}», «понятие \enquote*{т}» и «понятие \enquote*{о}», и эти понятия остаются теми же на протяжении всей книги.

\emph{Ахилл} : Тогда чтение «Пиквикского клуба» превратилось бы в неописуемо скучный кошмар. Это было бы упражнением в бессмысленности, какую бы идею я ни ассоциировал с каждой буквой.

\emph{Муравьед} : Верно. Естественного отображения отдельных букв на реальный мир просто не существует. Это отображение происходит на высшем уровне --- между словами и частями реального мира. Если вы хотите описать эту книгу, вы не должны делать это на уровне букв.

\emph{Ахилл} : Разумеется, нет! Я буду описывать сюжет, героев и так далее.

\emph{Муравьед} : Ну вот, видите! Вы не будете упоминать о минимальных кирпичиках, из которых построена книга, хотя она и существует благодаря им. Они являются способом передачи сообщения, а не самим сообщением.

\emph{Ахилл} : Ну хорошо --- а как насчет муравьиных колоний?

\emph{Муравьед} : Здесь вместо пассивных букв у нас имеются активные сигналы, а вместо пассивных слов --- активные символы; но в остальном идея остается та же.

\emph{Ахилл} : Вы хотите сказать, что нельзя установить соответствия между сигналами и вещами реального мира?

\emph{Муравьед} : Оказывается, это не удается сделать таким образом, чтобы активизирование новых сигналов имело бы смысл. Невозможно это и на низших уровнях --- например, уровне муравьев. Только на уровне символов эти схемы активизирования имеют смысл. Вообразите, например, что в один прекрасный день вы наблюдаете за тем, чем занимается Мура Вейник, и в это время я тоже захожу в гости. Сколько бы вы ни смотрели, вряд ли вы увидите нечто большее, чем простое перераспределение муравьев.

\emph{Ахилл} : Я уверен, что вы правы.

\emph{Муравьед} : И тем не менее я, наблюдая за высшим уровнем вместо низшего уровня, увижу, как «просыпаются» несколько дремлющих символов, которые затем становятся мыслью «А вот и очаровательный д-р~Муравьед --- какое удовольствие!»

\emph{Ахилл} : Это напоминает мне, как мы нашли различные уровни на картинке «МУ» --- по-крайней мере, их увидели трое из нас\ldots{}

\emph{Черепаха} : Какое удивительное совпадение, что между странной картинкой, на которую я наткнулась в «Хорошо темперированном клавире», и темой нашей беседы обнаружилось сходство.

\emph{Ахилл} : Вы думаете, это просто совпадение?

\emph{Черепаха} : Конечно.

\emph{Муравьед} : Что ж, я надеюсь, теперь вы понимаете, каким образом в г-же М.~Вейник зарождаются мысли при помощи манипуляции символами, составленными из сигналов, составленных из команд низшего уровня\ldots{} --- и так до самого низшего уровня --- муравьев.

\emph{Ахилл} : Почему вы называете это «манипуляцией символами»? Кто это делает, если сами символы активны? Что является этой действующей силой?

\emph{Муравьед} : Это возвращает нас к вопросу о цели, который вы уже поднимали раньше. Вы правы, символы активны. Но тем не менее, их действия не совсем свободны. Действия всех символов строго определяются общим состоянием системы, в которой они находятся. Таким образом, все система ответственна за то, каким образом её символы вызывают к жизни один другого; поэтому мы можем с полным правом сказать, что «действующая сила» --- вся система. По мере того, как символы действуют, состояние системы медленно меняется, приходя в соответствие с новыми условиями. Однако многие черты остаются неизменными. Именно эта, частично меняющаяся и частично стабильная, система является действующей силой. Мы можем дать имя этой системе. Например, Мура Вейник --- это та сила, которая манипулирует её символами. И про вас, Ахилл, можно сказать то же самое.

\emph{Ахилл} : Какое странное описание того, кто я такой. Не уверен, что полностью его понимаю\ldots{} Я еще подумаю над этим.

\emph{Черепаха} : Было бы очень интересно проследить за символами в вашем мозгу в тот момент, когда вы думаете о символах в вашем мозгу.

\emph{Ахилл} : Это для меня слишком сложно. У меня хватает проблем, когда я пытаюсь представить себе, как можно смотреть на муравьиную колонию и «читать» её на уровне символов. Я прекрасно понимаю, как можно воспринимать её на уровне муравьев и, с небольшим усилием, пожалуй, могу понять, каким было бы восприятие колонии на уровне сигналов; но на что было бы похоже восприятие колонии на уровне символов?

\emph{Муравьед} : Это умение достигается долгой практикой. Когда вы достигнете мастерства, подобного моему, вы сможете читать высший уровень муравьиной колонии с такой же легкостью, с какой прочли «МУ» на той картинке.

\emph{Ахилл} : Правда? Это, должно быть, удивительное ощущение.

\emph{Муравьед} : В каком-то смысле --- но оно также хорошо знакомо и вам, Ахилл.

\emph{Ахилл} : Знакомо мне? Что вы имеете в виду? Я никогда не рассматривал муравьиных колоний кроме как на уровне муравьев.

\emph{Муравьед} : Может быть. Но муравьиные колонии во многих смыслах не слишком-то отличаются от мозга.

\emph{Ахилл} : И мозгов никогда не видел и не читал!

\emph{Муравьед} : А как же ВАШ СОБСТВЕННЫЙ мозг? Разве вы не замечаете своих собственных мыслей? Разве не в этом заключается эссенция сознания? Что же еще вы делаете, как не читаете ваш мозг прямо на уровне символов?

\emph{Ахилл} : Никогда так об этом не думал. Вы хотите сказать, что я игнорирую все промежуточные уровни и вижу только самый высший?

\emph{Муравьед} : Именно так функционируют сознательные системы. Они воспринимают себя только на уровне символов и понятия не имеют о низших уровнях, таких, как сигналы.

\emph{Ахилл} : Значит ли это, что в мозгу есть активные символы, постоянно меняющиеся так, чтобы отразить состояние самого мозга в данный момент, оставаясь при этом всегда на уровне символов?

\emph{Муравьед} : Безусловно. В любой разумной системе есть символы, представляющие состояние мозга, и сами они --- часть именно того состояния, которое они символизируют. Поскольку сознание требует большого самосознания.

\emph{Ахилл} : Очень странная идея. Значит, хотя в моем мозгу происходит бурная деятельность, я способен воспринять её только на одном уровне --- уровне символов; при этом я полностью нечувствителен к низшим уровням. Это похоже на чтение романов Диккенса при помощи прямого зрительного восприятия, при полном незнании букв. Не могу себе представить, чтобы такая странная штука на самом деле могла случиться.

\emph{Краб} : Но именно это и произошло, когда вы прочитали «МУ», не замечая низших уровней, «ХОЛИЗМА» и «РЕДУКЦИОНИЗМА».

\emph{Ахилл} : Вы правы --- я действительно упустил из вида низшие уровни и заметил только самый высший. Интересно, не пропускаю ли я какие-нибудь типы значения также и на низших уровнях моего мозга, когда «считываю» только уровень символов? Как жаль, что высший уровень не содержит всей информации о низших уровнях. Прочитав его, мы могли бы узнать также о том, что сообщается на низших уровнях. Но, полагаю, было бы наивно надеяться, что на вершине закодировано что-либо о низе --- скорее всего, эта информация не просачивается наверх. Картинка «МУ», пожалуй, самый выразительный пример, там на верхнем уровне написано только «МУ», которое не имеет никакого отношения к уровням ниже!

\emph{Краб} : Совершенно верно. (Берет книгу, чтобы взглянуть на иллюстрацию поближе.) Гм-м-м\ldots{} В самых маленьких буквах есть что-то странное; они какие-то дрожащие\ldots{}

\emph{Муравьед} : Дайте-ка взглянуть. (Подносит книгу к глазам.) Кажется, здесь есть еще один уровень, который мы все пропустили!

\emph{Черепаха} : Говорите только за себя, д-р~Муравьед.

\emph{Ахилл} : Ох, не может быть! Можно мне посмотреть? (Пристально глядит на картинку.) Я знаю, что никто из вас в это не поверит, но значение этой картинки у нас прямо перед носом, только спрятанное у нее в глубине. Это всего-навсего одно слово, повторенное снова и снова, на манер мантры --- но слово весьма важное: «МУ»! Вот видите! То же самое, что и на высшем уровне! И никто из нас об этом не догадывался!

\emph{Краб} : Мы бы никогда не заметили, Ахилл, если бы не вы.

\emph{Муравьед} : Интересно, это совпадение между высшим и низшим уровнем случайно? Или это целенаправленный акт, кем-то совершенный?

\emph{Краб} : Как же мы это можем узнать?

\emph{Черепаха} : Я не вижу, как это можно сделать --- мы даже не знаем, почему эта иллюстрация оказалась у м-ра Краба в его издании «Хорошо темперированного клавира».

\emph{Муравьед} : Хотя мы и увлеклись интересной беседой, мне всё же удалось следить краем уха за этой четырехголосной фугой, такой длинной и сложной. Она удивительно прекрасна.

\emph{Черепаха} : Бесспорно; и вскоре вы услышите органный пункт.

\emph{Ахилл} : Органный пункт? Это то, что происходит, когда музыкальная пьеса слегка замедляется, останавливается на минуту-другую на одной ноте или аккорде, и после короткой паузы продолжается в нормальном темпе?

\emph{Черепаха} : Нет, вы путаете с «ферматой» --- нечто вроде музыкальной точки с запятой. Вы заметили, что одна такая была в прелюдии?

\emph{Ахилл} : Кажется, я её пропустил.

\emph{Черепаха} : Ничего, у вас еще будет случай услышать фермату --- в конце этой фуги их целых две.

\emph{Ахилл} : Отлично. Предупредите меня заранее, хорошо?

\emph{Черепаха} : Если вам угодно.

\emph{Ахилл} : Но скажите пожалуйста, что же такое органный пункт?

\emph{Черепаха} : Это когда какая-то нота продолжается одним из голосов (чаще всего, самым низким) полифонической пьесы, пока другие голоса ведут свои независимые темы. Здесь органный пункт --- нота ля. Слушайте внимательно!

\emph{Муравьед} : Ваше предложение понаблюдать за символами в мозгу Ахилла, когда они думают о себе самих, напомнило мне один случай, который произошел со мной, когда я в очередной раз навещал мою старую знакомую, М.~Вейник.

\emph{Краб} : Поделитесь с нами, пожалуйста.

\emph{Муравьед} : Мура Вейник чувствовала себя в тот день очень одинокой и была рада с кем-нибудь поболтать. В благодарность она пригласила меня угоститься самыми сочными муравьями, которых я мог найти. (Она всегда очень великодушна, когда дело доходит до муравьев.)

\emph{Ахилл} : Удивительно, кЛЯнусь честью!

\emph{Муравьед} : В тот момент я как раз наблюдал за символами, образующими её мысли, поскольку именно там заметил особенно аппетитных муравьишек.

\emph{Ахилл} : Вы меня удивЛЯете!

\emph{Муравьед} : Так что я отобрал себе самых толстых муравьев, бывших частью символа высшего уровня, который я в тот момент читал. Так случилось, что именно эти символы выражали мысль: «Не стесняйтесь, выбирайте муравьев потолще!»

\emph{Ахилл} : О-ЛЯ-ЛЯ!

\emph{Муравьед} : К несчастью для них, но к счастью для меня, букашечки и не подозревали о том, что они, все вместе, сообщали мне на уровне символов.

\emph{Ахилл} : Несчастная доЛЯ\ldots{} Какой удивительный оборот иногда принимают события. Они понятия не имели о том, в чем участвовали. Их действия были частью определенной схемы высшего уровня, но сами они об этом не подозревали. О, какая жалость --- и какая ирония судьбы --- что они пропустили это мимо ушей.

\emph{Краб} : Вы правы, г-жа Ч --- это был прелестный органный пункт.

\emph{Муравьед} : Я раньше ни одного не слыхал, но этот был настолько очевиден, что его невозможно было прослушать. Замечательно!

\emph{Ахилл} : Что? Органный пункт уже был? Как же я мог его не заметить, если он был так очевиден?

\emph{Черепаха} : Возможно, вы были так увлечены своим рассказом, что не обратили на него внимания. О, какая жалость --- и какая ирония судьбы --- что вы пропустили это мимо ушей.

\emph{Краб} : Скажите мне, а что, Мура Вейник живет в муравейнике?

\emph{Муравьед} : О, ей принадлежит большой кусок земли. Раньше им владел кто-то другой --- но это весьма грустная история. Так или иначе, её владения довольно обширны. Она живет роскошно по сравнению со многими другими колониями.

\emph{Ахилл} : Как же это совместить с коммунистической природой муравьиных колоний, которую вы нам раньше описали? Мне кажется, что проповедовать коммунизм, живя при этом в роскоши и изобилии, довольно непоследовательно!

\emph{Муравьед} : Коммунизм там только на уровне муравьев. В муравьиной колонии все муравьи работают на общее благо, иногда даже себе в ущерб. Это --- врожденное свойство М.~Вейник, и, насколько я знаю, она может ничего не знать об этом внутреннем коммунизме. Большинство людей не знают ничего о своих нейронах; они, возможно, даже довольны тем, что ничего не знают о собственном мозге. Люди --- весьма брезгливые создания! Мура Вейник тоже довольно брезглива --- она начинает нервничать, стоит ей только подумать о муравьях. Так что она пытается этого избежать всегда, когда только возможно. Я, честное слово, сомневаюсь, что она догадывается о коммунистическом обществе, встроенном в саму её структуру. Она сама --- ярый приверженец полной свободы. Знаете, laissez-faire, и тому подобное. Так что я нахожу вполне естественным, что она живет в роскошном поместье.

\emph{Черепаха} : Я только что перевернула страницу, следя за этой прелестной фугой «Хорошо темперированного клавира», и заметила, что приближается первая из двух фермат --- приготовьтесь, Ахилл!

\emph{Ахилл} : Я весь внимание.

\emph{Черепаха} : На соседней странице здесь нарисована престранная картинка.

\emph{Краб} : Еще одна? Что там на этот раз?

\emph{Черепаха} : Поглядите сами. (Передает ноты Крабу.)

\emph{Рис. 62. Рисунок автора. (Русский графический вариант выполнен переводчиком.)}

\emph{Краб} : Ага! Да это всего лишь буквы. Посмотрим, что здесь есть\ldots{} по нескольку штук «И»,~«С», «Б», «м»~и~«а». Как странно, первые три буквы растут, а последние две опять уменьшаются.

\emph{Муравьед} : Можно мне взглянуть?

\emph{Краб} : Разумеется.

\emph{Муравьед} : Рассматривая детали, вы совершенно упустили из виду главную картину. На самом деле, эти буквы --- «ф», «е», «р», «А», «X» --- и они вовсе не повторяются. Сначала они становятся меньше, а потом опять вырастают. Ахилл, а как ваше мнение?

\emph{Ахилл} : Погодите минутку. Гм-м\ldots{} Я вижу несколько заглавных букв, которые увеличиваются слева направо.

\emph{Черепаха} : Это какое-то слово?

\emph{Ахилл} : Э-э-э\ldots{} «И. С. Бах». О! Теперь я понимаю. Это имя Баха!

\emph{Черепаха} : Как странно, что вы видите это именно так. Мне кажется, это несколько прописных букв, уменьшающихся слева направо, и составляющих\ldots{} имя\ldots{} (Темп её речи все замедляется, особенно на последних словах; потом она останавливается на мгновение, и вдруг начинает говорить снова, будто ничего необычного не произошло.) --- «фермата.»

\emph{Ахилл} : Вы, видно, все никак не можете выбросить из головы Ферма. Вы видите Последнюю Теорему Ферма даже здесь.

\emph{Муравьед} : Вы были правы, г-жа Черепаха: я только что заметил премилую маленькую фермату в фуге.

\emph{Краб} : И я тоже!

\emph{Ахилл} : Вы говорите, что все это слышали, кроме меня? Я начинаю чувствовать себя совсем дураком.

\emph{Черепаха} : Ну что вы, Ахилл, не надо так говорить. Я уверен, что вы не пропустите Последнюю Фермату Фуги --- она прозвучит очень скоро. Но, д-р~Муравьед, возвращаясь к нашему разговору, что это за печальная история, о которой вы упомянули, говоря о прежнем владельце поместья М.~Вейник?

\emph{Муравьед} : Прежний его владелец был удивительной личностью, одной из самых творчески одаренных муравьиных колоний, которые когда-либо существовали. Его звали Иогей Себастей Фермовей; он был мураматиком по профессии, но мурзыкантом по призванию.

\emph{Ахилл} : Муравительно!

\emph{Муравьед} : Он был в расцвете творческих сил, когда его постигла безвременная кончина. Однажды, жарким летним днем, он грелся на солнышке. Вдруг с ясного неба грянула гроза, одна из тех, что бывают раз в сто лет. И.С.Ф. промок до последнего муравья. Гроза началась совершенно неожиданно и застала муравьев врасплох. Сложная структура, создававшаяся годами, погибла за какие-то минуты. Какая трагедия!

\emph{Ахилл} : Вы хотите сказать, что все муравьи утонули, и это было концом И.С.Ф.?

\emph{Муравьед} : Нет, не совсем. Муравьям удалось выжить: все они уцепились за травинки и щепки, крутящиеся в бешеных потоках. Но когда вода спала и оставила муравьев на их территории, там не оставалось никакой организации. Кастовая дистрибуция была совершенно разрушена, и муравьи оказались не способны своими силами восстановить прежнюю отлаженную структуру. Они были так же беспомощны, как кусочки Шалтая-Болтая, если бы те попытались собрать себя самих. Подобно всей королевской коннице и всей королевской рати, я пытался собрать бедного Фермовея. Я подкладывал сахар и сыр, в сумасшедшей надежде на то, что Фермовей появится опять\ldots{} (Вынимает носовой платок, вытирает глаза и сморкается.)

\emph{Ахилл} : Как великодушно с вашей стороны. Я и не знал, что у Муравьедов такое доброе сердце\ldots{}

\emph{Муравьед} : Но все мои усилия были бесполезны. Он ушел из жизни, и ничто не могло вызвать его обратно. Однако тут начало происходить что-то странное: в течение следующих месяцев муравьи, бывшие компонентами И.С.Ф., перегруппировались и сформировали новую организацию. Так родилась Мура Вейник.

\emph{Краб} : Потрясающе! Мура Вейник состоит из тех же муравьев, что прежде И.С.Ф.?

\emph{Муравьед} : Сначала так и было, но теперь некоторые старые муравьи умерли и были заменены новыми муравьями. Однако там все еще остаются муравьи эпохи И.С.Ф.

\emph{Краб} : Скажите, а проявляются ли время от времени черты старика И.С.Ф. в мадам М.~Вейник?

\emph{Муравьед} : Ни одной. У них нет ничего общего. И я не вижу, откуда бы тут взяться сходству. В конце концов, есть несколько различных способов перегруппировать отдельные части, чтобы получить их «сумму». Мура Вейник как раз и была новой суммой старых частей. Не БОЛЬШЕ суммы, заметьте --- просто определенный ТИП суммы.

\emph{Черепаха} : Кстати о суммах --- это мне напомнило теорию чисел. Там тоже бывает возможно разложить теорему на составляющие её символы, расположить их в новом порядке и получить новую теорему.

\emph{Муравьед} : Никогда об этом не слышал; хотя должен признаться, что в этой области я полнейший невежда.

\emph{Ахилл} : Я тоже в первый раз слышу --- а ведь я прекрасно осведомлен в этой области, хотя и не должен сам себя хвалить. Думаю, что г-жа Ч готовит один из своих сложных розыгрышей --- я её уже хорошо изучил.

\emph{Муравьед} : Кстати о теории чисел --- это мне напомнило опять об И.С.Ф. Как раз в этой области он прекрасно разбирался. Теория чисел обязана ему несколькими важными открытиями. А Мура Вейник, наоборот, удивительно несообразительна, когда речь заходит о чем-то, имеющем даже отдаленнейшее отношение к математике. К тому же, у нее довольно банальные вкусы в музыке, в то время как Себастей был необычайно одарен в этой области.

\emph{Ахилл} : Мне очень нравится теория чисел. Не расскажете ли вы нам о каком-нибудь из открытий Себастея?

\emph{Муравьед} : Отлично. (Делает паузу, чтобы отхлебнуть свой чай, и снова начинает.) Слышали ли вы о печально известной «Хорошо Проверенной Гипотезе» Фурми?

\emph{Ахилл} : Не уверен. Это звучит знакомо, но я не могу вспомнить, что это такое.

\emph{Муравьед} : Идея очень проста Француз Льер де Фурми, мураматик по призванию, но адвокей по профессии, читая классическую «Арифметику» Диофантея, наткнулся на страницу с уравнением

2 \textsuperscript{a} + 2 \textsuperscript{b} = 2 \textsuperscript{c}

Он тут же понял, что это уравнение имеет бесконечное множество решений \emph{a} , \emph{b} и \emph{с} , и записал на полях следующий замечательный комментарий.

Уравнение

n \textsuperscript{a}~+~n~\textsuperscript{b}~=~n \textsuperscript{c}

имеет решение в положительных целых числах \emph{а} , \emph{b} , \emph{с} , и~\emph{n} только при~\emph{n} = 2 (и в таком случае имеется бесконечное множество \emph{а} , \emph{b} , и \emph{с} , удовлетворяющих этому уравнению), но для \emph{n} \textgreater2 решений не существует. Я нашел совершенно замечательное доказательство этого --- к несчастью, такое крохотное, что оно будет почти невидимо, если написать его на полях.

С того года и в течение почти трехсот дней мураматики безуспешно пытаются сделать одно из двух либо доказать утверждение Фурми и таким образом очистить его репутацию --- в последнее время она слегка подпорчена скептиками, не верящими, что он действительно нашел доказательство --- или опровергнуть его утверждение, найдя контрпример множество четырех целых чисел \emph{а} , \emph{b} , \emph{с} , и \emph{n} , где~\emph{n} \textgreater{} 2, которое удовлетворяло бы этому уравнению. До недавнего времени все попытки в любом из этих двух направлений проваливались. Точнее, Гипотеза доказана лишь для определенных значений~\emph{n} --- в частности, для всех~\emph{n} до 125 000. Но никому не удавалось доказать её для ВСЕХ~\emph{n} --- никому, пока на сцене не появился Иогей Себастей Фермовей. Именно он нашел доказательство, очистившее репутацию Фурми. Теперь это известно под именем «Хорошо Проверенной Гипотезы Иогея Себастея Фермовея».

\emph{Рис. 63. Когда происходят перемещения колоний, муравьи иногда строят из собственных тел живые мосты. На этой фотографии (Льера Фурми) изображен подобный мост. Муравьи-работники колонии Eciton Burchelli сцепляются лапками и тарзальными челюстями; таким образом создается что-то вроде цепей. Видно, как по центру мосту переходит симбиотическая чешуйница, Trichatelura manni. (E. О. Вильсон, «Общества насекомых» (Е.О. Wilson, «The Insect Societies», стр. 62.)}

\emph{Ахилл} : Не лучше ли тогда называть это «Теоремой» вместо «Гипотезы,» поскольку настоящее доказательство уже найдено?

\emph{Муравьед} : Строго говоря, вы правы, но по традиции это зовется именно так.

\emph{Черепаха} : А какую музыку писал Себастей?

\emph{Муравьед} : Он был очень талантливым композитором. К несчастью, его лучшее сочинение покрыто тайной, поскольку оно никогда не было опубликовано. Некоторые думают, что Себастей держал свое сочинение в голове. Но те, кто настроены менее благожелательно, говорят, что на самом деле он никогда не писал подобного сочинения, а только хвастался направо и налево.

\emph{Ахилл} : И что же это было за великое сочинение?

\emph{Муравьед} : Это должно было быть гигантской прелюдией и фугой; в фуге предполагалось двадцать четыре голоса и двадцать четыре различных темы, по одной в каждом мажорном и минорном ключе.

\emph{Ахилл} : Было бы весьма трудно слушать такую двадцатичетырехголосную футу как целое!

\emph{Краб} : Уже не говоря о том, чтобы её сочинить!

\emph{Муравьед} : Все, что нам о ней известно, это её описание, оставленное Себастеем на полях его экземпляра «Прелюдий и фуг для органа» Букстехуде. Последними словами, которые он написал перед своей трагической кончиной, были следующие:

\emph{Я сочинил замечательную фугу. В ней я соединил силу 24 тональностей с силой 24 тем, получилась фуга с мощью в 24 голоса. К несчастью, она не помещается на полях.}

Этот несостоявшийся шедевр известен под именем «Последняя Фуга Фермовея».

\emph{Ахилл} : О, как это невыносимо трагично!

\emph{Черепаха} : Кстати о фугах: та фуга, которую мы слушаем, скоро закончится. Ближе к концу в её теме происходит странная вариация. (Переворачивает страницу «Хорошо темперированного клавира».) Что это у нас тут? Еще одна иллюстрация, да какая интересная! (Показывает её Крабу.)

\emph{Рис. 64. (Рисунок автора. Русский графический вариант выполнен переводчиком.)}

\emph{Краб} : Что это у нас тут? О, вижу это «ХОЛИЗМИОНИЗМ», написанное большими буквами, которые сначала уменьшаются, а затем снова возрастают до того же размера. Но в этом нет никакого смысла, поскольку это не настоящее слово. Надо же, подумать только! (Передает ноты Муравьеду)

\emph{Муравьед} : Что это у нас тут? О, вижу: это «РЕДУКЦХОЛИЗМ», написанное маленькими буквами, которые сначала увеличиваются, а затем снова уменьшаются до того же размера. Но в этом нет никакого смысла, поскольку это не настоящее слово. Подумать только, надо же! (Передает ноты Ахиллу.)

\emph{Ахилл} : Я знаю, что никто из вас в это не поверит, но на деле эта картинка состоит из слова «ХОЛИЗМ», написанного дважды, причем буквы в нем уменьшаются слева направо.

\emph{Черепаха} : Я знаю, что никто из вас в это не поверит, но на деле эта картинка состоит из слова «РЕДУКЦИОНИЗМ», написанного один раз, причем буквы в нем увеличиваются слева направо.

\emph{Aхилл} : Наконец-то! На этот раз я услышал новую вариацию темы! Я так рад, что вы мне на нее указали, г-жа Черепаха. Мне кажется, что я наконец начинаю понимать кое-что в искусстве слушания фуг.

\end{dialogue}

\end{document}
