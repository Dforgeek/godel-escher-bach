\documentclass[../main.tex]{subfiles}
\begin{document}

\DialogueChapter{Ария в ключе G}

\centerblock{
    \emph{Черепаха и Ахилл возвращаются с экскурсии по фабрике консервных ключей.}
}

\begin{dialogue}

\speak{Ахилл} Вы не возражаете, если я поменяю тему?

\emph{Черепаха} : Ради Бога.

\emph{Ахилл} : Хорошо. Я хотел вам рассказать, что несколько дней тому назад меня разбудил хулиганский телефонный звонок.

\emph{Черепаха} : Как интересно!

\emph{Ахилл} : Да уж\ldots{} Дело в том, что нахал сказал что-то совершенно бессмысленное. Он крикнул мне в ухо какую-то идиотскую фразу и повесил трубку\ldots{} хотя, кажется, прежде чем повесить трубку, он повторил эту бессмыслицу дважды.

\emph{Черепаха} : Вы помните, что именно он сказал?

\emph{Ахилл} : Наш разговор проходил так:

\textbf{Я} : Алло?

\textbf{Таинственный голос (дико орет)} : Предваренное цитатой себя самого, порождает ложь! Предваренное цитатой себя самого, порождает ложь!

\emph{(Щелчок. Короткие гудки)}

\emph{Черепаха} : Для хулиганского звонка это довольно необычно.

\emph{Ахилл} : Вот и я так подумал.

\emph{Черепаха} : Может быть, в этой кажущейся чепухе всё же есть какой-то смысл.

Ахилл: Кто знает\ldots{}

\emph{(Они входят в небольшой дворик, окруженный прелестными трехэтажными домами. В центре двора растет пальма; сбоку стоит башня. Около башни --- ступеньки, на которых сидит мальчик, занятый беседой с девушкой в окне.)}

\emph{Черепаха} : Куда это вы меня привели, Ахилл?

\emph{Ахилл} : Я хочу показать вам замечательный вид, открывающийся с этой башни.

\emph{Черепаха} : Ах, как мило!

\emph{(Они приближаются к мальчику, который смотрит на них с любопытством и говорит что-то девушке; оба хихикают. Вместо того, чтобы подниматься по лестнице, где сидит мальчишка, Ахилл и г-жа Ч поворачивают налево и спускаются по ступенькам, ведущим к небольшой деревянной двери.)}

\emph{Ахилл} : Вот и вход. Следуйте за мной.

\emph{(Ахилл открывает дверь. Они входят и начинают подниматься по крутой винтовой лесенке.)}

\emph{Черепаха (сопя и отдуваясь)} : Я не гожусь для таких упражнений, Ахилл. Еще далеко?

\emph{Ахилл} : Несколько пролетов\ldots{} но у меня есть идея. Вместо того, чтобы карабкаться по верхней стороне лестницы, почему бы вам не попробовать идти по нижней стороне?

\emph{Рис. 74. М. К. Эшер. «Сверху и снизу» (литография, 1947).}

\emph{Черепаха} : Как же ТАКОЕ возможно?

\emph{Ахилл} : Запросто, держитесь покрепче и переползайте на обратную сторону ступеней --- места там достаточно. Вы увидите, что по этой лестнице можно ходить так же хорошо снизу, как и сверху\ldots{}

\emph{Черепаха (переползая на обратную сторону ступенек)} : Ну как, правильно?

\emph{Ахилл} : Все верно, молодец!

\emph{Черепаха (слегка приглушенным голосом)} : Это упражнение меня слегка запутало. Куда мне теперь идти --- вверх или вниз?

\emph{Ахилл} : Держитесь того же направления, как раньше. На вашей стороне ступенек это будет ВНИЗ, а на моей --- ВВЕРХ.

\emph{Черепаха} : Надеюсь, вы не хотите сказать, что спускаясь по лестнице, я могу попасть на вершину башни?

\emph{Ахилл} : Почему-то получается именно так.

\emph{(И они начинают карабкаться по лестнице, одновременно описывая спирали --- Атлетический Ахилл на одной стороне, и Тяжеловесная Черепаха Тортилла на другой. Вскоре лестница кончается)}

Теперь вылезайте обратно, г-жа Черепаха. Дайте-ка я вам помогу.

\emph{(Он подает Черепахе руку и помогает ей забраться на верхнюю сторону ступенек)}

\emph{Черепаха} : Спасибо. Залезть обратно наверх было полегче.

\emph{(И они выходят на крышу, откуда открывается вид на город)}

Какая красота, Ахилл. Я рада, что вы привели меня наверх --- или, скорее, ВНИЗ.

\emph{Ахилл} : Я так и знал, что вам понравится.

\emph{Черепаха} : Знаете, возвращаясь к тому хулиганскому звонку, --- мне кажется, теперь я лучше понимаю, в чем дело.

\emph{Ахилл} : Да? Надеюсь, вы со мной поделитесь.

\emph{Черепаха} : С удовольствием. Вам не кажется, что выражение «предваряемый цитатой самого себя» звучит немного рекурсивно?

\emph{Ахилл} : Да. Немного Самую малость\ldots{}

\emph{Черепаха} : Можете ли вы вообразить себе что-либо, предваряемое собственной цитатой?

\emph{Ахилл} : Пожалуй, например, Мао, входящий в банкетный зал, где уже повешен плакат с каким-либо его изречением. Получается Мао, предваряемый цитатой самого себя.

\emph{Черепаха} : Какое у вас богатое воображение. Но давайте договоримся, что слово «предваряемый» будет относиться только к идее предварения на листе бумаги, а не к цитатам из государственных мужей.

\emph{Ахилл} : Ну ладно. Но тогда заодно скажите, что вы имеете в виду под «цитатой»?

\emph{Черепаха} : Когда вы говорите о каком-то слове или фразе, вы обычно заключаете их в кавычки. Например, я могу сказать:

В слове «философ» пять букв.

Я поставила «философ» в кавычки, чтобы указать, что я имею в виду СЛОВО «философ», а не философа собственной персоной. Это пример различия между «ИСПОЛЬЗОВАНИЕМ» и «УПОМИНАНИЕМ».

\emph{Ахилл} : Что?

\emph{Черепаха} : Позвольте мне объяснить. Когда я говорю:

Философы зарабатывают кучу денег, ---

я ИСПОЛЬЗУЮ слово, чтобы создать у вас в голове образ седобородого мудреца, окруженного мешками денег. Но заключая это --- или любое другое --- слово в кавычки, я тем самым лишаю его собственного значения и набора связанных с ним ассоциаций, и у меня остаются только значки на бумаге или звуки. Это называется «УПОМИНАНИЕ». При этом важен только типографский аспект слова, а его значение полностью игнорируется.

\emph{Ахилл} : Это похоже на использование скрипки в качестве мухобойки. Или, может быть, точнее было бы сказать «упоминание скрипки»? Тут в скрипке важна только её твердость --- любое другое её значение и возможное использование полностью игнорируются. Если подумать, то при этом мы обходимся ничуть не лучше и с мухой.

\emph{Черепаха} : Ваши сравнения не лишены смысла, хотя они и являются весьма нестандартной интерпретацией различия между ИСПОЛЬЗОВАНИЕМ и УПОМИНАНИЕМ. Теперь, пожалуйста, представьте что-либо, предваряемое собственной цитатой.

\emph{Ахилл} : Ну что ж\ldots{} Как насчет:

«ГИП-ГИП УРА» ГИП-ГИП УРА

\emph{Черепаха} : Здорово! А еще что-нибудь?

\emph{Ахилл} : Ладно:

«\enquote*{ПЛЮХ} --- ЭТО НЕ НАЗВАНИЕ КНИГИ»

«ПЛЮХ»~--- ЭТО НЕ НАЗВАНИЕ КНИГИ.

\emph{Черепаха} : Этот пример станет гораздо интереснее, если убрать из него «Плюх».

\emph{Ахилл} : Правда? Посмотрим:

«ЭТО НЕ НАЗВАНИЕ КНИГИ»

ЭТО НЕ НАЗВАНИЕ КНИГИ.

\emph{Черепаха} : Видите, у вас получилось предложение.

\emph{Ахилл} : И правда! Это предложение о фразе «это не название книги» --- и предложение преглупое.

\emph{Черепаха} : Почему преглупое?

\emph{Ахилл} : Потому, что оно совершенно бессмысленно. Вот вам еще одно в том же духе:

«ЗАВИСИТ ОТ ТОГО, СКОЛЬКО ДЕНЕГ У КОГО»

ЗАВИСИТ ОТ ТОГО, СКОЛЬКО ДЕНЕГ У КОГО.

Ну, и что это означает? Право слово, что за глупая игра.

\emph{Черепаха} : Ну что вы --- напротив, это очень серьезно. В действительности, эта операция предварения некоей фразы её собственной цитатой настолько важна, что я дам ей специальное имя.

\emph{Ахилл} : Да? Какого же названия удостоится эта глупая операция?

\emph{Черепаха} : Думаю, что я назову это «квайнированием» фразы.

\emph{Ахилл} : «Квайнирование»? Что это еще за слово?

\emph{Черепаха} : Если не ошибаюсь, это слово из тринадцати букв.

\emph{Ахилл} : Я имел в виду, почему вы выбрали именно эти тринадцать букв и именно в таком порядке.

\emph{Черепаха} : Ага, теперь я понимаю, что вы хотели сказать, спросив меня: «Что это еще за слово?» Видите ли, эту операцию изобрел философ по имени «Виллард Ван Орман Квайн», так что я назвал её в его честь. К сожалению, подробнее объяснить не могу. Почему его имя состоит именно из этих букв, и именно в таком порядке --- на этот вопрос у меня пока нет ответа. Но я готова попытаться ---

\emph{Ахилл} : Прошу вас, не утруждайтесь! Меня совсем не интересуют эти детали. Так или иначе, теперь я умею квайнировать фразы. Это довольно занимательно\ldots{} Вот еще одна квайнированная фраза:

«ЭТО ФРАГМЕНТ ПРЕДЛОЖЕНИЯ» ЭТО ФРАГМЕНТ ПРЕДЛОЖЕНИЯ.

Разумеется, это глупо, зато интересно. Вы берете кусочек предложения, квайнируете его, и оп-ля! перед вами что-то новое! В данном случае, это настоящее предложение.

\emph{Черепаха} : Попробуйте квайнировать фразу «это фуга без темы».

\emph{Ахилл} : Фуга без темы была бы ---

\emph{Черепаха} : --- аномалией, разумеется. Но не отвлекайтесь. Сначала квайны, а потом пьесы. Как говорится, сделал дело --- играй смело!

\emph{Ахилл} : Квайны, говорите? Хорошо:

«ЭТО ФУГА БЕЗ ТЕМЫ» ЭТО ФУГА БЕЗ ТЕМЫ

Мне кажется, что больше смысла было бы говорить о «предложении» вместо «фуги». Ну да ладно\ldots{} Дайте мне еще пример!

\emph{Черепаха} : Хорошо, вот вам напоследок такая фраза:

«ПОСЛЕ КВАЙНИРОВАНИЯ ДАЕТ ЛЮБОВНУЮ ПЕСНЬ ЧЕРЕПАХИ».

\emph{Ахилл} : Это совсем нетрудно:

«ПОСЛЕ КВАЙНИРОВАНИЯ ДАЕТ ЛЮБОВНУЮ ПЕСНЬ ЧЕРЕПАХИ».

ПОСЛЕ КВАЙНИРОВАНИЯ ДАЕТ ЛЮБОВНУЮ ПЕСНЬ ЧЕРЕПАХИ.

Гмм\ldots{} Что-то здесь не то. О, понятно --- это предложение говорит о себе самом! Видите?

\emph{Черепаха} : Что вы хотите сказать? Предложения не умеют говорить.

\emph{Ахилл} : Да, но они упоминают о каких-то вещах, и это предложение упоминает прямо, недвусмысленно и безошибочно о самом себе! Чтобы это увидеть, вы должны вспомнить, что такое квайнирование.

\emph{Черепаха} : Мне совсем не кажется, что это предложение говорит о себе самом. Покажите мне хотя бы одно «Я», или «это предложение», или что-нибудь в этом роде.

\emph{Ахилл} : Вы нарочно придуряетесь. Его красота как раз и заключается в том, что оно относится к себе самому, не называя себя при этом прямо.

\emph{Черепаха} : Придется вам разложить это для меня по полочкам --- я женщина простая и таких сложностей не понимаю.

\emph{Ахилл} : Вы ведете себя как Фома Неверующий. Ну ладно, постараюсь\ldots{} Представьте себе, что я придумываю предложение --- назовем его «предложением П» --- и оставляю в нем прочерк.

\emph{Черепаха} : Например?

\emph{Ахилл} : Вот так:

«~\_\_\_~ПОСЛЕ КВАЙНИРОВАНИЯ ДАЕТ ЛЮБОВНУЮ ПЕСНЬ ЧЕРЕПАХИ».

Теперь тема предложения П зависит от того, как вы заполните прочерк. Как только вы сделали выбор, тема определена: Это будет фраза, которую вы получите, кзайнировав то, что оказалось на месте прочерка. Назовем это «предложением К», поскольку оно получается в результате квайнирования.

\emph{Черепаха} : Что ж, это имеет смысл. Если бы на месте прочерка мы поставили бы «написано на старых банках горчицы, чтобы сохранять её свежей», тогда предложением К было бы:

«НАПИСАНО НА СТАРЫХ БАНКАХ ГОРЧИЦЫ, ЧТОБЫ СОХРАНЯТЬ её СВЕЖЕЙ»

НАПИСАНО НА СТАРЫХ БАНКАХ ГОРЧИЦЫ, ЧТОБЫ СОХРАНЯТЬ её СВЕЖЕЙ.

\emph{Ахилл} : Значит, Предложение П утверждает (не знаю, правда, насколько это верно), что Предложение К ---- Любовная Песнь Черепахи. Так или иначе, Предложение П здесь говорит не о себе самом, но о Предложении К. Согласны ли вы с этим?

\emph{Черепаха} : Безусловно --- и что за прелестная Песнь!

\emph{Ахилл} : Но теперь я хочу заполнить прочерк чем-то другим, а именно:

«ПОСЛЕ КВАЙНИРОВАНИЯ ДАЕТ ЛЮБОВНУЮ ПЕСНЬ ЧЕРЕПАХИ».

\emph{Черепаха} : Ах, боже мой! Вы слишком все усложняете. Боюсь, этот орешек окажется мне не по зубам\ldots{}

\emph{Ахилл} : О, не волнуйтесь --- я уверен, что скоро вы все поймете. Теперь Предложением К становится:

«ПОСЛЕ КВАЙНИРОВАНИЯ ДАЕТ ЛЮБОВНУЮ ПЕСНЬ ЧЕРЕПАХИ»

ПОСЛЕ КВАЙНИРОВАНИЯ ДАЕТ ЛЮБОВНУЮ ПЕСНЬ ЧЕРЕПАХИ.

\emph{Черепаха} : Постойте-ка, я, кажется, поняла! Предложение К теперь стало совершенно таким же, как и предложение П.

\emph{Ахилл} : И, поскольку Предложение К --- всегда тема предложения П, у нас получается петля: Предложение П теперь указывает на самого себя. Как видите, автореферентность здесь получилась вполне случайно. Обычно Предложения П и К совершенно не похожи --- но при правильном выборе темы в предложении П, квайнирование покажет вам этот магический трюк.

\emph{Черепаха} : Ловко, ничего не скажешь! Странно, почему я сама до этого не додумалась. Скажите, а следующее предложение тоже автореферентно?

«СОСТОИТ ИЗ ЧЕТЫРЕХ СЛОВ»

СОСТОИТ ИЗ ЧЕТЫРЕХ СЛОВ.

\emph{Ахилл} : Гм-м\ldots{} Трудно сказать. Это предложение относится не себе самому, но скорее ко фразе «состоит из четырех слов». Хотя, разумеется, эта фраза --- ЧАСТЬ предложения.

\emph{Черепаха} : Так что предложение говорит о своей части --- и что же?

\emph{Ахилл} : Это можно тоже рассматривать как автореференцию, не так ли?

\emph{Черепаха} : По моему мнению, отсюда еще далеко до настоящей автореферентности. Но не забивайте себе сейчас голову этими сложностями --- у вас еще будет время о них поразмыслить.

\emph{Ахилл} : Правда?

\emph{Черепаха} : Безусловно, будет. А пока, почему бы вам не попробовать квайнировать фразу «Предваряемый цитатой себя самого, производит ложь»?

\emph{Ахилл} : А, вы имеете в виду тот хулиганский звонок. Квайнирование этой фразы дает:

«ПРЕДВАРЯЕМЫЙ ЦИТАТОЙ СЕБЯ САМОГО, ПРОИЗВОДИТ ЛОЖЬ»

ПРЕДВАРЯЕМЫЙ ЦИТАТОЙ СЕБЯ САМОГО, ПРОИЗВОДИТ ЛОЖЬ.

Так вот что говорил тот негодяй! Я тогда его не понял. И правда, какое неприличное замечание! Да за такое надо в тюрьму сажать!

\emph{Черепаха} : Это почему же?

\emph{Ахилл} : Я от него просто заболеваю, в отличие от предыдущих высказываний, я не могу сказать, истинно ли оно или ложно. И чем больше я о нем думаю, тем больше запутываюсь. У меня от этой путаницы голова идет кругом. Интересно, что за лунатик изобрел подобный кошмар и мучает им по ночам честных людей?

\emph{Черепаха} : Кто знает\ldots{} Ну что, пора спускаться?

\emph{Ахилл} : В этом нет нужды --- мы уже на первом этаже. Зайдите обратно, и вы в этом убедитесь \emph{(Они заходят в башню и видят небольшую деревянную дверь)} Вот и выход --- следуйте за мной.

\emph{Черепаха} : Вы уверены? Я вовсе не хочу свалиться с третьего этажа и сломать себе панцирь.

\emph{Ахилл} : Разве я вас когда-нибудь обманывал?

\emph{(И он открывает дверь. Прямо перед ними сидит, по всей видимости, тот же самый мальчуган, болтающий с той же самой девушкой. Ахилл и г-жа Ч поднимаются по тем же ступенькам, по которым, как кажется, они раньше спускались, чтобы зайти в башню, и выходят во двор, кажущийся тем же самым двориком, в котором они уже побывали раньше.)}

Благодарю вас, г-жа Ч, за ваше объяснение по поводу того хулиганского звонка.

\emph{Черепаха} : А я вас --- за прелестную прогулку. Надеюсь, мы скоро увидимся опять.

\end{dialogue}

\end{document}
