\documentclass[../main.tex]{subfiles}
\begin{document}

\DialogueChapter{Ария в ключе G}

\centerblock{
    \emph{Черепаха и Ахилл возвращаются с экскурсии по фабрике консервных ключей.}
}

\begin{Dialogue}

\speak{Ахилл} Вы не возражаете, если я поменяю тему?

\speak{Черепаха} Ради Бога.

\speak{Ахилл} Хорошо. Я хотел вам рассказать, что несколько дней тому назад меня разбудил хулиганский телефонный звонок.

\speak{Черепаха} Как интересно!

\speak{Ахилл} Да уж\ldots{} Дело в том, что нахал сказал что-то совершенно бессмысленное. Он крикнул мне в ухо какую-то идиотскую фразу и повесил трубку\ldots{} хотя, кажется, прежде чем повесить трубку, он повторил эту бессмыслицу дважды.

\speak{Черепаха} Вы помните, что именно он сказал?

\speak{Ахилл} Наш разговор проходил так:

\begin{sublevel}

\speak{Я} Алло?

\speak{Таинственный голос (дико орет)} Предваренное цитатой себя самого, порождает ложь! Предваренное цитатой себя самого, порождает ложь!

\stage{\emph{(Щелчок. Короткие гудки)}}

\end{sublevel}

\speak{Черепаха} Для хулиганского звонка это довольно необычно.

\speak{Ахилл} Вот и я так подумал.

\speak{Черепаха} Может быть, в этой кажущейся чепухе всё же есть какой-то смысл.

\speak{Ахилл} Кто знает\ldots{}

\stage{\emph{(Они входят в небольшой дворик, окруженный прелестными трехэтажными домами. В центре двора растет пальма; сбоку стоит башня. Около башни \--- ступеньки, на которых сидит мальчик, занятый беседой с девушкой в окне.)}}

\speak{Черепаха} Куда это вы меня привели, Ахилл?

\speak{Ахилл} Я хочу показать вам замечательный вид, открывающийся с этой башни.

\speak{Черепаха} Ах, как мило!

\stage{\emph{(Они приближаются к мальчику, который смотрит на них с любопытством и говорит что-то девушке; оба хихикают. Вместо того, чтобы подниматься по лестнице, где сидит мальчишка, Ахилл и г-жа~Ч поворачивают налево и спускаются по ступенькам, ведущим к небольшой деревянной двери.)}}

\speak{Ахилл} Вот и вход. Следуйте за мной.

\stage{\emph{(Ахилл открывает дверь. Они входят и начинают подниматься по крутой винтовой лесенке.)}}

\speak{Черепаха (сопя и отдуваясь)} Я не гожусь для таких упражнений, Ахилл. Еще далеко?

\speak{Ахилл} Несколько пролетов\ldots{} но у меня есть идея. Вместо того, чтобы карабкаться по верхней стороне лестницы, почему бы вам не попробовать идти по нижней стороне?

% TODO: illustration 74
\emph{Рис. 74. М.\,К.~Эшер. «Сверху и снизу» (литография, 1947).}

\speak{Черепаха} Как же ТАКОЕ возможно?

\speak{Ахилл} Запросто, держитесь покрепче и переползайте на обратную сторону ступеней \--- места там достаточно. Вы увидите, что по этой лестнице можно ходить так же хорошо снизу, как и сверху\ldots{}

\speak{Черепаха (переползая на обратную сторону ступенек)} Ну как, правильно?

\speak{Ахилл} Все верно, молодец!

\speak{Черепаха (слегка приглушенным голосом)} Это упражнение меня слегка запутало. Куда мне теперь идти \--- вверх или вниз?

\speak{Ахилл} Держитесь того же направления, как раньше. На вашей стороне ступенек это будет ВНИЗ, а на моей \--- ВВЕРХ.

\speak{Черепаха} Надеюсь, вы не хотите сказать, что спускаясь по лестнице, я могу попасть на вершину башни?

\speak{Ахилл} Почему-то получается именно так.

\stage{\emph{(И они начинают карабкаться по лестнице, одновременно описывая спирали \--- Атлетический Ахилл на одной стороне, и Тяжеловесная Черепаха Тортилла на другой. Вскоре лестница кончается)}}

Теперь вылезайте обратно, г-жа Черепаха. Дайте-ка я вам помогу.

\stage{\emph{(Он подает Черепахе руку и помогает ей забраться на верхнюю сторону ступенек)}}

\speak{Черепаха} Спасибо. Залезть обратно наверх было полегче.

\stage{\emph{(И они выходят на крышу, откуда открывается вид на город)}}

Какая красота, Ахилл. Я рада, что вы привели меня наверх \--- или, скорее, ВНИЗ.

\speak{Ахилл} Я так и знал, что вам понравится.

\speak{Черепаха} Знаете, возвращаясь к тому хулиганскому звонку, \--- мне кажется, теперь я лучше понимаю, в чем дело.

\speak{Ахилл} Да? Надеюсь, вы со мной поделитесь.

\speak{Черепаха} С удовольствием. Вам не кажется, что выражение «предваряемый цитатой самого себя» звучит немного рекурсивно?

\speak{Ахилл} Да. Немного Самую малость\ldots{}

\speak{Черепаха} Можете ли вы вообразить себе что-либо, предваряемое собственной цитатой?

\speak{Ахилл} Пожалуй, например, Мао, входящий в банкетный зал, где уже повешен плакат с каким-либо его изречением. Получается Мао, предваряемый цитатой самого себя.

\speak{Черепаха} Какое у вас богатое воображение. Но давайте договоримся, что слово «предваряемый» будет относиться только к идее предварения на листе бумаги, а не к цитатам из государственных мужей.

\speak{Ахилл} Ну ладно. Но тогда заодно скажите, что вы имеете в виду под «цитатой»?

\speak{Черепаха} Когда вы говорите о каком-то слове или фразе, вы обычно заключаете их в кавычки. Например, я могу сказать:

% TODO: center
В слове «философ» пять букв.

Я поставила «философ» в кавычки, чтобы указать, что я имею в виду СЛОВО «философ», а не философа собственной персоной. Это пример различия между «ИСПОЛЬЗОВАНИЕМ» и «УПОМИНАНИЕМ».

\speak{Ахилл} Что?

\speak{Черепаха} Позвольте мне объяснить. Когда я говорю:

% TODO: center
Философы зарабатывают кучу денег, \---

я ИСПОЛЬЗУЮ слово, чтобы создать у вас в голове образ седобородого мудреца, окруженного мешками денег. Но заключая это \--- или любое другое \--- слово в кавычки, я тем самым лишаю его собственного значения и набора связанных с ним ассоциаций, и у меня остаются только значки на бумаге или звуки. Это называется «УПОМИНАНИЕ». При этом важен только типографский аспект слова, а его значение полностью игнорируется.

\speak{Ахилл} Это похоже на использование скрипки в качестве мухобойки. Или, может быть, точнее было бы сказать «упоминание скрипки»? Тут в скрипке важна только её твердость \--- любое другое её значение и возможное использование полностью игнорируются. Если подумать, то при этом мы обходимся ничуть не лучше и с мухой.

\speak{Черепаха} Ваши сравнения не лишены смысла, хотя они и являются весьма нестандартной интерпретацией различия между ИСПОЛЬЗОВАНИЕМ и УПОМИНАНИЕМ. Теперь, пожалуйста, представьте что-либо, предваряемое собственной цитатой.

\speak{Ахилл} Ну что ж\ldots{} Как насчет:

% TODO: center
«ГИП-ГИП УРА» ГИП-ГИП УРА

\speak{Черепаха} Здорово! А еще что-нибудь?

\speak{Ахилл} Ладно:

«\enquote*{ПЛЮХ} \--- ЭТО НЕ НАЗВАНИЕ КНИГИ»

% TODO: center
«ПЛЮХ» \--- ЭТО НЕ НАЗВАНИЕ КНИГИ.

\speak{Черепаха} Этот пример станет гораздо интереснее, если убрать из него «Плюх».

\speak{Ахилл} Правда? Посмотрим:

% TODO: center
«ЭТО НЕ НАЗВАНИЕ КНИГИ»

ЭТО НЕ НАЗВАНИЕ КНИГИ.

\speak{Черепаха} Видите, у вас получилось предложение.

\speak{Ахилл} И правда! Это предложение о фразе «это не название книги» \--- и предложение преглупое.

\speak{Черепаха} Почему преглупое?

\speak{Ахилл} Потому, что оно совершенно бессмысленно. Вот вам еще одно в том же духе:

% TODO: center
«ЗАВИСИТ ОТ ТОГО, СКОЛЬКО ДЕНЕГ У КОГО»

ЗАВИСИТ ОТ ТОГО, СКОЛЬКО ДЕНЕГ У КОГО.

Ну, и что это означает? Право слово, что за глупая игра.

\speak{Черепаха} Ну что вы \--- напротив, это очень серьезно. В действительности, эта операция предварения некоей фразы её собственной цитатой настолько важна, что я дам ей специальное имя.

\speak{Ахилл} Да? Какого же названия удостоится эта глупая операция?

\speak{Черепаха} Думаю, что я назову это «квайнированием» фразы.

\speak{Ахилл} «Квайнирование»? Что это еще за слово?

\speak{Черепаха} Если не ошибаюсь, это слово из тринадцати букв.

\speak{Ахилл} Я имел в виду, почему вы выбрали именно эти тринадцать букв и именно в таком порядке.

\speak{Черепаха} Ага, теперь я понимаю, что вы хотели сказать, спросив меня: «Что это еще за слово?» Видите ли, эту операцию изобрел философ по имени «Виллард Ван Орман Квайн», так что я назвал её в его честь. К сожалению, подробнее объяснить не могу. Почему его имя состоит именно из этих букв, и именно в таком порядке \--- на этот вопрос у меня пока нет ответа. Но я готова попытаться \---

\speak{Ахилл} Прошу вас, не утруждайтесь! Меня совсем не интересуют эти детали. Так или иначе, теперь я умею квайнировать фразы. Это довольно занимательно\ldots{} Вот еще одна квайнированная фраза:

% TODO: center
«ЭТО ФРАГМЕНТ ПРЕДЛОЖЕНИЯ» ЭТО ФРАГМЕНТ ПРЕДЛОЖЕНИЯ.

Разумеется, это глупо, зато интересно. Вы берете кусочек предложения, квайнируете его, и оп-ля! перед вами что-то новое! В данном случае, это настоящее предложение.

\speak{Черепаха} Попробуйте квайнировать фразу «это фуга без темы».

\speak{Ахилл} Фуга без темы была бы \---

\speak{Черепаха} \--- аномалией, разумеется. Но не отвлекайтесь. Сначала квайны, а потом пьесы. Как говорится, сделал дело \--- играй смело!

\speak{Ахилл} Квайны, говорите? Хорошо:

% TODO: center
«ЭТО ФУГА БЕЗ ТЕМЫ» ЭТО ФУГА БЕЗ ТЕМЫ

Мне кажется, что больше смысла было бы говорить о «предложении» вместо «фуги». Ну да ладно\ldots{} Дайте мне еще пример!

\speak{Черепаха} Хорошо, вот вам напоследок такая фраза:

% TODO: center
«ПОСЛЕ КВАЙНИРОВАНИЯ ДАЕТ ЛЮБОВНУЮ ПЕСНЬ ЧЕРЕПАХИ».

\speak{Ахилл} Это совсем нетрудно:

% TODO: center
«ПОСЛЕ КВАЙНИРОВАНИЯ ДАЕТ ЛЮБОВНУЮ ПЕСНЬ ЧЕРЕПАХИ».

ПОСЛЕ КВАЙНИРОВАНИЯ ДАЕТ ЛЮБОВНУЮ ПЕСНЬ ЧЕРЕПАХИ.

Гмм\ldots{} Что-то здесь не то. О, понятно \--- это предложение говорит о себе самом! Видите?

\speak{Черепаха} Что вы хотите сказать? Предложения не умеют говорить.

\speak{Ахилл} Да, но они упоминают о каких-то вещах, и это предложение упоминает прямо, недвусмысленно и безошибочно о самом себе! Чтобы это увидеть, вы должны вспомнить, что такое квайнирование.

\speak{Черепаха} Мне совсем не кажется, что это предложение говорит о себе самом. Покажите мне хотя бы одно «Я», или «это предложение», или что-нибудь в этом роде.

\speak{Ахилл} Вы нарочно придуряетесь. Его красота как раз и заключается в том, что оно относится к себе самому, не называя себя при этом прямо.

\speak{Черепаха} Придется вам разложить это для меня по полочкам \--- я женщина простая и таких сложностей не понимаю.

\speak{Ахилл} Вы ведете себя как Фома Неверующий. Ну ладно, постараюсь\ldots{} Представьте себе, что я придумываю предложение \--- назовем его «предложением~П» \--- и оставляю в нем прочерк.

\speak{Черепаха} Например?

\speak{Ахилл} Вот так:

«\makebox[3em]{\ulfill} ПОСЛЕ КВАЙНИРОВАНИЯ ДАЕТ ЛЮБОВНУЮ ПЕСНЬ ЧЕРЕПАХИ».

Теперь тема предложения~П зависит от того, как вы заполните прочерк. Как только вы сделали выбор, тема определена: Это будет фраза, которую вы получите, кзайнировав то, что оказалось на месте прочерка. Назовем это «предложением~К», поскольку оно получается в результате квайнирования.

\speak{Черепаха} Что ж, это имеет смысл. Если бы на месте прочерка мы поставили бы «написано на старых банках горчицы, чтобы сохранять её свежей», тогда предложением~К было бы:

% TODO: center
«НАПИСАНО НА СТАРЫХ БАНКАХ ГОРЧИЦЫ, ЧТОБЫ СОХРАНЯТЬ её СВЕЖЕЙ»

НАПИСАНО НА СТАРЫХ БАНКАХ ГОРЧИЦЫ, ЧТОБЫ СОХРАНЯТЬ её СВЕЖЕЙ.

\speak{Ахилл} Значит, Предложение~П утверждает (не знаю, правда, насколько это верно), что Предложение~К \--- Любовная Песнь Черепахи. Так или иначе, Предложение~П здесь говорит не о себе самом, но о Предложении~К. Согласны ли вы с этим?

\speak{Черепаха} Безусловно \--- и что за прелестная Песнь!

\speak{Ахилл} Но теперь я хочу заполнить прочерк чем-то другим, а именно:

% TODO: center
«ПОСЛЕ КВАЙНИРОВАНИЯ ДАЕТ ЛЮБОВНУЮ ПЕСНЬ ЧЕРЕПАХИ».

\speak{Черепаха} Ах, боже мой! Вы слишком все усложняете. Боюсь, этот орешек окажется мне не по зубам\ldots{}

\speak{Ахилл} О, не волнуйтесь \--- я уверен, что скоро вы все поймете. Теперь Предложением К становится:

% TODO: center
«ПОСЛЕ КВАЙНИРОВАНИЯ ДАЕТ ЛЮБОВНУЮ ПЕСНЬ ЧЕРЕПАХИ»

ПОСЛЕ КВАЙНИРОВАНИЯ ДАЕТ ЛЮБОВНУЮ ПЕСНЬ ЧЕРЕПАХИ.

\speak{Черепаха} Постойте-ка, я, кажется, поняла! Предложение~К теперь стало совершенно таким же, как и предложение~П.

\speak{Ахилл} И, поскольку Предложение~К \--- всегда тема предложения~П, у нас получается петля: Предложение~П теперь указывает на самого себя. Как видите, автореферентность здесь получилась вполне случайно. Обычно Предложения П~и~К совершенно не похожи \--- но при правильном выборе темы в предложении~П, квайнирование покажет вам этот магический трюк.

\speak{Черепаха} Ловко, ничего не скажешь! Странно, почему я сама до этого не додумалась. Скажите, а следующее предложение тоже автореферентно?

% TODO: center
«СОСТОИТ ИЗ ЧЕТЫРЕХ СЛОВ»

СОСТОИТ ИЗ ЧЕТЫРЕХ СЛОВ.

\speak{Ахилл} Гм-м\ldots{} Трудно сказать. Это предложение относится не себе самому, но скорее ко фразе «состоит из четырех слов». Хотя, разумеется, эта фраза \--- ЧАСТЬ предложения.

\speak{Черепаха} Так что предложение говорит о своей части \--- и что же?

\speak{Ахилл} Это можно тоже рассматривать как автореференцию, не так ли?

\speak{Черепаха} По моему мнению, отсюда еще далеко до настоящей автореферентности. Но не забивайте себе сейчас голову этими сложностями \--- у вас еще будет время о них поразмыслить.

\speak{Ахилл} Правда?

\speak{Черепаха} Безусловно, будет. А пока, почему бы вам не попробовать квайнировать фразу «Предваряемый цитатой себя самого, производит ложь»?

\speak{Ахилл} А, вы имеете в виду тот хулиганский звонок. Квайнирование этой фразы дает:

% TODO: center
«ПРЕДВАРЯЕМЫЙ ЦИТАТОЙ СЕБЯ САМОГО, ПРОИЗВОДИТ ЛОЖЬ»

ПРЕДВАРЯЕМЫЙ ЦИТАТОЙ СЕБЯ САМОГО, ПРОИЗВОДИТ ЛОЖЬ.

Так вот что говорил тот негодяй! Я тогда его не понял. И правда, какое неприличное замечание! Да за такое надо в тюрьму сажать!

\speak{Черепаха} Это почему же?

\speak{Ахилл} Я от него просто заболеваю, в отличие от предыдущих высказываний, я не могу сказать, истинно ли оно или ложно. И чем больше я о нем думаю, тем больше запутываюсь. У меня от этой путаницы голова идет кругом. Интересно, что за лунатик изобрел подобный кошмар и мучает им по ночам честных людей?

\speak{Черепаха} Кто знает\ldots{} Ну что, пора спускаться?

\speak{Ахилл} В этом нет нужды \--- мы уже на первом этаже. Зайдите обратно, и вы в этом убедитесь \emph{(Они заходят в башню и видят небольшую деревянную дверь)} Вот и выход \--- следуйте за мной.

\speak{Черепаха} Вы уверены? Я вовсе не хочу свалиться с третьего этажа и сломать себе панцирь.

\speak{Ахилл} Разве я вас когда-нибудь обманывал?

\stage{\emph{(И он открывает дверь. Прямо перед ними сидит, по всей видимости, тот же самый мальчуган, болтающий с той же самой девушкой. Ахилл и г-жа Ч поднимаются по тем же ступенькам, по которым, как кажется, они раньше спускались, чтобы зайти в башню, и выходят во двор, кажущийся тем же самым двориком, в котором они уже побывали раньше.)}}

Благодарю вас, г-жа~Ч, за ваше объяснение по поводу того хулиганского звонка.

\speak{Черепаха} А я вас \--- за прелестную прогулку. Надеюсь, мы скоро увидимся опять.

\end{Dialogue}

\end{document}
