\subsubsection{Магнификраб в пирожоре}

\emph{Чудный весенний денек, воскресенье. Черепаха и Ахилл отправились погулять за город, они решили взобраться на гору, на вершине которой, как они слышали, есть замечательная чайная, где к чаю подаются отличные пирожные.}

\emph{Ахилл} : Маг, о маг! Этот краб ---

Черепаха: Этот краб??

\emph{Ахилл} : Я хочу сказать, наш общий знакомый, м-р Краб, кажется мне совершенно магическим созданием\ldots{} Где вы еще найдете такого разумного краба? Он, может быть, раза в два умнее любого из своих собратьев. Даже в три. Или даже ---

\emph{Черепаха} : Боже мой! Как вы возвеличиваете Краба!

\emph{Ахилл} : Я всего лишь восхищаюсь его ---

\emph{Черепаха} : О, не надо ничего объяснять --- я тоже его поклонница. Кстати о поклонниках Краба --- говорила ли я вам, какое занятное письмо один из них недавно послал Крабу?

\emph{Ахилл} : Не помню. От кого было письмо?

\emph{Черепаха} : На нем была наклеена индийская марка; имя отправителя --- м-р Няунамар.

\emph{Ахилл} : Интересно, почему человек, который никогда не видел Краба, решил послать ему письмо? И откуда он узнал его адрес?

\emph{Черепаха} : Видимо, посчитал, что Краб --- математик. В письме содержатся некие результаты, которые\ldots{} О! Легок на помине! Вот и сам Краб, ползет вниз по холму!

\emph{Краб} : Пока! Приятно было с вами поболтать. Что ж, мне пора. Так наелся --- в меня уже ни кусочка не влезет! Я сам только что оттуда --- отличное местечко! Вы никогда не были в чайной на горе? Как дела, Ахилл? О, да это же Ахилл! Привет, привет! Не г-жа ли это Черепаха?

\emph{Черепаха} : Приветствую вас, м-р К Вы собрались сходить в чайную на горе?

\emph{Краб} : Точно --- как вы догадались? Я хочу отведать наполеонов --- мое любимое лакомство! Впрочем, я так голоден, что мог бы слопать и лягушку. О, да это же Ахилл! Как дела, Ахилл?

\emph{Ахилл} : Могло бы быть и хуже.

\emph{Краб} : Вот и славно! Но умоляю, не давайте мне прерывать вашей ученой беседы. Позвольте мне к вам присоединиться --- я постараюсь вам не мешать.

\emph{Черепаха} : Напротив --- я только что собиралась описать то странное письмо, которое вы получили несколько недель тому назад; но думаю, что Ахилл предпочтет получить эти сведения, так сказать, из первых клешней.

\emph{Краб} : Дело было так. Этот Няунамар, по-видимому, никогда не обучался математике, вместо этого он придумал свои собственные методы для открытия новых математических истин. Некоторые его идеи меня просто поразили --- я никогда не видел ничего подобного. Например, он прислал мне карту Индии, которую ему удалось раскрасить, по меньшей мере, в 1729 цветов!

\emph{Ахилл} : 1729! Вы сказали «1729»?

\emph{Краб} : Да, а почему вас это удивляет?

\emph{Ахилл} : Знаете ли, 1729 --- очень интересное число.

\emph{Краб} : И правда, мне это как-то не пришло в голову.

\emph{Ахилл} : В частности, 1729 --- номер того такси, в котором я ехал сегодня утром к Черепахе!

\emph{Краб} : Как интересно! А не могли бы вы мне сказать заодно и номер того троллейбуса, на котором вы поедете к г-же Черепахе завтра утром?

\emph{Ахилл (после минутного раздумья)} : Я точно не знаю, но мне кажется, что это будет очень большое число.

\emph{Черепаха} : У Ахилла на эти дела особый нюх.

\emph{Краб} : Да. Так вот, как я говорил, Няунамар в своем письме также доказал, что каждое четное простое число представляет собой сумму двух нечетных чисел и что не существует положительных целых решений уравнения:

а \textsuperscript{n} +~b \textsuperscript{n} = с \textsuperscript{n}, где~n = 0

\emph{Ахилл} : Что? Неужели ему удалось одним махом разрешить все эти классические задачки? Да он, кажется, настоящий гений!

\emph{Черепаха} : Но постойте, Ахилл --- неужели у вас не возникает ни тени сомнения?

\emph{Ахилл} : Что? Ах да, тени сомнения. Разумеется, возникают. Надеюсь, вы не думаете, что я поверил в то, что Краб действительно получил подобное письмо? Меня не так-то просто обвести вокруг пальца, знаете ли. Наверное, это письмо, г-жа Черепаха, на самом деле получили Вы.

\emph{Черепаха} : О, нет, Ахилл! То, что письмо было адресовано м-ру Крабу --- чистая правда. Я имела в виду, не возникают ли у вас сомнения по поводу содержания самого письма, его странных утверждений?

\emph{Ахилл} : Почему бы им возникнуть? Гм-м\ldots{} Разумеется, возникают. Я вообще чрезвычайный скептик, как вы оба отлично знаете. Меня очень трудно в чем-либо убедить, неважно, истина это или ложь.

\emph{Черепаха} : Отлично сказано, Ахилл. Поистине, вы знаток секретов собственного мозга.

\emph{Ахилл} : У вас не возникало мысли, друзья, что утверждения Няунамара могут быть ошибочными?

\emph{Краб} : Честно говоря, Ахилл, поскольку я такой консерватор, я сразу об этом подумал. На самом деле, мне сначала показалось, что это совершеннейшая утка. Но потом я решил, что мало кому удалось бы высосать из пальца такие странные и сложные результаты. В конце концов, все свелось к вопросу: «Что более вероятно: что письмо написано очень талантливым шарлатаном, или что его написал гениальный математик?» И вскоре я вычислил, что более вероятно первое.

\emph{Ахилл} : Но вы все же проверили его утверждения?

\emph{Краб} : К чему? Мой вероятностный аргумент намного лучше любого математического доказательства --- он вполне убедителен. Однако г-жа Черепаха настояла на строгом доказательстве и мне пришлось проверить все результаты Няунамара. К моему удивлению, они оказались истинными! Я, однако, никогда не пойму, как ему удалось их получить. Наверное, при помощи удивительной и непостижимой восточной магии. Нам этого понять не дано. Это единственное объяснение, которое имеет хоть какой-то смысл.

\emph{Черепаха} : М-р Краб всегда легко поддавался на удочку мистицизма и всяких странных идей --- но мне трудно поверить в подобное объяснение. Я совершенно уверена, что всем вычислениям Няунамара можно найти соответствие в ортодоксальной математике. Я вообще думаю, что, занимаясь математикой, невозможно намного отклониться от традиционного курса.

\emph{Ахилл} : Это интересное мнение. Думаю, что оно связано с Тезисом Черча-Тюринга и другими подобными темами.

\emph{Краб} : О, давайте оставим в стороне все эти технические детали --- в такой чудесный денек! --- и насладимся лучше тишиной леса, пением птиц, и игрой солнечных зайчиков на почках и молодых листочках Хо!

\emph{Черепаха} : Я --- за! Как известно, все поколения Черепах испокон веков наслаждались красотой природы.

\emph{Краб} : Как и все поколения Крабов.

\emph{Ахилл} : Вы, часом, не захватили с собой вашу флейту, г-н Краб?

\emph{Краб} : Разумеется захватил! Я ношу ее с собой везде. Хотите послушать мелодию-другую?

\emph{Ахилл} : Это было бы прелестно, в таком пасторальном окружении. Вы играете по памяти?

\emph{Краб} : К сожалению, это выше моих возможностей. Мне нужны ноты. Но это не проблема --- у меня с собой есть несколько очень милых пьесок.

\emph{(Краб открывает тонкую папку и вытаскивает несколько листов бумаги. На первом листе --- следующие символы:}

Ea:\textasciitilde Sa=0

\emph{Он кладет этот лист на небольшой пюпитр, прикрепленный к флейте, и начинает играть. Мелодия оказывается очень короткой.)}

\emph{Ахилл} : Очень мило. \emph{(Заглядывает в ноты и лицо его принимает недоуменное выражение.)} Но зачем вы прикрепили к флейте это высказывание теории чисел?

\emph{(Краб смотрит на свою флейту, на ноты, затем с удивлением оглядывается.)}

\emph{Краб} : Я вас не понимаю. Какое высказывание теории чисел?

\emph{Ахилл} : «Ноль не следует ни за каким натуральным числом». Вон там в ваших нотах!

\emph{Краб} : Это третий постулюд Пеано --- у него их пять, и я переложил каждый из них для флейты. Это очень простые и легко запоминающиеся мелодии.

\emph{Ахилл} : Не понимаю, как можно играть математическое суждение, как если бы оно было нотами!

\emph{Краб} : Я же вам сказал это вовсе не математическое суждение, а постулюд Пеано! Хотите послушать еще один?

\emph{Ахилл} : С удовольствием.

\emph{(Краб кладет на пюпитр другой лист бумаги, на этот раз Ахилл не спускает с него глаз.)}

Я следил за вами, вы смотрели на эту ФОРМУЛУ на листе. Вы уверены, что это музыкальная нотация? Я мог бы поклясться, что это больше напоминает нотацию, которая используется в формализованной теории чисел!

\emph{Краб} : Как странно. Насколько я могу судить, это ноты, а не математические формулы. Конечно, я весьма далек от математики\ldots{} Хотите послушать еще какую-нибудь мелодию?

\emph{Ахилл} : Прошу вас. У вас есть с собой и другие ноты?

\emph{Краб} : У меня их целая куча.

\emph{(Он достает новый лист и кладет его на пюпитр. На листе написаны следующие символы:}

\textasciitilde Ea:Eb:(SSa*SSb)=SSSSSSSSSSSSS0

\emph{Пока Краб играет, Ахилл не спускает глаз с бумаги.)}

Правда, мило?

\emph{Ахилл} : Да, премиленькая пьеска. Но должен признаться, что ваши ноты все более походят в моих глазах на теорию чисел.

\emph{Краб} : Ах, боже мой! Это моя обычная музыкальная запись. Я просто не понимаю, где вы видите все эти внемузыкальные понятия в этой простенькой записи! Это же совершенно из другой оперы!

\emph{Ахилл} : Не согласитесь ли вы сыграть пьесу моего собственного сочинения?

\emph{Краб} : С превеликим удовольствием. Она у вас с собой?

\emph{Ахилл} : Нет, но мне кажется, что я смогу запросто что-нибудь сочинить.

\emph{Черепаха} : Должна вас предупредить, Ахилл, что Краб --- строгий судья мелодий, написанных не им; так что не расстраивайтесь, если он не оценит ваших усилий по достоинству.

\emph{Ахилл} : Благодарю вас за предупреждение, но я все же попробую\ldots{}

\emph{(Он пишет:}

((SSS0*SSS0)+(SSSS0*SSSS0))=(SSSSS0*SSSSS0)

\emph{Краб берет у него бумагу, смотрит на нее в раздумье, кладет ее на пюпитр и начинает играть.)}

\emph{Краб} : Что ж, это весьма неплохо, Ахилл. Мне нравится этот странный ритм.

\emph{Ахилл} : Что же в нем странного?

\emph{Краб} : Может быть вам, как композитору, он кажется естественным, но для меня переход с 3/3 на 4/4 и, затем, на 5/5 звучит экзотически. Если у вас есть для меня еще что-нибудь, я с удовольствием это сыграю.

\emph{Ахилл} : Благодарю вас. Я никогда раньше не сочинял музыку и должен признаться, что я представлял себе это занятие не совсем так. Что же, попробую написать еще одну мелодию. \emph{(Набрасывает на листе следующую строчку.)}

\textasciitilde Ea:Eb:(SSa*SSb)=SSSSSSSSSSSSSS0

\emph{Краб} : Гм-м\ldots{} Мне кажется, что это точная копия моей предыдущей пьесы.

\emph{Ахилл} : О, нет! Я добавил одно S. Теперь их стало четырнадцать.

\emph{Краб} : Ах да, конечно. \emph{(Он начинает играть, и на лице его появляется гримаса недовольства.)}

\emph{Ахилл} : Я надеюсь, что мое сочинение не кажется вам неблагозвучным!

\emph{Краб} : Вы написали эту мелодию по мотивам моей пьесы. Но боюсь, Ахилл, что вы не оценили всех ее тонкостей\ldots{} Хотя как я могу требовать этого от вас после одного прослушивания? Иногда бывает трудно понять, на чем основана красота произведения. Ее легко спутать с поверхностными аспектами пьесы и имитировать их. Но сама красота лежит глубоко внутри; может быть, она вообще недоступна анализу.

\emph{Ахилл} : Боюсь, что я немного запутался в вашем ученом комментарии. Я понял, что моя пьеса не удовлетворяет вашим высоким требованиям --- но в чем именно заключается моя ошибка? Может быть, вы согласитесь мне растолковать?

\emph{Краб} : Вашу композицию можно поправить, вставив в конце три --- а, впрочем, можно и пять --- S. Это создаст тонкий и неожиданный эффект.

\emph{Ахилл} : Понятно.

\emph{Краб} : Но пьесу можно изменить и по-другому. Мне, например, кажется, что лучше всего прибавить слева еще одну тильду. Это создаст приятный баланс между началом и концом. Две тильды подряд никогда не помешают --- они придают мелодии этакий веселый изгиб, знаете ли.

\emph{Ахилл} : А что, если я послушаюсь обоих советов сразу и напишу следующую пьеску?

\textasciitilde\textasciitilde EaEb:(SSa*SSb)=SSSSSSSSSSSSSSSSS0

\emph{Краб (с выражением болезненного неудовольствия)} : Ахилл, вам необходимо запомнить следующее: никогда не пытайтесь выразить слишком многое в одной и той же пьесе. Так можно достичь того порога, за которым пьесу уже ничто не спасет. Любые попытки исправить дело только все ухудшат. Как раз это произошло сейчас с вашей пьесой. Вы ухватились сразу за оба моих предложения, и вместо того, чтобы добавить пьесе красоты, это нарушило равновесие и лишило пьесу всякой привлекательности.

\emph{Ахилл} : Почему такие похожие пьесы, как ваша, с 13 S, и моя, с 14 S, кажутся вам настолько разными? По-моему, в остальном они идентичны!

\emph{Краб} : О, небо! Между ними --- огромная разница. Это как раз тот случай, когда словами невозможно передать всего того, что чувствует душа. Осмелюсь сказать, что таких правил, которые определяли бы, красива ли какая-либо пьеса, не существует, да и не может существовать. Чувство Красоты принадлежит исключительно царству разума; разума, которому жизненный опыт сообщил глубину, превосходящую любые объяснения, основанные на наборе правил.

\emph{Ахилл} : Я навсегда запомню эти идеи о природе Красоты. Наверное, нечто похожее можно сказать и об Истине?

\emph{Краб} : Без сомнения. Истина и Красота соотносятся, как\ldots{} как\ldots{}

\emph{Ахилл} : Как математика и музыка?

\emph{Краб} : Вы читаете мои мысли! Откуда вы знаете, что я как раз об этом думал?

\emph{Черепаха} : Ахилл очень непрост, м-р Краб. Вы его недооцениваете.

\emph{Ахилл} : Вы думаете, что может существовать связь между истинностью или ложностью определенного математического утверждения и красотой (или отсутствием таковой) соответствующей музыкальной пьесы? Или же это лишь моя фантазия?

\emph{Краб} : Мне кажется, вы заходите слишком далеко. Говоря о связи музыки и математики, я выражался символически, понимаете? Что касается прямой связи между музыкальной пьесой и математическим утверждением, я питаю на этот счет глубочайшие сомнения. Позвольте мне дать вам скромный совет: не тратьте время на подобные пустые выдумки.

\emph{Ахилл} : Вы, без сомнения, правы. Это было бы совершенно бесполезно. Лучше я сосредоточусь на оттачивании моего музыкального вкуса, придумав еще несколько пьес. Согласитесь ли вы быть моим наставником, м-р Краб?

\emph{Краб} : Буду счастлив послужить проводником на вашем пути к пониманию музыки.

\emph{(Ахилл берет карандаш и погрузившись, казалось бы, в состояние глубочайшего сосредоточения, пишет:}

\&\#923;00aA'V\textasciitilde\&\#923;\&\#923;:b+cS(EE=0\&\#923;э((\textasciitilde d)\textless V(AS*+(\textgreater V

\emph{(У Краба от удивления лезут глаза на лоб.)}

Вы действительно хотите, чтобы я сыграл это\ldots{} это\ldots{} чем бы это ни было?

\emph{Ахилл} : О, пожалуйста, прошу вас!

\emph{(Краб начинает с трудом играть.)}

\emph{Черепаха} : Браво! Браво! Скажите, Ахилл, ваш любимый композитор, случайно, не Джон Кейдж?

\emph{Ахилл} : Сказать по правде, он мой любимый анти-композитор. Так или иначе, я рад, что вам понравилась МОЯ музыка.

\emph{Краб} : Может быть, прослушивание подобной какофонии кажется вам забавным, но я уверяю вас, что для чувствительного уха композитора эти чудовищные, режущие диссонансы и бессмысленные ритмы --- настоящее мучение. Я-то думал, Ахилл, что у вас есть музыкальный вкус. Неужели ваши предыдущие пьесы получились хорошо по чистой случайности?

\emph{Ахилл} : О, простите меня пожалуйста, м-р Краб. Я просто решил исследовать возможности вашей музыкальной нотации. Я хотел услышать, какие звуки получаются, когда я пишу некую последовательность символов, а заодно узнать, как вы оцениваете пьесы, написанные в разных стилях.

\emph{Краб (ворчливо)} : Я вам, знаете, не автоматическая музыкальная машина и не мусорное ведро для музыкальных отбросов.

\emph{Ахилл} : Умоляю, простите. Но я все же чувствую, что сочинение этой пьески меня многому научило и я уверен, что теперь смогу писать гораздо лучшую музыку. И если вы согласитесь сыграть еще одно мое сочинение, ваше мнение о моих композиторских способностях непременно изменится в лучшую сторону.

\emph{Краб} : Хорошо. Давайте попытаемся.

\emph{(Ахилл пишет:}

Aa:Ab:\textless(a*a)=(SS0*(b*b))эa=0\textgreater{}

\emph{и Краб играет.)}

Вы были правы, Ахилл. Кажется, ваши музыкальные способности к вам вернулись. Эта пьеса --- настоящая драгоценность! Как вам удалось ее написать? Я никогда не слыхал ничего подобного. Она следует всем законам гармонии, и, тем не менее, в ней есть некая --- как бы это сказать --- иррациональная привлекательность. Я не могу определить этого точнее, но именно это меня в ней притягивает.

\emph{Ахилл} : Я так и думал, что вам это понравится.

\emph{Черепаха} : Как называется ваше сочинение, Ахилл? Мне кажется, что ему подошло бы название «Песнь Пифагора». Вы наверное, помните, что Пифагор был одним из первых, кто стал изучать музыкальные звуки.

\emph{Ахилл} : Верно. Это очень хорошее название.

\emph{Краб} : Кроме того не Пифагор ли открыл, что отношение двух квадратов никогда не равняется двум?

\emph{Черепаха} : Думаю что вы правы. В свое время это считалась еретическим открытием, поскольку никто раньше не догадывался о существовании чисел --- таких как квадратный корень из двух --- которые нельзя представить, как отношение двух целых чисел. Это открытие было ужасно и для самого Пифагора. Он решил что в абстрактном мире чисел обнаружился неожиданный и кошмарный дефект, могу себе представить, в какое отчаяние пришел бедняга.

\emph{Ахилл} : Простите вы кажется сказали что-то о чае? Кстати --- где же та знаменитая чайная? Долго нам еще карабкаться?

\emph{Черепаха} : Не волнуйтесь мы будем на месте через пару минут.

\emph{Ахилл} : Гм-м. Я как раз успею просвистеть вам мотивчик, который сегодня утром услышал по радио.

\emph{Краб} : Погодите минутку, достану бумагу. Я хочу записать эту мелодию. \emph{(Копается в папке и вытаскивает чистый лист )} Готово.

\emph{(Ахилл начинает свистеть, мелодия оказывается довольно длинной. Краб быстро записывает, стараясь не отстать.)}

Можете ли вы повторить несколько последних тактов?

\emph{Ахилл} : Конечно.

\emph{(После нескольких повторений Краб, наконец, заканчивает и с гордостью показывает свою запись}

\textless((SSSSS0*SSSSS0)+(SSSSS0*SSSSS0))=(SSSSSSS0*SSSSSSS0)+(S0*S0))

\&\#923;\textasciitilde Eb:\textless Ec:(Sc+b)=((SSSSSSS0 SSSSSSS0)+(S0 S0))\&\#923;Ed:Ed':Ee:Ee'

\textless\textasciitilde\textless d=eVd=e'\textgreater\&\#923;\textless b=((Sd*Sd)+(Sd'*Sd'))Ab=((Se*Se)+(Se'*Se'))\textgreater\textgreater\textgreater\textgreater{}

\emph{Затем он берет флейту и играет записанную им мелодию)}

\emph{Черепаха} : Интересно это похоже на индийскую мелодию.

\emph{Краб} : Нет, мне кажется для индийской мелодии она слишком проста. Впрочем, я не специалист.

\emph{Черепаха} : Вот мы и пришли! Где мы сядем, на веранде?

\emph{Краб} : Если вы не возражаете, лучше сесть внутри, я уже и так слишком долго был на солнце.

\emph{(Они входят в чайную, полную народа, садятся за единственный свободный столик и заказывают чай с пирожными. Не проходит и получаса, как им приносят поднос с аппетитными сладостями, каждый выбирает свое любимое пирожное.)}

\emph{Ахилл} : Знаете, м р К, мне бы хотелось услышать ваше мнение о мелодии, которую я только что сочинил.

\emph{Краб} : Покажите, можете записать ее на этой салфетке.

\emph{(Ахилл пишет}

Aa:Eb:Ec:\textless\textasciitilde Ed:Ee:\textless(SSd*SSe)=bV(SSd*SSe)=c\textgreater\&\#923;(a+a)=(b+c)\textgreater{}

\emph{Краб и Черепаха с интересом изучают его запись)}

\emph{Черепаха} : Как вы думаете, м-р К, это красивая пьеса?

\emph{Краб} : Гм-м\ldots{} Я считаю\ldots{} Мне кажется\ldots{} \emph{(В явном замешательстве ерзает на стуле.)}

\emph{Ахилл} : В чем дело? Достоинства этой пьесы оказалось определить труднее, чем достоинства других моих сочинений?

\emph{Краб} : Э-э-э\ldots{} Нет, нет, это совсем не то. Как бы это сказать\ldots{} дело в том, что мне надо УСЛЫШАТЬ пьесу, чтобы иметь возможность о ней судить.

\emph{Ахилл} : Так за чем же дело стало? Прошу вас, сыграйте мою пьеску и скажите нам, находите ли вы ее красивой.

\emph{Краб} : Конечно, конечно\ldots{} Я был бы чрезвычайно рад сыграть вашу пьесу; вот только\ldots{}

\emph{Ахилл} : Что случилось? Вы не можете сыграть эту пьесу? Почему вы колеблетесь?

\emph{Черепаха} : Неужели вы не понимаете, Ахилл, что исполнить вашу просьбу было бы невежливо и неделикатно по отношению к посетителям и работникам этой замечательной чайной?

\emph{Краб (с внезапным облегчением)} : Верно! Мы не имеем права навязывать другим свою музыку.

\emph{Ахилл (подавленно)} : О, какая жалость\ldots{} А я-то ТАК надеялся узнать ваше мнение об этой мелодии!

\emph{Краб} : Ух ты! Чуть не вляпался!

\emph{Ахилл} : Как? Что значит это замечание?

\emph{Краб} : Да так, ничего. Просто тот господин чуть не наступил на пирожные, рассыпанные официантом минуту назад. Сегодня здесь вообще необычайное оживление, яблоку негде упасть.

\emph{Черепаха} : Дело в том, что сегодня будет известно, кто станет счастливым обладателем приза ежегодной лотереи. Раньше в этой лотерее участвовала и я, но уже давно отчаялась. Приз простой --- самовар и набор индийского чая, --- но получить его хочется многим, поэтому сегодня здесь столько народу.

\emph{Ахилл} : Вы хотите сказать, что сегодня --- день розыгрыша?

\emph{Краб} : Вот именно, Ахилл.

\emph{Ахилл} : А-а, понятно, я это запомню.

\emph{Краб} : Что ж, пожалуй, мне пора ползти домой. Эта суета вокруг розыгрыша меня порядком утомила, а мне еще предстоит утомительный спуск по крутому склону.

\emph{Ахилл} : До встречи; и спасибо за урок композиции!

\emph{Краб} : Я и сам получил большое удовольствие; надеюсь, что когда-нибудь мы продолжим наш обмен сочинениями.

\emph{Ахилл} : Буду ждать этого дня с нетерпением. До свидания!

\emph{Черепаха} : До свидания, м-р Краб.

\emph{(И краб уползает вниз по холму.)}

\emph{Ахилл} : Что за создание! Это ползет маг и волшебник музыки, блестящий флейтист и композитор. Мне кажется, что он в четыре раза умнее любого из своих собратьев. Или даже в пять ---

\emph{Черепаха} : Как вы уже сказали в начале --- и, кажется, собираетесь продолжать говорить во веки веков!

