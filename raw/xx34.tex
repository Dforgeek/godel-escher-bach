\subsubsection{Ария с разнообразным вариациями}

\emph{Черепаха пришла составить компанию своему другу Ахиллу, который в последнее время стал страдать бессоницей.}

\emph{Черепаха} : Сочувствую вам, Ахилл; бессоница --- пренеприятная вещь! Надеюсь, мое общество немного отвлечет вас от тех невыносимых мыслей, что не дают вам забыться сном. Может быть, мне удастся навеять на вас такую скуку, что вы, наконец, сможете заснуть --- в таком случае, мой визит принесет вам безусловную пользу.

\emph{Ахилл} : Увы\ldots{} на мне уже пробовали руку чемпионы нудности и скуки --- вы им и в подметки не годитесь --- и все, к сожалению, без толку. Нет, г-жа Ч, я пригласил вас с тем, чтобы вы развлекли меня своими рассказами о теории чисел --- надеюсь, что это скрасит мне долгие ночные часы. Видите ли, я обнаружил, что немного теории чисел весьма успокоительно действует на мои расстроенные нервы.

\emph{Черепаха} : Хорошенькая мысль, ничего не скажешь! Знаете, это мне напоминает о бедном графе Кайзерлинге.

\emph{Ахилл} : Кто это такой?

\emph{Черепаха} : О, был такой саксонский граф в восемнадцатом веке --- захудалый график, говоря по правде, но благодаря ему\ldots{} История довольно забавная --- хотите послушать?

\emph{Ахилл} : Прошу вас!

\emph{Черепаха} : Однажды наш добрый граф заболел бессоницей. Случилось так, что в том же городе жил тогда известный музыкант; граф Кайзерлинг решил заказать ему серию вариаций с тем, чтобы его придворный клавесинист играл бы их графу, когда тот не мог заснуть. Граф надеялся, что так время пролетит быстрей и приятней.

\emph{Ахилл} : И как, удалось местному композитору выполнить это требование?

\emph{Черепаха} : Кажется, да, поскольку граф его щедро вознаградил: он подарил ему золотой бокал с сотней луидоров внутри.

\emph{Ахилл} : Невероятно! Интересно, где сам граф раздобыл этот бокал с луидорами?

\emph{Черепаха} : Может быть, он увидел такой бокал в музее и тот ему понравился?

\emph{Ахилл} : Вы намекаете на то, что граф его попросту украл?

\emph{Черепаха} : Ну, зачем же так грубо\ldots{} Знаете, в те дни на это смотрели не так, и графам многое позволялось. Так или иначе, графу музыка явно пришлась по вкусу, поскольку он беспрестанно просил своего клавесиниста --- совсем еще мальчишку, по имени Гольдберг, --- сыграть ту или иную из этих тридцати вариаций. В результате, по иронии судьбы, вариации стали связаны с именем молодого Гольдберга, а не графа.

\emph{Ахилл} : Вы имеете в виду, что композитором был Бах, и что это произведение --- так называемые «Гольдберг-вариации»?

\emph{Черепаха} : Вы угадали! На самом деле, эта пьеса сначала именовалась «Ария с различными вариациями»; в ней тридцать вариаций. Знаете ли вы, как Бах построил эти великолепные вариации?

\emph{Ахилл} : Я весь внимание.

\emph{Черепаха} : Каждая из пьес, кроме последней, построена на одной и той же теме, которую он назвал «арией». В действительности, пьесы связаны скорее не общей мелодией, а одинаковым гармоническим фоном. Мелодии могут варьироваться, но в их основе --- одна и та же тема. Только в последней вариации Бах позволил себе некоторую вольность --- это что-то вроде «конца после конца». Там содержатся странные музыкальные идеи, почти не связанные с первоначальной темой --- а именно, две немецкие народные песенки. Эта вариация называется «quodlibet».

\emph{Ахилл} : А что еще необыкновенного в «Гольдберг-вариациях»?

\emph{Черепаха} : Каждая третья вариация построена в форме канона; в первом из них оба голоса вступают на одной и той же ноте; во втором, один из голосов вступает НА ОДНУ ступень ВЫШЕ, в третьем --- НА ДВЕ, и так далее, до последнего канона, в котором голоса отстоят ровно на девять интервалов. Десять канонов, и все они ---

\emph{Ахилл} : Подождите минутку. Мне помнится, я что-то слышал о том, что недавно обнаружили еще четырнадцать Гольдберг-канонов!

\emph{Черепаха} : Вы, случайно, прочли это не в том самом журнале, что недавно оповестил мир о сенсационном открытии четырнадцати новых дней в ноябре?

\emph{Ахилл} : Да нет, это правда. Некий музыковед по имени Вольф прослышал о том, что в Страсбурге хранится специальная копия «Гольдберг-вариаций»; он поехал туда и, к своему удивлению, на задней обложке этой копии, что-то вроде «конца после конца», он нашел четырнадцать новых канонов, основанных на первых восьми нотах темы. Так что на самом деле Гольдберговых вариаций сорок четыре, а не тридцать.

\emph{Черепаха} : Сорок четыре, пока какой-нибудь музыковед не обнаружит еще нескольких в самом невероятном месте. И хотя это кажется маловероятным, все еще возможно, что найдутся новые вариации, и затем еще новые, и еще\ldots{} Это может продолжаться до бесконечности! Мы можем так никогда и не узнать, когда у нас будут полные «Гольдберг-вариаций».

\emph{Ахилл} : Престранная идея. На самом деле, все считают, что это последнее открытие было просто счастливой случайностью, и что теперь у нас есть все существующие вариации. Но если предположить, что вы правы, и что найдутся еще какие-нибудь, мы должны быть к этому готовы. Тогда название «Гольдберг-вариаций» изменит свое значение и будет означать не только уже известные вариации, но и те, которые могут быть найдены в дальнейшем. Это число --- назовем его «g» --- разумеется, не бесконечно, но знать то, что g конечно, --- это не то же самое, что знать его величину. Следовательно, этой информации недостаточно, чтобы определить, когда найденная вариация окажется действительно последней.

\emph{Черепаха} : Это верно.

\emph{Ахилл} : Скажите мне, когда Бах создал эти знаменитые вариации?

\emph{Черепаха} : В 1742 году, когда он был кантором в Лейпциге.

\emph{Ахилл} : 1742? Гмм\ldots{} Это число мне о чем-то напоминает\ldots{}

\emph{Черепаха} : Естественно, так как это очень интересное число: это сумма двух нечетных простых чисел, 1729 и 13.

\emph{Ахилл} : Вот это да! Удивительно, ничего не скажешь! Интересно, как часто можно встретить четное число, обладающее тем же свойством. Посмотрим\ldots{}

6 = 3 + 3

8 = 3 + 5

10 = 3 + 7 = 5 + 5

12 = 5 + 7

14 = 3 + 11 = 7 + 7

16 = 3 + 13 = 5 + 11

18 = 5 + 13 = 7 + 11

20 = 3 + 17 = 7 + 13

22 = 3 + 19 = 5 + 17 = 11 + 11

24 = 5 + 19 = 7 + 17 = 11~+ 13

26 = 3 + 23 = 7 + 19 = 13 + 13

28 = 5 + 23 = 11 + 17

30 = 7 + 23 = 11 + 19 = 13 + 17

Смотрите-ка, согласно моей табличке это кажется весьма обычным явлением. Но я пока не вижу в таблице никакой простой закономерности.

\emph{Черепаха} : Может быть, здесь никакой закономерности и нет.

\emph{Ахилл} : Что вы, конечно есть! У меня просто не хватает проницательности, чтобы ее заметить.

\emph{Черепаха} : Вы кажетесь совершенно в этом уверенным.

\emph{Ахилл} : У меня нет ни малейших сомнений. Интересно\ldots{} может ли быть, что все четные числа, за исключением 4, могут быть представлены в виде суммы двух нечетных простых чисел?

\emph{Черепаха} : Гмм\ldots{} этот вопрос мне о чем-то напоминает\ldots{} Ах, да! Вы не первый, кто задает мне этот вопрос. Теперь припоминаю, в 1742 году меня о том же спрашивал математик-любитель в ---

\emph{Ахилл} : Вы сказали, в 1742 году? Простите, что перебиваю, но мне кажется, что 1742 --- очень интересное число, это разность двух нечетных простых чисел: 1745 и 3.

\emph{Черепаха} : Вот это да! Удивительно, ничего не скажешь! Интересно, как часто можно встретить четное число, обладающее тем же свойством.

\emph{Ахилл} : Прошу вас, не позволяйте мне отвлекать вас.

\emph{Черепаха} : Ах, да --- я говорила, что в 1742 году один математик-любитель --- к сожалению, не могу вспомнить его имени --- послал письмо Эйлеру, который в то время находился в Потсдаме при дворе короля Фридриха Великого, и\ldots{} История довольно забавная, хотите послушать?

\emph{Ахилл} : Я весь внимание!

\emph{Черепаха} : Так вот, в своем письме тот любитель предложил Эйлеру следующую недоказанную гипотезу «Любое четное число, большее двух, можно представить как сумму двух простых чисел.» Как же бишь его звали\ldots{}

\emph{Ахилл} : Я припоминаю, что уже читал об этом в какой-то математической книге. Кажется, его звали Купфергёдель.

\emph{Черепаха} : Гмм\ldots{} Нет, это звучит слишком длинно.

\emph{Ахилл} : Может, тогда Зильберэшер?

\emph{Черепаха} : Да нет, это все не то. У меня то имя прямо на языке вертится\ldots{} ах, да! Гольдбах! Гольдбах была его фамилия.

\emph{Ахилл} : Так я и думал.

\emph{Черепаха} : Да, и ваши попытки мне здорово помогли. Странно, как иногда нам приходится искать в памяти, словно в библиотеке, когда пытаешься найти книгу, не зная ее шифра. Но вернемся к числу 1742.

\emph{Ахилл} : Действительно. Я хотел спросить, доказал ли Эйлер, что догадка Гольдбаха верна?

\emph{Черепаха} : Он никогда не считал, что на нее стоит тратить время. Однако не все математики разделяли это пренебрежение. В действительности, многие пытались доказать «Гипотезу Гольдбаха» --- она стала известна под этим именем.

\emph{Ахилл} : И удалось кому-нибудь ее доказать?

\emph{Черепаха} : Пока нет, но некоторые математики были очень близки к успеху. Например, в 1931 году Шнирельман, русский специалист по теории чисел, доказал, что любое число, четное или нечетное, может быть представлено как сумма не более чем 300 000 простых чисел.

\emph{Ахилл} : Какой странный результат. Какая же от него польза?

\emph{Черепаха} : Он ограничил проблему, переведя ее в финитную область. До Шнирельмановского доказательства думали, что если брать большие и большие четные числа, то чтобы их представить, понадобится все большее количество простых чисел. Для того, чтобы представить некоторые четные числа, мог понадобиться миллиард простых чисел! Теперь известно, что это не так --- суммы 300 000 (или меньше) простых чисел всегда оказывается достаточно.

\emph{Ахилл} : Теперь понимаю.

\emph{Черепаха} : Вскоре, в 1937 году, один хитроумный тип по имени Виноградов --- тоже русский --- еще больше приблизился к желанному результату: он доказал, что любое достаточно большое нечетное число может быть представлено в виде суммы не более чем ТРЕХ нечетных простых чисел. Например, 1937 = 641 + 643 + 653. Можно сказать, что нечетное число, которое может быть представлено как сумма трех нечетных простых чисел, имеет «свойство Виноградова». Таким образом, все достаточно большие нечетные числа обладают свойством Виноградова.

\emph{Ахилл} : Хорошо --- но что означает «достаточно большие»?

\emph{Черепаха} : Это значит, что некоторое количество нечетных чисел могут не иметь этого свойства, но существует некое число --- назовем его «\textbf{v} » --- после которого все нечетные числа обладают свойством Виноградова. Однако сам Виноградов не знал величины этого \textbf{v} . Так что, в каком-то смысле, \textbf{v} похоже на \textbf{g} --- конечное, но неизвестное число «Гольдберг-вариаций». Знать, что \textbf{v} конечно, --- это не то же самое, что знать его величину. Следовательно, этой информации недостаточно, чтобы определить, когда нечетное число, представимое более, чем тремя простыми числами, окажется действительно последним.

\emph{Ахилл} : А-а, понятно. Значит, любое достаточно большое число 2N может быть представлено как сумма ЧЕТЫРЕХ простых чисел, если сначала представить 2N - 3 в виде суммы трех простых, и затем снова прибавить 3.

\emph{Черепаха} : Совершенно верно. Другая попытка, близко подошедшая к доказательству этой гипотезы, представлена следующей Теоремой: «Все четные числа могут быть представлены в виде суммы простого числа и произведения по меньшей мере двух простых чисел».

\emph{Ахилл} : Как я погляжу, этот вопрос о сумме двух простых чисел завел нас с вами в настоящие дебри. Интересно, а куда бы мы забрались, если бы стали исследовать РАЗНОСТИ двух нечетных простых чисел? Держу пари, что мне удастся кое-что понять в этой головоломке, если я опять составлю табличку, на этот раз представляя четные числа в виде разности двух нечетных простых чисел. Посмотрим\ldots{}

2 = 5 --- 3, 7 --- 5, 13 --- 11, 19 --- 17 и т. д.

4 = 7 --- 3, 11 --- 7, 17 --- 13, 23 --- 19 и т. д.

6 = 11---5, 13 --- 7, 17 --- 11, 19 --- 13 и т. д.

8 = 11---3, 13 --- 5, 19 --- 11, 31 --- 23 и т. д.

10 = 13 --- 3, 17 --- 7, 23 --- 13, 29---19 и т. д.

Батюшки мои! Кажется, что различным вариантам нет конца! Но я пока не вижу в таблице никакой простой закономерности.

\emph{Черепаха} : Может быть, здесь никакой закономерности и нет.

\emph{Ахилл} : Ах, опять эти ваши туманные рассуждения о хаосе! Увольте, прошу вас --- на этот раз я не хочу об этом слышать.

\emph{Черепаха} : Вы считаете, что любое четное число может каким-то образом быть представлено в виде разности двух нечетных простых чисел?

\emph{Ахилл} : Из моей таблички следует, что да. Но может быть, и нет\ldots{} Однако так мы далеко не уедем!

\emph{Черепаха} : Со должным уважением позволю себе заметить, что в эти материи можно проникнуть и поглубже.

\emph{Ахилл} : Забавно, насколько эта проблема схожа с первоначальным вопросом Гольдбаха. Надо бы назвать ее «Гольдбах-вариации».

\emph{Черепаха} : И правда. Однако позвольте мне указать вам на огромную разницу между Гипотезой Гольдбаха и этой Гольдбах-вариацией. Предположим, что некое четное число 2N обладает «свойством Гольдбаха», если оно равняется СУММЕ двух нечетных простых чисел, и «свойством Черепахи», если оно равняется РАЗНОСТИ двух нечетных простых чисел.

\emph{Ахилл} : Мне кажется, справедливей называть это «свойством Ахилла». В конце концов, эту задачу предложил я!

\emph{Черепаха} : Я собиралась предложить, чтобы мы считали, что число, у которого НЕТ свойства Черепахи, обладает «свойством Ахилла».

\emph{Ахилл} : А, ну ладно\ldots{}

\emph{Черепаха} : Как вы думаете, обладает ли триллион свойством Гольдбаха или свойством Черепахи? Разумеется, он может иметь оба свойства\ldots{}

\emph{Ахилл} : Я, конечно, могу над этим подумать, но сомневаюсь, чтобы я мог ответить на ваши вопросы.

\emph{Черепаха} : Не сдавайтесь так быстро. Представьте, что я попросила вас ответить на один из них. Над каким вопросом вы предпочли бы подумать?

\emph{Ахилл} : Наверное, мне пришлось бы бросить монетку. По-моему, между этими вопросами нет особой разницы.

\emph{Черепаха} : Ага! Вот тут вы и ошибаетесь. Между ними огромная разница! Если вы выберете свойство Гольдбаха, где идет речь о СУММАХ, то вам придется иметь дело только с простыми числами между двумя и триллионом, не так ли?

\emph{Ахилл} : Разумеется.

\emph{Черепаха} : А раз так, то ваш поиск рано или поздно ОБЯЗАТЕЛЬНО КОНЧИТСЯ\ldots{}

\emph{Ахилл} : А-а-а\ldots{} Понятно. С другой стороны, если я начну работать над представлением триллиона в форме РАЗНОСТИ двух простых чисел, я могу использовать сколь угодно большие числа. Они могут быть так велики, что мне придется просидеть за работой триллион лет.

\emph{Черепаха} : Хуже того, они могут вообще НЕ СУЩЕСТВОВАТЬ! В конце концов, именно в этом и состоял вопрос: существуют ли такие простые числа? Нас не интересовало, как велики они могут оказаться.

\emph{Ахилл} : Вы правы. Если бы они не существовали, мой поиск мог продолжаться вечно, и я не ответил бы ни да ни нет. И тем не менее, ответ был бы отрицательным.

\emph{Черепаха} : Таким образом, если у вас есть какое-то число и вы хотите проверить, обладает ли оно свойством Гольдбаха или свойством Черепахи, разница будет заключаться в следующем: поиск свойства Гольдбаха ОБЯЗАТЕЛЬНО ЗАКОНЧИТСЯ, в то время как поиск свойства Черепахи ПОТЕНЦИАЛЬНО БЕСКОНЕЧЕН --- у нас нет никаких гарантий. Он может запросто тянуться до бесконечности, не давая нам никаких ответов. И тем не менее, в некоторых случаях он может закончиться на первом же шаге.

\emph{Ахилл} : Я вижу, что между свойством Гольдбаха и свойством Черепахи действительно существует огромная разница.

\emph{Черепаха} : Вы правы; эти проблемы, столь схожие по виду, на самом деле имеют дело с весьма различными свойствами. Гипотеза Гольдбаха утверждает, что все четные числа обладают свойством Гольдбаха; вариация Гольдбаха --- что все четные числа обладают свойством Черепахи. Обе задачи еще не решены, и интересно то, что хотя они звучат очень похоже, в них идет речь об очень разных свойствах целых чисел.

\emph{Ахилл} : Я понимаю, что вы имеете в виду. Для любого четного числа свойство Гольдбаха --- вполне определенная вещь, так как мы знаем, что если поискать, всегда можно узнать, обладает ли данное число этим свойством. Свойство Черепахи, с другой стороны, гораздо менее определенно, так как одной грубой силой тут не возьмешь --- сколько ни ищи, а ответа можешь так и не найти.

\emph{Черепаха} : Все же мне кажется, должны существовать какие-нибудь способы получше; может быть с помощью одного из них мы смогли бы всегда доходить до конца, устанавливая, есть ли у данного числа свойство Черепахи.

\emph{Ахилл} : Но и тогда поиск кончался бы только в случае положительного ответа.

\emph{Черепаха} : Не обязательно. Должно существовать доказательство того, что если поиск продолжается дольше определенного времени, ответ должен быть отрицательным. Мне кажется, что можно найти и совершенно ИНОЙ способ поиска простых чисел, способ, не требующий грубой силы. Он гарантировал бы, что если такие числа существуют, мы их найдем. А если нет --- сможем это доказать. Так или иначе, в любом случае поиск завершался бы даже в случае отрицательного ответа. Но я не уверена, что все это можно доказать. Поиск в бесконечных пространствах --- дело непростое, знаете ли\ldots{}

\emph{Ахилл} : Значит, на данный момент вы не знаете ни одного способа конечного поиска для нахождения свойства Черепахи --- но тем не менее, такой способ МОЖЕТ существовать.

\emph{Черепаха} : Верно. Можно бы, конечно, начать поиск такого поиска --- но я не гарантирую, что подобный «мета-поиск», в свою очередь, окажется конечным.

\emph{Ахилл} : Удивительно то, что если у какого-нибудь четного числа --- скажем, у триллиона --- не оказалось бы свойства Черепахи, то этим оно было бы обязано бесконечному числу единиц информации. Забавно подумать, что все эта информация может быть собрана в пучок и названа, согласно вашему галантному предложению, «свойством Ахилла» одного триллиона. На самом деле, это свойство всей системы, а не одного числа.

\emph{Черепаха} : Это интересное наблюдение, Ахилл, но я все-таки думаю, что правильнее относить этот факт именно к триллиону. Представьте себе для примера простенькое утверждение «29 --- простое число». На самом деле это означает, что 2, умноженное на 2, не равно 29; 5, умноженное на 6, не равно 29, и так далее. Вы согласны?

\emph{Ахилл} : Ну, предположим\ldots{}

\emph{Черепаха} : Тем не менее вы спокойно можете собрать все эти факты вместе, связать их в пучок и привязать к числу 29, сказав «29 --- простое число», не так ли?

\emph{Ахилл} : Да\ldots{}

\emph{Черепаха} : И при этом число фактов бесконечно, поскольку мы можем также сказать, что «4444, умноженное на 3333, не равно 29».

\emph{Ахилл} : Строго говоря, вы правы. Однако мы оба знаем, что 29 не может равняться произведению двух чисел, каждое из которых больше него самого. А раз так, то, говоря «29 --- простое число», мы учитываем только ограниченное количество из всех фактов, известных нам об умножении.

\emph{Черепаха} : Вы можете смотреть на это и так, но учтите, что сам факт, что 29 не может быть произведении двух чисел, больших чем оно само, основан на структуре численной системы в целом. В этом смысле, этот факт сам включает в себя бесконечное множество фактов. Вам, Ахилл, никуда не деться от того, что, говоря «29 --- простое число», вы на самом деле утверждаете бесконечное множество фактов.

\emph{Ахилл} : Не знаю, не знаю --- мне это кажется лишь одним фактом.

\emph{Черепаха} : Это происходит потому, что эти бесконечные факты составляют часть вашего предыдущего знания; они косвенно влияют на то, как вы смотрите на вещи. Вы не замечаете бесконечности, так как она скрыта внутри образов, возникающих в вашем сознании.

\emph{Ахилл} : Наверное, вы правы. И все-таки мне кажется весьма странным объединить свойства системы чисел как целого и именовать результат «простотой числа 29».

\emph{Черепаха} : Может, оно и выглядит странным, но это весьма полезный способ смотреть на вещи. Давайте теперь вернемся к вашей гипотезе. Если, как вы предположили, триллион обладает свойством Ахилла, то какое бы простое число мы к нему ни прибавили, мы никогда не получим другого простого числа. В таком положении дел было бы виновато бесконечное количество отдельных математических «событий». Исходят ли все эти «события» из одного и того же источника? Имеют ли они общую причину? Если нет, то значит, за наш факт ответственно некое «бесконечное совпадение» , а не какая-либо закономерность.

\emph{Ахилл} : «Бесконечное совпадение»? В царстве натуральных чисел НИЧТО не бывает случайно --- любое событие там основано на некой регулярности, скрытой или явной. Возьмите вместо триллиона семерку --- она меньше и с ней полегче обращаться. У 7 есть свойство Ахилла.

\emph{Черепаха} : Вы уверены?

\emph{Ахилл} : Конечно, и вот почему. Если к 7 добавить 2, то вы получите 9 --- число не простое; если же добавить к 7 любое другое простое число, то вы будете складывать два нечетных простых числа, результатом чего будет четное число --- так что простого числа снова не получается. Так что здесь, если можно так выразиться, «Ахильность» семерки объясняется не бесконечным числом причин, но всего двумя, что весьма далеко от «бесконечного совпадения». Это только подтверждает то, что я уже сказал: для того, чтобы объяснить какую-либо арифметическую истину, нам никогда не понадобится бесконечное число причин. Если бы существовал такой арифметический факт, который был результатом бесконечного числа не связанных между собой совпадений, то мы никогда не смогли бы найти конечное доказательство этой истины --- а это просто смешно.

\emph{Рис. 71. М. К. Эшер. «Порядок и хаос» (литография, 1950).}

\emph{Черепаха} : Это звучит вполне разумно, и вы не первый, кто высказывает подобное мнение. Однако ---

\emph{Ахилл} : Неужели есть кто-нибудь, кто не согласен с этой точкой зрения? Такие люди должны верить в «бесконечные совпадения», в наличие хаоса среди самого совершенного, гармонического и прекрасного среди всех творений --- системы натуральных чисел.

\emph{Черепаха} : Может, так они и считают --- но задумывались ли вы когда-нибудь о том, что хаос может быть неотъемлемой частью красоты и гармонии?

\emph{Ахилл} : Хаос --- часть совершенства? Порядок и хаос в приятном единении? Да это же ересь!

\emph{Черепаха} : Ваш любимый художник, М. К. Эшер, как-то провел эту еретическую идею в одной из своих картин. Кстати, раз уж мы заговорили о хаосе, я думаю, вам будет интересно услышать о двух различных категориях поиска, каждая из которых непременно закончится.

\emph{Ахилл} : Разумеется.

\emph{Черепаха} : Пример первого --- нехаотичного --- вида поиска мы находим в проверке на свойство Гольдбаха. Надо просто перебирать простые числа, меньшие 2N, и если какая-нибудь пара таких чисел при сложении дает 2N, то следовательно 2N обладает свойством Гольдбаха; в противном случае, оно им не обладает. Подобная проверка не только наверняка закончится --- вы даже можете предсказать, КОГДА она закончится.

\emph{Ахилл} : Значит, это ПРЕДСКАЗУЕМО КОНЧАЮЩАЯСЯ проверка. Теперь вы, наверное, скажете мне, что некоторые теоретико-числовые свойства нуждаются в другого рода проверке, которая когда-либо кончится, но неизвестно, когда?

\emph{Черепаха} : Вы как в воду глядите, Ахилл. И существование подобного типа проверки доказывает, что системе натуральных чисел в некотором роде присущ хаос.

\emph{Ахилл} : Я бы сказал, что об этой проверке просто слишком мало известно. Если как следует поработать, я уверен, что можно было бы определить, как долго она продлится, прежде чем придет к концу. Ведь должен же быть какой-то смысл в структурах целых чисел! Никогда не поверю, что эта система хаотична и непредсказуема.

\emph{Черепаха} : Ваша интуитивная вера понятна, но не всегда оправдана. Разумеется, во многих случаях вы совершенно правы --- если мы чего-то не знаем, из этого еще не следует, что это вообще непознаваемо! Но есть и такие свойства целых чисел, для которых можно доказать существование конечной процедуры проверки, а также --- что невозможно заранее определить, как долго эта процедура будет продолжаться.

\emph{Ахилл} : В это трудно поверить. Словно сам черт забрался в божественно прекрасное здание натуральных чисел!

\emph{Черепаха} : Может быть, вам будет приятно узнать, что совсем не легко определить свойства, для которых существует конечная, но не ПРЕДСКАЗУЕМО конечная процедура проверки. Большинство «естественных» свойств целых чисел допускают предсказуемо конечные процедуры. Например, так можно проверить, является ли число простым, квадратом или десятой степенью какого-либо числа.

\emph{Ахилл} : Да, это нетрудно. Но мне любопытно узнать, что это за свойство, для которого существует конечная, но непредсказуемая процедура проверки?

\emph{Черепаха} : Это для меня слишком сложно, в особенности, когда я такая сонная. Лучше приведу вам пример свойства, которое весьма легко определить, но для которого неизвестна конечная процедура проверки. Заметьте, я не хочу сказать, что она никогда не будет открыта, --- просто пока она еще не найдена. Для начала надо выбрать какое-нибудь число --- предоставляю эту честь вам, Ахилл!

\emph{Ахилл} : Как насчет 15?

\emph{Черепаха} : Превосходно. Вы начинаете с вашего числа; если оно НЕЧЕТНО, вы умножаете его на три и прибавляете 1. Если оно ЧЕТНО, вы берете его половину. После этого мы повторяем процесс. Назовем число, которое таким образом рано или поздно превратится в 1, ИНТЕРЕСНЫМ, и число, которое не станет 1, НЕИНТЕРЕСНЫМ.

\emph{Ахилл} : Интересное ли число 15? Посмотрим:

15 НЕЧЕТНО, так что я превращаю его в 3n + 1: 46

46 ЧЕТНО, так что я делю его на два: 23

23 НЕЧЕТНО, так что я превращаю его в Зn + 1: 70

70 ЧЕТНО, так что я делю его на два: 35

35 НЕЧЕТНО, так что я превращаю его в Зn + 1: 106

106 ЧЕТНО, так что я делю его на два: 53

53 НЕЧЕТНО, так что я превращаю его в Зn + 1: 160

160 ЧЕТНО, так что я делю его на два: 80

80 ЧЕТНО, так что я делю его на два: 40

40 ЧЕТНО, так что я делю его на два: 20

20 ЧЕТНО, так что я делю его на два: 10

10 ЧЕТНО, так что я делю его на два: 5

5 НЕЧЕТНО, так что я превращаю его в Зn + 1: 16

16 ЧЕТНО, так что я делю его на два: 8

8 ЧЕТНО, так что я делю его на два: 4

4 ЧЕТНО, так что я делю его на два: 2

2 ЧЕТНО, так что я делю его на два: 1.

Ух ты! Ничего себе путешествьице, от 15 до 1! Но я все же достиг цели. Это значит, что 15 обладает свойством «интересности». Хотелось бы узнать, какие числа НЕинтересные\ldots{}

\emph{Черепаха} : Вы заметили, что в этом простом процессе числа то возрастают, то уменьшаются?

\emph{Ахилл} : Я особенно удивился, когда после 13 шагов я получил 16 --- число, всего на 1 большее того , с которого я начал! В каком-то смысле, я почти вернулся к началу --- но в другом смысле, я был весьма далек от начала. Странно и то, что чтобы решить задачку, мне пришлось добраться до 160. Интересно, почему так получилось?

\emph{Черепаха} : Потому что потолок у этой задачки бесконечно высок, и заранее невозможно сказать, как высоко нам придется забраться. На самом деле, возможно, что вам придется все время карабкаться вверх, и вверх, и вверх, и никогда не спускаться больше, чем на несколько шагов.

\emph{Ахилл} : Правда? Наверное, такое возможно --- но что за странным совпадением это было бы! Для этого нам должны все время попадаться нечетные числа, за редким исключением. Сомневаюсь, чтобы такое было возможно, хотя, конечно, я не мог бы в этом поклясться.

\emph{Черепаха} : Проверьте-ка число 27. Имейте в виду, я ничего не обещаю. Но все-таки попробуйте когда-нибудь --- просто так, для развлечения. И я посоветовала бы вам запастись для этого большим листом бумаги.

\emph{Ахилл} : Гммм\ldots{} Интересно\ldots{} Знаете, мне все еще кажется странным ассоциировать интересность (или неинтересность) с начальным числом, поскольку совершенно ясно, что это --- свойство всей системы чисел.

\emph{Черепаха} : Я понимаю, что вы имеете в виду, но это ничем не отличается от высказывания «29 --- простое число» или «золото --- дорогой металл». Оба утверждения приписывают единственному объекту свойство, которым тот обязан контексту целой системы.

\emph{Ахилл} : Вы, наверное, правы. Проблема «интересности» весьма непроста, так как величина чисел все время колеблется, то возрастая, то уменьшаясь. Здесь ДОЛЖНА быть какая-то регулярность, хотя на вид это выглядит довольно хаотично. Прекрасно понимаю, почему еще никто до сих пор не нашел для «интересности» такой процедуры проверки, которая обязательно кончается.

\emph{Черепаха} : Кстати о кончающихся и некончающихся процедурах --- это мне напоминает об одном из моих друзей; он сейчас работает над своей книгой.

\emph{Ахилл} : Ах, как занимательно! Как же она называется?

\emph{Черепаха} : «Медь, серебро, золото --- этот неразрушимый сплав». Не правда ли, звучит интересно?

\emph{Ахилл} : Честно говоря, я что-то не совсем понимаю. Что общего между собой у меди, серебра и золота?

\emph{Черепаха} : Это ясно, как день.

\emph{Ахилл} : Вот если бы книга называлась «Гориллы, серебро, золото» или «Эму, золото\ldots» --- тогда бы я еще мог понять\ldots{}

\emph{Черепаха} : Может быть, вы предпочли бы «Медь, серебро, бабуины»?

\emph{Ахилл} : Безусловно! Но это действительное название какое-то совсем слабенькое. Никто его не поймет.

\emph{Черепаха} : Я скажу моему другу. Он (как и его издатель) будет только рад поменять название на более завлекательное.

\emph{Ахилл} : Приятно слышать. Но почему наш разговор напомнил вам об этой книге?

\emph{Черепаха} : Ах, да. Видите ли, там будет Диалог, в котором автор постарается запутать читателей, заставив их искать конец.

\emph{Ахилл} : Забавно. Как же он это сделает?

\emph{Черепаха} : Вы, безусловно, замечали, как некоторые писатели стараются наращивать напряжение поближе к концу своих историй --- но читатель, держа книгу в руках, ЗНАЕТ, что рассказ подходит к концу. Таким образом, у него есть дополнительная информация, которая действует как предупреждение. Напряжение и неизвестность немного подпорчены физической сущностью книги. Было бы гораздо лучше, если бы в конце романов писатели оставляли прокладку потолще.

\emph{Ахилл} : Прокладку?

\emph{Черепаха} : Именно; я имею в виду кучу печатных страниц, не имеющих никакого отношения к истории, но маскирующих ее скорое окончание.

\emph{Ахилл} : А-а, понятно. Таким образом конец истории может отстоять на, скажем, пятьдесят или даже сто страниц от последней страницы книги?

\emph{Черепаха} : Да. Это привнесло бы некоторый элемент сюрприза, поскольку читатель не будет знать заранее, сколько страниц относятся к прокладке и сколько --- собственно к истории.

\emph{Ахилл} : Такая система была бы эффективной, если бы не есть одна проблема. Представьте себе, что ваша прокладка была бы очевидной --- скажем, чистые страницы, куча «А» или случайные буквы. Тогда она была бы совершенно бесполезной.

\emph{Черепаха} : Согласна. Она должна быть похожа на обычные печатные страницы.

\emph{Ахилл} : Но даже беглого взгляда на страницу из какой-либо истории зачастую хватает, чтобы отличить ее от страницы из другой истории.

\emph{Черепаха} : Это верно. Я всегда представляла это так: вы кончаете одну историю и тут же пишете еще что-то, что весьма похоже на продолжение --- но в действительности это только прокладка, никак не соотносящаяся с вашей историей. Эта прокладка --- что-то вроде «конца после конца». В ней могут быть странные литературные идеи, совершенно не имеющие отношения к первоначальной теме.

\emph{Ахилл} : Ловко! Но тогда вам не удастся сказать, где находится действительный конец. Он сольется с прокладкой.

\emph{Черепаха} : Вот и мы с моим другом-писателем пришли к такому же заключению. Жаль, эта идея мне очень нравилась.

\emph{Ахилл} : Послушайте, у меня есть предложение. Переход между историей и прокладкой может быть написан таким образом, что внимательный читатель сможет сказать, где кончается одна и начинается другая. Может быть, ему придется над этим посидеть. Может быть, будет вообще невозможно предсказать, сколько времени это у него отнимет. Но издатель сможет дать гарантию, что достаточно тщательный поиск всегда придет к концу, даже если мы и не знаем наперед, как долго он будет продолжаться.

\emph{Черепаха} : Прекрасно; но что означает «достаточно тщательный»?

\emph{Ахилл} : Это значит, что читатель должен будет искать в тексте некую крохотную, но важную деталь, которая укажет на действительный конец. И ему придется исхитриться, чтобы среди множества подобных деталей найти настоящую.

\emph{Черепаха} : Что-то вроде изменения частоты букв или длины слов? Внезапная россыпь грамматических ошибок?

\emph{Ахилл} : Совершенно верно. Какое-то шифрованное послание, которое поможет внимательному читателю найти конец книги. Еще можно вывести новых персонажей или придумать события, несоответствующие остальной истории. Наивный читатель проглотит это, не задумываясь, в то время как умудренный опытом человек сможет точно указать, где проходит граница.

\emph{Черепаха} : Какая оригинальная идея, Ахилл. Я расскажу о ней другу и, может быть, он захочет вставить ее в свой Диалог.

\emph{Ахилл} : Этим он окажет мне честь.

\emph{Черепаха} : Знаете, боюсь, что я совсем засыпаю, Ахилл. Пойду-ка, пожалуй, пока я еще в силах добраться до дому.

\emph{Ахилл} : Мне было очень приятно, что вы просидели у меня так долго в такой поздний час только лишь с тем, чтобы составить мне компанию. Уверяю вас, что ваши теоретико-численные рассказы явились прекрасным противоядием против моего обычного верчения в постели. Кто знает, может быть, мне даже удастся сегодня заснуть. В знак благодарности позвольте преподнести вам подарок.

\emph{Черепаха} : Ах, Ахилл, что за глупости\ldots{}

\emph{Ахилл} : Для меня это одно удовольствие, г-жа Ч. Подойдите-ка к комоду; на нем лежит маленькая старинная шкатулка.

\emph{(Черепаха подходит к комоду.)}

\emph{Черепаха} : Неужели вы имеете в виду эту золотую шкатулку?

\emph{Ахилл} : Ее самую. Пожалуйста, примите ее в знак нашей дружбы.

\emph{Черепаха} : Премного вам благодарна, Ахилл. Гмм\ldots{} Что это за имена математиков на крышке, да еще по-английски? Что за интересный список\ldots{}

\textbf{D} e ~M o r g a n

A \textbf{b} e l

B o \textbf{o} l e

В~r о \textbf{u} w e r

S i e r \textbf{p} i n s k i

W e i e r \textbf{s} t r a s s

\emph{Ахилл} : По идее, это должно быть Полным Списком Всех Великих Математиков. Только я никогда не мог понять, почему буквы, идущие вниз по диагонали, написаны жирным шрифтом.

\emph{Черепаха} : Смотрите, тут внизу написано: «Отнимите 1 от диагонали, и вы найдете Баха в Лейпциге».

\emph{Ахилл} : Я это тоже видел, но не могу сообразить, что бы это значило. Так я не запутывался с тех пор, когда пытался заниматься философией. Особенно меня тогда смутил Кант --- оригинально, но уж больно туманно\ldots{}

\emph{Черепаха} : Прошу вас, ни слова о философии --- я слишком устала. Лучше поползу-ка я домой. \emph{(Машинально открывает шкатулку.)} Ах! Глядите, здесь внутри куча золотых монет! Да это же луидоры!

\emph{Ахилл} : Вы доставите мне огромное удовольствие, приняв эти деньги, г-жа Ч.

\emph{Черепаха} : Но\ldots{} Но\ldots{}

\emph{Ахилл} : Пожалуйста, без возражений. Шкатулка и золото --- ваши. И спасибо вам за несравненный вечер.

\emph{Черепаха} : Как мило с вашей стороны. Надеюсь, вам удастся заснуть: выпейте стаканчик теплого молока, поставьте на патефон вашу любимую пластинку, и пусть вам приснится эта странная Гипотеза Гольдбаха и ее Вариации\ldots{} Спокойной вам ночи. \emph{(Она берет золотую шкатулку, полную луидоров, и направляется к двери. В этот момент раздается громкий стук.)} Кто бы это мог быть в такой поздний час, Ахилл?

\emph{Ахилл} : Понятия не имею. Все это весьма подозрительно\ldots{} Знаете что, спрячьтесь-ка на всякий случай за комодом!

\emph{Черепаха} : Отличная мысль. \emph{(Заползает за комод.)}

\emph{Ахилл} : Кто там?

\emph{Голос} : Откройте, полиция!

\emph{Ахилл} : Входите, дверь не заперта!

\emph{(Входят два дюжих полицейских в новеньких, с иголочки, формах, со сверкающими кокардами на фуражках.)}

\emph{Полицейский} : Я --- лейтенант Сильвер, а это --- копертан Гулд. Проживает ли здесь некто по имени Ахилл?

\emph{Ахилл} : Это я.

\emph{Полицейский} : Мистер Ахилл, у нас есть все основания подозревать, что в вашей квартире находится золотая шкатулка с сотней луидоров. Она была украдена сегодня вечером из музея.

\emph{Ахилл} : Ах, батюшки!

\emph{Полицейский} : Она должна находиться здесь, потому что, кроме вас, подозревать некого. Придется вам пройти с нами\ldots{} \emph{(Достает ордер на арест.)}

\emph{Ахилл} : Господи, как я счастлив, что вы наконец пришли! Весь вечер я мучился, слушая Черепашьи вариации на тему золотых шкатулок. Надеюсь, вы меня освободите! Прошу вас, господа, загляните за комод, и вы увидите там настоящего преступника!

\emph{(Полицейские заглядывают за комод; там, среди пыли и паутины, они видят дрожащую Черепаху с золотой шкатулкой в лапах.)}

\emph{Полицейский} : Ага! Вот она, злодейка! Никогда бы на нее не подумал --- но поскольку она поймана с поличным\ldots{}

\emph{Ахилл} : Уведите поскорее отсюда эту преступницу, любезные господа. Слава Богу, мне уже никогда не придется слышать ни о ней, ни о ее Золотых Вариациях.

