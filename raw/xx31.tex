\subsubsection{ГЛАВА XI: Мозг и мысль}

Новый взгляд на мысль

С ПОЯВЛЕНИЕМ компьютеров люди начали работать над созданием «думающих машин», при этом они стали свидетелями престранных вариаций на тему мысли. Были созданы программы, чье мышление так же походило на человеческое, как движение заводной куклы --- на движение человека. Все странности нашего мышления, его слабые и сильные стороны, причуды и изменчивость вышли на поверхность, когда мы получили возможность экспериментировать с самодельными формами мышления --- или приближений к мышлению. В результате в течение последних двадцати лет мы развили новый взгляд на то, чем является и чем не является мысль. За это время выяснилось много нового о малом и о большом масштабах «аппаратуры» нашего мозга. Эти исследования пока не смогли ответить на вопрос о том, как мозг работает с идеями, но они, тем не менее, дают нам некоторое представление о биологических механизмах, управляющих нашим мышлением.

В следующих двух главах мы попытаемся соединить наши знания об искусственном интеллекте с некоторыми фактами, которые нам удалось узнать благодаря хитроумным экспериментам с мозгом животных и исследованиям процессов мышления, проведенных специалистами в области психологии. Мы начали разговор об этом в «Прелюдии» и в «Муравьиной фуге»; теперь поговорим о том же на более глубоком уровне.

Интенсиональность и экстенсиональность

Мысль должна зависеть \emph{от отражения действительности аппаратурой мозга} . В предыдущих главах мы разработали формальные системы, отражающие области математической действительности с помощью символов. До какой степени подобные формальные системы могут служить моделями обращения мозга с идеями?

В системе \textbf{pr} и затем в других, более сложных системах мы видели, как значение, в ограниченном смысле этого слова, возникает из изоморфизма, соотносящего типографские символы с числами, арифметическими действиями и отношениями, а строчки типографских символов --- с высказываниями. В мозгу нет никаких типографских символов, но есть кое-что получше: активные элементы, которые могут хранить информацию, а также передавать ее и получать новую информацию от других активных элементов. Таким образом, у нас есть активные символы вместо пассивных типографских символов. В мозгу правила смешаны с самими символами, в то время как на бумаге символы --- это статичные единицы, а правила находятся у нас в голове. Благодаря строгости формальных систем, которые мы до сих пор рассматривали, читатель может заключить, что изоморфизм между символами и реальными вещами --- это жесткое взаимно однозначное соответствие, что-то вроде ниток, соединяющих марионетку с ведущей ее рукой. Однако важно понимать, что это вовсе не так. В той же ТТЧ понятие «пятьдесят» может быть выражено различными символами, скажем:

((SSSSSSSO*SSSSSSSO)+(SO*SO))

и

((SSSSSO*SSSSSO)+(SSSSSO*SSSSSO))

То, что обе эти записи обозначают один и тот же номер, вовсе не ясно априори. Вы можете работать с каждым из этих выражений независимо, пока не наткнетесь на какую-нибудь теорему, которая заставит вас воскликнуть: «Да это же \emph{то самое} число!»

В вашей голове могут соседствовать различные мысленные образы одного и того же человека, например:

\emph{Человек, чью книгу я послал несколько дней тому назад другу в Польшу.}

\emph{Незнакомец, заговоривший со мной и моими приятелями в кафе сегодня вечером.}

То что оба эти образа обозначают одного и того же человека, вовсе не ясно априори. Они могут находиться в вашей голове раздельно, пока, разговаривая с незнакомцем, вы не наткнетесь на тему, которая поможет вам понять, что эти образы относятся к одному и тому же человеку: «Да, вы же \emph{тот} самый человек!»

Не все мысленные описания человека обязательно соединяются с неким центральным символом, хранящим его имя. Описания могут рождаться и использоваться независимо. Мы можем изобретать несуществующих людей, придумывая их описания, совместить два описания, обнаружив, что они относятся к одному и тому же человеку, разделить одно описание на два, если обнаружим, что оно относится не к одному, а к двум предметам, и так далее. Это «исчисление описаний» находится в самом сердце мышления. Считается, что оно \emph{интенсионально} , а не \emph{экстенсионально} : это означает, что описания могут свободно «плавать на поверхности», а не стоять на якоре, привязанные к определенным, известным предметам. Интенсиональность мышления связана с его гибкостью, она дает нам возможность изобретать воображаемые миры, соединять разные описания в одно, разделять одно описание на два, и так далее.

Представьте себе, что подруга, взявшая у вас на время машину, звонит и говорит, что произошла авария машину занесло на мокрой дороге и она перевернулась, упав в кювет «Я чудом избежала смерти,» --- говорит она. В голове у вас появляются, одна за другой, соответствующие образы, которые становятся все реальнее по мере того как собеседница добавляет все новые детали; в конце рассказа вся картина стоит у вас перед глазами. Вдруг она, смеясь, сообщает вам что все это --- первоапрельская шутка, и что ни с ней, ни с машиной ничего не случилось! В некотором смысле, это ничего не меняет. История и образы вызванные ею не теряют своей жизненности и надолго остаются у вас в памяти. В дальнейшей вы можете считать вашу подругу плохим водителем, поскольку впечатление оставленное ее рассказом, не пропало, когда вы узнали, что это --- неправда. Выдумка и факт тесно переплетаются в нашем сознании, и это происходит потому, что мышление предполагает способность к изобретению сложных описаний и манипуляции ими, эти описания совсем не обязательно должны быть привязаны к реальным фактам или вещам.

В основе мышления --- гибкое, интенсиональное представление о мире. Как же физиологическая система, такая как мозг, позволяет производить подобное представление?

«Муравьи» мозга

Самые важные клетки мозга --- это нервные клетки или \emph{нейроны} ; их в мозгу около десяти миллиардов. (Интересно, что количество глиальных клеток, или глий, превосходит это число почти в десять раз. Считается, что глии играют второстепенную роль по сравнению с нейронами, поэтому мы не будем на них останавливаться.) У каждого нейрона есть несколько \emph{синапсов} (на компьютерном жаргоне, «портов ввода»), расположенных на \emph{дендритах} (и иногда --- на теле клетки), и один \emph{аксон} («канал вывода»). Ввод и вывод представляют собой электрохимические потоки, то есть движущиеся ионы. Между портом ввода и выводным каналом находится \emph{тело} клетки, где принимаются «решения».

Эти решения, которые нейрону приходится принимать иногда до тысячи раз в секунду, следующего типа: нужно ли ему \emph{возбудиться} --- то есть, послать по аксону ионы. Эти ионы рано или поздно достигнут входных портов других нейронов, которым придется тогда принимать такое же решение. Решение принимается очень просто: если сумма всех входных импульсов превышает некий порог, то нейрон возбуждается; в противном случае этого не происходит.

\emph{Рис. 65. Схема нейрона. (Взято из книги Д. Вулдриджа «Механика мозга» (D.Woold-ridge, «The Machinery of the Brain», стр. 6).)}

Некоторые входные импульсы могут быть негативными; они аннулируют позитивные импульсы, полученные из другого места. Так или иначе, на низшем уровне нашего разума царит простое сложение. Перефразируя знаменитое изречение Декарта, «я мыслю, значит я суммирую» (от латинского Cogito, ergo summo).

Хотя манера принятия решений кажется простой, ситуацию осложняет то, что у нейрона может быть до 200 000 отдельных входов; это означает, что для принятия решения нейрон должен манипулировать иногда 200 000 слагаемых. Как только решение принято, поток ионов устремляется по аксону к выходу. Однако они могут встретить на пути развилку или даже несколько. Тогда единый импульс разделяется и идет по нескольким ветвям аксона. Выхода достигают уже несколько импульсов, которые при этом могут прибыть к месту своего назначения в разное время, так как ветви аксона, по которым они двигаются, могут быть разной длины и иметь разное сопротивление. Важно, однако, то, что все они начались как единый импульс, испущенный телом клетки. После того, как нейрон возбудится, ему необходимо некоторое время, чтобы «восстановить силы»--- обычно это время измеряется миллисекундами, так что нейрон может возбуждаться до тысячи раз в секунду.

Более крупные структуры мозга

Мы только что описали «муравьев» мозга. А как насчет «команд» или «сигналов»? А насчет «символов»' Мы заметили, что, несмотря на сложность входных импульсов, каждый нейрон может ответить одним из двух способов --- либо возбуждаясь, либо нет. Это дает весьма небольшое количество информации. Безусловно, чтобы передавать и обрабатывать большой объем информации, необходимо участие множества нейронов. Можно предположить, что существуют более крупные структуры, состоящие из многих нейронов, которые работают с понятиями высшего уровня. Это, несомненно, верно; однако наивное предположение о том, что каждой идее соответствует определенная группа нейронов, скорее всего, неправильно.

Мозг состоит из различных анатомических частей, таких как головной мозг, мозжечок и гипоталамус (см. рис. 66). Головной мозг --- это самая большая часть человеческого мозга; он разделен на правое и левое полушария. Снаружи каждое из них покрыто слоями «коры», достигающей толщины в несколько миллиметров; эта оболочка так и называется \emph{корой головного мозга} . С анатомической точки зрения, именно размеры коры головного мозга --- наиболее бросающееся в глаза отличие между мозгом человека и мозгом менее разумных биологических видов. Мы не будем здесь подробно описывать различные части мозга, поскольку оказывается, что соотношение, которое можно установить между этими крупномасштабными органами и деятельностью, за которую они отвечают, весьма приблизительно.

Известно, например, что языковыми способностями в основном управляет одно из полушарий --- обычно левое. Мозжечок --- это область, управляющая двигательной активностью. Но каким образом эти области выполняют свои функции, остается загадкой.

\emph{Рис. 66 Человеческий мозг, вид слева. Странно, что зрительная область находится ближе к затылку. (Взято из книги Стивена Роуза «Мыслящий мозг» (Steven Rose, «The Conscious Brain»), стр. 50.)}

Соответствие между мозгами

Возникает очень важный вопрос: если мысли рождаются в мозгу, то чем отличается один мозг от другого? Чем отличается мой мозг от вашего? Безусловно,~~вы не думаете точно так же, как я --- да и нет двух людей, которые бы думали одинаково. Но при этом все мы имеем одинаковое анатомическое строение мозга. Насколько идентичны наши мозги? Распространяется ли это сходство на уровень нейронов? Для животных, стоящих на низшем уровне «иерархии мышления», таких, например, как земляной червь, ответ будет положительным. Процитирую по этому поводу выступление нейрофизиолога Дэвида Хубеля на конференции по общению с внеземными культурами:

Количество нервных клеток у червя измеряется, я думаю, тысячами. Интересно то, что мы можем указать на какой-либо нейрон определенного червя и затем найти точно соответствующий ему нейрон у другого червя того же вида.\footnote{Carl Sagan, ed., «Communication with Extraterrestrial Intelligence», стр. 78.}

Оказывается, мозги земляных червей изоморфны! Можно сказать, что существует всего один земляной червь.

Однако такое взаимооднозначное соответствие исчезает, как только мы обращаемся к высшим уровням иерархии мышления и количество нейронов возрастает, это подтверждает наше подозрение о том, что на свете --- не только один человек! И все же между человеческими мозгами существует большое сходство, если сравнивать их на уровне, промежуточном между нейронами и более крупными составляющими мозга. Какой из этого можно сделать вывод относительно того, как индивидуальные различия представлены в физиологии мозга? Можно ли, рассматривая связи между нейронами моего мозга, найти такие структуры, в которых закодированы мои знания, убеждения, надежды, страхи, симпатии и антипатии? Если мы считаем, что мысленный опыт расположен в мозгу, можно ли наши те места или те физические подсистемы мозга, где расположены знания и другие аспекты интеллектуальной жизни? Это будет основным вопросом этой и следующей глав.

Загадка местоположения мозговых процессов

В попытке найти ответ на этот вопрос, невролог Карл Лашли провел длинную серию экспериментов. В этих экспериментах, начавшихся около 1920 года и продолжавшихся много лет, он попытался обнаружить, где в мозгу у крысы хранится ее опыт по прохождению лабиринтов. В своей книге «Мыслящий мозг» Стивен Роуз описывает злоключения Лашли:

Лашли хотел определить где в коре головного мозга расположена память. Для этого он сначала тренировал крыс находить дорогу в лабиринте, а затем удалял~у них различные районы коры. После того как крысы выздоравливали он снова пускал их по лабиринту. К его удивлению, ему не удалось найти определенное место в мозгу ответственное за умение крыс находить дорогу к выходу. Вместо этого все крысы, у которых была удалена какая-либо часть коры, начинали страдать от тех или иных физических недостатков, серьезность которых была прямо пропорциональна количеству удаленной коры. Удаление коры повредило моторные и сенсорные способности животных, крысы начали хромать, подскакивать шататься или кататься по полу, но все они каким то образом, находили дорогу в лабиринте. Казалось что память расположена равномерно по всей коре. В своей последней статье «In Search of the Engram», опубликованной в 1950 году, Лашли мрачно заключил, что память вообще невозможна.\footnote{Steven Rose, «The Conscious Brain», стр. 251-2.}

Интересно, что в конце 1940-х годов, примерно в то же время, когда Лашли проводил свои эксперименты, в Канаде было найдено подтверждение противоположной точки зрения. Нейрохирург Вильдер Пенфильд изучал реакции пациентов во время операции над мозгом, вводя в различные области открытого мозга электроды и посылая слабые электрические импульсы, стимулирующие нейрон или нейроны, которых касался данный электрод. Эти импульсы были подобны импульсам, исходящим от других нейронов. Пенфильд обнаружил, что стимуляция определенных нейронов регулярно вызывает у пациентов специфические образы или чувства. Искусственно вызванные таким образом впечатления были самые разнообразные иногда пациенты испытывали странный, необъяснимый страх, иногда они видели цвета и слышали звуки --- но самыми впечатляющими были случаи когда пациенты вспоминали целую цепь событий из далекого прошлого, как, например, детский праздник дня рождения. Набор точек способных вызвать подобную реакцию, был весьма мал: практически речь шла об одном-единственном нейроне. Очевидно что результаты, полученные Пенфильдом разительно отличаются от заключения Лашли, поскольку из них вытекает, что специфические воспоминания хранятся в строго определенных зонах мозга.

Какой вывод можно из этого сделать? Возможным объяснением было бы то, что одно и то же воспоминание закодировано одновременно в нескольких местах, расположенных по всей коре~--- стратегия, которая могла развиться в процессе эволюции, как защита от возможной потери части коры в бою --- или во время экспериментов, проводимых нейрофизиологами. Другое возможное объяснение --- то, что воспоминания могут восстанавливаться на основе динамических процессов, распространенных по всему мозгу, но при этом могут вызываться возбуждением местных точек. Эта теория основана на современных телефонных сетях, где распределение междугородных звонков не известно заранее, а выбирается в момент данного звонка в зависимости от загруженности телефонных сетей по всей стране. Поломка части сетей не остановит звонки --- они будут просто направлены в обход испорченного места. В этом смысле любой звонок потенциально невозможно локализовать. И в то же время любой звонок соединяет всего две точки; в этом смысле локализовать его вполне возможно.

Определенность в обработке зрительных образов

Одно из самых интересных исследований по локализации мозговых процессов проводилось в последние пятнадцать лет Дэвидом Хюбелем и Торстеном Визелем из Харвардского университета. Они проследили путь зрительных впечатлений в мозгу у кошки: сначала возбуждаются нейроны на сетчатке, возбуждение распространяется по направлению к затылку, проходит через боковое коленчатое тело, работающее в качестве «ретрансляционной станции», и прибывает к зрительной коре в задней половине мозга. Прежде всего, в свете результатов Лашли кажется удивительным, что существуют определенные мозговые пути; но еще более замечательными оказались свойства нейронов, расположенных на различных участках этого пути.

Оказывается, что нейроны сетчатки прежде всего воспринимают контраст. Это происходит следующим образом, обычно каждый из этих нейронов возбуждается с постоянной скоростью. Когда на него падает свет, нейрон может начать возбуждаться быстрее, замедлиться, или совсем перестать возбуждаться. Однако это происходит только в том случае, когда соседние участки сетчатки менее освещены. Это означает, что существуют два типа нейронов: «центральные» и «периферийные». Первые посылают сигналы с большей скоростью, когда центр небольшой круглой зоны сетчатки, к которой они принадлежат, освещен, а периферия находится в темноте. Вторые, напротив, увеличивают скорость посылки импульсов тогда, когда центр круга находится в темноте, а внешнее кольцо освещено. «Увидев» светлый центр, периферийные нейроны \emph{замедляются} , и наоборот. Равномерное освещение не затрагивает ни тот, ни другой тип --- нейроны обоих типов продолжают посылать сигналы с обычной скоростью.

С сетчатки сигналы, посланные этими нейронами, направляются по оптическому нерву к боковому коленчатому телу, расположенному близко к центру мозга. Там мы находим прямое соответствие поверхности сетчатки, в том смысле, что нейроны коленчатого тела отвечают только на некоторые стимулы, падающие на определенные места сетчатки. В этом смысле коленчатое тело не представляет особого интереса --- это всего-навсего «ретрансляционная станция», и сигналы там не подвергаются дальнейшей обработке (хотя надо все же отдать ему должное --- коленчатое тело, по-видимому, усиливает чувствительность к световым контрастам). Образ на сетчатке закодирован в схеме сигналов, посылаемых нейронами бокового коленчатого тела, несмотря на то, что нейроны там расположены не на плоскости сетчатки, а в трехмерном блоке. Таким образом, хотя два измерения здесь соответствуют трем, информация тем не менее сохраняется: еще один пример изоморфизма. Возможно, у этого изменения количества измерений есть некий глубинный смысл, которого мы еще не понимаем полностью. Так или иначе, в нашем знании о зрении пока еще так много пробелов, что мы должны не расстраиваться, а радоваться, что нам удалось, хотя бы до определенного предела, понять данный этап.

Из бокового коленчатого тела сигналы поступают обратно в зрительную кору. Здесь они обрабатываются по-новому. Клетки зрительной коры подразделяются на три категории: простые, сложные, и сверхсложные. \emph{Простые} клетки весьма похожи на клетки сетчатки или бокового коленчатого тела они реагируют на освещенные и неосвещенные точки, когда те находятся в контрасте с окружением в определенных местах сетчатки. \emph{Сложные} клетки, с другой стороны, получают информацию от более чем сотни других клеток, и «видят» светлые и темные полосы , расположенные на сетчатке под определенными углами (см. рис. 67). \emph{Сверхсложные} клетки замечают углы, полосы и даже «языки», двигающиеся в определенных направлениях (еще раз см. рис. 67). Эти клетки настолько специализированы, что их иногда называют «сверхсложными клетками высшего порядка».

\emph{Рис. 67. Ответ на некоторые схемы стимулов различных типов нейронов. (а) Этот нейрон, видящий контуры, ищет вертикальные грани, освещенные слева и находящиеся в тени с правой стороны. Первая колонка показывает, как нейрон реагирует на угол наклона граней. Вторая колонка показывает, что позиция грани внутри «поля зрения» данного нейрона для него неважна (б) Сверхсложная клетка отвечает на стимулы более выборочно, в данном случае --- только когда спускающийся «язык» находится в центре поля зрения (в) Реакция гипотетической «клетки-бабушки» на различные типы стимулов, читатель может позабавиться, представив себе, как на те же стимулы отвечала бы «клетка-осьминог».}

«Клетка-бабушка»?

В связи с открытием в зрительной коре клеток, отвечающих на стимулы возрастающей сложности, некоторые исследователи стали задаваться вопросом, не соответствуют ли понятия отдельным клеткам --- скажем, у вас была бы «клетка-бабушка», возбуждающаяся только при виде вашей бабушки. Этот забавный пример «ультрасверхсложной клетки», разумеется, никем всерьез не принимается. Однако неясно, какая альтернативная теория более разумна. Одна возможность заключается в том, что на достаточно сложные зрительные стимулы отвечают некие более обширные комплексы нейронов. Разумеется, такие группы нейронов будут возбуждаться от комплекса сигналов, исходящих от многих сверхсложных клеток. Как конкретно это может происходить, пока является загадкой. Когда нам кажется, что мы приближаемся к порогу, за которым из «сигналов» рождается «символ», мы теряем след в этой дразняще неоконченной истории. Мы, впрочем, скоро к ней вернемся и постараемся дать еще кое-какие детали.

Ранее я упомянул о приблизительном, на анатомическом уровне, изоморфизме между человеческими мозгами и точном, на нейронном уровне, изоморфизме между мозгами земляных червей. Интересно, что можно проследить изоморфизм «среднего масштаба» между обрабатывающими зрительную информацию аппаратами кота, обезьяны и человека. Этот изоморфизм работает следующим образом. Во-первых, у всех трех видов обработка зрительной информации происходит в затылочной части коры головного мозга --- в области, так и называющейся зрительной корой. Во-вторых, у каждого вида зрительная кора разделена на три подучастка, называющиеся участками 17,18 и 19. Эти участки могут быть найдены в мозгу у любого нормального индивида каждого из этих трех видов. Внутри каждого участка можно пойти еще дальше и достигнуть «колонной организации» зрительной коры. Зрительные нейроны, расположенные перпендикулярно поверхности коры и направляющиеся по радиусу внутрь, к центру мозга, организованы в «колонки»; все сигналы поступают по радиусу --- в направлении колонок, а не между ними. Каждой колонке соответствует определенный небольшой участок сетчатки. Число колонок варьируется у разных индивидов, так что найти «ту же самую» колонку не удается. Наконец, внутри колонок есть слои, где обычно расположены простые нейроны, и слои, где расположены сложные нейроны. (Сверхсложные нейроны обычно находятся на участках 18 и 19, в то время как простые и сложные --- на участке 17.) По-видимому, на этом уровне изоморфизм кончается. На уровне отдельных нейронов, каждый кот, каждая обезьяна и каждый человек имеют уникальную структуру --- такую же уникальную, как отпечаток пальца или подпись.

Одно небольшое, но значительное различие между обработкой зрительной информации мозгом кота и мозгом обезьяны присутствует на этапе, на котором информация, полученная от обоих глаз, соединяется и образует единый сигнал высшего уровня. Оказывается, что у обезьян это происходит немного позднее, чем у котов; это дает сигналам каждого глаза больше времени для независимой обработки. Это неудивительно, поскольку мы предполагаем, что чем выше стоит данный тип в иерархии интеллекта, тем сложнее будут проблемы, решаемые его зрительным аппаратом; поэтому сигналы должны проходить более долгую обработку, прежде чем получить окончательный «ярлык». Это предположение было весьма убедительно подтверждено наблюдениями за зрительными способностями новорожденного теленка, рожденного, по-видимому, с полностью развитым зрительным аппаратом. Теленок пугается людей и собак, но прекрасно чувствует себя в окружении других телят. Возможно, его зрительная система целиком закодирована в мозгу еще до рождения и требует сравнительно небольшой работы коры. С другой стороны, человеческой зрительной системе, так сильно зависящей от коры, требуется несколько лет, чтобы развиться полностью.

Невральная воронка

Открытия, сделанные до сих пор в области организации мозга, интересны тем, что пока не удалось найти соответствия между крупномасштабной «аппаратурой» и «программным обеспечением высшего уровня» Например, зрительная кора --- это крупномасштабная часть аппаратуры, полностью посвященная обработке зрительной информации; однако все известные нам процессы, происходящие там, все еще протекают на низших уровнях. Ничего похожего на узнавание предметов пока в зрительной коре не обнаружено. Это значит, что никто пока не знает, где и каким образом информация, исходящая от сложных и сверхсложных клеток, превращается в узнанные формы, комнаты, картины, лица и так далее. Исследователи пытаются описать способ, при помощи которого множество реакций на низшем, нейронном уровне, словно проходя через воронку, сводится к меньшему числу реакций на высших уровнях, что, в конце концов, приводит к знаменитой «клетке-бабушке» или некоторой сложной нейронной сети, как та, о которой мы упомянули выше. Очевидно этот способ не может быть обнаружен на уровне анатомических частей мозга, скорее, его надо искать на более микроскопическом уровне.

Возможной альтернативой клетки-бабушки может быть постоянный набор нейронов --- скажем, несколько дюжин --- на узком конце «воронки», любой из них реагирует на появление бабушки в поле зрения. Подобно этому, для каждого отдельного предмета существовала бы специфическая сеть нейронов и некая «воронка», сводящая сложные впечатления к этой сети. Существуют более сложные альтернативы, основанные на той же идее, они включают сеть нейронов, которая может отвечать на стимул по-разному, вместо одного строго определенного способа. Такие сети соответствовали бы «символам» в нашем мозгу.

Необходим ли подобный процесс сужения? Возможно, что наш мозг узнает предметы по их «подписи» на зрительной коре --- то есть, по коллективным ответам простых, сложных и сверхсложных клеток. Может быть, мозгу не требуется никакое дальнейшее «сужение» впечатлений, чтобы узнать данный предмет. Однако эта теория представляет следующее затруднение Представьте себе, что вы смотрите на некую сцену. В вашем мозгу появляется «подпись»-отпечаток этой сцены; однако как вы перейдете от этого отпечатка к словесному описанию данной сцены? Например, когда вы смотрите на картины Эдуарда Вийара, французского постимпрессиониста, зачастую требуется несколько секунд, прежде чем вы различите человеческую фигуру. Предположительно, отпечаток увиденного появляется на зрительной коре в первую долю секунды --- при этом вы понимаете картину только через несколько секунд. Это только один пример весьма обычного явления --- чувства, что в момент узнавания у вас в мозгу что-то~«кристаллизуется»; это происходит не тогда, когда свет попадает на сетчатку, но позднее, после того как какая-то часть вашего интеллекта обработала сигналы на сетчатке.

Сравнение с кристаллизацией приводит на ум еще один замечательный образ, взятый из статистической механики: мириады микроскопических, не связанных между собой событий в некоей среде, которые формируют медленно растущие согласованные области. В результате эти мириады крохотных событий полностью изменяют среду: из хаотического множества независимых элементов она превращается в большую, стройную и связную структуру. Если считать, что первичные реакции нейронов представляют собой независимые события, в результате множества отдельных сигналов производящие определенный крупный «модуль» нейронов, то слово «кристаллизация» отлично сюда подходит.

Еще один аргумент в пользу «воронки» основан на факте, что существует множество различных сцен, которые обычно воспринимаются как один и тот же объект: бабушка может улыбаться или хмуриться, быть в шляпе или без, стоять в освещенном солнцем саду или в темной комнате, стоять далеко или близко, в профиль или в анфас и так далее. Все эти сцены производят весьма различные «подписи» на зрительной коре --- но все они заставляют вас сказать: «Здравствуй, бабуля.» Следовательно, некий сужающий процесс все же происходит в какой-то момент после образования зрительного отпечатка и перед тем, как вы произносите первое слово. Можно возразить, что этот процесс относится не к узнаванию бабушки, но к превращению этого впечатления в слова. Однако это разделение кажется искусственным, поскольку можно узнать бабушку и без необходимости выражать это знание словами. Было бы весьма неудобно обрабатывать всю информацию, полученную зрительной корой, так как большая часть этой информации может быть отброшена за ненадобностью, нам неинтересно знать, как падают на бабушкино лицо тени и сколько пуговиц у нее на блузке.

Другая проблема с теорией, отрицающей необходимость воронки --- это объяснение многих возможных интерпретаций для одной и той же «подписи» --- как, например, происходит с картиной Эшера «Выпуклое и вогнутое» (Рис. 23). Нам кажется очевидным то, что мы воспринимаем образ на экране телевизора не как набор \emph{точек} , но как определенные \emph{блоки} ; было бы странным, если бы восприятие происходило в момент появления гигантской точечной «подписи» в зрительной коре. Более вероятным кажется некое сужение, в результате которого возбуждаются специфические модули нейронов, каждый из которых ассоциируется с определенным понятием --- блоком --- в данной сцене.~~

Модули, участвующие в мышлении

Таким образом, мы приходим к заключению, что каждому понятию соответствует определенный модуль, некий «нейронный комплекс», состоящий из небольшой группы клеток. Однако, если принять эту теорию полностью, то возникает следующая проблема: эта теория предполагает, что расположение в мозгу подобных модулей может быть точно указано. Пока этого сделать не удалось, и некоторые данные, такие, например, как эксперименты Лашли, говорят против локализации. И все же, принимать окончательное решение еще рано. Могут существовать несколько копий одного модуля, расположенные в разных местах; кроме того, модули могут накладываться друг на друга. Обе эти возможности затруднили бы четкое разделение нейронов на группы. Может быть, нейронные комплексы подобны очень тонким блинчикам, уложенным слоями, которые иногда проходят друг сквозь друга --- а может быть, они похожи на длинных змей, закрученных друг вокруг друга, и иногда сплющенных, наподобие головы кобры. Они могут быть похожи на паутину --- или на электрические цепи, в которых сигналы перемещаются по траекториям, более причудливым, чем погоня голодной ласточки за комаром. Пока мы этого не знаем. Возможно даже то, что эти модули являются скорее программой, чем частью аппаратуры --- мы обсудим эту возможность в дальнейшем.

В связи с этими гипотетическими нейронными комплексами возникает множество вопросов. Например:

Распространяются ли они на низшие районы мозга, такие, как средний мозг, гипоталамус, и т. д.?

Может ли один и тот же нейрон принадлежать более, чем к одному комплексу?

К скольким комплексам может одновременно принадлежать один и тот же нейрон?

Сколько нейронов могут одновременно принадлежать к разным комплексам?

Совпадают ли комплексы в мозгах разных людей?

Находятся ли соответствующие комплексы в одинаковом месте в мозгах разных людей?

Перекрещиваются ли они одинаковым образом у разных людей?

С философской точки зрения самым важным вопросом является следующий: что означало бы наличие подобных модулей, например, клетки-бабушки? Обеспечило ли бы это более глубокое понимание нашего сознания? Или это знание приблизило бы нас к пониманию сознания не более, чем тот факт, что наш мозг состоит из нейронов и глий? Читая «Муравьиную фугу», вы, возможно, догадались, что мне кажется, что такое знание далеко продвинуло бы нас в понимании феномена сознания. Важнейшим шагом здесь является переход от описания состояния мозга на низшем уровне, нейрон за нейроном, к описанию этого состояния на высшем уровне, модуль за модулем. Или, возвращаясь к выразительной терминологии «Муравьиной фуги», мы хотим перенести описание состояния мозга с уровня \emph{сигналов} на уровень \emph{символов} .

Активные символы

В дальнейшем давайте называть эти гипотетические нейронные комплексы, нейронные модули, нейронные группы, нейронные сети и мультинейронные единицы \emph{символами} , как бы они не выглядели --- как блинчики, грабли, гремучие змеи, снежинки или даже как волны на воде. Описание состояния мозга в терминах символов уже было упомянуто в Диалоге. На что походило бы подобное описание? Какие типы понятий можно представить себе в виде символов? Каким образом подобные символы были бы связаны между собой? И как вся эта картина поможет нам понять, что такое сознание?

Во-первых, важно подчеркнуть, что символы бывают либо дремлющие, либо активированные. Активированный символ --- это тот, нейроны которого возбудились, когда внешние стимулы превысили определенный порог. Поскольку символ может быть возбужден различными способами, став активным, он может действовать по-разному. Это значит, что мы должны думать о символе не как о застывшей, но как о меняющейся, динамической единице. Таким образом, описывая состояние мозга, недостаточно сказать «символы А, В,\ldots{} N в данный момент активны»; вместо этого необходимо представить набор параметров для каждого активного символа, характеризуя некоторые аспекты того, как этот символ работает «изнутри». Интересен следующий вопрос: есть ли у каждого символа некие центральные нейроны, посылающие сигналы каждый раз, когда символ активизируется? Если такая «сердцевина» существует, мы могли бы называть ее «неизменной сердцевиной» символа. Соблазнительно предположить, что каждый раз, когда вы думаете, скажем, о водопаде, повторяется некий один и тот же нейронный процесс, хотя и видоизмененный немного в зависимости от контекста. Однако неясно, так ли это происходит на самом деле.

Что происходит, когда символ «просыпается»? Описывая этот процесс на низшем уровне, мы сказали бы, что возбуждаются многие из его нейронов. Однако подобные описания нас больше не интересуют. Описание на высшем уровне должно полностью игнорировать нейроны и сосредотачиваться исключительно на символах. Таким образом, на высшем уровне описание активного символа, в отличие от его дремлющего собрата, было бы следующим: «Он посылает сигналы, призванные разбудить, или активизировать, другие символы.» Разумеется, эти послания будут передаваться в виде потока нервных импульсов именно нейронами, но мы постараемся не использовать эту терминологию, чтобы избежать описания на низшем уровне. Иными словами, мы надеемся описать мыслительные процессы как отделенные непроницаемой переборкой от нейронных событий, так же, как поведение часов отделено от законов квантовой механики, или биология клетки --- от законов кварков.

Какие же преимущества предоставляет нам подобное описание на высшем уровне? Почему лучше сказать «Символы А и В активировали символ С», чем «Нейроны с 183 по 612 возбудили нейрон 75, и тот послал сигнал»? На этот вопрос мы уже ответили в «Муравьиной фуге»: это лучше, потому что символы \emph{символизируют} вещи, а нейроны --- нет. Символы --- отражения понятий в аппаратуре мозга. В то время, как возбуждение группой нейронов какого-то другого нейрона не соответствует никакому внешнему событию, активацию символов другими символами можно соотнести с событиями в реальном (или придуманном) мире. Символы соотносятся друг с другом при помощи посланий таким образом, что схема их возбуждения весьма напоминает действительные события, происходящие в масштабе реального мира, или могущие произойти в мире воображаемом. По сути, значение возникает здесь таким же образом, как оно возникало в системе~\textbf{pr} --- при помощи изоморфизма; только тут этот изоморфизм несравненно сложнее, тоньше, деликатнее и интенсиональнее.

Кстати, требования, чтобы символы были способны передавать сложные сообщения, вероятно, достаточно, чтобы исключить возможность того, что эту роль играют нейроны. Поскольку нейрон может посылать информацию только единственным путем и не может по желанию менять направление сигналов, у него просто не хватает мощи и гибкости, необходимых символу, чтобы действовать подобно объекту реального мира. В своей книге «Общества насекомых» Е. О. Вильсон высказывает похожую мысль о том, как сообщения распространяются внутри муравьиных колоний:

(Массовая коммуникация определяется как передача от группы к группе той информации, которую один индивид не способен передать другому.)\footnote{E. O. Wilson, «The Insect Societies», стр. 226.}

Кажется, что сравнение мозга с муравьиной колонией не так уж плохо! Следующий, очень важный вопрос касается природы и «размера» понятий, представленных в мозгу символами. О природе символов возникают такие вопросы: существует ли один символ для общего понятия водопада, или же разные символы для разных типов водопадов? Или верно и то и другое? О «размере» символов возникают такие вопросы: существует ли символ для целого рассказа? Или мелодии? Или шутки? Или же символы существуют на уровне слов, а фразы и предложения выражаются при помощи последовательной активации различных символов?

Рассмотрим проблему размера символов подробнее. Большинство мыслей, выраженных в предложениях, состоят из основных, сравнимых с атомами компонентов, которые мы дальше не анализируем. Обычно эти компоненты бывают размером со слово, иногда немного длиннее, иногда немного короче. Существительное «водопад», название «Ниагарский водопад», глагольный суффикс «-л», отмечающий прошедшее время, выражение «добро пожаловать», как и более длинные идиоматические выражения, являются примерами языковых атомов. Все это типичные мазки, которыми мы рисуем портреты более сложных понятий, таких, как сюжет фильма, обаяние города, природа сознания, и так далее. Подобные сложные идеи нельзя сравнить с отдельными мазками кисти. Кажется разумным предположить, что мазки языка --- это также и мазки мысли, и что символы представляют понятия приблизительно такого же размера. Таким образом, символ --- это нечто такое, для чего вы знаете слово, или готовую фразу, или с которым вы ассоциируете какое-либо имя собственное. Представлением же в мозгу более сложной идеи, такой, как неприятность в личной жизни, будет активация нескольких символов другими символами.

Классы и примеры

Описывая мышление, мы отмечаем общее различие между \emph{категориями} и \emph{индивидуумами} , или \emph{классами} и \emph{примерами} . (Иногда также используются термины «типы» и «образцы».) С первого взгляда может показаться, что каждый символ должен обязательно представлять либо класс, либо пример --- но это слишком большое упрощение. На самом деле, большинство символов могут представлять и то, и другое, в зависимости от контекста, в котором происходит их активация. Взгляните, например, на список приведенный ниже:

(1) печатное издание

(2) газета

(3) «Вечерняя Москва»

(4) Экземпляр «Вечорки» от 3 мая

(5) Мой экземпляр «Вечорки» от 3 мая

(6) Мой экземпляр «Вечорки» от 3 мая, в тот момент, когда я его купил (по сравнению с тем, каким он стал, полежав под плошкой с кошачьей едой).

Здесь строчки со второй по пятую играют обе роли. Строчка 4 --- пример общего класс строчки 3, и строчка 5 --- пример строчки 4. Строчка 6 --- это особый случай примера данного класса: \emph{проявление} . Разные состояния одного и того же предмета в разные моменты его жизни --- это его проявления. Интересно было бы узнать, понимают ли коровы, что веселый фермер, скармливающий им каждое утро щедрую порцию сена --- это один и тот же индивидуум, несмотря на его разные проявления?

Приведенный список кажется списком иерархии общности --- на вершине находится весьма широкая концептуальная категория, а внизу --- скромная конкретная вещь, локализованная во времени и пространстве. Однако идея о том, что «класс» должен обязательно быть очень широким и абстрактным, слишком ограничена. Дело в том, что наше мышление пользуется хитроумным принципом, который можно назвать \emph{принципом прототипа} :

\emph{Любой частный случай может служить примером некого класса случаев} .

Каждый знает, что отдельные события бывают настолько впечатляющими, что они надолго остаются в памяти и могут впоследствии служить моделями событий, в какой-то мере на них похожих. Таким образом, в каждом специфическом случае заложено семя целого класса подобных случаев. Идея о том, что в частном заложено общее, очень важна.

Возникает естественный вопрос: что представляют собой символы в мозгу, классы или примеры? Есть ли символы, представляющие исключительно классы, в то время как другие символы представляют только примеры? Или один и тот же символ может быть то символом класса, то символом примера, в зависимости от того, какие его части были активизированы? Последняя теория кажется привлекательной; можно предположить, что «слегка» активизированный символ может представлять класс, в то время как более глубокое и сложное возбуждение вызовет большее количество внутренних нейронных сигналов, и, следовательно, символ будет представлять частный пример. Однако если подумать хорошенько, это довольно странная идея: это означало бы, что активизировав символ «печатное издание» достаточно глубоко, можно прийти к сложному символу, соответствующему газете, лежащей под плошкой с минтаем для кошки Муськи. И что любое возможное проявление любого печатного издания может быть представлено в мозгу путем активации одного и того же символа «печатного издания». Это, пожалуй, слишком тяжелая ноша для одного единственного символа. Следовательно, можно заключить, что бок о бок с символами классов должны существовать и символы примеров, и что последние --- не просто разные способы активации символов класса, но самостоятельные единицы.

Отделение примеров от классов

С другой стороны, символы-примеры часто наследуют многие черты от классов, к которым эти примеры принадлежат. К примеру, если я скажу вам, что видел какой-нибудь фильм, вы начнете производить свежий символ для данного фильма; однако за отсутствием достаточной информации, вам придется опираться на уже существующий у вас символ-класс «фильм». Подсознательно вы будете предполагать, что фильм продолжался от одного до трех часов, что он показывался в местном кинотеатре, что в нем рассказывалась история о каких-то людях, и так далее. Эти предположения «встроены» в символ класса и служат для связи его с другими символами (возможности возбуждать другие символы); это, говоря на компьютерном жаргоне, \emph{параметры, выбираемые по умолчанию или стандартный выбор} . В любом новоиспеченном символе-примере эти параметры легко можно обойти; однако если это не оговорено специально, они будут унаследованы новым символом от символа-класса. Если эти параметры специально не исключены, они дадут вам предварительную информацию, чтобы представлять новый пример --- например, фильм, который я посмотрел --- с помощью вероятных допущений, основанных на неком «стереотипе», или символе-классе.

Новый символ-пример похож на ребенка, у которого еще нет собственных идей и опыта --- он старается имитировать родителей, надеясь во всем на их опыт и мнения. Но постепенно он приобретает собственный опыт и неизбежно начинает отдаляться от родителей. Спустя некоторое время, ребенок превращается во взрослого. Подобно этому, новорожденный символ-пример может постепенно отойти от класса-родителя и превратиться в самостоятельный класс, или прототип.

Чтобы лучше понять, как это происходит, представьте себе, что однажды вечером вы включаете радио и слышите трансляцию футбольного матча между незнакомыми вам командами. Сначала вы не знаете имен игроков ни в одной команде. Когда комментатор говорит: «Голяшкина сбивают с ног в тот момент, когда он собирался бить по воротам», вы понимаете только то, что одного из игроков незаконно атаковали. В вашей голове при этом активизируется символ класса «футболист», одновременно с символом «грубое действие». Затем, слыша фамилию Голяшкина еще несколько раз, вы начинаете создавать для него специальный символ, возможно, используя его имя как опорный пункт. Этот символ зависит, как ребенок, от символа-класса «футболист»; рождающийся у вас образ Голяшкина основан на вашем стереотипе футболиста, заложенном в соответственном символе. Но постепенно, пока вы слушаете репортаж, у вас накапливается новая информация о Голяшкине, и его символ становится все более независимым, меньше и меньше нуждаясь в возбуждении символа класса-родителя. Это может произойти за несколько минут, если Голяшкин отличится, сделав несколько удачных пасов и забив гол. Его товарищи по команде, однако, могут быть все еще представлены активацией класса-символа. Через несколько дней, после того, как вы прочитали несколько отчетов о матче, пуповина рвется, и Голяшкин может стоять на своих двоих. Теперь вы знаете, из какого он города и в какой команде играл раньше, узнаете его лицо, и так далее. В этот момент, Голяшкин для вас уже не абстрактный игрок, но человек, профессия которого --- футболист. Символ «Голяшкин» может быть активным, в то время как его родитель, символ-класс «футболист», может оставаться в пассивном состоянии. Когда-то символ «Голяшкин» был спутником, вращавшимся вокруг материнского символа, подобно тому, как искусственные спутники кружатся по орбите вокруг Земли, которая намного больше и массивнее их. Затем наступила промежуточная стадия, когда один символ был важнее другого, но при этом их можно было рассматривать как вращающиеся друг вокруг друга --- что-то вроде Земли и Луны. Наконец, новый символ становится автономным и может в свою очередь служить символом-классом, вокруг которого могут начать кружиться новые спутники --- символы, возникающие у других людей, не знакомых с Голяшкиным. Он может служить временным стереотипом --- пока у них не появится больше информации, и их символы-спутники также не станут независимыми.

Трудность отделения символов друг от друга

Эти этапы роста и отделения примера от класса можно различить по тому, как связаны между собой задействованные символы. Без сомнения, иногда будет очень трудно с уверенностью сказать, где начинается один символ и где кончается другой. Насколько «активен» какой-либо символ по сравнению с другим? Если они могут быть возбуждены отдельно друг от друга, то мы по праву можем называть их независимыми.

~Выше мы использовали метафору из области астрономии. Интересно то, что проблема движения планет весьма сложна; в действительности, общая проблема трех гравитационно взаимодействующих тел, таких, например, как Земля, Луна и Солнце, все еще не разрешена, даже после нескольких столетий поиска. Однако можно получить довольно точное приближение результата, когда одно из тел гораздо массивнее других (в нашем примере это Солнце). Тогда имеет смысл считать это тело неподвижным, а два других --- вращающимися вокруг него; после этого можно учесть взаимодействие двух спутников между собой. Это приближение требует разбивания системы две части: Солнце и некий «блок» --- систему Земля-Луна. Это, разумеется, только приближение, но оно помогает нам намного глубже понять всю систему. Так до какой же степени этот блок --- часть реальности, и до какой степени он --- измышление человеческого разума, наложение людьми определенной схемы на вселенную? Проблема «реальности» границ между «автономными» и «полуавтономными» блоками, как мы их воспринимаем, доставит нам немало забот, когда мы попытаемся соотнести эти понятия с символами в мозгу.

Весьма затруднительным вопросом, например, является вопрос о множественном числе. Как мы себе представляем, скажем, трех собак в чайной чашке? Или нескольких человек в лифте? Начинаем ли мы с символа-класса «собака» и затем снимаем с него три «копии»? Иными словами, используем ли мы символ «собака» как форму для отливки трех свежих символов-примеров? Или же мы одновременно активируем символы «собака» и «три»? Чем больше деталей мы добавляем к воображаемой сцене, тем менее приемлемыми кажутся обе эти теории. Скажем, у нас нет отдельного символа-примера для всех носов, усов или крупинок соли, которые мы когда-либо видели. Для таких множественных объектов мы пользуемся классами-символами; когда на улице мимо нас проходят люди с усами, мы активируем лишь класс-символ «усы», обычно не создавая при этом новых индивидуальных символов.

С другой стороны, как только мы начинаем различать людей, мы уже не можем опираться на общий символ-класс «человек». Очевидно, что необходимы отдельные символы-примеры для каждого отдельного человека. Смешно было бы воображать, что это может быть достигнуто путем «жонглирования» единственным символом, перебрасывая его между различными способами активации (по способу на каждого нового человека).

Между крайностями должно быть место для многих промежуточных случаев. Возможно, что в мозгу есть целая иерархия путей различения между классами и примерами, иерархия, порождающая символы --- и организации символов --- различной степени специфичности:

(1) несколько различных типов и степеней интенсивности активации символов-классов;

(2) одновременная согласованная активация нескольких символов-классов;

(3) активация одного символа-класса;

(4) активация одного символа-примера одновременно с активацией нескольких символов-классов;

(5) одновременная согласованная активация нескольких символов-примеров и символов-классов.

Это снова приводит нас к вопросу «когда символ является различимой подсистемой мозга?» Посмотрим, скажем, на второй пример --- одновременная согласованная активация нескольких символов-классов. Вполне возможно, что именно это и происходит, когда мы рассматриваем понятие «соната для фортепиано» (при этом активируются по-крайней мере два символа: «фортепиано» и «соната»). Но если эта пара символов активируется вместе достаточно часто, то разумно предположить, что рано или поздно между ними установится такая тесная связь, что они начнут действовать как некая единица каждый раз, когда они активированы соответствующим образом. Таким образом, в соответствующих условиях два или более символов могут действовать как один --- а это значит, что проблема подсчета символов в мозгу еще сложнее, чем нам казалось.

При некоторых условиях два ранее не связанных символа могут одновременно активироваться координированным путем. При этом они могут так подойти друг другу, что образуется новый символ, тесно связующий два прежних. Справедливо ли в таком случае утверждать, что новый символ «всегда был в мозгу, но до сих не был активирован» --- или же мы должны сказать, что он только что «создан»?

Если это звучит для вас слишком абстрактно, давайте рассмотрим конкретный пример: Диалог «Крабий канон». При написании этого Диалога два существующих символа --- «музыкальный канон-ракоход» и «словесный диалог» --- должны были быть активированы одновременно и им пришлось взаимодействовать. Как только это произошло, остальное было почти неизбежно: родился новый символ-класс, который в дальнейшем мог активироваться самостоятельно. Был ли он в моем мозгу всегда, в пассивном состоянии? В таком случае то же должно быть верно для любого человека, в чьем мозгу когда-либо имелись составляющие символы, даже если новый символ-класс никогда не был там активирован. Тогда, чтобы подсчитать количество символов в мозгу любого человека пришлось бы учитывать все \emph{пассивные} символы --- все возможные комбинации и комбинации всех возможных типов активации всех известных символов. Это включало бы даже фантастические создания, которые наш мозг изобретает во время сна --- странные смеси идей, которые просыпаются, когда их «хозяин» засыпает\ldots{} Существование этих «потенциальных символов» показывает, что представлять мозг, как строго определенную коллекцию символов в хорошо определенных состояниях, было бы слишком большим упрощением. Точно охарактеризовать состояние мозга на уровне символов гораздо сложнее.

Символы --- программное обеспечение или аппаратура?

Думая о громадном и непрерывно растущем количестве символов в мозгу, вы можете задаться вопросом --- а не наступит ли такой момент, когда мозг насытится, и в нем просто не окажется больше места для нового символа? Предположительно, такое могло бы произойти, если бы символы не пересекались и не накладывались бы один на другой --- если бы данный нейрон никогда не выступал бы в разных ролях. Тогда символы были бы подобны людям в лифте: «Осторожно. максимальная вместимость 350 275 символов!»

Однако это вовсе не обязательная черта моделей функционирования мозга. На самом деле, пересечение и сложная связь символов между собой скорее являются правилом; вероятно, каждый нейрон, вместо того, чтобы быть членом единственного символа, функционирует, как часть сотен различных символов.

Это звучит немного тревожно --- если дело обстоит именно так, почему бы тогда не считать, что каждый нейрон --- часть каждого существующего символа? Если так, то символы было бы невозможно локализовать --- каждый символ идентифицировался бы с целым мозгом. Это объяснило бы результаты, полученные Лашли при удалении частей коры головного мозга у крыс; однако нам пришлось бы отказаться от нашего первоначального намерения разделить мозг на отдельные физические подсистемы. Наша характеристика символов как «реализации понятий на уровне аппаратуры» оказывалась бы, в лучшем случае, слишком упрощенной. Ведь если бы каждый символ состоял из тех же нейронов, что и все остальные символы, то какой смысл был бы вообще говорить о различных символах? Какой была бы тогда «подпись» активации данного символа --- иными словами, как можно было бы отличить активацию символа А от активации символа В? Не разрушило ли бы это всю нашу теорию? Даже если \emph{полного} совпадения символов и не происходит, все же, чем больше они пересекаются, тем труднее будет нам поддерживать жизнь нашей теории. (Одна из возможностей пересечения символов представлена на рис. 68.)

Существует возможность спасти теорию, основанную на символах, даже когда те физически в значительной степени или даже полностью совпадают. Представьте себе поверхность пруда, на которой могут возникать самые различные типы волн. Аппаратура --- сама вода --- остается неизменной, но она может быть «возбуждена» по-разному. Подобные различные состояния --- программы --- одной и той же аппаратуры могут быть отличены друг от друга.

Предлагая эту аналогию, я не утверждаю, что все символы --- не что иное, как различные типы «волн», распространяющихся в однородной нейронной среде, которая не может быть подразделена на физически различимые символы. Однако вполне возможно, что для того, чтобы отличить активацию одного нейрона от активации другого, важно не только локализовать эти нейроны, но и точно определить соответствующие моменты их активации. Какой нейрон активировался раньше, и насколько? Сколько сигналов в секунду послал данный нейрон? Таким образом, разные символы могут сосуществовать в одном и том же наборе нейронов --- они характеризуются различными схемами активации. Разница между теорией, предполагающей физически различные символы, и теорией пересекающихся символов, различающихся друг от друга типом активации, в том, что первая предполагает реализацию понятий на уровне аппаратуры, а вторая --- частично на уровне аппаратуры и частично на уровне программ.

\emph{Рис. 68. На этой схематической диаграмме нейроны изображены в виде точек на плоскости. Два пересекающиеся пути нейронов отмечены разными оттенками серого цвета. Может случиться так, что два независимых нейронных сигнала одновременно устремляются по этим путям, проходя друг сквозь друга, как две волны на поверхности пруда (Рис. 52). Это иллюстрирует идею о том, что два активных символа могут частично состоять из одних и тех же нейронов, которые могут быть активированы одновременно. (Из книги Джона К. Экклса «Лицом к лицу с реальностью» (John С. Eccles, «Facing Reality»), стр. 21.)}

Отделяемость разума

Итак, в наших попытках понять процессы мышления мы столкнулись с двумя основными проблемами. Одна состоит в том, чтобы понять, каким образом активация нейронов на низшем уровне вызывает активацию символов на высшем уровне. Другая проблема --- в том, чтобы объяснить активацию символов на высшем уровне, не прибегая при этом к терминологии низшего, нейронного уровня. Если последнее возможно (как утверждает рабочая гипотеза, лежащая в основе большинства современных исследований по искусственному интеллекту), то интеллект может возникнуть и в других, отличных от мозга, типах аппаратуры. Таким образом, можно представить интеллект как характеристику, отделимую от аппаратуры, в которой она заключается --- иными словами, интеллект был бы заключен не в аппаратуре, а в программе. Это означало бы, что явления сознания и интеллекта --- это явления высшего порядка в том же смысле, как и многие другие сложные явления природы; они управляются своими законами высшего уровня, которые, разумеется, зависят от низшего уровня, но, тем не менее, могут быть от него отделены.

С другой стороны, если бы схемы активации символов оказались совершенно неосуществимыми без нейронов аппаратуры (или их симуляции), это означало бы, что интеллект неотделим от мозга, и что его гораздо труднее объяснить, чем какую-либо другую систему, основанную на иерархии законов на нескольких различных уровнях.

Вернемся к удивительному коллективному поведению, наблюдаемому в муравьиных колониях, поведению, в результате которого строятся огромные, сложные муравейники, хотя в приблизительно 100 000 нейронах муравьиного мозга почти наверняка не заложена никакая информация о структуре муравейника. Каким же образом, в таком случае, строится муравейник? Где находится нужная информация? Подумайте, например, над тем, где может находиться информация, необходимая для постройки арок, подобных тем что показаны на рис. 69. Она должна быть каким-то образом распространена по колонии, выражаясь в распределении каст, возрастов --- а также, возможно, в физических характеристиках самого муравьиного тела. То-есть, взаимодействие между муравьями настолько же определяется их шестиногостью, размером, и т. п., насколько оно определяется информацией, хранящейся у них в мозгу. Возможно ли создать Искусственную Муравьиную Колонию?

\emph{Рис. 69. Конструирование арки термитами-рабочими Macrotermes bellicosus. Каждая колонна надстраивается путем добавления шариков, сделанных из земли и экскрементов. С внешней стороны левой колонны можно видеть термита откладывающего на колонну такой шарик Другие работники, уже поднявшие в челюстях шарики на верх колонны, укладывают их на растущих концах. Когда колонна достигает определенной высоты, термиты, видимо, ориентируясь по запаху начинают наращивать колонну под углом к соседней. Законченная арка показана на заднем плане. (Рисунок Турида Холлдоблера из книги Е. О. Вильсона «Общества насекомых» (Е. О. Wilson, «The Insect Societies») стр. 230)}

Можно ли изолировать один символ?

Можно ли активировать один единственный символ не активируя при этом никаких других? Вероятно, нет. Подобно тому, как все вещи в мире существуют в контексте других вещей символы всегда пребывают в контакте с целыми созвездиями других символов. Это не означает, что символы невозможно отличить один от другого. Приведу простой пример в большинстве видов имеются мужские и женские особи, чьи роли тесно взаимосвязаны, однако это не значит что мужчину невозможно отличить от женщины. Каждый из них отражен в другом подобно стеклянным сферам в сети Индры. Рекурсивная связь функций F(n) и M(n) в главе V не мешает каждой функции иметь свои собственные характеристики. Связь между этими функциями сравнима с отношением между парой СРП вызывающих одна другую. Отсюда мы можем перейти к целой сети тесно взаимосвязанных схем --- гетерархии взаимодействия рекурсивных процедур. Связи здесь настолько сильны что ни одна схема не может быть активирована в изоляции но при этом активация каждой схемы своеобразна и легко отличима от активации других схем. Кажется что сравнение мозга с колонией СРП не так уж плохо!

Таким же образом символы со всеми их сложными связями между собой прочно сцеплены друг с другом и тем не менее различимы. Возможно что для этого необходимо идентифицировать нейронную сеть или ту же сеть плюс тип активации или же что нибудь совершенно в другом роде. В любом случае если символы --- части реальности то должен существовать способ их аккуратного отображения в мозгу. Однако если бы нам и удалось идентифицировать некоторые символы это еще не означало бы что их можно активировать по отдельности.

Символы насекомых

Способность производить примеры на основе классов и классы на основе примеров лежит в основе нашего интеллекта это одно из основных различий между процессом мышления человека и процессом мышления других животных. Конечно я сам никогда не принадлежал к другим видам и мне не приходилось испытывать на собственном опыте их способ мышления --- но со стороны очевидно что никакой другой вид не формирует общие понятия как это делаем мы и не воображает гипотетические миры --- варианты действительности помогающие нам принимать решения. Рассмотрим для примера, ставший знаменитым язык пчел --- танцы возвращающихся в улей пчел-работников при помощи которых они сообщают своим собратьям о том где есть нектар. Хотя у каждой пчелы может иметься рудиментарный набор символов, которые активируются этим танцем, нет основания предполагать, что запас символов в пчелином мозгу может быть расширен. Пчелы и другие насекомые по видимому не умеют обобщать --- то есть развивать новые символы классы на основе примеров, которые показались бы человеку почти идентичными.

Классический эксперимент с осами описан в книге Дина Вулдриджа «Механический человек» (Dean Wooldridge Mechanical Man):

Когда приходит время откладывать яйца, оса Sphex делает себе для этого нору и ищет сверчка, которого она жалит так чтобы не убить а парализовать. Она относит сверчка в нору, откладывает около него яйца закрывает нору и затем улетает чтобы никогда не вернуться. Через некоторое время из яиц вылупляются личинки осы, они питаются парализованным сверчком, который таким образом сохранялся свежим --- нечто вроде осиного эквивалента холодильника. С человеческой точки зрения, подобные сложно организованные и кажущиеся целенаправленными действия убедительно говорят о логике и осмысленности --- пока мы не обращаем внимание на некоторые детали. Например рутинные действия осы следующие: она относит парализованного сверчка к норе оставляет его у входа, заползает внутрь, проверить все ли в порядке, выходит наружу и лишь затем затаскивает сверчка в нору. Если отодвинуть сверчка на несколько сантиметров в сторону, пока оса занимается предварительным осмотром норы, она, выйдя наружу, снова подтащит сверчка к норе и оставит его на пороге, после чего она снова войдет в нору, проверить, все ли в порядке. Если снова отодвинуть сверчка на несколько сантиметров, пока оса внутри, она опять подтащит его к порогу и снова войдет проверять, все ли в порядке в норе. Она никогда не догадается затащить сверчка внутрь сразу. Однажды этот опыт был повторен сорок раз с тем же самым результатом.\footnote{Dean Wooldridge, «Mechanical Man», стр. 70.}

Кажется, что это поведение заложено в самой аппаратуре осиного мозга. В мозгу осы могут существовать рудиментарные символы, способные активировать друг друга; но там нет ничего похожего на человеческую способность видеть несколько примеров, как членов возможного класса и затем создавать этот символ-класс; нет там и ничего подобного человеческой способности спрашивать себя: «А что, если я сделаю так --- что из этого получится в моем гипотетическом мире?» Этот тип мышления требует способности создавать символы-примеры и затем обращаться с ними так, словно они представляют предметы в реальной ситуации, хотя эта ситуация может никогда не возникнуть на самом деле.

Символы-классы и воображаемые миры

Вернемся к первоапрельской шутке о взятой взаймы машине и к картинам, возникшим у вас в голове во время телефонного разговора. Для начала вы должны были активировать символы дороги, машины, человека за рулем. Понятие «дороги» весьма общее; у вас могут иметься в запасе некие «дремлющие» примеры, которые вы можете при случае вспомнить. «Дорога» --- это скорее символ-класс, чем символ-пример. Слушая рассказ, вы быстро активируете символы, представляющие все более конкретные примеры. Например, когда вы слышите, что дорога была мокрая, вы представляете себе некую конкретную картину, хотя вы знаете, что настоящая дорога, где произошла авария, может весьма отличаться от той, что встает перед вашим мысленным взором. Однако это неважно; необходимо только, чтобы ваш символ достаточно хорошо вписывался в историю --- то есть чтобы он, в свою очередь, мог активировать символы нужного типа.

По мере того, как рассказ продолжается, вы дополняете ваш мысленный образ дороги: там есть глубокий кювет, куда машина могла упасть. Значит ли это, что вы активируете символ «кювет» или что вы уточняете некоторые параметры в символе «дорога»? Без сомнения, верно и то и другое. Дело в том, что сеть нейронов, составляющих символ «дорога», может быть активирован разными путями, и вы выбираете, какая из его подсистем будет активирована в данный момент. Одновременно с этим, вы активируете символ «кювет», что, в свою очередь, влияет на способ активации символа «дорога», поскольку нейроны этих двух символов могут обмениваться сигналами друг с другом. (Это может показаться немного запутанным, поскольку я смешиваю здесь два уровня описания, пытаясь представить одновременно как символы, так и составляющие их нейроны.)

Не менее важными, чем имена существительные, являются глаголы, предлоги, и так далее. Они также активируют символы, которые начинают затем обмениваться сигналами. Схемы активации символов для глаголов и символов для имен существительных, разумеется, отличны друг от друга, что означает, что физически эти символы могут быть организованны по-разному. Например, символы для существительных могут быть расположены в каких-то определенных местах, в то время как символы для глаголов и предлогов могут иметь «щупальца» по всей коре; существует множество разных возможностей.

Когда рассказ окончен, вы узнаете, что вас разыграли, --- все это было только шуткой. Наше умение производить символы-примеры на основе символов-классов, подобно тому, как мы можем получить изображение монетки, заштриховав положенную на нее бумагу, позволяет нам представлять ситуации, не будучи при этом рабами действительности. Тот факт, что одни символы могут служить базой для создания других символов, дает нам некую мысленную свободу от окружающей реальности; мы можем создавать искусственные вселенные, где возможны любые события, которые мы можем описать как угодно детально. Но при этом у символов-классов, на ветвях которых расцветают эти воображаемые цветы, глубокие корни в реальной жизни.

Обычно символы играют изоморфные роли по отношению к возможным событиям, хотя иногда активируются символы, представляющие невозможные ситуации, --- например, туба, откладывающая яйца, или говорящая кошка. Граница между возможным и невозможным весьма нечетка. Воображая некое гипотетическое событие, мы приводим определенные символы в активное состояние и, в зависимости от того, насколько хорошо они взаимодействуют между собой (что, предположительно, отражается в том, насколько легко нам довести данную мысль до конца), мы говорим, что это событие «возможно» или «невозможно».

Таким образом, термины «возможно» и «невозможно» весьма субъективны. На самом деле большинство людей легко соглашаются с тем, какие события могут случиться и какие менее вероятны; это объясняется тем, что все мы имеем схожие структуры в мозгу. Однако существует пограничная зона, в которой субъективный характер воображаемых миров становится очевидным. Глубокое изучение того, какие именно воображаемые события люди считают «возможными» или «невозможными», пролило бы свет на поведение символов, лежащих в основе человеческой мысли.

Интуитивные законы физики

Когда рассказ был окончен, у вас в голове сложилась детальная модель того, что случилось; все предметы в этой модели повинуются физическим законам. Это значит, что те же законы скрыто присутствуют в самой схеме активации символов. Разумеется, фраза «физические законы» не означает здесь физических законов в том виде, как они излагаются учеными; скорее, имеются в виду интуитивные, блочные законы, которым мы повинуемся с тем, чтобы выжить.

Интересно то, что мы можем по желанию выдумать целую серию событий, идущих вразрез с законами физики. Например, если я попрошу вас вообразить, что две машины, идущие навстречу друг другу, вместо того, чтобы столкнуться, проходят одна сквозь другую, вы представите себе соответствующую сцену без труда. Интуитивные физические законы могут быть «отменены» законами воображаемыми; но то, как это происходит, как рождаются в мозгу подобные последовательности событий --- даже сама сущность всякого зрительного образа --- все еще является для нас глубочайшей загадкой.

Нет нужды говорить, что в нашем мозгу существуют интуитивные законы, описывающие поведение не только неодушевленных предметов, но и растений, животных, людей и государств --- иными словами, блочные законы биологии, психологии, социологии и так далее. Все внутренние представления подобных понятий с необходимостью включают черты блочных, обобщенных моделей: детерминизм здесь приносится в жертву ради простоты Наша модель реального мира способна предсказать только вероятность того или иного гипотетического события --- но она не предсказывает ничего с точностью физики.

Знание процедурное и знание декларативное

В науке об искусственном интеллекте различаются два типа знания: процедурное и декларативное. Знание называется \emph{декларативным} , если оно хранится в памяти явно, так что к нему имеют доступ не только программист, но и сама программа его можно «прочитать», словно энциклопедию или альманах. Обычно это значит, что такое знание локализовано, а не распространено по всей памяти. С другой стороны, \emph{процедурное} знание закодировано не в форме фактов, а в форме программ. Программист может взглянуть на них, и сказать: «Я знаю, что благодаря этим процедурам, программа „умеет`` писать русские предложения,» --- но сама программа может понятия не иметь, \emph{как именно} она это делает. Например, ее словарь может вообще не включать слова «русский», «предложение», и «писать»! Такое процедурное знание обычно разбросано по памяти в виде кусков, и на него невозможно указать пальцем. Это не отдельная деталь, но общее следствие работы программы. Иными словами, кусок процедурного знания --- это эпифеномен.

У большинства людей, наряду с глубоким процедурным знанием грамматики их родного языка, существует более слабое декларативное представление о ней. Эти два типа знания могут легко вступать в конфликт; например, носитель языка может пытаться научить иностранца выражениям, которые он сам не стал бы употреблять, но которые находятся в согласии с декларативным «книжным представлением», которому его когда-то научили в школе. Интуитивные, блочные законы физики и других дисциплин, о которых мы упомянули выше, представляют собой в основном процедурное знание; тот факт, что у паука восемь ног --- это в основном знание декларативное.

Между процедурным и декларативным существует множество переходных типов знания. Представьте себе, что вы пытаетесь вспомнить какую-то мелодию. Записана ли она у вас в мозгу нота за нотой? Сможет ли нейрохируг вынуть нервное волокно из вашего мозга и указать на нем, словно на магнитной ленте, каждую из последовательно записанных нот? Это означало бы, что мелодии хранятся в виде декларативного знания. Или же при попытке вспомнить мелодию в мозгу активируется множество символов, представляющих тональные соотношения, эмоциональные характеристики, ритмические особенности и так далее? Это означало бы, что мелодии хранятся в виде процедурного знания. На самом деле, возможно, что в записи мелодий в нашем мозгу участвуют оба эти типа знания.

Интересно то, что вспоминая мелодию, большинство людей не различают между возможными тональностями; им все равно, пропеть ли «В лесу родилась елочка» в до или в ми мажоре. Это означает, что в мозгу записаны не сами абсолютные тональности, а их соотношение. Однако у нас нет причин полагать, что это соотношение тональностей не может быть закодировано в декларативной форме. С другой стороны, некоторые мелодии запоминаются очень легко, в то время как другие никак не удается запомнить. Если бы все мелодии были закодированы в виде последовательности нот, сохранение в памяти любой мелодии должно было бы быть одинаково легким делом. Тот факт, что одни мелодии запоминаются легко, а другие --- нет, указывает, по-видимому, на существование в мозгу неких хорошо знакомых нам схем, которые активируются, когда мы слышим ту или иную мелодию. Чтобы воспроизвести данную мелодию, эти схемы должны быть активированы в том же порядке. Это возвращает нас к символам, активирующим один другого, вместо простой линейной последовательности закодированных декларативным образом нот или тональностей.

Откуда мозгу известно, когда кусок знания закодирован декларативным образом? Вообразите, например, что вас спрашивают. «Сколько человек живет в Санкт-Петербурге?» Каким-то образом вам сразу приходит на ум число пять миллионов; при этом вам нет нужды спрашивать себя: «Батюшки, как же я их всех смогу подсчитать?» Теперь представьте себе, что я вас спрашиваю: «Сколько стульев стоит у вас в столовой?» Здесь происходит обратное: вместо того, чтобы пытаться вытащить ответ из вашей мысленной картотеки, вы либо идете в столовую и считаете там стулья, либо мысленно представляете себе столовую и считаете стулья в воображаемой столовой. Вопросы были одного типа --- «сколько?..» --- но один из них заставил вас вытащить «кусок» декларативного знания, в то время как другой привел в действие процедурный метод нахождения ответа. Этот пример показывает, что у нас есть знания о том, как мы классифицируем наши собственные знания; более того, некоторые из этих метазнаний в свою очередь могут быть закодированны процедурно, так что вы используете их автоматически, не отдавая себе отчета в том, как именно вы это делаете.

Зрительные образы

Одним из самых замечательных и трудно описуемых свойств сознания является его способность создавать зрительные образы. Как мы создаем мысленный образ нашей гостиной? Или бурного горного ручья? Или апельсина? Как нам удается создавать эти образы бессознательно --- образы, которые дают нашим мыслям выразительность, цвет и глубину? С какого мысленного склада они достаются? С помощью какого волшебства нам удается смешивать два или три образа в один, даже не думая о том, как мы это делаем? Знания о том, как это делается, --- один из самых ярких примеров процедурных знаний, поскольку мы почти ничего не знаем о том, что такое зрительные образы.

Возможно, что мысленные образы основаны на нашей способности подавлять моторную деятельность. Я имею в виду следующее: когда вы воображаете себе апельсин, в коре вашего мозга могут возникнуть команды взять его, понюхать, осмотреть, и так далее. Ясно, что эти команды не могут быть исполнены, поскольку апельсин находится только у вас в воображении. Но они могут быть направлены по обычным каналам в мозжечок или другие подсистемы мозга, пока в некий критический момент «мысленный кран» не закрывается, предотвращая действительное исполнение команд. В зависимости от того, насколько далеко расположен этот «кран», образы могут казаться более или менее жизненными и натуральными. В гневе мы легко можем вообразить, что хватаем какой-то предмет и швыряем его, или пинаем что-то, хотя на самом деле мы этого не делаем, но чувствуем, что были весьма близки к этому. Возможно, «кран» перекрыл нервные импульсы в самый последний момент.

Вот еще один способ различить между доступным и недоступным видами знания при помощи образов. Вспомните, как вы представляли себе машину, скользящую на мокрой горной дороге. Без сомнения, в вашем воображении гора рисовалась намного большей, чем машина. Почему это происходит? Потому ли, что когда-то вы заметили, что машины обычно бывают меньше, чем горы, запомнили это наблюдение, и воспользовались им, при восстановлении в вашем воображении данной истории? Маловероятное предположение. Или же это случилось благодаря невидимым нам взаимодействиям неких символов, активированных в вашем мозгу? Ясно, что последнее кажется гораздо более вероятным. Знание о том, что машины меньше гор --- это не кусок запомненного материала; оно может быть получено \emph{дедуктивным путем} . Следовательно, оно, скорее всего, закодировано не в одном единственном символе, а является результатом активации и последующего взаимодействия многих символов, таких, например, как «сравнивать», «размер», «гора», «машина» и других. Это означает, что знания хранятся в мозгу не явно, не в каких-либо определенных местах ---~скорее, они распространены по большим участкам коры. Такие простые факты, как размер предметов, должны быть «собраны по частям», а не просто вынуты из памяти. Итак, даже в знании, которое может быть выражено словами, есть некие сложные, недоступные нашему взгляду процессы, которые подготавливают это знание к тому моменту, когда оно сможет быть выражено словесно.

Мы продолжим исследование объектов под названием «символы» еще в нескольких главах. В главах XVIII и XIX, посвященных искусственному интеллекту, мы будем говорить о возможных способах включения активных символов в программы. В следующей главе мы рассмотрим объяснения, которые модель мозговой деятельности, основанная на символах, предлагает сравнению мозгов разных людей.

