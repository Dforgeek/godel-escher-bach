\subsection{1}

Далее в книге, говоря о строчках, мы будем использовать следующие знаки: когда строчка будет напечатана тем же шрифтом, как и окружающий ее текст, она будет заключена в простые или двойные кавычки. Знаки препинания, принадлежащие тексту, будут, как того требует логика, вне кавычек. Например, первая буква этой фразы --- «Н», в то время, как первая буква «этой фразы» --- «э». Однако, когда строчка будет напечатана другим шрифтом или латинским алфавитом, кавычки использоваться не будут --- за исключением тех случаев, когда это будет абсолютно необходимо для ясности. Например, первая буква stola --- s.

