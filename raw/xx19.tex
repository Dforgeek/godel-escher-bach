\subsubsection{Канон с интервальны увеличением}

\emph{Ахилл и Черепаха только что доели превосходный ужин на двоих в лучшем китайском ресторане города.}

\emph{Ахилл} : Здорово вы управляетесь с палочками, г-жа Ч.

\emph{Черепаха} : Приходится --- я с детства питаю слабость к восточной кухню. Как насчет вас, Ахилл --- вам понравилось?

\emph{Ахилл} : Еще как! Я никогда раньше не пробовал китайской еды, и сегодняшний ужин был приятным знакомством с ней. А сейчас, если вы не торопитесь мы можем еще немного посидеть и поболтать.

\emph{Черепаха} : Что ж, с удовольствием побеседую с вами, пока мы пьем чай. Официант! (Подходит официант.) Пожалуйста, принесите наш счет. И еще немного чая! (Официант торопливо уходит.)

\emph{Ахилл} : Вы можете понимать больше меня в китайской кухне, г- жа Ч, но могу поспорить, что о японской поэзии я знаю побольше вас. Читали ли вы когда-нибудь хайку?

\emph{Черепаха} : Боюсь, что нет. Что это такое?

\emph{Ахилл} : Хайку --- это японская поэма, в которой семнадцать слогов. Правильнее сказать, что это мини-поэма, наводящая на размышление так,же, как благоуханный розовый лепесток или покрытые росой кувшинки в пруду. Обычно хайку состоит из группы пяти слогов, затем --- семи, и затем --- снова пяти.

\emph{Черепаха} : Такая краткость --- всего семнадцать слогов --- но где же здесь смысл?

\emph{Ахилл} : Смысл живет также в голове читателя --- не только в хайку.

\emph{Черепаха} : Гм-м-м\ldots{} Это утверждение наводит на размышления.

\emph{(Подходит официант со счетом, чайничком, полным чая, и парой печений «с сюрпризом» --- бумажкой, на которой написана судьба едока.)}

Премного благодарна. Еще чайку не желаете, Ахилл?

\emph{Ахилл} : Пожалуй. Эти печеньица выглядят весьма аппетитно. (Берет печенье, откусывает кусочек и начинает жевать.) Эй --- что эта за штуковина тут внутри? Клочок бумаги?

\emph{Черепаха} : Это ваша судьба, Ахилл. Во многих китайских ресторанах вместе со счетом подают печенья с судьбой-сюрпризом, чтобы смягчить удар. Завсегдатаи китайских ресторанов обычно считают их не за печенья, а за посланцев судьбы. К несчастью, вы, кажется, проглотили кусочек своей судьбы. Что там написано, на оставшемся клочке?

~\emph{Ахилл} : Странно --- все буквы сгрудились в кучу, нет никакого деления на слова. Может быть, это надо расшифровать? О, я понял если расставить промежутки там, где надо, получится: «НИС КЛАДУН ИЛ АДУ». Поистине, адская бессмыслица! Может быть, это что-то вроде хайку, от которого я отъел большинство слогов.

\emph{Черепаха} : В таком случае, ваша судьба теперь всего лишь 6/17 хайку. Веселенькие ассоциации все это вызывает. Колдуны, болота, черти, клады\ldots{} Что и говорить, картинка унилая\ldots{} унылая, я имею в виду. Это звучит как комментарий к новой форме искусства --- 6/17 хайку. Можно мне взглянуть?

\emph{Ахилл (протягивая Черепахе узкий клочок бумаги)} : Конечно.

\emph{Черепаха} : Но, Ахилл, в моей «расшифровке» получается нечто совершенно другое! Это вовсе не 6/17 хайку, а шестисложное послание --- и вот что в нем написано «НИ СКЛАДУ НИ ЛАДУ». Поистине, глубокий комментарий к этой новой форме искусства --- 6/17 хайку!

\emph{Ахилл} : Вы правы. Удивительно, что это послание содержит комментарий о самом себе!

\emph{Черепаха} : Я только передвинула рамку чтения на единицу --- сдвинула все промежутки между словами на один интервал.

\emph{Ахилл} : Посмотрим, какая судьба выпала сегодня вам.

\emph{Черепаха (ловко разламывая печенье, читает)} : «Судьбу едока не печенье содержит, а его рука».

\emph{Ахилл} : Ваша «судьба» тоже хайку, г-жа Черепаха --- по крайней мере, в ней семнадцать слогов. 5-7-5.

\emph{Черепаха} : Потрясающе! Я бы сама этого ни за что не заметила, Ахилл --- такие вещи только вы подмечаете. То, что меня больше всего удивило, это сам текст послания; разумеется, его можно интерпретировать по-разному.

\emph{Ахилл} : Наверное, мы все интерпретируем послания по-своему, когда с ними сталкиваемся\ldots{} (Лениво рассматривает чаинки на дне чашки.)

\emph{Черепаха} : Подлить вам чаю?

\emph{Ахилл} : Да, спасибо. Кстати, как поживает ваш товарищ, старый Краб? Я частенько о нем вспоминаю, с тех пор, как вы рассказали мне о его диковиной патефонной войне.

\emph{Черепаха} : Я ему о вас кое-что рассказала, и ему тоже не терпится с вами встретиться. У него все в порядке;на днях он приобрел новую штуковину из серии проигрывателей,~какой-то странный проигрыватель-автомат.

\emph{Ахилл} : Расскажите-ка мне об этом поподробнее. Обожаю эти автоматы --- кругом разноцветные огоньки, и когда опустишь монетку, машина играет глупые песни, которые так и окунают тебя в старое доброе прошлое\ldots{}

\emph{Черепаха} : Этот проигрыватель слишком велик, чтобы держать его дома, и Краб построил для него во дворе специальный навес.

\emph{Ахилл} : Не представляю себе, почему он такой большой? Может, в нем огромная коллекция пластинок?

\emph{Черепаха} : На самом деле, в нем всего одна запись.

\emph{Ахилл} : Что? Проигрыватель-автомат с одной пластинкой? Это уже само по себе противоречие! Почему же он так велик? Может, его единственная пластинка --- гигант двадцати футов в диаметре?

\emph{Черепаха} : Да нет, пластинка самая обыкновенная.

\emph{Ахилл} : Ах, г-жа Черепаха, не иначе как вы надо мной смеетесь. Ну скажите на милость, что это за автомат с единственной песней?

\emph{Черепаха} : Кто сказал хотя бы слово о единственной песне?

\emph{Ахилл} : Любой проигрыватель-автомат, с которым я когда-либо сталкивался, подчинялся фундаментальной аксиоме этих аппаратов: «одна пластинка, одна песня.»

\emph{Черепаха} : Этот автомат не таков, Ахилл. Единственная пластинка в нем расположена вертикально, и за ней находится небольшая, но сложная система рельсов, на которых подвешены проигрыватели. Когда вы нажимаете на пару кнопок, скажем, В-1, вы выбираете один из проигрывателей. Это пускает в действие механизм, и проигрыватель со скрипом отправляется по ржавым рельсам. Вскоре он прибывает к краю пластинки, и --- щелк! --- устанавливается в нужную позицию.

\emph{Ахилл} : И тогда пластинка начинает вращаться, и раздается музыка, правда?

\emph{Черепаха} : Не совсем. Пластинка остается неподвижной --- вращается сам проигрыватель.

\emph{Ахилл} : Я мог бы догадаться. Но каким же образом, если у вас только одна пластинка, вы можете выудить из этой сумасшедшей конструкции больше одной песни?

\emph{Черепаха} : Я и сама спрашивала Краба об этом. Он посоветовал мне попробовать самой. Я нашла в кармане монетку (ее хватало на три песни), засунула ее в щель и нажала наугад: В-1, С-3, и V-10.

\emph{Ахилл} : Значит, патефон В-1 поехал по рельсам, подкатился к вертикальной пластинке и стал вращаться?

\emph{Черепаха} : Точно. Получилась довольно приятная музыка, основанная на знаменитой старой мелодии В-А-С-H, которую, я полагаю, вы еще помните\ldots{}

\emph{Ахилл} : Могу ли я ее забыть?

\emph{Черепаха} : Это был патефон В-1. Когда мелодия закончилась, он отъехал назад, чтобы дать место патефону С-3.

\emph{Ахилл} : Неужели С-3 заиграл другую мелодию?

\emph{Черепаха} : Именно так.

\emph{Ахилл} : А, понимаю. Он проиграл другую сторону пластинки, или, может быть, другую полосу на этой стороне.

\emph{Черепаха} : Нет, на этой пластинке дорожки только с одной стороны и на ней только одна полоса.

\emph{Ахилл} : Ничего не понимаю. Получить разные песни из одной записи НЕВОЗМОЖНО!

\emph{Черепаха} : Я тоже так думала, пока не увидела проигрыватель м-ра Краба.

\emph{Ахилл} : Как звучала эта вторая песня?

\emph{Черепаха} : Это-то как раз интересно: она была основана на мелодии C-A-G-E.

\emph{Ахилл} : Но это совершенно иная мелодия!

\emph{Черепаха} : Верно.

\emph{Ахилл} : Кажется, Джон Кэйдж --- это композитор, создатель авангардистской музыки? Мне кажется, я читал о нем в одной из моих книг хайку.

\emph{Черепаха} : Точно. Многие его творения довольно известны, например, 4'33'' --- трехчастная пьеса, состоящая из безмолвий разной длины. Она необыкновенно выразительна --- если у вас есть вкус к подобным вещам.

\emph{Ахилл} : Что ж, если бы я находился в шумном ресторане, я с удовольствием поставил бы 4'33" Кэйджа на музыкальном автомате. Это могло бы быть некоторым облегчением!

\emph{Черепаха} : Правильно --- кому хочется слушать звон тарелок и стук ножей? Эта пьеса пришлась бы весьма кстати еще в одном месте, в Павильоне~Гигантских Кошек, во время кормления.

\emph{Ахилл} : Вы намекаете на то, что Кэйджу место в зверинце? Что ж, если учесть, что его фамилия в переводе с английского значит «клетка»\ldots{} Но вернемся к крабьему музыкальному автомату --- я ничего не понимаю. Как могут на одной и той же записи быть сразу В-А-С-H и C-A-G-E?

\emph{Черепаха} : Если вы посмотрите повнимательней, Ахилл, вы можете подметить, что между ними есть некоторая связь. Вот, взгляните: что у вас получится, если вы последовательно запишете интервалы мелодии В-А-С-H?

\emph{Ахилл} : Ну-ка, посмотрим\ldots{} Сначала она понижается на полтона, от В до А (я имею в виду немецкое В); затем поднимается на три полутона до С, и, наконец, опускается на полутон, до H. Получается следующая схема:

-1, +3, -1

\emph{Черепаха} : Совершенно верно. А как насчет C-A-G-E?

\emph{Ахилл} : Здесь мелодия сначала идет на три полутона вниз, потом поднимается на десять полутонов, и снова опускается на три полутона. Получается:

-3, +10, -3

Очень похоже на первую мелодию, правда?

\emph{Черепаха} : Действительно, похоже. В некотором смысле, у этих двух мелодий совершенно одинаковый «скелет». Вы можете получить C-A-G-E из~В-А-С-H, умножив все интервалы на 3,5 и беря ближайшее целое число.

\emph{Ахилл} : Вот это да! Это значит, что на звуковых дорожках записан только некий основной код, который разные проигрыватели интерпретируют по-разному?

\emph{Черепаха} : Я не уверена --- этот уклончивый Краб не посвятил меня во все детали. Но мне удалось услышать третью песню, произведенную на проигрывателе В-10.

\emph{Ахилл} : И как она звучала?

\emph{Черепаха} : Ее мелодия состояла из огромных интервалов: В-С-А-H.

Схема в полутонах была такая:

-10, +33, -10

Эта мелодия получается из C-A-G-E, если снова умножить интервалы на 3,3 и округлить результаты до ближайшего целого числа.

\emph{Ахилл} : Есть ли какое-то название у такого умножения интервалов?

\emph{Черепаха} : Его можно назвать «интервальным увеличением». Оно похоже на прием ритмического увеличения темы канона. При этом длительность всех нот мелодии умножается на какое-либо постоянное число. В результате мелодия замедляется. Здесь же интересным образом расширяется диапазон мелодии.

\emph{Ахилл} : Удивительно. Так что все три мелодии, что вы услышали, были интервальными увеличениями одной и той же схемы звуковых дорожек?

\emph{Черепаха} : Таково мое заключение.

\emph{Ахилл} : Забавно, когда мы увеличиваем В-А-С-H, у нас получается C-A-G-E, a когда мы опять увеличиваем C-A-G-E, то снова получаем В-А-С-H, только теперь он весь перевернут, словно В-А-С-H разнервничался, проходя через промежуточный этап C-A-G-E.

\emph{Черепаха} : Поистине, глубокий комментарий к этой новой форме искусства --- музыке Кэйджа.

