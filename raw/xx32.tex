\subsubsection{Англо-франко-германо-русская сюита}

By Lewis Carroll \footnote{Lewis Carroll, «The Annotated Alice» («Alice's Adventures in Wonderland» and «Through the Looking-Glass»). Введение и комментарии Мартина Гарднера (New York: Meridian Press, New American Library, 1960). Эта книга содержит все три версии. Оригинальные источники французского и немецкого текстов приводятся ниже. Русский перевод Д. Орловской.}\ldots{}

\ldots{} et Frank L. Warrin \footnote{Frank L. Warrin, «The New Yorker». 10 января 1931.}\ldots{}

\ldots{} und Robert Scott \footnote{Robert Scott, «The Jabberwock Traced to Its True Source», Macmillan's Magazine, февраль 1872.}\ldots{}

\ldots{} и Д. Г. Орловской.

'Twas brillig, and the slithy toves

Did gyre and gimble in the wabe:

All mimsy were the borogoves,

And the momeraths outgrabe.

Il~brilgue: les t\&\#244;ves lubricilleux

Se gyrent en vrillant dans le guave.

Enmim\&\#233;s sont les gougebosqueux

Et le momerade horsgrave.

Es brillig war. Die schlichten Toven

Wirrten und wimmelten in Waben;

Und aller mumsige Burggoven

Die mohmen Rath' ausgraben.

Воркалось. Хливкие шорьки

Пырялись по наве,

И хрюкотали зелюки,

Как мюмзики в мове.

«Beware the Jabberwock, my son!

The jaws that bite, the claws that catch!

Beware the Jubjub bird, and shun

The frumious Bandersnatch!»

«Garde-toi du Jaseroque, mon fils!

La gueule qui mord; la griffe qui prend!

Garde-toi de l'oiseau Jube, \&\#233;vite

Le frumieux Band-\&\#224;-prend!»

»Bewahre doch vor Jammerwoch!

Die Zahne knirschen, Krallen kratzen!

Bewahr' vor Jubjub-Vogel, vor

Frumiosen Banderschn\&\#228;tzchen!«

«О бойся Бармаглота, сын!

Он так свирлеп и дик,

А в глуще рымит исполин~---

Злопастный Брандашмыг.»

He took his vorpal sword in hand:

Long time the manxome foe he sought ---

So rested he by the Tumtum tree,

And stood awhile in thought.

Son glaive vorpal en main, il va-

T-\&\#224; la recherche du fauve manscant;

Pius arriv\&\#233; \&\#224; l'arbre T\&\#233;-t\&\#233;,

Il у reste, r\&\#233;fl\&\#233;chissant.

Er griff sein vorpals Schwertchen zu,

Er suchte lang das manchsam' Ding;

Dann, stehend unterm Tumtum Baum,

Er an-zu-denken-fing.

Но взял он меч, и взял он щит,

Высоких полон дум,

В глущобу путь его лежит,

Под дерево Тумтум.

And, as in uffish thought he stood,

The jabberwock, with eyes of flame,

Came whiffling through the tulgey wood,

And burbled as it came!

Pendant qu'il pense, tout uffus\&\#233;,

Le Jaseroque, \&\#224; l'oeil flambant,

Vient siblant par le bois tullegeais,

Et burbule en venant.

Als stand er tief in Andacht auf,

Die Jammerwochen's Augen-feuer

Durch turgen Wald mit Wiffek kam

Ein burbelnd Ungeheuer!

Он встал под дерево и ждет,

И вдруг граахнул гром ---

Летит ужасный Бармаглот

И пылкает огнем!

One, two! One, two! And through and through

The vorpal blade went snicker-snack!

He left it dead, and with its head

He went galumphing back.

Un deux, un deux, par le milieu,

Le glaive vorpal fait pat-\&\#224;-pan!

Le b\&\#234;te d\&\#233;faite, avec sa t\&\#234;te,

Il rentre gallomphant.

Eins, Zwei! Eins, Zwei! Und durch und durch

Sein vorpals Schwert zerschnifer-schnuck,

Da blieb es todt! Er, Kopf in Hand,

Gelaumfig zog zuruck.

Раз-два! Раз-два! Горит трава,

Взы-взы --- стрижает меч.

Ува! Ува! И голова

Барабардает с плеч.

«And hast thou slain the Jabberwock?

Come to my arms, my beamish boy!

О frabjous day! Callooh! Callay!»

He chortled in his joy.

«As-tu tu\&\#233; le Jaseroque?

Viens~\&\#224; mon coeur, fils rayonnais!

\&\#212;~jour frabbejais! Calleau! Callai!»

Il cortule dans sa joie.

»Und schlugst Du ja den Jammerwoch?

Umarme mich, mein Bohm'sches Kind!

О Freuden-Tag! О Halloo-Schlag!«

Er schortelt froh-gesinnt.

О, светозарный мальчик мой,

Ты победил в бою!

О, храброславленный герой,

Хвалу тебе пою!

'Twas brilhg, and the slithy toves

Did gyre and gimble in the wabe:

All mimsy were the borogoves,

And the momeraths outgrabe.

Il brilgue: les t\&\#244;ves lubncilleux

Se gyrent en vrillant dans le guave.

Enmim\&\#233;s sont les gougebosqueux

Et le momerade horsgrave.

Es brillig war. Die schlichten Toven

Wirrten und wimmelten in Waben;

Und aller mumsige Burggoven

Die mohmen Rath' ausgraben.

Воркалось. Хливкие шорьки

Пырялись по наве,

И хрюкотали зелюки,

Как мюмзики в мове.

